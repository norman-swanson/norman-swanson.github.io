%2multibyte Version: 5.50.0.2960 CodePage: 936


\documentclass[12pt]{article}
%%%%%%%%%%%%%%%%%%%%%%%%%%%%%%%%%%%%%%%%%%%%%%%%%%%%%%%%%%%%%%%%%%%%%%%%%%%%%%%%%%%%%%%%%%%%%%%%%%%%%%%%%%%%%%%%%%%%%%%%%%%%%%%%%%%%%%%%%%%%%%%%%%%%%%%%%%%%%%%%%%%%%%%%%%%%%%%%%%%%%%%%%%%%%%%%%%%%%%%%%%%%%%%%%%%%%%%%%%%%%%%%%%%%%%%%%%%%%%%%%%%%%%%%%%%%
\usepackage{amssymb}
\usepackage{amsfonts}
\usepackage{amsmath}
\usepackage{geometry}
\usepackage[onehalfspacing]{setspace}
\usepackage{setspace}
\usepackage{graphics}

\setcounter{MaxMatrixCols}{10}
%TCIDATA{OutputFilter=LATEX.DLL}
%TCIDATA{Version=5.00.0.2606}
%TCIDATA{Codepage=936}
%TCIDATA{<META NAME="SaveForMode" CONTENT="1">}
%TCIDATA{BibliographyScheme=Manual}
%TCIDATA{Created=Sunday, July 18, 2004 16:10:34}
%TCIDATA{LastRevised=Monday, October 23, 2023 17:02:36}
%TCIDATA{<META NAME="GraphicsSave" CONTENT="32">}
%TCIDATA{<META NAME="DocumentShell" CONTENT="General\Blank Document">}
%TCIDATA{Language=American English}
%TCIDATA{CSTFile=article.cst}
%TCIDATA{PageSetup=72,72,72,72,0}
%TCIDATA{AllPages=
%F=36,\PARA{038<p type="texpara" tag="Body Text" >\hfill \thepage}
%}


\newtheorem{theorem}{Theorem}
\newtheorem{acknowledgement}[theorem]{Acknowledgement}
\newtheorem{algorithm}[theorem]{Algorithm}
\newtheorem{axiom}[theorem]{Axiom}
\newtheorem{case}[theorem]{Case}
\newtheorem{claim}[theorem]{Claim}
\newtheorem{conclusion}[theorem]{Conclusion}
\newtheorem{condition}[theorem]{Condition}
\newtheorem{conjecture}[theorem]{Conjecture}
\newtheorem{corollary}[theorem]{Corollary}
\newtheorem{criterion}[theorem]{Criterion}
\newtheorem{definition}[theorem]{Definition}
\newtheorem{example}[theorem]{Example}
\newtheorem{exercise}[theorem]{Exercise}
\newtheorem{lemma}[theorem]{Lemma}
\newtheorem{notation}[theorem]{Notation}
\newtheorem{problem}[theorem]{Problem}
\newtheorem{proposition}[theorem]{Proposition}
\newtheorem{remark}[theorem]{Remark}
\newtheorem{solution}[theorem]{Solution}
\newtheorem{summary}[theorem]{Summary}
\newenvironment{proof}[1][Proof]{\noindent\textbf{#1.} }{\ \rule{0.5em}{0.5em}}
\renewcommand{\baselinestretch}{1.0} 
\textwidth=6.8in
\textheight=8.7in
\oddsidemargin=0in
\evensidemargin=0in
\topmargin=0in
\baselineskip=10pt
\linespread{1.5}
\input{tcilatex}
\geometry{left=1.25in,right=1.25in,top=1.25in,bottom=1.25in}

\begin{document}


\begin{center}
{\Large {Selecting the Relevant Variables for Factor Estimation }}

{\Large {in FAVAR Models$^{\ast }$}}

John C. Chao$^{1}$, Yang Liu${^2}$ and Norman R. Swanson$^{2}$

$^{1}$University of Maryland and $^{2}$Rutgers University

October 23, 2023

Abstract
\end{center}

\begin{spacing}{1.01}
{\linespread{1.1}
\noindent {\footnotesize {In this paper, we propose a new variable selection method that allows researchers to distinguish 
between variables that are relevant in the sense that they provide useful information for estimating underlying
latent factors and variables that are irrelevant in the sense that they do not load on underlying factors, in an FAVAR model. 
In our context, variable selection methods are needed because using too many irrelevant variables could lead to inconsistency
in factor estimation. Our procedure is designed to facilitate consistent factor estimation and can be viewed as the factor 
model analog of the type of multiple hypothesis testing or 
variable selection procedures that people use to select regressors when specifying linear regression. 
One key difference between our method and the typical multiple hypothesis testing procedure is that 
rather than controlling the overall Type I error at some fixed non-zero level, our procedure is completely 
consistent in the sense that the probability of both Type I and Type II errors go to zero asymptotically as 
sample sizes approach infinity.  Monte Carlo evidence indicates that our method has very good finite 
sample properties. Additionally, we analyze a real-time macroeconomic dataset, where it is 
shown that our method delivers factors that result in improved marginal predictive content, relative to 
cases where standard principal components as well as hard-thresholding methods are used in factor estimation.}}
}

\bigskip

\noindent \textit{Keywords: }Factor analysis, factor augmented vector
autoregression, forecasting, moderate deviation, principal components,
self-normalization, variable selection.

\noindent \textit{JEL Classification: }C32, C33, C38, C52, C53, C55.

\bigskip

{\scriptsize 

\noindent $^{\ast }$\textit{Corresponding Author:} Norman R. Swanson, Department of Economics, Rutgers University,
nswanson@econ.

\noindent John C. Chao, Department of Economics, University of Maryland, jcchao@umd.edu. Yang Liu, Department of Economics, Rutgers University,
yl1241@scarletmail.rutgers.edu. The authors are grateful to Matteo Barigozzi, Bin Chen, Rong Chen, Simon Freyaldenhoven, Domenico Giannone,
Yuan Liao, Esther Ruiz, Minchul Shin, Jim Stock, Timothy Vogelsang, Endong Wang, Xiye Yang, Peter Zadrozny, Bo Zhou and seminar participants at the University of Glasgow, 
the University of California Riverside, the Federal
Reserve Bank of Philadelphia, the 2022 North America Summer Meeting of the Econometric Society, the 2022 International Association of Applied Econometrics Association meetings, 
the DC-MD-VA Economtrics Workshop, the 2022 NBER-NSF Time Series Conference, 
the Spring 2023 Rochester Conference in Econometrics, and the 8th Annual Conference of the Society for Economic Measurement for useful comments. Chao thanks the University of Maryland for research support.}
\end{spacing}

\newpage

\noindent \noindent \setcounter{page}{2}

\section{Introduction}

\noindent \qquad As a result of the astounding rate at which raw information
is currently being accumulated, there is a clear need for variable
selection, dimension reduction and shrinkage techniques when analyzing big
data using machine learning methodologies. This has led to a profusion of
novel research in areas ranging from the analysis of high dimensional and/or
high frequency datasets to the development of new statistical learning
methods. Needless to say, there are many critical unanswered questions in
this burgeoning literature. One such question, which we address in this
paper stems from the work of Bai and Ng (2002), Stock and Watson (2002a,b),
and Forni, Hallin, Lippi, and Reichlin (2005). In these papers, the authors
develop methods for constructing forecasts based on factor-augmented
regression models. An obvious appeal of using factor analytical methods for
this problem is the capacity for dimension reduction, so that in terms of
the specification of the forecasting equation, employment of a factor
structure allows the parsimonious representation of information embedded in
a possibly high-dimensional vector of predictor variables\footnote{%
In addition to the greater variety of data that are being collected now, an
important source of high dimensionality in economic datasets is the use of
disaggregate, as opposed to aggregate data (see e.g. Qiu and Qu (2021)).
Disaggregate data may be more informative than aggregate data in situations
where there is information loss in the process of aggregation.}.

Within this context, we note that a key assumption commonly used in the
literature to obtain consistent factor estimation is the so-called factor
pervasiveness assumption, which requires that $\Gamma ^{\prime }\Gamma /N$\
converges to a positive definite matrix as the number of time series
variables, $N\rightarrow \infty $, where $\Gamma $\ denotes the loading
matrix of the factor model. Since this assumption imposes certain conditions
on how the variables in a given dataset load on the underlying latent
factors, it is of interest to have econometric tools which allow researchers
to check the empirical content of this assumption for the particular
datasets they are using. Along these lines, our paper explores situations
where the pervasiveness assumption may not hold because one is working with
a dataset where some of the variables are irrelevant, in the sense that they
do not load on the underlying latent factors. If a sufficient number of such
irrelevant variables exist, inconsistency in factor estimation may result if
one naively includes all available variables when estimating the underlying
factors, without regard to whether they are relevant or not. See Chao, Qiu,
and Swanson (2023), for a particularly pathological example where an
estimated factor, $\widehat{f}_{t},$\ approaches $0$\ in probability,
regardless of what the true value of $f_{t}$\ happens to be - a situation
which can arise when the underlying factors are nonpervasive. Not being able
to obtain consistent estimates of the underlying factors will clearly cause
problems for empirical researchers, such as when the objective is to
estimate forecast functions that incorporate estimated factors. On the other
hand, if one pre-screens the variables and successfully prunes out the
irrelevant ones, then consistent estimation can be achieved, under
appropriate conditions. For this reason, a main contribution of this paper
is to introduce a novel variable selection procedure which allows empirical
researchers to correctly distinguish the relevant from the irrelevant
variables prior to factor estimation, with probability approaching one. We
study this problem within a factor-augmented VAR (FAVAR) framework - a setup
which has the advantage that it allows time series forecasts to be made
using information sets much richer than those used in traditional VAR
models. While the present paper focuses on the development of a variable
selection procedure and the analysis of its asymptotic properties, we show
in Chao, Qiu, and Swanson (2023) that the use of our methodology will allow
the conditional mean function of a factor-augmented forecast equation to be
consistently estimated in a wide range of situations, including cases where
violation of factor pervasiveness is such that consistent estimation is
precluded in the absence of variable pre-screening.\footnote{%
See Theorem 4.2 of Chao, Qiu, and Swanson (2023). A proof of Theorem 4.2 can
be found in Appendix A of that paper.} Monte Carlo experiments indicate that
our procedure has very good finite sample properties, in the sense that when
sample sizes are large, such as the case where the number of observations, $%
T=600$ and $N=1000$, then both Type I and II error rates are very close to
zero. Moreover, even in the smaller sample case where $T=100,$ and $N=100$,
Type I\ and II error rates are usually less than 0.05, and are often much
smaller than that. In order to illustrate the empirical relevance of our
procedure, we also carry out real-time forecasting experiments using a
variety of macroeconomic variables from the well-known FRED-MD database. In
the illustration, we compare our method for pre-selecting variables prior to
factor estimation with two alternative methods for factor construction:
standard PCA, which does not pre-screen the variables, and a hard
thresholding method that is used in the empirical literaure. We find that
our method leads to appreciably more precise predictions when forecasting
using factor-augmented autoregressions than when forecasting with factors
constructed using the other two methods. Overall, we feel that the
theoretical and experimental results detailed in this paper add to the
nascent literature that considers the problem of factor estimation under
various relaxations of the conventional factor pervasiveness assumption
(see, for example, the interesting papers by Giglio, Xiu, and Zhang (2021),
Freyaldenhoven (2021a,b), and Bai and Ng (2021)).

The variable selection procedure reported here is related to the well-known
supervised principal components method proposed by Bair, Hastie, Paul, and
Tibshirani (2006) \textbf{(}henceforth,\textbf{\ }Bair et al. (2006)\textbf{)%
}. Additionally, our procedure is related to recent work by Giglio, Xiu, and
Zhang (2021), who propose a method for selecting test assets, with the
objective of estimating risk premia in a Fama-MacBeth type framework. A
crucial difference between the variable selection method proposed in our
paper and those proposed in these papers is that we use a score statistic
that is self-nomalized, whereas the aforementioned papers do not make use of
statistics that involve self-normalization. An important advantage of
self-normalized statistics is their ability to accommodate a much wider
range of possible tail behavior in the underlying distributions, relative to
their non-self-normalized counterparts. In addition, the type of models
studied in Bair et al. (2006) and Giglio, Xiu, and Zhang (2021) differ
significantly from the FAVAR model studied here. In particular, Bair et al.
(2006) study a one-factor model in an $i.i.d.$ Gaussian framework, thus,
precluding complications associated with the introduction of dependence and
non-normality. Giglio, Xiu, and Zhang (2021), on the other hand, make
certain high-level assumptions which can accommodate some dependence both
cross-sectionally and intertemporally, but the model that they consider is
very different from the dynamic vector time series model studied in the
sequel.\footnote{%
Another interesting recent paper on factor estimation is Ahn and Bae (2022).
This paper uses partial least squares instead of principal component methods
to estimate a factor-based forecasting equation, and thus utilizes an
approach that differs from the one taken in this paper. In addition, Ahn and
Bae (2022) assume factor pervasiveness so that issues of variable selection,
which are the main focus of this paper, do not arise in their paper.}

Before continuing, it should be noted that although our self-normalized
statistics can be viewed as generalizations of the score statistic
introduced in Bair et al. (2006), our use and interpretation of these
statistics differ from that given in their paper. While these authors viewed
the primary function of this type of statistic as providing a method for
picking variables which have predictive content for the target variable of
interest, we view these statistics as being primarily useful for determining
which variables are relevant for factor estimation in the sense that they
load on the underlying factors. Although we agree that these statistics do
provide information about the predictive content of variables; it is
important to note that in the context of the FAVAR model studied here, they
do not allow one to construct a variable selection procedure which, with
probability approaching one, correctly classifies variables on the basis of
their predictive content. On the other hand, as mentioned previously, the
use of these statistics does allow one to correctly identify variables which
are relevant for the purpose of consistent factor estimation. The intuition
behind why this is so is illustrated in an example, which we give as Remark
2.2(a) in Section 2 of the paper. In the example, we elucidate why we feel
that it is important to first try to identify all relevant variables and use
them to get as precise of an estimate of the latent factors as possible,
prior to the use of said factors in forecasting models. Any questions about
predictability can subsequently be addressed via specification testing or
variable selection performed directly on the factor-augmented forecasting
equation, after one plugs in the estimated factors.

Our variable selection procedure also differs substantially from the
approach to multiple hypothesis testing taken in much of the traditional
econometrics/statistics literature. In particular, we show that important
moderate deviation results obtained recently by Chen, Shao, Wu, and Xu
(2016) can be used to help control the probability of a Type I error, i.e.,
the error that an irrelevant variable which is not informative about the
underlying factors is falsely selected as a relevant variable. This is so
even in situations where the number of irrelevant variables may be very
large. Hence, we are able to design a variable selection procedure where the
probability of a Type I error goes to zero, as the sample sizes grow to
infinity. This fact, taken together with the fact that the probability of a
Type II error for our procedure also goes to zero asymptotically, allows us
to establish that our variable selection procedure is completely consistent,
in the sense that the probabilities of both Type I and Type II errors go to
zero in the limit. This property of complete consistency is important
because if we try simply to control the probability of a Type I error at
some predetermined non-zero level, which is the typical approach in multiple
hypothesis testing, then we will not in general be able to estimate the
factors consistently, even up to an invertible matrix transformation, and in
consequence, we will have fallen short of our ultimate goal of obtaining a
consistent estimate of the conditional mean function of the factor-augmented
forecasting equation.

The rest of the paper is organized as follows. In Section 2, we discuss the
FAVAR model and the assumptions that we impose on this model. We also
describe our variable selection procedure and provide theoretical results
establishing the complete consistency of this procedure. Section 3 presents
the results of a promising Monte Carlo study on the finite sample
performance of our variable selection method. Section 4 offers some
concluding remarks. Proofs of the main theorems and of two key supporting
lemmas are provided in the Appendix to this paper. In addition, some further
technical results are reported in an Online Appendix, Chao, Liu and Swanson
(2023).

Before proceeding, we first say a few words about some of the frequently
used notation in this paper. Throughout, let $\lambda _{\left( j\right)
}\left( A\right) $, $\lambda _{\max }\left( A\right) $, and $\lambda _{\min
}\left( A\right) $ denote, respectively, the $j^{th}$ largest eigenvalue,
the maximal eigenvalue, and the minimal eigenvalue of a square matrix $A$.
Similarly, let $\sigma _{\left( j\right) }\left( B\right) $, $\sigma _{\max
}\left( B\right) $, and $\sigma _{\min }\left( B\right) $ denote,
respectively, the $j^{th}$ largest singular value, the maximal singular
value, and the minimal singular value of a matrix $B$, which is not
restricted to be a square matrix. In addition, let $\left\Vert a\right\Vert
_{2}$ denote the usual Euclidean norm when applied to a (finite-dimensional)
vector $a$. Also, for a matrix $A$, $\left\Vert A\right\Vert _{2}\equiv \max
\left\{ \sqrt{\lambda \left( A^{\prime }A\right) }:\lambda \left( A^{\prime
}A\right) \text{ is an eigenvalue of }A^{\prime }A\right\} $ denotes the
matrix spectral norm. For two sequences, $\left\{ x_{T}\right\} $ and $%
\left\{ y_{T}\right\} $, write $x_{T}\sim y_{T}$ if $x_{T}/y_{T}=O\left(
1\right) $ and $y_{T}/x_{T}=O\left( 1\right) $, as $T\rightarrow \infty $.
Furthermore, let $\left\vert z\right\vert $ denote the absolute value or the
modulus of the number $z$; let $\left\lfloor \cdot \right\rfloor $ denote
the floor function, so that $\left\lfloor x\right\rfloor $ gives the integer
part of the real number $x$, and let $\iota _{p}=\left( 1,1,...,1\right)
^{\prime }$ denote a $p\times 1$ vector of ones. Finally, for a sequence of
random variables $u_{i,t+m},u_{i,t+m+1},u_{i,t+m+2},....$; we let $\sigma
\left( u_{i,t+m},u_{i,t+m+1},u_{i,t+m+2},....\right) $ denote the $\sigma $%
-field generated by this sequence of random variables.

\section{\noindent Model, Assumptions, and Variable Selection in High
Dimensions}

\noindent \qquad Consider the following $p^{th}$-order factor-augmented
vector autoregression (FAVAR):%
\begin{equation}
W_{t+1}=\mu +A_{1}W_{t}+\cdot \cdot \cdot +A_{p}W_{t-p+1}+\varepsilon _{t+1}%
\text{,}  \label{FAVAR}
\end{equation}%
where%
\begin{eqnarray*}
\underset{\left( d+K\right) \times 1}{W_{t+1}} &=&\left( 
\begin{array}{c}
\underset{d\times 1}{Y_{t+1}} \\ 
\underset{K\times 1}{F_{t+1}}%
\end{array}%
\right) \text{, }\underset{\left( d+K\right) \times 1}{\varepsilon _{t+1}}%
=\left( 
\begin{array}{c}
\underset{d\times 1}{\varepsilon _{t+1}^{Y}} \\ 
\underset{K\times 1}{\varepsilon _{t+1}^{F}}%
\end{array}%
\right) \text{, }\underset{\left( d+K\right) \times 1}{\mu }=\left( 
\begin{array}{c}
\underset{d\times 1}{\mu _{Y}} \\ 
\underset{K\times 1}{\mu _{F}}%
\end{array}%
\right) ,\text{ and} \\
\text{ }\underset{\left( d+K\right) \times \left( d+K\right) }{A_{g}}
&=&\left( 
\begin{array}{cc}
\underset{d\times d}{A_{YY,g}} & \underset{d\times K}{A_{YF,g}} \\ 
\underset{K\times d}{A_{FY,g}} & \underset{K\times K}{A_{FF,g}}%
\end{array}%
\right) ,\text{ for }g=1,...,p.
\end{eqnarray*}%
Here, $Y_{t}$ denotes the vector of observable economic variables, and $%
F_{t} $ is a vector of unobserved (latent) factors. In our analysis of this
model, it will often be convenient to rewrite the FAVAR in several
alternative forms, which will facilitate writing down assumptions and
conditions used in the sequel. We thus briefly outline two alternative
representations of the above model. First, it is easy to see that the system
of equations given in (\ref{FAVAR}) can be written in the form:%
\begin{eqnarray}
Y_{t+1} &=&\mu _{Y}+A_{YY}\underline{Y}_{t}+A_{YF}\underline{F}%
_{t}+\varepsilon _{t+1}^{Y},  \label{Y component FAVAR} \\
F_{t+1} &=&\mu _{F}+A_{FY}\underline{Y}_{t}+A_{FF}\underline{F}%
_{t}+\varepsilon _{t+1}^{F},  \label{F component FAVAR}
\end{eqnarray}%
where $\underset{d\times dp}{A_{YY}}=\left( 
\begin{array}{cccc}
A_{YY,1} & A_{YY,2} & \cdots & A_{YY,p}%
\end{array}%
\right) $, $\underset{d\times Kp}{A_{YF}}=\left( 
\begin{array}{cccc}
A_{YF,1} & A_{YF,2} & \cdots & A_{YF,p}%
\end{array}%
\right) $, $\underset{K\times dp}{A_{FY}}=\left( 
\begin{array}{cccc}
A_{FY,1} & A_{FY,2} & \cdots & A_{FY,p}%
\end{array}%
\right) $, $\underset{K\times Kp}{A_{FF}}=\left( 
\begin{array}{cccc}
A_{FF,1} & A_{FF,2} & \cdots & A_{FF,p}%
\end{array}%
\right) $, $\underset{dp\times 1}{\underline{Y}_{t}}=\left( 
\begin{array}{cccc}
Y_{t}^{\prime } & Y_{t-1}^{\prime } & \cdots & Y_{t-p{\LARGE +}1}^{\prime }%
\end{array}%
\right) ^{\prime }$, and $\underset{Kp\times 1}{\underline{F}_{t}}=\left( 
\begin{array}{cccc}
F_{t}^{\prime } & F_{t-1}^{\prime } & \cdots & F_{t-p{\LARGE +}1}^{\prime }%
\end{array}%
\right) ^{\prime }$. Another useful representation of the FAVAR model is the
so-called companion form, wherein the $p^{th}$-order model given in
expression (\ref{FAVAR}) is written in terms of a first-order model:%
\begin{equation*}
\underset{\left( d+K\right) p\times 1}{\underline{W}_{t}}=\alpha +A%
\underline{W}_{t-1}+E_{t}\text{,}
\end{equation*}%
where $\underline{W}_{t}=\left( 
\begin{array}{ccccc}
W_{t}^{\prime } & W_{t-1}^{\prime } & \cdots & W_{t-p{\LARGE +}2}^{\prime }
& W_{t-p{\LARGE +}1}^{\prime }%
\end{array}%
\right) ^{\prime }$ and where%
\begin{equation}
\alpha =\left( 
\begin{array}{c}
\mu \\ 
0 \\ 
\vdots \\ 
0 \\ 
0%
\end{array}%
\right) \text{, }A=\left( 
\begin{array}{ccccc}
A_{1} & A_{2} & \cdots & A_{p-1} & A_{p} \\ 
I_{d+K} & 0 & \cdots & 0 & 0 \\ 
0 & I_{d+K} & \ddots & \vdots & 0 \\ 
\vdots & \ddots & \ddots & 0 & \vdots \\ 
0 & \cdots & 0 & I_{d+K} & 0%
\end{array}%
\right) \text{, and }E_{t}=\left( 
\begin{array}{c}
\varepsilon _{t} \\ 
0 \\ 
\vdots \\ 
0 \\ 
0%
\end{array}%
\right) \text{.}  \label{companion form notations}
\end{equation}

In addition to the observations on $Y_{t}$, suppose that the data set
available to researchers includes a vector of time series variables which
are related to the unobserved factors in the following manner:%
\begin{equation}
Z_{t}=\text{ }\Gamma \underline{F}_{t}+u_{t}\text{,}
\label{overspecified factor model}
\end{equation}%
where $\underset{N\times 1}{Z_{t}}=\left( Z_{1t},Z_{2t},...,Z_{Nt}\right)
^{\prime }$. Assume, however, that not all components of $Z_{t}$ provide
useful information for estimating the unobserved vector $\underline{F}_{t}$,
so that the $N\times Kp$ parameter matrix $\Gamma $ may have some rows whose
elements are all zero. More precisely, let the $1\times Kp$ vector $\gamma
_{i}^{\prime }$ denote the $i^{th}$ row of $\Gamma $, and assume that the
rows of the matrix $\Gamma $ can be divided into two classes:%
\begin{eqnarray}
H &=&\left\{ k\in \left\{ 1,....,N\right\} :\gamma _{k}=0\right\} \text{ and}
\label{H} \\
H^{c} &=&\left\{ k\in \left\{ 1,....,N\right\} :\gamma _{k}\neq 0\right\} 
\text{.}  \label{Hc}
\end{eqnarray}%
Now, let $\mathcal{P}$ be a permutation matrix which reorders the components
of $Z_{t}$ such that $\mathcal{P}Z_{t}=\left( 
\begin{array}{cc}
Z_{t}^{\left( 1\right) \prime } & Z_{t}^{\left( 2\right) \prime }%
\end{array}%
\right) ^{\prime }$, where%
\begin{eqnarray}
\underset{N_{1}\times 1}{Z_{t}^{\left( 1\right) }} &=&\Gamma _{1}\underline{F%
}_{t}+u_{t}^{\left( 1\right) }  \label{Z(1)} \\
\underset{N_{2}\times 1}{Z_{t}^{\left( 2\right) }} &=&u_{t}^{\left( 2\right)
}\text{.}  \label{Z(2)}
\end{eqnarray}%
The above representation suggests that the components of $Z_{t}^{\left(
1\right) }$ can be interpreted as the relevant variables for the purpose of
factor estimation, as the information that they supply will be helpful in
estimating $\underline{F}_{t}$. On the other hand, the components of the
subvector $Z_{t}^{\left( 2\right) }$ are irrelevant variables (or pure
\textquotedblleft noise\textquotedblright\ variables), as they do not load
on the underlying factors and only add noise if they are included in the
factor estimation process. Given that an empirical researcher will typically
not have prior knowledge as to which variables are elements of $%
Z_{t}^{\left( 1\right) }$ and which are elements of $Z_{t}^{\left( 2\right)
} $, it will be nice to have a variable selection procedure which will allow
us to properly identify the components of $Z_{t}^{\left( 1\right) }$ and to
use only these variables when we try to estimate $\underline{F}_{t}$. On the
other hand, if we unknowingly include too many components of $Z_{t}^{\left(
2\right) }$ in the estimation process, then inconsistent factor estimation
can arise. This is demonstrated in an example analyzed recently in Chao, Qiu
and Swanson (2023a) which considers a setting similar to the specification
given in expressions (\ref{overspecified factor model})-(\ref{Z(2)}) above,
but for the case of a simple one-factor model. More precisely, Chao, Qiu,
and Swanson (2023) give an example which shows that, in this situation
without variable pre-screening, the usual principal-component-based factor
estimator $\widehat{f}_{t}\overset{p}{\rightarrow }$ $0$ regardless of the
true value $f_{t}$ under the additional rate condition that $N/\left(
TN_{1}^{\left( 1+\kappa \right) }\right) =c+o\left( N_{1}^{-1}\right) $,
where $c$ and $\kappa $ are constants such that $0<c<\infty $ and $0<\kappa
<1$ and where $N_{1}$ is the number of relevant variables, $N_{2}$ is the
number of irrelevant variables, and $N=N_{1}+N_{2}$. This example shows the
kind of severe inconsistency in factor estimation that could result if the
commonly assumed condition of factor pervasiveness (which essentially
requires that $N_{1}\sim N$) does not hold\footnote{%
The reason why we refer to the result given in Chao, Qiu, and Swanson (2023)
as a severe form of inconsistency in factor estimation is because
inconsistency of this type will preclude the consistent estimation of the
conditional mean function of a factor-augmented forecast equation. This is
different from the case where the factors may be estimated consistently up
to a non-zero scalar multiplication or, more generally, up to an invertible
matrix transformation. In the latter case, consistent estimation of the
conditional mean function of a factor-augmented forecast equation can still
be attained.}.

It should be noted that, in a recent paper, Bai and Ng (2021) provide
results which show that factors can still be estimated consistently in
certain situations where factor loadings are weaker than implied by the
conventional pervasiveness assumption; although, as might be expected, in
such cases the rate of convergence of the factor estimator is slower and
additional assumptions are needed. To understand the relationship between
their results and our setup, note that a key condition for the consistency
result given in their paper, when expressed in terms of our setup, is the
assumption that $N/\left( TN_{1}\right) \rightarrow 0$. When violation of
the factor pervasiveness condition is more severe than that characterized by
this rate condition (i.e., if $N/\left( TN_{1}\right) \rightarrow c_{1},$
for some positive constant $c_{1}$ or if $N/\left( TN_{1}\right) \rightarrow
\infty )$, then factors will be estimated inconsistently unless there is
some method which can correctly identify the relevant variables, and only
these variables are used to estimate the factors. Indeed, in Chao, Qiu, and
Swanson (2023), we add to the results given in Bai and Ng (2021) by giving a
result (Theorem 2.1 of Chao, Qiu, and Swanson (2023)) which shows that if
one pre-screens variables using the variable selection method proposed
below, then consistent factor estimation can be achieved, even if the rate
condition that $N/\left( TN_{1}\right) \rightarrow 0$ is not satisfied. In
general, knowledge about the severity with which the conventional factor
pervasiveness assumption may be violated must ultimately be gathered on a
case-by-case basis, and depends on the dataset used for a particular study.
Along these lines, various authors have already documented cases where the
empirical evidence shows that the underlying factors are quite weak,
suggesting that there may be rather severe violation of the assumption of
factor pervasiveness. For example, see Jagannathan and Wang (1998), Kan and
Zhang (1999), Harding (2008), Kleibergen (2009), Onatski (2012), Bryzgalova
(2016), Burnside (2016), Gospodinov, Kan, and Robotti (2017), Anatolyev and
Mikusheva (2021), and Freyaldenhoven (2021a,b). In such cases, it is of
interest to explore the possibility that weakness in loadings is not uniform
across all variables, but rather is due to the fact that only a fraction of
the $Z_{it}$ variables loads significantly on the underlying factors.
Furthermore, even if the empirical situation of interest is one where,
strictly speaking, the condition $N/\left( TN_{1}\right) \rightarrow 0$ does
hold, it may still be beneficial in some such instances to do variable
pre-screening. This is particularly true in situations where the condition $%
N/\left( TN_{1}\right) \rightarrow 0$ is \textquotedblleft
barely\textquotedblright\ satisfied, in which case one would expect to pay a
rather hefty finite sample price for not pruning out variables that do not
load significantly on the underlying factors, since these variables may add
unwanted noise to the estimation process. For these reasons, we believe that
there is a need to develop methods which will enable empirical researchers
to pre-screen the components of $Z_{t},$ so that variables which are
informative and helpful to the estimation process can be properly
identified. In summary, our paper aims to build on the results developed by
Bai and Ng (2021) and others by introducing additional tools for situations
where factor estimator properties may be impacted by failure of the
conventional pervasiveness assumption.

To provide a variable selection procedure with provable guarantees, we must
first specify a number of conditions on the FAVAR model defined above.

\noindent \textbf{Assumption 2-1: }Suppose that:%
\begin{equation}
\det \left\{ I_{\left( d+K\right) }-A_{1}z-\cdot \cdot \cdot
-A_{p}z^{p}\right\} =0,\text{ implies that }\left\vert z\right\vert >1\text{.%
}  \label{stability cond}
\end{equation}

\noindent \textbf{Assumption 2-2: }Let $\varepsilon _{t}$ satisfy the
following set of conditions: (a) $\left\{ \varepsilon _{t}\right\} $ is an
independent sequence of random vectors with $E\left[ \varepsilon _{t}\right]
=0$ $\forall t$; (b) there exists a positive constant $C$ such that $%
\sup_{t}E\left\Vert \varepsilon _{t}\right\Vert _{2}^{6}\leq C<\infty $; and
(c) $\varepsilon _{t}$ admits a density $g_{\varepsilon _{t}}$ such that,
for some positive constant $M<\infty $, $\sup_{t}\dint \left\vert
g_{\varepsilon _{t}}\left( \upsilon -u\right) -g_{\varepsilon _{t}}\left(
\upsilon \right) \right\vert d\upsilon \leq M\left\Vert u\right\Vert $,
whenever $\left\Vert u\right\Vert \leq \overline{\kappa }$ for some constant 
$\overline{\kappa }>0$.

\noindent \textbf{Assumption 2-3: }Let $u_{i,t}$ be the $i^{th}$ element of
the error vector $u_{t}$ in expression (\ref{overspecified factor model}),
and we assume that it satisfies the following conditions: (a) $E\left[
u_{i,t}\right] =0$ for all $i$ and $t$; (b) there exists a positive constant 
$\overline{C}$ such that $\sup_{i,t}E\left\vert u_{i,t}\right\vert ^{7}\leq 
\overline{C}<\infty $, and there exists a constant $\underline{C}>0$ such
that $\inf_{i,t}E\left[ u_{i,t}^{2}\right] \geq \underline{C}$; and (c)
define $\mathcal{F}_{i,-\infty }^{t}=\sigma \left(
....,u_{i,t-2},u_{i,t-1},u_{t}\right) $, $\mathcal{F}_{i,t+m}^{\infty
}=\sigma \left( u_{i,t+m},u_{i,t+m+1},u_{i,t+m+2},....\right) $, and $\beta
_{i}\left( m\right) =\sup_{t}E\left[ \sup \left\{ \left\vert P\left( B|%
\mathcal{F}_{i,-\infty }^{t}\right) -P\left( B\right) \right\vert :B\in 
\mathcal{F}_{i,t+m}^{\infty }\right\} \right] $. Assume that there exist
constants $a_{1}>0$ and $a_{2}>0$ such that%
\begin{equation*}
\beta _{i}\left( m\right) \leq a_{1}\exp \left\{ -a_{2}m\right\} ,\text{ for
all }i\text{.}
\end{equation*}

\noindent \textbf{Assumption 2-4: }$\varepsilon _{t}$ and $u_{i,s}$ are
independent, for all $i,t,$ and $s$.

\noindent \textbf{Assumption 2-5: }There exists a positive constant $%
\overline{C},$ such that $\sup_{i\in H^{{\large c}}}\left\Vert \gamma
_{i}\right\Vert _{2}\leq \overline{C}<\infty $ and $\left\Vert \mu
\right\Vert _{2}\leq \overline{C}<\infty $, where $\mu =\left( \mu
_{Y}^{\prime },\mu _{F}^{\prime }\right) ^{\prime }$.

\noindent \textbf{Assumption 2-6: }Let $A$ be as defined in expression (\ref%
{companion form notations}) above, and let the modulus of the eigenvalues of
the matrix $I_{\left( d+K\right) p}-A$ be sorted so that:%
\begin{equation*}
\left\vert \lambda ^{\left( 1\right) }\left( I_{\left( d+K\right)
p}-A\right) \right\vert \geq \left\vert \lambda ^{\left( 2\right) }\left(
I_{\left( d+K\right) p}-A\right) \right\vert \geq \cdot \cdot \cdot \geq
\left\vert \lambda ^{\left( \left( d+K\right) p\right) }\left( I_{\left(
d+K\right) p}-A\right) \right\vert =\overline{\phi }_{\min }\text{.}
\end{equation*}%
Suppose that there is a constant $\underline{C}>0$ such that%
\begin{equation}
\sigma _{\min }\left( I_{\left( d+K\right) p}-A\right) \geq \underline{C}%
\overline{\phi }_{\min }  \label{lower bd I-A}
\end{equation}%
In addition, there exists a positive constant $\overline{C}<\infty $ such
that, for all positive integer $j$, 
\begin{equation}
\sigma _{\max }\left( A^{j}\right) \leq \overline{C}\max \left\{ \left\vert
\lambda _{\max }\left( A^{j}\right) \right\vert ,\left\vert \lambda _{\min
}\left( A^{j}\right) \right\vert \right\} .  \label{upper bd A}
\end{equation}

\noindent \textbf{Remark 2.1:}

\noindent \textbf{(a)} Note that Assumption 2-1 is the stability condition
that one typically assumes for a stationary VAR process. One difference is
that we allow for possible heterogeneity in the distribution of $\varepsilon
_{t}$ across time, so that our FAVAR process is not necessarily a strictly
stationary process. Under Assumption 2-1, there exists a vector moving
average representation for the FAVAR process.

\noindent \textbf{(b) }It is well known that $\det \left\{ I_{\left(
d+K\right) }-Az\right\} =\det \left\{ I_{\left( d+K\right) }-A_{1}z-\cdot
\cdot \cdot -A_{p}z^{p}\right\} ,$ where $A$ is the coefficient matrix of
the companion form given in expression (\ref{companion form notations}). It
follows that Assumption 2-1 is equivalent to the condition that $\det
\left\{ I_{\left( d+K\right) }-Az\right\} =0$ implies that $\left\vert
z\right\vert >1$. In addition, Assumption 2-1 is also, of course, equivalent
to the assumption that all eigenvalues of $A$ have modulus less than $1$.

\noindent \textbf{(c)} Assumption 2-6 imposes a condition whereby the
extreme singular values of the matrices $A^{j}$ and $I_{\left( d+K\right)
p}-A$ have bounds that depend on the extreme eigenvalues of these matrices.
More primitive conditions for such a relationship between the singular
values and the eigenvalues of a (not necessarily symmetric) matrix have been
studied in the linear algebra literature. In fact, it is easy to show that
Assumption 2-6 holds automatically if the matrix $A$ is diagonalizable, even
if it is not symmetric. Assumptions 2-6, on the other hand, takes into
account other situations where expressions (\ref{lower bd I-A}) and (\ref%
{upper bd A}) are valid even though the matrix $A$ is not diagonalizable.

\noindent \textbf{(d)} Note that Assumptions 2-1, 2-2, and 2-6 together
imply that the process $\left\{ W_{t}\right\} $ generated by the FAVAR model
given in expression (\ref{FAVAR}) is a $\beta $-mixing process with $\beta $%
-mixing coefficient satisfying $\beta _{W}\left( m\right) \leq a_{1}\exp
\left\{ -a_{2}m\right\} $, for some positive constants $a_{1}$ and $a_{2}$,
with $\beta _{W}\left( m\right) =\sup_{t}E\left[ \sup \left\{ \left\vert
P\left( B|\mathcal{A}_{-\infty }^{t}\right) -P(B)\right\vert :B\in \mathcal{A%
}_{t+m}^{\infty }\right\} \right] $, and with

\noindent $\mathcal{A}_{-\infty }^{t}=\sigma \left(
...,W_{t-2},W_{t-1},W_{t}\right) $ and $\mathcal{A}_{t+m}^{\infty }=\sigma
\left( W_{t+m},W_{t+m+1},W_{t+m+2},....\right) $\footnote{%
This can be shown by applying Theorem 2.1 of Pham and Tran (1985). A proof
of this result is also given in Chao, Liu, and Swanson (2023). See, in
particular, Lemma OA-11 and its proof in Chao, Liu, and Swanson (2023).}.
Note, in addition, that Assumption 2-2 (c) rules out situations such as that
given in the famous counterexample presented by Andrews (1984) which shows
that a first-order autoregression with errors having a discrete Bernoulli
distribution is not $\alpha $-mixing, even if it satisfies the stability
condition. Conditions similar to Assumption 2-2(c) have also appeared in
previous papers, such as Gorodetskii (1977) and Pham and Tran (1985), which
seek to provide sufficient conditions for establishing the $\alpha $ or $%
\beta $ mixing properties of linear time series processes.

\medskip

Our variable selection procedure is based on a self-normalized statistic and
makes use of some pathbreaking moderate deviation results for weakly
dependent processes recently obtained by Chen, Shao, Wu, and Xu (2016). An
advantage of using a self-normalized statistic, as discussed in Remark
2.2(b) below, is that it allows the range of the moderate deviation
approximation to be wider relative to their non-self-normalized counterparts%
\textbf{.} To accommodate data dependence, we consider self-nomalized
statistics that are constructed from observations which are first split into
blocks in a manner similar to the kind of construction one would employ in
implementing a block bootstrap or in proving a central limit theorem using
the blocking technique. Two such statistics are proposed in this paper. The
first of these statistics has the form of an $\ell _{\infty }$ norm and is
given by: 
\begin{equation}
\max_{1\leq \ell \leq d}\left\vert S_{i,\ell ,T}\right\vert =\max_{1\leq
\ell \leq d}\left\vert \frac{\overline{S}_{i,\ell ,T}}{\sqrt{\overline{V}%
_{i,\ell ,T}}}\right\vert ,  \label{max statistic}
\end{equation}%
where 
\begin{equation}
\overline{S}_{i,\ell ,T}=\dsum\limits_{r=1}^{q}\dsum\limits_{t=\left(
r-1\right) \tau +p}^{\left( r-1\right) \tau +\tau _{1}+p-1}Z_{it}y_{\ell ,t%
{\LARGE +}1}\text{ and }\overline{V}_{i,\ell ,T}=\dsum\limits_{r=1}^{q}\left[
\dsum\limits_{t=\left( r-1\right) \tau +p}^{\left( r-1\right) \tau +\tau
_{1}+p-1}Z_{it}y_{\ell ,t{\LARGE +}1}\right] ^{2}\text{.}
\label{num and denom max stat}
\end{equation}%
Here, $Z_{it}$ denotes the $i^{th}$ component of $Z_{t}$ , $y_{\ell ,t+1}$
denotes the $\ell ^{th}$ component of $Y_{t+1}$, $\tau _{1}=\left\lfloor
T_{0}^{\alpha _{{\large 1}}}\right\rfloor $, and $\tau _{2}=\left\lfloor
T_{0}^{\alpha _{{\large 2}}}\right\rfloor $, where $1>\alpha _{1}\geq \alpha
_{2}>0$, $\tau =\tau _{1}+\tau _{2}$, $q=\left\lfloor T_{0}/\tau
\right\rfloor $, and $T_{0}=T-p+1$. Note that the statistic given in
expression (\ref{max statistic}) can be interpreted as the maximum of the
(self-normalized) sample covariances between the $i^{th}$ component of $%
Z_{t} $ and the components of $Y_{t+1}$. Our second statistic has the form
of a pseudo-$L_{1}$ norm and is given by: 
\begin{equation*}
\dsum\limits_{\ell =1}^{d}\varpi _{\ell }\left\vert S_{i,\ell ,T}\right\vert
=\dsum\limits_{\ell =1}^{d}\varpi _{\ell }\left\vert \frac{\overline{S}%
_{i,\ell ,T}}{\sqrt{\overline{V}_{i,\ell ,T}}}\right\vert ,
\end{equation*}%
where $\overline{S}_{i,\ell ,T}$ and $\overline{V}_{i,\ell ,T}$ are as
defined in (\ref{num and denom max stat}) above and where $\left\{ \varpi
_{\ell }:\ell =1,..,d\right\} $ denotes pre-specified weights, such that $%
\varpi _{\ell }\geq 0,$ for every $\ell \in \left\{ 1,...,d\right\} $ and $%
\dsum\nolimits_{\ell =1}^{d}\varpi _{\ell }=1$. Both of these statistics
employ a blocking scheme similar to that proposed in Chen, Shao, Wu, and Xu
(2016), where, in order to keep the effects of dependence under control, the
construction of these statistics is based only on observations in every
other block. To see this, note that if we write out the \textquotedblleft
numerator\textquotedblright\ term $\overline{S}_{i,\ell ,T}$ in greater
detail, we have that:%
\begin{eqnarray}
\overline{S}_{i,\ell ,T} &=&\dsum\limits_{t=p}^{\tau _{1}+p-1}Z_{it}y_{\ell
,t{\LARGE +}1}+\dsum\limits_{t=\tau +p}^{\tau +\tau _{1}+p-1}Z_{it}y_{\ell ,t%
{\LARGE +}1}  \notag \\
&&+\dsum\limits_{t=2\tau +p}^{2\tau +\tau _{1}+p-1}Z_{it}y_{\ell ,t{\LARGE +}%
1}+\cdot \cdot \cdot +\dsum\limits_{t=\left( q-1\right) \tau +p}^{\left(
q-1\right) \tau +\tau _{1}+p-1}Z_{it}y_{\ell ,t{\LARGE +}1}
\label{sum of Z times y}
\end{eqnarray}%
Comparing the first term and the second term on the right-hand side of
expression (\ref{sum of Z times y}), we see that the observations $%
Z_{it}y_{\ell ,t{\LARGE +}1}$, for $t=\tau _{1}+p,...,\tau +p-1$, have not
been included in the construction of the sum. Similar observations hold when
comparing the second and the third terms, and so on.

It should also be pointed out that although we make use of some of their
fundamental results on moderate deviation, both the model studied in our
paper and the objective of our paper are very different from that of Chen,
Shao, Wu, and Xu (2016). Whereas Chen, Shao, Wu, and Xu \textbf{(}2016%
\textbf{)} focus their analysis on problems of testing and inference for the
mean of a scalar weakly dependent time series using self-normalized
Student-type test statistics, our paper applies the self-normalization
approach to a variable selection problem in a FAVAR setting. Indeed, the
problem which we study is in some sense more akin to a model selection
problem rather than a multiple hypothesis testing problem. In order to
consistently estimate the factors (at least up to an invertible matrix
transformation), we need to develop a variable selection procedure whereby
both the probability of a false positive and the probability of a false
negative converge to zero as $N_{1}$, $N_{2}$, $T\rightarrow \infty $%
\footnote{%
Here, a false positive refers to mis-classifying a variable, $Z_{it},$ as a
relevant variable for the purpose of factor estimation when its factor
loading $\gamma _{i}^{\prime }=0$, whereas a false negative refers to the
opposite case, where $\gamma _{i}^{\prime }\neq 0,$ but the variable $Z_{it}$
is mistakenly classified as irrelevant.}. This is different from the typical
multiple hypothesis testing approach whereby one tries to control the
familywise error rate (or, alternatively, the false discovery rate), so that
it is no greater than $0.05,$ say, but does not try to ensure that this
probability goes to zero as the sample size grows.

To determine whether the $i^{th}$ component of $Z_{t}$ is a relevant
variable for the purpose of factor estimation, we propose the following
procedure. Define $i\in \widehat{H}^{c}$ to indicate that the procedure has
classified $Z_{it}$ to be a relevant variable for the purpose of factor
estimation. Similarly\textbf{,} define $i\in \widehat{H}$ to indicate that
the procedure has classified $Z_{it}$ to be an irrelevant variable. Now, let 
$\mathbb{S}_{i,T}^{+}$ denote either the statistic $\max_{1\leq \ell \leq
d}\left\vert S_{i,\ell ,T}\right\vert $ or the statistic $%
\dsum\nolimits_{\ell =1}^{d}\varpi _{\ell }\left\vert S_{i,\ell
,T}\right\vert .$\footnote{%
It should be noted that the denominator of the statistic $S_{i,\ell ,T}=%
\overline{S}_{i,\ell ,T}/\sqrt{\overline{V}_{i,\ell ,T}}$ does not
correspond to the use of an HAR standard error constructed using the fixed $%
b $ (or fixed smoothing) approach pioneered by Kiefer and Vogelsang (2002a,
2002b), even in the case without any truncation. Hence, our statistic
differs from the usual Studentized statistic that is normalized by an HAR
estimator. This can be shown by straightforward calculations for the case of
the Bartlett kernel, for example. For interesting discussions of different
approaches to self-normalization in the statistics and probability
literature, refer to Z. Zhou and X. Shao (2013), X. Chen, Q-M. Shao, W.B.
Wu, and L. Xu (2016), and the references cited therein.} Our variable
selection procedure is based on the decision rule: 
\begin{equation}
i\in \left\{ 
\begin{array}{cc}
\widehat{H}^{c} & \text{ if }\mathbb{S}_{i,T}^{+}\geq \Phi ^{-1}\left( 1-%
\frac{\varphi }{2N}\right) \\ 
\widehat{H} & \text{if }\mathbb{S}_{i,T}^{+}<\Phi ^{-1}\left( 1-\frac{%
\varphi }{2N}\right)%
\end{array}%
\right. ,  \label{var selection decision rule}
\end{equation}%
where $\Phi ^{-1}\left( \cdot \right) $ denotes the quantile function or the
inverse of the cumulative distribution function of the standard normal
random variable, and where $\varphi $ is a tuning parameter which may depend
on $N$. Some conditions on $\varphi $ will be given in Assumption 2-10 below.

\noindent \textbf{Remark 2.2:}

\noindent \textbf{(a) }As noted in the introduction, the statistics proposed
in this paper can be viewed as self-normalized versions of the score
statistic introduced in Bair et al. (2006), where to accommodate for time
series dependence in the data, we make the additional modification of
constructing these statistics using block sums. However, our use of these
statistics here differs from how Bair et al. (2006) apply and interpret
their score statistic. In particular, within the FAVAR framework studied
here, we view these self-normalized score statistics as being primarily
useful for identifying variables that are relevant for factor estimation and
not so much for identifying variables which have predictive content for the
target variable of interest. The following example illustrates this and
provides some intuition. Consider the following two-factor FAVAR model:%
\begin{eqnarray}
y_{t{\LARGE +}1} &=&a_{YY}y_{t}+\alpha _{YF,1}f_{1,t}+\varepsilon _{t{\LARGE %
+}1}^{Y}  \notag \\
f_{1,t{\LARGE +}1}
&=&a_{FY,1}y_{t}+a_{FF,11}f_{1,t}+a_{FF,12}f_{2,t}+\varepsilon _{1,t{\LARGE +%
}1}^{F}  \notag \\
f_{2,t{\LARGE +}1}
&=&a_{FY,2}y_{t}+a_{FF,21}f_{1,t}+a_{FF,22}f_{2,t}+\varepsilon _{2,t{\LARGE +%
}1}^{F}\text{,}  \label{FAVAR Example}
\end{eqnarray}%
with factor equation having the form 
\begin{equation}
Z_{t}=\Gamma F_{t}+u_{t},  \label{factor eqn Example}
\end{equation}%
where $F_{t}=\left( 
\begin{array}{cc}
f_{1,t} & f_{2,t}%
\end{array}%
\right) ^{{\Large \prime }}$. Note that, under the specification given by
expressions (\ref{FAVAR Example}) and (\ref{factor eqn Example}), the factor 
$f_{2,t}$ has no predictive content for future values of $y_{t},$ whereas
the factor $f_{1,t}$ does have predictive content. Now, write the companion
form:%
\begin{equation*}
W_{t{\LARGE +}1}=AW_{t}+\varepsilon _{t{\LARGE +}1},
\end{equation*}%
where%
\begin{equation*}
W_{t}=\left( 
\begin{array}{c}
y_{t} \\ 
f_{1,t} \\ 
f_{2,t}%
\end{array}%
\right) \text{, }\varepsilon _{t}=\left( 
\begin{array}{c}
\varepsilon _{t{\LARGE +}1}^{Y} \\ 
\varepsilon _{1,t{\LARGE +}1}^{F} \\ 
\varepsilon _{2,t{\LARGE +}1}^{F}%
\end{array}%
\right) \text{, and }A=\left( 
\begin{array}{ccc}
a_{YY} & \alpha _{YF,1} & 0 \\ 
a_{FY,1} & a_{FF,11} & a_{FF,12} \\ 
a_{FY,2} & a_{FF,21} & a_{FF,22}%
\end{array}%
\right) ,
\end{equation*}%
Here, under Assumption 2-1, we have the vector moving-average representation:%
\begin{equation*}
W_{t{\LARGE +}1}=\dsum\limits_{j=0}^{\infty }A^{j}\varepsilon _{t{\LARGE +}%
1},
\end{equation*}%
It follows that the components of $W_{t{\LARGE +}1}$ have the univariate MA
representations: 
\begin{equation*}
y_{t{\LARGE +}1}=\dsum\limits_{j=0}^{\infty }e_{1}^{{\Large \prime }%
}A^{j}\varepsilon _{t{\LARGE +}1-j}\text{, }f_{1,t}=\dsum\limits_{k=0}^{%
\infty }e_{2}^{{\Large \prime }}A^{k}\varepsilon _{t-k}\text{, and }%
f_{2,t}=\dsum\limits_{k=0}^{\infty }e_{3}^{{\Large \prime }}A^{k}\varepsilon
_{t-k},
\end{equation*}%
with $e_{1}=\left( 
\begin{array}{ccc}
1 & 0 & 0%
\end{array}%
\right) ^{{\Large \prime }}$, $e_{2}=\left( 
\begin{array}{ccc}
0 & 1 & 0%
\end{array}%
\right) ^{{\Large \prime }}$, and $e_{3}=\left( 
\begin{array}{ccc}
0 & 0 & 1%
\end{array}%
\right) ^{{\Large \prime }}$. Let $Z_{it}$ and $Z_{jt}$ be, respectively,
the $i^{th}$ and the $j^{th}$ components of $Z_{t}$ (with $i\neq j$).
Suppose that $Z_{it}$ loads only on the second factor but not the first, so
that $\gamma _{i}^{{\large \prime }}=\left( 
\begin{array}{cc}
0 & \gamma _{i2}%
\end{array}%
\right) $, where $\gamma _{i2}\neq 0$;\ and suppose that $Z_{jt}$ loads only
on the first factor but not the second, so that $\gamma _{j}^{{\large \prime 
}}=\left( 
\begin{array}{cc}
\gamma _{j1} & 0%
\end{array}%
\right) $, where $\gamma _{j1}\neq 0.$ Hence, both $Z_{it}$ and $Z_{jt}$ are
relevant variables for factor estimation, but $Z_{jt}$ has predictive
content for $y_{t{\LARGE +}1}$ whereas $Z_{it}$ does not. Consider the score
statistics associated with $Z_{it}$ and $Z_{jt}:$ 
\begin{eqnarray*}
S_{i} &=&\dsum\limits_{t=1}^{T-1}Z_{it}y_{t{\LARGE +}1}=\dsum%
\limits_{t=1}^{T-1}\left( \gamma _{i}^{{\large \prime }}F_{t}+u_{it}\right)
y_{t{\LARGE +}1}=\dsum\limits_{t=1}^{T-1}\gamma _{i2}f_{2,t}y_{t{\LARGE +}%
1}+\dsum\limits_{t=1}^{T-1}u_{it}y_{t{\LARGE +}1}\text{ and} \\
S_{j} &=&\dsum\limits_{t=1}^{T-1}Z_{jt}y_{t{\LARGE +}1}=\dsum%
\limits_{t=1}^{T-1}\left( \gamma _{j}^{{\large \prime }}F_{t}+u_{jt}\right)
y_{t{\LARGE +}1}=\dsum\limits_{t=1}^{T-1}\gamma _{j1}f_{1,t}y_{t{\LARGE +}%
1}+\dsum\limits_{t=1}^{T-1}u_{jt}y_{t{\LARGE +}1}.
\end{eqnarray*}%
Note that, when both $\sigma _{21}\neq 0$ and $\sigma _{31}\neq 0$, where $%
\sigma _{21}$ and $\sigma _{31}$ are, respectively, the $\left( 2,1\right)
^{th}$ and the $\left( 3,1\right) ^{th}$ elements of the error covariance
matrix $\Sigma _{\varepsilon }$, the expected values of $S_{i}$ and $S_{j}$
will not, in general, be properly centered at zero, i.e., 
\begin{eqnarray*}
E\left[ S_{i}\right] &=&\dsum\limits_{t=1}^{T-1}E\left[ Z_{it}y_{t{\LARGE +}%
1}\right] \\
&=&\gamma _{i2}\dsum\limits_{t=1}^{T-1}E\left[ f_{2,t}y_{t{\LARGE +}1}\right]
+\dsum\limits_{t=1}^{T-1}E\left[ u_{it}y_{t{\LARGE +}1}\right] \\
&=&\gamma _{i2}\dsum\limits_{t=1}^{T-1}\dsum\limits_{k=0}^{\infty
}\dsum\limits_{\ell =0}^{\infty }e_{3}^{{\Large \prime }}A^{k}E\left[
\varepsilon _{t-k}\varepsilon _{t{\LARGE +}1-\ell }^{{\Large \prime }}\right]
\left( A^{{\Large \prime }}\right) ^{\ell }e_{1} \\
&=&\gamma _{i2}\dsum\limits_{t=1}^{T-1}\dsum\limits_{k=0}^{\infty }e_{3}^{%
{\Large \prime }}A^{k}\Sigma _{\varepsilon }\left( A^{{\Large \prime }%
}\right) ^{k{\LARGE +}1}e_{1} \\
&\neq &0
\end{eqnarray*}%
and%
\begin{eqnarray*}
E\left[ S_{j}\right] &=&\dsum\limits_{t=1}^{T-1}E\left[ Z_{jt}y_{t{\LARGE +}%
1}\right] \\
&=&\gamma _{j1}\dsum\limits_{t=1}^{T-1}E\left[ f_{1,t}y_{t{\LARGE +}1}\right]
+\dsum\limits_{t=1}^{T-1}E\left[ u_{jt}y_{t{\LARGE +}1}\right] \\
&=&\gamma _{j1}\dsum\limits_{t=1}^{T-1}\dsum\limits_{k=0}^{\infty
}\dsum\limits_{\ell =0}^{\infty }e_{2}^{{\Large \prime }}A^{k}E\left[
\varepsilon _{t-k}\varepsilon _{t{\LARGE +}1-\ell }^{{\Large \prime }}\right]
\left( A^{{\Large \prime }}\right) ^{\ell }e_{1} \\
&=&\gamma _{j1}\dsum\limits_{t=1}^{T-1}\dsum\limits_{k=0}^{\infty }e_{2}^{%
{\Large \prime }}A^{k}\Sigma _{\varepsilon }\left( A^{{\Large \prime }%
}\right) ^{k{\LARGE +}1}e_{1} \\
&\neq &0
\end{eqnarray*}%
Hence, both statistics, when appropriately normalized, will diverge with
probability approaching one, as $T\rightarrow \infty .$ This makes the right
inference about the relevance of both of these variables, since the
divergence of these statistics implies that the null hypothesis $%
H_{0}:\gamma _{i}=0$ (i.e., $Z_{it}$ is irrelevant) as well as the null
hypothesis $H_{0}:\gamma _{j}=0$ (i.e., $Z_{jt}$ is irrelevant) will both be
rejected with probability approaching one. However, if we were to interpret
these statistics as providing inference about the predictive content of the
variables $Z_{it}$ and $Z_{jt}$; then, we would have made the wrong
inference about $Z_{it}$, since it loads only on $f_{2,t}$ which is not
helpful in predicting $y_{t{\LARGE +}1}$. On the other hand, suppose instead
that $\gamma _{i2}=0$ and $\gamma _{j1}=0$ so that $\gamma _{i}^{{\large %
\prime }}=\left( 
\begin{array}{cc}
0 & \gamma _{i2}%
\end{array}%
\right) =\left( 
\begin{array}{cc}
0 & 0%
\end{array}%
\right) $ and $\gamma _{j}^{{\large \prime }}=\left( 
\begin{array}{cc}
\gamma _{j1} & 0%
\end{array}%
\right) =\left( 
\begin{array}{cc}
0 & 0%
\end{array}%
\right) $, and, thus, both $Z_{it}$ and $Z_{jt}$ are now irrelevant
variables. Then, under this alternative scenario, we would have%
\begin{eqnarray*}
E\left[ S_{i}\right] &=&\gamma
_{i2}\dsum\limits_{t=1}^{T-1}\dsum\limits_{j=1}^{\infty }e_{3}^{{\Large %
\prime }}A^{j-1}\Sigma _{\varepsilon }\left( A^{{\Large \prime }}\right)
^{j}e_{1}=0 \\
E\left[ S_{j}\right] &=&\gamma
_{j1}\dsum\limits_{t=1}^{T-1}\dsum\limits_{j=1}^{\infty }e_{2}^{{\Large %
\prime }}A^{j-1}\Sigma _{\varepsilon }\left( A^{{\Large \prime }}\right)
^{j}e_{1}=0
\end{eqnarray*}%
so that both statistics are now properly centered at zero, and neither will
diverge, when appropriately normalized, as $T\rightarrow \infty $. Hence,
under this alternative scenario, given an appropriate threshold or critical
value, we will also make the correct inference asymptotically about the fact
that both $Z_{it}$ and $Z_{jt}$ are irrelevant variables in this case.

Note that, for ease of presentation, we have constructed this example based
on a simple score statistic for which construction does not involve a
blocking scheme or self-normalization. Of course, the same story holds for
the more complicated score statistics discussed in this paper. Formal
results showing that our self-normalized statistics correctly identify the
relevant variables with probability approaching one are given below in
Theorems 1 and 2.

\noindent \textbf{(b) }To understand why using the quantile function of the
standard normal as the threshold function for our procedure is a natural
choice, note first that, by a slight modification of the arguments given in
the proof of Lemma A2\footnote{%
The statement and proof of Lemma A2 are provided below in the Appendix to
this paper.}, we can show that, as $T\rightarrow \infty $%
\begin{equation}
P\left( \left\vert S_{i,\ell ,T}\right\vert \geq z\right) =2\left[ 1-\Phi
\left( z\right) \right] \left( 1+o\left( 1\right) \right) ,
\label{moderate dev result}
\end{equation}%
which holds for all $i$ and $\ell $ and for all $z$ such that

\noindent $0\leq z\leq c_{0}\min \left\{ T^{\left( 1-{\large \alpha }%
_{1}\right) /6}/L\left( T\right) ,T^{\alpha _{2}/2}\right\} $, where $%
L\left( T\right) $ denotes a slowly varying function such that $L\left(
T\right) \rightarrow \infty $ but $L\left( T\right) /T^{\left( 1-{\large %
\alpha }_{1}\right) /6}\rightarrow 0$ as $T\rightarrow \infty $. In view of
expression (\ref{moderate dev result}), we can interpret moderate deviation
as providing an asymptotic approximation of the (two-sided) tail behavior of
the self-normalized statistic, $S_{i,\ell ,T},$ based on the tails of the
standard normal distribution. An important advantage of using
self-normalized statistics in this context is that the range for which this
standard normal approximation is valid (i.e., the range $0\leq z\leq
c_{0}\min \left\{ T^{\left( 1-{\large \alpha }_{1}\right) /6}/L\left(
T\right) ,T^{\alpha _{2}/2}\right\} $) is wider for self-normalized
statistics relative to their non-self-normalized counterparts. Now, suppose
initially that we wish simply to control the probability of a Type I error
for testing the null hypothesis $H_{0}:\gamma _{i}=0$ (i.e., the $i^{th}$
variable does not load on the underlying factors) at some fixed significance
level $\alpha $. Then, expression (\ref{moderate dev result}) suggests that
a natural way to do this is to set $z=\Phi ^{-1}\left( 1-\alpha /2\right) $.
This is because, given that the quantile function $\Phi ^{-1}\left( \cdot
\right) $ is, by definition, the inverse function of the cdf $\Phi \left(
\cdot \right) $, we have that: 
\begin{equation*}
P\left( \left\vert S_{i,\ell ,T}\right\vert \geq \Phi ^{-1}\left( 1-\alpha
/2\right) \right) =2\left[ 1-\Phi \left( \Phi ^{-1}\left( 1-\alpha /2\right)
\right) \right] \left( 1+o\left( 1\right) \right) =\alpha \left( 1+o\left(
1\right) \right) ,
\end{equation*}%
so that the probability of a Type I error is controlled at the desired level 
$\alpha $ asymptotically. Note also that an advantage of moderate deviation
theory is that it gives a characterization of the relative approximation
error, as opposed to the absolute approximation error. As a result, the
approximation given is useful and meaningful even when $\alpha $ is very
small, which is of importance to us since we are interested in situations
where we might want to let $\alpha $ go to zero, as sample size approaches
infinity.

We give the above example to provide some intuition concerning the form of
the threshold function that we have specified. The variable selection
problem that we actually consider is more complicated than what is
illustrated by this example, since we need to control the probability of a
Type I error (or of a false positive) not just for a single test involving
the $i^{th}$ variable but for all variables simultaneously. Moreover, as
noted previously, we also need the probability of a false positive to go to
zero asymptotically, if we want to be able to estimate the factors
consistently, even up to an invertible matrix transformation. We show in
Theorem 1 below that these objectives can all be accomplished using the
threshold function specified in expression (\ref{var selection decision rule}%
), since a threshold function of this form makes it easy for us to properly
control the probability of a false positive in large samples.

\noindent \textbf{(c)} The threshold function used here is reminiscent of
the one employed in Belloni, Chen, Chernozhukov, and Hansen (2012) and
further studied in Belloni, Chernozhukov, and Hansen (2014). The latter
paper focuses on developing a variable screening methodology for a partially
linear treatment effects model. In that paper, a threshold function that is
similar to ours is used to set the penalty level for a lasso-based procedure
for selecting the terms in a series expansion of the nonlinear component of
their model under conditions of sparsity. In spite of the similarity in the
form of the threshold function used, the nature of the variable selection
problem studied in the two above papers is quite different from that
investigated in our paper. In particular, Belloni, Chenozhukov, and Hansen
(2014) do not require their variable selection procedure to be completely
consistent, nor do they provide a result showing that the probability of
both Type I and Type II error vanishes asymptotically as sample sizes
approach infinity. As noted in Belloni, Chernozhukov, and Hansen (2014),
perfect variable selection is not needed in the type of regression settings
considered in their paper if the goal is to approximate the nonlinear
functions in their model sufficiently well so that the post-selection
estimators of the treatment effect parameter will have good asymptotic
properties. Here, we instead argue that having a variable selection
procedure that is completely consistent is quite useful given our objective
of ensuring that good factor estimates can be obtained in a high-dimensional
latent factor model. This is because, as noted earlier, if the probability
of a Type I error is only controlled at some fixed nonzero level
asymptotically, then consistent factor estimation may not be possible. In
addition, the precision with which the latent factors are estimated will be
reduced if we have a variable selection procedure where the probability of a
Type II error does not go to zero. As a result of these differences in setup
and objectives, the conditions that we specify for setting the tuning
parameter $\varphi $\ will also be quite different from those in Belloni,
Chen, Chernozhukov, and Hansen (2012) and Belloni, Chernozhukov, and Hansen
(2014).

Under appropriate conditions, the variable selection procedure described
above can be shown to be consistent, in the sense that both the probability
of a false positive, i.e. $P\left( i\in \widehat{H}^{c}|i\in H\right) $, and
the probability of a false negative, i.e., $P\left( i\in \widehat{H}|i\in
H^{c}\right) $, approach zero as $N_{1},N_{2},T\rightarrow \infty $. To show
this result, we must first state a number of additional assumptions.

\noindent \textbf{Assumption 2-7: }There exists a positive constant $%
\underline{c}$ such that for all $\tau \geq 1$ and $\tau _{1}\geq 1$: 
\begin{equation*}
\min_{1\leq \ell \leq d}\min_{i\in H}\min_{r\in \left\{ 1,...,q\right\}
}E\left\{ \left[ \frac{1}{\sqrt{\tau _{1}}}\dsum\limits_{t=\left( r-1\right)
\tau +p}^{\left( r-1\right) \tau +\tau _{1}+p-1}y_{\ell ,t{\LARGE +}1}u_{it}%
\right] ^{2}\right\} \geq \underline{c},
\end{equation*}%
where, as defined earlier, $\tau _{1}=\left\lfloor T_{0}^{\alpha _{{\large 1}%
}}\right\rfloor $, $\tau _{2}=\left\lfloor T_{0}^{\alpha _{{\large 2}%
}}\right\rfloor $ for $1>\alpha _{1}\geq \alpha _{2}>0$ and $q=\left\lfloor 
\frac{T_{0}}{\tau _{1}+\tau _{2}}\right\rfloor $, and $T_{0}=T-p+1$.

\noindent \textbf{Assumption 2-8: }Let $i\in H^{c}=\left\{ k\in \left\{
1,....,N\right\} :\gamma _{k}\neq 0\right\} $. Suppose that there exists a
positive constant, $\underline{c},$ such that, for all $N_{1},N_{2},$and $T$
sufficiently large:%
\begin{eqnarray*}
&&\min_{1\leq \ell \leq d}\min_{{\large i\in }H^{{\large c}}}\left\vert 
\frac{\mu _{i,\ell ,T}}{q\tau _{1}}\right\vert \\
&=&\min_{1\leq \ell \leq d}\min_{{\large i\in }H^{{\large c}}}\left\vert 
\frac{1}{q}\dsum\limits_{r=1}^{q}\frac{1}{\tau _{1}}\dsum\limits_{t=\left(
r-1\right) \tau +p}^{\left( r-1\right) \tau +\tau _{1}+p-1}\gamma
_{i}^{\prime }\left\{ E\left[ \underline{F}_{t}\right] \mu _{Y,\ell }+E\left[
\underline{F}_{t}\underline{Y}_{t}^{\prime }\right] \alpha _{YY,\ell }+E%
\left[ \underline{F}_{t}\underline{F}_{t}^{\prime }\right] \alpha _{YF,\ell
}\right\} \right\vert \\
&\geq &\underline{c}>0,
\end{eqnarray*}%
where $\mu _{Y,\ell }=e_{\ell ,d}^{\prime }\mu _{Y}$, $\alpha _{YY,\ell
}=A_{YY}^{\prime }e_{\ell ,d}$, and $\alpha _{YF,\ell }=A_{YF}^{\prime
}e_{\ell ,d}.$ Here, $e_{\ell ,d}$ is a $d\times 1$ elementary vector whose $%
\ell ^{th}$ component is $1$ and all other components are $0$.

\noindent \textbf{Assumption 2-9: }Suppose that, as $N_{1}$, $N_{2}$, and $%
T\rightarrow \infty $, the following rate conditions hold:

\begin{enumerate}
\item[(a)] $\sqrt{\ln N}/\min \left\{ T^{\left( 1-{\large \alpha }%
_{1}\right) /6},T^{\alpha _{2}/2}\right\} \rightarrow 0$, where $1>\alpha
_{1}\geq \alpha _{2}>0$ and $N=N_{1}+N_{2}$.

\item[(b)] $N_{1}/T^{3\alpha _{{\large 1}}}\rightarrow 0$ where $\alpha _{1}$
is as defined in part (a) above.
\end{enumerate}

\noindent \textbf{Assumption 2-10: }Let $\varphi $ satisfy the following two
conditions: (a) $\varphi \rightarrow 0$ as $N_{1},N_{2}\rightarrow \infty $,
and (b) there exists some constant $a>0,$ such that $\varphi \geq 1/N^{a},$
for all $N_{1},N_{2}$ sufficiently large.

\smallskip

\noindent \textbf{Remark 2.3: }Assumption 2-8 imposes the condition that
there exists a positive constant, $\underline{c},$ such that, for all $%
N_{1},N_{2},$ and $T$ sufficiently large: 
\begin{eqnarray*}
&&\min_{1\leq \ell \leq d}\min_{{\large i\in }H^{{\large c}}}\left\vert 
\frac{1}{q}\dsum\limits_{r=1}^{q}\frac{1}{\tau _{1}}\dsum\limits_{t=\left(
r-1\right) \tau +p}^{\left( r-1\right) \tau +\tau _{1}+p-1}\gamma
_{i}^{\prime }\left\{ E\left[ \underline{F}_{t}\right] \mu _{Y,\ell }+E\left[
\underline{F}_{t}\underline{Y}_{t}^{\prime }\right] \alpha _{YY,\ell }+E%
\left[ \underline{F}_{t}\underline{F}_{t}^{\prime }\right] \alpha _{YF,\ell
}\right\} \right\vert \\
&\geq &\underline{c}>0\text{.}
\end{eqnarray*}%
This is a fairly mild condition which allows us to differentiate the
alternative hypothesis, $i\in H^{c},$ from the null hypothesis, $i\in H,$
since if $i\in H$, then it is clear that:%
\begin{equation*}
\frac{\mu _{i,\ell ,T}}{q\tau _{1}}=\frac{1}{q}\dsum\limits_{r=1}^{q}\frac{1%
}{\tau _{1}}\dsum\limits_{t=\left( r-1\right) \tau +p}^{\left( r-1\right)
\tau +\tau _{1}+p-1}\gamma _{i}^{\prime }\left\{ E\left[ \underline{F}_{t}%
\right] \mu _{Y,\ell }+E\left[ \underline{F}_{t}\underline{Y}_{t}^{\prime }%
\right] \alpha _{YY,\ell }+E\left[ \underline{F}_{t}\underline{F}%
_{t}^{\prime }\right] \alpha _{YF,\ell }\right\} =0,
\end{equation*}%
given that $\gamma _{i}=0$. Note that this assumption does rule out certain
specialized situations, such as the case when $\mu _{Y,\ell }=0$, $\alpha
_{YY,\ell }=0$, and $\alpha _{YF,\ell }=0,$ for some $\ell \in \left\{
1,...,d\right\} $. However, we do not consider such cases to be of much
practical interest since, for example, if $\mu _{Y,\ell }=0$, $\alpha
_{YY,\ell }=0$, and $\alpha _{YF,\ell }=0$ for some $\ell $ then expression (%
\ref{Y component FAVAR}) above implies that the $\ell ^{th}$ component of $%
Y_{t{\LARGE +}1}$ will have the representation $y_{\ell ,t{\LARGE +}1}=\mu
_{Y,\ell }+\underline{Y}_{t}^{\prime }\alpha _{YY,\ell }+\underline{F}%
_{t}^{\prime }\alpha _{YF,\ell }+\varepsilon _{\ell ,t{\LARGE +}%
1}^{Y}=\varepsilon _{\ell ,t{\LARGE +}1}^{Y}$, so that, in this case, $%
y_{\ell ,t{\LARGE +}1}$ depends neither on $\underline{Y}_{t}=\left(
Y_{t}^{\prime },Y_{t-1}^{\prime },...,Y_{t-p{\LARGE +}1}^{\prime }\right)
^{\prime }$ nor on $\underline{F}_{t}=\left( F_{t}^{\prime },F_{t-1}^{\prime
},...,F_{t-p{\LARGE +}1}^{\prime }\right) $. This is, of course, an
unrealistic model for $y_{\ell ,t{\LARGE +}1}$ since it would not even be a
dependent process in this case.

\medskip

The following two theorems give our main theoretical results on the variable
selection procedure described above.

\noindent \textbf{Theorem 1: }\textit{Let\ }$H=\left\{ k\in \left\{
1,....,N\right\} :\gamma _{k}=0\right\} $\textit{. Suppose that Assumptions
2-1, 2-2, 2-3, 2-4, 2-5, 2-6, 2-7, 2-9 (a) and 2-10 hold. Let }$\Phi
^{-1}\left( \cdot \right) $\textit{\ denote the inverse of the cumulative
distribution function of the standard normal random variable, or,
alternatively, the quantile function of the standard normal distribution.
Then the following statements are true:}

\begin{enumerate}
\item[(a)] \textit{Let }$\left\{ \varpi _{\ell }:\ell =1,..,d\right\} $%
\textit{\ be pre-specified weights such that }$\varpi _{\ell }\geq 0$\textit{%
\ for every }$\ell \in \left\{ 1,...,d\right\} $\textit{\ and }$%
\dsum\nolimits_{\ell =1}^{d}\varpi _{\ell }=1$\textit{, then:}%
\begin{equation*}
P\left( \max_{{\large i\in }H}\dsum\limits_{\ell =1}^{d}\varpi _{\ell
}\left\vert S_{i,\ell ,T}\right\vert \geq \Phi ^{-1}\left( 1-\frac{\varphi }{%
2N}\right) \right) =O\left( \frac{N_{2}\varphi }{N}\right) =o\left( 1\right) 
\text{,}
\end{equation*}%
\textit{where }$N=N_{1}+N_{2}$\textit{.}

\item[(b)] 
\begin{equation*}
P\left( \max_{{\large i\in }H}\max_{1\leq \ell \leq d}\left\vert S_{i,\ell
,T}\right\vert \geq \Phi ^{-1}\left( 1-\frac{\varphi }{2N}\right) \right)
=O\left( \frac{N_{2}\varphi }{N}\right) =o\left( 1\right) \text{. }
\end{equation*}
\end{enumerate}

\noindent \textbf{Theorem 2: }\textit{Let }$H^{c}=\left\{ k\in \left\{
1,....,N\right\} :\gamma _{k}\neq 0\right\} $\textit{. Suppose that
Assumptions 2-1, 2-2, 2-3, 2-5, 2-6, 2-8, 2-9, and 2-10 hold. Then the
following statements are true.}

\begin{enumerate}
\item[(a)] \textit{Let }$\left\{ \varpi _{\ell }:\ell =1,..,d\right\} $%
\textit{\ be pre-specified weights such that }$\varpi _{\ell }\geq 0$\textit{%
\ for every }$\ell \in \left\{ 1,...,d\right\} $\textit{\ and }$%
\dsum\nolimits_{\ell =1}^{d}\varpi _{\ell }=1$\textit{, then: }%
\begin{equation*}
P\left( \min_{{\large i\in }H^{{\large c}}}\dsum\limits_{\ell =1}^{d}\varpi
_{\ell }\left\vert S_{i,\ell ,T}\right\vert \geq \Phi ^{-1}\left( 1-\frac{%
\varphi }{2N}\right) \right) \rightarrow 1\text{.}
\end{equation*}

\item[(b)] 
\begin{equation*}
P\left( \min_{{\large i\in }H^{{\large c}}}\max_{1\leq \ell \leq
d}\left\vert S_{i,\ell ,T}\right\vert \geq \Phi ^{-1}\left( 1-\frac{\varphi 
}{2N}\right) \right) \rightarrow 1\text{.}
\end{equation*}
\end{enumerate}

\noindent \textbf{Remark 2.4:}\noindent \noindent

\noindent \textbf{(a)} Theorem 1 shows that, for both of our statistics, the
probability of a false positive approaches zero uniformly over all ${\large %
i\in }$ $H$ as $N_{1},N_{2},T\rightarrow \infty $. The results of Theorem 2
further imply that, for both of our statistics, the probability of a false
negative also approaches zero, uniformly over all ${\large i\in }$ $H^{%
{\large c}}$ as $N_{1},N_{2},T\rightarrow \infty $. Together, these two
theorems show that our procedure is (completely) consistent in the sense
that the probability of committing a misclassification error vanishes as $%
N_{1},N_{2},T\rightarrow \infty $.

\noindent \textbf{(b) }Note that our variable selection procedure also
delivers a consistent estimate of $N_{1}$ (i.e., $\widehat{N}_{1})$; this is
shown in Lemma D-15 part (a) of Chao, Qiu, and Swanson (2023), where we
establish that $\widehat{N}_{1}/N_{1}\overset{p}{\rightarrow }1$. The
estimator $\widehat{N}_{1}$ is useful to applied researchers implementing
the methodology developed in this paper, and also to empiricists interested
in assessing the rate condition for consistent factor estimation, given in
Assumption A4 of Bai and Ng (2021). This is another way in which the methods
developed in this paper built upon the work of Bai and Ng (2021).

\noindent \textbf{(c) }In addition, note that knowledge of the number of
factors is not needed to implement our variable selection procedure. In the
case where the number of factors needs to be determined empirically, an
applied researcher can first use our procedure to select the relevant
variables and then apply an information criterion such as that proposed in
Bai and Ng (2002) to estimate the number of factors.

\section{\noindent Monte Carlo Study}

In this section, we report some simulation results on the finite sample
performance of our variable selection procedure. The model used in the Monte
Carlo study is the following tri-variate FAVAR(1) process: 
\begin{eqnarray}
W_{t} &=&\mu +AW_{t-1}+\varepsilon _{t},  \label{W eqn} \\
Z_{t} &=&\gamma F_{t}+u_{t}\text{,}  \label{Z eqn}
\end{eqnarray}%
where%
\begin{equation*}
W_{t}=\left( 
\begin{array}{c}
Y_{1t} \\ 
Y_{2t} \\ 
F_{t}%
\end{array}%
\right) \text{, }\mu =\left( 
\begin{array}{c}
2 \\ 
1 \\ 
2%
\end{array}%
\right) \text{, }A=\left( 
\begin{array}{ccc}
0.9 & 0.3 & 0.5 \\ 
0 & 0.7 & 0.1 \\ 
0 & 0.6 & 0.7%
\end{array}%
\right) \text{, and }\gamma =\left( 
\begin{array}{c}
\iota _{N_{1}} \\ 
\underset{N_{2}\times 1}{0}%
\end{array}%
\right) ,
\end{equation*}%
with $\iota _{N_{1}}$ denoting an $N_{1}\times 1$ vector of ones. We
consider different configurations of $N$, $N_{1}$, and $T,$ as given below.
For the error process in equation (\ref{W eqn}), we take $\left\{
\varepsilon _{t}\right\} \equiv i.i.d.N\left( 0,\Sigma _{\varepsilon
}\right) $, where: 
\begin{equation*}
\Sigma _{\varepsilon }=\left( 
\begin{array}{ccc}
1.3 & 0.99 & 0.641 \\ 
0.99 & 0.81 & 0.009 \\ 
0.641 & 0.009 & 5.85%
\end{array}%
\right) \text{.}
\end{equation*}%
The error process, $\left\{ u_{it}\right\} ,$ in equation (\ref{Z eqn}) is
allowed to exhibit both temporal and cross-sectional dependence and also
conditional heteroskedasticity. More specifically, we let $%
u_{it}=0.8u_{it-1}+\zeta _{it}$, and following the approach for modeling
cross-sectional dependence given in the Monte Carlo design of Stock and
Watson (2002a), we specify: $\zeta _{it}=\left( 1+b^{2}\right) \eta
_{it}+b\eta _{i+1,t}+b\eta _{i-1,t}$, and set $b=1$. In addition, $\eta
_{it}=\omega _{it}\xi _{it},$ with $\left\{ \xi _{it}\right\} \equiv
i.i.d.N\left( 0,1\right) $ independent of $\left\{ \varepsilon _{t}\right\} $%
, and $\omega _{it}$ follows a GARCH(1,1) process given by: $\omega
_{it}^{2}=1+0.9\omega _{it-1}^{2}+0.05\eta _{it-1}^{2}$. To study the
effects of varying the tuning parameter, we consider specifications where $%
\varphi =\left( \ln \ln N\right) ^{-\vartheta }$ for $\vartheta =0.1,0.5,1$
and also $\varphi =N^{-\vartheta }$ for $\vartheta =0.2,0.4,0.6$.\footnote{%
We have also obtained simulation results for the cases where $\varphi
=\left( \ln N\right) ^{-\vartheta }$ for $\vartheta =0.1,0.5,1$ and where $%
\varphi =N^{-\vartheta }$ for $\vartheta =0.3,0.5,0.7$. The results obtained
for these cases are qualitatively similar to the results reported in this
paper. Hence, due to space considerations, we do not report these results
here, but they are available from the authors upon request.} We also attempt
to shed light on the effects of using blocks of different sizes on the
performance of our procedure. To do this, for $T=100$, we set $\tau _{1}=2$, 
$3$, $4$, and $5$; for $T=200$, we set $\tau _{1}=5$, $6$, $8$, and $10$;
and for $T=600$, we set $\tau _{1}=6$, $8$, $10$, and $12$. Due to space
considerations, we only report Monte Carlo results for the statistic $%
\dsum\nolimits_{\ell =1}^{d}\varpi _{\ell }\left\vert S_{i,\ell
,T}\right\vert $. Simulation results for the statistic $\max_{1\leq \ell
\leq d}\left\vert S_{i,\ell ,T}\right\vert $ have also been obtained by the
authors and are qualitatively similar to the results reported here for $%
\dsum\nolimits_{\ell =1}^{d}\varpi _{\ell }\left\vert S_{i,\ell
,T}\right\vert $. The results for $\max_{1\leq \ell \leq d}\left\vert
S_{i,\ell ,T}\right\vert $ are available from the authors upon request. In
addition, since $d=2$ in our Monte Carlo setup, we set $\varpi _{1}=\varpi
_{2}=1/2$. Results are gathered in Table 1, where FPR denotes the
\textquotedblleft False Positive Rate\textquotedblright\ or the
\textquotedblleft Type I\textquotedblright\ error rate, i.e., the proportion
of cases where an irrelevant variable $Z_{it}$, with associated coefficient $%
\gamma _{i}=0$ is erroneously selected as a relevant variable. FNR denotes
the \textquotedblleft False Negative Rate\textquotedblright\ or the
\textquotedblleft Type II\textquotedblright\ error rate, i.e., the
proportion of cases where a relevant variable is erroneously identified as
being irrelevant.

Looking across each row of the table, note that FPRs decrease when moving
from left to right, whereas FNRs increase. This is not surprising, because
moving from $\varphi =\left( \ln \ln N\right) ^{-0.1}$ to $\varphi =N^{-0.6}$
for a given $N$ results in smaller values of the tuning parameter $\varphi $%
, and the specified threshold $\Phi ^{-1}\left( 1-\frac{\varphi }{2N}\right) 
$ thus becomes larger. Overall, these results indicate that choosing $%
\varphi $ in the range between $\left( \ln \ln N\right) ^{-0.1}$ and $%
N^{-0.4}$\ leads to very good performance, since within this range, neither
FPR nor FNR exceeds $0.1$ in any of the cases studied here. In fact, both
are smaller than $0.05$ in a vast majority of the cases. In contrast,
choosing $\varphi =N^{-0.6}$ can lead to high FNRs, as such a choice of $%
\varphi $ can set our threshold at such a high level that our procedure ends
up having very little power.

Looking down the columns of the table, note that FPR tends to increase as $%
\tau _{1}$ increases, whereas FNR tends to decrease as $\tau _{1}$
increases. As an explanation for this result, note first that the smaller is 
$\tau _{1}$ relative to $\tau $, the larger is $\tau _{2}$ (since $\tau
=\tau _{1}+\tau _{2}$), and thus the larger is the number of observations
removed when constructing the self-normalized block sums. Intuitively, this
can lead to better accommodation of the effects of dependence and better
moderate deviation approximations under the null hypothesis, resulting in a
lower FPR. However, removal of a larger number of observations can also lead
to a reduction in power, when the alternative hypothesis is correct, so that
a negative consequence of having a smaller $\tau _{1}$ relative to $\tau $
is that FNR will tend to be higher in this case. The opposite, of course,
occurs when we try to specify a larger $\tau _{1}$ relative to $\tau $.
\begin{table}[tbp]
{{\textbf{Table 1: Monte Carlo Results for $\mathbb{S}_{i,T}^{+}=\dsum%
\nolimits_{\ell =1}^{d}\varpi _{\ell }\left\vert S_{i,\ell ,T}\right\vert $}}
\newline
\newline
\newline
\hspace{0.75in}} \vspace{0.5in} {\small 
\begin{tabular}{|l|l|c|c|c|c|c|c|}
\hline
&  & $N=100$ & $N_{1}=50$ & $T=100$ & $\tau =5$ &  &  \\ \hline
&  & \multicolumn{1}{|l|}{{\scriptsize {$\varphi =\left( \ln \ln N\right)
^{-0.1}$}}} & \multicolumn{1}{|l|}{{\scriptsize {$\varphi =\left( \ln \ln
N\right) ^{-0.5}$}}} & \multicolumn{1}{|l|}{{\scriptsize {$\varphi =\left(
\ln \ln N\right) ^{-1}$}}} & \multicolumn{1}{|l|}{$\varphi =N^{-0.2}$} & $%
\varphi =N^{-0.4}$ & $\varphi =N^{-0.6}$ \\ \hline\hline
$\tau _{1}=2$ & FPR & 0.03916 & 0.03350 & 0.02678 & 0.01460 & 0.00382 & 
0.00076 \\ \hline
& FNR & 0.00046 & 0.00068 & 0.00104 & 0.00284 & 0.01674 & 0.09412 \\ \hline
$\tau _{1}=3$ & FPR & 0.04544 & 0.03902 & 0.03110 & 0.01810 & 0.00526 & 
0.00092 \\ \hline
& FNR & 0.00022 & 0.00032 & 0.00052 & 0.00172 & 0.01100 & 0.06942 \\ \hline
$\tau _{1}=4$ & FPR & 0.05408 & 0.04650 & 0.03756 & 0.02224 & 0.00702 & 
0.00162 \\ \hline
& FNR & 0.00016 & 0.00024 & 0.00034 & 0.00118 & 0.00828 & 0.05194 \\ \hline
$\tau _{1}=5$ & FPR & 0.06332 & 0.05462 & 0.04558 & 0.02796 & 0.00924 & 
0.00232 \\ \hline
& FNR & 0.00014 & 0.00018 & 0.00034 & 0.00084 & 0.00574 & 0.03948 \\ \hline
&  & $N=200$ & $N_{1}=100$ & $T=100$ & $\tau =5$ &  &  \\ \hline
$\tau _{1}=2$ & FPR & 0.01913 & 0.01470 & 0.01068 & 0.00486 & 0.00064 & 
0.00002 \\ \hline
& FNR & 0.00206 & 0.00282 & 0.00449 & 0.01415 & 0.09966 & 0.48356 \\ \hline
$\tau _{1}=3$ & FPR & 0.02341 & 0.01842 & 0.01365 & 0.00657 & 0.00098 & 
0.00005 \\ \hline
& FNR & 0.00143 & 0.00190 & 0.00315 & 0.00921 & 0.07372 & 0.40894 \\ \hline
$\tau _{1}=4$ & FPR & 0.02869 & 0.02306 & 0.01733 & 0.00841 & 0.00133 & 
0.00004 \\ \hline
& FNR & 0.00111 & 0.00145 & 0.00224 & 0.00661 & 0.05564 & 0.34279 \\ \hline
$\tau _{1}=5$ & FPR & 0.03506 & 0.02903 & 0.02194 & 0.01124 & 0.00213 & 
0.00017 \\ \hline
& FNR & 0.00086 & 0.00112 & 0.00172 & 0.00477 & 0.04258 & 0.28620 \\ \hline
&  & $N=400$ & $N_{1}=200$ & $T=200$ & $\tau =10$ &  &  \\ \hline
\multicolumn{1}{|c|}{$\tau _{1}=5$} & FPR & 0.00214 & 0.00148 & 0.00090 & 
0.00030 & 2.5$\times $10$^{-5}$ & 0.00000 \\ \hline
\multicolumn{1}{|c|}{} & FNR & 7.5$\times $10$^{-5}$ & 0.00016 & 0.00040 & 
0.00231 & 0.06894 & 0.67266 \\ \hline
\multicolumn{1}{|c|}{$\tau _{1}=6$} & FPR & 0.00249 & 0.00166 & 0.00104 & 
0.00034 & 0.00002 & 0.00000 \\ \hline
\multicolumn{1}{|c|}{} & FNR & 0.00004 & 0.00009 & 0.00025 & 0.00148 & 
0.05058 & 0.60968 \\ \hline
\multicolumn{1}{|c|}{$\tau _{1}=8$} & FPR & 0.00337 & 0.00235 & 0.00142 & 
0.00046 & 0.00004 & 0.00000 \\ \hline
\multicolumn{1}{|c|}{} & FNR & 0.00001 & 0.00002 & 0.00008 & 0.00068 & 
0.02712 & 0.48133 \\ \hline
\multicolumn{1}{|c|}{$\tau _{1}=10$} & FPR & 0.00484 & 0.00350 & 0.00220 & 
0.00079 & 7.5$\times $10$^{-5}$ & 5.0$\times $10$^{-6}$ \\ \hline
\multicolumn{1}{|c|}{} & FNR & 0.00001 & 0.00001 & 0.00002 & 0.00034 & 
0.01535 & 0.36382 \\ \hline
&  & $N=1000$ & $N_{1}=500$ & $T=600$ & $\tau =12$ &  &  \\ \hline
\multicolumn{1}{|c|}{$\tau _{1}=6$} & FPR & 0.00155 & 0.00121 & 0.00086 & 
0.00038 & 0.00006 & 0.00001 \\ \hline
\multicolumn{1}{|c|}{} & FNR & 0.00000 & 0.00000 & 0.00000 & 0.00000 & 
0.00000 & 0.00000 \\ \hline
\multicolumn{1}{|c|}{$\tau _{1}=8$} & FPR & 0.00201 & 0.00153 & 0.00106 & 
0.00049 & 8.2$\times $10$^{-5}$ & 1.4$\times $10$^{-5}$ \\ \hline
\multicolumn{1}{|c|}{} & FNR & 0.00000 & 0.00000 & 0.00000 & 0.00000 & 
0.00000 & 0.00000 \\ \hline
\multicolumn{1}{|c|}{$\tau _{1}=10$} & FPR & 0.00274 & 0.00216 & 0.00155 & 
0.00072 & 0.00016 & 3.2$\times $10$^{-5}$ \\ \hline
\multicolumn{1}{|c|}{} & FNR & 0.00000 & 0.00000 & 0.00000 & 0.00000 & 
0.00000 & 0.00000 \\ \hline
\multicolumn{1}{|c|}{$\tau _{1}=12$} & FPR & 0.00421 & 0.00332 & 0.00242 & 
0.00115 & 0.00028 & 6.0$\times $10$^{-5}$ \\ \hline
\multicolumn{1}{|c|}{} & FNR & 0.00000 & 0.00000 & 0.00000 & 0.00000 & 
0.00000 & 0.00000 \\ \hline
\end{tabular}%
} {\footnotesize 
\begin{minipage}{1\columnwidth}
{\noindent Notes: False positive and negative rates are reported for various values of 
$N, N_1,$ and $T$. Results are based on 1000 simulations. See Section 3 for complete details.}
\end{minipage}
}
\end{table}

Our results also show that when the sample sizes are large enough such as
the cases presented in the last panel of the table, where $T=600$ and $N=1000
$, then both FPR and FNR are very close to zero for all of the cases that we
consider. Moreover, even in the extreme case where $T=100$ and $N=100$, FPR
and NPR rates are usually less than 0.05, and are often much smaller than
that. This is in accord with the results of our theoretical analysis, which
shows that our variable selection procedure is completely consistent in the
sense that both the probability of a false positive and the probability of a
false negative approach zero, as the sample sizes go to infinity. 

\noindent \newpage 

\section{\noindent Empirical Illustration}

In this illustration, we forecast eight target variables from the monthly
real-time macroeconomic FRED-MD dataset maintained by St. Louis Federal
Reserve Bank. We follow the data cleaning methods outlined on the FRED-MD
data website, as well as removing all discontinued series, when
pre-processing the data, yielding a dataset, $\mathbf{X}$, containing $N=97$
variables for the period 1973:3 to 2022:9. The full list of all
macroeconomic variables and their transformations is available upon request
from the authors.

Of note is that the dataset used here is \textquotedblleft
truly\textquotedblright\ real-time, in the sense that a \textquotedblleft
vintage\textquotedblright\ of data is available at calendar date in our
sample period. Consider the value of industrial production for January 2020.
In February 2020, the government reported a \textquotedblleft first
release\textquotedblright\ value for January. In March 2020, however, they
further updated their \textquotedblleft estimate\textquotedblright\ of
industrial production for January. Namely, they reported a \textquotedblleft
second release\textquotedblright\ for January. This process of revision
continues indefinitely. Namely, as the government changes data collection
and processing methodology, collects new data and/or revises definitions of
variables, new releases are reported. A \textquotedblleft
vintage\textquotedblright\ of data is a date, say February 2020. For
industrial production, there is a whole vector of truly real-time data that
includes different releases, all available in February 2020. For example,
this vintage includes a 1st release value for January 2020, a 2nd release
value for December 2019 is included in this vintage, a 3rd release value for
November 2019, etc. In this sense, there is an entirely unique vintage of
industrial production available each month, and the values of the calendar
dated observations in each vintage change because the government updates
historical values of the variable every month. Using this type of data
allows the practitioner to truly simulate a forecasting environment in which
models are updated at each point in time using data that were actually
available at that time. If we were to simply collect industrial production
data from a website today, calendar dated observations in our dataset from
2020 would reflect revisions that occurred after 2020. For variables that
are subject to revision, this means that forecasting experiments of the type
carried out in this paper would be invalid, in the sense that they would be
utilizing \textquotedblleft future data\textquotedblright\ as explanatory
variables when estimating forecasting model regressions, if it were the case
that the correct vintage of data was not used at each point in time when
estimating models and constructing forecasts. For further discussion of the
structure of real-time datasets, as well as methods for real-time
forecasting, refer to Swanson (1996), Swanson and van Dijk (2006), and Kim
and Swanson (2018).

The eight target variables for which we construct predictions are:
Industrial Production (INDPRO), Civilian Unemployment Rate (UNRATE), Housing
Starts: new, privately owned (HOUST), Housing Permits: new, privately owned
(PERMIT), Real M2 Money Stock (M2REAL), 10-Year Government Treasury Bond
Rate (R10), CPI - All Items (CPI), S\&P Common Stock Price Index - Composite
(S\&P 500). Our experiments compare variable selection and dimension
reduction methods when used to estimate and/or select factors and observable
variables for inclusion in forecasting models of the form: 
\begin{equation}
y_{t+h}={\alpha }+{\beta _{h}}(L)y_{t}+{\gamma _{h}}(L)\boldsymbol{F_{t}}%
+\epsilon _{t+h},
\end{equation}%
where $y_{t}$ is a scalar target variable to be predicted, ${\beta _{h}}(L)$
and ${\gamma _{h}}(L)$ are finite order lag polynomials and where $\epsilon
_{t}$ is a stochastic disturbance term. Lags in this model are selected
using the Schwarz Information Criterion (SIC), and our benchmark model sets
the coefficients in $\gamma _{h}(L)=0$. In the sequel, we carry out variable
selection and dimension reduction using seven different methods.

\begin{table}[h]
{\small \noindent \textbf{Table 2: }Empirical Illustration - Target Forecast
Variables{$^{\ast }$} }
\par
{\scriptsize \ 
\begin{tabular}{p{6cm}p{2cm}p{3cm}}
\hline\hline
Target Variable & Abbreviation & Data Transformation \\ \hline
Industrial Production & INDPRO & $\Delta log(y_t)$ \\ 
Civilian Unemployment Rate & UNRATE & $y_t$ \\ 
Housing Starts (new, privately owned) & HOUST & $log(y_t)$ \\ 
Housing Permits (new, privately owned) & PERMIT & $log(y_t)$ \\ 
Real M2 Money Stock & M2REAL & $\Delta log(y_t)$ \\ 
10-Year Government Treasury Bond Rate & R10 & $y_t$ \\ 
CPI (all items) & CPI & $\Delta log(y_t)$ \\ 
S\&P Common Stock Price Index (composite) & S\&P500 & $\Delta log(y_t)$ \\ 
\hline\hline
\end{tabular}
}
\par
{\scriptsize \vspace{0.5cm} }
\par
{\scriptsize \noindent $^*$ Notes: This table lists the target forecast
variables that are predicted in our empirical illustration, \noindent and
associated data transformations. }
\end{table}

\noindent \textit{Principal Components Analysis (PCA):} Excluding the target
variable, apply PCA to $\mathbf{X}$ and estimate latent factors, $%
\boldsymbol{F_t}$, with the number of factors determined using the $PC_{p2}$
criterion in Bai and Ng (2002). The maximum number of the factors is set
equal to eight, following the findings of McCracken and Ng (2016), who
introduce and examine the dataset that we utilize in our experiments.

\noindent \textit{Hard Thresholding:} For each variable in $\mathbf{X}$, and
forecast horizon, $h$, perform a regression of $y_{t+h}$ on lags of $y_{t}$
and on $X_{i,t}$, where $X_{i,t}$ is a scalar variable in $\mathbf{X}$, for $%
i=1,...,N$, and lags of $y_{t}$ are selected using the SIC. Let $t_{i}$
denote the $t$ statistic associated with $X_{it-h}$ in the regression, and
select variables, $X_{it}$ if $|t_{i}|$ $>1.28$. If the number of selected
variables is greater than 20, utilize PCA to estimate factors for inclusion
in the above forecasting equation, otherwise use the AR(SIC) model. As
models are re-estimated at each point in time, this approach is a hybrid, in
the sense that some models may include factors as regressors, while others
may be simple AR(SIC) models. Note that in our experiments, less than 10\%
of the total number of forecasting periods involved replacing the
thresholding model with our AR(SIC) benchmark.

\noindent \textit{Chao-Swanson Variable Selection}: Use the variable
selection method introduced in this paper to select variables. Then, use PCA
to estimate factors for inclusion in the forecasting equation. There are
three tuning parameters in the CS method, including: $\tau ,\tau _{1}$, and $%
\varphi $. We set $\{\tau =5,\tau _{1}=3,5\}$ and $\{\tau =10,\tau
_{1}=6,8\} $ and consider the following values for $\varphi $: 
\begin{equation*}
\varphi =\left\{ 
\begin{array}{rclrcl}
(lnlnN)^{-0.1} & (lnlnN)^{-0.6} & (lnN)^{-0.1} & (lnN)^{-0.6} & N^{-0.1} & 
N^{-0.6} \\ 
(lnlnN)^{-0.2} & (lnlnN)^{-0.7} & (lnN)^{-0.2} & (lnN)^{-0.7} & N^{-0.2} & 
N^{-0.7} \\ 
(lnlnN)^{-0.3} & (lnlnN)^{-0.8} & (lnN)^{-0.3} & (lnN)^{-0.8} & N^{-0.3} & 
N^{-0.8} \\ 
(lnlnN)^{-0.4} & (lnlnN)^{-0.9} & (lnN)^{-0.4} & (lnN)^{-0.9} & N^{-0.4} & 
N^{-0.9} \\ 
(lnlnN)^{-0.5} & (lnlnN)^{-1} & (lnN)^{-0.5} & (lnN)^{-1} & N^{-0.5} & N^{-1}
\\ 
&  &  &  &  & 
\end{array}%
\right.
\end{equation*}%
Different tuning parameters select different numbers of variables and we
exclude tuning parameter permutations that select less than 25 variables for
use in factor construction. In this method, the tuning parameter used for
each value of $h$ and target variable is selected by partitioning a
\textquotedblleft training dataset\textquotedblright\ consisting of the
first 10 years on data in our sample into an in-sample period of 7 years and
an out-of-sample period of 3 years. The tuning parameter is set equal to
that yielding the smallest mean square forecast error (MSFE) after
constructing real-time predictions based on models estimated at each point
in time prior to the construction of each new prediction for the
out-of-sample period. Note that in our experiments, less than 10\% of the
total number of forecasting periods involved replacing the selected
variables with those from the previous period.

In summary, we carry out truly real-time $h$-month ahead predictions using
monthly data, with $h=$1, 3, 6, and 12. Our \textquotedblleft full sample
forecasting period\textquotedblright\ is 2000:1-2022:9 (when reporting
results for this period, we omit predictions for 2008:1-2008:12 and
2020:1-2020:12, in order to mitigate the influence of predictions made
during the 2008 Financial Crisis and the Covid-19 period). However, even
though some predictions are omitted in our \textquotedblleft
full-sample\textquotedblright , data from these extraordinary periods in
history still affect the estimated models used when predicting other
periods. For this reason, this first set of results, where it is clearly
seen that the impact of these periods on estimated models is severe, is not
included here, but is available upon request from the authors. In the
sequel, we report results for the out-of-sample period 2000:1-2007:12. All
factors and forecasting equations are re-estimated at each point in time,
prior to the construction of each new forecast, using rolling windows of
length 120 observations. Additionally, in-sample estimation periods used
when constructing our $h=$3, 6, and 12-step ahead forecasts are adjusted so
that the forecast period remains the same regardless of forecast horizon.

Forecasting performance is evaluated using point mean squared forecast
errors (MSFEs), where MSFE=$\frac{1}{P} \sum_{t=1}^T(y_{j,t}-\hat{y}%
_{j,t})^2 $, and $\hat{y}_{j,t}$ denotes the prediction for target variable $%
y_j$ that is made using data that are truly available in real-time at period 
$t$. In our tabulated results, MSFEs, relative to that of the benchmark
AR(SIC) model are reported. Additionally, we report the results of Giacomini
and White (GW) tests (see Giacomini and White (2006)), which can be viewed
as conditional Diebold-Mariano (DM) predictive accuracy tests (see Diebold
and Mariano (1995)). Recall that the null hypothesis of the DM test when
formulated using the conditioning approach of Giacomini and White is: $%
H_{0}: \text{E}[L(\hat{\epsilon}_{t+h}^{(1)}) | G_t ] - \text{E}[L(\hat{%
\epsilon}_{t+h}^{(2)}) | G_t ] = 0 $, where the $\hat{\epsilon}_{t+h}^{(i)}$
are prediction errors associated with model $i$, for $i=1,2$, and $G_t$
denotes the conditioning set, which includes the model and estimated
parameters. Here, $L(\cdot)$ is a quadratic loss function, and the test
statistic is $\text{DM}_{P} = P^{-1} \sum\limits_{t=1}^P \frac{d_{t+h}}{\hat{%
\sigma}_{\bar{d}}}$, where $d_{t+h} = [\hat{\epsilon}_{t+h}^{(1)}]^{2} - [%
\hat{\epsilon}_{t+h}^{(2)}]^{2}$, $\bar{d}$ denotes the mean of $d_{t+h}$, $%
\hat{\sigma}_{\bar{d}}$ is a heteroskedasticity and autocorrelation
consistent estimate of the standard deviation of $\bar{d}$, and $P$ denotes
the number of ex-ante predictions used to construct the test statistic.%
\footnote{%
In this paper, we report test results for the Wald version of this test
statistic (see Giacomini and White (2006) for further details).} If the
statistic is ``significantly negative'', then Model 1 is preferred to Model
2, and in our context, where we report relative MSFEs, rejection indicates
that the benchmark AR(SIC) model is preferred if the relative MSFE is
greater than one, and the converse if it is less than one.

Turning to our empirical findings, note first that a summary of the target
variables in our experiments is contained in Table 2. Table 3 contains
results for our full sample forecasting period. In this table, all entries
are relative MSFEs, as discussed above. Additionally, bolded entries
indicate the \textquotedblleft MSFE-best" method for a particular target
variable and forecast horizon. Since relative MSFEs are reported, however,
if the lowest relative MSFE is greater than 1, it is not bolded, as this
means that the AR(SIC) benchmark yields the MSFE-best predictions. Starred
entries denote rejection of the null hypothesis of equal forecast accuracy
when comparing the model associated with a given method against the AR(SIC)
benchmark. Turning to the results in this table, a number of conclusions can
be made.

\begin{table}[h!]
{\small \noindent \textbf{Table 3: }Empirical Illustration - Real-Time
Predictive Accuracy Experiments{$^{\ast }$} }
\par
{\scriptsize 
\begin{tabular}{llccc}
\hline\hline
&  & \multicolumn{3}{c}{{\textbf{Factor Estimation Method}}} \\ \hline
& \multicolumn{1}{c}{Target Variable} & Principal Components Analysis & Hard
Thresholding & CS Variable Selection \\ \hline
& INDPRO & \textbf{0.971} & 1.025 & 1.092 \\ 
& UNRATE & 1.098 & \textbf{0.966} & 0.984 \\ 
& HOUST & 1.000 & 0.882 ** & \textbf{0.877} \\ 
h=1 & PERMIT & 1.009 & 1.005 & 1.042 \\ 
& M2REAL & 1.06 & 1.052 * & 1.19 * \\ 
& R10 & 1.086 & 1.107 & 1.051 \\ 
& CPI & 1.08 & 1.125 & 1.161 \\ 
& S\&P500 & 1.142 & 1.104 & 1.118 \\ \hline
& INDPRO & 1.042 & 1.042 & 1.083 \\ 
& UNRATE & 0.839 & \textbf{0.691 **} & 0.785 * \\ 
& HOUST & 0.979 & 0.777 & \textbf{0.766} ** \\ 
h=3 & PERMIT & 1.021 & 0.971 & \textbf{0.895} \\ 
& M2REAL & 0.979 & \textbf{0.907} & 1.035 \\ 
& R10 & 1.187 & 1.225 & \textbf{1.109} \\ 
& CPI & 1.022 & 1.048 * & \textbf{0.993} \\ 
& S\&P500 & 1.186 & 1.213 & 1.055 \\ \hline
& INDPRO & 1.145 & 1.043 & \textbf{0.965 *} \\ 
& UNRATE & 0.561 * & \textbf{0.518 **} & 0.617 * \\ 
& HOUST & 0.818 & 0.696 * & \textbf{0.665 *} \\ 
h=6 & PERMIT & 0.855 & 0.825 & \textbf{0.744} \\ 
& M2REAL & 1.031 & 1.049 & \textbf{0.994} \\ 
& R10 & 1.513 & 1.046 & 1.064 ** \\ 
& CPI & 1.034 & 1.05 & 1.081 * \\ 
& S\&P500 & 1.126 * & 1.276 ** & 1.166 \\ \hline
& INDPRO & 1.251 * & 1.096 * & 1.004 \\ 
& UNRATE & 0.632 & 0.489 & \textbf{0.471} \\ 
& HOUST & 0.693 & 0.605 & \textbf{0.605} \\ 
h=12 & PERMIT & 0.681 & 0.650 & \textbf{0.621} \\ 
& M2REAL & 1.087 & 1.036 & \textbf{0.984} \\ 
& R10 & 0.992 & 0.641 & \textbf{0.638 *} \\ 
& CPI & 1.024 & 1.131 & 1.064 ** \\ 
& S\&P500 & 1.188 & 1.311 * & 1.046 \\ \hline\hline
\end{tabular}
}
\par
{\scriptsize \vspace{0.5cm} }
\par
{\scriptsize \noindent $^{\ast }$ Notes: See notes to Table 2. Tabulated
entries are relative mean squared forecast error (MSFEs) for our 8 target
variables, for forecast horizons of h=1,3,6, and 12 months ahead. The
AR(SIC) benchmark model is in the denominator of the reported MSFEs, so that
entries that are less than unity indicate that our factor and variable
augmented forecast regressions yield lower MSFEs than those associated with
the AR(SIC) benchmark. The forecast period is 2000:1-2007:12, and all models
are estimated prior to the construction of each monthly forecast. In cases
where at least one big data method outperfroms the AR(SIC) benchmark model,
entries in bold denote the big data method yielding the lowest relative MSFE
for a given target variable and forecast horizon. Starred entries indicate
rejection of the null hypothesis of equal conditional predictive ability, at
significance levels $p=0.01$ ($\ast \ast \ast $), $p=0.05$, ($\ast \ast $),
and $p=0.10$ ($\ast $). See Section 4 for complete details. }
\end{table}

First, comparing the standard PCA method, which does not involve variable
pre-selection, with our method (called CS hereafter), CS beats the PCA in
terms of relative MSFE in 22 out of 32 cases across all forecast horizons
and all target variables examined here. Moreover, for the longer forecast
horizons of $h=6$ and $h=12$, CS beats PCA in 12 out of 16 cases. Second,
comparing the CS method with the hard thresholding method for variable
selection (called THRESH hereafter), we see that CS wins in 19 out of 32
overall cases, and, for the longer horizons of $h=6$ and $h=12$, CS wins in
12 out of 16 cases. In addition, it should be noted that there is one tie
between CS and THRESH, so that overall CS performed as least as well if not
better than THRESH in 20 out of 32 overall cases and in 13 out of the 16
longer horizon cases. Finally, it should be noted that the CS method beats
the AR(SIC) benchmark model in 16 out of 32 overall cases, but if we focus
just on the longer horizons of $h=6$ and $h=12$, we see that our method
outperforms the benchmark model in 10 out of the 16 cases. In summary, our
method performs well overall, and performs particularly well relative to the
other two methods (and also the benchmark method) in the longer horizon
cases.

\section{\noindent Conclusion}

In this paper, we propose a new variable selection procedure based on two
alternative self-normalized score statistics and provide asymptotic analyses
showing that our procedure, based on either of these statistics, correctly
identify the set of variables which load significantly on the underlying
factors, with probability approaching one as the sample sizes go to
infinity. Our research is motivated by the observation that inconsistency in
factor estimation could result in high dimensional settings when the
conventional assumption of factor pervasiveness does not hold. Hence, in
such settings, it is particularly important to pre-screen the variables in
terms of their association with the underlying factors prior to estimation.
We conduct a small Monte Carlo study which yields encouraging evidence about
the finite sample properties of our variable selection procedure. Moreover,
we present empirical evidence suggesting that use of our procedure yields
factors that improve the forecasting performance of factor augmented
regressions, relative to the case when principal component analysis or hard
thresholding methods are used to construct factors. It is also worth noting
that in a companion paper (Chao, Qiu, and Swanson, 2023) we prove that
consistent estimation of factors (up to an invertible matrix transformation)
can be achieved by estimating factors using only those variables selected by
our method, and this is so even in situations where the standard
pervasiveness assumption does not hold. In addition, in the same paper, we
further show that by plugging factors estimated in such a manner into the
factor-augmented forecasting equation implied by the FAVAR model, the
conditional mean function of the forecasting equation can be consistently
estimated, even for the case of multi-step ahead forecasts. In sum, the
collective body of results discussed in this paper indicates that the
variable selection methodology introduced in this paper can be useful to
empirical researchers as they engage in the important tasks of factor
estimation and the construction of point forecasts based on factor-augmented
forecasting equations.

\section{\noindent Appendix: Proofs of Theorems}

\noindent \qquad This appendix contains the proofs of the main results of
the paper: Theorems 1 and 2. In addition, two key supporting lemmas, Lemmas
A1and A2, along with their proofs are also given here. Additional technical
results are available in an Online Appendix, Chao, Liu, and Swanson (2023).

\noindent

\noindent \textbf{Proof of Theorem 1: }To show part (a), first set $z=\Phi
^{-1}\left( 1-\frac{\varphi }{2N}\right) $, where $N=N_{1}+N_{2}$. Note
that, under Assumption 2-10, we can easily show that

\noindent $\Phi ^{-1}\left( 1-\frac{\varphi }{2N}\right) \leq \sqrt{2\left(
1+a\right) }\sqrt{\ln N}$, for all $N_{1},N_{2}$ sufficiently large.%
\footnote{%
An explicit proof of this result is given in Chao, Liu, and Swanson (2023).
In particular, this inequality is shown in part (b) of Lemma OA-15 in Chao,
Liu, and Swanson (2023).} By part (a) of Assumption 2-9, $\sqrt{\ln N}/\min
\left\{ T^{\left( 1-{\large \alpha }_{1}\right) /6},T^{\alpha
_{2}/2}\right\} \rightarrow 0$ as $N_{1},N_{2},T\rightarrow \infty $; this,
in turn, implies that, for some positive constant $c_{0}$, $\Phi ^{-1}\left(
1-\frac{\varphi }{2N}\right) $ satisfies the inequality constraint $0\leq
\Phi ^{-1}\left( 1-\frac{\varphi }{2N}\right) \leq c_{0}\min \left\{
T^{\left( 1-{\large \alpha }_{1}\right) /6},T^{\alpha _{2}/2}\right\} $ for
all $N_{1},N_{2},T$ sufficiently large, so that $\Phi ^{-1}\left( 1-\frac{%
\varphi }{2N}\right) $ lies within the range of values of $z$ for which the
moderate deviation inequality given in Lemma A2 holds. Thus, plugging $\Phi
^{-1}\left( 1-\frac{\varphi }{2N}\right) $ into the moderate deviation
inequality (\ref{moderate deviation bd}) given in Lemma A2 below, we see
that there exists a positive constant $A$ such that:

\noindent ${\small P}\left( \left\vert S_{i,\ell ,T}\right\vert \geq \Phi
^{-1}\left( 1-\frac{\varphi }{2N}\right) \right) $

\noindent ${\small \leq 2}\left[ 1-\Phi \left( \Phi ^{-1}\left( 1-\frac{%
\varphi }{2N}\right) \right) \right] \left\{ 1+A\left[ 1+\Phi ^{-1}\left( 1-%
\frac{\varphi }{2N}\right) \right] ^{3}T^{-\frac{{\large 1-\alpha }_{{\large %
1}}}{{\large 2}}}\right\} $

\noindent ${\small =2}\left[ 1-\left( 1-\frac{\varphi }{2N}\right) \right]
\left\{ 1+A\left[ 1+\Phi ^{-1}\left( 1-\frac{\varphi }{2N}\right) \right]
^{3}T^{-\frac{{\large 1-\alpha }_{{\large 1}}}{{\large 2}}}\right\} $

\noindent ${\small =}\frac{\varphi }{N}\left\{ 1+A\left[ 1+\Phi ^{-1}\left(
1-\frac{\varphi }{2N}\right) \right] ^{3}T^{-\frac{{\large 1-\alpha }_{%
{\large 1}}}{{\large 2}}}\right\} ,$

\noindent for $\ell \in \left\{ 1,...,d\right\} $, for $i\in H=\left\{ k\in
\left\{ 1,....,N\right\} :\gamma _{k}=0\right\} $, and for all $%
N_{1},N_{2},T $ sufficiently large. Next, note that:

\noindent ${\small P}\left( \max_{i\in H}\dsum\limits_{\ell =1}^{d}\varpi
_{\ell }\left\vert S_{i,\ell ,T}\right\vert \geq \Phi ^{-1}\left( 1-\frac{%
\varphi }{2N}\right) \right) $

\noindent ${\small \leq P}\left( \dbigcup\limits_{i\in
H}\dbigcup\limits_{1\leq \ell \leq d}\left\{ \left\vert S_{i,\ell
,T}\right\vert \geq \Phi ^{-1}\left( 1-\frac{\varphi }{2N}\right) \right\}
\right) \text{ }\left( \text{since }0\leq \varpi _{\ell }\leq 1\text{ and }%
\dsum\limits_{\ell =1}^{d}\varpi _{\ell }=1\right) $

\noindent ${\small \leq }\dsum\limits_{i\in H}\dsum\limits_{\ell =1}^{d}%
{\small P}\left( \left\vert S_{i,\ell ,T}\right\vert \geq \Phi ^{-1}\left( 1-%
\frac{\varphi }{2N}\right) \right) \text{ }\left( \text{by union bound}%
\right) $

\noindent ${\small \leq }\dsum\limits_{i\in H}\dsum\limits_{\ell =1}^{d}%
\frac{\varphi }{N}\left\{ 1+A\left[ 1+\Phi ^{-1}\left( 1-\frac{\varphi }{2N}%
\right) \right] ^{3}T^{-\left( 1-\alpha _{{\large 1}}\right) \frac{{\large 1}%
}{{\large 2}}}\right\} $

\noindent ${\small =d}\frac{N_{2}\varphi }{N}\left\{ 1+A\left[ 1+\Phi
^{-1}\left( 1-\frac{\varphi }{2N}\right) \right] ^{3}T^{-\left( 1-\alpha _{%
{\large 1}}\right) \frac{{\large 1}}{{\large 2}}}\right\} $

\noindent Using the inequality $\Phi ^{-1}\left( 1-\frac{\varphi }{2N}%
\right) \leq \sqrt{2\left( 1+a\right) }\sqrt{\ln N}$ discussed above, we
further obtain, for all $N_{1},N_{2},T$ sufficiently large:

\noindent ${\small P}\left( \max_{i\in H}\dsum\limits_{\ell =1}^{d}\varpi
_{\ell }\left\vert S_{i,\ell ,T}\right\vert \geq \Phi ^{-1}\left( 1-\frac{%
\varphi }{2N}\right) \right) {\small \leq }\frac{dN_{2}\varphi }{N}\left\{ 1+%
\frac{A}{T^{\left( {\large 1-\alpha }_{{\large 1}}\right) {\large /2}}}\left[
1+\Phi ^{-1}\left( 1-\frac{\varphi }{2N}\right) \right] ^{3}\right\} $

\noindent ${\small \leq }\frac{dN_{2}\varphi }{N}\left\{ 1+2^{2}AT^{-\frac{%
\left( {\large 1-\alpha }_{{\large 1}}\right) }{{\large 2}}}+2^{2}A\left[
\Phi ^{-1}\left( 1-\frac{\varphi }{2N}\right) \right] ^{3}T^{-\frac{\left( 
{\large 1-\alpha }_{{\large 1}}\right) }{{\large 2}}}\right\} $

\noindent $\left( \text{by the inequality }\left\vert
\dsum\limits_{i=1}^{m}a_{i}\right\vert ^{r}\leq
c_{r}\dsum\limits_{i=1}^{m}\left\vert a_{i}\right\vert ^{r}\text{ where }%
c_{r}=m^{r-1}\text{ for }r\geq 1\right) $

\noindent ${\small \leq }\frac{dN_{2}\varphi }{N}\left\{ 1+4AT^{-\frac{%
\left( {\large 1-\alpha }_{{\large 1}}\right) }{{\large 2}}}+4A\left[ \sqrt{%
2\left( 1+a\right) }\sqrt{\ln N}\right] ^{3}T^{-\frac{\left( {\large %
1-\alpha }_{{\large 1}}\right) }{{\large 2}}}\right\} $

\noindent ${\small =}\frac{dN_{2}\varphi }{N}\left\{ 1+4AT^{-\frac{\left( 
{\large 1-\alpha }_{{\large 1}}\right) }{{\large 2}}}+2^{\frac{{\large 7}}{%
{\large 2}}}A\left( 1+a\right) ^{\frac{{\large 3}}{{\large 2}}}\frac{\left(
\ln N\right) ^{\frac{{\Large 3}}{{\large 2}}}}{T^{\frac{{\large 1-\alpha }_{%
{\large 1}}}{{\large 2}}}}\right\} .$

\noindent Finally, note that the rate condition given in part (a) of
Assumption 2-9

\noindent (i.e., $\sqrt{\ln N}/\min \left\{ T^{\left( 1-{\large \alpha }%
_{1}\right) /6},T^{\alpha _{2}/2}\right\} \rightarrow 0$ as $%
N_{1},N_{2},T\rightarrow \infty $) implies that

\noindent $\left( \ln N\right) ^{\frac{{\Large 3}}{{\large 2}}}/T^{\frac{%
{\large 1-\alpha }_{{\large 1}}}{{\large 2}}}\rightarrow 0$ as $%
N_{1},N_{2},T\rightarrow \infty $, from which it follows that:

\noindent ${\small P}\left( \max_{i\in H}\dsum\limits_{\ell =1}^{d}\varpi
_{\ell }\left\vert S_{i,\ell ,T}\right\vert \geq \Phi ^{-1}\left( 1-\frac{%
\varphi }{2N}\right) \right) $

\noindent ${\small \leq }\frac{dN_{2}\varphi }{N}\left\{ 1+4AT^{-\frac{%
\left( {\large 1-\alpha }_{{\large 1}}\right) }{{\large 2}}}+2^{\frac{%
{\large 7}}{{\large 2}}}A\left( 1+a\right) ^{\frac{{\large 3}}{{\large 2}}}%
\frac{\left( \ln N\right) ^{\frac{{\Large 3}}{{\large 2}}}}{T^{\frac{{\large %
1-\alpha }_{{\large 1}}}{{\large 2}}}}\right\} =\frac{dN_{2}\varphi }{N}%
\left[ 1+o\left( 1\right) \right] =O\left( \frac{N_{2}\varphi }{N}\right)
=o\left( 1\right) $.

Next, to show part (b), note that, by a similar argument as that given for
part (a) above, we have:

\noindent $P\left( \max_{{\large i\in }H}\max_{1\leq \ell \leq d}\left\vert
S_{i,\ell ,T}\right\vert \geq \Phi ^{-1}\left( 1-\frac{\varphi }{2N}\right)
\right) $

\noindent $=P\left( \dbigcup\limits_{i\in H}\dbigcup\limits_{1\leq \ell \leq
d}\left\{ \left\vert S_{i,\ell ,T}\right\vert \geq \Phi ^{-1}\left( 1-\frac{%
\varphi }{2N}\right) \right\} \right) $

\noindent $\leq \frac{dN_{2}\varphi }{N}\left\{ 1+\frac{4A}{T^{\left( 
{\large 1-\alpha }_{{\large 1}}\right) /{\large 2}}}+\frac{2^{\frac{{\large 7%
}}{{\large 2}}}A\left( 1+a\right) ^{\frac{{\large 3}}{{\large 2}}}\left( \ln
N\right) ^{\frac{{\Large 3}}{{\large 2}}}}{T^{\left( {\large 1-\alpha }_{%
{\large 1}}\right) /{\large 2}}}\right\} =\frac{dN_{2}\varphi }{N}\left[
1+o\left( 1\right) \right] =O\left( \frac{N_{2}\varphi }{N}\right) =o\left(
1\right) ${\small . }$\square $

\noindent \textbf{Proof of Theorem 2: }To show part (a), let $\overline{S}%
_{i,\ell ,T}$ and $\overline{V}_{i,\ell ,T}$ be as defined in expression (%
\ref{num and denom max stat}), and note that:

\noindent ${\small P}\left( \min_{{\large i\in }H^{{\large c}%
}}\dsum\limits_{\ell =1}^{d}\varpi _{\ell }\left\vert S_{i,\ell
,T}\right\vert \geq \Phi ^{-1}\left( 1-\frac{\varphi }{2N}\right) \right) $

\noindent ${\small =P}\left( \min_{{\large i\in }H^{{\large c}%
}}\dsum\limits_{\ell =1}^{d}\varpi _{\ell }\left\vert \frac{\overline{S}%
_{i,,\ell ,T}-\mu _{i,\ell ,T}}{\sqrt{\overline{V}_{i,\ell ,T}}}+\frac{\mu
_{i,,\ell ,T}}{\sqrt{\overline{V}_{i,\ell ,T}}}\right\vert \geq \Phi
^{-1}\left( 1-\frac{\varphi }{2N}\right) \right) $

\noindent ${\small \geq P}\left( \min_{{\large i\in }H^{{\large c}%
}}\dsum\limits_{\ell =1}^{d}\varpi _{\ell }\left\{ \left\vert \frac{\mu
_{i,\ell ,T}}{\sqrt{\overline{V}_{i,\ell ,T}}}\right\vert -\left\vert \frac{%
\overline{S}_{i,\ell ,T}-\mu _{i,\ell ,T}}{\sqrt{\overline{V}_{i,\ell ,T}}}%
\right\vert \right\} \geq \Phi ^{-1}\left( 1-\frac{\varphi }{2N}\right)
\right) $

\noindent ${\small =P}\left( \min_{{\large i\in }H^{{\large c}%
}}\dsum\limits_{\ell =1}^{d}\varpi _{\ell }\left\{ \left\vert \frac{\mu
_{i,\ell ,T}}{\sqrt{\overline{V}_{i,\ell ,T}}}\right\vert \left[
1-\left\vert \frac{\sqrt{\overline{V}_{i,\ell ,T}}}{\mu _{i,\ell ,T}}%
\right\vert \left\vert \frac{\overline{S}_{i,\ell ,T}-\mu _{i,\ell ,T}}{%
\sqrt{\overline{V}_{i,\ell ,T}}}\right\vert \right] \right\} \geq \Phi
^{-1}\left( 1-\frac{\varphi }{2N}\right) \right) $

\noindent ${\small =P}\left( \min_{{\large i\in }H^{{\large c}%
}}\dsum\limits_{\ell =1}^{d}\varpi _{\ell }\left\{ \left\vert \frac{\mu
_{i,\ell ,T}}{\sqrt{\overline{V}_{i,\ell ,T}}}\right\vert \left[
1-\left\vert \frac{\overline{S}_{i,\ell ,T}-\mu _{i,\ell ,T}}{\mu _{i,\ell
,T}}\right\vert \right] \right\} \geq \Phi ^{-1}\left( 1-\frac{\varphi }{2N}%
\right) \right) {\small ,}$

\noindent where ${\small \mu }_{i,\ell ,T}{\small =}\dsum\nolimits_{r=1}^{q}%
\dsum\nolimits_{t=b_{1}\left( r\right) }^{b_{2}\left( r\right) }{\small %
\gamma }_{i}^{\prime }\left\{ E\left[ \underline{F}_{t}\right] \mu _{Y,\ell
}+E\left[ \underline{F}_{t}\underline{Y}_{t}^{\prime }\right] \alpha
_{YY,\ell }+E\left[ \underline{F}_{t}\underline{F}_{t}^{\prime }\right]
\alpha _{YF,\ell }\right\} ,$for

\noindent $b_{1}\left( r\right) =\left( r-1\right) \tau +p$ and $b_{2}\left(
r\right) =b_{1}\left( r\right) +\tau _{1}-1${\small . }Next, let

\noindent ${\small \pi }_{i,\ell ,T}{\small =}\dsum\nolimits_{r=1}^{q}\left(
\dsum\nolimits_{t=b_{1}\left( r\right) }^{b_{2}\left( r\right) }\left\{
\gamma _{i}^{\prime }E\left[ \underline{F}_{t}\right] \mu _{Y,\ell }+\gamma
_{i}^{\prime }E\left[ \underline{F}_{t}\underline{Y}_{t}^{\prime }\right]
\alpha _{YY,\ell }+\gamma _{i}^{\prime }E\left[ \underline{F}_{t}\underline{F%
}_{t}^{\prime }\right] \alpha _{YF,\ell }\right\} \right) ^{2}$, and we see
that, under Assumption 2-8, there exists a positive constant $\underline{c}$
such that for every $\ell \in \left\{ 1,...,d\right\} $ and for all $%
N_{1},N_{2},$ and $T$ sufficiently large:

$\min_{{\large i\in }H^{{\large c}}}\left\{ \pi _{i,\ell ,T}/\left( q\tau
_{1}^{2}\right) \right\} $

\noindent ${\small =}\min_{{\large i\in }H^{{\large c}}}\frac{1}{q}%
\dsum\limits_{r=1}^{q}\left( \frac{1}{\tau _{1}}\dsum\limits_{t=b_{1}\left(
r\right) }^{b_{2}\left( r\right) }\left\{ \gamma _{i}^{\prime }E\left[ 
\underline{F}_{t}\right] \mu _{Y,\ell }+\gamma _{i}^{\prime }E\left[ 
\underline{F}_{t}\underline{Y}_{t}^{\prime }\right] \alpha _{YY,\ell
}+\gamma _{i}^{\prime }E\left[ \underline{F}_{t}\underline{F}_{t}^{\prime }%
\right] \alpha _{YF,\ell }\right\} \right) ^{2}$

\noindent ${\small =}\min_{{\large i\in }H^{{\large c}}}\frac{1}{q}%
\dsum\limits_{r=1}^{q}\left( \frac{1}{\tau _{1}}\dsum\limits_{t=b_{1}\left(
r\right) }^{b_{2}\left( r\right) }E\left[ \gamma _{i}^{\prime }\underline{F}%
_{t}y_{\ell ,t{\LARGE +}1}\right] \right) ^{2}$

\noindent $\geq \min_{{\large i\in }H^{{\large c}}}\left( \frac{1}{q}%
\dsum\limits_{r=1}^{q}\frac{1}{\tau _{1}}\dsum\limits_{t=b_{1}\left(
r\right) }^{b_{2}\left( r\right) }E\left[ \gamma _{i}^{\prime }\underline{F}%
_{t}y_{\ell ,t{\LARGE +}1}\right] \right) ^{2}${\small \ \ }$\left( \text{by
Jensen's inequality}\right) $

\noindent ${\small =}\min_{{\large i\in }H^{{\large c}}}\left\vert \frac{1}{q%
}\dsum\limits_{r=1}^{q}\frac{1}{\tau _{1}}\dsum\limits_{t=b_{1}\left(
r\right) }^{b_{2}\left( r\right) }\left\{ \gamma _{i}^{\prime }E\left[ 
\underline{F}_{t}\right] \mu _{Y,\ell }+\gamma _{i}^{\prime }E\left[ 
\underline{F}_{t}\underline{Y}_{t}^{\prime }\right] \alpha _{YY,\ell
}+\gamma _{i}^{\prime }E\left[ \underline{F}_{t}\underline{F}_{t}^{\prime }%
\right] \alpha _{YF,\ell }\right\} \right\vert ^{2}$

\noindent ${\small \geq }$ $\underline{c}^{2}{\small >0}$ \ $\left( \text{in
light of Assumption 2-8}\right) .$

\noindent It follows that for all $N_{1},N_{2},$ and $T$ sufficiently large:

\noindent ${\small P}\left( \min_{{\large i\in }H^{{\large c}%
}}\dsum\limits_{\ell =1}^{d}\varpi _{\ell }\left\vert S_{i,\ell
,T}\right\vert \geq \Phi ^{-1}\left( 1-\frac{\varphi }{2N}\right) \right) $

\noindent ${\small \geq P}\left( \min_{{\large i\in }H^{{\large c}%
}}\dsum\limits_{\ell =1}^{d}\varpi _{\ell }\left\{ \left\vert \frac{\mu
_{i,\ell ,T}}{\sqrt{\overline{V}_{i,\ell ,T}}}\right\vert \left[
1-\left\vert \frac{\overline{S}_{i,\ell ,T}-\mu _{i,\ell ,T}}{\mu _{i,\ell
,T}}\right\vert \right] \right\} \geq \Phi ^{-1}\left( 1-\frac{\varphi }{2N}%
\right) \right) $

\noindent ${\small =P}\left( \min_{{\large i\in }H^{{\large c}%
}}\dsum\limits_{\ell =1}^{d}\varpi _{\ell }\left\{ \left\vert \frac{\sqrt{q}%
\left[ \mu _{i,\ell ,T}/\left( q\tau _{1}\right) \right] }{\sqrt{\pi
_{i,\ell ,T}/\left( q\tau _{1}^{2}\right) }}\right\vert \left\vert \frac{%
\sqrt{\pi _{i,\ell ,T}/\left( q\tau _{1}^{2}\right) }}{\sqrt{\pi _{i,\ell
,T}/\left( q\tau _{1}^{2}\right) }+\sqrt{\overline{V}_{i,\ell ,T}/\left(
q\tau _{1}^{2}\right) }-\sqrt{\pi _{i,\ell ,T}/\left( q\tau _{1}^{2}\right) }%
}\right\vert \right. \right. $

\bigskip\ {\small \ \ \ \ \ \ \ \ \ \ \ \ \ \ \ \ \ \ \ \ \ \ \ \ \ \ \ \ \
\ }$\left. \left. \times \left[ 1-\left\vert \frac{\overline{S}_{i,\ell
,T}-\mu _{i,\ell ,T}}{\mu _{i,\ell ,T}}\right\vert \right] \right\} \geq
\Phi ^{-1}\left( 1-\frac{\varphi }{2N}\right) \right) $

\noindent ${\small =P}\left( \min_{{\large i\in }H^{{\large c}%
}}\dsum\limits_{\ell =1}^{d}\varpi _{\ell }\left\{ \left\vert \frac{\sqrt{q}%
\left[ \mu _{i,\ell ,T}/\left( q\tau _{1}\right) \right] }{\sqrt{\pi
_{i,\ell ,T}/\left( q\tau _{1}^{2}\right) }}\right\vert \left\vert \frac{1}{%
1+\left( \sqrt{\overline{V}_{i,\ell ,T}}-\sqrt{\pi _{i,\ell ,T}}\right) /%
\sqrt{\pi _{i,\ell ,T}}}\right\vert \right. \right. $

{\small \ \ \ \ \ \ \ \ \ \ \ \ \ \ \ \ \ \ \ \ \ \ \ \ \ \ \ \ \ \ \ }$%
\left. \left. \times \left[ 1-\left\vert \frac{\overline{S}_{i,\ell ,T}-\mu
_{i,\ell ,T}}{\mu _{i,\ell ,T}}\right\vert \right] \right\} \geq \Phi
^{-1}\left( 1-\frac{\varphi }{2N}\right) \right) $

\noindent ${\small \geq P}\left( \min_{{\large i\in }H^{{\large c}%
}}\dsum\limits_{\ell =1}^{d}\varpi _{\ell }\left\{ \left\vert \frac{\sqrt{q}%
\left( \mu _{i,\ell ,T}/\left( q\tau _{1}\right) \right) }{\sqrt{\pi
_{i,\ell ,T}/\left( q\tau _{1}^{2}\right) }}\right\vert \frac{1}{1+\max_{k%
{\large \in }H^{{\large c}}}\left\vert \sqrt{\overline{V}_{k,\ell ,T}}-\sqrt{%
\pi _{k,\ell ,T}}\right\vert /\sqrt{\pi _{k,\ell ,T}}}\right. \right. $

$\ \ \ \ \ \ \ \ \ \ \ \ \ \ \ \ \ \ \ \ \ \ \left. \left. \times \left[
1-\max_{k{\large \in }H^{{\large c}}}\left\vert \frac{\overline{S}_{k,\ell
,T}-\mu _{k,\ell ,T}}{\mu _{k,\ell ,T}}\right\vert \right] \right\} \geq
\Phi ^{-1}\left( 1-\frac{\varphi }{2N}\right) \right) $

\noindent ${\small \geq P}\left( \min_{{\large i\in }H^{{\large c}%
}}\dsum\limits_{\ell =1}^{d}\varpi _{\ell }\left\{ \left\vert \frac{\sqrt{q}%
\left[ \mu _{i,\ell ,T}/\left( q\tau _{1}\right) \right] }{\sqrt{\pi
_{i,\ell ,T}/\left( q\tau _{1}^{2}\right) }}\right\vert \frac{1}{1+\max_{k%
{\large \in }H^{{\large c}}}\sqrt{\left\vert \overline{V}_{k,\ell ,T}-\pi
_{k,\ell ,T}\right\vert /\pi _{k,\ell ,T}}}\right. \right. $

\ \ \ \ $\ \ \ \ \ \ \ \ \ \ \ \ \ \ \ \ \ \ \ \left. \left. \times \left[
1-\max_{k{\large \in }H^{{\large c}}}\left\vert \frac{\overline{S}_{k,\ell
,T}-\mu _{k,\ell ,T}}{\mu _{k,\ell ,T}}\right\vert \right] \right\} \geq
\Phi ^{-1}\left( 1-\frac{\varphi }{2N}\right) \right) $

$\left( \text{making use of the inequality }\left\vert \sqrt{x}-\sqrt{y}%
\right\vert \leq \sqrt{\left\vert x-y\right\vert }\text{ for }x\geq 0\text{
and }y\geq 0\text{ }\right) $

\noindent ${\small =P}\left( \min_{{\large i\in }H^{{\large c}%
}}\dsum\limits_{\ell =1}^{d}\varpi _{\ell }\left\{ \left\vert \frac{\sqrt{q}%
\left[ \mu _{i,\ell ,T}/\left( q\tau _{1}\right) \right] }{\sqrt{\pi
_{i,\ell ,T}/\left( q\tau _{1}^{2}\right) }}\right\vert \frac{1-\max_{k%
{\large \in }H^{{\large c}}}\left\vert \mathcal{E}_{k,\ell ,T}\right\vert }{%
1+\max_{k{\large \in }H^{{\large c}}}\sqrt{\left\vert \mathcal{V}_{k,\ell
,T}\right\vert }}\right\} {\small \geq \Phi }^{-1}\left( 1-\frac{\varphi }{2N%
}\right) \right) $,

\noindent where $\mathcal{E}_{k,\ell ,T}=\left( \overline{S}_{k,\ell ,T}-\mu
_{k,\ell ,T}\right) /\mu _{k,\ell ,T}$ and $\mathcal{V}_{k,\ell ,T}=\left( 
\overline{V}_{k,\ell ,T}-\pi _{k,\ell ,T}\right) /\pi _{k,\ell ,T}$. By part
(a) of Lemma QA-16 (given in the Online Appendix, Chao, Liu, and Swanson,
2023), there exists a sequence of positive numbers $\left\{ \epsilon
_{T}\right\} $ such that, as $T\rightarrow \infty $, $\epsilon
_{T}\rightarrow 0$ and $P\left( \max_{1\leq \ell \leq d}\max_{k{\large \in }%
H^{{\large c}}}\left\vert \mathcal{E}_{k,\ell ,T}\right\vert \geq \epsilon
_{T}\right) \rightarrow 0$. In addition, by the result of part (b) of Lemma
QA-16, there exists a sequence of positive numbers $\left\{ \epsilon
_{T}^{\ast }\right\} $ such that, as $T\rightarrow \infty $, $\epsilon
_{T}^{\ast }\rightarrow 0$ and $P\left( \max_{1\leq \ell \leq d}\max_{k%
{\large \in }H^{{\large c}}}\left\vert \mathcal{V}_{k,\ell ,T}\right\vert
\geq \epsilon _{T}^{\ast }\right) \rightarrow 0$. Further define $\overline{%
\mathbb{E}}_{T}=\max_{1\leq \ell \leq d}\max_{k{\large \in }H^{{\large c}%
}}\left\vert \mathcal{E}_{k,\ell ,T}\right\vert $ and $\overline{\mathbb{V}}%
_{T}=\max_{1\leq \ell \leq d}\max_{k{\large \in }H^{{\large c}}}\left\vert 
\mathcal{V}_{k,\ell ,T}\right\vert $; and note that, for all $N_{1},N_{2}$,
and $T$ sufficiently large,

\noindent ${\small P}\left( \min_{{\large i\in }H^{{\large c}%
}}\dsum\limits_{\ell =1}^{d}\varpi _{\ell }\left\vert S_{i,\ell
,T}\right\vert \geq \Phi ^{-1}\left( 1-\frac{\varphi }{2N}\right) \right) $

\noindent ${\small \geq P}\left( \min_{{\large i\in }H^{{\large c}%
}}\dsum\limits_{\ell =1}^{d}\varpi _{\ell }\left\{ \left\vert \frac{\sqrt{q}%
\left[ \mu _{i,\ell ,T}/\left( q\tau _{1}\right) \right] }{\sqrt{\pi
_{i,\ell ,T}/\left( q\tau _{1}^{2}\right) }}\right\vert \frac{1-\max_{k%
{\large \in }H^{{\large c}}}\left\vert \mathcal{E}_{k,\ell ,T}\right\vert }{%
1+\max_{k{\large \in }H^{{\large c}}}\sqrt{\left\vert \mathcal{V}_{k,\ell
,T}\right\vert }}\right\} {\small \geq \Phi }^{-1}\left( 1-\frac{\varphi }{2N%
}\right) \right) $

\noindent ${\small \geq P}\left( \frac{1-\max_{1\leq \ell \leq d}\max_{k%
{\large \in }H^{{\large c}}}\left\vert \mathcal{E}_{k,\ell ,T}\right\vert }{%
1+\max_{1\leq \ell \leq d}\max_{k{\large \in }H^{{\large c}}}\sqrt{%
\left\vert \mathcal{V}_{k,\ell ,T}\right\vert }}\min_{{\large i\in }H^{%
{\large c}}}\dsum\limits_{\ell =1}^{d}\varpi _{\ell }\left\vert \frac{\sqrt{q%
}\left[ \mu _{i,\ell ,T}/\left( q\tau _{1}\right) \right] }{\sqrt{\pi
_{i,\ell ,T}/\left( q\tau _{1}^{2}\right) }}\right\vert \geq {\small \Phi }%
^{-1}\left( 1-\frac{\varphi }{2N}\right) \right) $

\noindent ${\small =P}\left( \frac{1-\overline{\mathbb{E}}_{T}}{1+\sqrt{%
\overline{\mathbb{V}}_{T}}}\min_{{\large i\in }H^{{\large c}%
}}\dsum\limits_{\ell =1}^{d}\varpi _{\ell }\left\vert \frac{\mu _{i,\ell ,T}%
}{\sqrt{\pi _{i,\ell ,T}}}\right\vert \geq {\small \Phi }^{-1}\left( 1-\frac{%
\varphi }{2N}\right) \right) $

\noindent ${\small \geq P}\left( \left\{ \left\vert \frac{1-\epsilon _{T}}{1+%
\sqrt{\epsilon _{T}^{\ast }}}\right\vert \min_{{\large i\in }H^{{\large c}%
}}\dsum\limits_{\ell =1}^{d}\varpi _{\ell }\left\vert \frac{\mu _{i,\ell ,T}%
}{\sqrt{\pi _{i,\ell ,T}}}\right\vert \geq {\small \Phi }^{-1}\left( 1-\frac{%
\varphi }{2N}\right) \right\} \cap \left\{ \overline{\mathbb{E}}%
_{T}<\epsilon _{T}\right\} \cap \left\{ \overline{\mathbb{V}}_{T}<\epsilon
_{T}^{\ast }\right\} \right) $

\noindent ${\small +P}\left( \left\{ \frac{1-\overline{\mathbb{E}}_{T}}{1+%
\sqrt{\overline{\mathbb{V}}_{T}}}\min_{{\large i\in }H}\dsum\limits_{\ell
=1}^{d}\varpi _{\ell }\left\vert \frac{\mu _{i,\ell ,T}}{\sqrt{\pi _{i,\ell
,T}}}\right\vert \geq {\small \Phi }^{-1}\left( 1-\frac{\varphi }{2N}\right)
\right\} \cap \left\{ \overline{\mathbb{E}}_{T}\geq \epsilon _{T}\cup 
\overline{\mathbb{V}}_{T}\geq \epsilon _{T}^{\ast }\right\} \right) $

\noindent ${\small \geq P}\left( \left\{ \left\vert \frac{1-\epsilon _{T}}{1+%
\sqrt{\epsilon _{T}^{\ast }}}\right\vert \min_{{\large i\in }H^{{\large c}%
}}\dsum\limits_{\ell =1}^{d}\varpi _{\ell }\left\vert \frac{\mu _{i,\ell ,T}%
}{\sqrt{\pi _{i,\ell ,T}}}\right\vert \geq {\small \Phi }^{-1}\left( 1-\frac{%
\varphi }{2N}\right) \right\} \cap \left\{ \overline{\mathbb{E}}%
_{T}<\epsilon _{T}\right\} \cap \left\{ \overline{\mathbb{V}}_{T}<\epsilon
_{T}^{\ast }\right\} \right) $

\noindent ${\small +P}\left( \left\{ \frac{1-\overline{\mathbb{E}}_{T}}{1+%
\sqrt{\overline{\mathbb{V}}_{T}}}\min_{{\large i\in }H}\dsum\limits_{\ell
=1}^{d}\varpi _{\ell }\left\vert \frac{\mu _{i,\ell ,T}}{\sqrt{\pi _{i,\ell
,T}}}\right\vert \geq {\small \Phi }^{-1}\left( 1-\frac{\varphi }{2N}\right)
\right\} \cap \left\{ \overline{\mathbb{E}}_{T}\geq \epsilon _{T}\right\}
\right) $

\noindent ${\small =P}\left( \left\{ \left\vert \frac{1-\epsilon _{T}}{1+%
\sqrt{\epsilon _{T}^{\ast }}}\right\vert \min_{{\large i\in }H^{{\large c}%
}}\dsum\limits_{\ell =1}^{d}\varpi _{\ell }\left\vert \frac{\mu _{i,\ell ,T}%
}{\sqrt{\pi _{i,\ell ,T}}}\right\vert \geq {\small \Phi }^{-1}\left( 1-\frac{%
\varphi }{2N}\right) \right\} \cap \left\{ \overline{\mathbb{E}}%
_{T}<\epsilon _{T}\right\} \cap \left\{ \overline{\mathbb{V}}_{T}<\epsilon
_{T}^{\ast }\right\} \right) $

${\small +o}\left( 1\right) {\small .}$

\noindent where the last equality above follows from the fact that

\noindent $P\left( \left\{ \frac{1-\overline{\mathbb{E}}_{T}}{1+\sqrt{%
\overline{\mathbb{V}}_{T}}}\min_{{\large i\in }H}\dsum\limits_{\ell
=1}^{d}\varpi _{\ell }\left\vert \frac{\mu _{i,\ell ,T}}{\sqrt{\pi _{i,\ell
,T}}}\right\vert \geq \Phi ^{-1}\left( 1-\frac{\varphi }{2N}\right) \right\}
\cap \left\{ \overline{\mathbb{E}}_{T}\geq \epsilon _{T}\right\} \right) $

\noindent $\leq P\left( \overline{\mathbb{E}}_{T}\geq \epsilon _{T}\right)
=o\left( 1\right) $

\noindent Moreover, making use of Assumption 2-8, the result given in Lemma
A1, and the fact that $q=\left\lfloor T_{0}/\tau \right\rfloor \sim
T^{1-\alpha _{{\large 1}}}$, we see that, there exists positive constants $%
\underline{c}$ and $\overline{C}$ such that:

\noindent $\min_{{\large i\in }H^{{\large c}}}\dsum\limits_{\ell =1}^{d}%
{\small \varpi }_{\ell }\left\vert \frac{\mu _{i,\ell ,T}}{\sqrt{\pi
_{i,\ell ,T}}}\right\vert {\small =}\min_{{\large i\in }H^{{\large c}%
}}\dsum\limits_{\ell =1}^{d}{\small \varpi }_{\ell }\frac{\sqrt{q}\left\vert
\mu _{i,\ell ,T}/\left( q\tau _{1}\right) \right\vert }{\sqrt{\pi _{i,\ell
,T}/\left( q\tau _{1}^{2}\right) }}$

\noindent ${\small \geq }\sqrt{q}\dsum\limits_{\ell =1}^{d}{\small \varpi }%
_{\ell }\frac{\min_{{\large i\in }H^{{\large c}}}\left\vert \mu _{i,\ell
,T}/\left( q\tau _{1}\right) \right\vert }{\sqrt[\backslash ]{\max_{{\large %
i\in }H^{{\large c}}}\pi _{i,\ell ,T}/\left( q\tau _{1}^{2}\right) }}{\small %
\geq }\sqrt{q}\dsum\limits_{\ell =1}^{d}{\small \varpi }_{\ell }\frac{%
\underline{c}}{\sqrt{C}}{\small =}\sqrt{q}\frac{\underline{c}}{\sqrt{C}}%
{\small \sim }\sqrt{q}{\small \sim }\sqrt{\frac{T_{0}}{\tau }}{\small \sim T}%
^{\left( 1-\alpha _{{\large 1}}\right) /2}$ .

\noindent On the other hand, applying the inequality

\noindent $\Phi ^{-1}\left( 1-\frac{\varphi }{2N}\right) \leq \sqrt{2\left(
1+a\right) }\sqrt{\ln N}\sim \sqrt{\ln N}$,\footnote{%
As noted previously, an explicit proof of this result is given in Chao, Liu,
and Swanson (2023). In particular, this inequality is shown in part (b) of
Lemma QA-15 in Chao, Liu, and Swanson (2023).} we further deduce that,

\noindent as $N_{1},N_{2},T\rightarrow \infty $,%
\begin{equation*}
\frac{1}{\Phi ^{-1}\left( 1-\frac{\varphi }{2N}\right) }\min_{{\large i\in }%
H^{{\large c}}}\dsum\limits_{\ell =1}^{d}\varpi _{\ell }\left\vert \frac{\mu
_{i,\ell ,T}}{\sqrt{\pi _{i,\ell ,T}}}\right\vert \geq \frac{\underline{c}}{%
\sqrt{C}}\sqrt{\frac{q}{2\left( 1+a\right) \ln N}}\sim \sqrt[\backslash ]{%
\frac{T^{\left( 1-\alpha _{{\large 1}}\right) }}{\ln N}}\rightarrow \infty 
\text{.}
\end{equation*}%
This is true because the condition $\sqrt{\ln N}/\min \left\{ T^{\left( 1-%
{\large \alpha }_{1}\right) /6},T^{\alpha _{2}/2}\right\} \rightarrow 0$ as $%
N_{1},N_{2},T\rightarrow \infty $ \ (as specified in Assumption 2-9 part
(a)) implies that $\ln N/T^{\left( 1-\alpha _{{\large 1}}\right)
}\rightarrow 0$ as $N_{1},N_{2},T\rightarrow \infty $. Hence, there exists a
natural number $M$ such that, for all $N_{1}\geq M,N_{2}\geq M$, and $T\geq
M $, we have $\left\vert \frac{1-\epsilon _{T}}{1+\sqrt{\epsilon _{T}^{\ast }%
}}\right\vert \min_{{\large i\in }H^{{\large c}}}\dsum\limits_{\ell =1}^{d}%
{\small \varpi }_{\ell }\left\vert \frac{\mu _{i,\ell ,T}}{\sqrt{\pi
_{i,\ell ,T}}}\right\vert {\small \geq \Phi }^{-1}\left( 1-\frac{\varphi }{2N%
}\right) $ so that:

\noindent ${\small P}\left( \min_{{\large i\in }H^{{\large c}%
}}\dsum\limits_{\ell =1}^{d}\varpi _{\ell }\left\vert S_{i,\ell
,T}\right\vert \geq \Phi ^{-1}\left( 1-\frac{\varphi }{2N}\right) \right) $

\noindent ${\small \geq P}\left( \left\{ \left\vert \frac{1-\epsilon _{T}}{1+%
\sqrt{\epsilon _{T}^{\ast }}}\right\vert \min_{{\large i\in }H^{{\large c}%
}}\dsum\limits_{\ell =1}^{d}\varpi _{\ell }\left\vert \frac{\mu _{i,\ell ,T}%
}{\sqrt{\pi _{i,\ell ,T}}}\right\vert \geq \Phi ^{-1}\left( 1-\frac{\varphi 
}{2N}\right) \right\} \cap \left\{ \overline{\mathbb{E}}_{T}<\epsilon
_{T}\right\} \cap \left\{ \overline{\mathbb{V}}_{T}<\epsilon _{T}^{\ast
}\right\} \right) $

$+o\left( 1\right) $

\noindent ${\small =P}\left( \left\{ \overline{\mathbb{E}}_{T}<\epsilon
_{T}\right\} \cap \left\{ \overline{\mathbb{V}}_{T}<\epsilon _{T}^{\ast
}\right\} \right) {\small +o}\left( 1\right) $

$\left( \text{for all }N_{1}\geq M,N_{2}\geq M,\text{and }T\geq M\right) $

\noindent ${\small \geq P}\left( \overline{\mathbb{E}}_{T}<\epsilon
_{T}\right) {\small +P}\left( \overline{\mathbb{V}}_{T}<\epsilon _{T}^{\ast
}\right) {\small -1+o}\left( 1\right) $ $\left( \text{using the inequality}%
\right. $

$\left. P\left\{ \dbigcap\limits_{i=1}^{m}A_{i}\right\} \geq
\dsum\limits_{i=1}^{m}P\left( A_{i}\right) -\left( m-1\right) \text{ in
Chao, Liu, and Swanson (2023) Lemma OA-14}\right) $

\noindent ${\small =1-P}\left( \overline{\mathbb{E}}_{T}\geq \epsilon
_{T}\right) {\small +1-P}\left( \overline{\mathbb{V}}_{T}\geq \epsilon
_{T}^{\ast }\right) {\small -1+o}\left( 1\right) $

\noindent ${\small =1-P}\left( \overline{\mathbb{E}}_{T}\geq \epsilon
_{T}\right) {\small -P}\left( \overline{\mathbb{V}}_{T}\geq \epsilon
_{T}^{\ast }\right) {\small +o}\left( 1\right) $

\noindent $=1+o\left( 1\right) ${\small .}

Next, to show part (b), note that, by applying the result in part (a), we
have that:

\noindent $P\left( \min_{{\large i\in }H^{{\large c}}}\max_{1\leq \ell \leq
d}\left\vert S_{i,\ell ,T}\right\vert \geq \Phi ^{-1}\left( 1-\frac{\varphi 
}{2N}\right) \right) $

\noindent $\geq P\left( \min_{{\large i\in }H^{{\large c}}}\dsum\limits_{%
\ell =1}^{d}\varpi _{\ell }\left\vert S_{i,\ell ,T}\right\vert \geq \Phi
^{-1}\left( 1-\frac{\varphi }{2N}\right) \right) =1+o\left( 1\right) $. $%
\square $

\bigskip

\noindent \textbf{Lemma A1: }Let $\underline{Y}_{t}=\left( 
\begin{array}{cccc}
Y_{t}^{\prime } & Y_{t-1}^{\prime } & \cdots & Y_{t-p{\LARGE +}1}^{\prime }%
\end{array}%
\right) ^{\prime }$ and $\underline{F}_{t}=\left( 
\begin{array}{cccc}
F_{t}^{\prime } & F_{t-1}^{\prime } & \cdots & F_{t-p{\LARGE +}1}^{\prime }%
\end{array}%
\right) ^{\prime }$, and define $b_{1}\left( r\right) =\left( r-1\right)
\tau +p$ and $b_{2}\left( r\right) =b_{1}\left( r\right) +\tau _{1}-1$.
Under Assumptions 2-1, 2-2, 2-5, 2-6, and 2-9(b); there exists a positive
constant $C$ such that:

\noindent $\max_{1\leq \ell \leq d,\text{ }i\in H^{{\large c}}}\left( \frac{%
\pi _{i,\ell ,T}}{q\tau _{1}^{2}}\right) $

\noindent $=\max_{1\leq \ell \leq d,\text{ }i\in H^{{\large c}}}\frac{1}{q}%
\dsum\limits_{r=1}^{q}\left( \frac{1}{\tau _{1}}\dsum%
\limits_{t=b_{1}(r)}^{b_{2}(r)}\gamma _{i}^{\prime }\left\{ E\left[ 
\underline{F}_{t}\right] \mu _{Y,\ell }+E\left[ \underline{F}_{t}\underline{Y%
}_{t}^{\prime }\right] \alpha _{YY,\ell }+E\left[ \underline{F}_{t}%
\underline{F}_{t}^{\prime }\right] \alpha _{YF,\ell }\right\} \right) ^{2}$

\noindent $\leq C<\infty ,$ for all $N_{1},N_{2},T$ sufficiently large.

\noindent \textbf{Proof of Lemma A1:} \ To proceed, let $\phi _{\max }=\max
\left\{ \left\vert \lambda _{\max }\left( A\right) \right\vert ,\left\vert
\lambda _{\min }\left( A\right) \right\vert \right\} $ and, for $\ell \in
\left\{ 1,...,d\right\} $, let $e_{\ell ,d}$ denote a $d\times 1$ elementary
vector whose $\ell ^{th}$ component is $1$ and all other components are $0$.
Now, note that:

\noindent $\max_{1\leq \ell \leq d,\text{ }i\in H^{{\large c}}}\left\{ \pi
_{i,\ell ,T}/\left( q\tau _{1}^{2}\right) \right\} $

\noindent $=\max_{1\leq \ell \leq d,\text{ }i\in H^{{\large c}}}\frac{1}{q}%
\dsum\limits_{r=1}^{q}\left( \frac{1}{\tau _{1}}\dsum\limits_{t=b_{1}\left(
r\right) }^{b_{2}\left( r\right) }\gamma _{i}^{\prime }\left\{ E\left[ 
\underline{F}_{t}\right] \mu _{Y,\ell }+E\left[ \underline{F}_{t}\underline{Y%
}_{t}^{\prime }\right] \alpha _{YY,\ell }+E\left[ \underline{F}_{t}%
\underline{F}_{t}^{\prime }\right] \alpha _{YF,\ell }\right\} \right) ^{2}$

\noindent $\leq \max_{1\leq \ell \leq d,\text{ }i\in H^{{\large c}}}\frac{1}{%
q}\dsum\limits_{r=1}^{q}\left( \frac{1}{\tau _{1}}\dsum\limits_{t=b_{1}%
\left( r\right) }^{b_{2}\left( r\right) }\left\{ E\left[ \left\vert \gamma
_{i}^{\prime }\underline{F}_{t}\right\vert \right] \left\vert \mu _{Y,\ell
}\right\vert +E\left[ \left\vert \gamma _{i}^{\prime }\underline{F}_{t}%
\underline{Y}_{t}^{\prime }A_{YY}^{\prime }e_{\ell ,d}\right\vert \right]
\right. \right. $

$\ \ \ \ \ \ \ \ \ \ \ \ \ \ \ \ \ \ \ \ \ \ \ \ \ \ \ \ \ \ \ \left. \left.
+E\left[ \left\vert \gamma _{i}^{\prime }\underline{F}_{t}\underline{F}%
_{t}^{\prime }A_{YF}^{\prime }e_{\ell ,d}\right\vert \right] \right\}
\right) ^{2}$ $\left( \text{by triangle and Jensen's inequalities}\right) $

\noindent $\leq \max_{i\in H^{{\large c}}}\frac{1}{q}\dsum\limits_{r=1}^{q}%
\left( \frac{1}{\tau _{1}}\dsum\limits_{t=b_{1}\left( r\right)
}^{b_{2}\left( r\right) }\left\{ \sqrt{\left\Vert \gamma _{i}\right\Vert
_{2}^{2}}\sqrt{E\left\Vert \underline{F}_{t}\right\Vert _{2}^{2}}\max_{1\leq
\ell \leq d}\left\vert \mu _{Y,\ell }\right\vert \right. \right. $

\noindent\ $\ \ \ \ \ \ \ \ \ \ \ \ \ \ \ \ \ \ \ \ \ \ \ \ \ \ \ +\sqrt{%
\gamma _{i}^{\prime }E\left[ \underline{F}_{t}\underline{F}_{t}^{\prime }%
\right] \gamma _{i}}\sqrt{\max_{1\leq \ell \leq d}e_{\ell ,d}^{\prime
}A_{YY}E\left[ \underline{Y}_{t}\underline{Y}_{t}^{\prime }\right]
A_{YY}^{\prime }e_{\ell ,d}}$

$\ \ \ \ \ \ \ \ \ \ \ \ \ \ \ \ \ \ \ \ $\ $\ \left. \left. +\sqrt{\gamma
_{i}^{\prime }E\left[ \underline{F}_{t}\underline{F}_{t}^{\prime }\right]
\gamma _{i}}\sqrt{\max_{1\leq \ell \leq d}e_{\ell ,d}^{\prime }A_{YF}E\left[ 
\underline{F}_{t}\underline{F}_{t}^{\prime }\right] A_{YF}^{\prime }e_{\ell
,d}}\right\} \right) ^{2}$

\noindent $\leq \left( \max_{i\in H^{{\large c}}}\left\Vert \gamma
_{i}\right\Vert _{2}^{2}\right) \frac{1}{q}\dsum\limits_{r=1}^{q}\left( 
\frac{1}{\tau _{1}}\dsum\limits_{t=b_{1}\left( r\right) }^{b_{2}\left(
r\right) }\left\{ \sqrt{E\left\Vert \underline{F}_{t}\right\Vert _{2}^{2}}%
\max_{1\leq \ell \leq d}\left\vert \mu _{Y,\ell }\right\vert \right. \right. 
$

\ \ \ \ \ \ \ \ \ \ \ \ \ \ \ \ \ \ \ \ \ \ \ \ \ \ \ \ \ \ $+\sqrt{%
E\left\Vert \underline{F}_{t}\right\Vert _{2}^{2}}\sqrt{E\left\Vert 
\underline{Y}_{t}\right\Vert _{2}^{2}}\sqrt{\max_{1\leq \ell \leq d}e_{\ell
,d}^{\prime }A_{YY}A_{YY}^{\prime }e_{\ell ,d}}$

$\ \ \ \ \ \ \ \ \ \ \ \ \ \ \ \ \ \ \ \ \ \ \ \ \ \ \ $\ $\ \left. \left.
+E\left\Vert \underline{F}_{t}\right\Vert _{2}^{2}\sqrt{\max_{1\leq \ell
\leq d}e_{\ell ,d}^{\prime }A_{YF}A_{YF}^{\prime }e_{\ell ,d}}\right\}
\right) ^{2}$

\noindent $\leq \left( \max_{i\in H^{{\large c}}}\left\Vert \gamma
_{i}\right\Vert _{2}^{2}\right) \frac{1}{q}\dsum\limits_{r=1}^{q}\left( 
\frac{1}{\tau _{1}}\dsum\limits_{t=b_{1}\left( r\right) }^{b_{2}\left(
r\right) }\left\{ \sqrt{E\left\Vert \underline{F}_{t}\right\Vert _{2}^{2}}%
\max_{1\leq \ell \leq d}\left\vert \mu _{Y,\ell }\right\vert \right. \right. 
$

$\ \ \ \ \ \ \ \ \ \ \ \ \ \ \ \ \ \ \ \ \left. \left. +\sqrt{E\left\Vert 
\underline{F}_{t}\right\Vert _{2}^{2}}\sqrt{E\left\Vert \underline{Y}%
_{t}\right\Vert _{2}^{2}}C^{\dagger }\phi _{\max }+E\left\Vert \underline{F}%
_{t}\right\Vert _{2}^{2}C^{\dagger }\phi _{\max }\right\} \right) ^{2},$

\noindent where the last inequality follows from the fact that, by making
use of Assumption 2-6, it is easy to show that there exists a constant $%
C^{\dagger }>0$ such that

\noindent $\sqrt{\max_{1\leq \ell \leq d}e_{\ell ,d}^{\prime
}A_{YY}A_{YY}^{\prime }e_{\ell ,d}}\leq \left\Vert A_{YY}\right\Vert _{2}%
\sqrt{\max_{1\leq \ell \leq d}e_{\ell ,d}^{\prime }e_{\ell ,d}}=\left\Vert
A_{YY}\right\Vert _{2}\leq C^{\dagger }\phi _{\max }$ and,

\noindent similarly, $\sqrt{\max_{1\leq \ell \leq d}e_{\ell ,d}^{\prime
}A_{YF}A_{YF}^{\prime }e_{\ell ,d}}\leq \left\Vert A_{YF}\right\Vert
_{2}\leq C^{\dagger }\phi _{\max }$.\footnote{%
Explicit proofs of these two inequalities are given in Chao, Liu, and
Swanson (2023). In particular, these inequalities are shown in parts (a) and
(b) of Lemma OA-7 in Chao, Liu, and Swanson (2023).} Hence,

\medskip

\noindent\ $\ \ \ \ \max_{1\leq \ell \leq d}\max_{k{\large \in }H^{{\large c}%
}}\left\{ \pi _{i,\ell ,T}/\left( q\tau _{1}^{2}\right) \right\} $

\noindent \noindent ${\small \leq }$ $\left( \max_{i\in H^{{\large c}%
}}\left\Vert \gamma _{i}\right\Vert _{2}^{2}\right) \frac{1}{q}%
\dsum\limits_{r=1}^{q}\left( \frac{1}{\tau _{1}}\dsum\limits_{t=b_{1}\left(
r\right) }^{b_{2}\left( r\right) }\left\{ \sqrt{E\left\Vert \underline{F}%
_{t}\right\Vert _{2}^{2}}\max_{1\leq \ell \leq d}\left\vert \mu _{Y,\ell
}\right\vert \right. \right. $

$\left. \left. +\sqrt{E\left\Vert \underline{F}_{t}\right\Vert _{2}^{2}}%
\sqrt{E\left\Vert \underline{Y}_{t}\right\Vert _{2}^{2}}C^{\dagger }\phi
_{\max }+E\left\Vert \underline{F}_{t}\right\Vert _{2}^{2}C^{\dagger }\phi
_{\max }\right\} \right) ^{2}$

\noindent ${\small \leq }\left( \max_{i\in H^{{\large c}}}\left\Vert \gamma
_{i}\right\Vert _{2}^{2}\right) \frac{1}{q}\dsum\limits_{r=1}^{q}\frac{1}{%
\tau _{1}}\dsum\limits_{t=b_{1}\left( r\right) }^{b_{2}\left( r\right) }%
{\small E}\left\Vert \underline{F}_{t}\right\Vert _{2}^{2}\left( \left\Vert
\mu _{Y}\right\Vert _{2}^{2}+\left[ \sqrt{E\left\Vert \underline{Y}%
_{t}\right\Vert _{2}^{2}}+\sqrt{E\left\Vert \underline{F}_{t}\right\Vert
_{2}^{2}}\right] C^{\dagger }\phi _{\max }\right) ^{2}$

\noindent \noindent $\leq C<\infty $,

\noindent for some positive constant $C$ such that

\noindent $C\geq \left( \max_{i\in H^{{\large c}}}\left\Vert \gamma
_{i}\right\Vert _{2}^{2}\right) E\left\Vert \underline{F}_{t}\right\Vert
_{2}^{2}\left( \left\Vert \mu _{Y}\right\Vert _{2}^{2}+\left[ \sqrt{%
E\left\Vert \underline{Y}_{t}\right\Vert _{2}^{2}}+\sqrt{E\left\Vert 
\underline{F}_{t}\right\Vert _{2}^{2}}\right] C^{\dagger }\phi _{\max
}\right) ^{2}$, where such a constant exists because $\max_{i\in H^{{\large c%
}}}\left\Vert \gamma _{i}\right\Vert _{2}^{2}$ and $\left\Vert \mu
_{Y}\right\Vert _{2}^{2}$ are both bounded given Assumption 2-5; because $%
0<\phi _{\max }<1$ given Assumption 2-1; and because, under Assumptions 2-1,
2-2(a)-(b), 2-5, and 2-6; one can easily show that there exists a constant $%
C^{\ast }>0$ such that $E\left\Vert \underline{F}_{t}\right\Vert
_{2}^{2}\leq $ $C^{\ast }$ and $E\left\Vert \underline{Y}_{t}\right\Vert
_{2}^{2}\leq \left( E\left\Vert \underline{Y}_{t}\right\Vert _{2}^{6}\right)
^{1/3}\leq $ $C^{\ast }$.\footnote{%
An explicit proof that, under Assumptions 2-1, 2-2(a)-(b), 2-5, and 2-6;
there exists some positive constant $C^{\#}$ such that $E\left\Vert 
\underline{F}_{t}\right\Vert _{2}^{6}\leq $ $C^{\#}$ and $E\left\Vert 
\underline{Y}_{t}\right\Vert _{2}^{6}\leq $ $C^{\#}$ is given in Chao, Liu,
and Swanson (2023). See Lemma OA-5 in Chao, Liu, and Swanson (2023).} $%
\square $

\medskip

\noindent \textbf{Lemma A2: }Suppose that Assumptions 2-1, 2-2, 2-3, 2-4,
2-5, 2-6, and 2-7 hold. Let $\Phi \left( \cdot \right) $ denote the
cumulative distribution function of the standard normal random variable.
Then, there exists a positive constant $A$ such that 
\begin{equation}
P\left( \left\vert S_{i,\ell ,T}\right\vert \geq z\right) \leq 2\left[
1-\Phi \left( z\right) \right] \left\{ 1+A\left( 1+z\right) ^{3}T^{-\left(
1-\alpha _{{\large 1}}\right) \frac{{\large 1}}{{\large 2}}}\right\}
\label{moderate deviation bd}
\end{equation}%
for $i\in H=\left\{ k\in \left\{ 1,....,N\right\} :\gamma _{k}=0\right\} $,
for $\ell \in \left\{ 1,...,d\right\} $, for $T$ sufficiently large, and for
all $z$ such that $0\leq z\leq c_{0}\min \left\{ T^{\left( 1-\alpha _{%
{\large 1}}\right) {\large /}6},T^{\alpha _{2}/2}\right\} $ with $c_{0}$
being a positive constant.

\noindent \textbf{Proof of Lemma A2: }Note first that, for any $i$ such that

\noindent $i\in H=\left\{ k\in \left\{ 1,....,N\right\} :\gamma
_{k}=0\right\} $, the formula for $S_{i,\ell ,T}$ reduces to:

\noindent $S_{i,\ell ,T}=\left( \dsum\nolimits_{r=1}^{q}\left[
\dsum\nolimits_{t=\left( r-1\right) \tau +p}^{\left( r-1\right) \tau +\tau
_{1}+p-1}y_{\ell ,t{\LARGE +}1}u_{it}\right] ^{2}\right) ^{-\frac{{\large 1}%
}{{\large 2}}}\dsum\nolimits_{r=1}^{q}\dsum\nolimits_{t=\left( r-1\right)
\tau +p}^{\left( r-1\right) \tau +\tau _{1}+p-1}y_{\ell ,t{\LARGE +}1}u_{it}$%
.

\noindent Hence, to verify the conditions of Theorem 4.1 of Chen, Shao, Wu,
and Xu (2016), we set $X_{it}=u_{it}y_{\ell ,t{\LARGE +}1}$, and note that $E%
\left[ X_{it}\right] =E\left[ u_{it}y_{\ell ,t{\LARGE +}1}\right] =E_{Y}%
\left[ E\left[ u_{it}\right] y_{\ell ,t{\LARGE +}1}\right] =0$ , where the
second equality follows by the law of iterated expectations given that
Assumption 2-4 implies the independence of $u_{it}$ and $y_{\ell ,t{\LARGE +}%
1}$ and where the third equality follows by Assumption \ 2-3(a). Hence, the
first part of condition (4.1) of Chen, Shao, Wu, and Xu (2016) is fulfilled.
Moreover, in light of Assumption 2-3(b) and in light of the fact that, under
Assumptions 2-1, 2-2(a)-(b), 2-5, and 2-6; one can show by straightforward
calculations that there exists a positive constant $\overline{C}$ such that $%
E\left\Vert \underline{Y}_{t}\right\Vert _{2}^{6}\leq $ $\overline{C}$; we
see that there exists some positive constant $c_{1}$ such that, for every $%
\ell \in \left\{ 1...,d\right\} $,%
\begin{eqnarray*}
E\left[ \left\vert X_{it}\right\vert ^{\frac{{\large 31}}{{\large 10}}}%
\right] &=&E\left[ \left\vert u_{it}y_{\ell ,t+1}\right\vert ^{\frac{{\large %
31}}{{\large 10}}}\right] \leq \left( E\left\vert u_{it}\right\vert ^{\frac{%
{\large 186}}{{\large 29}}}\right) ^{\frac{{\large 29}}{{\large 60}}}\left(
E\left\vert y_{\ell ,t+1}\right\vert ^{6}\right) ^{\frac{{\large 31}}{%
{\large 60}}} \\
&\leq &\left[ \left( E\left\vert u_{it}\right\vert ^{\frac{{\large 186}}{%
{\large 29}}}\right) ^{\frac{{\large 29}}{{\large 186}}}\right] ^{\frac{%
{\large 31}}{{\large 10}}}\left[ E\left(
\dsum\limits_{k=1}^{d}\dsum\limits_{j=0}^{p-1}y_{k,t+1-j}^{2}\right) ^{3}%
\right] ^{\frac{{\large 31}}{{\large 60}}} \\
&\leq &\left[ \left( E\left\vert u_{it}\right\vert ^{7}\right) ^{\frac{%
{\large 1}}{{\large 7}}}\right] ^{\frac{{\large 31}}{{\large 10}}}\left[
\left( E\left\Vert \underline{Y}_{t+1}\right\Vert _{2}^{6}\right) ^{\frac{%
{\large 1}}{{\large 6}}}\right] ^{\frac{{\large 31}}{{\large 10}}}\leq
c_{1}^{\frac{{\large 31}}{{\large 10}}}\text{,}
\end{eqnarray*}%
where the first and third inequalities above follow, respectively, by H\"{o}%
lder's and Liapunov's inequalities. Hence, the second part of condition
(4.1) of Chen, Shao, Wu, and Xu (2016) is also fulfilled with $r=\frac{31}{10%
}>2$. Moreover, note that, by Assumption 2-7, for all $r\geq 1$ and $\tau
_{1}\geq 1:$ 
\begin{equation*}
E\left\{ \left[ \dsum\limits_{t=\left( r-1\right) \tau +p}^{\left(
r-1\right) \tau +\tau _{1}+p-1}X_{it}\right] ^{2}\right\} =\tau _{1}E\left\{ %
\left[ \frac{1}{\sqrt{\tau _{1}}}\dsum\limits_{t=\left( r-1\right) \tau
+p}^{\left( r-1\right) \tau +\tau _{1}+p-1}y_{\ell ,t{\LARGE +}1}u_{it}%
\right] ^{2}\right\} \geq \tau _{1}\underline{c}\text{,}
\end{equation*}%
so that condition (4.2) of Chen, Shao, Wu, and Xu (2016) is satisfied here.
Now, making use of Assumption 2-3(c) and Assumption 2-4 and applying Theorem
2.1 of Pham and Tran (1985), it can be shown that $\left\{ \left( y_{\ell ,t%
{\LARGE +}1},u_{it}\right) ^{\prime }\right\} $ is $\beta $ mixing with $%
\beta $ mixing coefficient satisfying $\beta \left( m\right) \leq \overline{a%
}_{1}\exp \left\{ -a_{2}m\right\} $ for some constants $\overline{a}_{1}>0$
and $a_{2}>0$. Next, define $X_{it}=y_{\ell ,t{\LARGE +}1}u_{it}$, and note
that $\left\{ X_{it}\right\} $ is a $\beta $-mixing process with $\beta $%
-mixing coefficient $\beta _{X,m}$ satisfying the condition $\beta
_{X,m}\leq a_{1}\exp \left\{ -a_{2}m\right\} $ for some constant $a_{1}>0$
and for all $m$ sufficiently large, given that measurable functions of a
finite number of $\beta $-mixing random variables are also $\beta $-mixing,
with $\beta $-mixing coefficients having the same order of magnitude%
\footnote{%
For $\alpha $-mixing and $\phi $-mixing, this result is given in Theorem
14.1 of Davidson (1994). However, using essentially the same argument as
that given in the proof of Theorem 14.1, one can also prove a similar result
for $\beta $-mixing. For an explicit proof of this result, see Lemma OA-2
part (a) in Chao, Liu, and Swanson (2023).}. It follows that $\left\{
X_{it}\right\} $ satisfies the $\beta $ mixing condition (2.1) stipulated in
Chen, Shao, Wu, and Xu (2016) for all $i\in H$. Hence, by applying Theorem
4.1 of Chen, Shao, Wu, and Xu (2016) for the case where $\delta =1$\footnote{%
Note that Theorem 4.1 of Chen, Shao, Wu and Xu (2016) requires that $%
0<\delta \leq 1$ and $\delta <r-2$. These conditions are satisfied here
given that we choose $\delta =1$ and $r=31/10$.}, we obtain the Cram\'{e}%
r-type moderate deviation result%
\begin{equation}
\frac{P\left\{ \overline{S}_{i,\ell ,T}/\sqrt{\overline{V}_{i,\ell ,T}}\geq
z\right\} }{1-\Phi \left( z\right) }=1+O\left( 1\right) \left( 1+z\right)
^{3}T^{-\left( 1-\alpha _{{\large 1}}\right) \frac{{\large 1}}{{\large 2}}},
\label{PSV greater than z}
\end{equation}%
which holds for all $0\leq z\leq c_{0}\min \left\{ T^{\left( 1-{\large %
\alpha }_{1}\right) /6},T^{\alpha _{2}/2}\right\} $ and for $\left\vert
O\left( 1\right) \right\vert \leq A$, where $A$ is an absolute constant and
where $\overline{S}_{i,\ell ,T}$ and $\overline{V}_{i,\ell ,T}$ are as
defined in expression (\ref{num and denom max stat}).

Next, consider obtaining a moderate deviation result for

\noindent $P\left\{ -\overline{S}_{i,\ell ,T}/\sqrt{\overline{V}_{i,\ell ,T}}%
\geq z\right\} /\left[ 1-\Phi \left( z\right) \right] $. As $\overline{S}%
_{i,\ell ,T}=\dsum\nolimits_{r=1}^{q}\dsum\nolimits_{t=\left( r-1\right)
\tau +p}^{\left( r-1\right) \tau +\tau _{1}+p-1}\left( -u_{it}y_{\ell ,t%
{\LARGE +}1}\right) $, we can take $X_{it}=-u_{it}y_{\ell ,t{\LARGE +}1}$,
and note that, by calculations similar to those given above, we have $E\left[
X_{it}\right] =E\left[ -u_{it}y_{\ell ,t{\LARGE +}1}\right] =0$, $E\left[
\left\vert X_{it}\right\vert ^{\frac{{\large 31}}{{\large 10}}}\right] =E%
\left[ \left\vert -u_{it}y_{\ell ,t{\LARGE +}1}\right\vert ^{\frac{{\large 31%
}}{{\large 10}}}\right] =E\left[ \left\vert u_{it}y_{\ell ,t{\LARGE +}%
1}\right\vert ^{\frac{{\large 31}}{{\large 10}}}\right] \leq c_{1}^{\frac{%
{\large 31}}{{\large 10}}}$, and 
\begin{equation*}
E\left\{ \left[ \dsum\limits_{t=\left( r-1\right) \tau +p}^{\left(
r-1\right) \tau +\tau _{1}+p-1}X_{it}\right] ^{2}\right\} =E\left\{ \left[
\dsum\limits_{t=\left( r-1\right) \tau +p}^{\left( r-1\right) \tau +\tau
_{1}+p-1}\left( -u_{it}y_{\ell ,t{\LARGE +}1}\right) \right] ^{2}\right\}
\geq \underline{c}\tau _{1}.
\end{equation*}

\noindent Moreover, it is easily seen that $\left\{ X_{it}\right\} $ (with $%
X_{it}=-u_{it}y_{\ell ,t{\LARGE +}1}$) also satisfies the $\beta $ mixing
condition (2.1) stipulated in Chen, Shao, Wu, and Xu (2016) for every $i$.
Thus, by applying Theorem 4.1 of Chen, Shao, Wu, and Xu (2016), we also
obtain the Cram\'{e}r-type moderate deviation result%
\begin{equation}
\frac{P\left\{ -\overline{S}_{i,\ell ,T}/\sqrt{\overline{V}_{i,\ell ,T}}\geq
z\right\} }{1-\Phi \left( z\right) }=1+O\left( 1\right) \left( 1+z\right)
^{3}T^{-\left( 1-\alpha _{{\large 1}}\right) \frac{{\large 1}}{{\large 2}}},
\label{PSV less than -z}
\end{equation}%
which holds for all $0\leq z\leq c_{0}\min \left\{ T^{\left( 1-{\large %
\alpha }_{1}\right) /6},T^{\alpha _{2}/2}\right\} $ and for $\left\vert
O\left( 1\right) \right\vert \leq A$ with $A$ being an absolute constant.
Next, note that:

\noindent $\left\vert \frac{P\left( \left\vert S_{i,\ell ,T}\right\vert \geq
z\right) }{2\left[ 1-\Phi \left( z\right) \right] }-1\right\vert =\left\vert 
\frac{P\left( \left\vert \overline{S}_{i,\ell ,T}/\sqrt{\overline{V}_{i,\ell
,T}}\right\vert \geq z\right) }{2\left[ 1-\Phi \left( z\right) \right] }%
-1\right\vert $

\noindent\ $\ \ \ \ \ \ \ \ \ \ \ \ \ \ \ \ \ \ \ \ \ \ \ =\left\vert \frac{%
P\left( \left\{ \overline{S}_{i,\ell ,T}/\sqrt{\overline{V}_{i,\ell ,T}}\geq
z\right\} \cup \left\{ -\overline{S}_{i,\ell ,T}/\sqrt{\overline{V}_{i,\ell
,T}}\geq z\right\} \right) }{2\left[ 1-\Phi \left( z\right) \right] }%
-1\right\vert $

\noindent\ $\ \ \ \ \ \ \ \ \ \ \ \ \ \ \ \ \ \ \ \ \ \ \ =\left\vert \frac{%
P\left( \overline{S}_{i,\ell ,T}/\sqrt{\overline{V}_{i,\ell ,T}}\geq
z\right) +P\left( -\overline{S}_{i,\ell ,T}/\sqrt{\overline{V}_{i,\ell ,T}}%
\geq z\right) }{2\left[ 1-\Phi \left( z\right) \right] }-1\right\vert $

\noindent\ $\ \ \ \ \ \ \ \ \ \ \ \ \ \ \ \ \ \ \ \ \ \ \ \ \ \ \left( \text{%
since }\left\{ \overline{S}_{i,\ell ,T}/\sqrt{\overline{V}_{i,\ell ,T}}\geq
z\right\} \cap \left\{ -\overline{S}_{i,\ell ,T}/\sqrt{\overline{V}_{i,\ell
,T}}\geq z\right\} =\varnothing \text{ w.p.1}\right) $

\noindent\ $\ \ \ \ \ \ \ \ \ \ \ \ \ \ \ \ \ \ \ \ \ \ \ \leq \frac{1}{2}%
\left\vert \frac{P\left( \overline{S}_{i,\ell ,T}/\sqrt{\overline{V}_{i,\ell
,T}}\geq z\right) }{1-\Phi \left( z\right) }-1\right\vert +\frac{1}{2}%
\left\vert \frac{P\left\{ -\overline{S}_{i,\ell ,T}/\sqrt{\overline{V}%
_{i,\ell ,T}}\geq z\right\} }{1-\Phi \left( z\right) }-1\right\vert .$

\noindent Thus, in light of expressions (\ref{PSV greater than z}) and (\ref%
{PSV less than -z}), we have that: 
\begin{eqnarray*}
&&\left\vert \frac{P\left( \left\vert S_{i,\ell ,T}\right\vert \geq z\right) 
}{2\left[ 1-\Phi \left( z\right) \right] }-1\right\vert \\
&\leq &\frac{1}{2}\left\vert \frac{P\left( \overline{S}_{i,\ell ,T}/\sqrt{%
\overline{V}_{i,\ell ,T}}\geq z\right) }{1-\Phi \left( z\right) }%
-1\right\vert +\frac{1}{2}\left\vert \frac{P\left\{ -\overline{S}_{i,\ell
,T}/\sqrt{\overline{V}_{i,\ell ,T}}\geq z\right\} }{1-\Phi \left( z\right) }%
-1\right\vert \\
&\leq &\frac{A}{2}\left( 1+z\right) ^{3}T^{-\left( 1-\alpha _{{\large 1}%
}\right) \frac{{\large 1}}{{\large 2}}}+\frac{A}{2}\left( 1+z\right)
^{3}T^{-\left( 1-\alpha _{{\large 1}}\right) \frac{{\large 1}}{{\large 2}}%
}=A\left( 1+z\right) ^{3}T^{-\left( 1-\alpha _{{\large 1}}\right) \frac{%
{\large 1}}{{\large 2}}}
\end{eqnarray*}%
It then follows that:%
\begin{equation}
-A\left( 1+z\right) ^{3}T^{-\left( 1-\alpha _{{\large 1}}\right) \frac{%
{\large 1}}{{\large 2}}}\leq \frac{P\left( \left\vert S_{i,\ell
,T}\right\vert \geq z\right) }{2\left[ 1-\Phi \left( z\right) \right] }%
-1\leq A\left( 1+z\right) ^{3}T^{-\left( 1-\alpha _{{\large 1}}\right) \frac{%
{\large 1}}{{\large 2}}}  \label{double sided ineq}
\end{equation}%
where $S_{i,\ell ,T}=$ $\overline{S}_{i,\ell ,T}/\sqrt{\overline{V}_{i,\ell
,T}}$. Focusing on the right-hand part of the inequality in (\ref{double
sided ineq}), we have that:

\noindent $P\left( \left\vert S_{i,\ell ,T}\right\vert \geq z\right) /\left(
2\left[ 1-\Phi \left( z\right) \right] \right) -1\leq A\left( 1+z\right)
^{3}T^{-\left( 1-\alpha _{{\large 1}}\right) \frac{{\large 1}}{{\large 2}}}$%
. Simple rearrangement of this inequality then leads to the desired result:%
\begin{equation*}
P\left( \left\vert S_{i,\ell ,T}\right\vert \geq z\right) \leq 2\left[
1-\Phi \left( z\right) \right] \left\{ 1+A\left( 1+z\right) ^{3}T^{-\left(
1-\alpha _{{\large 1}}\right) \frac{{\large 1}}{{\large 2}}}\right\} ,
\end{equation*}

\noindent\ \noindent which holds for all $i\in H=\left\{ k\in \left\{
1,....,N\right\} :\gamma _{k}=0\right\} $, for every $\ell \in \left\{
1,...,d\right\} $, for all $T$ sufficiently large, and for all $z$ such that 
$0\leq z\leq c_{0}\min \left\{ T^{\left( 1-{\large \alpha }_{1}\right)
/6},T^{\alpha _{2}/2}\right\} $. $\square $

\begin{thebibliography}{99}
\bibitem{} \noindent Ahn, S. C. and J. Bae (2022): \textquotedblleft
Forecasting with Partial Least Squares When a Large Number of Predictors Are
Available,\textquotedblright\ Working Paper, Arizona State University and
University of Glasgow.

\bibitem{} Anatolyev, S. and A. Mikusheva (2021): \textquotedblleft Factor
Models with Many Assets: Strong Factors, Weak Factors, and the Two-Pass
Procedure,\textquotedblright\ \textit{Journal of Econometrics}, forthcoming.

\bibitem{} Andrews, D.W.K. (1984): \textquotedblleft Non-strong Mixing
Autoregressive Processes,\textquotedblright\ \textit{Journal of Applied
Probability}, 21, 930-934.

\bibitem{} Bai, J. and S. Ng (2002): \textquotedblleft Determining the
Number of Factors in Approximate Factor Models,\textquotedblright\ \textit{%
Econometrica}, 70, 191-221.

\bibitem{} Bai, J. and S. Ng (2008): \textquotedblleft Forecasting Economic
Time Series Using Targeted Predictors,\textquotedblright\ \textit{Journal of
Econometrics}, 146, 304-317.

\bibitem{} Bai, J. and S. Ng (2021): \textquotedblleft Approximate Factor
Models with Weaker Loading,\textquotedblright\ Working Paper, Columbia
University.

\bibitem{} Bair, E., T. Hastie, D. Paul, and R. Tibshirani (2006):
\textquotedblleft Prediction by Supervised Principal
Components,\textquotedblright\ \textit{Journal of the American Statistical
Association}, 101, 119-137.

\bibitem{} Belloni, A., D. Chen, V. Chernozhukov, and C. Hansen (2012):
\textquotedblleft Sparse Models and Methods for Optimal Instruments with an
Application to Eminent Domain,\textquotedblright\ \textit{Econometrica}, 80,
2369-2429.

\bibitem{} Belloni, A., V. Chernozhukov, and C. Hansen (2014):
\textquotedblleft Inference on Treatment Effects after Selection among
High-Dimensional Controls,\textquotedblright\ \textit{Review of Economic
Studies}, 81, 608-650.

\bibitem{} Bryzgalova, S. (2016): \textquotedblleft Spurious Factors in
Linear Asset Pricing Models,\textquotedblright\ Working Paper, Stanford
Graduate School of Business.

\bibitem{} Burnside, C. (2016): \textquotedblleft Identification and
Inference in Linear Stochastic Discount Factor Models with Excess
Returns,\textquotedblright\ \textit{Journal of Financial Econometrics}, 14,
295-330.

\bibitem{} Chao, J. C., Y. Qiu, and N. R. Swanson (2023): \textquotedblleft
Consistent Factor Estimation and Forecasting in Factor-Augmented VAR
Models,\textquotedblright\ Working Paper, Rutgers University and University
of Maryland.

\bibitem{} Chao, J. C., Y. Liu, and N. R. Swanson (2023): Online Appendix to
\textquotedblleft Selecting the Relevant Variables for Factor Estimation in
VAR Models,\textquotedblright\ Working Paper, Rutgers University and
University of Maryland.

\bibitem{} Chen, X., Q. Shao, W. B. Wu, and L. Xu (2016): \textquotedblleft
Self-normalized Cram\'{e}r-type Moderate Deviations under
Dependence,\textquotedblright\ \textit{Annals of Statistics}, 44, 1593-1617.

\bibitem{} Davidson. J. (1994): \textit{Stochastic Limit Theory: An
Introduction for Econometricians}. New York: Oxford University Press.

\bibitem{} Diebold, F.X. and R.S. Mariano (1995): \textquotedblleft
Comparing Predictive Accuracy,\textquotedblright\ \textit{Journal of
Business and Economic Statistics}, 20, 134-144.

\bibitem{} Forni, M., M. Hallin, M. Lippi, and L. Reichlin (2005):
\textquotedblleft The Generalized Dynamic Factor Model, One-Sided Estimation
and Forecasting,\textquotedblright\ \textit{Journal of the American
Statistical Association}, 100, 830-840.

\bibitem{} Freyaldenhoven, S. (2021a): \textquotedblleft Factor Models with
Local Factors - Determining the Number of Relevant
Factors,\textquotedblright\ \textit{Journal of Econometrics}, forthcoming.

\bibitem{} Freyaldenhoven, S. (2021b): \textquotedblleft Identification
through Sparsity in Factor Models: The $\ell _{1}$-Rotation
Criterion,\textquotedblright\ Working Paper, Federal Reserve Bank of
Philadelphia.

\bibitem{} Giacomini, R. and H. White (2006): \textquotedblleft Tests of
Conditional Predictive Ability,\textquotedblright\ \textit{Econometrica},
74, 1545-1578.

\bibitem{} Giglio, S., D. Xiu, and D. Zhang (2021): \textquotedblleft Test
Assets and Weak Factors,\textquotedblright\ Working Paper, Yale School of
Management and the Booth School of Business, University of Chicago.

\bibitem{} Goroketskii, V. V. (1977): \textquotedblleft On the Strong Mixing
Property for Linear Sequences,\textquotedblright\ \textit{Theory of
Probability and Applications}, 22, 411-413.

\bibitem{} Gospodinov, N., R. Kan, and C. Robotti (2017): \textquotedblleft
Spurious Inference in Reduced-Rank Asset Pricing Models,\textquotedblright\ 
\textit{Econometrica}, 85, 1613-1628.

\bibitem{} Harding, M. C. (2008): \textquotedblleft Explaining the Single
Factor Bias of Arbitrage Pricing Models in Finite
Samples,\textquotedblright\ \textit{Economics Letters}, 99, 85-88.

\bibitem{} Jagannathan, R. and Z. Wang (1998): \textquotedblleft An
Asymptotic Theory for Estimating Beta-Pricing Models Using Cross-Sectional
Regression,\textquotedblright\ \textit{Journal of Finance}, 53, 1285-1309.

\bibitem{} Kan, R. and C. Zhang (1999): \textquotedblleft Two-Pass Tests of
Asset Pricing Models with Useless Factors,\textquotedblright\ \textit{%
Journal of Finance}, 54, 203-235.

\bibitem{} Kiefer, N. M. and T. J. Vogelsang (2002a): \textquotedblleft
Heteroskedasticity-Autocorrelation Robust Standard Errors Using the Bartlett
Kernel without Truncation,\textquotedblright\ \textit{Econometrica}, 70,
2093-2095.

\bibitem{} Kiefer, N. M. and T. J. Vogelsang (2002b): \textquotedblleft
Heteroskedasticity-Autocorrelation Robust Testing Using Bandwidth Equal to
Sample Size,\textquotedblright\ \textit{Econometric Theory}, 18, 1350-1366.

\bibitem{} Kim, H.-H. and N.R. Swanson (2018): \textquotedblleft Methods for
Backcasting, Nowcasting and Forecasting Using Factor-MIDAS: With an
Application to Korean GDP,\textquotedblright\ \textit{Journal of Forecasting}%
, 37, 281-302.

\bibitem{} Kleibergen, F. (2009): \textquotedblleft Tests of Risk Premia in
Linear Factor Models,\textquotedblright\ \textit{Journal of Econometrics},
149, 149-173.

\bibitem{} McCracken, M.W. and S. Ng (2016): \textquotedblleft FRED-MD: A
Monthly Database for Macroeconomic Research,\textquotedblright\ \textit{%
Journal of Business and Economic Statistics}, 34, 574-589.

\bibitem{} Onatski, A. (2012): \textquotedblleft Asymptotics of the
Principal Components Estimator of Large Factor Models with Weakly
Influential Factors,\textquotedblright\ \textit{Journal of Econometrics},
168, 244-258.

\bibitem{} Pham, T. D. and L. T. Tran (1985): \textquotedblleft Some Mixing
Properties of Time Series Models,\textquotedblright\ \textit{Stochastic
Processes and Their Applications}, 19, 297-303.

\bibitem{} Qiu, A. and Z. Qu (2021): \textquotedblleft Modeling Regime
Switching in High-Dimensional Data with Applications to U.S. Business
Cycles,\textquotedblright\ Working Paper, Boston University.

\bibitem{} Stock, J. H. and M. W. Watson (2002a): \textquotedblleft
Forecasting Using Principal Components from a Large Number of
Predictors,\textquotedblright\ \textit{Journal of the American Statistical
Association}, 97, 1167-1179.

\bibitem{} Stock, J. H. and M. W. Watson (2002b): \textquotedblleft
Macroeconomic Forecasting Using Diffusion Indexes,\textquotedblright\ 
\textit{Journal of Business and Economic Statistics}, 20, 147-162.

\bibitem{} Swanson, N.R. (1996): \textquotedblleft Forecasting Using
First-Available Versus Fully Revised Economic Time-Series
Data,\textquotedblright\ \textit{Studies in Nonlinear Dynamics and
Econometrics}, 1, 47-64.

\bibitem{} Swanson, N.R. and D. van Dijk (2006): \textquotedblleft Are
Statistical Reporting Agencies Getting It Right? Data Rationality and
Business Cycle Asymmetry,\textquotedblright\ \textit{Journal of Business and
Economic Statistics}, 24, 24-42.

\bibitem{} Zhou, Z. and X. Shao (2013): \textquotedblleft Inference for
Linear Models with Dependent Errors,\textquotedblright\ \textit{Journal of
the Royal Statistical Society Series B}, 75, 323-343.
\end{thebibliography}

\end{document}
