%2multibyte Version: 5.50.0.2953 CodePage: 936
%\geometry{left=1in,right=1in,top=1in,bottom=1in}
%\pagestyle{fancy}   %to establish a new page style.
%\fancyhf{}              %to clear the header and footer.
%\rhead{\thepage}  %to force the page number to the top right of the page.
%\renewcommand{\headrulewidth}{0pt} %on the next line.
%\renewcommand{\footrulewidth}{0pt} %on the next line.
%\pagestyle{fancy}
%\lhead{Emprical Comparison of Forecasting Method for Factor Model} %Leave the left of the header empty
%\chead{} %Leave the center of the header empty
%\rhead{\thepage} %Display this text on the right of the header
%\lfoot{By Author} %Display this text on the left of the footer
%\cfoot{} %Leave the center of the footer empty
%\rfoot{Page:\ \thepage} %Print the page number in the right footer
%\renewcommand{\headrulewidth}{0pt} %Do not print a rule below the header
%\renewcommand{\footrulewidth}{0pt} %Do not print a rule above the footer 
%\usepackage{bm}


\documentclass[12pt]{article}
%%%%%%%%%%%%%%%%%%%%%%%%%%%%%%%%%%%%%%%%%%%%%%%%%%%%%%%%%%%%%%%%%%%%%%%%%%%%%%%%%%%%%%%%%%%%%%%%%%%%%%%%%%%%%%%%%%%%%%%%%%%%%%%%%%%%%%%%%%%%%%%%%%%%%%%%%%%%%%%%%%%%%%%%%%%%%%%%%%%%%%%%%%%%%%%%%%%%%%%%%%%%%%%%%%%%%%%%%%%%%%%%%%%%%%%%%%%%%%%%%%%%%%%%%%%%
\usepackage{multirow}
\usepackage{amssymb}
\usepackage{amsmath}
\usepackage{enumerate}
\usepackage{graphics}
\usepackage{epsfig}
\usepackage{booktabs}
\usepackage{array}
\usepackage{latexsym}
\usepackage[nomarginpar,ignoremp,noheadfoot]{geometry}
\usepackage[singlespacing]{setspace}
\usepackage{fancyhdr}
\usepackage{lscape}
\usepackage{longtable}
\usepackage[sort&compress]{natbib}

\setcounter{MaxMatrixCols}{10}
%TCIDATA{OutputFilter=LATEX.DLL}
%TCIDATA{Version=5.50.0.2953}
%TCIDATA{Codepage=936}
%TCIDATA{<META NAME="SaveForMode" CONTENT="1">}
%TCIDATA{BibliographyScheme=BibTeX}
%TCIDATA{Created=Thursday, June 19, 2008 15:00:19}
%TCIDATA{LastRevised=Sunday, December 09, 2012 10:47:39}
%TCIDATA{<META NAME="GraphicsSave" CONTENT="32">}
%TCIDATA{<META NAME="DocumentShell" CONTENT="Standard LaTeX\Blank - Standard LaTeX Article">}
%TCIDATA{Language=American English}
%TCIDATA{CSTFile=article.cst}
%TCIDATA{PageSetup=72,72,72,72,0}
%TCIDATA{EvenPages=
%H=36
%F=36
%}

%TCIDATA{OddPages=
%H=36
%F=36
%}

%TCIDATA{FirstPage=
%H=36
%F=36
%}


\newtheorem{theorem}{Theorem}
\newtheorem{acknowledgement}[theorem]{Acknowledgement}
\newtheorem{algorithm}[theorem]{Algorithm}
\newtheorem{axiom}[theorem]{Axiom}
\newtheorem{case}[theorem]{Case}
\newtheorem{claim}[theorem]{Claim}
\newtheorem{conclusion}[theorem]{Conclusion}
\newtheorem{condition}[theorem]{Condition}
\newtheorem{conjecture}[theorem]{Conjecture}
\newtheorem{corollary}[theorem]{Corollary}
\newtheorem{criterion}[theorem]{Criterion}
\newtheorem{definition}[theorem]{Definition}
\newtheorem{example}[theorem]{Example}
\newtheorem{exercise}[theorem]{Exercise}
\newtheorem{lemma}[theorem]{Lemma}
\newtheorem{notation}[theorem]{Notation}
\newtheorem{problem}[theorem]{Problem}
\newtheorem{proposition}[theorem]{Proposition}
\newtheorem{remark}[theorem]{Remark}
\newtheorem{solution}[theorem]{Solution}
\newtheorem{summary}[theorem]{Summary}
\newcolumntype{x}[1]{>{\centering\arraybackslash}p{#1}}
\newcolumntype{y}[1]{>{\raggedright\arraybackslash}p{#1}}
\newenvironment{proof}[1][Proof]{\noindent\textbf{#1.} }{\ \rule{0.5em}{0.5em}}
\input{tcilatex}
\geometry{headheight=15pt,headsep=1cm,vmargin={2.3cm,2.3cm},hmargin={2.3cm,2.3cm}}
\setlength{\footskip}{1cm}
\setlength{\bibsep}{1.5pt}

\begin{document}


\begin{center}
{\Large Forecasting Financial and Macroeconomic Variables Using Data
Reduction Methods: New Empirical Evidence}$^{\ast }$

\bigskip

{\large Hyun Hak Kim}$^{1}${\large \ and Norman R. Swanson}$^{2}$

$^{1}${\large Bank of Korea and }$^{2}${\large Rutgers University}

\bigskip

{\large October 2010 (revised December 2012)}

\bigskip

{\large Abstract}
\end{center}

{\footnotesize In this paper, we empirically assess the predictive accuracy
of a large group of models that are specified using principle components and
other shrinkage techniques, including Bayesian model averaging and various
bagging, boosting, least angle regression and related methods. Our results
suggest that model averaging does not dominate other well designed
prediction model specification methods, and that using \textquotedblleft
hybrid\textquotedblright\ combination factor/shrinkage methods often yields
superior predictions. More specifically, when using recursive estimation
windows, which dominate other \textquotedblleft windowing\textquotedblright\
approaches, \textquotedblleft hybrid\textquotedblright\ models are mean
square forecast error \textquotedblleft best\textquotedblright\ around 1/3
of the time, when used to predict 11 key macroeconomic indicators at various
forecast horizons. Baseline linear (factor) models also \textquotedblleft
win\textquotedblright\ around 1/3 of the time, as do model averaging
methods. Interestingly, these broad findings change noticably when
considering different sub-samples. For example, when used to predict only
recessionary periods, \textquotedblleft hybrid\textquotedblright\ models
\textquotedblleft win\textquotedblright\ in 7 of 11 cases, when condensing
findings across all \textquotedblleft windowing\textquotedblright\
approaches, estimation methods, and models, while model averaging does not
\textquotedblleft win\textquotedblright\ in a single case. However, in
expansions, and during the 1990s, model averaging wins almost 1/2 of the
time. Overall, combination factor/shrinkage methods \textquotedblleft
win\textquotedblright\ approximately 1/2 of the time in 4 of 6 different
sample periods. Ancillary findings based on our forecasting experiments
underscore the advantages of using recursive estimation strategies, and
provide new evidence of the usefulness of yield and yield-spread variables
in nonlinear prediction model specification.}

\setcounter{page}{0} \thispagestyle{empty}

\bigskip

\noindent \textit{Keywords}: bagging, Bayesian model averaging, boosting,
diffusion index, elastic net, forecasting, least angle regression,
non-negative garotte, prediction, reality check, ridge regression.

\noindent \textit{JEL Classification:} C1, C22, C52, C58.

\noindent \_\_\_\_\_\_\_\_\_\_\_\_\_\_\_\_\_\_\_\_\_\_\_\_\_\_\_

$^{\ast }${\footnotesize \ Hyun Hak Kim (khdouble@bok.or.kr), The Bank of
Korea, 55 Namdaemunno, Jung-Gu, Seoul 100-794, Korea. Norman R. Swanson \
(nswanson@econ.rutgers.edu), Department of Economics, Rutgers University, 75
Hamilton Street, New Brunswick, NJ 08901, USA. This paper was prepared for
the Sir Clive W.J. Granger Memorial Conference held at Nottingham University
in May 2010. We are very grateful to the organizers, Rob Taylor and David
Harvey, for hosting this special event. We have benefited from numerous
useful comments made on earlier versions of this paper by the editor, Graham
Elliott, and by an anonymous referee. In addition to thanking the editor and
referee, the authors wish to thank the seminar participants from the
conference, as well as at the Bank of Canada and Rutgers University, for
useful comments. Additional thanks are owed to Nii Armah, Dongjun Chung,
Valentina Corradi, David Hendry, Gary Koop, John Landon-Lane, Fuchun Li,
Greg Tkacz, Hiroki Tsurumi, and Hal White for numerous useful suggestions on
an earlier version of this paper. The views stated herein are those of the
authors and not necessarily those of the Bank of Korea. }\setlength{%
\baselineskip}{1.3\baselineskip}

\section{Introduction}

Technological advances over the last five decades have led to impressive
gains in not only computational power, but also in the quantity of available
financial and macroeconomic data. Indeed, there has been something of a race
going on in recent years, as technology, both computational and theoretical,
has been hard pressed to keep up with the ever increasing mountain of (big)
data available for empirical use. From a computational perspective, this has
helped spur the development of data shrinkage techniques, for example. In
economics, one of the most widely applied of these is diffusion index
methodology. Diffusion index techniques offer a simple and sensible approach
for extracting common factors that underlie the dynamic evolution of large
numbers of variables. To be more specific, let $Y$ be a time series vector
of dimension $(T\times 1),$ let $X$ be a time-series predictor matrix of
dimension $\left( T\times N\right) ,$ and define the following factor model,
where $F_{t}$ denotes a $1\times r$ vector of unobserved common factors that
can be extracted from $X_{t}.$ Namely, let $X_{t}=F_{t}\Lambda ^{\prime
}+e_{t},$ where $e_{t}$ is an $1\times N$ vector of disturbances and $%
\Lambda $ is an $N\times r$ coefficient matrix. Using common factors
extracted from the above model, Stock and Watson (2002a,b) as well as \cite%
{BN06EMTR} examine linear autoregressive (AR) forecasting models augmented
by the inclusion of common factors.

In this paper, we use the forecasting models of Stock and Watson (2002a,b)
and Bai and Ng (2006a) as a starting point. In particular, we first estimate
unobserved factors, say $\hat{F}_{t},$ and then forecast a scalar target
variable, $Y_{t+h},$ using observed variables and $\hat{F}_{t}.$ We then
draw on the fact that even though factor models are now widely used, several
issues remain outstanding, such as the determination of the (number of)
factors to be used in subsequent prediction model specification (see e.g., 
\cite{BN02EMC, BN06JoE, BN08JEMC}). In light of this, and in order to add
functional flexibility, we implement prediction models where the numbers and
functions of factors are selected using a variety of shrinkage methods. In
this sense, we add to the recent work of \cite{SW05WP} as well as \cite%
{BN08JEMC,BN08JAE}, who survey several methods for shrinkage in the context
of factor augmented autoregression\ models. Shrinkage methods considered in
this paper include bagging, boosting, Bayesian model averaging, simple model
averaging, ridge regression, least angle regression, elastic net and the
non-negative garotte. We also evaluate various linear models, and hence also
add to the recent work of \cite{PPT2010}, who carry out a broad examination
of factor-augmented vector autoregression models.

In summary, the purpose of this paper is to empirically assess the
predictive accuracy of various linear models; pure principal component
models; principal components models where the factors are constructed using
subsets of variables first selected based on shrinkage techniques; principle
components models where the factors are first constructed, and are then
refined using shrinkage methods; models constructed by directly applying
shrinkage methods (other than principle components) to the data; and a
number of model averaging methods.\ The \textquotedblleft
horse-race\textquotedblright\ that we carry out allows us to provide new
evidence on the usefulness of factors in general as well as on various
related issues such as whether model averaging \textquotedblleft
wins\textquotedblright\ as often as is usually found to be the case in
empirical investigations of this sort.

The variables that we predict include a variety of macroeconomic variables
that are useful for evaluating the state of the economy. More specifically,
forecasts are constructed for eleven series, including: the unemployment
rate, personal income less transfer payments, the 10 year Treasury-bond
yield, the consumer price index, the producer price index, non-farm payroll
employment, housing starts, industrial production, M2, the S\&P 500 index,
and gross domestic product. These variables constitute 11 of the 14
variables (for which long data samples are available) that the Federal
Reserve takes into account, when formulating the nation's monetary policy.
In particular, as has been noted in \cite{ArSwan10} and on the Federal
Reserve Bank of New York's website: \textit{\textquotedblleft In formulating
the nation's monetary policy, the Federal Reserve considers a number of
factors, including the economic and financial indicators which follow, as
well as the anecdotal reports compiled in the Beige Book. Real Gross
Domestic Product (GDP); Consumer Price Index (CPI); Nonfarm Payroll
Employment Housing Starts; Industrial Production/Capacity Utilization;
Retail Sales; Business Sales and Inventories; Advance Durable Goods
Shipments, New Orders and Unfilled Orders; Lightweight Vehicle Sales; Yield
on 10-year Treasury Bond; S\&P 500 Stock Index; M2.\textquotedblright }

Our finding can be summarized as follows. First, for a number of our target
variables, we find that various sophisticated shrinkage methods, such as
component-wise boosting, bagging,ridge regression, least angle regression,
the elastic net, and the non-negative garotte yield predictions with lower
mean square forecast errors (MSFEs) than a variety of benchmark linear
autoregressive forecasting models constructed using only observable
variables. Moreover, these shrinkage methods, when used in conjunction with
diffusion indexes, yield a surprising number of MSFE \textquotedblleft
best\textquotedblright\ models, hence suggesting that \textquotedblleft
hybrid\textquotedblright\ models that combine diffusion index methodology
with other shrinkage techniques offer a convenient way to filter the
information contained in large-scale economic datasets, particularly if they
are specified using sophisticated shrinkage techniques. More specifically,
when using recursive estimation windows, which dominate other
\textquotedblleft windowing\textquotedblright\ approaches, \textquotedblleft
hybrid\textquotedblright\ models are MSFE \textquotedblleft
best\textquotedblright\ around 1/3 of the time, when used to predict 11 key
macroeconomic indicators at various forecast horizons. Baseline linear
(factor) models also \textquotedblleft win\textquotedblright\ around 1/3 of
the time, as do model averaging methods. Interestingly, these broad findings
change noticably when considering different sub-samples. For example, when
used to predict only recessionary periods, \textquotedblleft
hybrid\textquotedblright\ models \textquotedblleft win\textquotedblright\ in
7 of 11 cases, when condensing findings across all \textquotedblleft
windowing\textquotedblright\ approaches, estimation methods, and models,
while model averaging does not \textquotedblleft win\textquotedblright\ in a
single case. However, in expansions, and during the 1990s, model averaging
wins almost 1/2 of the time. Overall, combination factor/shrinkage methods
\textquotedblleft win\textquotedblright\ approximately 1/2 of the time in 4
of 6 different sample periods. Ancillary findings based on our forecasting
experiments underscore the advantages of using recursive estimation
strategies\footnote{%
For further discussion of estimation windows and the related issue of
structural breaks, see \cite{PPT2010}.}, and provide new evidence of the
usefulness of yield and yield-spread variables in nonlinear prediction model
specification.

Although we leave many important issues to future research, such as the
prevalence of structural breaks other than level shifts, and the use of even
more general nonlinear methods for describing the data series that we
examine, we believe that results presented in this paper add not only to the
diffusion index literature, but also to the extraordinary collection of
papers on forecasting that Clive W.J. Granger wrote during his decades long
research career. Indeed, as we and others have said many times, we believe
that Sir Clive W.J. Granger is in many respects the father of time series
forecasting, and we salute his innumerable contributions in areas from
predictive accuracy testing, model selection analysis, and forecast
combination, to forecast loss function analysis, forecasting using
nonstationary data, and nonlinear forecasting model specification.

The rest of the paper is organized as follows. In the next section we
provide a brief survey of diffusion index models.\ In Section 3, we briefly
survey the robust shrinkage estimation methods used in our prediction
experiments. Data, forecasting methods, and benchmark forecasting models are
discussed in Sections 4 and 5, while empirical results are presented in
Section 6. Concluding remarks are gathered in Section 7.

\section{Diffusion Index Models}

Recent forecasting studies using large-scale datasets and pseudo
out-of-sample forecasting include: \cite{ABM02CEPR}, \cite%
{BoNg05IJCB,BoNg06JEMC}, \cite{FHLR05JASA}, and \cite%
{SW99JME,SW02JASA,SW05WP,SW05bNBER,SW2006HEF}. \cite{SW2006HEF} discuss in
some detail the literature on the use of diffusion indices for forecasting.
In the following brief discussion of diffusion index methodology, we follow 
\cite{SW02JASA}.

Let $X_{tj}$\ be the observed datum for the $j-$th cross-sectional unit at
time $t$, for $t=1,...,T$\ and $j=1,...,N.$\ We begin with the following
model:%
\begin{equation}
X_{tj}=F_{t}\Lambda _{j}^{\prime }+e_{tj},  \label{eq75}
\end{equation}%
where $F_{t}$\ is a $1$\ $\times $\ $r$\ vector of common factors, $\Lambda
_{j}$\ is an $1$\ $\times r$\ vector of factor loadings associated with $%
F_{t}$, and $e_{tj}$\ is the idiosyncratic component of $X_{tj}$. The
product $F_{t}\Lambda _{j}^{\prime }$\ is called the common component of $%
X_{tj}$. This is a useful dimension reducing factor representation of the
data, particularly when $r<<N$, as is usually assumed to be the case in the
empirical literature. Following \cite{BN02EMC}, the whole panel of data $%
X=\left( X_{1},...,X_{N}\right) $, where $X_{i},$ $i=1,...,N,$ is a $T$\ \ $%
\times $\ $1$\ vector of observations on a single variable, can be
represented as in (\ref{eq75}). \cite{CK86JFE,CK88JFE,CK93JF} note that the
factors can be consistently estimated by principal components, as $%
N\rightarrow \infty ,$\ even if $e_{tj}$\ is weakly cross-sectionally
correlated. Similarly, \cite{FHLR05JASA} and \cite{SW02JASA} discuss
consistent estimation of the factors when $N,T\rightarrow \infty .$ We work
with high-dimensional factor models that allow both $N$\ and $T$\ to tend to
infinity, and in which $e_{tj}$\ may be serially and cross-sectionally
correlated, so that the covariance matrix of $e_{t}=\left(
e_{t1},...,e_{tN}\right) $\ does not have to be a diagonal matrix. We will
also assume that $\{F_{t}\}$\ and $\{e_{tj}\}$\ are two groups of mutually
independent stochastic variables. Furthermore, it is well known that if $%
\Lambda =\left( \Lambda _{1},...,\Lambda _{N}\right) ^{\prime }$\ for $%
F_{t}\Lambda ^{\prime }$\ $=F_{t}QQ^{-1}\Lambda ^{\prime }$\ , a
normalization is needed in order to identify the factors, where $Q$\ is a
nonsingular matrix. Assuming that $(\Lambda ^{\prime }\Lambda /N)\rightarrow
I_{r}$, we restrict $Q$\ to be orthonormal. This assumption, together with
others noted in \cite{SW02JASA} and \cite{BN02EMC}, enables us to identify
the factors up to a change of sign and consistently estimate them up to an
orthonormal transformation.

With regard to choice of $r$, note that \cite{BN02EMC} provide one solution
to the problem of choosing the number of factors. They establish convergence
rates for factor estimates under consistent estimation of the number of
factors, $r,$ and propose panel criteria to consistently estimate the number
of factors. Namely, \cite{BN02EMC} define selection criteria of the form $%
PC\left( r\right) =V\left( r,\hat{F}\right) +rh\left( N,T\right) ,$ where $%
h\left( \cdot \right) $ is a penalty function. In this paper, the following
version is used (for discussion, see \cite{BN02EMC} and \cite{ArSwan08}):%
\begin{equation}
SIC(r)=V\left( r,\hat{F}\right) +r\hat{\sigma}^{2}\left( \frac{\left(
N+T-r\right) \ln \left( NT\right) }{NT}\right) .
\end{equation}%
A consistent estimate of the true number of factors is $\hat{r}=\arg 
\underset{0\leq r\leq r_{\max }}{\min }SIC(r).$ In a number of our models,
we use this criterion for choosing the number of factors. However, as
discussed above, we also use a variety of shrinkage methods to specify
numbers and functions of factors to be used in prediction models. In
addition, shrinkage methods are also directly implemented, yielding
\textquotedblleft factor-free\textquotedblright\ prediction models. Finally,
shrinkage methods are used to \textquotedblleft
pre-select\textquotedblright\ subsets of $X$ for subsequent use in the
construction of factors.

The basic structure of the forecasting models examined is the same as that
examined in \cite{ABM02CEPR}, \cite%
{BN02EMC,BN06JoE,BN06EMTR,BN08JEMC,BN08JAE}, \cite{BoNg05IJCB} and \cite%
{SW02JASA,SW05WP,SW05bNBER,SW2006HEF}. In particular, we consider models of
the following generic form:

\begin{equation}
Y_{t+h}=W_{t}\beta _{W}+F_{t}\beta _{F}+\varepsilon _{t+h},  \label{eq41a}
\end{equation}%
where $h$ is the forecast horizon, $Y_{t}$ is the scalar valued
\textquotedblleft target\textquotedblright\ variable to be forecasted, $%
W_{t} $ is a $1\times s$ vector of observable variables, including lags of $%
Y_{t}$, $\varepsilon _{t}$ is a disturbance term, and the $\beta $'s are
parameters estimated using least squares. Forecasts of $Y_{t+h}$\ based on (%
\ref{eq41a}) involve a two step procedure because both the regressors and
coefficients in the forecasting equations are unknown. The data $X_{t}$\ are
first used to estimate the factors, $\hat{F}_{t},$\ by means of principal
components. With the estimated factors in hand, we obtain the estimators $%
\hat{\beta}_{F}$\ and $\hat{\beta}_{W}$\ by regressing $Y_{t+h}$\ on $\hat{F}%
_{t}$\ and $W_{t}$. Of note is that if $\sqrt{T}/N\rightarrow 0$, then the
generated regressor problem does not arise, in the sense that least squares
estimates of $\hat{\beta}_{F}$\ and $\hat{\beta}_{W}$\ are $\sqrt{T}$\
consistent and asymptotically normal (see \cite{BN08JEMC}). As discussed
above, one aspect of our empirical analysis is that we use various shrinkage
methods for estimating $\hat{\beta}_{F}$\ and then compare the predictive
accuracy of the resulting forecasting models.\footnote{%
We refer the reader to \cite{SW02JASA,SW05WP,SW05bNBER,SW99JME} and \cite%
{BN02EMC,BN08JAE,BN08JEMC} for a detailed explanation of this procedure, and
to \cite{CK86JFE,CK88JFE,CK93JF}, \cite{FHLR05JASA} and \cite{ArSwan08} for
further detailed discussion of factor augmented autoregression models.
Finally, note that \cite{DH99JASA} also analyze the properties of forecasts
constructed using principal components when $N$\ and $T$\ are large,
although they carry out their analysis under the assumption that the error
processes $\left\{ e_{tj},\varepsilon _{t+h}\right\} $\ are
cross-sectionally and serially $iid.$\ }

\section{Robust Estimation Techniques}

We consider a variety of \textquotedblleft robust\textquotedblright\
estimation techniques including statistical learning algorithms (bagging and
boosting), as well as various penalized regression methods including ridge
regression, least angle regression, the elastic net, and the non-negative
garotte. We also consider forecast combination in the form of Bayesian model
averaging.

The following sub-sections provide summary details on implementation of the
above methods in contexts where in a first step we estimate factors using
principal components analysis, while in a second step we select factor
weights using shrinkage. Approaches in which we first directly implement
shrinkage to select an \textquotedblleft informative\textquotedblright\ set
of variables for: (i) direct use in prediction model construction; or (ii)
use in a second step where factors are constructed for subsequent use in
prediction model construction, follow immediately. Note that all variables
are assumed to be standardized in the sequel. Algorithms for the methods
outlined below are given in key papers that we cite as well as discussed in
detail in an earlier working paper version of the current paper (see \cite%
{KS11WP}).

\subsection{Statistical Learning (Bagging and Boosting)}

\subsubsection{Bagging}

Bagging, which is short for \textquotedblleft bootstrap
aggregation\textquotedblright , was introduced by \cite{Bre96ML}. Bagging
involves first drawing bootstrap samples from in-sample \textquotedblleft
training\textquotedblright\ data, and then constructing predictions, which
are later combined. If a bootstrap sample based predictor is defined as $%
\hat{Y}_{b}^{\ast }=\hat{\beta}_{b}^{\ast }X_{b}^{\ast },$ where $b=1,...,B$
denotes the $b$-th bootstrap sample drawn from the original dataset, then
the bagging predictor is $\hat{Y}^{Bagging}=\frac{1}{B}\underset{b=1}{%
\overset{B}{\Sigma }}\hat{Y}_{b}^{\ast }$. In this paper, we follow \cite%
{BuYu02AnSta} and \cite{SW05WP} who note that, asymptotically, the bagging
estimator can be represented in shrinkage form. Namely: 
\begin{equation}
\hat{Y}_{t+h}^{Bagging}=W_{t}\hat{\beta}_{W}+\underset{j=1}{\overset{r}{%
\Sigma }}\psi \left( t_{j}\right) \hat{\beta}_{Fj}\hat{F}_{t,j}  \label{eq98}
\end{equation}%
where $\hat{Y}_{t+h}^{Bagging}$ is the forecast of $Y_{t+h}$ made using data
through time $t,$ and $\hat{\beta}_{W}$ is the least squares (LS)$\ $%
estimator from a regression of $Y_{t+h}$ on $W_{t},$ where $W_{t}$ is a
vector of lags of $Y_{t}$ as in (\ref{eq41a}) including a vector of ones, $%
\hat{\beta}_{Fj}$ is a LS estimator from a regression of residuals, $%
Z_{t}=Y_{t+h}-W_{t}\hat{\beta}_{W}$ on $\hat{F}_{T-h,j}$, and $t_{j}$ is the
t-statistic associated with\textrm{\ }$\hat{\beta}_{Fj}$, defined as $\sqrt{T%
}\hat{\beta}_{Fj}/s_{e},$ where $s_{e}$, is a Newey-West standard error, and 
$\psi $ is a function specific to the forecasting method. In the current
context we set:%
\begin{equation}
\psi \left( t\right) =1-\Phi \left( t+c\right) +\Phi \left( t-c\right)
+t^{-1}[\phi \left( t-c\right) -\phi \left( t+c\right) ],  \label{eq53}
\end{equation}%
where $c$ is the pretest critical value, $\phi $ is the standard normal
density and $\Phi $ is the standard normal CDF. In this paper, we follow 
\cite{SW05WP}, and set the pretest critical value for bagging, $c$ to be $%
1.96$.

\subsubsection{Boosting}

Boosting (see e.g., \cite{FrSch97JCSS}) is a procedure that builds on a
user-determined set of functions (e.g., least square estimators), often
called \textquotedblleft learners\textquotedblright ,\ and uses the set
repeatedly on filtered data which are typically outputs from previous
iterations of the learning algorithm. The output of a boosting algorithm
generally takes the form:%
\begin{equation*}
\hat{Y}^{M}=\underset{m=1}{\overset{M}{\Sigma }}\kappa _{m}f\left( X;\beta
_{m}\right) ,
\end{equation*}%
where the $\kappa _{m}$ can be interpreted as weights, and $f\left( X;\beta
_{m}\right) $ are functions of the panel dataset, $X.$ \cite{Friedman01AnSta}
introduce \textquotedblleft $L_{2}$ Boosting\textquotedblright , which takes
the simple approach of refitting \textquotedblleft base
learners\textquotedblright\ to residuals from previous iterations.\footnote{%
Other extensions of the boosting problem discussed by \cite{Friedman01AnSta}
are given in \cite{RMR99boostingmethodology} and \cite{SS06AdaBoo}.} \cite%
{BUYU03JASA} develop a boosting algorithm fitting \textquotedblleft
learners\textquotedblright\ using one predictor at a time, in contexts where
large numbers of predictors are available, and data are $iid$. \cite{BN08JAE}%
\ modify this algorithm to handle time-series. We use their
\textquotedblleft Component-Wise $L_{2}$Boosting\textquotedblright\
algorithm in the sequel, with least squares \textquotedblleft
learners\textquotedblright .

As an example, consider the case where boosting is done on the original $%
W_{t}$ data as well as factors, $\hat{F}_{t},$ constructed using principal
components analysis, and denote the scalar output of the boosting algorithm
as $\hat{\mu}^{M}\left( \hat{F}_{t}\right) .$ Then, predictions are
constructed using the following model:%
\begin{equation}
\hat{Y}_{t+h}^{Boosting}=W_{t}\hat{\beta}_{W}+\hat{\mu}^{M}\left( \hat{F}%
_{t}\right) .  \label{eq104}
\end{equation}%
Evidently, when shrinkage is done directly on $X_{t}$, then $\hat{F}_{t}$ in
the above expression is suitably replaced with $X_{t}$.

\subsection{Penalized Regression (Least Angle Regression, Elastic Net, and
Non-Negative Garotte)}

Ridge regression, which was introduced by \cite{HE70T}, is likely the most
well known penalized regression method (see e.g., \cite{KS11WP}) for further
discussion). Ridge regression is characterized by an $L_{2}$ penalty
function, while several recent advances in penalized regression have
centered to some extent on $L_{1}$ penalty functions. For example, there has
been much recent research examining the properties of $L_{1}$ penalty
functions, using the so-called \textquotedblleft lasso\textquotedblright\
(least absolute shrinkage and selection operator) regression method, as
introduced by \cite{Tib96lasso}, and various hybrids and generalizations
thereof. Examples of these include least angle regression, the elastic net,
and the non-negative garotte, all of which are implemented in our prediction
experiments.

\subsubsection{Least Angle Regression (LAR)}

Least Angle Regression\ (LAR), as introduced by \cite{EHJT04AnSta}, is based
on a model-selection approach known as forward \ stage-wise regression,
which has been extensively used to examine cross-sectional data (for further
details, see \cite{EHJT04AnSta} and \cite{BN08JEMC}). \cite{GC08} extend 
\cite{BN08JEMC} to time series forecasting with many predictors. We
implement the algorithm of \cite{GC08}\ when constructing the LAR estimator.

Like many other stage-wise regression approaches, we start with $\hat{\mu}%
^{0}$ $=\bar{Y},$ the mean of the target variable, use the residuals after
fitting $W_{t}$ to the target variable, and construct a first estimate, $%
\hat{\mu}=X_{t}\hat{\beta},$ in stepwise fashion, using standardized data,
and using $M$ iterations, say. Possible explanatory variables are
incrementally examined, and they are added to the estimator function, $%
\widehat{\mu },$ according to their explanatory power. Following the same
notation as that used above, in the case where shrinkage is done solely on
common factors, the objective is to construct predictions, 
\begin{equation*}
\hat{Y}_{t+h}^{LAR}=W_{t}\hat{\beta}_{W}+\hat{\mu}^{M}(\hat{F}_{t}).
\end{equation*}

\subsubsection{Elastic Net (EN)}

\cite{ZH05JRSS} point out that the lasso has undesirable properties when $T$
is greater than $N$ or when there is a group of variables amongst which all
pairwise correlations are very high. They develop a new regularization
method that they claim remedies the above problems. The so-called elastic
net (EN) simultaneously carries out automatic variable selection and
continuous shrinkage. Its name comes from the notion that it is similar in
structure to a stretchable fishing net that retains \textquotedblleft all
the big fish\textquotedblright (see \cite{ZH05JRSS}). In this paper, we use
the algorithm of \cite{BN08JEMC}, who modify the naive EN to use time series
rather than cross sectional data. To fix ideas, assume again that we are
interested in $X$ and $Y,$ and that variables are standardized. For any
fixed non-negative $\eta _{1}$ and $\eta _{2}$, the elastic net criterion is
defined as:%
\begin{equation}
L\left( \eta _{1},\eta _{2},\beta \right) =\left\vert Y-X\beta \right\vert
^{2}+\eta _{2}\left\vert \beta \right\vert ^{2}+\eta _{1}\left\vert \beta
\right\vert _{1},  \label{eq86}
\end{equation}%
where $\left\vert \beta \right\vert ^{2}=\overset{N}{\underset{j}{\tsum }}%
(\beta _{j})^{2}$ and $\left\vert \beta \right\vert _{1}=\overset{N}{%
\underset{j}{\tsum }}\left\vert \beta _{j}\right\vert $.{\LARGE \ }The
solution to this problem is the so-called naive elastic net, given as:%
\begin{equation}
\hat{\beta}^{NEN}=\frac{\left( \left\vert \hat{\beta}^{LS}\right\vert -\eta
_{1}/2\right) _{pos}}{1+\eta _{2}}sign\left\{ \hat{\beta}^{LS}\right\} .
\label{eq91}
\end{equation}%
where $\hat{\beta}^{LS}$ is the least square estimator of $\beta $ and $%
sign\left( \cdot \right) $ equals $\pm 1.$ Here, $``pos"$ denotes the
positive part of the term in parentheses. \cite{ZH05JRSS}, in the context of
above naive elastic net, point out that there is double shrinkage in this
criterion, which does not help to reduce the variance and may lead to
additional bias; and so that they propose a version of the elastic net in
which this double shrinkage is corrected. In this context, the elastic net
estimator, $\hat{\beta}^{EN},$ is defined as:%
\begin{equation}
\hat{\beta}^{EN}=\left( 1+\eta _{2}\right) \hat{\beta}^{NEN},  \label{eq96}
\end{equation}%
where $\eta _{2}$ is a constant, usually \textquotedblleft
optimized\textquotedblright\ via cross validation methods. \cite{ZH05JRSS}
propose an algorithm called \textquotedblleft LAR-EN\textquotedblright\ to
estimate $\hat{\beta}^{EN}$.\footnote{%
We use their algorithm, which is discussed in more detail in Kim and Swanson
(2011).} In the current context, $\hat{\beta}^{EN}$ is either the
coefficient vector associated with the $\hat{F}_{t}$ in a forecasting model
of the variety given in (\ref{eq41a}), assuming that $\psi \left( \cdot
\right) =1$, or is a coefficient vector constructed by operating directly on
the panel dataset, $X.$

\subsubsection{NON-NEGATIVE GAROTTE (NNG)}

The non-negative garotte\ (NNG), was introduced by \cite{Bre95TMC}. This
method is a scaled version of the least square estimator with shrinkage
factors, and is closely related to the EN and LAR. We follow \cite{YL07JRSS}%
, who develop an efficient garotte algorithm and prove consistency in
variable selection. As far as we know, this method has previously not been
used in the time series econometrics literature. As usual, we begin by
considering standardized $X$\textbf{\ }and\textbf{\ }$Y$. Assume that the
following shrinkage factor is given: $q\left( \zeta \right) =\left(
q_{1}\left( \zeta \right) ,q_{2}\left( \zeta \right) ,...,q_{N}\left( \zeta
\right) \right) ^{\prime },$ where $\zeta >0$ is a tuning parameter$.$ The
objective is to choose the shrinkage factor in order to minimize: 
\begin{equation}
\frac{1}{2}\left\Vert Y-Gq\right\Vert ^{2}+T\zeta \underset{j=1}{\overset{N}{%
\tsum }}q_{j},\text{ \ \ \ \ \ \ subject to }q_{j}>0,\text{ }j=1,..,N,
\label{eq10}
\end{equation}%
where $G$ $=(G_{1},..,G_{N})^{\prime }$, $G_{j}=X_{j}\widehat{\beta }%
_{j}^{LS},$ and $\widehat{\beta }^{LS}$ is the least squares estimator. The
NNG\ estimator of the regression coefficient vector is defined as $\hat{\beta%
}_{j}^{NNG}=q_{j}\left( \zeta \right) \hat{\beta}_{j}^{LS},$ and the
estimate of $Y$ is defined as $\widehat{\mu }=X\hat{\beta}^{NNG}\left( \zeta
\right) $, so that predictions can be formed in a manner that is analogous
to that discussed in the previous subsections. Assuming, for example, that $%
X^{\prime }X=I$, the minimizer of expression (\ref{eq10}) has the following
explicit form: $q_{j}\left( \zeta \right) =\left( 1-\frac{\zeta }{(\hat{\beta%
}_{j}^{LS})^{2}}\right) _{+},$ \ \ for $j=1,...,N.$ This ensures that the
shrinking factor may be identically zero for redundant predictors. The
disadvantage of the NNG is its dependence on the ordinary least squares
estimator, which can be especially problematic in small samples. However, 
\cite{Zou06JASA} shows that the NNG with ordinary least squares is
consistent, if $N\ $is fixed, as $T\rightarrow \infty .$ Our approach is to
start the algorithm with the least squares estimator, as in \cite{Yuan07JMLR}%
.

\subsection{Bayesian Model Averaging}

In recent years, Bayesian model averaging (BMA) has been applied to many
forecasting problems, and has frequently been shown to yield improved
predictive accuracy, relative to approaches based on the use of individual
models. For this reason, we include BMA in our prediction experiments; and
we view it as one of our benchmark modeling approaches. For further
discussion of BMA in a forecasting context, see \cite{KP04EJ}, \cite%
{W08JE,W09JF}, and \cite{KS11WP}.

In addition, for a concise discussion of general BMA methodology, see \cite%
{HMRV99StatSci} and \cite{CGM01}. The basic idea of BMA\ starts with
supposing interest focuses on $Q$\ possible models, denoted by $%
M_{1},...,M_{Q}$, say$.$ In forecasting contexts, BMA involves averaging
target predictions, $Y_{t+h}$ from the candidate models, with weights
appropriately chosen. In a very real sense, thus, it resembles bagging. The
key difference is that BMA puts little weight on implausible models, and is
different from other varieties of shrinkage discussed above that operate
directly on regressors. The algorithm that we use for implementation of BMA
follows closely \cite{CGM01}, \cite{FLS01JECM}, and \cite{KP04EJ}. For
complete details, see \cite{KS11WP}.

\section{Data}

Following a long tradition in the diffusion index literature, we examine
monthly data observations on 144 U.S.\ macroeconomic time series for the
period 1960:01 - 2009:5 ($N=144,T=593$)\footnote{%
This is an updated and expanded version of the Stock and Watson (2005a,b)
dataset.}. Forecasts are constructed for eleven variables, including: the
unemployment rate, personal income less transfer payments, the 10 year
Treasury-bond yield, the consumer price index, the producer price index,
non-farm payroll employment, housing starts, industrial production, M2, the
S\&P 500 index, and gross domestic product.\footnote{%
Note that gross domestic product is reported quaterly. We interpolate these
data to a monthly frequency following \cite{CL71RES}.} These variables
constitute 11 of the 14 variables (for which long data samples are
available) that the Federal Reserve takes into account, when formulating
monetary policy, as noted in \cite{ArSwan10} and on the Federal Reserve Bank
of New York's website. Table 1 lists the eleven variables. The third row of
the table gives the transformation of the variable used in order to induce
stationarity. In general, logarithms were taken for all nonnegative series
that were not already in rates (see \cite{SW02JASA,SW05WP} for complete
details). Note that the full list of predictor variables is provided in the
appendix to an earlier working paper version of the current paper, and is
available upon request from the authors.

\section{Forecasting Methodology}

Using the transformed dataset, factors are estimated as discussed in Section
2. After estimating factors, the alternative methods outlined in the
previous sections are used to form forecasting models and predictions.
Summarizing, we consider four \textquotedblleft specification
types\textquotedblright :

\textit{Specification Type 1 without lags (SP1)}\textbf{:} Principal
components are first constructed, and then prediction models are formed
using the shrinkage methods of Section 3 to select functions of and weights
for the factors to be used in our prediction models of the type given in (%
\ref{eq41a}).

\textit{Specification Type 1 with lags (SP1L)}\textbf{:} This is the same as
SP1, except that lags are included in the datasets used to construct
principal components.

\textit{Specification Type 2 (SP2):} Principal component models of the type
given in (\ref{eq41a}) are constructed using subsets of variables from the
largescale dataset that are first selected via application of the shrinkage
methods of Section 3. This is different from the above approach of
estimating factors using all of the variables in $X$.

\textit{Specification Type 3 (SP3):} Prediction models are constructed using
only the shrinkage methods discussed in Section 3, without use of factor
analysis at any stage.

In our prediction experiments, pseudo out-of-sample forecasts are calculated
for prediction horizons $h=1,3,$ and $12$. All estimation, including lag
selection, shrinkage, and factor construction is done anew, at each point in
time, prior to the construction of each new prediction, using both recursive
and rolling estimation windows. Note that at each estimation period, the
number of factors included will be different, following the testing approach
discussed in Section 2. Note also that lags of the target predictor
variables are also included in the set of explanatory variables, in all
cases. Selection of the number of lags to include is done using the SIC.
Various out-of-sample periods are examined, including periods from
1992:10-2009:5 (i.e., $P=200),$ 1984:6-2009:5 (i.e., $P=300)$, and
1972:2-2009:5 $(i.e.,P=400)$, where $P$ denotes the length of the
out-of-sample period, and in subsequent discussion, $R$ is the in-sample
estimation period. In-sample estimation periods are adjusted so that $P$
remains fixed, regardless of forecast horizon. In our rolling estimation
scheme, the in-sample estimation period used to calibrate our prediction
models is fixed at length 12 years. The recursive estimation scheme begins
with the same in-sample period of 12 years, and a new observation is added
to this sample prior to the re-estimation and construction of each new
forecast, as we iterate through the ex-ante prediction period. Note that the
actual observations being predicted as well as the number of predictions in
our ex-ante prediction period remains fixed, regardless of forecast horizon,
in order to facilitate comparison across forecast horizons as well as models.

Forecast performance is evaluated using mean square forecast error (MSFE),
defined as:%
\begin{equation}
MSFE_{i,h}=\overset{T-h+1}{\underset{t=R-h+2}{\tsum }}\left( Y_{t+h}-\hat{Y}%
_{i,t+h}\right) ^{2},  \label{eq68}
\end{equation}%
where $\widehat{Y}_{i,t+h}$ is the forecast at horizon $h$ for the $i-$th
model. Forecast accuracy is evaluated using point MSFEs as well as the
so-called DM predictive accuracy test of \cite{DM95}, which is implemented
using quadratic loss, and which has a null hypothesis that the two models
being compared have equal predictive accuracy. DM\ test statistics have
asymptotic $N(0,1)$ limiting distributions, under the assumption that
parameter estimation error vanishes as $T,P,R\rightarrow \infty $, and
assuming that each pair of models being compared is nonnested$.$ Namely, the
null hypothesis of the test is $H_{0}:E\left[ l\left( \varepsilon
_{t+h|t}^{1}\right) \right] -E\left[ l\left( \varepsilon _{t+h|t}^{2}\right) %
\right] =0,$ where $\varepsilon _{t+h|t}^{i}$ is $i-$th model's prediction
error and $l\left( \cdot \right) $ is the quadratic loss function. The
actual statistic in this case is constructed as: $DM=P^{-1}%
\sum_{i=1}^{P}d_{t}/\hat{\sigma}_{\overline{d}},$ where $d_{t}=\left( 
\widehat{\varepsilon _{t+h|t}^{1}}\right) ^{2}-\left( \widehat{\varepsilon
_{t+h|t}^{2}}\right) ^{2},$ $\overline{d}$ is the mean of $d_{t}$, $\hat{%
\sigma}_{\overline{d}}$ is a heteroskedasticity and autocorrelation robust
estimator of the standard deviation of $\overline{d}$, and $\widehat{%
\varepsilon _{t+h|t}^{1}}$and $\widehat{\varepsilon _{t+h|t}^{2}}$ are
estimates of the true prediction errors $\varepsilon _{t+h|t}^{1}$and $%
\varepsilon _{t+h|t}^{2}$. Thus, if the statistic is negative and
significantly different from zero, then Model 1 is preferred over Model 2.
Related reality check tests due to White (2000) are also constructed (see
below for further discussion). \ \ \ \ \ \ \ \ \ \ \ \ \ \ \ \ \ \ \ \ \ \ \
\ \ \ \ \ \ \ \ \ \ \ \ \ \ \ \ \ \ \ \ \ \ \ \ \ \ \ \ \ \ \ \ \ \ \ \ \ \
\ \ \ \ \ \ \ \ \ \ \ \ \ \ \ \ \ \ \ \ \ \ \ \ \ \ \ \ \ \ \ \ \ \ \ \ \ \
\ \ \ \ \ \ \ \ \ \ \ \ \ \ \ \ \ \ \ \ \ \ \ \ \ \ \ \ \ \ \ \ \ \ \ \ \ \
\ \ 

In addition to the various forecast model specification approaches discussed
above, we form predictions using the following \textquotedblleft
benchmark\textquotedblright\ models, all of which are estimated using least
squares.

\textbf{Univariate Autoregression: }Forecasts from a univariate AR(p) model
are computed as $\hat{Y}_{t+h}^{AR}=\hat{\alpha}+\hat{\phi}\left( L\right)
Y_{t},$ with lags , $p$, selected using the SIC.

\textbf{Multivariate Autoregression: }Forecasts from an ARX(p) model are
computed as $Y_{t+h}^{ARX}=\hat{\alpha}+\hat{\beta}Z_{t}+\hat{\phi}\left(
L\right) Y_{t},$ where $Z_{t}$ is a set of lagged predictor variables
selected using the SIC. Dependent variable lags are also selected using the
SIC. Selection of the exogenous predictors includes choosing up to six
variables prior to the construction of each new prediction model.

\textbf{Principal Components Regression: }Forecasts from principal
components regression are computed as $\hat{Y}_{t+h}^{PCR}=\hat{\alpha}+\hat{%
\gamma}\hat{F}_{t},$\textbf{\ }where $\hat{F}_{t}$ is estimated via
principal components using $\left\{ X_{t}\right\} _{t=1}^{T},$ as in
equation (\ref{eq41a}).

\textbf{Factor Augmented Autoregression}: Based on equation (\ref{eq41a}),
forecasts are computed as $Y_{t+h}^{h}=\hat{\alpha}+\hat{\beta}_{F}\hat{F}%
_{t}+\hat{\beta}_{W}\left( L\right) Y_{t}.$ This model combines an AR(p)
model, with lags selected using the SIC, with the above principal component
regression model.

\textbf{Combined Bivariate ADL\ Model}: As in \cite{SW05WP},\ we implement a
combined bivariate autoregressive distributed lag (ADL) model. Forecasts are
constructed by combining individual forecasts computed from bivariate ADL
models. The $i$-th ADL model includes $p_{i,x}$ lags of $X_{i,t},$ $\ $and $%
p_{i,y}$ lags of $Y_{t},$ and has the form $\hat{Y}_{t+h}^{ADL}=\hat{\alpha}+%
\hat{\beta}_{i}\left( L\right) X_{i,t}+\hat{\phi}_{i}\left( L\right) Y_{t}.$
The combined forecast is $\hat{Y}_{T+h|T}^{Comb,h}=\underset{t=1}{\overset{n}%
{\Sigma }}w_{i}\hat{Y}_{T+h|T}^{ADL,h}$. Here, we set $w_{i}=1/N$. There are
a number of studies that compare the performance of combining methods in
controlled experiments, including: \cite{C89IJF}, \cite{DL96NBER}, \cite%
{NH02FC}, and \cite{Timm05CEPR}; and in the literature on factor models, 
\cite{SW04JF,SW05WP,SW2006HEF}, and the references cited therein. In this
literature, combination methods typically outperform individual forecasts.
This stylized fact is sometimes called the \textquotedblleft forecast
combining puzzle.\textquotedblright

\textbf{Mean Forecast Combination: }To further examine the issue of forecast
combination, we form forecasts as the simple average of the forecasting
models summarized in Table 2.

\section{Empirical Results}

In this section, we discuss the results of our prediction experiments. For
the case where models are estimated using recursive data windows, detailed
results are gathered in Tables 3-6. Analogous results based on rolling
estimation are omitted for the sake of brevity, although they are available
upon request from the authors. Summary results are contained in Tables 7-11.

Tables \ref{tb3}-\ref{tb6} report MSFEs and the results of DM predictive
accuracy tests for all alternative forecasting models, using Specification
Type 1 without lags (Table \ref{tb3}), Specification Type 1 with lags (Table %
\ref{tb4}), Specification Type 2 (Table \ref{tb5}), and Specification Type 3
(Table \ref{tb6}). Results in these tables are for the out-of-sample period
1984:6-2009:5 ($P=300$). Other out-of-sample results ($P=200$ and $400$) are
available upon request from the authors. Panels A-C in these tables contain
results for $h=1,3$ and $12$ month ahead predictions, respectively. In each
panel, the first row of entries reports MSFEs for benchmark AR(SIC)\ models,
and all other rows report MSFEs relative to those of the AR(SIC) model.
Thus, entries greater than unity imply point MSFEs greater than those of our
AR(SIC) model. Entries in bold denote MSFE \textquotedblleft
best\textquotedblright\ models for a given variable, forecast horizon, and
specification type. For example, in Panel C of Table 3, the MSFE
\textquotedblleft best\textquotedblright\ model for unemployment (UR), when $%
h=12$, is principal component regression (PCR), with a relative MSFE of
0.974. Results from DM predictive accuracy tests, for which the null
hypothesis is that of equal predictive accuracy between the benchmark model
(defined to be the AR(SIC) model), and the model listed in the first column
of the tables, are reported using a single star (denoting rejection at the
10\% level), and a double star (denoting rejection at the 5\% level).

There are no models that uniformly yield the lowest MSFEs, across both
forecast horizon and variable. However, various models perform quite well,
including in particular our benchmark FAAR and PCR models. This supports the
oft reported result that models that incorporate common factors offer a
convenient way to filter the information contained in large-scale economic
datasets. Various other results are also apparent, upon inspection of
tables. For example, turning to Table 3, which summarizes results for
Specification Type 1 without lags (SP1), notice that in Panel A, AR(SIC) and
ARX(SIC) models \textquotedblleft win\textquotedblright\ for only 2 of 11
variables (SNP and GDP). A similar results holds for $h=3$ and $h=12$ (see
Panels B and C). Notice also that model averaging (BMA and \textquotedblleft
Mean\textquotedblright ) wins only 27\% of the time, across all
variable/horizon permutations.

In Table \ref{tb4}, we report the results for Specification Type 1 with lags
(SP1L). In Panel A ($h=1)$, benchmark models including AR(SIC), ARX(SIC),
and CADL\ yields lowest point MSFEs for 7 of 11 variables. This suggests
that including lags (of factors) does not appreciably improve model forecast
performance, given that the structure of these benchmark models do not
change when lags are added to our factor specification methods, and that
they \textquotedblleft win\textquotedblright\ less frequently under SP1 than
under SP1L.\footnote{%
Recall that the addition of lags when moving from SP1 to SP1L only involves
including lags of explanatory variables used in diffusion index
construction. As AR, ARX, and CADL models do not include diffusion indexes,
they are thus unchanged when constructed under either SP1 or SP1L} Indeed,
comparison of the results in Tables \ref{tb3} and \ref{tb4} suggests that
there is little advantage to using lags of factors when constructing
predictions in our context. Instead, it appears that the more important
determinant of model performance is the type of combination factor/shrinkage
type model (elsewhere called our \textquotedblleft hybrid\textquotedblright\
model) employed when constructing forecasts. Note however, that there are
(possibly) $some$ lagged variables used in the construction of diffusion
indexes under $all$ specification types, including SP1 and SP1L. In
particular, all datasets used to construct factors may contain lagged values
of the $target$ variable. An exception to the above finding concerning the
usefulness of SP1L as a specification method is as follows. Under SP1L, all
models (except CADL)\ yield lower MSFEs than AR(SIC), when used to forecast
GDP, for $h=1$. However, under SP1, few models yield lower MSFEs than
AR(SIC). This result may be due to the interpolation approach used to
construct our monthly GDP figures.

For Specification Type 2 (SP2), results are reported in Table 5. In this
table, FAAR\ and PCR\ models are omitted since all diffusion indexes used in
the prediction models involve using variables first selected via shrinkage,
and our benchmark \textquotedblleft pure\textquotedblright\ factor models
use no shrinkage. In Specification Type 3 (SP3), FAAR\ and PCR\ are also
omitted (since no SP3 specifications use diffusion indices). In Table \ref%
{tb5}, our \textquotedblleft hybrid\textquotedblright\ SP2 type models (that
use bagging, boosting, ridge, LAR, EN, and NNG) are MSFE \textquotedblleft
best\textquotedblright\ for 3, 3, and 2 out of 11 variables, for $h=1,3$ and 
$\ 12,$ respectively. On the other hand, linear benchmark models (AR, ARX\
and CADL) yield the lowest MSFEs for\ 6, 5, 4 out of 11 variables, for $%
h=1,3 $ and $\ 12,$ respectively. Thus, SP2 type factor models
\textquotedblleft win\textquotedblright\ roughly 1/3 of the time. In Table
6, our \textquotedblleft pure shrinkage\textquotedblright\ (SP3) type models
that use bagging, boosting, ridge, LAR, EN, or NNG are MSFE\
\textquotedblleft best\textquotedblright\ for 5, 5, 4 out of 11 target
variables, for $h=1,3$ and $12,$ respectively. Of note is that bagging
performs very poorly in SP3, so that \textquotedblleft
Mean\textquotedblright\ does not yield reasonable MSFEs in most cases. We
also constructed \textquotedblleft Mean\textquotedblright\ predictions
without bagging, and although results improved, our overall model rankings
did not change.

Entries in Tables 3-6 only summarize a subset of our experiments results,
and are hence not as informative as we would like them to be. For this
reason, we also provide a series of \textquotedblleft
summary\textquotedblright\ tables, denoted Tables 7-11, from which a number
of further interesting findings can be drawn.

Table \ref{tb7} summarizes MSFE \textquotedblleft best\textquotedblright\
forecast models $by$ specification type (Panels A-D), as well as $across$
all specification types (Panel E). Entries in the Panels A-D correspond to
entries in bold in Tables 3-6, respectively. For example, when forecasting
unemployment (UR), FAAR\ under Specification Type 1 without lags yields the
lowest MSFE, for $h=1$. When comparing results for individual specification
types, note that benchmark AR and ARX models sometimes \textquotedblleft
win\textquotedblright . However, when the MSFE \textquotedblleft
best\textquotedblright\ model across all four specifications is selected,
these benchmark models never win (see Panel E). This is interesting, as it
suggests that the oft noted dominance of simple AR type models in
forecasting experiments may, to some extent, be due to a lack of suitable
alternative \textquotedblleft model / specification
method\textquotedblright\ combinations against which to compare the AR
model. Turning again to Panel E, note that, for $h=1,$ a model that
incorporates either shrinkage (including bagging, boosting, ridge, LAR, EN,
NNG, and \textquotedblleft Mean\textquotedblright )\footnote{%
BMA and simple arithmetic model averaging (collectively called
\textquotedblleft Mean\textquotedblright ) are included in this subset of
models because these model averaging methods involve the combination of
various shrinkage type models.}, or is a \textquotedblleft
hybrid\textquotedblright\ model, \textquotedblleft wins\textquotedblright\
for 6 of 11 variables (models not in this subset include FAAR, PCR, and
CADL). Notice, also, that this number increases from 6 out of 11 to 10 out
of 11 ($h=3)$ and 9 out of 11 ($h=12),$ when longer forecast horizons are
used. This is strong evidence in favor of using shrinkage and
\textquotedblleft hybrid\textquotedblright\ modeling approaches. These
summary finding remain qualitatively the same for out-of-sample periods not
reported here (i.e., $P=200$ and $400$).

Table \ref{tb8} (Panel A), summarizes the number of MSFE \textquotedblleft
wins\textquotedblright\ for each model, across $all$ specification types,
for two cases. In the first case (\textquotedblleft Recursive Estimation
Window\textquotedblright ), results are reported only for recursively
estimated models. In the second case (\textquotedblleft Recursive and
Rolling Estimation Windows\textquotedblright ), results are summarized $%
across$ both \textquotedblleft windowing\textquotedblright\ methods. In
Panel B, the number of \textquotedblleft wins\textquotedblright\ are
reported for the two windowing approaches (\textquotedblleft Winners by
Estimation Window\textquotedblright ) and by specification type
(\textquotedblleft Winners by Specification Type\textquotedblright ). From
Panel A, we see that our findings based on results reported in Table 7,
which are based only on recursive estimation, remain essentially the same
when rolling estimation is also considered (i.e., compare the first 4
columns of entries in the table with the second 4 columns of entries). This
is not surprising, given that inspection of Panel B of the table indicates
that recursive estimation often \textquotedblleft
dominates\textquotedblright\ rolling estimation. However, it is noteworthy
that recursive estimation do not $always$ lead to MSFE \textquotedblleft
best\textquotedblright\ models, particularly for $h=12.$

In Table 8, notice also that SP1 clearly dominates for $h=1$, while SP2
dominates for $h=12.$ This suggests that longer horizon forecasts, which are
generally know to be more difficult to accurately construct, are best
tackled using more parsimonious model specification methods (i.e.,
pre-select the variables to use in diffusion index construction via
shrinkage methods, as is done under SP2). Also, note that as the forecast
horizon increases, model averaging methods \textquotedblleft
win\textquotedblright\ more frequently; which is, again, not surprising (see
Panel A).

In order to shed light on the importance of sub-samples in our empirical
analysis, we report summary results for a variety of NBER business-cycle
related sample periods in Table 9, for $h=1$. Although conclusions based
upon inspection of this table are largely in accord with those reported
above, two additional noteworthy findings are worth stressing. First, in
Panel A of the table, note that when MSFE \textquotedblleft
best\textquotedblright\ models are tabulated by specification type, our
model averaging methods perform quite well, particularly for SP2 and SP3.
This finding can be confirmed by carefully examining the results reported in
Tables 3-6 (i.e., compare the MSFE \textquotedblleft best\textquotedblright\
models, denoted by bold entries). However, notice that when results are
summarized across $all$ specification types (see Panel B of the table), then
the model averaging methods yield MSFE \textquotedblleft
best\textquotedblright\ predictions in far fewer cases.\ This is because
SP1, where model averaging clearly \textquotedblleft wins\textquotedblright\
the least, is the predominant winner when comparing results across $all$
specification types, as mentioned previously. Moreover, it is clear that our
\textquotedblleft hybrid\textquotedblright\ model building approach whereby
we first construct factors and thereafter use shrinkage methods to estimate
functions of and weights for factors to be used in our prediction models
(i.e., SP1) is the dominant specification type. In summary, when more
complicated specification methods are used, model averaging methods fare
worse and \textquotedblleft hybrid\textquotedblright\ methods fare better.
Still, pure factor type models sometimes perform well, particularly for the
long expansion period from 1982-1990. Second, careful examination of Panel B
of this table indicated that our so-called \textquotedblleft Nonlinear
Factor\textquotedblright\ models (i.e., all shrinkage/factor combination
models - see footnote to Table 9) exhibit business cycle dependent
performance. In particular, the \textquotedblleft Nonlinear
Factor\textquotedblright\ models are MSFE \textquotedblleft
best\textquotedblright\ for 7 of 11 variables in recessionary periods (in
recessions, \textquotedblleft Mean\textquotedblright\ - or model averaging -
\textquotedblleft wins\textquotedblright\ for 0 of 11 variables, while
\textquotedblright Linear Factor\textquotedblright\ models and simple AR
type models each win twice). On the other hand, \textquotedblleft Nonlinear
Factor\textquotedblright\ models \textquotedblleft win\textquotedblright\
for only 4 of 11 variable (or approximately 1/3 of the time) during
expansionary periods (in expansions, \textquotedblleft
Mean\textquotedblright\ models \textquotedblleft win\textquotedblright\ for
5 of 11 variables, while \textquotedblleft Linear Factor\textquotedblright\
models and simple AR type models each \textquotedblleft
win\textquotedblright\ once). Additionally, for all but 1 other sub-sample,
\textquotedblleft Nonlinear Factor\textquotedblright\ models win for 4 or 5
of 11 variables. Indeed, it is only during the 01:11-07:11 period that
\textquotedblleft Nonlinear Factor\textquotedblright\ models do not win more
than 1/3 of the time, suggesting that during extremely stable periods there
is less to be gained by using more sophisticated nonlinear models.

Table \ref{tb10} reports reality check statistics based on \cite{W2000} and 
\cite{CS2007}. In order to facilitate construction of these statistics as
well as simulation of critical values (following CS (2007)), we re-ran all
experiments, fixing the number of lags and the number of factors used in our
recursive and rolling window based forecast model construction. The reality
check tests the null hypothesis that a given benchmark model, say Model $1$
yields equal or better predictive performance than all competitors, say
Models $2,...,n$. The alternative is that at least one models among Models $%
2,...,n$ outperforms the benchmark. Formally, we test: 
\begin{eqnarray*}
H_{0} &:&\underset{i=2,...,n}{\max }E\left( g\left( u_{1,t+1}\right)
-g\left( u_{i,t+1}\right) \right) \leq 0 \\
H_{A} &:&\underset{i=2,...,n}{\max }E\left( g\left( u_{1,t+1}\right)
-g\left( u_{i,t+1}\right) \right) >0,
\end{eqnarray*}%
where $u_{i,t+1}=y_{t+1}-\kappa _{i}\left( Z_{t},\theta _{i}\right) $, $%
\kappa _{i}$ is $i$-th conditional mean function, $\theta _{i}$ are model
parameters, $Z_{t}$ is the set of predictors, and $g$ is a given loss
function, here assumed to be quadratic. Following White (2000), define the
statistic $S_{p}=\max_{i=2,...,n}S_{P}\left( 1,i\right) $ with%
\begin{equation*}
S_{P}\left( 1,i\right) =\frac{1}{\sqrt{P}}\underset{t=R}{\overset{T-1}{\sum }%
}\left( g\left( \hat{u}_{1,t+1}\right) -g\left( \hat{u}_{i,t+1}\right)
\right) ,i=2,...,n
\end{equation*}%
where $\hat{u}_{i,t+1}=y_{t+1}-\kappa _{i}\left( Z_{t},\hat{\theta}%
_{i,t}\right) $ and $\kappa _{i}\left( Z_{t},\hat{\theta}_{i,t}\right) $ is
the estimated conditional mean under model $i$. See CS (2007) for technical
details about computing critical values. We executed all tests using three
different block lengths ($l=2,5,$ and $10),$ although we report results only
for $l=5,$ since our findings were not affected by block length. Double and
single starred entries denote rejection of the null hypothesis at 5\% and
10\% significance levels, respectively. We consider various test scenarios,
including: (I) set the AR(SIC) model as the benchmark, and compare this
model against a variety of alternative models, including, for example, all
models under SP1, SP2, or SP3; (II) set our factor augmented autoregressive
model (FAAR) as the benchmark; and (III) set the point MSFE
\textquotedblleft best\textquotedblright\ model from each specification type
as the benchmark (see Table \ref{tb7} for summary of \textquotedblleft
best\textquotedblright\ models).

As evident from inspection of the tabulated results, under scenario (I), the
null is almost always rejected, particularly under SP1, as expected given
our findings that are discussed above. Under scenario (II), the null
hypothesis that the FAAR model is at least as \textquotedblleft
MSFE-accurate\textquotedblright\ as any other model is rejected for
approximately 9 out of 11 variables, at all forecast horizons. This supports
our earlier finding concerning the effectiveness of our \textquotedblleft
hybrid\textquotedblright\ models. Namely, pure factor models alone do not
usually \textquotedblleft win\textquotedblright\ when compared with either
shrinkage and/or \textquotedblleft hybrid\textquotedblright\ models. Indeed,
when FAAR is compared with pure shrinkage models with no diffusion indices
(i.e., SP3) the null is rejected for 6 out of 11 variables when $h=1$,
whereas for $h=12$ the null is rejected for all variables. Thus, pure
shrinkage models outperform pure linear factor models in many cases.
However, it is only when \textquotedblleft hybrid\textquotedblright\ models
are used that pure linear factor models are generally \textquotedblleft
dominated\textquotedblright . In summary, our reality check results are
largely in agreement with our point MSFE results.

As a final measure of the relevance of our findings, we constructed a simple
measure of forecast accuracy \textquotedblleft percentage
differences\textquotedblright\ between various AR, \textquotedblleft
Mean\textquotedblright , and MSFE \textquotedblleft best\textquotedblright\
models discussed above, as reported in Table \ref{tb11}. The
\textquotedblleft percentage differences\textquotedblright\ were calculated
using the following formula:%
\begin{equation*}
\frac{1}{T}\sum_{t}\left\vert \left\vert \frac{\hat{Y}_{1,t}-Y_{t}}{Y_{t}}%
\right\vert -\left\vert \frac{\hat{Y}_{2,t}-Y_{t}}{Y_{t}}\right\vert
\right\vert \times 100\%,
\end{equation*}%
where $Y_{t}$ is the actual value of the target variable at time $t.$
Additionally, $\hat{Y}_{1,t}$ and $\hat{Y}_{2,t}$ are the forecasts from the
two models being compared. In rare occurrences where $Y_{t}$ is zero, it is
replaced by $Y_{t-1}.$ Turning to the tabulated results, in the first row of
entries, \textquotedblleft AR\ vs. SP1Best\textquotedblright\ denotes the
\textquotedblleft percentage difference\textquotedblright\ between AR(SIC)
forecasts and those of the MSFE \textquotedblleft best\textquotedblright\
model under SP1. The value of 11.07\% in the first numerical entry of the
table (upper left), thus indicates a marked improvement in predictive
accuracy when using \textquotedblleft SP1best\textquotedblright\ (i.e., FAAR
- see Table 3) instead of AR(SIC). This \textquotedblleft percentage
difference\textquotedblright\ might be assumed to have significant impact
when predictions are used to calibrate macroeconomic policy decisions, for
example. Although results vary somewhat, it is clear upon inspection of the
tabulated entries that \textquotedblleft percentage
differences\textquotedblright\ often range from around 5\% to 15\%, and for
some variables, such as unemployment, the 10-year Treasury bond rate, CPI,
PPI, and housing starts, most \textquotedblleft percentage
differences\textquotedblright\ lie in this range.

Given the importance of factors in our forecasting experiments, it would
seem worthwhile to examine which variables contribute to the estimated
factors used in our MSFE \textquotedblleft best\textquotedblright\ models, $%
across$ all specification and estimation window types. This is done in
Figure 1, where we report the ten most frequently selected variables for a
variety of MSFE \textquotedblleft best\textquotedblright\ models and
forecast horizons. Keeping in mind that factors are re-estimated at each
point in time, prior to each new prediction being constructed, a 45 degree
line denotes cases for which a particular variable is selected each time a
forecast is constructed. For example, in Panel A, note that the BAA Bond
Yield - Federal Funds Rate spread is always selected as a predictor when
constructing factors to forecast the Producer Price Index for $h=1$. For
SP1, variables are selected based on the $A(j)$\ and $M(j)$ statistics of 
\cite{BN06EMTR} and \cite{ArSwan08}, and for SP2, we directly observe
variables that are selected via shrinkage, and subsequently used to
construct diffusion indexes. The list of selected variables does not vary
much, for SP1. On the other hand, in Panels D\ and F, we see that frequently
selected variables are not selected all the time. For example, in Panel D,
CPI:Apparel is selected over all periods and the 3 month Treasury bill yield
is selected continuously, but only after 1979. Of \ further note is that
interest-rate related variables (i.e. Treasury bills rates, Treasury bond
rates, and spreads with Federal Funds Rate) are frequently selected, across
all specification type, estimation window types, and forecast horizons. This
confirms that in addition to their well established usefulness in linear
models, yields and spreads remain important in nonlinear modelling contexts.

\section{Concluding Remarks}

In this paper we empirically examine approaches to combining factor
modelling methods and robust (shrinkage based) estimation techniques,
including bagging, boosting, ridge regression, least angle regression, the
elastic net, and the non-negative garotte. In particular, we present the
results of a \textquotedblleft horse-race\textquotedblright\ in which
mean-square-forecast-error (MSFE) \textquotedblleft best\textquotedblright\
models are found, using reality check and related predictive accuracy
tests,\ for a variety of forecast horizons, estimation \textquotedblleft
windowing\textquotedblright\ schemes and sample periods. Our empirical
models include \textquotedblleft pure\textquotedblright\ common factor
models, which are estimated using principal components methods, simple
linear models, and sophisticated \textquotedblleft hybrid\textquotedblright\
models that combine latent factor modelling approaches with the various
shrinkage based techniques outlined above. For the majority of the target
variables that we forecast, we find that \textquotedblleft
hybrid\textquotedblright\ forecasting models outperform our benchmark and
pure factor type models. This suggests that diffusion index methodology is
particularly useful when combined with other shrinkage methods (see also, 
\cite{BN08JEMC,BN08JAE}, and \cite{SW05WP}). We also find that model
averaging methods perform surprisingly poorly. Given the rather extensive
empirical evidence suggesting the usefulness of model averaging when
specifying linear prediction models, this is taken as further evidence of
the usefulness of more sophisticated nonlinear modelling approaches.
Finally, we find that model performance is rather sensitive to the state of
the business cycle. Our \textquotedblleft hybrid\textquotedblright\ models,
for example, are most accurate during recessions, while model averaging
methods are never chosen as \textquotedblleft best\textquotedblright\ during
these times.

\newpage 
\bibliographystyle{apalike}
\bibliography{KimHyunHak_Bib_new}
\newpage

%TABLE 1 %%%%%%%%%%%%%%%%%%%%%%%%%%%%%%%%%%%%%%%%%%%%%%%%%%%%%%%%%%%%%%%%%%%%%%%%%%%%%%%%%%%%%%%%%%%%%%%%%%%%%%%%%%%%
\begin{table}[htbp]
\caption{Target Forecasting Variables *}
\label{tb1}
\begin{center}
\begin{tabular}{ccc}
\toprule {\footnotesize Series} & {\footnotesize Abbreviation} & $\ \ \ \ \
\ \ \ \ \ Y_{t+h}${\footnotesize \ \ \ \ \ \ \ \ \ \ \ } \\ \hline
\multicolumn{1}{l}{\footnotesize Unemployment Rate} & {\footnotesize UR} & $%
{\footnotesize Z}_{t+1}{\footnotesize -Z}_{t}$ \\ 
\multicolumn{1}{l}{\footnotesize Personal Income Less transfer payments} & 
{\footnotesize PI} & $\ln\left( Z_{t+1}/Z_{t}\right) $ \\ 
\multicolumn{1}{l}{\footnotesize 10-Year Treasury Bond} & {\footnotesize TB}
& ${\footnotesize Z}_{t+1}{\footnotesize -Z}_{t}$ \\ 
\multicolumn{1}{l}{\footnotesize Consumer Price Index} & {\footnotesize CPI}
& $\ln\left( Z_{t+1}/Z_{t}\right) $ \\ 
\multicolumn{1}{l}{\footnotesize Producer Price Index} & {\footnotesize PPI}
& $\ln\left( Z_{t+1}/Z_{t}\right) $ \\ 
\multicolumn{1}{l}{\footnotesize Nonfarm Payroll Employment} & 
{\footnotesize NPE} & $\ln\left( Z_{t+1}/Z_{t}\right) $ \\ 
\multicolumn{1}{l}{\footnotesize Housing Starts} & {\footnotesize HS} & $%
\ln\left( Z_{t}\right) $ \\ 
\multicolumn{1}{l}{\footnotesize Industrial Production} & {\footnotesize IPX}
& $\ln\left( Z_{t+1}/Z_{t}\right) $ \\ 
\multicolumn{1}{l}{\footnotesize M2} & {\footnotesize M2} & $\ln\left(
Z_{t+1}/Z_{t}\right) $ \\ 
\multicolumn{1}{l}{\footnotesize S\&P 500 Index} & {\footnotesize SNP} & $%
\ln\left( Z_{t+1}/Z_{t}\right) $ \\ 
\multicolumn{1}{l}{\footnotesize Gross Domestic Product} & {\footnotesize GNP%
} & $\ln\left( Z_{t+1}/Z_{t}\right) $ \\ 
\bottomrule &  & 
\end{tabular}
{\footnotesize 
\begin{minipage}{1\columnwidth} \footnotesize * Notes: Data used in model estimation and prediction construction are monthly U.S. figures for the period 1960:1-2009:5. Data transformations used in prediction experiments are given in the last column of the table. See Section 4 for further details. 
\end{minipage}}
\end{center}
\end{table}

\newpage 
% TABLE 2 --------------------------------------------------------------------------------------------------------------------------------
\begin{table}[htbp]
\caption{Models and Methods Used In Real-Time Forecasting Experiments*}
\label{tb2}
\begin{center}
{\footnotesize 
\begin{tabular}{ll}
\toprule Method & \multicolumn{1}{c}{Description} \\ 
\midrule AR(SIC) & Autoregressive model with lags selected by the SIC \\ 
ARX & Autoregressive model with exogenous regressors \\ 
CADL & Combined autoregressive distributed lag model \\ 
FAAR & Factor augmented autoregressive model \\ 
PCR & Principal components regression \\ 
Bagging & Bagging with shrinkage, $c = 1.96$ \\ 
Boosting & Component boosting, $M = 50$ \\ 
BMA1 & Bayesian model averaging with g-prior $= 1/T$ \\ 
BMA2 & Bayesian model averaging with g-prior $= 1/N^2$ \\ 
Ridge & Ridge regression \\ 
LAR & Least angle regression \\ 
EN & Elastic net \\ 
NNG & Non-negative garotte \\ 
Mean & Arithmetic mean \\ 
\bottomrule & 
\end{tabular}
} {\footnotesize 
\begin{minipage}{1\columnwidth} \footnotesize * Notes: This table summarizes the model specification methods used in the construction of prediction models. In addition to directly estimating the above pure linear and factor models (i.e., AR, ARX, CADL, FAAR, PCR), three different combined factor and shrinkage type prediction ``specification methods'' are used in our forecasting experiments, including: Specification Type 1 - Principal components are first constructed, and then prediction models are formed using the above shrinkage methods (including Bagging, Boosting, Ridge, LAR, EN, and NNG) to select functions of and weights for the factors to be used in our prediction models. Specification Type 2 - Principal component models are constructed using subsets of variables from the large-scale dataset that are first selected via application of the above shrinkage methods (ranging from bagging to NNG). This is different from the above approach of estimating factors using all of the variables. 
Specification  Type 3 - Prediction models are constructed using only the above shrinkage methods (including Bagging, Boosting, Ridge, LAR, EN, and NNG), without use of factor analysis at any stage. See Sections 3 and 4 for complete details. 
\end{minipage}}
\end{center}
\end{table}

%TABLE 3 Specification Type 1 without lags  %%%%%%%%%%%%%%%%%%%%%%%%%%%%%%%%%%%%%%%%%%%%%%%%%%%%%%%%%%%%%%%%%%%%%%%%%%%%%%%%%%%%%%%%%%%%%%
\begin{table}[htbp]
\caption{Relative Mean Square Forecast Errors of Specification Type 1
Without Lags*}
\label{tb3}{\scriptsize 
\begin{tabular}{llllllllllll}
\multicolumn{12}{c}{} \\ 
\multicolumn{12}{c}{{\footnotesize {Panel A: Recursive, $h=1$}}} \\ 
\multicolumn{12}{c}{} \\ 
\toprule Method & UR & PI & TB & CPI & PPI & NPE & HS & IPX & M2 & SNP & GDP
\\ 
\midrule AR(SIC) & 6.257 & 0.007 & 18.650 & 0.002 & 0.010 & 0.000 & 1.316 & 
0.010 & 0.003 & \textbf{0.400 } & 0.006 \\ 
ARX(SIC) & 1.011 & 0.941 & 1.136 & 0.966 & 0.992 & 1.347** & 1.088 & 0.913 & 
1.181 & 1.135 & \textbf{0.836 } \\ 
CADL & 0.976** & 1.031 & 0.991 & 1.026 & 1.018 & \textbf{0.920* } & 0.988 & 
0.975 & 1.060 & 1.009 & 1.075 \\ 
FAAR & \textbf{0.889 } & 0.907 & 1.029 & 0.897 & 0.941 & 1.014 & 1.054 & 
0.865* & \textbf{0.953 } & 1.046 & 0.964 \\ 
PCR & 0.935 & \textbf{0.856 } & 1.046 & \textbf{0.858 } & \textbf{0.906 } & 
1.276** & 2.084** & 0.864* & 1.472** & 1.046 & 0.875 \\ 
Bagging & 0.917 & 1.039 & 0.962 & 1.129 & 1.039 & 1.373** & 1.147 & 1.126 & 
0.963 & 1.010 & 1.001 \\ 
Boosting & 0.972 & 0.994 & 0.929 & 0.961 & 0.990 & 0.965 & 1.003 & 0.882** & 
0.997 & 1.008 & 1.032 \\ 
BMA1 & 0.974 & 0.984 & 0.941 & 0.961 & 0.994 & 0.974 & 1.019 & 0.875** & 
1.014 & 1.027 & 1.047 \\ 
BMA2 & 0.980 & 0.990 & 0.952 & 0.955 & 0.991 & 1.002 & 1.012 & 0.870** & 
1.009 & 1.028 & 1.040 \\ 
Ridge & 0.984 & 0.990 & 0.959 & 0.969 & 0.988 & 1.052 & 1.000 & \textbf{%
0.864**} & 1.003 & 1.035 & 1.013 \\ 
LAR & 0.963* & 0.984 & 0.957** & 0.991 & 0.991 & 0.970** & 1.003 & 0.950** & 
0.987 & 1.005 & 1.021 \\ 
EN & 0.963* & 0.984 & 0.957** & 0.991 & 0.991 & 0.970** & 1.003 & 0.950** & 
0.987 & 1.005 & 1.021 \\ 
NNG & 0.987* & 0.992 & 0.995 & 0.991 & 0.993 & 0.978** & 0.999 & 0.986** & 
0.995 & 1.003 & 1.005 \\ 
Mean & 0.924** & 0.952* & \textbf{0.924 } & 0.927 & 0.953 & 0.939 & \textbf{%
0.974 } & 0.884** & 0.969 & 1.006 & 0.939** \\ 
\bottomrule &  &  &  &  &  &  &  &  &  &  & 
\end{tabular}
\begin{tabular}{llllllllllll}
\multicolumn{12}{c}{{\footnotesize {Panel B: Recursive, $h=3$}}} \\ 
\multicolumn{12}{c}{} \\ 
\toprule Method & UR & PI & TB & CPI & PPI & NPE & HS & IPX & M2 & SNP & GDP
\\ 
\midrule AR(SIC) & 6.288 & 0.006 & 22.471 & 0.002 & 0.011 & 0.000 & 2.540 & 
0.010 & 0.003 & 0.432 & 0.005 \\ 
ARX(SIC) & 1.088 & \textbf{0.953 } & 1.039 & 0.976 & 0.945 & 1.149 & 1.144 & 
1.030 & 1.028 & 1.042 & 1.135** \\ 
CADL & 0.994 & 0.999 & 0.986** & 1.049 & 1.038 & \textbf{0.873**} & 0.979* & 
0.960 & 1.066 & 1.014 & 1.019 \\ 
FAAR & 1.013 & 0.975 & 1.024 & 0.985 & 0.962 & 0.998 & 1.174 & 1.014 & 1.019
& 1.069** & 1.193** \\ 
PCR & \textbf{0.975 } & 0.964 & 0.992 & 0.989 & \textbf{0.931 } & 1.107 & 
1.757** & 1.013 & 1.295** & 1.069 & 1.178** \\ 
Bagging & 1.129** & 1.144** & \textbf{0.981 } & 1.121 & 0.987 & 1.640** & 
\textbf{0.905 } & 0.982 & 1.009 & \textbf{0.991 } & 1.056** \\ 
Boosting & 1.020 & 1.000 & 1.009 & 0.992 & 0.989 & 1.069 & 1.039 & 1.015 & 
1.010 & 1.014 & 1.042 \\ 
BMA1 & 1.062 & 1.001 & 1.012 & 1.012 & 0.991 & 1.101 & 1.039 & 1.047 & 1.017
& 1.013 & 1.074 \\ 
BMA2 & 1.034 & 0.987 & 1.020 & 1.009 & 0.994 & 1.097 & 1.056 & 1.023 & 1.015
& 1.021 & 1.080 \\ 
Ridge & 1.011 & 0.965 & 1.013 & 0.994 & 1.003 & 1.086 & 1.100 & 0.986 & 1.005
& 1.032 & 1.099** \\ 
LAR & 1.035 & 1.006 & 1.012 & 0.998 & 0.996 & 1.054 & 1.033 & 1.043 & 1.002
& 1.005 & 1.012 \\ 
EN & 1.035 & 1.006 & 1.012 & 0.998 & 0.996 & 1.054 & 1.033 & 1.043 & 1.002 & 
1.005 & 1.012 \\ 
NNG & 1.004 & 1.002 & 1.014 & 0.997 & 0.999 & 1.014 & 1.000 & 1.002 & 0.998
& 1.013 & \textbf{0.998 } \\ 
Mean & 1.003 & 0.967 & 0.993 & \textbf{0.963 } & 0.956 & 0.992 & 0.990 & 
\textbf{0.951 } & \textbf{0.988 } & 1.012 & 1.031 \\ 
\bottomrule &  &  &  &  &  &  &  &  &  &  & 
\end{tabular}
\begin{tabular}{llllllllllll}
\multicolumn{12}{c}{{\footnotesize {Panel C: Recursive, $h=12$}}} \\ 
\multicolumn{12}{c}{} \\ 
\toprule Method & UR & PI & TB & CPI & PPI & NPE & HS & IPX & M2 & SNP & GDP
\\ 
\midrule AR(SIC) & 7.579 & \textbf{0.007 } & 21.740 & 0.002 & 0.011 & 0.001
& 12.047 & 0.012 & 0.004 & 0.437 & 0.005 \\ 
ARX(SIC) & 1.026 & 1.022 & 1.043 & 0.984 & 1.008 & 0.992 & 0.933 & 1.037 & 
\textbf{0.927 } & 1.030 & 1.023 \\ 
CADL & 0.985** & 1.012 & 0.983** & 0.995 & 1.029 & \textbf{0.786**} & 0.964**
& \textbf{0.973 } & 0.974 & 1.016 & 1.061 \\ 
FAAR & 0.982 & 1.079** & 0.992 & 0.976 & 1.000 & 1.071 & 0.950 & 1.100** & 
0.976 & 1.017 & 1.039 \\ 
PCR & \textbf{0.974 } & 1.052 & \textbf{0.973 } & 1.075 & 1.028 & 1.079 & 
1.220** & 1.093** & 1.110** & 1.008 & 1.013 \\ 
Bagging & 1.051 & 1.022 & 1.015 & 0.973 & 1.010 & 1.366** & \textbf{0.843* }
& 1.016 & 0.983 & 1.024 & 1.010 \\ 
Boosting & 0.996 & 1.046 & 0.999 & 0.963 & 0.988 & 1.095** & 1.007 & 1.051 & 
1.011 & 1.004 & 1.002 \\ 
BMA1 & 1.004 & 1.058** & 0.996 & 0.968 & 0.992 & 1.126** & 1.032 & 1.065 & 
1.018 & 1.000 & 1.003 \\ 
BMA2 & 0.997 & 1.060** & 0.993 & 0.969 & 0.992 & 1.120** & 1.033 & 1.066 & 
1.014 & 0.999 & 1.004 \\ 
Ridge & 0.975 & 1.058** & 0.985 & 0.966 & 0.990 & 1.079 & 1.039 & 1.070 & 
1.001 & 1.003 & 1.018 \\ 
LAR & 0.990 & 1.020 & 0.997 & 0.999 & 0.998 & 1.064** & 0.964 & 1.020 & 1.000
& 1.001 & 0.999 \\ 
EN & 0.990 & 1.020 & 0.997 & 0.999 & 0.998 & 1.064** & 0.964 & 1.020 & 1.000
& 1.001 & 1.000 \\ 
NNG & 0.990 & 1.004 & 0.995 & 0.999 & 0.998 & 1.006 & 0.993** & 0.998 & 0.999
& 1.001 & 1.001 \\ 
Mean & 0.977 & 1.008 & 0.986 & \textbf{0.954 } & \textbf{0.980 } & 0.996 & 
0.933** & 1.004 & 0.957** & \textbf{0.998 } & \textbf{0.992 } \\ 
\bottomrule \  &  &  &  &  &  &  &  &  &  &  &  \\ 
&  &  &  &  &  &  &  &  &  &  & 
\end{tabular}%
} 
\begin{minipage}{1\columnwidth} \scriptsize * Notes: See notes to Tables 1 and 2. Numerical entries in this table are mean square forecast errors (MSFEs) based on the use of various recursively estimated prediction models. Forecasts are monthly, for the period 1984:6-2009:5 ($P=300$). Models and target variables are given in Tables 1 and 2. Forecast horizons reported on include $h=1,3$ and $12$. Entries in the first row, corresponding to our benchmark AR(SIC) model, are actual MSFEs, while all other entries are relative MSFEs, such that numerical values less than unity constitute cases for which the alternative model has lower point MSFE than the AR(SIC) model. Entries in bold denote point MSFE ``best'' models for a given variable and forecast horizon.  The results from Diebold and Mariano (1995) predictive accuracy tests, for which the null hypothesis is that of equal predictive accuracy between the benchmark model (defined to be the AR(SIC) model), and the model listed in the first column of the table, 
are reported using a single star (denoting rejection at the 10\% level), and a double star (denoting rejection at the 5\% level). See Sections 4 and 5 for complete details.
\end{minipage}
\end{table}

%TABLE 4 Specification Type 1 with lags  %%%%%%%%%%%%%%%%%%%%%%%%%%%%%%%%%%%%%%%%%%%%%%%%%%%%%%%%%%%%%%%%%%%%%%%%%%%%%%%%%%%%%%%%%%%%%%
\begin{table}[htbp]
\caption{Relative Mean Square Forecast Errors of Specification Type 1 With
Lags*}
\label{tb4}
\begin{center}
{\scriptsize 
\begin{tabular}{llllllllllll}
\multicolumn{12}{c}{} \\ 
\multicolumn{12}{c}{{\footnotesize {Panel A: Recursive, $h=1$}}} \\ 
\multicolumn{12}{c}{} \\ 
\toprule Method & UR & PI & TB & CPI & PPI & NPE & HS & IPX & M2 & SNP & GDP
\\ 
\midrule AR(SIC) & 6.257 & 0.007 & 18.650 & 0.002 & 0.010 & 0.000 & 1.316 & 
0.010 & \textbf{0.003 } & \textbf{0.400 } & 0.006 \\ 
ARX(SIC) & 1.011 & 0.941 & 1.136 & \textbf{0.966 } & 0.992 & 1.347** & 1.088
& \textbf{0.913 } & 1.181 & 1.135 & 0.836 \\ 
CADL & 0.976** & 1.031 & \textbf{0.991 } & 1.026 & 1.018 & \textbf{0.920* }
& \textbf{0.988 } & 0.975 & 1.060 & 1.009 & 1.075 \\ 
FAAR & 1.069 & 0.882 & 1.851** & 1.392 & 1.121 & 1.985** & 3.631** & 1.209 & 
1.454 & 1.390** & 0.799 \\ 
PCR & 1.013 & 0.881 & 1.676** & 1.359 & 1.120 & 1.837** & 4.869** & 1.163 & 
1.700** & 1.371 & 0.889 \\ 
Bagging & 1.310** & 0.984 & 1.762** & 1.477** & 1.120 & 3.494** & 3.905** & 
1.225 & 1.229 & 1.131 & 0.804 \\ 
Boosting & 1.038 & 0.897 & 1.465** & 1.310 & 1.101 & 1.758** & 3.714** & 
1.104 & 1.419 & 1.224** & 0.795 \\ 
BMA1 & 1.042 & 0.890 & 1.544** & 1.328 & 1.097 & 1.758** & 3.711** & 1.124 & 
1.446 & 1.253** & 0.802 \\ 
BMA2 & 1.028 & 0.888 & 1.538** & 1.344 & 1.103 & 1.762** & 3.720** & 1.111 & 
1.430 & 1.242** & 0.805 \\ 
Ridge & 1.033 & 0.891 & 1.636** & 1.272 & 1.086 & 1.735** & 3.664** & 1.125
& 1.478** & 1.270** & 0.784 \\ 
LAR & 1.068 & 0.901 & 1.395** & 1.276 & 1.094 & 1.782** & 3.823** & 1.072 & 
1.360 & 1.210** & 0.791 \\ 
EN & 1.069 & 0.902 & 1.396** & 1.275 & 1.094 & 1.777** & 3.825** & 1.073 & 
1.360 & 1.210** & 0.788 \\ 
NNG & 1.070 & 0.907 & 1.408** & 1.275 & 1.097 & 1.779** & 3.844** & 1.072 & 
1.362 & 1.207** & 0.784 \\ 
Mean & \textbf{0.966 } & \textbf{0.876 } & 1.259** & 1.046 & \textbf{0.973 }
& 1.441** & 2.681** & 0.986 & 1.143 & 1.150** & \textbf{0.745**} \\ 
\bottomrule &  &  &  &  &  &  &  &  &  &  & 
\end{tabular}
\begin{tabular}{llllllllllll}
\multicolumn{12}{c}{{\footnotesize {Panel B: Recursive, $h=3$}}} \\ 
\multicolumn{12}{c}{} \\ 
\toprule Method & UR & PI & TB & CPI & PPI & NPE & HS & IPX & M2 & SNP & GDP
\\ 
\midrule AR(SIC) & 6.288 & 0.006 & 22.471 & 0.002 & 0.011 & 0.000 & 2.540 & 
0.010 & 0.003 & \textbf{0.432 } & \textbf{0.005 } \\ 
ARX(SIC) & 1.088 & \textbf{0.953 } & 1.039 & 0.976 & 0.945 & 1.149 & 1.144 & 
1.030 & 1.028 & 1.042 & 1.135** \\ 
CADL & \textbf{0.994 } & 0.999 & 0.986** & 1.049 & 1.038 & \textbf{0.873**}
& \textbf{0.979* } & \textbf{0.960 } & 1.066 & 1.014 & 1.019 \\ 
FAAR & 1.024 & 0.986 & 1.175** & 0.806** & 0.900 & 1.543** & 2.490** & 
1.228** & 1.170 & 1.115 & 1.170** \\ 
PCR & 1.021 & 0.974 & 1.151 & 0.822* & 0.906 & 1.456** & 3.539** & 1.201** & 
1.444** & 1.088 & 1.126 \\ 
Bagging & 1.122 & 1.061** & 1.102 & 1.041 & 1.034 & 2.088** & 2.465** & 1.146

& 0.989 & 1.031 & 1.101 \\ 
Boosting & 1.064 & 1.024 & 1.001 & 0.803** & 0.907 & 1.572** & 2.638** & 
1.148** & 1.065 & 1.023 & 1.131 \\ 
BMA1 & 1.053 & 1.032 & 1.009 & 0.788** & 0.894 & 1.565** & 2.641** & 1.161**
& 1.097 & 1.032 & 1.141 \\ 
BMA2 & 1.059 & 1.028 & 0.990 & 0.790** & 0.900 & 1.570** & 2.647** & 1.173**
& 1.091 & 1.030 & 1.142 \\ 
Ridge & 1.035 & 1.010 & 1.094 & 0.788** & 0.886 & 1.518** & 2.617** & 1.154
& 1.121 & 1.081 & 1.127 \\ 
LAR & 1.078 & 1.036 & 0.979 & 0.857* & 0.907 & 1.610** & 2.690** & 1.151** & 
1.007 & 1.024 & 1.104 \\ 
EN & 1.081 & 1.036 & 0.979 & 0.858* & 0.906 & 1.598** & 2.691** & 1.137 & 
1.006 & 1.025 & 1.099 \\ 
NNG & 1.075 & 1.035 & 0.973 & 0.857* & 0.913 & 1.580** & 2.687** & 1.114 & 
1.003 & 1.019 & 1.093 \\ 
Mean & 0.998 & 0.974 & \textbf{0.969 } & \textbf{0.786**} & \textbf{0.871**}
& 1.312** & 1.984** & 1.045 & \textbf{0.983 } & 1.020 & 1.046 \\ 
\bottomrule &  &  &  &  &  &  &  &  &  &  & 
\end{tabular}
\begin{tabular}{llllllllllll}
\multicolumn{12}{c}{{\footnotesize {Panel C: Recursive, $h=12$}}} \\ 
\multicolumn{12}{c}{} \\ 
\toprule Method & UR & PI & TB & CPI & PPI & NPE & HS & IPX & M2 & SNP & GDP
\\ 
\midrule AR(SIC) & 7.579 & 0.007 & 21.740 & 0.002 & 0.011 & 0.001 & 12.047 & 
0.012 & 0.004 & \textbf{0.437 } & 0.005 \\ 
ARX(SIC) & 1.026 & 1.022 & 1.043 & 0.984 & 1.008 & 0.992 & \textbf{0.933 } & 
1.037 & \textbf{0.927 } & 1.030 & 1.023 \\ 
CADL & 0.985** & 1.012 & \textbf{0.983**} & 0.995 & 1.029 & \textbf{0.786**}
& 0.964** & \textbf{0.973 } & 0.974 & 1.016 & 1.061 \\ 
FAAR & 1.155 & 1.083 & 1.148 & 0.956 & 1.025 & 1.115 & 1.135 & 1.204** & 
1.049 & 1.117 & 1.093 \\ 
PCR & 1.170** & 1.065 & 1.163 & 0.973 & 1.030 & 1.128 & 1.410** & 1.190** & 
1.177** & 1.115** & 1.089 \\ 
Bagging & 1.089 & 1.025 & 1.105 & 1.001 & 0.986 & 1.582** & 1.413** & 1.094**
& 0.968 & 1.072 & 0.997 \\ 
Boosting & \textbf{0.963 } & 1.033 & 1.020 & 0.926 & 1.012 & 1.070 & 1.210**
& 1.063 & 0.992 & 1.011 & 0.987 \\ 
BMA1 & 0.981 & 1.037 & 1.015 & 0.927 & 1.012 & 1.075 & 1.229** & 1.085** & 
0.996 & 1.011 & 0.998 \\ 
BMA2 & 0.979 & 1.034 & 1.021 & 0.924 & 1.012 & 1.067 & 1.231** & 1.088** & 
0.986 & 1.011 & 0.992 \\ 
Ridge & 1.057 & 1.038 & 1.090 & 0.925 & 1.009 & 1.055 & 1.205** & 1.122** & 
1.037 & 1.064 & 1.043 \\ 
LAR & 0.981 & 1.013 & 1.017 & 0.933 & 0.988 & 1.060** & 1.196** & 1.011 & 
0.976 & 1.006 & 0.984 \\ 
EN & 0.982 & 1.012 & 1.016 & 0.934 & 0.987 & 1.052** & 1.199** & 1.009 & 
0.978 & 1.005 & 0.984 \\ 
NNG & 0.979 & 1.021 & 1.010 & 0.930 & 0.988 & 1.044** & 1.216** & 1.003 & 
1.011 & 1.003 & 0.985 \\ 
Mean & 0.971 & \textbf{0.991 } & 1.007 & \textbf{0.885**} & \textbf{0.963 }
& 0.977 & 1.052 & 1.013 & 0.940* & 1.011 & \textbf{0.981 } \\ 
\bottomrule \  &  &  &  &  &  &  &  &  &  &  &  \\ 
&  &  &  &  &  &  &  &  &  &  & 
\end{tabular}%
} 
\begin{minipage}{1\columnwidth} \scriptsize * Notes: See notes to Table 3. Note that numerical entries for AR(SIC), ARX(SIC) and CADL models are identical to those given for Specification Type 1 without lags (see Table 1), since those benchmark models do not involve factors or shrinkage. See Section 5 for further details. 
\end{minipage}
\end{center}
\end{table}

%TABLE 5 Specification Type 2  %%%%%%%%%%%%%%%%%%%%%%%%%%%%%%%%%%%%%%%%%%%%%%%%%%%%%%%%%%%%%%%%%%%%%%%%%%%%%%%%%%%%%%%%%%%%%%
\begin{table}[htbp]
\caption{Relative Mean Square Forecast Errors of Specification Type 2*}
\label{tb5}\centering
{\scriptsize 
\begin{tabular}{llllllllllll}
\multicolumn{12}{c}{} \\ 
\multicolumn{12}{c}{{\footnotesize {Panel A: Recursive, $h=1$}}} \\ 
\multicolumn{12}{c}{} \\ 
\toprule Method & UR & PI & TB & CPI & PPI & NPE & HS & IPX & M2 & SNP & GDP
\\ 
\midrule AR(SIC) & 6.257 & 0.007 & 18.650 & 0.002 & 0.010 & 0.000 & 1.316 & 
0.010 & 0.003 & \textbf{0.400 } & 0.006 \\ 
ARX(SIC) & 1.011 & 0.941 & 1.136 & \textbf{0.966 } & 0.992 & 1.347** & 1.088
& \textbf{0.913 } & 1.181 & 1.135 & \textbf{0.836 } \\ 
CADL & 0.976** & 1.031 & \textbf{0.991 } & 1.026 & 1.018 & 0.920* & \textbf{%
0.988 } & 0.975 & 1.060 & 1.009 & 1.075 \\ 
Bagging & 1.492** & 1.194 & 1.282** & 1.266 & \textbf{0.952 } & 1.823** & 
9.795** & 1.316** & 1.655 & 1.081 & 1.089 \\ 
Boosting & 1.465** & 0.940 & 1.545** & 1.606 & 1.206 & 1.196 & 1.282 & 1.287
& 1.157 & 1.119 & 1.257 \\ 
BMA1 & 1.544** & 1.012 & 1.595** & 1.707 & 1.057 & 2.558** & 1.337** & 
1.529** & 1.246 & 1.049 & 1.793** \\ 
BMA2 & 1.021 & 0.934 & 1.603** & 1.416 & 1.444 & 1.219 & 1.229 & 1.570** & 
1.237 & 1.638 & 1.093 \\ 
Ridge & 1.089 & 0.996 & 1.654** & 1.447 & 1.477 & 1.329** & 1.206 & 1.363**
& 1.157 & 1.734 & 1.077 \\ 
LAR & \textbf{0.891**} & 0.890 & 1.388** & 1.402** & 1.119 & 1.173** & 
1.368** & 0.961 & 1.108 & 1.083 & 1.332** \\ 
EN & 0.893** & \textbf{0.885 } & 1.441** & 1.402** & 1.100 & 1.172** & 
1.367** & 0.961 & 1.115 & 1.123 & 1.289** \\ 
NNG & 0.959 & 0.925 & 1.394** & 1.404** & 1.118 & 1.114** & 1.383** & 0.943
& 1.142 & 1.976 & 1.083 \\ 
Mean & 0.965 & 0.917 & 1.131 & 1.142 & 0.973 & \textbf{0.864**} & 1.229 & 
1.008 & \textbf{0.996 } & 1.021 & 1.000 \\ 
\bottomrule &  &  &  &  &  &  &  &  &  &  & 
\end{tabular}
\begin{tabular}{llllllllllll}
\multicolumn{12}{c}{{\footnotesize {Panel B: Recursive, $h=3$}}} \\ 
\multicolumn{12}{c}{} \\ 
\toprule Method & UR & PI & TB & CPI & PPI & NPE & HS & IPX & M2 & SNP & GDP
\\ 
\midrule AR(SIC) & 6.288 & 0.006 & 22.471 & 0.002 & 0.011 & 0.000 & 2.540 & 
0.010 & \textbf{0.003 } & \textbf{0.432 } & \textbf{0.005 } \\ 
ARX(SIC) & 1.088 & 0.953 & 1.039 & 0.976 & 0.945 & 1.149 & 1.144 & 1.030 & 
1.028 & 1.042 & 1.135** \\ 
CADL & 0.994 & 0.999 & 0.986** & 1.049 & 1.038 & \textbf{0.873**} & \textbf{%
0.979* } & 0.960 & 1.066 & 1.014 & 1.019 \\ 
Bagging & 1.056 & 1.016 & 1.042 & 1.001 & 0.960 & 1.312** & 2.267** & 1.124
& 1.369** & 1.086 & 1.164** \\ 
Boosting & 0.941 & 0.962 & 0.998 & 1.147 & 1.050 & 1.005 & 1.172 & 1.084 & 
1.157 & 1.019 & 1.153** \\ 
BMA1 & 0.992 & 0.958 & 0.989 & 1.090 & 1.139 & 1.151 & 1.178 & 0.971 & 1.144
& 1.010 & 1.168 \\ 
BMA2 & \textbf{0.937 } & 0.974 & 1.014 & 1.021 & 0.970 & 0.908 & 1.312 & 
1.013 & 1.199 & 1.051 & 1.223** \\ 
Ridge & 0.943 & 0.975 & 0.993 & 1.086 & 0.978 & 0.961 & 1.276** & 1.053 & 
1.183 & 1.055 & 1.216** \\ 
LAR & 0.980 & 1.016 & 1.009 & 0.938 & 0.934 & 1.179** & 1.371** & 0.958 & 
1.121 & 1.029 & 1.175** \\ 
EN & 0.982 & 1.004 & 0.998 & 0.937 & \textbf{0.934 } & 1.201** & 1.379** & 
0.985 & 1.118 & 1.026 & 1.147 \\ 
NNG & 1.001 & 0.990 & 1.004 & \textbf{0.936 } & 0.953 & 1.107** & 1.356** & 
1.054 & 1.174 & 1.019 & 1.111 \\ 
Mean & 0.949 & \textbf{0.952* } & \textbf{0.970 } & 0.941 & 0.936* & 0.949 & 
1.106 & \textbf{0.936 } & 1.060 & 1.011 & 1.066 \\ 
\bottomrule &  &  &  &  &  &  &  &  &  &  & 
\end{tabular}
\begin{tabular}{llllllllllll}
\multicolumn{12}{c}{{\footnotesize {Panel C: Recursive, $h=12$}}} \\ 
\multicolumn{12}{c}{} \\ 
\toprule Method & UR & PI & TB & CPI & PPI & NPE & HS & IPX & M2 & SNP & GDP
\\ 
\midrule AR(SIC) & 7.579 & 0.007 & 21.740 & 0.002 & \textbf{0.011 } & 0.001
& 12.047 & 0.012 & 0.004 & 0.437 & 0.005 \\ 
ARX(SIC) & 1.026 & 1.022 & 1.043 & \textbf{0.984 } & 1.008 & 0.992 & 0.933 & 
1.037 & 0.927 & 1.030 & 1.023 \\ 
CADL & 0.985** & 1.012 & \textbf{0.983**} & 0.995 & 1.029 & \textbf{0.786**}
& 0.964** & 0.973 & 0.974 & 1.016 & 1.061 \\ 
Bagging & 0.971 & 1.051 & 1.021 & 2.434** & 1.503** & 1.004 & 1.193** & 1.030
& 1.151 & 1.010 & 1.049 \\ 
Boosting & 0.968 & 0.999 & 1.025 & 1.039 & 1.061 & 1.059 & 0.952 & 1.115** & 
1.006 & 1.002 & 0.994 \\ 
BMA1 & 1.065 & 0.986 & 0.999 & 1.050 & 1.095 & 1.009 & 1.002 & 1.009 & 0.919
& \textbf{0.981 } & 0.984 \\ 
BMA2 & 0.980 & 1.025 & 1.027 & 1.079 & 1.092** & 1.021 & 1.013 & 1.046 & 
1.055 & 1.007 & 1.026 \\ 
Ridge & 0.981 & 1.017 & 1.017 & 1.064 & 1.105** & 0.951 & 0.992 & 1.036 & 
1.041 & 1.009 & 1.019 \\ 
LAR & 0.984 & 0.997 & 1.011 & 0.995 & 1.007 & 1.083 & 1.005 & \textbf{0.972 }
& 0.974 & 1.003 & 0.999 \\ 
EN & 0.985 & 1.000 & 1.010 & 1.023 & 1.006 & 1.083 & 1.002 & 1.014 & 0.977 & 
1.003 & 1.000 \\ 
NNG & 0.992 & 1.002 & 1.004 & 1.005 & 1.024 & 1.011 & 1.036 & 0.993 & 1.004
& 1.004 & 0.988 \\ 
Mean & \textbf{0.950 } & \textbf{0.976 } & 0.994 & 0.987 & 1.014 & 0.900** & 
\textbf{0.930**} & 0.974 & \textbf{0.914**} & 0.996 & \textbf{0.970 } \\ 
\bottomrule \  &  &  &  &  &  &  &  &  &  &  &  \\ 
&  &  &  &  &  &  &  &  &  &  & 
\end{tabular}%
} 
\begin{minipage}{1\columnwidth} \scriptsize * Notes: See notes to Tables 3-4. Note that FAAR and PCR model results are not included in this table, since these models are not constructed under Specification Type 2.
\end{minipage}
\end{table}

%TABLE 6 Specification Type 3  %%%%%%%%%%%%%%%%%%%%%%%%%%%%%%%%%%%%%%%%%%%%%%%%%%%%%%%%%%%%%%%%%%%%%%%%%%%%%%%%%%%%%%%%%%%%%%
\begin{table}[htbp]
\caption{Relative Mean Square Forecast Errors of Specification Type 3*}
\label{tb6}{\scriptsize 
\begin{tabular}{llllllllllll}
\multicolumn{12}{c}{} \\ 
\multicolumn{12}{c}{{\footnotesize {Panel A: Recursive, $h=1$}}} \\ 
\multicolumn{12}{c}{} \\ 
\toprule Method & UR & PI & TB & CPI & PPI & NPE & HS & IPX & M2 & SNP & GDP
\\ 
\midrule AR(SIC) & 6.257 & 0.007 & 18.650 & 0.002 & 0.010 & 0.000 & 1.316 & 
0.010 & 0.003 & 0.400 & 0.006 \\ 
ARX(SIC) & 1.011 & \textbf{0.941 } & 1.136 & 0.966 & 0.992 & 1.347** & 1.088
& 0.913 & 1.181 & 1.135 & 0.836 \\ 
CADL & \textbf{0.976**} & 1.031 & 0.991 & 1.026 & 1.018 & \textbf{0.920* } & 
\textbf{0.988 } & 0.975 & 1.060 & 1.009 & 1.075 \\ 
Bagging & 6.843** & 1.142 & 5.235** & 5.337 & 2.120 & 13.78** & 10.14** & 
3.521** & 4.217 & 11.14 & 3.637 \\ 
Boosting & 1.010** & 0.985 & \textbf{0.991**} & 0.947 & 0.983 & 1.143 & 1.011
& \textbf{0.893 } & 1.118 & 1.024 & \textbf{0.826 } \\ 
BMA1 & 1.137** & 1.010 & 1.062** & 0.989 & 1.069 & 2.150** & 1.409** & 
0.966** & 1.080 & 1.162 & 0.842** \\ 
BMA2 & 0.980 & 0.998 & 1.081** & 0.996 & 1.027 & 1.440 & 1.301 & 0.905** & 
1.115 & 1.121 & 0.834 \\ 
Ridge & 1.564 & 1.274 & 1.354** & 1.051 & 1.186 & 2.258** & 1.231 & 1.322**
& 1.334 & 1.417 & 1.079 \\ 
LAR & 0.990** & 0.998 & 1.000** & 1.017** & 1.004 & 0.981** & 1.003** & 1.000
& \textbf{0.992 } & \textbf{0.998 } & 0.847** \\ 
EN & 0.990** & 0.996 & 1.000** & 1.016** & 1.004 & 0.982** & 1.001** & 1.000
& 0.993 & 0.998 & 0.850** \\ 
NNG & 0.996 & 0.998 & 0.998** & 0.999** & 1.000 & 0.989** & 1.003** & 0.995
& 1.001 & 1.001 & 1.000 \\ 
Mean & 1.012 & 0.954 & 1.365 & \textbf{0.928 } & \textbf{0.937 } & 1.822** & 
1.013 & 0.928 & 1.001 & 1.102 & 0.865 \\ 
\bottomrule &  &  &  &  &  &  &  &  &  &  & 
\end{tabular}
\begin{tabular}{llllllllllll}
\multicolumn{12}{c}{{\footnotesize {Panel B: Recursive, $h=3$}}} \\ 
\multicolumn{12}{c}{} \\ 
\toprule Method & UR & PI & TB & CPI & PPI & NPE & HS & IPX & M2 & SNP & GDP
\\ 
\midrule AR(SIC) & 6.288 & 0.006 & 22.471 & 0.002 & 0.011 & 0.000 & 2.540 & 
0.010 & 0.003 & 0.432 & 0.005 \\ 
ARX(SIC) & 1.088 & 0.953 & 1.039 & \textbf{0.976 } & \textbf{0.945 } & 1.149
& 1.144 & 1.030 & 1.028 & 1.042 & 1.135** \\ 
CADL & \textbf{0.994 } & 0.999 & \textbf{0.986**} & 1.049 & 1.038 & \textbf{%
0.873**} & 0.979* & 0.960 & 1.066 & 1.014 & 1.019 \\ 
Bagging & 13.34 & 2.042 & 5.843 & 7.926 & 3.300 & 19.95** & 12.30** & 3.530
& 8.547** & 13.84 & 19.75** \\ 
Boosting & 1.012 & \textbf{0.939 } & 1.023 & 0.991 & 1.061 & 0.966 & \textbf{%
0.934 } & 0.983 & 1.034 & 1.008 & 1.033** \\ 
BMA1 & 1.033 & 0.956 & 1.258 & 1.125 & 1.098 & 1.072 & 1.301 & 1.014 & 1.074
& 1.088 & 1.116 \\ 
BMA2 & 1.035 & 0.968 & 1.027 & 1.062 & 1.051 & 1.030 & 1.219 & 1.044 & 0.991
& 1.050 & 1.090** \\ 
Ridge & 1.471 & 1.239 & 1.974 & 1.261 & 1.350 & 1.647 & 1.148** & 1.460 & 
1.148 & 1.261 & 1.662** \\ 
LAR & 0.996 & 0.988 & 1.002 & 1.006 & 1.005 & 0.981** & 0.994** & 0.962 & 
0.986 & 0.999 & 1.002** \\ 
EN & 0.994 & 0.988 & 1.002 & 1.006 & 1.005 & 0.982** & 0.995** & \textbf{%
0.955 } & 0.990 & \textbf{0.999 } & \textbf{0.999 } \\ 
NNG & 0.999 & 0.998 & 1.000 & 0.999 & 0.999 & 1.001** & 1.006** & 1.000 & 
1.001 & 1.003 & 1.002 \\ 
Mean & 1.130 & 0.968* & 1.536 & 1.035 & 1.010* & 2.469 & 1.012 & 1.208 & 
\textbf{0.936 } & 1.059 & 1.202 \\ 
\bottomrule &  &  &  &  &  &  &  &  &  &  & 
\end{tabular}
\begin{tabular}{llllllllllll}
\multicolumn{12}{c}{{\footnotesize {Panel C: Recursive, $h=12$}}} \\ 
\multicolumn{12}{c}{} \\ 
\toprule Method & UR & PI & TB & CPI & PPI & NPE & HS & IPX & M2 & SNP & GDP
\\ 
\midrule AR(SIC) & 7.579 & 0.007 & 21.740 & 0.002 & 0.011 & 0.001 & 12.047 & 
0.012 & 0.004 & 0.437 & 0.005 \\ 
ARX(SIC) & 1.026 & 1.022 & 1.043 & 0.984 & 1.008 & 0.992 & 0.933 & 1.037 & 
\textbf{0.927 } & 1.030 & 1.023 \\ 
CADL & 0.985** & 1.012 & \textbf{0.983**} & 0.995 & 1.029 & \textbf{0.786**}
& 0.964** & \textbf{0.973 } & 0.974 & 1.016 & 1.061 \\ 
Bagging & 2.681 & 2.070 & 9.818 & 3.783** & 7.000** & 10.45 & 6.306** & 9.277
& 9.428 & 12.94 & 8.207 \\ 
Boosting & \textbf{0.958 } & \textbf{0.998 } & 1.022 & 0.953 & 1.002 & 0.915
& 0.899 & 1.026** & 0.973 & 1.006 & 1.002 \\ 
BMA1 & 1.209 & 1.035 & 1.162 & 1.579 & 1.253 & 1.669 & 2.078 & 1.112 & 1.045
& 1.038 & 1.203 \\ 
BMA2 & 1.133 & 1.011 & 1.054 & 1.383 & 1.126** & 1.365 & 1.783 & 1.063 & 
1.004 & 1.021 & 1.036 \\ 
Ridge & 1.321 & 1.178 & 1.863 & 1.210 & 1.306** & 1.124 & 1.129 & 1.598 & 
1.083 & 1.358 & 1.692 \\ 
LAR & 0.983 & 1.002 & 1.001 & 1.004 & 0.997 & 0.968 & 0.928 & 1.006 & 0.983
& 1.000 & 1.001 \\ 
EN & 0.985 & 1.002 & 1.001 & 0.992 & \textbf{0.995 } & 0.982 & 0.984 & 0.999
& 0.981 & \textbf{0.999 } & 1.000 \\ 
NNG & 0.996 & 1.002 & 1.000 & 0.999 & 1.000 & 0.999 & 0.998 & 1.001 & 1.001
& 1.001 & \textbf{0.998 } \\ 
Mean & 1.118 & 1.154 & 1.076 & \textbf{0.950 } & 1.020 & 1.185** & \textbf{%
0.897**} & 1.526 & 0.996** & 1.062 & 1.080 \\ 
\bottomrule \  &  &  &  &  &  &  &  &  &  &  &  \\ 
&  &  &  &  &  &  &  &  &  &  & 
\end{tabular}%
} 
\begin{minipage}{1\columnwidth} \scriptsize * Notes: See notes to Tables 3-5.
\end{minipage}
\end{table}

%TABLE 7 %%%%%%%%%%%%%%%%%%%%%%%%%%%%%%%%%%%%%%%%%%%%%%%%%%%%%%%%%%%%%%%%%%%%%%%%%%%%%%%%%%%%%%%%%%%%%%
\begin{table}[tbph]
\caption{Forecast Experiment Summary Results*}
\label{tb7}
{}
\par
\begin{center}
Best Forecasting Methods by Target Variable and Specification Type
\end{center}
\par
\centering{\scriptsize 
\begin{tabular}{lccccccccccc}
\multicolumn{12}{c}{} \\ 
\multicolumn{12}{c}{{\footnotesize {Panel A: Specification Type 1 Without
Lags}}} \\ 
\multicolumn{12}{c}{} \\ 
\toprule & UR & PI & TB & CPI & PPI & NPE & HS & IPX & M2 & SNP & GDP \\ 
\midrule$h=1$ & FAAR & PCR & Mean & PCR & PCR & CADL & Mean & Ridge & FAAR & 
AR & ARX \\ 
$h=3$ & PCR & ARX & Bag & Mean & PCR & CADL & Bag & Mean & Mean & Bag & NNG
\\ 
$h=1$2 & PCR & AR & PCR & Mean & Mean & CADL & Bag & CADL & ARX & Mean & Mean
\\ 
\bottomrule &  &  &  &  &  &  &  &  &  &  & 
\end{tabular}
\begin{tabular}{lccccccccccc}
\multicolumn{12}{c}{{\footnotesize {Panel B: Specification Type 1 With Lags}}
} \\ 
\multicolumn{12}{c}{} \\ 
\toprule & UR & PI & TB & CPI & PPI & NPE & HS & IPX & M2 & SNP & GDP \\ 
\midrule$h=1$ & Mean & Mean & CADL & ARX & Mean & CADL & CADL & ARX & AR & AR
& Mean \\ 
$h=3$ & CADL & ARX & Mean & Mean & Mean & CADL & CADL & CADL & Mean & AR & AR
\\ 
$h=12$ & Boost & Mean & CADL & Mean & Mean & CADL & ARX & CADL & ARX & AR & 
Mean \\ 
\bottomrule &  &  &  &  &  &  &  &  &  &  & 
\end{tabular}
\begin{tabular}{lccccccccccc}
\multicolumn{12}{c}{{\footnotesize {Panel C: Specification Type 2}}} \\ 
\multicolumn{12}{c}{} \\ 
\toprule & UR & PI & TB & CPI & PPI & NPE & HS & IPX & M2 & SNP & GDP \\ 
\midrule$h=1$ & LAR & EN & CADL & ARX & Bag & Mean & CADL & ARX & Mean & AR
& ARX \\ 
$h=3$ & BMA2 & Mean & Mean & NNG & EN & CADL & CADL & Mean & AR & AR & AR \\ 
$h=1$2 & Mean & Mean & CADL & ARX & AR & CADL & Mean & LAR & Mean & BMA1 & 
Mean \\ 
\bottomrule &  &  &  &  &  &  &  &  &  &  & 
\end{tabular}
\begin{tabular}{lccccccccccc}
\multicolumn{12}{c}{{\footnotesize {Panel D: Specification Type 3}}} \\ 
\multicolumn{12}{c}{} \\ 
\toprule & UR & PI & TB & CPI & PPI & NPE & HS & IPX & M2 & SNP & GDP \\ 
\midrule$h=1$ & CADL & ARX & Boost & Mean & Mean & CADL & CADL & Boost & LAR
& LAR & Boost \\ 
$h=3$ & CADL & Boost & CADL & ARX & ARX & CADL & Boost & EN & Mean & EN & EN
\\ 
$h=12$ & Boost & Boost & CADL & Mean & EN & CADL & Mean & CADL & ARX & EN & 
NNG \\ 
\bottomrule &  &  &  &  &  &  &  &  &  &  & 
\end{tabular}
\begin{tabular}{lccccccccccc}
\multicolumn{12}{c}{{\footnotesize {Panel E: All Specification Types}}} \\ 
\multicolumn{12}{c}{} \\ 
\toprule & UR & PI & TB & CPI & PPI & NPE & HS & IPX & M2 & SNP & GDP \\ 
\midrule\multirow{2}{.5in}{$h=1$} & SP1 & SP1 & SP1 & SP1 & SP1 & SP2 & SP1
& SP1 & SP1 & SP3 & SP1L \\ 
& FAAR & PCR & Mean & PCR & PCR & Mean & Mean & Ridge & FAAR & LAR & Mean \\ 
\midrule\multirow{2}{.5in}{$h=3$} & SP2 & SP3 & SP1L & SP1L & SP1L & - & SP1
& SP2 & SP3 & SP1 & SP1 \\ 
& BMA2 & Boost & Mean & Mean & Mean & CADL & Bag & Mean & Mean & Bag & NNG
\\ 
\midrule\multirow{2}{.5in}{$h=12$} & SP2 & SP2 & SP1 & SP1L & SP1L & - & SP1
& SP2 & SP2 & SP2 & SP2 \\ 
& Mean & Mean & PCR & Mean & Mean & CADL & Bag & LAR & Mean & BMA1 & Mean \\ 
\bottomrule &  &  &  &  &  &  &  &  &  &  & 
\end{tabular}
} 
\begin{minipage}{1\columnwidth} \scriptsize * Notes: See notes to Tables 1-5. Entries in first four panels denote the method yielding the lowest point MSFE for given target variable. These entries correspond to entries in bold in Tables 3-6. Each pair of entries in Panel E  denotes the MSFE ``best'' method and specification type for given target variables and forecast horizon, using recursive estimation. 
The way to read entries in Panels E is as follows. When forecasting unemployment (UR), the FAAR model under Specification Type 1 without lags yields the lowest MSFE, for $h=1$, across all specification types.  
\end{minipage}
\end{table}

%TABLE 8 Forecast Experiment Summary Results %%%%%%%%%%%%%%%%%%%%%%%%%%%%%%%%%%%%%%%%%%%%%%%%%%%%%%%%%%%%%%%%%%%%%%%%
\begin{table}[htbp]
\caption{Forecast Experiment Summary Results*}
\label{tb8}\centering
{\scriptsize %\begin{tabular}{lcccccc}
\begin{tabular}{lx{1.5cm}x{1.5cm}x{1.5cm}x{1.5cm}x{1.5cm}x{1.5cm}}
% PANEL A -------------------------------------------------------------------------
\multicolumn{7}{c}{} \\ 
\multicolumn{7}{c}{\footnotesize{Panel A: Summary of MSFE ``best'' Models Across All Specification Types}}\\
\multicolumn{7}{c}{} \\ \toprule
{} & \multicolumn{3}{x{4.5cm}}{Recursive Estimation Windows} & \multicolumn{3}{x{4.5cm}}{Recursive and Rolling Estimation Windows} \\
{} & { $h=1$} & { $h=3$} & {$h=12$} & {$h=1$} & {$h=3$} & {$h=12$} \\ \cmidrule(r){2-4} \cmidrule(r){5-7}
AR(SIC)&	0&	0&	0&	0&	0&	0	\\
ARX(SIC)&	0&	0&	0&	0&	0&	0	\\
CADL &	0&	1&	1&	0&	1&	1	\\
FAAR&	2&	0&	0&	2&	0&	0	\\
PCR&	3&	0&	1&	3&	0&	1	\\
Bagging&	0&	2&	1&	0&	2&	0	\\
Boosting&	0&	1&	0&	0&	1&	1	\\
BMA1&	0&	0&	1&	0&	0&	3	\\
BMA2&	0&	1&	0&	0&	1&	0	\\
Ridge&	1&	0&	0&	1&	0&	0	\\
LAR&	1&	0&	1&	0&	0&	1	\\
EN&	0&	0&	0&	0&	0&	0	\\
NNG&	0&	1&	0&	2&	1&	0	\\
Mean&	4&	5&	6&	3&	5&	4	\\ \bottomrule
% PANEL B -------------------------------------------------------------------------
\multicolumn{7}{c}{} \\ 
\multicolumn{7}{c}{\footnotesize{Panel B: Summary of MSFE-"best" Models }}\\
\multicolumn{7}{c}{} \\ \toprule
{} & \multicolumn{3}{x{4.5cm}}{Winners by Estimation Window Type} & \multicolumn{3}{x{4.5cm}}{Winners by Specification Type} \\
{} & {$ h=1$} & {$ h=3$} & {$h=12$} & {$ h=1$} & {$ h=3$} & {$h=12$} \\ 
\cmidrule(r){2-4} \cmidrule(r){5-7}
Specification Type 1 without lags&	\multicolumn{6}{c}{}						\\
{             Recursive}&	6&	10&	4&	\multirow{2}{.2in}{8}&	\multirow{2}{.2in}{4}&	\multirow{2}{.2in}{3}	\\
{             Rolling}&	5&	1&	7&	&	&		\\
Specification Type 1 with lags&	\multicolumn{6}{c}{}						\\
{             Recursive}&	8&	10&	7&	\multirow{2}{.2in}{1}&	\multirow{2}{.2in}{3}&	\multirow{2}{.2in}{2}	\\
{             Rolling}&	3&	1&	4&	&	&		\\
Specification Type 2&	\multicolumn{6}{c}{}						\\
{             Recursive}&	7&	7&	6&	\multirow{2}{.2in}{1}&	\multirow{2}{.2in}{2}&	\multirow{2}{.2in}{6}	\\
{             Rolling}&	4&	4&	5&	&	&		\\
Specification Type 3&	\multicolumn{6}{c}{}						\\
{             Recursive}&	6&	10&	7&	\multirow{2}{.2in}{1}&	\multirow{2}{.2in}{2}&	\multirow{2}{.2in}{0}	\\
{             Rolling}&	5&	1&	4&	&	&		\\ \bottomrule
\multicolumn{7}{c}{} \\ 
\end{tabular}} 
\begin{minipage}{1\columnwidth} \scriptsize * Notes: See notes to Tables 2,3 and 7. Numerical entries in Panel A denote the number of ``wins'' of each forecasting method, at a given forecasting horizon. The lefthand set of 3 columns summarizes results for recursive estimation, and correpond to entries in Panel B of Table 7. The righthand set of 3 columns summarizes results based on both recursive and rolling estimation methods.  
\end{minipage}
\end{table}

\newpage 
%TABLE 9 Forecast Experiment Summary Results %%%%%%%%%%%%%%%%%%%%%%%%%%%%%%%%%%%%%%%%%%%%%%%%%%%%%%%%%%%%%%%%%%%%%%%%
\begin{landscape}
\begin{table}[htbp]
\caption{Forecast Experiment Summary Results*}
\label{tb9}\centering
{\scriptsize 
\begin{tabular}{cx{.86cm}x{.86cm}x{.86cm}x{.86cm}x{.86cm}x{.86cm}x{.86cm}x{.86cm}x{.86cm}x{.86cm}x{.86cm}x{.86cm}x{.86cm}x{.86cm}x{.86cm}x{.86cm}}
\multicolumn{17}{c}{} \\ 
% PANEL A -----------------------------------------------------------------------------------------------------------
%\multicolumn{13}{c}{\footnotesize{Panel A: MSFE ``Wins'' by Specification Type  $h=1$, Recursive Estimation}}\\
\multicolumn{5}{c}{} & \multicolumn{7}{x{6.5cm}}{\footnotesize{Panel A: Wins By Specification Types, $h=1$, Recursive Estimation}} & \multicolumn{5}{c}{}\\
\multicolumn{17}{c}{} \\ \toprule
 & \multicolumn{4}{c}{Specification  Type 1 Without Lags} & \multicolumn{4}{c}{Specification Type 1 With Lags} & \multicolumn{4}{c}{Specification Type 2} & \multicolumn{4}{c}{Specification Type 3} \\
\cmidrule(r){2-5} \cmidrule(r){6-9} \cmidrule(r){10-13} \cmidrule(r){14-17}
Subsample & Mean	& Linear Factor & Nonlinear Factor & Other & Mean	& Linear Factor & Nonlinear Factor & Other & Mean	& Linear Factor & Nonlinear Factor & Other & Mean	& Linear Factor & Nonlinear Factor & Other \\ 
\cmidrule(r){2-5} \cmidrule(r){6-9} \cmidrule(r){10-13} \cmidrule(r){14-17}
% DATA INPUT STARTS HERE
84:06 $\symbol{126}$ 90:06&	3&	3&	4&	1&	4&	0&	1&	6&	0&	0&	5&	6&	4&	1&	4&	2	\\
91:03 $\symbol{126}$ 01:02&	5&	2&	4&	0&	4&	0&	1&	6&	3&	0&	1&	7&	4&	0&	4&	3	\\
01:11 $\symbol{126}$ 07:11&	0&	4&	4&	2&	2&	1&	1&	7&	1&	0&	1&	9&	5&	1&	1&	4	\\
Expansion&	4&	4&	2&	1&	4&	1&	0&	6&	2&	0&	3&	6&	5&	1&	4&	1	\\
Recession&	1&	4&	2&	4&	1&	1&	2&	7&	0&	2&	3&	6&	0&	2&	7&	2	\\ \bottomrule
% DATA INPUT ENDS HERE
%\bottomrule
\multicolumn{17}{c}{} \\
% PANEL B -----------------------------------------------------------------------------------------------------------
%\multicolumn{13}{c}{\footnotesize{Panel B: MSFE ``Wins'' Across All Specification Types  $h=1$, Recursive Estimation}}\\
\multicolumn{5}{c}{} & \multicolumn{7}{x{7.5cm}}{\footnotesize{Panel B: Wins Across All Specification Types, $h=1$, Recursive Estimation}} & \multicolumn{5}{c}{}\\
\multicolumn{17}{c}{} \\ \toprule
  & \multicolumn{4}{c}{Specification Type 1 Without Lags} & \multicolumn{4}{c}{Specification Type 1 With Lags} & \multicolumn{4}{c}{Specification Type 2} & \multicolumn{4}{c}{Specification Type 3} \\
\cmidrule(r){2-5} \cmidrule(r){6-9} \cmidrule(r){10-13} \cmidrule(r){14-17}
Subsample & Mean	& Linear Factor & Nonlinear Factor & Other & Mean	& Linear Factor & Nonlinear Factor & Other & Mean	& Linear Factor & Nonlinear Factor & Other & Mean	& Linear Factor & Nonlinear Factor & Other\\ 
\cmidrule(r){2-5} \cmidrule(r){6-9} \cmidrule(r){10-13} \cmidrule(r){14-17}
% DATA INPUT STARTS HERE
84:06 $\symbol{126}$ 90:06&	0&	1&	1&	0&	0&	0&	1&	3&	0&	0&	3&	1&	1&	0&	0&	0	\\
91:03 $\symbol{126}$ 01:02&	1&	1&	3&	0&	2&	0&	0&	1&	1&	0&	1&	0&	1&	0&	0&	0	\\
01:11 $\symbol{126}$ 07:11&	0&	2&	1&	1&	0&	1&	0&	1&	0&	0&	0&	3&	2&	0&	0&	0	\\
Expansion&	1&	0&	1&	0&	0&	0&	3&	1&	2&	1&	0&	0&	2&	0&	0&	0	\\
Recession&	0&	0&	1&	2&	0&	1&	1&	0&	0&	1&	2&	0&	0&	0&	3&	0	\\ \bottomrule
% DATA INPUT ENDS HERE
%\bottomrule
\multicolumn{17}{c}{} 
\end{tabular}}
\begin{minipage}{1\columnwidth} \scriptsize * Notes: See notes to Tables 3 and 8. Entries in Panel A enumerate how many times each ``group'' of forecasting methods yields the lowest MSFE for each target variable and for each specification type. The ``groups'' are defined as follows.
\textquotedblleft Mean\textquotedblright  includes: BMA, Combined-ADL and Mean. \textquotedblleft Linear Factor\textquotedblright  includes: FAAR and PCR. \textquotedblleft Nonlinear Factor\textquotedblright  includes: all shrinkage/factor combination models (i.e., see Specification Types 1 and 2 in Table 2). Finally, \textquotedblleft Other\textquotedblright  includes our linear AR(SIC) and ARX(SIC) models. 
See Section 3 for further details. Entries in Panel B enumerate how many times each set of forecasting methods yields lowest MSFE across all specification types. 
Subsamples follow business cycles dated by the NBER. The first relevant NBER dated expansion started in 82:11, but our forecasting experiments start in 84:6, so that the start date of first subsample is set at 84:6. \textquotedblleft Expansion\textquotedblright  includes all of expansion dates in the forecasting period and \textquotedblleft Recession\textquotedblright  includes all of contraction dates in the forecasting period. 
\end{minipage}
\end{table}
\end{landscape} 

%TABLE 10 Reality Check  %%%%%%%%%%%%%%%%%%%%%%%%%%%%%%%%%%%%%%%%%%%%%%%%%%%%%%%%%%%%%%%%%%%%%%%%%%%%%%%%%%%%%%%%%%%%%%
\newpage 
%\begin{longtable}{y{2cm}y{.9cm}y{.9cm}y{.9cm}y{.9cm}y{.9cm}y{.9cm}y{.9cm}y{.9cm}y{.9cm}y{.9cm}y{.9cm}}
\LTcapwidth=\textwidth
\centering
\begin{longtable}{y{2cm}lllllllllll}
\caption{\normalsize {Reality Check Statistics for Various Forecast Experiments*}}\\
\label{tb10}\\
%{\scriptsize 
%\begin{tabular}{lccccccccccc}
%\begin{tabular}{llllllllllll}
%\multicolumn{12}{c}{} \\
\multicolumn{12}{c}{{\footnotesize {Panel A: Recursive Estimation, $h=1$}}} \\ \multicolumn{12}{c}{} \\
\toprule \multicolumn{1}{c}{{\scriptsize {Models}}} & \multicolumn{1}{c}{{\scriptsize {UR}}} & \multicolumn{1}{c}{{\scriptsize {PI}}} & \multicolumn{1}{c}{{\scriptsize {TB}}} & \multicolumn{1}{c}{{\scriptsize {CPI}}} & \multicolumn{1}{c}{{\scriptsize {PPI}}} & \multicolumn{1}{c}{{\scriptsize {NPE}}} & \multicolumn{1}{c}{{\scriptsize {HS}}}& \multicolumn{1}{c}{{\scriptsize {IPX}}} & \multicolumn{1}{c}{{\scriptsize {M2}}} & \multicolumn{1}{c}{{\scriptsize {SNP}}} & \multicolumn{1}{c}{{\scriptsize {GDP}}} \\ \midrule
\endfirsthead
%\multicolumn{12}{c}{} \\
%\toprule {\scriptsize {Cases}} & {\scriptsize {UR}} & {\scriptsize {PI}} & {\scriptsize {TB}} & {\scriptsize {CPI}} & {\scriptsize {PPI}} & {\scriptsize {NPE}} & {\scriptsize {HS}}& {\scriptsize {IPX}} & {\scriptsize {M2}} & {\scriptsize {SNP}} & {\scriptsize {GDP}} \\ \midrule
%\multicolumn{12}{c}{{\footnotesize {Continued from previous page}}} \\ 
%\endhead
%
%\multicolumn{12}{c}{{\footnotesize {Continued in next page}}} \\ 
%\endfoot
%
%\\
%\endlastfoot
%
{\scriptsize {AR vs. SP1s}}&	{\scriptsize { 0.402* }}&	{\scriptsize { 5.800**}}&	{\scriptsize { 0.809* }}&	{\scriptsize { 1.550* }}&	{\scriptsize { 5.429* }}&	{\scriptsize { 0.132* }}&	{\scriptsize { 0.019* }}&	{\scriptsize { 0.818* }}&	{\scriptsize { 0.859* }}&	{\scriptsize {-0.069  }}&	{\scriptsize{ 5.982* }}	\\
{\scriptsize {AR vs. SP1FAARs}}&	{\scriptsize { 0.402* }}&	{\scriptsize { 5.800**}}&	{\scriptsize { 0.809* }}&	{\scriptsize { 1.550* }}&	{\scriptsize { 5.429* }}&	{\scriptsize { 0.101* }}&	{\scriptsize { 0.019* }}&	{\scriptsize { 0.818* }}&	{\scriptsize { 0.859* }}&	{\scriptsize {-0.069  }}&	{\scriptsize{ 4.563* }}	\\
{\scriptsize {AR vs. SP1LFAARs}}&	{\scriptsize { 0.122* }}&	{\scriptsize { 4.977**}}&	{\scriptsize { 0.096* }}&	{\scriptsize { 0.375* }}&	{\scriptsize { 1.582* }}&	{\scriptsize { 0.132* }}&	{\scriptsize { 0.009* }}&	{\scriptsize { 0.520* }}&	{\scriptsize {-1.090  }}&	{\scriptsize {-0.208  }}&	{\scriptsize{ 9.301* }}	\\
{\scriptsize {AR vs. SP2s}}&	{\scriptsize { 0.395**}}&	{\scriptsize { 3.986**}}&	{\scriptsize {-1.251  }}&	{\scriptsize {-1.265  }}&	{\scriptsize { 1.845* }}&	{\scriptsize { 0.149* }}&	{\scriptsize {-0.168  }}&	{\scriptsize { 0.345* }}&	{\scriptsize { 1.716* }}&	{\scriptsize {-1.146  }}&	{\scriptsize{ 1.984**}}	\\
{\scriptsize {AR vs. SP3s}}&	{\scriptsize { 0.168* }}&	{\scriptsize { 2.245**}}&	{\scriptsize { 0.620**}}&	{\scriptsize { 1.072**}}&	{\scriptsize { 4.217* }}&	{\scriptsize { 0.031* }}&	{\scriptsize {-0.001  }}&	{\scriptsize { 0.640**}}&	{\scriptsize { 0.924**}}&	{\scriptsize { 0.048* }}&	{\scriptsize{ 7.098* }}	\\
{\scriptsize {SP1FAAR vs. SP1Shrinks}}&	{\scriptsize {-1.029  }}&	{\scriptsize { 2.061* }}&	{\scriptsize { 1.126* }}&	{\scriptsize { 0.425* }}&	{\scriptsize { 1.993**}}&	{\scriptsize { 1.246* }}&	{\scriptsize { 0.603* }}&	{\scriptsize { 0.007* }}&	{\scriptsize {-0.189  }}&	{\scriptsize { 0.984* }}&	{\scriptsize{ 3.251* }}	\\
{\scriptsize {SP1FAAR vs. SP1Ls}}&	{\scriptsize {-0.279  }}&	{\scriptsize { 1.237* }}&	{\scriptsize {-2.469  }}&	{\scriptsize {-1.624  }}&	{\scriptsize {-1.855  }}&	{\scriptsize {-0.705  }}&	{\scriptsize {-1.236  }}&	{\scriptsize {-0.726  }}&	{\scriptsize {-3.463  }}&	{\scriptsize {-1.980  }}&	{\scriptsize{ 7.989* }}	\\
{\scriptsize {SP1FAAR vs. SP2s}}&	{\scriptsize {-0.006  }}&	{\scriptsize { 0.246* }}&	{\scriptsize {-0.934  }}&	{\scriptsize {-2.390  }}&	{\scriptsize {-1.591  }}&	{\scriptsize { 0.173* }}&	{\scriptsize {-0.126  }}&	{\scriptsize {-0.465  }}&	{\scriptsize { 0.857* }}&	{\scriptsize {-0.093  }}&	{\scriptsize{ 0.673* }}	\\
{\scriptsize {SP1FAAR vs. SP3s}}&	{\scriptsize {-0.234  }}&	{\scriptsize {-1.495  }}&	{\scriptsize { 0.937* }}&	{\scriptsize {-0.053  }}&	{\scriptsize { 0.780**}}&	{\scriptsize { 0.054* }}&	{\scriptsize { 0.040**}}&	{\scriptsize {-0.170  }}&	{\scriptsize { 0.065**}}&	{\scriptsize { 1.101* }}&	{\scriptsize{ 5.786* }}	\\
{\scriptsize {SP1Best vs. SP1s}}&	{\scriptsize {-0.103  }}&	{\scriptsize {-2.061  }}&	{\scriptsize {-0.046  }}&	{\scriptsize {-0.425  }}&	{\scriptsize {-1.993  }}&	{\scriptsize {-0.031  }}&	{\scriptsize {-0.010  }}&	{\scriptsize {-0.001  }}&	{\scriptsize {-0.189  }}&	{\scriptsize {-0.069  }}&	{\scriptsize{-1.419  }}	\\
{\scriptsize {SP1Best vs. SP1Ls}}&	{\scriptsize {-0.279  }}&	{\scriptsize {-0.082  }}&	{\scriptsize {-3.595  }}&	{\scriptsize {-2.049  }}&	{\scriptsize {-0.385  }}&	{\scriptsize {-0.862  }}&	{\scriptsize {-0.130  }}&	{\scriptsize {-0.733  }}&	{\scriptsize {-0.346  }}&	{\scriptsize {-3.033  }}&	{\scriptsize{ 3.319* }}	\\
{\scriptsize {SP1Best vs. SP2s}}&	{\scriptsize {-0.006  }}&	{\scriptsize {-1.814  }}&	{\scriptsize {-2.060  }}&	{\scriptsize {-2.815  }}&	{\scriptsize {-3.584  }}&	{\scriptsize { 0.016* }}&	{\scriptsize {-0.187  }}&	{\scriptsize {-0.472  }}&	{\scriptsize { 0.857* }}&	{\scriptsize {-1.146  }}&	{\scriptsize{-3.998  }}	\\
{\scriptsize {SP1Best vs. SP3s}}&	{\scriptsize {-0.234  }}&	{\scriptsize {-3.555  }}&	{\scriptsize {-0.188  }}&	{\scriptsize {-0.478  }}&	{\scriptsize {-1.213  }}&	{\scriptsize {-0.102  }}&	{\scriptsize {-0.020  }}&	{\scriptsize {-0.178  }}&	{\scriptsize { 0.065* }}&	{\scriptsize { 0.048* }}&	{\scriptsize{ 1.116* }}	\\
{\scriptsize {SP1LBest vs. SP1Ls}}&	{\scriptsize {-0.352  }}&	{\scriptsize {-0.193  }}&	{\scriptsize {-0.096  }}&	{\scriptsize {-0.375  }}&	{\scriptsize {-1.095  }}&	{\scriptsize {-1.323  }}&	{\scriptsize {-0.089  }}&	{\scriptsize {-0.368  }}&	{\scriptsize {-1.090  }}&	{\scriptsize {-0.208  }}&	{\scriptsize{-1.404  }}	\\
{\scriptsize {SP2Best vs. SP2s}}&	{\scriptsize {0.000  }}&	{\scriptsize {-0.003  }}&	{\scriptsize {-0.096  }}&	{\scriptsize {-0.375  }}&	{\scriptsize {-0.136  }}&	{\scriptsize {-0.016  }}&	{\scriptsize {-0.001  }}&	{\scriptsize {-0.284  }}&	{\scriptsize {-0.172  }}&	{\scriptsize {-0.208  }}&	{\scriptsize{-5.982  }}	\\
{\scriptsize {SP3Best vs. SP3s}}&	{\scriptsize {-0.081  }}&	{\scriptsize {-0.119  }}&	{\scriptsize {-0.520  }}&	{\scriptsize {-0.496  }}&	{\scriptsize {-3.236  }}&	{\scriptsize {-0.102  }}&	{\scriptsize {-0.009  }}&	{\scriptsize {-0.040  }}&	{\scriptsize {-0.775  }}&	{\scriptsize {-0.009  }}&	{\scriptsize{-0.750  }}	\\
{\scriptsize {SP3Best vs. SP1s}}&	{\scriptsize { 0.234  }}&	{\scriptsize { 0.344  }}&	{\scriptsize { 0.188  }}&	{\scriptsize { 0.478  }}&	{\scriptsize { 0.121  }}&	{\scriptsize {-0.031  }}&	{\scriptsize { 0.001  }}&	{\scriptsize { 0.178  }}&	{\scriptsize {-0.006  }}&	{\scriptsize {-0.117  }}&	{\scriptsize{-2.535  }}	\\
{\scriptsize {SP3Best vs. SP1Ls}}&	{\scriptsize {-0.045  }}&	{\scriptsize { 0.261  }}&	{\scriptsize {-3.406  }}&	{\scriptsize {-1.571  }}&	{\scriptsize {-0.263  }}&	{\scriptsize {-0.862  }}&	{\scriptsize {-0.129  }}&	{\scriptsize {-0.555  }}&	{\scriptsize {-0.353  }}&	{\scriptsize {-3.081  }}&	{\scriptsize{ 2.203  }}	\\
{\scriptsize {SP3Best vs. SP2s}}&	{\scriptsize { 0.227  }}&	{\scriptsize { 0.162  }}&	{\scriptsize {-1.872  }}&	{\scriptsize {-2.337  }}&	{\scriptsize {-0.237  }}&	{\scriptsize { 0.016  }}&	{\scriptsize {-0.018  }}&	{\scriptsize {-0.295  }}&	{\scriptsize { 0.079  }}&	{\scriptsize {-1.194  }}&	{\scriptsize{-5.113  }}	\\ \bottomrule
%
\pagebreak
%\multicolumn{12}{c}{} \\
\multicolumn{12}{c}{{\footnotesize {Panel B: Recursive Estimation, $h=3$}}} \\ 
\multicolumn{12}{c}{} \\
\toprule \multicolumn{1}{c}{{\scriptsize {Models}}} & \multicolumn{1}{c}{{\scriptsize {UR}}} & \multicolumn{1}{c}{{\scriptsize {PI}}} & \multicolumn{1}{c}{{\scriptsize {TB}}} & \multicolumn{1}{c}{{\scriptsize {CPI}}} & \multicolumn{1}{c}{{\scriptsize {PPI}}} & \multicolumn{1}{c}{{\scriptsize {NPE}}} & \multicolumn{1}{c}{{\scriptsize {HS}}}& \multicolumn{1}{c}{{\scriptsize {IPX}}} & \multicolumn{1}{c}{{\scriptsize {M2}}} & \multicolumn{1}{c}{{\scriptsize {SNP}}} & \multicolumn{1}{c}{{\scriptsize {GDP}}} \\ \midrule
%
{\scriptsize {AR vs. SP1s}}&	{\scriptsize {  0.092*  }}&	{\scriptsize {  1.723**}}&	{\scriptsize {  0.241*  }}&	{\scriptsize {  0.519*  }}&	{\scriptsize {  4.492*  }}&	{\scriptsize {  0.284*  }}&	{\scriptsize {  0.139*  }}&	{\scriptsize {  0.293*  }}&	{\scriptsize {  0.240*  }}&	{\scriptsize {  0.221**}}&	{\scriptsize {  0.059*  }}	\\
{\scriptsize {AR vs. SP1FAARs}}&	{\scriptsize {  0.092*  }}&	{\scriptsize {  1.338**}}&	{\scriptsize {  0.241*  }}&	{\scriptsize {  0.519*  }}&	{\scriptsize {  4.492*  }}&	{\scriptsize {  0.017*  }}&	{\scriptsize {  0.139*  }}&	{\scriptsize {  0.293*  }}&	{\scriptsize {  0.240*  }}&	{\scriptsize {  0.221**}}&	{\scriptsize {  0.059*  }}	\\
{\scriptsize {AR vs. SP1LFAARs}}&	{\scriptsize {  0.023*  }}&	{\scriptsize {  1.723**}}&	{\scriptsize {  0.404**}}&	{\scriptsize {  3.072*  }}&	{\scriptsize {  8.413*  }}&	{\scriptsize {  0.284*  }}&	{\scriptsize {  0.030*  }}&	{\scriptsize {  0.241**}}&	{\scriptsize {  0.345*  }}&	{\scriptsize {-0.340    }}&	{\scriptsize {-0.539    }}	\\
{\scriptsize {AR vs. SP2s}}&	{\scriptsize {  0.286*  }}&	{\scriptsize {  2.075**}}&	{\scriptsize {  0.301**}}&	{\scriptsize {  1.432**}}&	{\scriptsize {  5.018*  }}&	{\scriptsize {  0.237*  }}&	{\scriptsize {-0.423    }}&	{\scriptsize {  0.410**}}&	{\scriptsize {  0.604*  }}&	{\scriptsize {-0.290    }}&	{\scriptsize {-0.980    }}	\\
{\scriptsize {AR vs. SP3s}}&	{\scriptsize {  0.063*  }}&	{\scriptsize {  3.273**}}&	{\scriptsize {  0.009*  }}&	{\scriptsize {  1.237**}}&	{\scriptsize {  1.685*  }}&	{\scriptsize {  0.288**}}&	{\scriptsize {  0.096*  }}&	{\scriptsize {  0.417**}}&	{\scriptsize {  2.432**}}&	{\scriptsize {  0.111**}}&	{\scriptsize {  0.037*  }}	\\
{\scriptsize {SP1FAAR vs. SP1Shrinks}}&	{\scriptsize {  1.407*  }}&	{\scriptsize {  0.415*  }}&	{\scriptsize {  0.546*  }}&	{\scriptsize {  0.310*  }}&	{\scriptsize {  1.986*  }}&	{\scriptsize {  0.114*  }}&	{\scriptsize {  3.936*  }}&	{\scriptsize {  0.377*  }}&	{\scriptsize {  0.622*  }}&	{\scriptsize {  1.954*  }}&	{\scriptsize {  5.530*  }}	\\
{\scriptsize {SP1FAAR vs. SP1Ls}}&	{\scriptsize {  0.057*  }}&	{\scriptsize {  0.038*  }}&	{\scriptsize {  0.709*  }}&	{\scriptsize {  2.863*  }}&	{\scriptsize {  5.907*  }}&	{\scriptsize {-0.700    }}&	{\scriptsize {-1.188    }}&	{\scriptsize {-0.184    }}&	{\scriptsize {  0.727*  }}&	{\scriptsize {  1.252*  }}&	{\scriptsize {  4.175*  }}	\\
{\scriptsize {SP1FAAR vs. SP2s}}&	{\scriptsize {  0.335*  }}&	{\scriptsize {  1.151*  }}&	{\scriptsize {  0.606*  }}&	{\scriptsize {  1.222*  }}&	{\scriptsize {  2.512*  }}&	{\scriptsize {  0.231**}}&	{\scriptsize {-0.168    }}&	{\scriptsize {  0.493*  }}&	{\scriptsize {  0.986**}}&	{\scriptsize {  1.442*  }}&	{\scriptsize {  4.491**}}	\\
{\scriptsize {SP1FAAR vs. SP3s}}&	{\scriptsize {  0.111*  }}&	{\scriptsize {  2.350*  }}&	{\scriptsize {  0.314*  }}&	{\scriptsize {  1.027*  }}&	{\scriptsize {-0.821    }}&	{\scriptsize {  0.282**}}&	{\scriptsize {  0.351*  }}&	{\scriptsize {  0.501*  }}&	{\scriptsize {  2.813**}}&	{\scriptsize {  1.844*  }}&	{\scriptsize {  5.508*  }}	\\
{\scriptsize {SP1Best vs. SP1s}}&	{\scriptsize {-0.070    }}&	{\scriptsize {-0.385    }}&	{\scriptsize {-0.055    }}&	{\scriptsize {-0.168    }}&	{\scriptsize {-0.889    }}&	{\scriptsize {-0.267    }}&	{\scriptsize {-0.109    }}&	{\scriptsize {-0.052    }}&	{\scriptsize {-0.210    }}&	{\scriptsize {-0.221    }}&	{\scriptsize {-0.059    }}	\\
{\scriptsize {SP1Best vs. SP1Ls}}&	{\scriptsize {-0.084    }}&	{\scriptsize {-0.076    }}&	{\scriptsize {  0.164*  }}&	{\scriptsize {  2.553*  }}&	{\scriptsize {  0.392*  }}&	{\scriptsize {-0.979    }}&	{\scriptsize {-0.158    }}&	{\scriptsize {-0.561    }}&	{\scriptsize {  0.011*  }}&	{\scriptsize {-0.702    }}&	{\scriptsize {-1.355    }}	\\
{\scriptsize {SP1Best vs. SP2s}}&	{\scriptsize {  0.194*  }}&	{\scriptsize {  0.352*  }}&	{\scriptsize {  0.060*  }}&	{\scriptsize {  0.913*  }}&	{\scriptsize {  0.526*  }}&	{\scriptsize {-0.047    }}&	{\scriptsize {-0.562    }}&	{\scriptsize {  0.116*  }}&	{\scriptsize {  0.364*  }}&	{\scriptsize {-0.512    }}&	{\scriptsize {-1.039    }}	\\
{\scriptsize {SP1Best vs. SP3s}}&	{\scriptsize {-0.030    }}&	{\scriptsize {  1.550*  }}&	{\scriptsize {-0.232    }}&	{\scriptsize {  0.717*  }}&	{\scriptsize {-2.807    }}&	{\scriptsize {  0.004*  }}&	{\scriptsize {-0.043    }}&	{\scriptsize {  0.124*  }}&	{\scriptsize {  2.192*  }}&	{\scriptsize {-0.110    }}&	{\scriptsize {-0.022    }}	\\
{\scriptsize {SP1LBest vs. SP1Ls}}&	{\scriptsize {-0.140    }}&	{\scriptsize {-0.761    }}&	{\scriptsize {-0.061    }}&	{\scriptsize {-0.023    }}&	{\scriptsize {-0.991    }}&	{\scriptsize {-2.838    }}&	{\scriptsize {-0.303    }}&	{\scriptsize {-0.241    }}&	{\scriptsize {-0.134    }}&	{\scriptsize {-0.340    }}&	{\scriptsize {-0.539    }}	\\
{\scriptsize {SP2Best vs. SP2s}}&	{\scriptsize {-0.008    }}&	{\scriptsize {-0.035    }}&	{\scriptsize {-0.115    }}&	{\scriptsize {-0.437    }}&	{\scriptsize {-0.142    }}&	{\scriptsize {-0.047    }}&	{\scriptsize {-0.003    }}&	{\scriptsize {-0.169    }}&	{\scriptsize {-0.060    }}&	{\scriptsize {-0.340    }}&	{\scriptsize {-0.539    }}	\\
{\scriptsize {SP3Best vs. SP3s}}&	{\scriptsize {-0.040    }}&	{\scriptsize {-0.892    }}&	{\scriptsize {-0.177    }}&	{\scriptsize {-0.885    }}&	{\scriptsize {-1.918    }}&	{\scriptsize {-0.004    }}&	{\scriptsize {-0.066    }}&	{\scriptsize {-0.147    }}&	{\scriptsize {-2.158    }}&	{\scriptsize {-0.088    }}&	{\scriptsize {-0.037    }}	\\
{\scriptsize {SP3Best vs. SP1s}}&	{\scriptsize {  0.030    }}&	{\scriptsize {-0.194    }}&	{\scriptsize {  0.055    }}&	{\scriptsize {-0.717    }}&	{\scriptsize {  0.089    }}&	{\scriptsize {-0.271    }}&	{\scriptsize {  0.004    }}&	{\scriptsize {-0.124    }}&	{\scriptsize {-0.219    }}&	{\scriptsize {  0.110    }}&	{\scriptsize {  0.022    }}	\\
{\scriptsize {SP3Best vs. SP1Ls}}&	{\scriptsize {-0.054    }}&	{\scriptsize {-0.231    }}&	{\scriptsize {  0.219    }}&	{\scriptsize {  1.835    }}&	{\scriptsize {  0.481    }}&	{\scriptsize {-0.983    }}&	{\scriptsize {-0.154    }}&	{\scriptsize {-0.684    }}&	{\scriptsize {-0.209    }}&	{\scriptsize {-0.592    }}&	{\scriptsize {-1.333    }}	\\
{\scriptsize {SP3Best vs. SP2s}}&	{\scriptsize {  0.224    }}&	{\scriptsize {-0.120    }}&	{\scriptsize {  0.115    }}&	{\scriptsize {  0.195    }}&	{\scriptsize {  0.142    }}&	{\scriptsize {-0.051    }}&	{\scriptsize {-0.052    }}&	{\scriptsize {-0.007    }}&	{\scriptsize {-0.183    }}&	{\scriptsize {-0.402    }}&	{\scriptsize {-1.017    }}	\\ \bottomrule

\pagebreak
\multicolumn{12}{c}{{\footnotesize {Panel C: Recursive Estimation, $h=12$}}} \\ 
\multicolumn{12}{c}{} \\
\toprule \multicolumn{1}{c}{{\scriptsize {Models}}} & \multicolumn{1}{c}{{\scriptsize {UR}}} & \multicolumn{1}{c}{{\scriptsize {PI}}} & \multicolumn{1}{c}{{\scriptsize {TB}}} & \multicolumn{1}{c}{{\scriptsize {CPI}}} & \multicolumn{1}{c}{{\scriptsize {PPI}}} & \multicolumn{1}{c}{{\scriptsize {NPE}}} & \multicolumn{1}{c}{{\scriptsize {HS}}}& \multicolumn{1}{c}{{\scriptsize {IPX}}} & \multicolumn{1}{c}{{\scriptsize {M2}}} & \multicolumn{1}{c}{{\scriptsize {SNP}}} & \multicolumn{1}{c}{{\scriptsize {GDP}}} \\ \midrule
%
{\scriptsize {AR vs. SP1s}}&	{\scriptsize {  0.114*  }}&	{\scriptsize {-0.159    }}&	{\scriptsize {  0.344*  }}&	{\scriptsize {  0.663*  }}&	{\scriptsize {  1.242**}}&	{\scriptsize {  1.135*  }}&	{\scriptsize {  1.090*  }}&	{\scriptsize {  0.185**}}&	{\scriptsize {  1.646**}}&	{\scriptsize {  0.040*  }}&	{\scriptsize {  0.236*  }}	\\
{\scriptsize {AR vs. SP1FAARs}}&	{\scriptsize {  0.114*  }}&	{\scriptsize {-0.159    }}&	{\scriptsize {  0.344*  }}&	{\scriptsize {  0.663*  }}&	{\scriptsize {  1.242**}}&	{\scriptsize {  0.024*  }}&	{\scriptsize {  1.090*  }}&	{\scriptsize {  0.015*  }}&	{\scriptsize {  0.956**}}&	{\scriptsize {  0.040*  }}&	{\scriptsize {  0.236*  }}	\\
{\scriptsize {AR vs. SP1LFAARs}}&	{\scriptsize {  0.163*  }}&	{\scriptsize {  0.354*  }}&	{\scriptsize {  0.217**}}&	{\scriptsize {  1.641**}}&	{\scriptsize {  2.325**}}&	{\scriptsize {  1.135**}}&	{\scriptsize {  0.469*  }}&	{\scriptsize {  0.185**}}&	{\scriptsize {  1.646*  }}&	{\scriptsize {-0.078    }}&	{\scriptsize {  0.606*  }}	\\
{\scriptsize {AR vs. SP2s}}&	{\scriptsize {  0.215*  }}&	{\scriptsize {  1.360*  }}&	{\scriptsize {  0.142*  }}&	{\scriptsize {  1.072**}}&	{\scriptsize {  0.906*  }}&	{\scriptsize {  0.915*  }}&	{\scriptsize {  0.202*  }}&	{\scriptsize {  0.324*  }}&	{\scriptsize {  2.350**}}&	{\scriptsize {-0.007    }}&	{\scriptsize {  1.044*  }}	\\
{\scriptsize {AR vs. SP3s}}&	{\scriptsize {  0.183*  }}&	{\scriptsize {  1.378**}}&	{\scriptsize {  0.007*  }}&	{\scriptsize {  1.423**}}&	{\scriptsize {  2.383**}}&	{\scriptsize {  0.825*  }}&	{\scriptsize {  0.704*  }}&	{\scriptsize {  0.298**}}&	{\scriptsize {  3.192**}}&	{\scriptsize {  0.241*  }}&	{\scriptsize {  0.367*  }}	\\
{\scriptsize {SP1FAAR vs. SP1Shrinks}}&	{\scriptsize {  0.334*  }}&	{\scriptsize {  2.923*  }}&	{\scriptsize {  0.247*  }}&	{\scriptsize {  0.319*  }}&	{\scriptsize {  1.240*  }}&	{\scriptsize {  4.014*  }}&	{\scriptsize {  7.455*  }}&	{\scriptsize {  0.711*  }}&	{\scriptsize {  0.419*  }}&	{\scriptsize {  0.480*  }}&	{\scriptsize {  1.436*  }}	\\
{\scriptsize {SP1FAAR vs. SP1Ls}}&	{\scriptsize {  0.082*  }}&	{\scriptsize {  3.436*  }}&	{\scriptsize {-0.183    }}&	{\scriptsize {  1.298*  }}&	{\scriptsize {  2.324*  }}&	{\scriptsize {  0.501*  }}&	{\scriptsize {-0.703    }}&	{\scriptsize {  0.671*  }}&	{\scriptsize {  0.813*  }}&	{\scriptsize {  0.362*  }}&	{\scriptsize {  1.805*  }}	\\
{\scriptsize {SP1FAAR vs. SP2s}}&	{\scriptsize {  0.134*  }}&	{\scriptsize {  4.442*  }}&	{\scriptsize {  0.046*  }}&	{\scriptsize {  0.729*  }}&	{\scriptsize {  0.905*  }}&	{\scriptsize {  1.292*  }}&	{\scriptsize {-0.143    }}&	{\scriptsize {  1.020*  }}&	{\scriptsize {  1.813*  }}&	{\scriptsize {  0.433*  }}&	{\scriptsize {  2.243*  }}	\\
{\scriptsize {SP1FAAR vs. SP3s}}&	{\scriptsize {  0.102*  }}&	{\scriptsize {  4.460*  }}&	{\scriptsize {-0.089    }}&	{\scriptsize {  1.079*  }}&	{\scriptsize {  2.382*  }}&	{\scriptsize {  1.202*  }}&	{\scriptsize {  0.359*  }}&	{\scriptsize {  0.993*  }}&	{\scriptsize {  2.655*  }}&	{\scriptsize {  0.680*  }}&	{\scriptsize {  1.566*  }}	\\
{\scriptsize {SP1Best vs. SP1s}}&	{\scriptsize {-0.003    }}&	{\scriptsize {-0.159    }}&	{\scriptsize {-0.126    }}&	{\scriptsize {-0.135    }}&	{\scriptsize {-0.516    }}&	{\scriptsize {-1.094    }}&	{\scriptsize {-0.621    }}&	{\scriptsize {-0.170    }}&	{\scriptsize {-0.690    }}&	{\scriptsize {-0.021    }}&	{\scriptsize {-0.196    }}	\\
{\scriptsize {SP1Best vs. SP1Ls}}&	{\scriptsize {  0.049*  }}&	{\scriptsize {  0.035*  }}&	{\scriptsize {-0.431    }}&	{\scriptsize {  0.979*  }}&	{\scriptsize {  0.108*  }}&	{\scriptsize {-1.011    }}&	{\scriptsize {-0.145    }}&	{\scriptsize {-0.209    }}&	{\scriptsize {-0.030    }}&	{\scriptsize {-0.118    }}&	{\scriptsize {  0.369*  }}	\\
{\scriptsize {SP1Best vs. SP2s}}&	{\scriptsize {  0.101*  }}&	{\scriptsize {  1.360*  }}&	{\scriptsize {-0.202    }}&	{\scriptsize {  0.410*  }}&	{\scriptsize {-0.335    }}&	{\scriptsize {-0.220    }}&	{\scriptsize {-0.888    }}&	{\scriptsize {  0.139*  }}&	{\scriptsize {  0.705*  }}&	{\scriptsize {-0.047    }}&	{\scriptsize {  0.808*  }}	\\
{\scriptsize {SP1Best vs. SP3s}}&	{\scriptsize {  0.069*  }}&	{\scriptsize {  1.378*  }}&	{\scriptsize {-0.336    }}&	{\scriptsize {  0.760*  }}&	{\scriptsize {  1.141*  }}&	{\scriptsize {-0.309    }}&	{\scriptsize {-0.386    }}&	{\scriptsize {  0.112*  }}&	{\scriptsize {  1.546*  }}&	{\scriptsize {  0.200*  }}&	{\scriptsize {  0.131*  }}	\\
{\scriptsize {SP1LBest vs. SP1Ls}}&	{\scriptsize {-0.378    }}&	{\scriptsize {-0.354    }}&	{\scriptsize {-0.217    }}&	{\scriptsize {-0.560    }}&	{\scriptsize {-1.448    }}&	{\scriptsize {-10.113    }}&	{\scriptsize {-2.209    }}&	{\scriptsize {-0.185    }}&	{\scriptsize {-0.296    }}&	{\scriptsize {-0.078    }}&	{\scriptsize {-0.095    }}	\\
{\scriptsize {SP2Best vs. SP2s}}&	{\scriptsize {-0.005    }}&	{\scriptsize {-0.136    }}&	{\scriptsize {-0.217    }}&	{\scriptsize {-0.842    }}&	{\scriptsize {-0.091    }}&	{\scriptsize {-0.662    }}&	{\scriptsize {-0.022    }}&	{\scriptsize {-0.139    }}&	{\scriptsize {-0.070    }}&	{\scriptsize {-0.080    }}&	{\scriptsize {-0.508    }}	\\
{\scriptsize {SP3Best vs. SP3s}}&	{\scriptsize {-0.057    }}&	{\scriptsize {-0.613    }}&	{\scriptsize {-0.210    }}&	{\scriptsize {-0.744    }}&	{\scriptsize {-2.084    }}&	{\scriptsize {-0.309    }}&	{\scriptsize {-0.203    }}&	{\scriptsize {-0.112    }}&	{\scriptsize {-1.546    }}&	{\scriptsize {-0.046    }}&	{\scriptsize {-0.303    }}	\\
{\scriptsize {SP3Best vs. SP1s}}&	{\scriptsize {-0.069    }}&	{\scriptsize {-0.154    }}&	{\scriptsize {  0.126    }}&	{\scriptsize {-0.760    }}&	{\scriptsize {-0.114    }}&	{\scriptsize {-1.111    }}&	{\scriptsize {  0.039    }}&	{\scriptsize {-0.282    }}&	{\scriptsize {-0.224    }}&	{\scriptsize {-0.200    }}&	{\scriptsize {-0.131    }}	\\
{\scriptsize {SP3Best vs. SP1Ls}}&	{\scriptsize {-0.020    }}&	{\scriptsize {-0.102    }}&	{\scriptsize {-0.305    }}&	{\scriptsize {  0.218    }}&	{\scriptsize {-0.006    }}&	{\scriptsize {-1.011    }}&	{\scriptsize {-0.106    }}&	{\scriptsize {-0.322    }}&	{\scriptsize {-0.184    }}&	{\scriptsize {-0.319    }}&	{\scriptsize {  0.239    }}	\\
{\scriptsize {SP3Best vs. SP2s}}&	{\scriptsize {  0.032    }}&	{\scriptsize {-0.002    }}&	{\scriptsize {-0.075    }}&	{\scriptsize {-0.351    }}&	{\scriptsize {-0.148    }}&	{\scriptsize {-0.220    }}&	{\scriptsize {-0.050    }}&	{\scriptsize {  0.027    }}&	{\scriptsize {-0.084    }}&	{\scriptsize {-0.248    }}&	{\scriptsize {  0.677    }}	\\ \bottomrule

%\end{tabular}%
%} 
\multicolumn{12}{c}{} \\
\begin{minipage}{1.1\columnwidth} \scriptsize * Notes: See notes to Tables 1, 2 and 7. 
Numerical entries are reality check statistics due to White (2000) and Corradi and Swanson (2007), 
calculated for the forecast period 1984:6-2009:05, and summarize various results from Tables 3-6. 
The null hypothesis of the statistic is that no alternative model ``outperforms'' the benchmark model. 
Starred entries denote rejection of the null hypothesis at 5\% (**) and 10\% (*) siginificance levels. See Section 6 for further details. 
The first entry under the header ``Model'' denotes the benchmark model, and the second entry denotes the group of alternative 
models against which the benchmark is compared. Here, AR denotes our autoregressive benchmark model, 
SP1s denotes all SP1  models, excluding the benchmark, SP2s and SP3s are analogously defined. A mnemonic with ``Best'' appended onto it denotes the point MSFE ``best'' model for a particular
specification type. The notion ``FAARs'' and "Shrinks'' refers to all pure factor models, and all pure shrinkage (i.e., Bagging, Boosting, Ridge, LAR, EN, and NNG) models, respectively.
Finally, SP1FAAR, when used as a benchmark, refer to the FAAR model estimation under SP1. 
\end{minipage}
\end{longtable}

%TABLE 11 Percentage Difference  %%%%%%%%%%%%%%%%%%%%%%%%%%%%%%%%%%%%%%%%%%%%%%%%%%%%%%%%%%%%%%%%%%%%%%%%%%%%%%%%%%%%%%%%%%%%%%
\newpage 
%\begin{longtable}{y{2cm}y{.9cm}y{.9cm}y{.9cm}y{.9cm}y{.9cm}y{.9cm}y{.9cm}y{.9cm}y{.9cm}y{.9cm}y{.9cm}}
\LTcapwidth=\textwidth
\begin{longtable}{y{3cm}lllllllllll}
\caption{\normalsize {Predictive Accuracy Percentage Differences Based on the Comparison of Various Pairs of Forecasting Models*}}\\
\label{tb11}\\
%{\scriptsize 
%\begin{tabular}{lccccccccccc}
%\begin{tabular}{llllllllllll}
%\multicolumn{12}{c}{} \\
%\multicolumn{12}{c}{{\footnotesize {Panel A: Recursive, $h=1$}}} \\ \multicolumn{12}{c}{} \\
%\toprule \multicolumn{1}{c}{{\scriptsize {Cases}}} & \multicolumn{1}{c}{{\scriptsize {UR}}} & \multicolumn{1}{c}{{\scriptsize {PI}}} & \multicolumn{1}{c}{{\scriptsize {TB}}} & \multicolumn{1}{c}{{\scriptsize {CPI}}} & \multicolumn{1}{c}{{\scriptsize {PPI}}} & \multicolumn{1}{c}{{\scriptsize {NPE}}} & \multicolumn{1}{c}{{\scriptsize {HS}}}& \multicolumn{1}{c}{{\scriptsize {IPX}}} & \multicolumn{1}{c}{{\scriptsize {M2}}} & \multicolumn{1}{c}{{\scriptsize {SNP}}} & \multicolumn{1}{c}{{\scriptsize {GDP}}} \\ \midrule
%\endfirsthead
%\multicolumn{12}{c}{} \\
%\toprule {\scriptsize {Cases}} & {\scriptsize {UR}} & {\scriptsize {PI}} & {\scriptsize {TB}} & {\scriptsize {CPI}} & {\scriptsize {PPI}} & {\scriptsize {NPE}} & {\scriptsize {HS}}& {\scriptsize {IPX}} & {\scriptsize {M2}} & {\scriptsize {SNP}} & {\scriptsize {GDP}} \\ \midrule
%\multicolumn{12}{c}{{\footnotesize {Continued from previous page}}} \\ 
%\endhead
%
%\multicolumn{12}{c}{{\footnotesize {Continued in next page}}} \\ 
%\endfoot
%
%\\
%\endlastfoot
%
\multicolumn{12}{c}{{\footnotesize {Panel A: Recursive Estimation, $h=1$}}} \\ \multicolumn{12}{c}{} \\
\toprule \multicolumn{1}{c}{{\scriptsize {Models}}} & \multicolumn{1}{c}{{\scriptsize {UR}}} & \multicolumn{1}{c}{{\scriptsize {PI}}} & \multicolumn{1}{c}{{\scriptsize {TB}}} & \multicolumn{1}{c}{{\scriptsize {CPI}}} & \multicolumn{1}{c}{{\scriptsize {PPI}}} & \multicolumn{1}{c}{{\scriptsize {NPE}}} & \multicolumn{1}{c}{{\scriptsize {HS}}}& \multicolumn{1}{c}{{\scriptsize {IPX}}} & \multicolumn{1}{c}{{\scriptsize {M2}}} & \multicolumn{1}{c}{{\scriptsize {SNP}}} & \multicolumn{1}{c}{{\scriptsize {GDP}}} \\ \midrule

\endfirsthead
{\scriptsize {AR vs. SP1Best}}&	{\scriptsize {11.07\%}}&	{\scriptsize {  0.10\%}}&	{\scriptsize {  4.68\%}}&	{\scriptsize {  1.44\%}}&	{\scriptsize {  4.23\%}}&	{\scriptsize {  0.02\%}}&	{\scriptsize {  1.25\%}}&	{\scriptsize {  0.14\%}}&	{\scriptsize {  0.06\%}}&	{\scriptsize {  0.00\%}}&	{\scriptsize {  0.23\%}}	\\
{\scriptsize {AR vs. SP1LBest}}&	{\scriptsize {  5.04\%}}&	{\scriptsize {  0.04\%}}&	{\scriptsize {  1.84\%}}&	{\scriptsize {  0.75\%}}&	{\scriptsize {  1.22\%}}&	{\scriptsize {  0.02\%}}&	{\scriptsize {  4.97\%}}&	{\scriptsize {  0.11\%}}&	{\scriptsize {  0.04\%}}&	{\scriptsize {  0.47\%}}&	{\scriptsize {  0.04\%}}	\\
{\scriptsize {AR vs. SP2Best}}&	{\scriptsize {  2.21\%}}&	{\scriptsize {  0.03\%}}&	{\scriptsize {  3.99\%}}&	{\scriptsize {  0.14\%}}&	{\scriptsize {  0.46\%}}&	{\scriptsize {  0.07\%}}&	{\scriptsize {  2.31\%}}&	{\scriptsize {  0.00\%}}&	{\scriptsize {  0.06\%}}&	{\scriptsize {  0.58\%}}&	{\scriptsize {  0.00\%}}	\\
{\scriptsize {AR vs. SP3Best}}&	{\scriptsize {  4.00\%}}&	{\scriptsize {  0.12\%}}&	{\scriptsize {  0.58\%}}&	{\scriptsize {  0.00\%}}&	{\scriptsize {  0.01\%}}&	{\scriptsize {  0.01\%}}&	{\scriptsize {  1.60\%}}&	{\scriptsize {  0.36\%}}&	{\scriptsize {  0.49\%}}&	{\scriptsize {  0.89\%}}&	{\scriptsize {  0.27\%}}	\\
{\scriptsize {AR vs. SP1Mean}}&	{\scriptsize {  5.36\%}}&	{\scriptsize {  0.03\%}}&	{\scriptsize {  4.68\%}}&	{\scriptsize {  0.78\%}}&	{\scriptsize {  1.20\%}}&	{\scriptsize {  0.02\%}}&	{\scriptsize {  1.25\%}}&	{\scriptsize {  0.09\%}}&	{\scriptsize {  0.03\%}}&	{\scriptsize {  0.34\%}}&	{\scriptsize {  0.06\%}}	\\
{\scriptsize {AR vs. SP1LMean}}&	{\scriptsize {  5.04\%}}&	{\scriptsize {  0.04\%}}&	{\scriptsize {  4.61\%}}&	{\scriptsize {  0.75\%}}&	{\scriptsize {  1.22\%}}&	{\scriptsize {  0.02\%}}&	{\scriptsize {  1.20\%}}&	{\scriptsize {  0.11\%}}&	{\scriptsize {  0.04\%}}&	{\scriptsize {  0.38\%}}&	{\scriptsize {  0.04\%}}	\\
{\scriptsize {AR vs. SP2Mean}}&	{\scriptsize {  1.70\%}}&	{\scriptsize {  0.13\%}}&	{\scriptsize {  3.99\%}}&	{\scriptsize {  0.14\%}}&	{\scriptsize {  0.46\%}}&	{\scriptsize {  0.07\%}}&	{\scriptsize {  7.16\%}}&	{\scriptsize {  0.14\%}}&	{\scriptsize {  0.22\%}}&	{\scriptsize {  0.51\%}}&	{\scriptsize {  0.14\%}}	\\
{\scriptsize {AR vs. SP3Mean}}&	{\scriptsize {  1.77\%}}&	{\scriptsize {  0.10\%}}&	{\scriptsize {  4.33\%}}&	{\scriptsize {  0.12\%}}&	{\scriptsize {  0.16\%}}&	{\scriptsize {  0.07\%}}&	{\scriptsize {  6.32\%}}&	{\scriptsize {  0.13\%}}&	{\scriptsize {  0.19\%}}&	{\scriptsize {  0.41\%}}&	{\scriptsize {  0.13\%}}	\\
{\scriptsize {SP1Mean vs. SP1Best}}&	{\scriptsize {  7.90\%}}&	{\scriptsize {  0.08\%}}&	{\scriptsize {  0.00\%}}&	{\scriptsize {  0.91\%}}&	{\scriptsize {  3.22\%}}&	{\scriptsize {  0.03\%}}&	{\scriptsize {  0.00\%}}&	{\scriptsize {  0.07\%}}&	{\scriptsize {  0.03\%}}&	{\scriptsize {  0.34\%}}&	{\scriptsize {  0.19\%}}	\\
{\scriptsize {SP1LMean vs. SP1LBest}}&	{\scriptsize {  0.00\%}}&	{\scriptsize {  0.00\%}}&	{\scriptsize {  3.36\%}}&	{\scriptsize {  0.00\%}}&	{\scriptsize {  0.00\%}}&	{\scriptsize {  0.03\%}}&	{\scriptsize {  4.58\%}}&	{\scriptsize {  0.00\%}}&	{\scriptsize {  0.05\%}}&	{\scriptsize {  0.49\%}}&	{\scriptsize {  0.00\%}}	\\
{\scriptsize {SP2Mean vs. SP2Best}}&	{\scriptsize {  2.71\%}}&	{\scriptsize {  0.13\%}}&	{\scriptsize {  0.00\%}}&	{\scriptsize {  0.00\%}}&	{\scriptsize {  0.00\%}}&	{\scriptsize {  0.00\%}}&	{\scriptsize {  8.64\%}}&	{\scriptsize {  0.14\%}}&	{\scriptsize {  0.23\%}}&	{\scriptsize {  0.53\%}}&	{\scriptsize {  0.14\%}}	\\
{\scriptsize {SP3Mean vs. SP3Best}}&	{\scriptsize {  3.63\%}}&	{\scriptsize {  0.12\%}}&	{\scriptsize {  4.22\%}}&	{\scriptsize {  0.12\%}}&	{\scriptsize {  0.17\%}}&	{\scriptsize {  0.07\%}}&	{\scriptsize {  6.70\%}}&	{\scriptsize {  0.32\%}}&	{\scriptsize {  0.30\%}}&	{\scriptsize {  0.81\%}}&	{\scriptsize {  0.22\%}}	\\
{\scriptsize {SP1Best vs. SP2Mean}}&	{\scriptsize {18.04\%}}&	{\scriptsize {  0.17\%}}&	{\scriptsize {15.44\%}}&	{\scriptsize {  6.26\%}}&	{\scriptsize {  6.00\%}}&	{\scriptsize {  0.09\%}}&	{\scriptsize {  8.98\%}}&	{\scriptsize {  0.21\%}}&	{\scriptsize {  0.27\%}}&	{\scriptsize {  1.18\%}}&	{\scriptsize {  0.30\%}}	\\
{\scriptsize {SP1Best vs. SP2Best}}&	{\scriptsize {13.66\%}}&	{\scriptsize {  0.12\%}}&	{\scriptsize {11.17\%}}&	{\scriptsize {  5.44\%}}&	{\scriptsize {  5.88\%}}&	{\scriptsize {  0.06\%}}&	{\scriptsize {  3.23\%}}&	{\scriptsize {  0.18\%}}&	{\scriptsize {  0.16\%}}&	{\scriptsize {  1.06\%}}&	{\scriptsize {  0.26\%}}	\\
{\scriptsize {SP1Best vs. SP3Mean}}&	{\scriptsize {17.45\%}}&	{\scriptsize {  0.14\%}}&	{\scriptsize {10.26\%}}&	{\scriptsize {  6.23\%}}&	{\scriptsize {  5.88\%}}&	{\scriptsize {  0.08\%}}&	{\scriptsize {  7.47\%}}&	{\scriptsize {  0.17\%}}&	{\scriptsize {  0.21\%}}&	{\scriptsize {  0.87\%}}&	{\scriptsize {  0.22\%}}	\\
{\scriptsize {SP1Best vs. SP3Best}}&	{\scriptsize {13.55\%}}&	{\scriptsize {  0.14\%}}&	{\scriptsize {  4.68\%}}&	{\scriptsize {  5.36\%}}&	{\scriptsize {  5.88\%}}&	{\scriptsize {  0.02\%}}&	{\scriptsize {  1.37\%}}&	{\scriptsize {  0.34\%}}&	{\scriptsize {  0.49\%}}&	{\scriptsize {  0.89\%}}&	{\scriptsize {  0.21\%}}	\\
{\scriptsize {SP2Best vs. SP3Mean}}&	{\scriptsize {  2.77\%}}&	{\scriptsize {  0.12\%}}&	{\scriptsize {11.53\%}}&	{\scriptsize {  0.13\%}}&	{\scriptsize {  0.38\%}}&	{\scriptsize {  0.08\%}}&	{\scriptsize {  6.46\%}}&	{\scriptsize {  0.15\%}}&	{\scriptsize {  0.22\%}}&	{\scriptsize {  0.98\%}}&	{\scriptsize {  0.17\%}}	\\
{\scriptsize {SP2Best vs. SP3Best}}&	{\scriptsize {  4.47\%}}&	{\scriptsize {  0.14\%}}&	{\scriptsize {10.22\%}}&	{\scriptsize {  0.14\%}}&	{\scriptsize {  0.28\%}}&	{\scriptsize {  0.07\%}}&	{\scriptsize {  3.48\%}}&	{\scriptsize {  0.35\%}}&	{\scriptsize {  0.49\%}}&	{\scriptsize {  1.38\%}}&	{\scriptsize {  0.28\%}}	\\ \bottomrule
%
%\pagebreak
\multicolumn{12}{c}{} \\
\multicolumn{12}{c}{} \\\multicolumn{12}{c}{{\footnotesize {Panel B: Recursive Estimation, $h=3$}}} \\ 
\multicolumn{12}{c}{} \\
\toprule \multicolumn{1}{c}{{\scriptsize {Models}}} & \multicolumn{1}{c}{{\scriptsize {UR}}} & \multicolumn{1}{c}{{\scriptsize {PI}}} & \multicolumn{1}{c}{{\scriptsize {TB}}} & \multicolumn{1}{c}{{\scriptsize {CPI}}} & \multicolumn{1}{c}{{\scriptsize {PPI}}} & \multicolumn{1}{c}{{\scriptsize {NPE}}} & \multicolumn{1}{c}{{\scriptsize {HS}}}& \multicolumn{1}{c}{{\scriptsize {IPX}}} & \multicolumn{1}{c}{{\scriptsize {M2}}} & \multicolumn{1}{c}{{\scriptsize {SNP}}} & \multicolumn{1}{c}{{\scriptsize {GDP}}} \\ \midrule
%
{\scriptsize {AR vs. SP1Best}}&	{\scriptsize {  8.56\%}}&	{\scriptsize {  0.12\%}}&	{\scriptsize {  3.97\%}}&	{\scriptsize {  0.73\%}}&	{\scriptsize {  3.94\%}}&	{\scriptsize {  0.03\%}}&	{\scriptsize {  1.80\%}}&	{\scriptsize {  0.11\%}}&	{\scriptsize {  0.03\%}}&	{\scriptsize {  0.18\%}}&	{\scriptsize {  0.02\%}}	\\
{\scriptsize {AR vs. SP1LBest}}&	{\scriptsize {10.27\%}}&	{\scriptsize {  0.10\%}}&	{\scriptsize {  0.76\%}}&	{\scriptsize {  1.52\%}}&	{\scriptsize {  2.07\%}}&	{\scriptsize {  0.03\%}}&	{\scriptsize {  4.14\%}}&	{\scriptsize {  0.13\%}}&	{\scriptsize {  0.04\%}}&	{\scriptsize {  0.07\%}}&	{\scriptsize {  0.00\%}}	\\
{\scriptsize {AR vs. SP2Best}}&	{\scriptsize {  2.73\%}}&	{\scriptsize {  0.07\%}}&	{\scriptsize {  2.27\%}}&	{\scriptsize {  0.15\%}}&	{\scriptsize {  0.04\%}}&	{\scriptsize {  0.05\%}}&	{\scriptsize {  6.54\%}}&	{\scriptsize {  0.10\%}}&	{\scriptsize {  0.07\%}}&	{\scriptsize {  0.00\%}}&	{\scriptsize {  0.00\%}}	\\
{\scriptsize {AR vs. SP3Best}}&	{\scriptsize {  3.52\%}}&	{\scriptsize {  0.07\%}}&	{\scriptsize {  4.05\%}}&	{\scriptsize {  0.01\%}}&	{\scriptsize {  0.66\%}}&	{\scriptsize {  0.16\%}}&	{\scriptsize {  0.28\%}}&	{\scriptsize {  0.23\%}}&	{\scriptsize {  0.53\%}}&	{\scriptsize {  0.62\%}}&	{\scriptsize {  0.00\%}}	\\
{\scriptsize {AR vs. SP1Mean}}&	{\scriptsize {  6.05\%}}&	{\scriptsize {  0.04\%}}&	{\scriptsize {  1.76\%}}&	{\scriptsize {  0.73\%}}&	{\scriptsize {  1.26\%}}&	{\scriptsize {  0.02\%}}&	{\scriptsize {  2.05\%}}&	{\scriptsize {  0.11\%}}&	{\scriptsize {  0.03\%}}&	{\scriptsize {  0.18\%}}&	{\scriptsize {  0.05\%}}	\\
{\scriptsize {AR vs. SP1LMean}}&	{\scriptsize {  4.88\%}}&	{\scriptsize {  0.04\%}}&	{\scriptsize {  3.34\%}}&	{\scriptsize {  0.73\%}}&	{\scriptsize {  2.07\%}}&	{\scriptsize {  0.02\%}}&	{\scriptsize {  2.00\%}}&	{\scriptsize {  0.09\%}}&	{\scriptsize {  0.04\%}}&	{\scriptsize {  0.22\%}}&	{\scriptsize {  0.05\%}}	\\
{\scriptsize {AR vs. SP2Mean}}&	{\scriptsize {  1.46\%}}&	{\scriptsize {  0.11\%}}&	{\scriptsize {  2.01\%}}&	{\scriptsize {  0.15\%}}&	{\scriptsize {  0.30\%}}&	{\scriptsize {  0.07\%}}&	{\scriptsize {  6.54\%}}&	{\scriptsize {  0.11\%}}&	{\scriptsize {  0.24\%}}&	{\scriptsize {  0.23\%}}&	{\scriptsize {  0.10\%}}	\\
{\scriptsize {AR vs. SP3Mean}}&	{\scriptsize {  1.77\%}}&	{\scriptsize {  0.11\%}}&	{\scriptsize {  2.73\%}}&	{\scriptsize {  0.13\%}}&	{\scriptsize {  0.34\%}}&	{\scriptsize {  0.06\%}}&	{\scriptsize {  5.72\%}}&	{\scriptsize {  0.11\%}}&	{\scriptsize {  0.22\%}}&	{\scriptsize {  0.27\%}}&	{\scriptsize {  0.10\%}}	\\
{\scriptsize {SP1Mean vs. SP1Best}}&	{\scriptsize {  6.22\%}}&	{\scriptsize {  0.09\%}}&	{\scriptsize {  3.98\%}}&	{\scriptsize {  0.00\%}}&	{\scriptsize {  2.79\%}}&	{\scriptsize {  0.05\%}}&	{\scriptsize {  2.43\%}}&	{\scriptsize {  0.00\%}}&	{\scriptsize {  0.00\%}}&	{\scriptsize {  0.19\%}}&	{\scriptsize {  0.03\%}}	\\
{\scriptsize {SP1LMean vs. SP1LBest}}&	{\scriptsize {  7.89\%}}&	{\scriptsize {  0.07\%}}&	{\scriptsize {  3.00\%}}&	{\scriptsize {  0.84\%}}&	{\scriptsize {  0.00\%}}&	{\scriptsize {  0.04\%}}&	{\scriptsize {  4.25\%}}&	{\scriptsize {  0.17\%}}&	{\scriptsize {  0.04\%}}&	{\scriptsize {  0.19\%}}&	{\scriptsize {  0.05\%}}	\\
{\scriptsize {SP2Mean vs. SP2Best}}&	{\scriptsize {  3.12\%}}&	{\scriptsize {  0.14\%}}&	{\scriptsize {  2.16\%}}&	{\scriptsize {  0.00\%}}&	{\scriptsize {  0.27\%}}&	{\scriptsize {  0.08\%}}&	{\scriptsize {  0.00\%}}&	{\scriptsize {  0.13\%}}&	{\scriptsize {  0.26\%}}&	{\scriptsize {  0.23\%}}&	{\scriptsize {  0.10\%}}	\\
{\scriptsize {SP3Mean vs. SP3Best}}&	{\scriptsize {  2.78\%}}&	{\scriptsize {  0.13\%}}&	{\scriptsize {  3.25\%}}&	{\scriptsize {  0.13\%}}&	{\scriptsize {  0.42\%}}&	{\scriptsize {  0.10\%}}&	{\scriptsize {  5.78\%}}&	{\scriptsize {  0.20\%}}&	{\scriptsize {  0.31\%}}&	{\scriptsize {  0.50\%}}&	{\scriptsize {  0.10\%}}	\\
{\scriptsize {SP1Best vs. SP2Mean}}&	{\scriptsize {  9.08\%}}&	{\scriptsize {  0.12\%}}&	{\scriptsize {  4.41\%}}&	{\scriptsize {  8.52\%}}&	{\scriptsize {  7.55\%}}&	{\scriptsize {  0.07\%}}&	{\scriptsize {  7.34\%}}&	{\scriptsize {  0.14\%}}&	{\scriptsize {  0.24\%}}&	{\scriptsize {  0.40\%}}&	{\scriptsize {  0.13\%}}	\\
{\scriptsize {SP1Best vs. SP2Best}}&	{\scriptsize {  9.24\%}}&	{\scriptsize {  0.11\%}}&	{\scriptsize {  6.84\%}}&	{\scriptsize {  8.23\%}}&	{\scriptsize {  4.84\%}}&	{\scriptsize {  0.05\%}}&	{\scriptsize {  7.62\%}}&	{\scriptsize {  0.18\%}}&	{\scriptsize {  0.09\%}}&	{\scriptsize {  0.45\%}}&	{\scriptsize {  0.11\%}}	\\
{\scriptsize {SP1Best vs. SP3Mean}}&	{\scriptsize {  8.12\%}}&	{\scriptsize {  0.11\%}}&	{\scriptsize {  3.10\%}}&	{\scriptsize {  8.50\%}}&	{\scriptsize {  7.22\%}}&	{\scriptsize {  0.06\%}}&	{\scriptsize {  5.72\%}}&	{\scriptsize {  0.11\%}}&	{\scriptsize {  0.23\%}}&	{\scriptsize {  0.31\%}}&	{\scriptsize {  0.11\%}}	\\
{\scriptsize {SP1Best vs. SP3Best}}&	{\scriptsize {11.42\%}}&	{\scriptsize {  0.08\%}}&	{\scriptsize {  5.41\%}}&	{\scriptsize {  8.12\%}}&	{\scriptsize {  4.96\%}}&	{\scriptsize {  0.14\%}}&	{\scriptsize {  1.80\%}}&	{\scriptsize {  0.20\%}}&	{\scriptsize {  0.54\%}}&	{\scriptsize {  0.70\%}}&	{\scriptsize {  0.02\%}}	\\
{\scriptsize {SP2Best vs. SP3Mean}}&	{\scriptsize {  2.82\%}}&	{\scriptsize {  0.11\%}}&	{\scriptsize {  4.90\%}}&	{\scriptsize {  0.13\%}}&	{\scriptsize {  0.29\%}}&	{\scriptsize {  0.06\%}}&	{\scriptsize {  6.37\%}}&	{\scriptsize {  0.14\%}}&	{\scriptsize {  0.23\%}}&	{\scriptsize {  0.44\%}}&	{\scriptsize {  0.13\%}}	\\
{\scriptsize {SP2Best vs. SP3Best}}&	{\scriptsize {  5.18\%}}&	{\scriptsize {  0.08\%}}&	{\scriptsize {  6.55\%}}&	{\scriptsize {  0.15\%}}&	{\scriptsize {  0.30\%}}&	{\scriptsize {  0.17\%}}&	{\scriptsize {  7.37\%}}&	{\scriptsize {  0.25\%}}&	{\scriptsize {  0.56\%}}&	{\scriptsize {  0.62\%}}&	{\scriptsize {  0.11\%}}	\\ \bottomrule
%
%\pagebreak
\multicolumn{12}{c}{} \\
\multicolumn{12}{c}{{\footnotesize {Panel C: Recursive Estimation, $h=12$}}} \\ 
\multicolumn{12}{c}{} \\
\toprule \multicolumn{1}{c}{{\scriptsize {Models}}} & \multicolumn{1}{c}{{\scriptsize {UR}}} & \multicolumn{1}{c}{{\scriptsize {PI}}} & \multicolumn{1}{c}{{\scriptsize {TB}}} & \multicolumn{1}{c}{{\scriptsize {CPI}}} & \multicolumn{1}{c}{{\scriptsize {PPI}}} & \multicolumn{1}{c}{{\scriptsize {NPE}}} & \multicolumn{1}{c}{{\scriptsize {HS}}}& \multicolumn{1}{c}{{\scriptsize {IPX}}} & \multicolumn{1}{c}{{\scriptsize {M2}}} & \multicolumn{1}{c}{{\scriptsize {SNP}}} & \multicolumn{1}{c}{{\scriptsize {GDP}}} \\ \midrule
%
{\scriptsize {AR vs. SP1Best}}&	{\scriptsize {  8.95\%}}&	{\scriptsize {  0.00\%}}&	{\scriptsize {  5.02\%}}&	{\scriptsize {  1.05\%}}&	{\scriptsize {  2.24\%}}&	{\scriptsize {  0.08\%}}&	{\scriptsize {  8.61\%}}&	{\scriptsize {  0.19\%}}&	{\scriptsize {  0.11\%}}&	{\scriptsize {  0.13\%}}&	{\scriptsize {  0.02\%}}	\\
{\scriptsize {AR vs. SP1LBest}}&	{\scriptsize {  7.59\%}}&	{\scriptsize {  0.04\%}}&	{\scriptsize {  0.86\%}}&	{\scriptsize {  0.63\%}}&	{\scriptsize {  2.67\%}}&	{\scriptsize {  0.08\%}}&	{\scriptsize {  4.75\%}}&	{\scriptsize {  0.19\%}}&	{\scriptsize {  0.19\%}}&	{\scriptsize {  0.00\%}}&	{\scriptsize {  0.03\%}}	\\
{\scriptsize {AR vs. SP2Best}}&	{\scriptsize {  3.70\%}}&	{\scriptsize {  0.01\%}}&	{\scriptsize {  0.00\%}}&	{\scriptsize {  0.00\%}}&	{\scriptsize {  0.02\%}}&	{\scriptsize {  0.06\%}}&	{\scriptsize {  3.39\%}}&	{\scriptsize {  0.12\%}}&	{\scriptsize {  0.00\%}}&	{\scriptsize {  0.00\%}}&	{\scriptsize {  0.05\%}}	\\
{\scriptsize {AR vs. SP3Best}}&	{\scriptsize {  1.67\%}}&	{\scriptsize {  0.19\%}}&	{\scriptsize {  3.75\%}}&	{\scriptsize {  0.17\%}}&	{\scriptsize {  0.66\%}}&	{\scriptsize {  0.18\%}}&	{\scriptsize {  3.85\%}}&	{\scriptsize {  0.23\%}}&	{\scriptsize {  0.56\%}}&	{\scriptsize {  0.74\%}}&	{\scriptsize {  0.00\%}}	\\
{\scriptsize {AR vs. SP1Mean}}&	{\scriptsize {  5.78\%}}&	{\scriptsize {  0.04\%}}&	{\scriptsize {  1.21\%}}&	{\scriptsize {  1.05\%}}&	{\scriptsize {  2.24\%}}&	{\scriptsize {  0.03\%}}&	{\scriptsize {  4.29\%}}&	{\scriptsize {  0.09\%}}&	{\scriptsize {  0.04\%}}&	{\scriptsize {  0.13\%}}&	{\scriptsize {  0.02\%}}	\\
{\scriptsize {AR vs. SP1LMean}}&	{\scriptsize {  5.94\%}}&	{\scriptsize {  0.04\%}}&	{\scriptsize {  1.92\%}}&	{\scriptsize {  0.63\%}}&	{\scriptsize {  2.67\%}}&	{\scriptsize {  0.03\%}}&	{\scriptsize {  4.75\%}}&	{\scriptsize {  0.09\%}}&	{\scriptsize {  0.05\%}}&	{\scriptsize {  0.24\%}}&	{\scriptsize {  0.03\%}}	\\
{\scriptsize {AR vs. SP2Mean}}&	{\scriptsize {  1.39\%}}&	{\scriptsize {  0.12\%}}&	{\scriptsize {  2.00\%}}&	{\scriptsize {  0.15\%}}&	{\scriptsize {  0.27\%}}&	{\scriptsize {  0.07\%}}&	{\scriptsize {  3.39\%}}&	{\scriptsize {  0.09\%}}&	{\scriptsize {  0.26\%}}&	{\scriptsize {  0.23\%}}&	{\scriptsize {  0.13\%}}	\\
{\scriptsize {AR vs. SP3Mean}}&	{\scriptsize {  1.83\%}}&	{\scriptsize {  0.11\%}}&	{\scriptsize {  2.86\%}}&	{\scriptsize {  0.11\%}}&	{\scriptsize {  0.12\%}}&	{\scriptsize {  0.05\%}}&	{\scriptsize {  3.78\%}}&	{\scriptsize {  0.10\%}}&	{\scriptsize {  0.23\%}}&	{\scriptsize {  0.32\%}}&	{\scriptsize {  0.11\%}}	\\
{\scriptsize {SP1Mean vs. SP1Best}}&	{\scriptsize {  5.45\%}}&	{\scriptsize {  0.04\%}}&	{\scriptsize {  4.39\%}}&	{\scriptsize {  0.00\%}}&	{\scriptsize {  0.00\%}}&	{\scriptsize {  0.10\%}}&	{\scriptsize {  5.58\%}}&	{\scriptsize {  0.25\%}}&	{\scriptsize {  0.08\%}}&	{\scriptsize {  0.00\%}}&	{\scriptsize {  0.00\%}}	\\
{\scriptsize {SP1LMean vs. SP1LBest}}&	{\scriptsize {  4.14\%}}&	{\scriptsize {  0.00\%}}&	{\scriptsize {  1.74\%}}&	{\scriptsize {  0.00\%}}&	{\scriptsize {  0.00\%}}&	{\scriptsize {  0.10\%}}&	{\scriptsize {  0.00\%}}&	{\scriptsize {  0.25\%}}&	{\scriptsize {  0.16\%}}&	{\scriptsize {  0.24\%}}&	{\scriptsize {  0.00\%}}	\\
{\scriptsize {SP2Mean vs. SP2Best}}&	{\scriptsize {  2.85\%}}&	{\scriptsize {  0.11\%}}&	{\scriptsize {  2.00\%}}&	{\scriptsize {  0.15\%}}&	{\scriptsize {  0.29\%}}&	{\scriptsize {  0.10\%}}&	{\scriptsize {  0.00\%}}&	{\scriptsize {  0.17\%}}&	{\scriptsize {  0.26\%}}&	{\scriptsize {  0.23\%}}&	{\scriptsize {  0.11\%}}	\\
{\scriptsize {SP3Mean vs. SP3Best}}&	{\scriptsize {  1.89\%}}&	{\scriptsize {  0.18\%}}&	{\scriptsize {  3.19\%}}&	{\scriptsize {  0.14\%}}&	{\scriptsize {  0.64\%}}&	{\scriptsize {  0.13\%}}&	{\scriptsize {  4.14\%}}&	{\scriptsize {  0.21\%}}&	{\scriptsize {  0.34\%}}&	{\scriptsize {  0.64\%}}&	{\scriptsize {  0.11\%}}	\\
{\scriptsize {SP1Best vs. SP2Mean}}&	{\scriptsize {  7.61\%}}&	{\scriptsize {  0.15\%}}&	{\scriptsize {  3.89\%}}&	{\scriptsize {  7.93\%}}&	{\scriptsize {  9.25\%}}&	{\scriptsize {  0.09\%}}&	{\scriptsize {  5.35\%}}&	{\scriptsize {  0.15\%}}&	{\scriptsize {  0.23\%}}&	{\scriptsize {  0.31\%}}&	{\scriptsize {  0.13\%}}	\\
{\scriptsize {SP1Best vs. SP2Best}}&	{\scriptsize {  8.70\%}}&	{\scriptsize {  0.06\%}}&	{\scriptsize {  4.95\%}}&	{\scriptsize {  7.86\%}}&	{\scriptsize {  9.12\%}}&	{\scriptsize {  0.11\%}}&	{\scriptsize {  9.82\%}}&	{\scriptsize {  0.27\%}}&	{\scriptsize {  0.11\%}}&	{\scriptsize {  0.26\%}}&	{\scriptsize {  0.07\%}}	\\
{\scriptsize {SP1Best vs. SP3Mean}}&	{\scriptsize {  6.80\%}}&	{\scriptsize {  0.14\%}}&	{\scriptsize {  2.79\%}}&	{\scriptsize {  7.89\%}}&	{\scriptsize {  9.13\%}}&	{\scriptsize {  0.07\%}}&	{\scriptsize {  3.56\%}}&	{\scriptsize {  0.15\%}}&	{\scriptsize {  0.19\%}}&	{\scriptsize {  0.31\%}}&	{\scriptsize {  0.10\%}}	\\
{\scriptsize {SP1Best vs. SP3Best}}&	{\scriptsize {10.63\%}}&	{\scriptsize {  0.19\%}}&	{\scriptsize {  3.24\%}}&	{\scriptsize {  7.94\%}}&	{\scriptsize {  9.58\%}}&	{\scriptsize {  0.10\%}}&	{\scriptsize {  6.44\%}}&	{\scriptsize {  0.30\%}}&	{\scriptsize {  0.51\%}}&	{\scriptsize {  0.69\%}}&	{\scriptsize {  0.02\%}}	\\
{\scriptsize {SP2Best vs. SP3Mean}}&	{\scriptsize {  3.18\%}}&	{\scriptsize {  0.15\%}}&	{\scriptsize {  4.42\%}}&	{\scriptsize {  0.10\%}}&	{\scriptsize {  0.53\%}}&	{\scriptsize {  0.08\%}}&	{\scriptsize {  5.56\%}}&	{\scriptsize {  0.17\%}}&	{\scriptsize {  0.21\%}}&	{\scriptsize {  0.38\%}}&	{\scriptsize {  0.09\%}}	\\
{\scriptsize {SP2Best vs. SP3Best}}&	{\scriptsize {  3.93\%}}&	{\scriptsize {  0.20\%}}&	{\scriptsize {  3.75\%}}&	{\scriptsize {  0.17\%}}&	{\scriptsize {  0.60\%}}&	{\scriptsize {  0.20\%}}&	{\scriptsize {  5.55\%}}&	{\scriptsize {  0.22\%}}&	{\scriptsize {  0.56\%}}&	{\scriptsize {  0.77\%}}&	{\scriptsize {  0.08\%}}	\\ \bottomrule
%
%\end{tabular}%
%} 
\multicolumn{12}{c}{} \\
\begin{minipage}{1\columnwidth} \scriptsize * Notes: See notes to Table 10. Mnemonics used under the header ``Models'' are defined in Table 10.
Numerical entries are average percentage differences when comparing actual and predicted values of a given target variable, defined as follows: 
$T^{-1} \sum_{t}\left\vert \left\vert \frac{\hat{Y}_{1,t}-Y_{t}}{Y_{t}}\right\vert
-\left\vert \frac{\hat{Y}_{2,t}-Y_{t}}{Y_{t}}\right\vert \right\vert \times
100\%$,
where $Y_{t}$ is the actual value of the target variable at time $t.$
Additionally, $\hat{Y}_{1,t}$ and $\hat{Y}_{2,t}$ are the forecasts from the
two models being compared. In rare occurences where $Y_{t}$ is zero, it is
replaced by $Y_{t-1}.$
\end{minipage}
\end{longtable}

%\newpage 

%\begin{equation*}
%\begin{figure}[tbp]
%\centering
%\FRAME{itbpFU}{6.2179in}{7.9267in}{0in}{\Qct{Figure 1: Most Frequently Selected
%Variables by Various Specification Types*}}{\Qlb{fig1}}{Figure}{\special%
%{language "Scientific Word";type "GRAPHIC";display "USEDEF";valid_file "T";width 2.2087in;height 3.3053in;depth
%0in;original-width 48.6111in;original-height 73.0559in;cropleft "0";croptop
%"1";cropright "1";cropbottom "0";tempfilename
%'MBJ2JM01.bmp';tempfile-properties "XPR";}}
%\end{equation*}
%\begin{minipage}{1\columnwidth} \scriptsize *Notes: Panels in this figure depict the 10 most commonly selected variables for use in factor construction, across the entire prediction period from 1984:6-2009:5, where factors are re-estimated at each point in time, prior to each new prediction being constructed. 45 degree lines denote cases for which a particular variables is selected every time.  All models reported on are MSFE-best models, across Specification Types 1 and 2, and estimation window types. For example, in Panels A and B, the BAA Bond Yield - Federal Funds Rate spread is the most frequently selected predictor when constructing factors to forecast the Producer Price Index and Housing Starts, respectively. Note that in Panel E, the 10 most commonly selected variables by EN are picked at every point in time. 
%\end{minipage}
%\end{figure}

%\begin{equation*}
\begin{figure}[tbp]
\centering
\FRAME{itbpF}{6.2179in}{7.9267in}{0in}{\Qct{Figure 1: Most Frequently
Selected Variables by Various Specification Types*}}{\Qlb{fig1}}{Figure}{%
\special{language "Scientific Word";type "GRAPHIC";maintain-aspect-ratio
TRUE;display "USEDEF";valid_file "T";width 6.2179in;height 7.9267in;depth
0in;original-width 48.6111in;original-height 73.0559in;cropleft "0";croptop
"1";cropright "1";cropbottom "0";tempfilename
'MERT7F00.gif';tempfile-properties "XPR";}} %\end{equation*}
\begin{minipage}{1\columnwidth} \scriptsize *Notes: Panels in this figure depict the 10 most commonly selected variables for use in factor construction, across the entire prediction period from 1984:6-2009:5, where factors are re-estimated at each point in time, prior to each new prediction being constructed. 45 degree lines denote cases for which a particular variables is selected every time.  All models reported on are MSFE-best models, across Specification Types 1 and 2, and estimation window types. For example, in Panels A and B, the BAA Bond Yield - Federal Funds Rate spread is the most frequently selected predictor when constructing factors to forecast the Producer Price Index and Housing Starts, respectively. Note that in Panel E, the 10 most commonly selected variables by EN are picked at every point in time. 
\end{minipage}
\end{figure}

\end{document}
