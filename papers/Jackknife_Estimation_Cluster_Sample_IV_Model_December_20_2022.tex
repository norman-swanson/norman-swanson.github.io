%2multibyte Version: 5.50.0.2960 CodePage: 936


\documentclass[11pt]{article}
%%%%%%%%%%%%%%%%%%%%%%%%%%%%%%%%%%%%%%%%%%%%%%%%%%%%%%%%%%%%%%%%%%%%%%%%%%%%%%%%%%%%%%%%%%%%%%%%%%%%%%%%%%%%%%%%%%%%%%%%%%%%%%%%%%%%%%%%%%%%%%%%%%%%%%%%%%%%%%%%%%%%%%%%%%%%%%%%%%%%%%%%%%%%%%%%%%%%%%%%%%%%%%%%%%%%%%%%%%%%%%%%%%%%%%%%%%%%%%%%%%%%%%%%%%%%
\usepackage{amssymb}
\usepackage{amsfonts}
\usepackage{amsmath}
\usepackage{geometry}
\usepackage[onehalfspacing]{setspace}
\usepackage{setspace}

\setcounter{MaxMatrixCols}{10}
%TCIDATA{OutputFilter=LATEX.DLL}
%TCIDATA{Version=5.00.0.2606}
%TCIDATA{Codepage=936}
%TCIDATA{<META NAME="SaveForMode" CONTENT="1">}
%TCIDATA{BibliographyScheme=Manual}
%TCIDATA{Created=Sunday, July 18, 2004 16:10:34}
%TCIDATA{LastRevised=Tuesday, December 20, 2022 19:26:29}
%TCIDATA{<META NAME="GraphicsSave" CONTENT="32">}
%TCIDATA{<META NAME="DocumentShell" CONTENT="General\Blank Document">}
%TCIDATA{Language=American English}
%TCIDATA{CSTFile=article.cst}
%TCIDATA{PageSetup=72,72,72,72,0}
%TCIDATA{AllPages=
%F=36,\PARA{038<p type="texpara" tag="Body Text" >\hfill \thepage}
%}


\newtheorem{theorem}{Theorem}
\newtheorem{acknowledgement}[theorem]{Acknowledgement}
\newtheorem{algorithm}[theorem]{Algorithm}
\newtheorem{axiom}[theorem]{Axiom}
\newtheorem{case}[theorem]{Case}
\newtheorem{claim}[theorem]{Claim}
\newtheorem{conclusion}[theorem]{Conclusion}
\newtheorem{condition}[theorem]{Condition}
\newtheorem{conjecture}[theorem]{Conjecture}
\newtheorem{corollary}[theorem]{Corollary}
\newtheorem{criterion}[theorem]{Criterion}
\newtheorem{definition}[theorem]{Definition}
\newtheorem{example}[theorem]{Example}
\newtheorem{exercise}[theorem]{Exercise}
\newtheorem{lemma}[theorem]{Lemma}
\newtheorem{notation}[theorem]{Notation}
\newtheorem{problem}[theorem]{Problem}
\newtheorem{proposition}[theorem]{Proposition}
\newtheorem{remark}[theorem]{Remark}
\newtheorem{solution}[theorem]{Solution}
\newtheorem{summary}[theorem]{Summary}
\newenvironment{proof}[1][Proof]{\noindent\textbf{#1.} }{\ \rule{0.5em}{0.5em}}
\renewcommand{\baselinestretch}{1.3} 
\textwidth=6.8in
\textheight=8.7in
\oddsidemargin=0in
\evensidemargin=0in
\topmargin=0in
\baselineskip=10pt
\linespread{1.3}
\input{tcilatex}
\geometry{left=1in,right=1in,top=1.25in,bottom=1.25in}

\begin{document}

\title{\textbf{Jackknife Estimation of a Cluster-Sample IV Regression Model
with Many Weak Instruments\thanks{%
Corresponding author: John C. Chao, Department of Economics, 7343 Preinkert
Drive, University of Maryland, chao@econ.umd.edu. Norman R. Swanson,
Department of Economics, 9500 Hamilton Street, Rutgers University,
nswanson@econ.rutgers.edu. Tiemen Woutersen, Department of Economics, 1130 E
Helen Street, University of Arizona, woutersen@arizona.edu. The authors
would like to thank the Editor, Xiaohong Chen, an Associate Editor, and an
anonymous referee for very helpful comments and suggestions. The authors
also owe special thanks to Jerry Hausman and Whitney Newey for many
discussions on the topic of this paper over a number of years. In addition,
thanks are owed to the particpants of the 2019 MIT Conference Honoring
Whitney Newey for comments on and advice given on an earlier version of this
work. Finally, the authors wish to thank Miriam Arden for excellent research
assistance. Chao thanks the University of Maryland for research support, and
Woutersen's work was supported by an Eller College of Management Research
Grant.}}}
\author{John C. Chao, Norman R. Swanson, and Tiemen Woutersen}
\date{\textit{First Draft}: September 5, 2020 \ \ \textit{This Version}:
December 20, 2022}
\maketitle

\begin{spacing}{1.0}
\begin{abstract}
This paper proposes new jackknife IV estimators that are robust to the
effects of many weak instruments and error heteroskedasticity in a cluster
sample setting with cluster-specific effects and possibly many included
exogenous regressors. The estimators that we propose are designed to
properly partial out the cluster-specific effects and included exogenous
regressors while preserving the re-centering property of the jackknife
methodology. To the best of our knowledge, our proposed procedures provide
the first consistent estimators under many weak instrument asymptotics in
the setting considered. We also present results on the asymptotic normality
of our estimators and show that t-statistics based on said estimators are asymptotically normal under the
null and consistent under fixed alternatives. Monte Carlo results
show that our t-statistics perform better in controlling size in
finite samples than those based on alternative jackknife IV procedures
previously introduced in the literature.
\end{abstract}
\end{spacing}

\noindent \textit{Keywords: }Cluster sample, instrumental variables,
heteroskedasticity, jackknife, many weak instruments, panel data

\medskip

\noindent \textit{JEL classification: }C12, C13, C23, C26, C38

\noindent \setcounter{page}{1}

\section{Introduction}

\noindent \qquad The problem of endogeneity remains central to research in
economics and econometrics. The key reason for this is that there are many
different regression settings for which endogeneity is an issue, but for
which valid estimators are not currently available. One such setting
involves the case where the objective is to estimate an IV regression with
fixed effects using panel or cluster-sampled data in situations where the
number of available instruments may be large, but where the instruments
themselves are all only weakly correlated with the endogenous regressors.
There is now a substantial literature on estimation and inference under many
weak instruments, including Chao and Swanson (2005), Stock and Yogo (2005),
Hansen, Hausman, and Newey (2008), Hausman et al. (2012), Chao et al. (2012,
2014), Bekker and Crudu (2015), Crudu, Mellace, and S\'{a}ndor (2020), and
Mikusheva and Sun (2021). However, the analyses given in these papers are
for cross-sectional data, thus precluding panel data or cluster sampling
settings where there is additional unobserved heterogeneity modeled by fixed
or cluster-specific effects. Moreover, even in the cross-sectional context,
2SLS and the LIML estimators are not well behaved under many weak
instruments. In particular, Chao and Swanson (2005) and Stock and Yogo
(2005) show that the 2SLS estimator is inconsistent under many weak
instrument asymptotics, even when the errors are homoskedastic. In addition,
Hausman et al. (2012) and Chao et al. (2012) point out that LIML is also
inconsistent under many weak instruments when there is error
heteroskedasticity. Estimators which are currently known to be robust to the
effects of many weak instruments in cross sectional settings with error
heteroskedasticity all have a jackknife form, as discussed in Chao and
Swanson (2004), Chao et al. (2012), and Hausman et al. (2012). These include
the JIVE1 and JIVE2 estimators studied in Angrist, Imbens, and Krueger
(1999), for example. For further discussion, see Phillips and Hale (1977),
Blomquist and Dahlberg (1999), Ackerberg and Devereux (2009), and Bekker and
Crudu (2015). These papers again only study various versions of the
jackknife IV estimator in a cross-sectional setup without fixed effects.

The goal of this paper is to consider the problem of many weak instruments
in a panel data or cluster-sampling framework with fixed or cluster specific
effects. In addition to the presence of unobserved heterogeneity, our setup
allows for additional (included) exogenous regressors which appear in both
the outcome, or structural, equation and in the first-stage equations. To
consistently estimate the structural parameter vector of interest in an IV
regression with fixed or cluster-specific effects, we propose three new
estimators, which we refer to by the acronyms FEJIV, FELIM, and FEFUL. These
estimators are so named as they are modified versions and generalizations,
respectively, of the jackknife IV (JIV), the LIML, and the Fuller (1977)
estimators. In contrast to the original JIV, LIML, and Fuller estimators,
our new estimators are designed to be robust to the effects of many weak
instruments and error heteroskedasticity, even in the presence of additional
complications caused by having fixed or cluster-specific effects and many
included exogenous regressors. To achieve consistency in our setting
requires an estimator that not only properly partials out additional
covariates and cluster-specific effects, but at the same time must also be
properly centered in a form similar to a degenerate U-statistic. It turns
out that accomplishing both of these objectives simultaneously is quite
challenging. While a number of innovative JIV-type estimators have been
proposed recently (see, for example, the improved jackknife estimator IJIVE
of Ackerberg and Devereux (2009), as well as the UJIVE estimator of Koles%
\'{a}r (2013)), due to the aforementioned difficulties, these estimators are
not consistent when applied to our setting under many weak instrument
asymptotics, as we shall elaborate on in greater detail in Section 3. On the
other hand, the estimation procedures that we introduce here are carefully
designed to properly partial out fixed or cluster-specific effects and
included exogenous regressors, while preserving the re-centering property of
the jackknife methodology. To the best of our knowledge, the estimators
presented here are the first consistent estimators under many weak
instrument asymptotics in an IV regression model with fixed or
cluster-specific effects and possibly many included exogenous regressors. In
addition to consistency, we also establish the asymptotic normality of the
FELIM and FEFUL estimators\footnote{%
We do not provide a formal proof of the asymptotic normality of the FEJIV
estimator because the results of our Monte Carlo study, as reported in
Section 6, show that FELIM and FEFUL tend to have better finite sample
properties than FEJIV. For this reason, we shall focus the presentation of
our theoretical results on FELIM\ and FEFUL only. However, one can easily
show, by slightly modifying the arguments that we give for FELIM and FEFUL,
that FEJIV is also asymptotically normal, under many weak instrument
asymptotics. Note also that our simulation finding regarding the properties
of FEJIV are consistent with the findings of Davidson and MacKinnon (2006).}.

This paper also provides a number of results showing that hypothesis testing
procedures based on FELIM and FEFUL are robust to the effects of many weak
instruments. In particular, we construct t-statistics based on these two
estimators and show that, when the null hypothesis is true, these
t-statistics converge to an asymptotic standard normal distribution under
both many weak instrument asymptotics and also standard asymptotics.
Moreover, our t-statistics are shown to be consistent in the sense that
under fixed alternatives they diverge, with probability approaching one, in
the direction of the alternative hypothesis.

The many-weak-instrument asymptotic framework used in the sequel to analyze
the performance of FELIM and FEFUL was first proposed in Chao and Swanson
(2005). This framework extends earlier work by Morimune (1983) and Bekker
(1994) on what has become known in the IV literature as the many-instrument
asymptotics or \textquotedblleft Bekker asymptotics\textquotedblright ,
whereby a large sample approximation is carried out by considering an
alternative sequence where the number of instruments is allowed to approach
infinity as the sample size grows to infinity. A key difference between the
Bekker asymptotic framework and the many-weak-instrument asymptotic
framework is the rate of growth of the so-called concentration parameter. As
has been pointed out by Phillips (1983) and Rothenberg (1984), among others,
the concentration parameter is the natural measure of instrument strength in
a linear IV model. In the original papers by Morimune (1983) and Bekker
(1994), the concentration parameter is assumed to grow at the same rate as
the sample size, which is also what is assumed under standard (strong but
fixed number of instruments) asymptotics, whereas the many-weak-instrument
asymptotic framework allows the concentration parameter to grow at a rate
much slower than the sample size, thus allowing for much weaker instruments.
Let $\mu _{n}^{2}$ be a sequence that gives the rate of growth of the
concentration parameter, and let $K_{2,n}$ denote the number of instruments.
Chao and Swanson (2005) show that for consistent point estimation to be
possible, a sufficient condition is $\sqrt{K_{2,n}}/\mu _{n}^{2}\rightarrow
0 $, as $K_{2,n}$, $\mu _{n}^{2},n\rightarrow \infty $. This allows for the
possibility that $\mu _{n}^{2}$ is of an order smaller than $K_{2,n}$ which,
in turn, can be of an order much smaller than the sample size $n$. The
original Bekker framework, on the other hand, requires $K_{2,n},\mu
_{n}^{2}, $ and $n$ to all be of the same order of magnitude. Recent work by
Mikusheva and Sun (2021) indicates that the condition $\sqrt{K_{2,n}}/\mu
_{n}^{2}\rightarrow 0$, as $K_{2,n}$, $\mu _{n}^{2},n\rightarrow \infty $ is
not only sufficient but also necessary for consistency in point estimation
and hypothesis testing.\footnote{%
An alternative to the asymptotic framework considered here is the weak
instrument asymptotic framework proposed in Staiger and Stock (1997). The
Staiger-Stock framework considers a setting where $\mu _{n}^{2}=O\left(
1\right) $, in which case the IV model is not point identified. We do not
consider the Staiger-Stock framework in this paper because our focus is on
consistency of point estimation and on test consistency.}

The rest of the paper is organized as follows. Section 2 provides some brief
motivation for our paper. Section 3 states the model, defines the FELIM,
FEFUL, and FEJIV estimators, and provides an explanation of how our
estimators improve upon various alternative jackknife IV estimators that
have previously been proposed in the literature. Analytical results
presented in Section 4 establish that our estimators are consistent and
asymptotically normally distributed. Section 5 shows how to estimate the
variances of the estimators and also provides asymptotic results for
t-statistics based on our estimators. Section 6 contains the results of a
series of Monte Carlo experiments in which the relative performance of our
estimators is compared with that of extant estimators in the literature.
Section 7 concludes. Proofs of Theorem 1, Corollary 1, Theorems 4-5, and
Corollaries 2-3 are presented in the Appendix to this paper. The proofs of
Theorems 2 and 3 are longer and are given in a supplemental Appendix%
\footnote{%
The supplemental Appendix can be viewed at the URL: http://econweb.umd.edu/%
\symbol{126}chao/Research/research\_files/
\par
\noindent
Supplemental\_Appendix\_to\_Jackknife\_Estimation\_Cluster\_Sample\_IV%
\_Model\_December\_20\_2022.pdf}.

Before proceeding, we will first say a few words about some of the commonly
used notations in this paper. In what follows, we use $\lambda _{\min
}\left( A\right) $, $\lambda _{\max }\left( A\right) $, and $tr\left(
A\right) $ to denote, respectively, the minimal eigenvalue, the maximal
eigenvalue, and the trace of a square matrix $A,$ whereas $A^{\prime }$
denotes the transpose of a (not necessarily square) matrix $A$. $\left\Vert
a\right\Vert _{2}$ denotes the usual Euclidean norm when applied to a
(finite-dimensional) vector $a$. On the other hand, for a matrix $A$, $%
\left\Vert A\right\Vert _{2}\equiv \max \left\{ \sqrt{\lambda \left(
A^{\prime }A\right) }:\lambda \left( A^{\prime }A\right) \text{ is an
eigenvalue of }A^{\prime }A\right\} $ denotes the matrix spectral norm,
while $\left\Vert A\right\Vert _{F}\equiv \sqrt{tr\left\{ A^{\prime
}A\right\} }$ denotes the Frobenius norm and $\left\Vert A\right\Vert
_{\infty }\equiv \max_{1\leq i\leq
m_{n}}\dsum\nolimits_{j=1}^{m_{n}}\left\vert a_{ij}\right\vert $ (i.e., the
maximal row sum of an $m_{n}\times m_{n}$ matrix). In addition, we use $%
A\circ B$ to denote the Hadamard product of two conformable matrices $A$ and 
$B$ (i.e., $A\circ B\equiv \left[ a_{ij}b_{ij}\right] ,$ for $A=\left[ a_{ij}%
\right] $ and $B=\left[ b_{ij}\right] ).$ We take $D\left( a\right) $ to be
a diagonal matrix whose diagonal elements correspond with the elements of
the vector $a,$ while $D\left( A\right) $ is taken to be a diagonal matrix
whose diagonal elements are the same as the diagonal elements of the square
matrix $A$. Furthermore, we will let $\iota _{p}=\left( 1,1,...,1\right)
^{\prime }$ denote a $p\times 1$ vector of ones, and we take the shorthand $%
a.s.n.$ to mean almost surely for all $n$ sufficiently large. Finally, we
use CS and T, respectively, to denote the Cauchy-Schwarz and the triangle
inequality, and the abbreviation w.p.a.1 stands for \textquotedblleft with
probability approaching one\textquotedblright .

\section{Some Background and Motivation}

In this section, we briefly discuss some of the issues that arise when one
needs to partial out additional covariates in a setting with many weak
instruments, with the hope that such a discussion will provide the necessary
background to help readers gain a stronger intuitive feel for the estimation
procedures which we will introduce in subsequent sections. To offer a point
of contrast, we will start by first reviewing some basic aspects of IV
estimation under many weak instruments in the context of a simple,
cross-sectional model with a single endogenous regressor and no additional
covariate, i.e.,

\begin{eqnarray*}
\underset{n\times 1}{y} &=&\underset{1\times 1}{\delta _{0}}\underset{%
n\times 1}{x}+\underset{n\times 1}{\varepsilon }\text{,} \\
\underset{n\times 1}{x} &=&\underset{n\times K_{2}}{Z_{2}}\underset{%
K_{2}\times 1}{\pi _{n}}+\underset{n\times 1}{u}
\end{eqnarray*}%
Here, $y$ is vector of observations on the outcome variable, $x$ is the
vector of observations on the endogenous regressor, and $Z_{2}$ is a
non-random matrix of observations on the $K_{2}$ instruments. In addition,
we intentionally specify the coefficient vector $\pi _{n}$ of the
first-stage equation to depend on $n$ to allow for local-to-zero modeling of
weak instruments\footnote{%
See Assumption 3 in section 3 for the type of (generalized) local-to-zero
structure which we assume for the more general cluster-sample/panel-data IV
regression setting studied in this paper.}. Even in this simple setup, it is
well-known that, in the presence of many weak instruments and error
heteroskedasticity, the usual IV-type estimator such as 2SLS and LIML will
not have desirable asymptotic properties. To see this, consider the case of
the 2SLS estimator, which in this case, can be decomposed as%
\begin{equation}
\widehat{\delta }_{2SLS}-\delta _{0}=\left( x^{\prime }P^{Z_{2}}x\right)
^{-1}x^{\prime }P^{Z_{2}}\varepsilon =\left( \pi _{n}^{\prime }Z_{2}^{\prime
}Z_{2}\pi _{n}+2\pi _{n}^{\prime }Z_{2}^{\prime }u+u^{\prime
}P^{Z_{2}}u\right) ^{-1}\left( \pi _{n}^{\prime }Z_{2}^{\prime }\varepsilon
+u^{\prime }P^{Z_{2}}\varepsilon \right)  \label{2SLS decomp}
\end{equation}%
where $\widehat{\delta }_{2SLS}$ is of course obtained by minimizing the
objective function

\noindent $\widehat{Q}_{2SLS}\left( \delta \right) =\left( y-x^{\prime
}\delta \right) ^{\prime }P^{Z_{2}}\left( y-x^{\prime }\delta \right) $ with 
$P^{Z_{2}}=Z_{2}\left( Z_{2}^{\prime }Z_{2}\right) ^{-1}Z_{2}^{\prime }$.
Under conventional asymptotics with a fixed number of strong instruments,
the asymptotic behavior of the denominator $x^{\prime }P^{Z_{2}}x$ will be
dominated by the concentration parameter $\pi _{n}^{\prime }Z_{2}^{\prime
}Z_{2}\pi _{n}$ which in this case grows at the rate of the sample size $n$,
whereas $\pi _{n}^{\prime }Z_{2}^{\prime }\varepsilon =O_{p}\left( \sqrt{n}%
\right) $ and $u^{\prime }P^{Z_{2}}\varepsilon =O_{p}\left( 1\right) $ so
that, in some sense, the signal in the denominator overwhelms the noise
elements in the numerator, leading to the consistency of the 2SLS estimator.
Viewed from this perspective, the problem caused by having weak instruments
is that the signal component as represented by the concentration parameter $%
\pi ^{\prime }Z_{2}^{\prime }Z_{2}\pi $, is now weaker and grows at some
rate $\mu _{n}^{2}$ which is much slower than $n$. On the other hand, the
problem caused by many instruments is that it inflates one of the noise
components $u^{\prime }P^{Z_{2}}\varepsilon $ which now grows, in
probability, at the rate $K_{2}$. This combination of having stronger noise
and a weaker signal can then lead to inconsistency of the 2SLS estimator
when $\mu _{n}^{2}/K_{2}=O\left( 1\right) $. Note also that under
conventional, strong-instrument asymptotics the term $u^{\prime
}P^{Z_{2}}\varepsilon $ is of a lower order relative to $\pi _{n}^{\prime
}Z_{2}^{\prime }\varepsilon $ but this will no longer be true when $\mu
_{n}^{2}/K_{2}=O\left( 1\right) $, so having sufficiently many weak
instruments leads to a reshuffling of the order of magnitude of the terms in
the numerator of expression (\ref{2SLS decomp}).

Now, one way to fix this problem in the case with no additional covariates
is to use one of the JIVE estimators proposed in Angrist, Imbens, and
Krueger (1999). As an illustration, consider the JIVE2 estimator proposed in
that paper which can be obtained by minimizing a modified 2SLS objective
function whereby the diagonal elements of the projection matrix $P^{Z_{2}}$
are removed; that is, the JIVE2 estimator is obtained by minimizing the
objective function 
\begin{equation*}
\widehat{Q}_{JIVE2}\left( \delta \right) =\left( y-x^{\prime }\delta \right)
^{\prime }\left[ P^{Z_{2}}-D\left( P^{Z_{2}}\right) \right] \left(
y-x^{\prime }\delta \right)
\end{equation*}%
where $D\left( P^{Z_{2}}\right) $ is the diagonal matrix whose diagonal
elements are the same as those of $P^{Z_{2}}$. The reason why such
\textquotedblleft jackknife-type" modification helps is that if we do a
decomposition of JIVE2 similar to the decomposition given for 2SLS in
expression (\ref{2SLS decomp}) above, we obtain%
\begin{equation*}
\widehat{\delta }_{JIVE2}-\delta _{0}=\left( x^{\prime }\left[
P^{Z_{2}}-D\left( P^{Z_{2}}\right) \right] x\right) ^{-1}\left( \pi
_{n}^{\prime }Z_{2}^{\prime }\left[ P^{Z_{2}}-D\left( P^{Z_{2}}\right) %
\right] \varepsilon +u^{\prime }\left[ P^{Z_{2}}-D\left( P^{Z_{2}}\right) %
\right] \varepsilon \right) \text{.}
\end{equation*}%
Comparing the JIVE2 bilinear term $u^{\prime }\left[ P^{Z_{2}}-D\left(
P^{Z_{2}}\right) \right] \varepsilon $ with its counterpart $u^{\prime
}P^{Z_{2}}\varepsilon $ for the 2SLS estimator, we see that the former has a
smaller order of magnitude than the latter under a many instrument
asymptotic regime, so that, in particular, $u^{\prime }\left[
P^{Z_{2}}-D\left( P^{Z_{2}}\right) \right] \varepsilon =O_{p}\left( \sqrt{%
K_{2}}\right) $ whereas $u^{\prime }P^{Z_{2}}\varepsilon =O_{p}\left(
K_{2}\right) $. The reason why this is the case is related to the so-called
concentration of measure phenomenon that has been studied in the probability
literature. Note that, under the assumption that $\left( \varepsilon
_{i},u_{i}\right) $ is independent of $\left( \varepsilon _{j},u_{j}\right) $
for all $i\neq j$ (where $\varepsilon _{i}$ and $u_{i}$ denote the $i^{th}$
component of $\varepsilon $ and $u$ respectively); $E\left[ u^{\prime }\left[
P^{Z_{2}}-D\left( P^{Z_{2}}\right) \right] \varepsilon \right] =0$, even
under heteroskedasticity, whereas $E\left[ u^{\prime }P^{Z_{2}}\varepsilon %
\right] \neq 0$, so that the former, being a properly centered bilinear
form, will have a lower order of magnitude than the latter, which is not
properly centered at zero\footnote{%
To give perhaps a more familiar example of the concentration of measure
phenomenon, we can consider a simple case where $W_{1},...,W_{n}$ is a
sequence of independent random variables such that $\sup_{i}E\left[ W_{i}^{2}%
\right] <\infty $ and $E\left[ W_{i}\right] \neq 0$ for all $i$. In this
case, it is well-known that $\dsum\nolimits_{i=1}^{n}W_{i}=O_{p}\left(
n\right) $ whereas $\dsum\nolimits_{i=1}^{n}\left( W_{i}-\mu _{i}\right)
=O_{p}\left( \sqrt{n}\right) $, so that the order of magnitude in
probability of the uncentered sum $\dsum\nolimits_{i=1}^{n}W_{i}$ is much
larger than that of the properly centered sum $\dsum\nolimits_{i=1}^{n}%
\left( W_{i}-\mu _{i}\right) $. In other words, the sum of an independent
sequence of random variables will concentrate more sharply in a much
narrower range around its mean. It follows also that if it had been the case
that $E\left[ W_{i}\right] =0$ for all $i$; then, we would have $%
\dsum\nolimits_{i=1}^{n}W_{i}=O_{p}\left( \sqrt{n}\right) $, so the order of
magnitude in this case is smaller than in the case where $E\left[ W_{i}%
\right] \neq 0$. Moreover, the concentration of measure phenomenon is known
to exist more generally, not just for sums of independent random variables
but also for Lipschitz functions of such variables and for multilinear
forms. See, for example, Tao (2012) for additional discussion.}. It follows
that JIVE2 will be more robust to the effects of many weak instruments in
the sense that it will be consistent as long as the concentration parameter
grows fast enough so that $\sqrt{K_{2}}/\mu _{n}^{2}\rightarrow 0$, whereas
the consistency of the 2SLS requires the stronger condition that $K_{2}/\mu
_{n}^{2}\rightarrow 0$.

Consider next a more realistic model with additional covariates%
\begin{eqnarray*}
\underset{n\times 1}{y} &=&\underset{1\times 1}{\delta _{0}}\underset{%
n\times 1}{x}+\underset{n\times K_{1}}{Z_{1}}\underset{K_{1}\times 1}{%
\varphi }+\underset{n\times 1}{\varepsilon }\text{,} \\
\underset{n\times 1}{x} &=&\underset{n\times K_{1}}{Z_{1}}\underset{%
K_{1}\times 1}{\beta }+\underset{n\times K_{2}}{Z_{2}}\underset{K_{2}\times 1%
}{\pi _{n}}+\underset{n\times 1}{u}
\end{eqnarray*}%
To see why it is not as straightforward as one might think to generalize the
JIVE2 estimator discussed previously to this setting, consider the IJIVE2
(the improved JIVE2) estimator discussed in Evdokimov and Koles\'{a}r
(2018). To construct the IJIVE2 estimator, one first partials out the
covariates $Z_{1}$ to obtain the system of equations 
\begin{eqnarray}
\widetilde{y} &=&\delta _{0}\widetilde{x}+\widetilde{\varepsilon }
\label{partial out structural eqn} \\
\widetilde{x} &=&\widetilde{Z}_{2}\pi +\widetilde{u}
\label{partial out 1st stage eqn}
\end{eqnarray}%
(where $\widetilde{y}=M^{Z_{1}}y$, $\widetilde{x}=M^{Z_{1}}x$, $\widetilde{Z}%
_{2}=M^{Z_{1}}Z_{2}$, $\widetilde{\varepsilon }=M^{Z_{1}}\varepsilon $, $%
\widetilde{u}=M^{Z_{1}}u$ and $M^{Z_{1}}=I_{n}-Z_{1}\left( Z_{1}^{\prime
}Z_{1}\right) ^{-1}Z_{1}^{\prime }$) and then construct a JIVE2 estimator
based on the representation given in expressions (\ref{partial out
structural eqn})-(\ref{partial out 1st stage eqn}). It is easy to see that
this estimation strategy leads equivalently to an estimator that minimizes
that objective function%
\begin{equation*}
\widehat{Q}_{IJIVE2}\left( \delta \right) =\left( \widetilde{y}-\widetilde{x}%
^{\prime }\delta \right) ^{\prime }\left[ P^{\widetilde{Z}_{2}}-D\left( P^{%
\widetilde{Z}_{2}}\right) \right] \left( \widetilde{y}-\widetilde{x}^{\prime
}\delta \right)
\end{equation*}%
and the deviation of this IJIVE2 estimator, $\widehat{\delta }_{IJIVE2}$,
from the true value, $\delta _{0}$, can be decomposed as 
\begin{equation*}
\widehat{\delta }_{IJIVE2}-\delta _{0}=\left( \widetilde{x}^{\prime }\left[
P^{\widetilde{Z}_{2}}-D\left( P^{\widetilde{Z}_{2}}\right) \right] 
\widetilde{x}\right) ^{-1}\left( \pi _{n}^{\prime }\widetilde{Z}_{2}^{\prime
}\left[ P^{\widetilde{Z}_{2}}-D\left( P^{\widetilde{Z}_{2}}\right) \right] 
\widetilde{\varepsilon }+\widetilde{u}^{\prime }\left[ P^{\widetilde{Z}%
_{2}}-D\left( P^{\widetilde{Z}_{2}}\right) \right] \widetilde{\varepsilon }%
\right)
\end{equation*}%
Again, if we focus on the term $\widetilde{u}^{\prime }\left[ P^{\widetilde{Z%
}_{2}}-D\left( P^{\widetilde{Z}_{2}}\right) \right] \widetilde{\varepsilon }$%
, we can show by simple manipulation that, since $P^{\widetilde{Z}_{2}}=%
\widetilde{Z}_{2}\left( \widetilde{Z}_{2}^{\prime }\widetilde{Z}_{2}\right)
^{-1}\widetilde{Z}_{2}^{\prime }=M^{Z_{1}}Z_{2}\left( Z_{2}^{\prime
}M^{Z_{1}}Z_{2}\right) Z_{2}^{\prime }M^{Z_{1}}$, 
\begin{eqnarray*}
\widetilde{u}^{\prime }\left[ P^{\widetilde{Z}_{2}}-D\left( P^{\widetilde{Z}%
_{2}}\right) \right] \widetilde{\varepsilon } &=&u^{\prime }M^{Z_{1}}\left[
M^{Z_{1}}Z_{2}\left( Z_{2}^{\prime }M^{Z_{1}}Z_{2}\right) Z_{2}^{\prime
}M^{Z_{1}}-D\left( P^{\widetilde{Z}_{2}}\right) \right] M^{Z_{1}}\varepsilon
\\
&=&u^{\prime }\left[ P^{\widetilde{Z}_{2}}-M^{Z_{1}}D\left( P^{\widetilde{Z}%
_{2}}\right) M^{Z_{1}}\right] \varepsilon
\end{eqnarray*}%
By an easy calculation, one can show that $E\left\{ u^{\prime }\left[ P^{%
\widetilde{Z}_{2}}-M^{Z_{1}}D\left( P^{\widetilde{Z}_{2}}\right) M^{Z_{1}}%
\right] \varepsilon \right\} \neq 0$, so this term is not properly centered
at zero, even under the usual assumption that $\left( \varepsilon
_{i},u_{i}\right) $ is independent of $\left( \varepsilon _{j},u_{j}\right) $
for all $i\neq j$, as long as there is error heteroskedasticity. Note that,
the matrix $P^{\widetilde{Z}_{2}}-M^{Z_{1}}D\left( P^{\widetilde{Z}%
_{2}}\right) M^{Z_{1}}$ in the middle of the bilinear form in $u$ and $%
\varepsilon $ turns out not to have zero diagonal elements because, in some
sense, the process of partialing out $Z_{1}$ has interfered with the process
of jackknife recentering in this case. Our basic point in presenting this
example here is simply to show that it is not as easy as it might seem to
construct an IV estimator which simultaneously partial out all additional
covariates and at the same time preserve the recentering property of the
jackknife methodology. As we will show in the remaining sections of this
paper, such estimators can be constructed, however, even in a more general
cluster-sample/panel-data setting with fixed effects and with many
stochastic instruments and included exogenous regressors.

\section{Model, Assumptions, and Estimation Procedures}

The more general model that we consider in this paper is a cluster-sample IV
regression model

\begin{eqnarray}
\underset{1\times 1}{y_{\left( i,t\right) }} &=&X_{\left( i,t\right)
}^{\prime }\delta _{0}+\varphi _{n}^{\prime }Z_{1,\left( i,t\right) }+\alpha
_{i}+\varepsilon _{\left( i,t\right) },  \label{structural eqn} \\
X_{\left( i,t\right) } &=&\Phi _{n}^{\prime }Z_{1,\left( i,t\right) }+\Pi
_{n}^{\prime }Z_{2,\left( i,t\right) }+\xi _{i}+U_{\left( i,t\right) },
\label{1st stage eqn}
\end{eqnarray}%
where $i=1,...,n,$ $t=1,...,T_{i},$ and the total sample size is given by $%
m_{n}=\dsum\nolimits_{i=1}^{n}T_{i}$. The notation $\left( i,t\right) :%
\mathbb{N\times N\rightarrow }$ $\mathbb{N}$ denotes a pairing function
which maps an ordered pair of natural numbers into a natural number, so
that, in particular, we have $\left( 1,1\right) =1,...,\left( 1,T_{1}\right)
=T_{1},\left( 2,1\right) =T_{1}+1,...,\left( n,T_{n}\right) =m_{n}$. This is
just a notational device used to convert a double index into a single index,
thus, facilitating certain vectorization and summation operations while
still allowing one to keep track of both $i$ and $t$. In this setup, we take 
$X_{\left( i,t\right) }$ to be a $d\times 1$ vector of endogenous
regressors, and we let $Z_{1,\left( i,t\right) }$ denote a $K_{1,n}\times 1$
vector of included exogenous variables and let $Z_{2,\left( i,t\right) }$
denote a $K_{2,n}\times 1$ vector of instruments, for $i=1,2,...,n$ and $%
t=1,...,T_{i}$ (or, equivalently, for $\left( i,t\right) =1,...,m_{n}$). To
allow for the possibility that $Z_{1,\left( i,t\right) }$ and $Z_{2,\left(
i,t\right) }$ may be weakly correlated with the endogenous variables $%
y_{\left( i,t\right) }$ and $X_{\left( i,t\right) }$, we let each of the
coefficient parameters $\varphi _{n}$, $\Phi _{n}$, and $\Pi _{n}$ to
possibly have a (generalized) local-to-zero structure which we will specify
more precisely later in Assumptions 3 and 4. In addition, $\alpha _{i}$ and $%
\xi _{i}$ in the above equations denote unobserved or individual effects
interpreted as \textquotedblleft fixed effects\textquotedblright\ in the
sense that although we do not necessarily require $\alpha _{i}$ and $\xi
_{i} $ to be (non-random) constants, they are allowed to be correlated with
the exogenous variables $Z_{1,\left( i,t\right) }$ and $Z_{2,\left(
i,t\right) }$, unlike the typical assumptions specified in a traditional
\textquotedblleft random effects\textquotedblright\ model. More precise
assumptions on the model given by equations (\ref{structural eqn}) and (\ref%
{1st stage eqn}) are given below.

We will develop some additional notations before proceeding. First, let

\noindent $Z_{1}=\left( Z_{1,\left( 1,1\right) },..,Z_{1,\left(
1,T_{1}\right) },...,Z_{1,\left( n,1\right) },..,Z_{1,\left( n,T_{n}\right)
}\right) ^{\prime }$ be an $m_{n}\times K_{1,n}$ matrix of observations on
the include exogenous variables and let $Z_{2}=\left( Z_{2,\left( 1,1\right)
},..,Z_{2,\left( 1,T_{1}\right) },...,Z_{2,\left( n,1\right)
},..,Z_{2,\left( n,T_{n}\right) }\right) ^{\prime }$ be an $m_{n}\times
K_{2,n}$ matrix of observations on the instruments. Also, define the $%
m_{n}\times K_{n}$ matrix $Z=\left[ 
\begin{array}{cc}
Z_{1} & Z_{2}%
\end{array}%
\right] $, where $K_{n}=K_{1,n}+K_{2,n}$. Now, let $y$ and $X$ be defined
similar to $Z_{1}$ and $Z_{2}$ by stacking the observations across the index 
$\left( i,t\right) =1,...,m_{n}$; and we can write the model given by
equations (\ref{structural eqn}) and (\ref{1st stage eqn}) more succinctly as%
\begin{eqnarray}
\underset{m_{n}\times 1}{y} &=&X\delta _{0}+Z_{1}\varphi _{n}+Q\alpha
+\varepsilon \text{,}  \label{structural eqn stacked} \\
\underset{m_{n}\times d}{X} &=&Z_{1}\Phi _{n}+Z_{2}\Pi _{n}+Q\Xi +U\text{,}
\label{1st stage eqn stacked}
\end{eqnarray}%
where $\alpha =\left( \alpha _{1},...,\alpha _{n}\right) ^{\prime }$, $\Xi
=\left( \xi _{1},...,\xi _{n}\right) ^{\prime }$, and $\underset{m_{n}\times
n}{Q}=\left( 
\begin{array}{cccc}
e_{1,n}\iota _{T_{1}}^{\prime } & e_{2,n}\iota _{T_{2}}^{\prime } & \cdots & 
e_{n,n}\iota _{T_{n}}^{\prime }%
\end{array}%
\right) ^{\prime }$ with $e_{j,n}$ being an $n\times 1$ elementary vector
whose $j^{th}$ component is $1$ and all other components are $0$. Note that
our setup allows the clusters to be of possibly different sizes, so that our
model can also be interpreted as a possibly unbalanced panel data model. For
notational convenience, we have suppressed the dependence of $y$, $X$, $%
Z_{1} $, $Z_{2}$, $Q$, $\varepsilon $, and $U$ on $n$ but have made explicit
the dependence of $\varphi _{n}$, $\Phi _{n}$, and $\Pi _{n}$ on $n$ to
highlight the fact that these parameters may have a local-to-zero structure.

Making use of these notations, we can write down the following assumptions
for our model.

\medskip

\noindent \textbf{Assumption 1: }Let $\mathcal{F}_{n}^{Z}=\sigma \left(
Z\right) $ (i.e., the $\sigma $-algebra generated by $Z)$. Assume the
following conditions are satisfied (i) Conditional on $\mathcal{F}_{n}^{Z}$, 
$\left( \varepsilon _{\left( 1,1\right) },U_{\left( 1,1\right) }^{\prime
}\right) ,...,\left( \varepsilon _{\left( 1,T_{1}\right) },U_{\left(
1,T_{1}\right) }^{\prime }\right) ,$

\noindent $.....,\left( \varepsilon _{\left( n,1\right) },U_{\left(
n,1\right) }^{\prime }\right) ,...,\left( \varepsilon _{\left(
n,T_{n}\right) },U_{\left( n,T_{n}\right) }^{\prime }\right) $ are mutually
independent. (ii) $E\left[ \varepsilon _{\left( i,t\right) }|\mathcal{F}%
_{n}^{Z}\right] =0$ and

\noindent $E\left[ U_{\left( i,t\right) }|\mathcal{F}_{n}^{Z}\right] =0$ $%
a.s.$, for $\left( i,t\right) =1,...,m_{n}$.

\medskip

\noindent \textbf{Assumption 2: }There exists a constant $C\geq 1$ such that
for all $n$

\noindent (i) $\max_{1\leq \left( i,t\right) \leq m_{n}}E\left[ \varepsilon
_{\left( i,t\right) }^{8}|\mathcal{F}_{n}^{Z}\right] \leq C<\infty $ $a.s.$%
and $\max_{1\leq \left( i,t\right) \leq m_{n}}E\left[ \left\Vert U_{\left(
i,t\right) }\right\Vert _{2}^{8}|\mathcal{F}_{n}^{Z}\right] \leq C<\infty $ $%
a.s.$ and (ii) $\inf_{1\leq \left( i,t\right) \leq m_{n}}\lambda _{\min
}\left( \Omega _{\left( i,t\right) }\right) \geq 1/C>0$ $\ a.s.$, where $%
\Omega _{\left( i,t\right) }=E\left[ \nu _{\left( i,t\right) }\nu _{\left(
i,t\right) }^{\prime }|\mathcal{F}_{n}^{Z}\right] $ with $\nu _{\left(
i,t\right) }=\left( 
\begin{array}{cc}
\varepsilon _{\left( i,t\right) } & U_{\left( i,t\right) }^{\prime }%
\end{array}%
\right) ^{\prime }$.\bigskip

\noindent \textbf{Assumption 3: }Let $\Pi _{n}=\Upsilon D_{\mu }/\sqrt{n}$,
where $D_{\mu }=diag\left( \mu _{1,n},..,\mu _{d,n}\right) $. Also, let $\mu
_{n}^{\min }=\min_{1\leq k\leq d}$ $\mu _{k,n}$ and let $K_{2,n}$ denote the
number of instruments or the number of columns of $Z_{2}$. The following
conditions are assumed on the diagonal elements $\mu _{1,n},..,\mu _{d,n},$
as $n\rightarrow \infty $. (i) Either $\mu _{k,n}=\sqrt{n}$ or $\mu _{k,n}/%
\sqrt{n}\rightarrow 0,$ for $k\in \left\{ 1,...,d\right\} $. (ii) $\mu
_{n}^{\min }\rightarrow \infty ,$ as $n\rightarrow \infty $, such that$\sqrt{%
K_{2,n}}/\left( \mu _{n}^{\min }\right) ^{2}\rightarrow 0$ (iii) $\lambda
_{\min }\left( H_{n}\right) \geq 1/C>0$ and $\lambda _{\max }\left( \Upsilon
^{\prime }Z_{2}^{\prime }Z_{2}\Upsilon /n\right) \leq C<\infty $ $a.s.,$ for
all $n$ sufficiently large, where $H_{n}=\Upsilon ^{\prime }Z_{2}^{\prime
}M^{\left( Z_{1},Q\right) }Z_{2}\Upsilon /n$. Here, we take $M^{\left(
Z_{1},Q\right) }=M^{Q}-M^{Q}Z_{1}\left( Z_{1}^{\prime }M^{Q}Z_{1}\right)
^{-1}Z_{1}^{\prime }M^{Q}$ where $M^{Q}=I_{m_{n}}-Q\left( Q^{\prime
}Q\right) ^{-1}Q^{\prime }$, so that $M^{\left( Z_{1},Q\right) }$ is a
projection matrix which projects into the orthogonal complement of the space
spanned by the columns of the matrix $\left[ 
\begin{array}{cc}
Z_{1} & Q%
\end{array}%
\right] $.

\medskip

\noindent \textbf{Assumption 4: }Let $\Phi _{n}=\Theta D_{\kappa }/\sqrt{n}$
and $\varphi _{n}=\gamma \tau _{n}/\sqrt{n}$, where $\underset{d\times d}{%
D_{\kappa }}=diag\left( \kappa _{1,n},..,\kappa _{d,n}\right) $ and $\tau
_{n}$ is a sequence of positive real numbers. The following conditions are
assumed on $\kappa _{1,n},..,\kappa _{d,n}$ and on $\tau _{n}$ as $%
n\rightarrow \infty $: (i) either $\kappa _{\ell ,n}=\sqrt{n}$ or $\kappa
_{\ell ,n}/\sqrt{n}\rightarrow 0,$ for $\ell \in \left\{ 1,...,d\right\} $;
(ii) either $\tau _{n}=\sqrt{n}$ or $\tau _{n}/\sqrt{n}\rightarrow 0.$%
\medskip

Assumption 3 is general enough to accommodate a range of situations
including both cases where there are strong instruments and cases where the
instruments are weaker. In particular, when $\mu _{1,n}=\cdot \cdot \cdot
=\mu _{d,n}=\mu _{n}^{\min }=\sqrt{n}$, our model specializes to the more
classical situation where the instruments are strong. On the other hand, the
cases where some of the $\mu _{j,n}$'s $\left( j=1,..,d\right) $ grow at a
rate slower than $\sqrt{n}$ correspond to cases where at least some of the
components of the parameter vector of interest $\delta $ are weakly
identified. By allowing for the possibility that different $\mu _{j,n}$'s
may grow at different rates, our setup also allows for heterogeneity in how
strongly the different components of $\delta $ are identified. Note,
however, that we do require that $\sqrt{K_{2,n}}/\left( \mu _{n}^{\min
}\right) ^{2}\rightarrow 0$, since this is both a sufficient and a necessary
condition for consistent estimation of $\delta $.\footnote{%
The sufficiency part of this condition has been demonstrated in various
settings by Chao and Swanson (2005), Hausman et al (2012), and Chao et al
(2012); whereas the necessity part of this condition has been proved
recently by Mikusheva and Sun (2021).}

It should be noted that an interesting paper by Antoine and Renault (2012)
has also modeled heterogeneity in instrument weakness in a way similar to
Assumption 3. However, our setup here differs from that of Antoine and
Renault (2012) in several respects. First of all, Antoine and Renault (2012)
consider a GMM setup with a fixed number of moment conditions. Hence,
Antoine and Renault (2012) allow for nonlinearity in their framework but do
not consider the case where the number of instruments/moment conditions may
be large, as we do here in our linear setup. In addition, the parameter
vector in the Antoine-Renault setup is of fixed dimension. In contrast,
although our parameter vector of interest $\delta $ is also of fixed
dimension; our model contains a large number of additional nuisance and
incidental parameters, given that we allow for many included exogenous
regressors and for the presence of fixed effects. Thus, the paper by Antoine
and Renault (2012) does not consider the kind of problems associated with
having to eliminate a large number of nuisance and incidental parameters
that we do here in our paper. Given these differences, we view our analysis
here as being largely complementary to that of Antoine and Renault (2012).

Assumption 4 allows for possible local-to-zero modeling of the coefficients
of $Z_{1}$ both in the outcome (or structural) equation and in the
first-stage equations. In the special case where $\kappa _{1,n}=\cdot \cdot
\cdot =\kappa _{d,n}=\tau _{n}=\sqrt{n}$ and $\mu _{1,n}=\cdot \cdot \cdot
=\mu _{d,n}=\mu _{n}^{\min }=\sqrt{n}$, our model becomes a standard
textbook linear IV model (or limited information simultaneous equations
model) with strong instruments. However, by allowing for the possibility
that some of the $\kappa _{j,n}$'s and/or $\tau _{n}$ may grow at a rate
slower than $\sqrt{n}$, we also accommodate situations where the additional
covariates may only be weakly correlated with $y_{\left( i,t\right) }$
and/or with some elements of \ $X_{\left( i,t\right) }$.

\medskip

\noindent \textbf{Assumption 5: }(i) $m_{n}\rightarrow \infty $ as $%
n\rightarrow \infty ,$ such that $m_{n}\sim n$. (ii) $K_{1,n},K_{2,n}%
\rightarrow \infty ,$ as $n\rightarrow \infty ,$ such that $%
K_{1,n}^{2}/n=O\left( 1\right) $ and $K_{2,n}^{2}/n=o\left( 1\right) $.
(iii) Let $M^{Q}=I_{m_{n}}-Q\left( Q^{\prime }Q\right) ^{-1}Q^{\prime }$.
There exists a positive constant $\underline{C}$ such that $\lambda _{\min
}\left( Z^{\prime }M^{Q}Z\right) \geq \underline{C}$ $>0$ $a.s.,$ for all $n$
sufficiently large. (iv) Let $P^{\perp }=P^{\left( Z,Q\right) }-P^{\left(
Z_{1},Q\right) }=M^{\left( Z_{1},Q\right) }Z_{2}\left( Z_{2}^{\prime
}M^{\left( Z_{1},Q\right) }Z_{2}\right) ^{-1}Z_{2}^{\prime }M^{\left(
Z_{1},Q\right) }$ and $P^{Z_{1}^{\perp }}=M^{Q}Z_{1}\left( Z_{1}^{\prime
}M^{Q}Z_{1}\right) ^{-1}Z_{1}^{\prime }M^{Q}$, where $M^{\left(
Z_{1},Q\right) }$ is as defined in part (iii) of Assumption 3 and where $%
P^{\left( Z,Q\right) }$ and $P^{\left( Z_{1},Q\right) }$ are projection
matrices that project into the column space of $\left[ 
\begin{array}{cc}
Z & Q%
\end{array}%
\right] $ and $\left[ 
\begin{array}{cc}
Z_{1} & Q%
\end{array}%
\right] $, respectively\footnote{%
Note that $P^{\left( Z,Q\right) }$ and $P^{\left( Z_{1},Q\right) }$can be
given the explicit representations $P^{\left( Z,Q\right)
}=P^{Z}+M^{Z}Q\left( Q^{\prime }M^{Z}Q\right) ^{-1}Q^{\prime }M^{Z}$ and $%
P^{\left( Z_{1},Q\right) }=P^{Z_{1}}+M^{Z_{1}}Q\left( Q^{\prime
}M^{Z_{1}}Q\right) ^{-1}Q^{\prime }M^{Z_{1}}$, where $P^{Z}=Z\left(
Z^{\prime }Z\right) ^{-1}Z^{\prime }$, $P^{Z_{1}}=Z_{1}\left( Z_{1}^{\prime
}Z_{1}\right) ^{-1}Z_{1}^{\prime }$, $M^{Z_{1}}=I_{m_{n}}-P^{Z_{1}}$, and $%
M^{Z}=I_{m_{n}}-P^{Z}$.}. Assume that $\max_{1\leq \left( i,t\right) \leq
m_{n}}P_{\left( i,t\right) ,\left( i,t\right) }^{Z_{1}^{\perp
}}=O_{a.s.}\left( K_{1,n}/n\right) $ and $\max_{1\leq \left( i,t\right) \leq
m_{n}}P_{\left( i,t\right) ,\left( i,t\right) }^{\perp }=O_{a.s.}\left(
K_{2,n}/n\right) $\footnote{%
More primitive, sufficient conditions for $\max_{1\leq \left( i,t\right)
\leq m_{n}}P_{\left( i,t\right) ,\left( i,t\right) }^{Z_{1}^{\perp
}}=O_{a.s.}\left( K_{1,n}/n\right) $ and $\max_{1\leq \left( i,t\right) \leq
m_{n}}P_{\left( i,t\right) ,\left( i,t\right) }^{\perp }=O_{a.s.}\left(
K_{2,n}/n\right) $ are given in Lemma OA-20 of the Additional Online
Appendix, which can be found at the URL:
\par
\noindent http://econweb.umd.edu/\symbol{126}chao/Research/research%
\_files/Additional\_Online\_Appendix\_Jackknife\_Estimation\_
\par
\noindent Cluster\_Sample\_IV\_Model\_December\_20\_2022.pdf}.

\medskip

\noindent \textbf{Assumption 6: }(i)\textbf{\ }$\min_{1\leq i\leq
n}T_{i}\geq 3$ for all $n$; (ii)\textbf{\ }There exists a positive integer $%
\overline{T}\geq 3,$ such that $\max_{1\leq i\leq n}T_{i}\leq \overline{T}%
<\infty ,$ for all $n$.

\medskip

\noindent \textbf{Assumption 7: }Assume that $\max_{1\leq \left( i,t\right)
\leq m_{n}}\left\Vert \Upsilon ^{\prime }Z_{2}^{\prime }M^{\left(
Z_{1},Q\right) }e_{\left( i,t\right) }\right\Vert _{2}/\sqrt{n}=o_{p}\left(
1\right) $, where $e_{\left( i,t\right) }$ is an $m_{n}\times 1$ elementary
vector whose $\left( i,t\right) ^{th}$ component is $1$ and all components
are $0$ for $\left( i,t\right) \in \left\{ 1,2,...,m_{n}\right\} $.

\newline
Note that Assumption 7 is similar to a condition given in Assumption 3 of
Cattaneo, Jansson, and Newey (2018). As noted in that paper, this assumption
comes close to providing a minimal condition for the central limit theorem
to hold.

\medskip

\noindent \textbf{Assumption 8: }Let $\rho _{n}=E\left[ U^{\prime
}M^{Q}\varepsilon \right] /E\left[ \varepsilon ^{\prime }M^{Q}\varepsilon %
\right] $. Let the limit of $\rho _{n}$ exists, so that $\rho
_{n}\rightarrow \rho $ ,\ as $n\rightarrow \infty ,$ for some fixed $d\times
1$ vector $\rho \in \mathcal{S}_{\rho }$, where $\mathcal{S}_{\rho }$
denotes some compact subset of $\mathbb{R}^{d}$.\bigskip

To estimate the parameter (vector) of interest $\delta $ in equation (\ref%
{structural eqn}), we propose three new jackknife-type IV estimators. We
shall use the acronyms FEJIV, FELIM, and FEFUL to denote, respectively, the
Fixed Effect Jackknife IV, the Fixed Effect LIML, and the Fixed Effect
Fuller estimator.

\begin{enumerate}
\item \textbf{FEJIV: }%
\begin{equation*}
\widehat{\delta }_{J}=\left( X^{\prime }AX\right) ^{-1}X^{\prime }Ay\text{,}
\end{equation*}%
where $A=P^{\perp }-M^{\left( Z,Q\right) }D_{\widehat{\vartheta }}M^{\left(
Z,Q\right) }$, with $M^{\left( Z,Q\right) }=I_{m_{n}}-P^{\left( Z,Q\right) }$
and with $P^{\perp }$ as previously defined in Assumption 5. In addition, $%
D_{\widehat{\vartheta }}$ denotes an $m_{n}\times m_{n}$ diagonal matrix,
whose diagonal elements $\widehat{\vartheta }=\left( 
\begin{array}{cccc}
\widehat{\vartheta }_{1} & \widehat{\vartheta }_{2} & \cdots & \widehat{%
\vartheta }_{m_{n}}%
\end{array}%
\right) ^{\prime }$, when stacked into a vector, correspond to the solution
of the system of linear equations $d_{P^{\perp }}=\left( M^{\left(
Z,Q\right) }\circ M^{\left( Z,Q\right) }\right) \vartheta $, where $%
d_{P^{\perp }}$ is an $m_{n}\times 1$ vector containing the diagonal
elements of the projection matrix $P^{\perp }$.\footnote{%
In Lemma 1 below, we show that, under mild conditions, the system of linear
equations, $d_{P^{\perp }}=\left( M^{\left( Z,Q\right) }\circ M^{\left(
Z,Q\right) }\right) \vartheta $, always has a unique solution.}

\item \textbf{FELIM: }The FELIM estimator $\widehat{\delta }_{L}$ is the
estimator that minimizes the objective function 
\begin{equation}
\widehat{Q}_{FELIM}\left( \delta \right) =\frac{\left( y-X\delta \right)
^{\prime }A\left( y-X\delta \right) }{\left( y-X\delta \right) ^{\prime
}M^{\left( Z_{1},Q\right) }\left( y-X\delta \right) },  \label{FELIM obj fn}
\end{equation}%
where $A$ is as defined above in the definition of FEJIV and where $%
M^{\left( Z_{1},Q\right) }$ is as defined in Assumption 3. $\widehat{\delta }%
_{L}$ has the explicit representation%
\begin{equation}
\widehat{\delta }_{L}=\left( X^{\prime }\left[ A-\widehat{\ell }%
_{L}M^{\left( Z_{1},Q\right) }\right] X\right) ^{-1}\left( X^{\prime }\left[
A-\widehat{\ell }_{L}M^{\left( Z_{1},Q\right) }\right] y\right) ,
\label{FELIM}
\end{equation}%
where $\widehat{\ell }_{L}$ is the smallest root of the determinantal
equation $\det \left\{ \overline{X}^{\prime }A\overline{X}-\ell \overline{X}%
^{\prime }M^{\left( Z_{1},Q\right) }\overline{X}\right\} $

$=0$ with $\overline{X}=\left[ 
\begin{array}{cc}
y & X%
\end{array}%
\right] $.

\item \textbf{FEFUL: }The FEFUL estimator $\widehat{\delta }_{F}$ is defined
as follows:\textbf{\ }%
\begin{equation*}
\widehat{\delta }_{F}=\left( X^{\prime }\left[ A-\widehat{\ell }%
_{F}M^{\left( Z_{1},Q\right) }\right] X\right) ^{-1}\left( X^{\prime }\left[
A-\widehat{\ell }_{F}M^{\left( Z_{1},Q\right) }\right] y\right) ,
\end{equation*}%
where $\widehat{\ell }_{F}=\left[ \widehat{\ell }_{L}-\left( 1-\widehat{\ell 
}_{L}\right) C/m_{n}\right] /\left[ 1-\left( 1-\widehat{\ell }_{L}\right)
C/m_{n}\right] $ for some constant $C$ and where$\widehat{\ell }_{L}$ is as
previously defined in the definition of FELIM given above. For the Monte
Carlo results reported in section 6, we shall take $C=1$.\medskip
\end{enumerate}

To help develop some intuition for these new estimators, it is easiest if we
focus the discussion on FEJIV. To proceed, note first that, under our setup,
it is not difficult to show that%
\begin{equation*}
\widehat{\delta }_{J}-\delta _{0}=\left( X^{\prime }AX\right) ^{-1}X^{\prime
}A\varepsilon =\left( X^{\prime }AX\right) ^{-1}\left( \Pi _{n}^{\prime
}Z_{2}^{\prime }A\varepsilon +U^{\prime }A\varepsilon \right) ,
\end{equation*}%
where the \textquotedblleft numerator\textquotedblright\ of the right-hand
side of this equation is again written in a familiar form as the sum of a
linear form $\Pi _{n}^{\prime }Z_{2}^{\prime }A\varepsilon $ plus a bilinear
form $U^{\prime }A\varepsilon $. Next, note that an elementary result from
linear algebra states that if $A=MDM$, where $A$ is a square matrix, $D$ is
a diagonal matrix, and $M$ is a symmetric matrix, then $a=\left( M\circ
M\right) d$, where $a=\left( a_{11},a_{22},...,a_{m_{n},m_{n}}\right)
^{\prime }$ and $d=\left( d_{11},d_{22},...,d_{m_{n},m_{n}}\right) ^{\prime
} $ are vectors whose elements are the diagonal elements of the matrices $A$
and $D$, respectively. Put in words, this result states that the vector of
diagonal elements of $A$ is a linear transformation of the vector of
diagonal elements of $D$, with the transformation matrix given by $\left(
M\circ M\right) $. Since in the definition of $\widehat{\delta }_{J}$, we
have specified $A=P^{\perp }-M^{\left( Z,Q\right) }D_{\widehat{\vartheta }%
}M^{\left( Z,Q\right) }$, it follows that by choosing the diagonal elements
of $D_{\widehat{\vartheta }}$ to satisfy the system of linear equations $%
d_{P^{\perp }}=\left( M^{\left( Z,Q\right) }\circ M^{\left( Z,Q\right)
}\right) \vartheta $, where $d_{P^{\perp }}=\left( P_{11}^{\perp
},P_{22}^{\perp },...,P_{m_{n},m_{n}}^{\perp }\right) ^{\prime }$, we would,
by construction, end up with a matrix $A$ whose diagonal elements $%
A_{11},...,A_{m_{n},m_{n}}$ are all zero. This, in turn, leads to the
bilinear form $U^{\prime }A\varepsilon $ having the characteristics of a
degenerate U-statistic, with expectation that is properly centered at zero.
As discussed in the previous section, this proper centering is important, as
it reduces the order of magnitude of the bilinear term $U^{\prime
}A\varepsilon $ and, thus, allows $\widehat{\delta }_{J}$ to be both
consistent and asymptotically normal under many weak instrument asymptotics
so long as $\sqrt{K_{2,n}}/\left( \mu _{n}^{\min }\right) ^{2}\rightarrow 0$%
. In addition, write $\widehat{\delta }_{J}-\delta _{0}=\left( X^{\prime
}AX\right) ^{-1}X^{\prime }A\left( Z_{1}\varphi _{n}+Q\alpha +\varepsilon
\right) $, and note that 
\begin{eqnarray}
X^{\prime }A\left( Z_{1}\varphi _{n}+Q\alpha +\varepsilon \right) &=&\left(
Z_{1}\Phi _{n}+Z_{2}\Pi _{n}+Q\Xi +U\right) ^{\prime }\left[ P^{\perp
}-M^{\left( Z,Q\right) }D_{\widehat{\vartheta }}M^{\left( Z,Q\right) }\right]
\left( Z_{1}\varphi _{n}+Q\alpha +\varepsilon \right)  \notag \\
&=&\Pi _{n}^{\prime }Z_{2}^{\prime }P^{\perp }\varepsilon +U^{\prime }\left[
P^{\perp }-M^{\left( Z,Q\right) }D_{\widehat{\vartheta }}M^{\left(
Z,Q\right) }\right] \varepsilon \text{.}  \label{numerator FEJIV}
\end{eqnarray}%
Looking at equation (\ref{numerator FEJIV}), we see that the design of the
matrix $A$ allows fixed effects and the included exogenous regressors $Z_{1}$
to be partialed out on both sides of $A$ in the above expression, and this
is done in such a way so that the proper centering of the bilinear form $%
U^{\prime }\left[ P^{\perp }-M^{\left( Z,Q\right) }D_{\widehat{\vartheta }%
}M^{\left( Z,Q\right) }\right] \varepsilon $ is still preserved. FELIM and
FEFUL are a bit more complicated than FEJIV to discuss, but they share the
same basic design as FEJIV; and, in consequence, they will also be
consistent and asymptotically normal under many weak instrument asymptotics,
as we will show in the theorems below.

In contrast, jackknife IV estimators currently available in the literature
do not fully accomplish the dual goals of being both properly centered and
of having all cluster-specific effects and additional covariates properly
partialed out. To be more specific, we will briefly discuss a number of
jackknife IV estimators that have been proposed in the literature. The paper
by Angrist, Imbens, and Krueger (1999) consider the JIVE1 and JIVE2
estimators of the parameter vector $\delta $, but in a cross-sectional setup
without either fixed effects or included exogenous regressors. Hence, these
authors do not explicitly study the more general version of these estimators
that partials out additional covariates. Hausman et al. (2012) introduce
jackknife versions of LIML\ and Fuller estimators called HLIM\ and HFUL, but
they do so in a cross-sectional context where there are no fixed effects and
where only a small number of included exogenous regressors is allowed, so
that the problem of having to partial out fixed effects and a potentially
large number of included exogenous variables is not studied in that paper.
In addition, the symmetric jackknife IV (SJIVE) estimator proposed by Bekker
and Crudu (2015) is formulated in a setting without fixed effects and with
no included exogenous regressors. Hence, that paper also does not consider
issues related to having to partial out additional covariates.

In a recent paper, Evdokimov and Koles\'{a}r (2018) examine a number of
interesting jackknife IV estimators that allow for partialing out of
additional covariates. In the previous section, we have already discussed
the IJIVE2 estimator from that paper in the context of a simple
cross-sectional IV model. Here, we shall briefly examine the other
estimators considered in Evdokimov and Koles\'{a}r (2018) and provide some
discussion about how these estimators might perform under many weak
instruments asymptotics when applied to our more general cluster-sample
setting here with fixed effects. For this purpose, it is easiest to consider
the case where there is only one endogenous regressor. In this case, note
that $D_{\mu }=\mu _{n}=\mu _{n}^{\min }$ since $d=1$, and we shall use $x$, 
$\pi _{n}$, $\phi _{n}$, $\upsilon $, and $u$ in lieu of $X$, $\Pi _{n}$, $%
\Phi _{n}$, $\Upsilon $, and $U$ to emphasize the fact that, in the one
endogenous regressor case; $x$, $\pi _{n}$, $\phi _{n}$, and $u$ are vectors
and not matrices.

Consider first the IJIVE1 estimator studied in that paper. This estimator
was originally proposed by Ackerberg and Devereux (2009) and is further
analyzed in the grouped data setting by Evdokimov and Koles\'{a}r (2018)%
\footnote{%
It should be noted that this estimator was originally referred to in
Ackerberg and Devereux (2009) as simply IJIVE. However, since Edokimov and
Koles\'{a}r (2018) introduced a variant of this estimator in their paper
which they called IJIVE2, they renamed the original IJIVE estimator IJIVE1.}%
. Using our notation, the estimator can be written in the form%
\begin{eqnarray*}
\widehat{\delta }_{IJIVE1} &=&\left( x^{\prime }M^{\left( Z_{1},Q\right) }%
\left[ P^{\perp }-D\left( P^{\perp }\right) \right] \left[ I_{m_{n}}-D\left(
P^{\perp }\right) \right] ^{-1}M^{\left( Z_{1},Q\right) }x\right) ^{-1} \\
&&\times \left( x^{\prime }M^{\left( Z_{1},Q\right) }\left[ P^{\perp
}-D\left( P^{\perp }\right) \right] \left[ I_{m_{n}}-D\left( P^{\perp
}\right) \right] ^{-1}M^{\left( Z_{1},Q\right) }y\right) .
\end{eqnarray*}%
Now, it is easily seen that the deviation of this estimator from the true
value $\delta _{0}$ can be written as%
\begin{eqnarray}
\widehat{\delta }_{IJIVE1}-\delta _{0} &=&\left( x^{\prime }A_{IJ1}x\right)
^{-1}x^{\prime }A_{IJ1}\left( Z_{1}\varphi _{n}+Q\alpha +\varepsilon \right)
\notag \\
&=&\left( x^{\prime }A_{IJ1}x\right) ^{-1}\left( \Pi _{n}^{\prime
}Z_{2}^{\prime }A_{IJ1}\varepsilon +U^{\prime }A_{IJ1}\varepsilon \right) 
\text{,}  \label{IJIVE1}
\end{eqnarray}%
where $A_{IJ1}=M^{\left( Z_{1},Q\right) }\left[ P^{\perp }-D\left( P^{\perp
}\right) \right] \left[ I_{m_{n}}-D\left( P^{\perp }\right) \right]
^{-1}M^{\left( Z_{1},Q\right) }$. Straightforward

\noindent calculations further show that the $\left( i,t\right) ^{th}$
diagonal element of the matrix $A_{IJ1}$ is given by%
\begin{equation*}
A_{IJ1,\left( i,t\right) ,\left( i,t\right) }=\dsum\limits_{\left(
j,s\right) =1}^{m_{n}}\frac{M_{\left( j,s\right) ,\left( i,t\right)
}^{\left( Z_{1},Q\right) }}{1-P_{\left( j,s\right) ,\left( j,s\right)
}^{\perp }}\left[ P_{\left( i,t\right) ,\left( j,s\right) }^{\perp
}-M_{\left( i,t\right) ,\left( j,s\right) }^{\left( Z_{1},Q\right)
}P_{\left( j,s\right) ,\left( j,s\right) }^{\perp }\right] \neq 0,
\end{equation*}%
for $\left( i,t\right) =1,...,m_{n}$, so that $u^{\prime }A_{IJ1}\varepsilon 
$, the bilinear form on the right-hand side of equation (\ref{IJIVE1})
above, will not be a degenerate U-statistic and will not be properly
centered at the origin. Hence, similar to what we have pointed out
previously about IJIVE2, the problem here is that, although the matrix $%
\left[ P^{\perp }-D\left( P^{\perp }\right) \right] \left[ I_{m_{n}}-D\left(
P^{\perp }\right) \right] ^{-1}$ does have a \textquotedblleft jackknife
form\textquotedblright\ in the sense that the elements of its main diagonal
are all zero, it defines a bilinear form not with respect to $u$ and $%
\varepsilon $ but with respect to the projected vectors $\widehat{u}%
=M^{\left( Z_{1},Q\right) }u$ and $\widehat{\varepsilon }=M^{\left(
Z_{1},Q\right) }\varepsilon $. Note, however, that in general the $\left(
i,t\right) ^{th}$ element of $\widehat{u}$ will contain not just the $\left(
i,t\right) ^{th}$ element of $u$ but other elements as well, and similarly
for $\widehat{\varepsilon }$. In consequence, merely having the diagonal
elements zeroed out in this case is not sufficient for the bilinear form $%
u^{\prime }A_{IJ1}\varepsilon =\widehat{u}^{\prime }\left[ P^{\perp
}-D\left( P^{\perp }\right) \right] \left[ I_{m_{n}}-D\left( P^{\perp
}\right) \right] ^{-1}\widehat{\varepsilon }$ to have expectation equal to
zero. Again, we have a situation where the process of partialing out the
covariates has interfered with the process of jackknife recentering.

Another estimator studied in Evdokimov and Koles\'{a}r (2018) is the UJIVE
estimator, which was first introduced in Koles\'{a}r (2013) and then further
analyzed in the grouped data setting by Evdokimov and Koles\'{a}r (2018).
This estimator takes the form 
\begin{eqnarray*}
\widehat{\delta }_{UJIVE} &=&\left( x^{\prime }\left[ \widetilde{P}^{\left(
Z,Q\right) }D\left( M^{\left( Z,Q\right) }\right) ^{-1}-\widetilde{P}%
^{\left( Z_{1},Q\right) }D\left( M^{\left( Z_{1},Q\right) }\right) ^{-1}%
\right] x\right) ^{-1} \\
&&\times \left( x^{\prime }\left[ \widetilde{P}^{\left( Z,Q\right) }D\left(
M^{\left( Z,Q\right) }\right) ^{-1}-\widetilde{P}^{\left( Z_{1},Q\right)
}D\left( M^{\left( Z_{1},Q\right) }\right) ^{-1}\right] y\right) ,
\end{eqnarray*}%
where $Z=\left[ 
\begin{array}{cc}
Z_{1} & Z_{2}%
\end{array}%
\right] $, $\widetilde{P}^{\left( Z,Q\right) }=P^{\left( Z,Q\right)
}-D\left( P^{\left( Z,Q\right) }\right) $, and $\widetilde{P}^{\left(
Z_{1},Q\right) }=P^{\left( Z_{1},Q\right) }-D\left( P^{\left( Z_{1},Q\right)
}\right) $. Now, the deviation of the UJIVE estimator from the true value $%
\delta _{0}$ can be written as%
\begin{eqnarray*}
\widehat{\delta }_{UJIVE}-\delta _{0} &=&\left( \frac{x^{\prime }A_{UJ}x}{%
\mu _{n}^{2}}\right) ^{-1}\left( \frac{x^{\prime }A_{UJ}Z_{1}\varphi
_{n}+x^{\prime }A_{UJ}Q\alpha +\phi _{n}^{\prime }Z_{1}^{\prime
}A_{UJ}\varepsilon +\pi _{n}^{\prime }Z_{2}^{\prime }A_{UJ}\varepsilon
+u^{\prime }A_{UJ}\varepsilon }{\mu _{n}^{2}}\right) \\
&=&\left( \frac{x^{\prime }A_{UJ}x}{\mu _{n}^{2}}\right) ^{-1}\left( \frac{%
x^{\prime }A_{UJ}Z_{1}\varphi _{n}+x^{\prime }A_{UJ}Q\alpha +\pi
_{n}^{\prime }Z_{2}^{\prime }A_{UJ}\varepsilon +u^{\prime }A_{UJ}\varepsilon 
}{\mu _{n}^{2}}\right)
\end{eqnarray*}%
where $A_{UJ}=\left[ P^{\left( Z,Q\right) }-D\left( P^{\left( Z,Q\right)
}\right) \right] D\left( M^{\left( Z,Q\right) }\right) ^{-1}-\left[
P^{\left( Z_{1},Q\right) }-D\left( P^{\left( Z_{1},Q\right) }\right) \right]
D\left( M^{\left( Z_{1},Q\right) }\right) ^{-1}$. Note first that the
diagonal elements of the matrix $A_{UJ}$ are all equal to zero, so the
bilinear term for this estimator, $U^{\prime }A_{UJ}\varepsilon $, is
properly centered. However, this estimator has a bias problem that arises
from the presence of the term $x^{\prime }A_{UJ}Z_{1}\varphi _{n}/\mu
_{n}^{2}$, which can be nonnegligible and even large in order of magnitude.
To see this, observe first that simple manipulation shows that $%
A_{UJ}=M^{\left( Z_{1},Q\right) }D\left( M^{\left( Z_{1},Q\right) }\right)
^{-1}-M^{\left( Z,Q\right) }D\left( M^{\left( Z,Q\right) }\right) ^{-1}$.
Using this identity, we can write%
\begin{eqnarray}
\frac{x^{\prime }A_{UJ}Z_{1}\varphi _{n}}{\mu _{n}^{2}} &=&\frac{\pi
_{n}^{\prime }Z_{2}^{\prime }M^{\left( Z_{1},Q\right) }D\left( M^{\left(
Z_{1},Q\right) }\right) ^{-1}Z_{1}\varphi _{n}}{\mu _{n}^{2}}  \notag \\
&&+\frac{u^{\prime }M^{\left( Z_{1},Q\right) }D\left( M^{\left(
Z_{1},Q\right) }\right) ^{-1}Z_{1}\varphi _{n}}{\mu _{n}^{2}}-\frac{%
u^{\prime }M^{\left( Z,Q\right) }D\left( M^{\left( Z,Q\right) }\right)
^{-1}Z_{1}\varphi _{n}}{\mu _{n}^{2}}.  \label{XAphi}
\end{eqnarray}%
Note that the term on the right-hand side of (\ref{XAphi}) which can be
particularly large in order of magnitude is $\pi _{n}^{\prime }Z_{2}^{\prime
}M^{\left( Z_{1},Q\right) }D\left( M^{\left( Z_{1},Q\right) }\right)
^{-1}Z_{1}\varphi _{n}/\mu _{n}^{2}$. In fact, one can show that%
\begin{equation*}
\frac{\pi _{n}^{\prime }Z_{2}^{\prime }M^{\left( Z_{1},Q\right) }D\left(
M^{\left( Z_{1},Q\right) }\right) ^{-1}Z_{1}\varphi _{n}}{\mu _{n}^{2}}=%
\frac{\tau _{n}}{\mu _{n}}\frac{\upsilon ^{\prime }Z_{2}^{\prime }M^{\left(
Z_{1},Q\right) }D\left( M^{\left( Z_{1},Q\right) }\right) ^{-1}Z_{1}\gamma }{%
n}=O_{a.s.}\left( \frac{\tau _{n}}{\mu _{n}}\right) \text{.}
\end{equation*}%
Hence, this estimator will be inconsistent as long as $\mu _{n}=O\left( \tau
_{n}\right) $. This will certainly be true in weak instrument cases where $%
\mu _{n}=o\left( \tau _{n}\right) $, but can also occur even in strong
instrument cases where $\mu _{n}\sim \sqrt{n}$ if the included exogenous
regressors enter significantly into the structural equation of interest, in
which case $\tau _{n}\sim \sqrt{n}$. Our Monte Carlo results reported in
section 6 also confirm that UJIVE can have a large median bias relative to
its competitors when there are included exogenous regressors that enter
significantly into the structural equation of interest\footnote{%
It should be noted, however, that UJIVE may perform well under many weak
instrument asymptotics in the special case where the equation of interest
contains no included exogenous regressors and only fixed effects. This is
not only because in this case there is no term of the form
\par
\noindent $x^{\prime }A_{UJ}Z_{1}\varphi _{n}/\mu _{n}^{2}=\tau
_{n}x^{\prime }A_{UJ}Z_{1}\gamma /\left( \mu _{n}^{2}\sqrt{n}\right) $, but
also because, in this case,%
\begin{equation*}
\frac{\pi _{n}^{\prime }Z_{2}^{\prime }A_{UJ}Q\alpha }{\mu _{n}^{2}}=\frac{%
\pi _{n}^{\prime }Z_{2}^{\prime }\left[ M^{Q}D\left( M^{Q}\right)
^{-1}-M^{\left( Z_{2},Q\right) }D\left( M^{\left( Z_{2},Q\right) }\right)
^{-1}\right] Q\alpha }{\mu _{n}\sqrt{n}}=0
\end{equation*}%
\par
\noindent so that, without the contaminating effects of the included
exogenous regressors, UJIVE does properly partial out the fixed effects. We
conjecture that, in this setting, UJIVE might be consistent so long as $%
\sqrt{K_{2,n}}/\left( \mu _{n}^{\min }\right) ^{2}\rightarrow 0$, but we
have yet to obtain a formal proof of this result.}.

Since our setup essentially has a panel data structure, one may also wonder
if it is possible to simply first difference away the fixed effects and then
do a jackknife-type recentering. A problem with this strategy occurs if the
IV regression contains, in addition to fixed effects, other included
exogenous regressors which cannot be eliminated by first-differencing. In
that case, one will have to do a projection to partial out these included
exogenous regressors, leading to the same problem as we have discussed
previously with regard to IJIVE1 and IJIVE2. In fact, the problem will be
worse in this case due to the serial correlation in the errors induced by
the first-differencing. Moreover, even if there are no additional included
exogenous regressors, the serial correlation induced by first differencing
causes additional complications. In particular, let $P^{Z}=Z\left( Z^{\prime
}Z\right) ^{-1}Z^{\prime }$ denote the projection matrix of the instruments%
\footnote{%
Here, we let $Z$ denote the matrix of observations on the instruments
because we are referring to a case where there are no included exogenous
variables, $Z_{1}$.}. Then, to achieve proper jackknife recentering in this
case requires the removal not only of the elements on the main diagonal of $%
P^{Z}$ but also the elements on the superdiagonal and the subdiagonal of $%
P^{Z}$, so that with serial correlation proper recentering is attained only
at the cost of greater information loss. Finally, the presence of serial
correlation also makes the large sample covariance matrix of a jackknife IV
estimator under many weak instrument asymptotics both more complicated and
more difficult to estimate. Hence, we believe that our approach for removing
fixed or cluster-specific effects has certain advantages over any
alternative procedure that is based on first-differencing. It should be
noted that a recent panel data paper by Hsiao and Zhou (2018) does take the
approach of constructing a jackknife IV estimator after first-differencing
the data. However, the objective and focus of that paper differs greatly
from ours. First of all, the panel data simultaneous equations model
specified in Hsiao and Zhou (2018) does not allow for the degree of
instrument weakness that we consider. In addition, the model that they
consider does not have error heteroskedasticity or included exogenous
regressors. If we apply their estimator to our setting, the estimator will
not be consistent in the case where $K_{2,n}\sim \left( \mu _{n}^{\min
}\right) ^{2}$ or in the case where $K_{2,n}/\left( \mu _{n}^{\min }\right)
^{2}\rightarrow \infty ,$ but $\sqrt{K_{2,n}}/\left( \mu _{n}^{\min }\right)
^{2}\rightarrow 0$. Still, it should be stressed that in their setting with
strong instruments and error homoskedasticity their estimator does have good
asymptotic properties.

Turning our attention back to the equation $d_{P^{\perp }}=\left( M^{\left(
Z,Q\right) }\circ M^{\left( Z,Q\right) }\right) \vartheta $, note that in
order for this system of linear equations to have a unique solution, we need
the matrix $\left( M^{\left( Z,Q\right) }\circ M^{\left( Z,Q\right) }\right) 
$ to be invertible. The following lemma provides sufficient conditions for
the invertibility of $\left( M^{\left( Z,Q\right) }\circ M^{\left(
Z,Q\right) }\right) $.

\medskip

\noindent \textbf{Lemma 1: }Suppose that Assumptions 5 and 6(i) are
satisfied. Then, there exists a positive constant $C$ such that $\lambda
_{\min }\left( M^{\left( Z,Q\right) }\circ M^{\left( Z,Q\right) }\right)
\geq C>0$ $a.s.$, for all $n$ sufficiently large\footnote{%
A proof of Lemma 1 is given in section 2 of the Additional Online Appendix
for this paper. This online appendix can be viewed at the URL:
\par
\noindent http://econweb.umd.edu/\symbol{126}chao/Research/research%
\_files/Additional\_Online\_Appendix\_Jackknife\_Estimation\_
\par
\noindent Cluster\_Sample\_IV\_Model\_December\_20\_2022.pdf}.

\smallskip

It should be noted that a more general result on conditions for the
invertibility of Hadamard products has been given previously in Cattaneo,
Jansson, and Newey (2018)\footnote{%
See, in particular, the analysis given in Section 3 of their Supplemental
Appendix.}. However, we choose to present a specialization of their result
because it shows that, in the context of our cluster-sampling setup, a key
condition for ensuring the invertibility of $\left( M^{\left( Z,Q\right)
}\circ M^{\left( Z,Q\right) }\right) $ is $\min_{1\leq i\leq n}T_{i}\geq 3$,
which we explicitly assume in Assumption 6 part (i) above.

A further observation is that, in analyzing estimators that are obtained
from minimizing a variance ratio (e.g., FELIM), it is often convenient to
first consider the objective function in the form $Q\left( \beta \right)
=\left( \beta ^{\prime }\overline{X}^{\prime }A\overline{X}\beta \right)
/\left( \beta ^{\prime }\overline{X}^{\prime }M^{\left( Z_{1},Q\right) }%
\overline{X}\beta \right) $, where $\overline{X}=\left[ y,X\right] $ and
where $\beta $ is a $\left( d+1\right) \times 1$ vector, not initially
normalized to identify the dependent variable from the regressors. In this
setting, one would first minimize the objective function $Q\left( \beta
\right) $ to obtain a minimizer $\widetilde{\beta }=\left( 
\begin{array}{cc}
\widetilde{\beta }_{1} & \widetilde{\beta }_{2}^{\prime }%
\end{array}%
\right) ^{\prime }$, with $\widetilde{\beta }_{1}$ being a scalar and $%
\widetilde{\beta }_{2}$ a $d\times 1$ vector and subsequently normalize the
last $d$ components of $\widetilde{\beta }$ to obtain an estimator $%
\widetilde{\delta }=-\widetilde{\beta }_{2}/\widetilde{\beta }_{1}$ for the
coefficients of the endogenous regressors $X$. The following assumption
ensures that this subsequent normalization is well-defined. Moreover, in the
proof of Lemma S2-11 given in the Additional Online Appendix to this paper,
we show that, by following this procedure, we end up with exactly the FELIM
estimator $\widehat{\delta }_{L}$, that satisfies the first-order conditions
of the objective function given by (\ref{FELIM obj fn}) and that also has
explicit representation given by equation (\ref{FELIM}) above\footnote{%
The proof of Lemma S2-11 is given in section 1 of the Additional Online
Appendix, which, in turn, can be found at the URL:
\par
\noindent http://econweb.umd.edu/\symbol{126}chao/Research/research%
\_files/Additional\_Online\_Appendix\_Jackknife\_Estimation\_
\par
\noindent Cluster\_Sample\_IV\_Model\_December\_20\_2022.pdf}.

\medskip

\noindent \textbf{Assumption 9: }Consider the variance-ratio objective
function

\noindent $Q\left( \beta \right) =\left( \beta ^{\prime }\overline{X}%
^{\prime }A\overline{X}\beta \right) /\left( \beta ^{\prime }\overline{X}%
^{\prime }M^{\left( Z_{1},Q\right) }\overline{X}\beta \right) $, where $%
\beta \in \overline{B}=\left\{ \beta \in \mathbb{R}^{d+1}:\left\Vert \beta
\right\Vert _{2}=1\right\} $. Let $\widetilde{\beta }$ be a $\left(
d+1\right) \times 1$ vector that minimizes the objective function $Q\left(
\beta \right) ,$ among all $\beta \in \overline{B}$ (i.e., $\widetilde{\beta 
}=\arg \min_{\beta \in \overline{B}}Q\left( \beta \right) )$. Partition $%
\widetilde{\beta }=\left( \widetilde{\beta }_{1},\widetilde{\beta }%
_{2}^{\prime }\right) ^{\prime }$ as defined above and assume that there
exists a positive constant $\underline{C}$ such that 
\begin{equation}
\left\vert \widetilde{\beta }_{1}\right\vert \geq \underline{C}>0\text{ \ }%
a.s.\text{ for all }n\text{ sufficiently large.}  \label{Cond 9}
\end{equation}

\medskip

\noindent Note that constraining $\beta $ (so that $\left\Vert \beta
\right\Vert _{2}=1)$ is not restrictive since we are dealing with an
objective function $Q\left( \beta \right) $ that is a ratio of quadratic
forms in $\beta $. More precisely, let $\overline{\beta }=\arg \min_{\beta
\in \mathbb{R}^{d+1}}Q\left( \beta \right) $, where $\overline{\beta }\neq 0$%
, and let $\widetilde{\beta }=\overline{\beta }/\left\Vert \overline{\beta }%
\right\Vert _{2}$ so that $\left\Vert \widetilde{\beta }\right\Vert _{2}=1$.
Then, $Q\left( \overline{\beta }\right) =\left( \overline{\beta }^{\prime }%
\overline{X}^{\prime }A\overline{X}\overline{\beta }\right) /\left( 
\overline{\beta }^{\prime }\overline{X}^{\prime }M^{\left( Z_{1},Q\right) }%
\overline{X}\overline{\beta }\right) $

\noindent $=\left( \left\Vert \overline{\beta }\right\Vert _{2}^{-1}%
\overline{\beta }^{\prime }\overline{X}^{\prime }A\overline{X}\overline{%
\beta }\left\Vert \overline{\beta }\right\Vert _{2}^{-1}\right) /\left(
\left\Vert \overline{\beta }\right\Vert _{2}^{-1}\overline{\beta }^{\prime }%
\overline{X}^{\prime }M^{\left( Z_{1},Q\right) }\overline{X}\overline{\beta }%
\left\Vert \overline{\beta }\right\Vert _{2}^{-1}\right) =Q\left( \widetilde{%
\beta }\right) $, so any minimal value of $Q\left( \beta \right) $ obtained
by minimizing $\beta $ over all $\beta \in \mathbb{R}^{d+1}$ can also be
achieved by some $\widetilde{\beta }$ such that $\left\Vert \widetilde{\beta 
}\right\Vert _{2}=1$.

\section{\noindent Consistency and Asymptotic Normality \newline
of Point Estimators}

\noindent \textbf{Theorem 1: }Let $\overline{\delta }_{n}=\left( X^{\prime }%
\left[ A-\overline{\ell }_{n}M^{\left( Z_{1},Q\right) }\right] X\right)
^{-1}\left( X^{\prime }\left[ A-\overline{\ell }_{n}M^{\left( Z_{1},Q\right)
}\right] y\right) $, for some sequence $\overline{\ell }_{n},$ such that $%
\overline{\ell }_{n}=o_{p}\left( \left[ \mu _{n}^{\min }\right]
^{2}/n\right) =o_{p}\left( 1\right) $. Then, under Assumptions 1-6, $%
\left\Vert D_{\mu }\left( \overline{\delta }_{n}-\delta _{0}\right) /\mu
_{n}^{\min }\right\Vert _{2}\overset{p}{\rightarrow }0$ and $\left\Vert 
\overline{\delta }_{n}-\delta _{0}\right\Vert _{2}\overset{p}{\rightarrow }0$%
, as $n\rightarrow \infty $

\medskip

Special cases of the class of estimators that satisfy the conditions of
Theorem 1, and are thus consistent in the sense described in the theorem,
include FEJIV $\widehat{\delta }_{J,n}$, FELIM $\widehat{\delta }_{L,n}$,
and FEFUL $\widehat{\delta }_{F,n}$. Evidently, the main difference between
these estimators is the different specifications of $\overline{\ell }_{n}$. $%
\widehat{\delta }_{J,n}$ takes $\overline{\ell }_{n}=0,$ for all $n$; $%
\widehat{\delta }_{L,n}$ takes $\overline{\ell }_{n}=\widehat{\ell }_{L,n},$
where $\widehat{\ell }_{L,n}$ is the smallest root of the determinantal
equation $\det \left\{ \overline{X}^{\prime }A\overline{X}-\ell \overline{X}%
^{\prime }M^{\left( Z_{1},Q\right) }\overline{X}\right\} =0$; and $\widehat{%
\delta }_{F,n}$ takes $\overline{\ell }_{n}=\widehat{\ell }_{F,n}$ $=\left[ 
\widehat{\ell }_{L}-\left( 1-\widehat{\ell }_{L}\right) C/m_{n}\right] /%
\left[ 1-\left( 1-\widehat{\ell }_{L}\right) C/m_{n}\right] $, as described
earlier. Hence, by verifying that, in all three cases, $\overline{\ell }_{n}$
satisfies the condition $\overline{\ell }_{n}=o_{p}\left( \left[ \mu
_{n}^{\min }\right] ^{2}/n\right) =o_{p}\left( 1\right) $, we can easily
specialize the consistency result of Theorem 1 to establish the consistency
of FEJIV, FELIM, and FEFUL. These results are given in the following
corollary.

\medskip

\noindent \textbf{Corollary 1: }Under Assumptions 1-6 and 9, the following
results hold as $n\rightarrow \infty $.

\noindent (a) $\left\Vert D_{\mu }\left( \widehat{\delta }_{J,n}-\delta
_{0}\right) /\mu _{n}^{\min }\right\Vert _{2}\overset{p}{\rightarrow }0$ and 
$\left\Vert \widehat{\delta }_{J,n}-\delta _{0}\right\Vert _{2}\overset{p}{%
\rightarrow }0$. (b) $\left\Vert D_{\mu }\left( \widehat{\delta }%
_{L,n}-\delta _{0}\right) /\mu _{n}^{\min }\right\Vert _{2}\overset{p}{%
\rightarrow }0$ and $\left\Vert \widehat{\delta }_{L,n}-\delta
_{0}\right\Vert _{2}\overset{p}{\rightarrow }0$. (c) $\left\Vert D_{\mu
}\left( \widehat{\delta }_{F,n}-\delta _{0}\right) /\mu _{n}^{\min
}\right\Vert _{2}\overset{p}{\rightarrow }0$ and $\left\Vert \widehat{\delta 
}_{F,n}-\delta _{0}\right\Vert _{2}\overset{p}{\rightarrow }0$.

\medskip

The next two results establish asymptotic normality for the FELIM and FEFUL
estimators, under two different cases: (i) Case I: $K_{2,n}/\left( \mu
_{n}^{\min }\right) ^{2}=O\left( 1\right) $ and (ii) Case II: $%
K_{2,n}/\left( \mu _{n}^{\min }\right) ^{2}\rightarrow \infty ,$ but $\sqrt{%
K_{2,n}}/\left( \mu _{n}^{\min }\right) ^{2}\rightarrow 0$. The FEJIV
estimator can also be shown to have an asymptotic normal distribution under
both Cases I and II. However, we choose to focus our theoretical analysis on
FELIM\ and FEFUL\ because, as noted previously, the results of our Monte
Carlo study indicate that FELIM and FEFUL have better finite sample
properties than FEJIV.

To facilitate the statement of the next two results, define%
\begin{eqnarray}
\Lambda _{I,n} &=&H_{n}^{-1}\left( \Sigma _{1,n}+\Sigma _{2,n}\right)
H_{n}^{-1}=H_{n}^{-1}\Sigma _{n}H_{n}^{-1}\text{,}  \label{Asy Var I} \\
\Lambda _{II,n} &=&\frac{\left( \mu _{n}^{\min }\right) ^{2}}{K_{2,n}}%
H_{n}^{-1}\Sigma _{2,n}H_{n}^{-1}\text{,}  \label{Asy Var II}
\end{eqnarray}%
where $H_{n}=\Upsilon ^{\prime }Z_{2}^{\prime }M^{\left( Z_{1},Q\right)
}Z_{2}\Upsilon /n$, $\Sigma _{1,n}=VC\left( \Upsilon ^{\prime }Z_{2}^{\prime
}M^{\left( Z_{1},Q\right) }\varepsilon /\sqrt{n}|\mathcal{F}_{n}^{Z}\right) $

\noindent $=\Upsilon ^{\prime }Z_{2}^{\prime }M^{\left( Z_{1},Q\right)
}D_{\sigma ^{2}}M^{\left( Z_{1},Q\right) }Z_{2}\Upsilon /n$, and $\Sigma
_{2,n}=D_{\mu }^{-1}\Sigma _{2,n}^{\ast }D_{\mu }^{-1}$, with 
\begin{eqnarray}
\Sigma _{2,n}^{\ast } &=&VC\left( \underline{U}^{\prime }A\varepsilon |%
\mathcal{F}_{n}^{Z}\right)  \notag \\
&=&\dsum\limits_{\substack{ \left( i,t\right) ,\left( j,s\right) =1  \\ %
\left( i,t\right) \neq \left( j,s\right) }}^{m_{n}}A_{\left( i,t\right)
,\left( j,s\right) }^{2}E\left[ \varepsilon _{\left( i,t\right) }^{2}|%
\mathcal{F}_{n}^{Z}\right] E\left[ \underline{U}_{\left( j,s\right) }%
\underline{U}_{\left( j,s\right) }^{\prime }|\mathcal{F}_{n}^{Z}\right] 
\notag \\
&&+\dsum\limits_{\substack{ \left( i,t\right) ,\left( j,s\right) =1  \\ %
\left( i,t\right) \neq \left( j,s\right) }}^{m_{n}}A_{\left( i,t\right)
,\left( j,s\right) }^{2}E\left[ \underline{U}_{\left( i,t\right)
}\varepsilon _{\left( i,t\right) }|\mathcal{F}_{n}^{Z}\right] E\left[
\varepsilon _{\left( j,s\right) }\underline{U}_{\left( j,s\right) }^{\prime
}|\mathcal{F}_{n}^{Z}\right] \text{.}  \label{sigma2 star}
\end{eqnarray}%
In addition, $\Sigma _{n}=\Sigma _{1,n}+\Sigma _{2,n}$ and $\underline{U}%
_{\left( i,t\right) }=U_{\left( i,t\right) }-\rho \varepsilon _{\left(
i,t\right) }$ for $\left( i,t\right) =1,...,m_{n}$. Here, for any random
vector $x$, $VC\left( x|\mathcal{F}_{n}^{Z}\right) $ denotes the conditional
variance-covariance matrix of $x$ given $\mathcal{F}_{n}^{Z}$. Moreover, let 
$D_{\sigma ^{2}}=diag\left( \sigma _{\left( 1,1\right) }^{2},....,\sigma
_{\left( n,T_{n}\right) }^{2}\right) =diag\left( \sigma _{1}^{2},....,\sigma
_{m_{n}}^{2}\right) $, where $\sigma _{\left( i,t\right) }^{2}=\left[
\varepsilon _{\left( i,t\right) }^{2}|\mathcal{F}_{n}^{Z}\right] ,$ for $%
\left( i,t\right) =1,...,m_{n}$ and where, for notational convenience, we
suppress the dependence of $\sigma _{\left( i,t\right) }^{2}$ on $\mathcal{F}%
_{n}^{Z}$.

As evident from the results given below, $\Lambda _{I,n}$ and $\Lambda
_{II,n}$ are the (conditional) covariance matrices of FELIM\ (and also of
FEFUL) in large samples under Cases I and II, respectively.

\medskip

\noindent \textbf{Theorem 2: } Let Assumptions 1-9 be satisfied. Then, under
Case I where $K_{2,n}/\left( \mu _{n}^{\min }\right) ^{2}=O\left( 1\right) $%
, the following results hold: $\Lambda _{I,n}$ is positive definite $a.s.$
for all $n$ sufficiently large; and, as $n\rightarrow \infty $, $\Lambda
_{I,n}^{-1/2}D_{\mu }\left( \widehat{\delta }_{L,n}-\delta _{0}\right) 
\overset{d}{\rightarrow }N\left( 0,I_{d}\right) $ and $\Lambda
_{I,n}^{-1/2}D_{\mu }\left( \widehat{\delta }_{F,n}-\delta _{0}\right) 
\overset{d}{\rightarrow }N\left( 0,I_{d}\right) $.

\medskip

\noindent \textbf{Theorem 3: }Let Assumptions 1-9 be satisfied, let $%
\widetilde{L}_{n}$ be a $q\times d$ matrix with $1\leq q\leq d$, and let
there exists a positive constant $C$ such that $\left\Vert \widetilde{L}%
_{n}\right\Vert _{2}\leq C<\infty $ and $\lambda _{\min }\left( \widetilde{L}%
_{n}\Lambda _{II,n}\widetilde{L}_{n}^{\prime }\right) \geq 1/C>0$ $a.s.n$.
Then, under Case II where $\left( \mu _{n}^{\min }\right)
^{2}/K_{2,n}=o\left( 1\right) ,$ but $\sqrt{K_{2,n}}/\left( \mu _{n}^{\min
}\right) ^{2}\rightarrow 0$, the following results hold: as $n\rightarrow
\infty $, $\left( \mu _{n}^{\min }/\sqrt{K_{2,n}}\right) \left( \widetilde{L}%
_{n}\Lambda _{II,n}\widetilde{L}_{n}^{\prime }\right) ^{-1/2}\widetilde{L}%
_{n}D_{\mu }\left( \widehat{\delta }_{L,n}-\delta _{0}\right) \overset{d}{%
\rightarrow }N\left( 0,I_{q}\right) $ and $\left( \mu _{n}^{\min }/\sqrt{%
K_{2,n}}\right) \left( \widetilde{L}_{n}\Lambda _{II,n}\widetilde{L}%
_{n}^{\prime }\right) ^{-1/2}\widetilde{L}_{n}D_{\mu }\left( \widehat{\delta 
}_{F,n}-\delta _{0}\right) \overset{d}{\rightarrow }N\left( 0,I_{q}\right) $.

\section{Covariance Matrix Estimation and Hypothesis\newline
Testing}

\noindent \qquad To consistently estimate the asymptotic covariance matrix
of FELIM and FEFUL, we propose the following estimators%
\begin{equation}
\widehat{V}_{L}=\widehat{H}_{L}^{-1}\widehat{\Sigma }_{L}\widehat{H}_{L}^{-1}%
\text{ and }\widehat{V}_{F}=\widehat{H}_{F}^{-1}\widehat{\Sigma }_{F}%
\widehat{H}_{F}^{-1}\text{,}  \label{Var Est LIM and FUL}
\end{equation}%
where%
\begin{eqnarray*}
\widehat{H}_{L} &=&X^{\prime }\left[ A-\widehat{\ell }_{L,n}M^{\left(
Z_{1},Q\right) }\right] X\text{, }\widehat{H}_{F}=X^{\prime }\left[ A-%
\widehat{\ell }_{F,n}M^{\left( Z_{1},Q\right) }\right] X \\
\widehat{\Sigma }_{L} &=&X^{\prime }AD\left( J\left[ \widehat{\varepsilon }%
_{L}\circ \widehat{\varepsilon }_{L}\right] \right) AX-\widehat{\rho }%
_{L}\left( \widehat{\varepsilon }_{L}\circ \widehat{\varepsilon }_{L}\right)
^{\prime }J\left( A\circ A\right) J\left( \widehat{\varepsilon }_{L}\iota
_{d}^{\prime }\circ M^{\left( Z,Q\right) }X\right) \\
&&-\left( \widehat{\varepsilon }_{L}\iota _{d}^{\prime }\circ M^{\left(
Z,Q\right) }X\right) ^{\prime }J\left( A\circ A\right) J\left( \widehat{%
\varepsilon }_{L}\circ \widehat{\varepsilon }_{L}\right) \widehat{\rho }%
_{L}^{\prime }+\widehat{\rho }_{L}\widehat{\rho }_{L}^{\prime }\left( 
\widehat{\varepsilon }_{L}\circ \widehat{\varepsilon }_{L}\right) ^{\prime
}J\left( A\circ A\right) J\left( \widehat{\varepsilon }_{L}\circ \widehat{%
\varepsilon }_{L}\right) \\
&&+\left( \widehat{\varepsilon }_{L}\iota _{d}^{\prime }\circ \widehat{%
\underline{U}}_{L}\right) ^{\prime }J\left( A\circ A\right) J\left( \widehat{%
\varepsilon }_{L}\iota _{d}^{\prime }\circ \widehat{\underline{U}}%
_{L}\right) \text{,} \\
\widehat{\Sigma }_{F} &=&X^{\prime }AD\left( J\left[ \widehat{\varepsilon }%
_{F}\circ \widehat{\varepsilon }_{F}\right] \right) AX-\widehat{\rho }%
_{F}\left( \widehat{\varepsilon }_{F}\circ \widehat{\varepsilon }_{F}\right)
^{\prime }J\left( A\circ A\right) J\left( \widehat{\varepsilon }_{F}\iota
_{d}^{\prime }\circ M^{\left( Z,Q\right) }X\right) \\
&&-\left( \widehat{\varepsilon }_{F}\iota _{d}^{\prime }\circ M^{\left(
Z,Q\right) }X\right) ^{\prime }J\left( A\circ A\right) J\left( \widehat{%
\varepsilon }_{F}\circ \widehat{\varepsilon }_{F}\right) \widehat{\rho }%
_{F}^{\prime }+\widehat{\rho }_{F}\widehat{\rho }_{F}^{\prime }\left( 
\widehat{\varepsilon }_{F}\circ \widehat{\varepsilon }_{F}\right) ^{\prime
}J\left( A\circ A\right) J\left( \widehat{\varepsilon }_{F}\circ \widehat{%
\varepsilon }_{F}\right) \\
&&+\left( \widehat{\varepsilon }_{F}\iota _{d}^{\prime }\circ \widehat{%
\underline{U}}_{F}\right) ^{\prime }J\left( A\circ A\right) J\left( \widehat{%
\varepsilon }_{F}\iota _{d}^{\prime }\circ \widehat{\underline{U}}%
_{F}\right) \text{.}
\end{eqnarray*}%
and where $J=\left[ M^{Q}\circ M^{Q}\right] ^{-1}$, $\widehat{\varepsilon }%
_{L}=M^{\left( Z,Q\right) }\left( y-X\widehat{\delta }_{L}\right) $, $%
\widehat{\varepsilon }_{F}=M^{\left( Z,Q\right) }\left( y-X\widehat{\delta }%
_{F}\right) $,

\noindent $\widehat{\underline{U}}_{L}=M^{\left( Z,Q\right) }X-\widehat{%
\varepsilon }_{L}\widehat{\rho }_{L}^{\prime }$, and $\widehat{\underline{U}}%
_{F}=M^{\left( Z,Q\right) }X-\widehat{\varepsilon }_{F}\widehat{\rho }%
_{F}^{\prime }$. In addition, let

\noindent $\widehat{\rho }_{L}=\left[ X^{\prime }M^{\left( Z,Q\right)
}\left( y-X\widehat{\delta }_{L}\right) \right] /\left[ \left( y-X\widehat{%
\delta }_{L}\right) ^{\prime }M^{\left( Z,Q\right) }\left( y-X\widehat{%
\delta }_{L}\right) \right] $ and

\noindent $\widehat{\rho }_{F}=\left[ X^{\prime }M^{\left( Z,Q\right)
}\left( y-X\widehat{\delta }_{F}\right) \right] /\left( y-X\widehat{\delta }%
_{F}\right) ^{\prime }M^{\left( Z,Q\right) }\left( y-X\widehat{\delta }%
_{F}\right) $ denote estimators of the parameter $\rho =\lim_{n\rightarrow
\infty }E\left[ U^{\prime }M^{Q}\varepsilon \right] /E\left[ \varepsilon
^{\prime }M^{Q}\varepsilon \right] ,$ based on $\widehat{\delta }_{L}$ and $%
\widehat{\delta }_{F}$, respectively.

Our next result shows the consistency of the covariance matrix estimators
given in equation (\ref{Var Est LIM and FUL}) under both Cases I and II%
\footnote{%
It can be shown that an estimator of the asymptotic covariance matrix of
FEJIV, which will be consistent under both Case I and II, is given by%
\begin{equation*}
\widehat{V}_{J,n}=\widehat{H}^{-1}\widehat{\Sigma }_{J}\widehat{H}%
^{-1}=\left( X^{\prime }AX\right) ^{-1}\left[ X^{\prime }AD_{\widehat{%
\varsigma }_{J}}AX+\left( \widehat{\varepsilon }_{J}\circ \widehat{U}\right)
^{\prime }J\left( A\circ A\right) J\left( \widehat{\varepsilon }_{J}\circ 
\widehat{U}\right) \right] \left( X^{\prime }AX\right) ^{-1},
\end{equation*}%
where $D_{\widehat{\varsigma }_{J}}=diag\left( \widehat{\varsigma }%
_{J,\left( 1,1\right) },..,\widehat{\varsigma }_{J,\left( 1,T_{1}\right)
},..,\widehat{\varsigma }_{J,\left( n,1\right) },..,\widehat{\varsigma }%
_{J,\left( n,T_{n}\right) }\right) $, $\widehat{\varsigma }_{J,\left(
i,t\right) }=e_{\left( i,t\right) }^{\prime }J\left( \widehat{\varepsilon }%
_{J}\circ \widehat{\varepsilon }_{J}\right) $, $\widehat{\varepsilon }%
_{J}=M^{\left( Z,Q\right) }\left( y-X\widehat{\delta }_{J}\right) $, and $%
\widehat{U}=M^{\left( Z,Q\right) }X$. Note also that the standard error used
for FEJIV in our Monte Carlo study given in section 6 is based on the above
formula.}.

\bigskip

\noindent \textbf{Theorem 4: }If Assumptions 1-6 and 8-9 are satisfied;
then, the following statements are true:

\begin{enumerate}
\item[(a)] For Case I, where $K_{2,n}/\left( \mu _{n}^{\min }\right)
^{2}=O\left( 1\right) $, $D_{\mu }\widehat{V}_{L}D_{\mu }=\Lambda
_{I,n}+o_{p}\left( 1\right) $ and $D_{\mu }\widehat{V}_{F}D_{\mu }=\Lambda
_{I,n}+o_{p}\left( 1\right) $, where $\Lambda _{I,n}$ is as defined in
equation (\ref{Asy Var I}).

\item[(b)] For Case II, where $K_{2,n}/\left( \mu _{n}^{\min }\right)
^{2}\rightarrow \infty $ but $\sqrt{K_{2,n}}/\left( \mu _{n}^{\min }\right)
^{2}\rightarrow 0$, $\left[ \left( \mu _{n}^{\min }\right) ^{2}/K_{2,n}%
\right] D_{\mu }\widehat{V}_{L}D_{\mu }=\Lambda _{II,n}+o_{p}\left( 1\right) 
$ and $\left[ \left( \mu _{n}^{\min }\right) ^{2}/K_{2,n}\right] D_{\mu }%
\widehat{V}_{F}D_{\mu }=\Lambda _{II,n}+o_{p}\left( 1\right) $, where $%
\Lambda _{II,n}$ is as defined in equation (\ref{Asy Var II}).
\end{enumerate}

\bigskip

Theorem 5 below provides results on the asymptotic properties of
t-statistics associated with the FELIM and FEFUL estimators when testing a
general linear hypothesis of the form $H_{0}:c^{\prime }\delta _{0}=r$. We
show that the t-ratio based on our estimators has an asymptotic standard
normal distribution under the null hypothesis, as long as $\sqrt{K_{2,n}}%
/\left( \mu _{n}^{\min }\right) ^{2}\rightarrow 0$. Moreover, our results
show that, under the same rate condition, the tests are also consistent, as
our test statistics diverge with probability approaching one under fixed
alternatives. Some additional conditions are needed to obtain these results
if we were to allow for general heterogeneity in instrument weakness where
the diagonal elements $\mu _{g,n}$ $\left( g=1,..,d\right) $ of $D_{\mu }$
can diverge at different rates. These conditions are given in Assumption 10.

\medskip

\noindent \textbf{Assumption 10: }Consider testing the null hypothesis $%
H_{0}:c^{\prime }\delta _{0}=r$. Let 
\begin{equation*}
\mu _{n}^{\ast }\left( c\right) =\min \left\{ \mu _{g,n}|g\in \left\{
1,...,d\right\} \text{, }c_{g}\neq 0\right\}
\end{equation*}

\noindent\ \noindent where $c_{g}$ is the $g^{th}$ component of the vector $%
c $, and let $\mu _{n}^{\ast }\left( c\right) D_{\mu }^{-1}c\rightarrow
c_{\ast }\neq 0$ as $n\rightarrow \infty $. Assume that $c_{\ast }^{\prime
}\Lambda _{II,n}c_{\ast }\geq \underline{C}$ $a.s.$ for all $n$ sufficiently
large for some positive constant $\underline{C}$.

\medskip

\noindent \textbf{Theorem 5: }If Assumptions 1-10 are satisfied; then, the
following statements are true for the t-statistics $\mathbb{T}_{L}=\left(
c^{\prime }\widehat{\delta }_{L}-r\right) /\sqrt{c^{\prime }\widehat{V}_{L}c}
$ and $\mathbb{T}_{F}=\left( c^{\prime }\widehat{\delta }_{F}-r\right) /%
\sqrt{c^{\prime }\widehat{V}_{F}c}$.

\begin{enumerate}
\item[a.] For Case I, where $K_{2,n}/\left( \mu _{n}^{\min }\right)
^{2}=O\left( 1\right) $:

\begin{enumerate}
\item[(i)] Under $H_{0}:c^{\prime }\delta _{0}=r$, $\mathbb{T}_{L}\overset{d}%
{\rightarrow }N\left( 0,1\right) $ and $\mathbb{T}_{F}\overset{d}{%
\rightarrow }N\left( 0,1\right) .$

\item[(ii) ] Under $H_{1}:c^{\prime }\delta _{0}\neq r$, with probability
approaching one, as $n\rightarrow \infty $, the following results hold: $%
\mathbb{T}_{L}\rightarrow +\infty $ \ if $c^{\prime }\delta _{0}>r$; $%
\mathbb{T}_{L}\rightarrow -\infty $ \ if $c^{\prime }\delta _{0}<r$; $%
\mathbb{T}_{F}\rightarrow +\infty $\ if $c^{\prime }\delta _{0}>r$; and $%
\mathbb{T}_{F}\rightarrow -\infty $ \ if $c^{\prime }\delta _{0}<r$.
\end{enumerate}

\item[b.] For Case II, where $K_{2,n}/\left( \mu _{n}^{\min }\right)
^{2}\rightarrow \infty $ but $\sqrt{K_{2,n}}/\left( \mu _{n}^{\min }\right)
^{2}\rightarrow 0$:

\begin{enumerate}
\item[(i)] Under $H_{0}:c^{\prime }\delta _{0}=r$, $\mathbb{T}_{L}\overset{d}%
{\rightarrow }N\left( 0,1\right) $ and $\mathbb{T}_{F}\overset{d}{%
\rightarrow }N\left( 0,1\right) .$

\item[(ii)] Under $H_{1}:c^{\prime }\delta _{0}\neq r$, with probability
approaching one, as $n\rightarrow \infty $, the following results hold: $%
\mathbb{T}_{L}\rightarrow +\infty $ \ if $c^{\prime }\delta _{0}>r$; $%
\mathbb{T}_{L}\rightarrow -\infty $ \ if $c^{\prime }\delta _{0}<r$; $%
\mathbb{T}_{F}\rightarrow +\infty $\ if $c^{\prime }\delta _{0}>r$; and $%
\mathbb{T}_{F}\rightarrow -\infty $ \ if $c^{\prime }\delta _{0}<r$.
\end{enumerate}
\end{enumerate}

\noindent .\medskip

In looking over the proof of Theorem 5, one can see that the condition
stipulating $c_{\ast }^{\prime }\Lambda _{II,n}c_{\ast }\geq \underline{C}$ $%
a.s.n.$ for some constant $\underline{C}>0$, given in Assumption 10, is only
needed in Case II where $K_{2,n}/\left( \mu _{n}^{\min }\right)
^{2}\rightarrow \infty $ but $\sqrt{K_{2,n}}/\left( \mu _{n}^{\min }\right)
^{2}\rightarrow 0$. This is because in this case the covariance matrix
estimator is dominated by the contribution of the bilinear term and, when
appropriately normalized, this matrix takes the form%
\begin{eqnarray}
&&\Lambda _{II,n}  \notag \\
&=&\frac{\left( \mu _{n}^{\min }\right) ^{2}}{K_{2,n}}D_{\mu }\left( \Pi
_{n}^{\prime }Z_{2}^{\prime }M^{\left( Z_{1},Q\right) }Z_{2}\Pi _{n}\right)
^{-1}\Sigma _{2,n}^{\ast }\left( \Pi _{n}^{\prime }Z_{2}^{\prime }M^{\left(
Z_{1},Q\right) }Z_{2}\Pi _{n}\right) ^{-1}D_{\mu }  \notag \\
&=&\frac{\left( \mu _{n}^{\min }\right) ^{2}}{K_{2,n}}\left( D_{\mu }^{-1}%
\frac{D_{\mu }\Upsilon ^{\prime }Z_{2}^{\prime }M^{\left( Z_{1},Q\right)
}Z_{2}\Upsilon D_{\mu }}{n}D_{\mu }^{-1}\right) ^{-1}D_{\mu }^{-1}\Sigma
_{2,n}^{\ast }D_{\mu }^{-1}  \notag \\
&&\times \left( D_{\mu }^{-1}\frac{D_{\mu }\Upsilon ^{\prime }Z_{2}^{\prime
}M^{\left( Z_{1},Q\right) }Z_{2}\Upsilon D_{\mu }}{n}D_{\mu }^{-1}\right)
^{-1}  \notag \\
&=&H_{n}^{-1}\left[ \frac{\left( \mu _{n}^{\min }\right) ^{2}}{K_{2,n}}%
D_{\mu }^{-1}\Sigma _{2,n}^{\ast }D_{\mu }^{-1}\right] H_{n}^{-1}
\label{Lambda II Calcul}
\end{eqnarray}%
where $H_{n}=\Upsilon ^{\prime }Z_{2}^{\prime }M^{\left( Z_{1},Q\right)
}Z_{2}\Upsilon /n$ and $\Sigma _{2,n}^{\ast }$ is as defined in expression (%
\ref{sigma2 star}) above. Now, the matrix $\Lambda _{II}$ is singular in
large sample when heterogeneity in instrument weakness of a general form is
allowed because, even though $K_{2,n}^{-1}\Sigma _{2,n}^{\ast }$ can be
shown to be positive definite almost surely as $K_{2,n}$, $n\rightarrow
\infty $\footnote{%
A proof of the asymptotically positive definiteness of $K_{2,n}^{-1}\Sigma
_{2,n}^{\ast }$ is given in Lemma S2-3 part (b) of the Additional Online
Appendix to this paper, which can be found at the URL:
\par
\noindent http://econweb.umd.edu/\symbol{126}chao/Research/research%
\_files/Additional\_Online\_Appendix\_Jackknife\_Estimation\_
\par
\noindent Cluster\_Sample\_IV\_Model\_December\_20\_2022.pdf}; the matrix $%
\left( \mu _{n}^{\min }\right) D_{\mu }^{-1}$ converges to a singular
diagonal matrix where some of the diagonal elements are zero, except in the
special case where $D_{\mu }=\left( \mu _{n}^{\min }\right) \cdot I_{d}$. It
follows that the matrix $\left[ \left( \mu _{n}^{\min }\right) ^{2}/K_{2,n}%
\right] D_{\mu }^{-1}\Sigma _{2,n}^{\ast }D_{\mu }^{-1}$in expression (\ref%
{Lambda II Calcul}) will in general be a singular matrix asymptotically. By
following through the derivation given in expression (\ref{Lambda II Calcul}%
), we see that this problem occurs because, under Case II, the covariance
matrix has a \textquotedblleft denominator" term, i.e., $D_{\mu }\Upsilon
^{\prime }Z_{2}^{\prime }M^{\left( Z_{1},Q\right) }Z_{2}\Upsilon D_{\mu }/n$%
, which depends on $D_{\mu }$ but the \textquotedblleft numerator" term $%
\Sigma _{2,n}^{\ast }$ does not. Due to this asymmetry, in trying to
properly standardize $D_{\mu }\Upsilon ^{\prime }Z_{2}^{\prime }M^{\left(
Z_{1},Q\right) }Z_{2}\Upsilon D_{\mu }/n$ so that its inverse will exist
asymptotically, we end up, in some sense, transferring the singularity to
the \textquotedblleft numerator". This also explains why this same problem
does not arise under Case I, where $K_{2,n}/\left( \mu _{n}^{\min }\right)
^{2}=O\left( 1\right) $, since in that case the covariance matrix is
dominated in the \textquotedblleft numerator" by $D_{\mu }\Upsilon ^{\prime
}Z_{2}^{\prime }M^{\left( Z_{1},Q\right) }D_{\sigma ^{2}}M^{\left(
Z_{1},Q\right) }Z_{2}\Upsilon D_{\mu }/n$, \ the contribution of the linear
term, which is affected by heterogeneity in instrument weakness in the same
way as the \textquotedblleft denominator", so that, upon proper
normalization, the ill effect of this heterogeneity is removed via
cancellation.

It should be pointed out that there are important special cases of interest
where Assumption 10 either holds automatically or can be shown to hold under
mild additional conditions. One such special case is where instrument
weakness is homogeneous, i.e., the case where $\mu _{1,n}=\cdot \cdot \cdot
=\mu _{d,n}=\mu _{n}^{\min }$. In this case, the asymptotic singularity of $%
\Lambda _{II,n}$ does not arise, so that Assumption 10 is fulfilled without
additional side conditions, allowing us to easily obtain the following
corollary to Theorem 5.

\medskip

\noindent \textbf{Corollary 2: }Let Assumptions 1-9 be satisfied. Assume
further that the diagonal matrix $D_{\mu }$ in Assumption 3 takes the form $%
D_{\mu }=\mu _{n}^{\min }\cdot I_{d}$ (i.e., $\mu _{1,n}=\cdot \cdot \cdot
=\mu _{d,n}=\mu _{n}^{\min })$. Then, the following statements are true for
the t-statistics $\mathbb{T}_{L}=\left( c^{\prime }\widehat{\delta }%
_{L}-r\right) /\sqrt{c^{\prime }\widehat{V}_{L}c}$ and $\mathbb{T}%
_{F}=\left( c^{\prime }\widehat{\delta }_{F}-r\right) /\sqrt{c^{\prime }%
\widehat{V}_{F}c}$.

\begin{enumerate}
\item[a.] For Case I, where $K_{2,n}/\left( \mu _{n}^{\min }\right)
^{2}=O\left( 1\right) $:

\begin{enumerate}
\item[(i)] Under $H_{0}:c^{\prime }\delta _{0}=r$, $\mathbb{T}_{L}\overset{d}%
{\rightarrow }N\left( 0,1\right) $ and $\mathbb{T}_{F}\overset{d}{%
\rightarrow }N\left( 0,1\right) .$

\item[(ii) ] Under $H_{1}:c^{\prime }\delta _{0}\neq r$, with probability
approaching one as $n\rightarrow \infty $, the following results hold: $%
\mathbb{T}_{L}\rightarrow +\infty $ \ if $c^{\prime }\delta _{0}>r$; $%
\mathbb{T}_{L}\rightarrow -\infty $ \ if $c^{\prime }\delta _{0}<r$; $%
\mathbb{T}_{F}\rightarrow +\infty $\ if $c^{\prime }\delta _{0}>r$; and $%
\mathbb{T}_{F}\rightarrow -\infty $ \ if $c^{\prime }\delta _{0}<r$.
\end{enumerate}

\item[b.] For Case II, where $K_{2,n}/\left( \mu _{n}^{\min }\right)
^{2}\rightarrow \infty $ but $\sqrt{K_{2,n}}/\left( \mu _{n}^{\min }\right)
^{2}\rightarrow 0$:

\begin{enumerate}
\item[(i)] Under $H_{0}:c^{\prime }\delta _{0}=r$, $\mathbb{T}_{L}\overset{d}%
{\rightarrow }N\left( 0,1\right) $ and $\mathbb{T}_{F}\overset{d}{%
\rightarrow }N\left( 0,1\right) .$

\item[(ii)] Under $H_{1}:c^{\prime }\delta _{0}\neq r$, with probability
approaching one as $n\rightarrow \infty $, the following results hold: $%
\mathbb{T}_{L}\rightarrow +\infty $ \ if $c^{\prime }\delta _{0}>r$; $%
\mathbb{T}_{L}\rightarrow -\infty $ \ if $c^{\prime }\delta _{0}<r$; $%
\mathbb{T}_{F}\rightarrow +\infty $\ if $c^{\prime }\delta _{0}>r$; and $%
\mathbb{T}_{F}\rightarrow -\infty $ \ if $c^{\prime }\delta _{0}<r$.
\end{enumerate}
\end{enumerate}

\medskip

Corollary 2 is of interest because the case where the degree of instrument
weakness is homogeneous and does not vary across the different first-stage
equations is one which is often assumed in previous papers on weak and/or
many instruments. This includes the well-known papers by Bekker (1994),
Staiger and Stock (1997) and Kleibergen (2002). In addition, note that the
case where there is only one endogenous regressor is also obviously a
special case of the setup considered in Corollary 2.

Another special case of interest is where we test a null hypothesis
involving only one coefficient, such as testing the significance of a
particular parameter. This case is important because it is the most frequent
use of the t-statistic by empirical researchers. In this case, we show in
the corollary below that, under mild additional conditions, the t-test based
on our proposed estimators will be robust to many weak instruments, even if
there is heterogeneity in instrument weakness of a general form.

\medskip

\noindent \textbf{Assumption 10*: }Let $e_{\ell }$ denote a $d\times 1$
elementary vector whose $\ell ^{th}$ component is $1$ and all other
components are $0$, and write $D_{\mu }$ in the form%
\begin{equation}
D_{\mu }=\left( 
\begin{array}{cc}
\underset{d_{1}\times d_{1}}{D_{1}} & 0 \\ 
0 & \left( \mu _{n}^{\min }\right) \cdot I_{d_{2}}%
\end{array}%
\right) \text{,}  \label{Dmu}
\end{equation}%
where $D_{1}=diag\left( \mu _{1,n},..,\mu _{d_{1},n}\right) $, where $d_{1}$
and $d_{2}$ are positive integers such that $d_{1}+d_{2}=d$, and where $%
\left( \mu _{n}^{\min }\right) /\mu _{g,n}\rightarrow 0,$ as $n\rightarrow
\infty ,$ for $g\in \left\{ 1,...,d_{1}\right\} $. Partition $H_{n}^{-1}$ as 
$H_{n}^{-1}=\overline{H}_{n}=\left( 
\begin{array}{cc}
\overline{H}_{1\cdot }^{\prime } & \overline{H}_{2\cdot }^{\prime }%
\end{array}%
\right) ^{\prime }$, where $\overline{H}_{1\cdot }$ is $d_{1}\times d$ and $%
\overline{H}_{2\cdot }$ is $d_{2}\times d$. Assume that there exists a
positive constant $C_{\ast }$ such that $e_{\ell }^{\prime }\overline{H}%
_{2\cdot }^{\prime }\overline{H}_{2\cdot }e_{\ell }\geq C_{\ast }>0$ $a.s.$%
for all $n$ sufficiently large.

\medskip

\noindent \textbf{Corollary 3: }Let Assumptions 1-9 and 10* be satisfied;
and let $e_{\ell }$ be as defined in Assumption 10* above. Consider testing
the null hypothesis $H_{0}:e_{\ell }^{\prime }\delta _{0}=r$, using either
the t-statistic, $\mathbb{T}_{L}=\left( e_{\ell }^{\prime }\widehat{\delta }%
_{L}-r\right) /\sqrt{e_{\ell }^{\prime }\widehat{V}_{L}e_{\ell }}$ or the
t-statistic, $\mathbb{T}_{F}=\left( e_{\ell }^{\prime }\widehat{\delta }%
_{F}-r\right) /\sqrt{e_{\ell }^{\prime }\widehat{V}_{F}e_{\ell }}$.

\begin{enumerate}
\item[a.] For Case I, where $K_{2,n}/\left( \mu _{n}^{\min }\right)
^{2}=O\left( 1\right) $, the following results hold for $\ell \in \left\{
1,...,d\right\} $.

\begin{enumerate}
\item[(i)] Under $H_{0}:e_{\ell }^{\prime }\delta _{0}=r$, $\mathbb{T}_{L}%
\overset{d}{\rightarrow }N\left( 0,1\right) $ and $\mathbb{T}_{F}\overset{d}{%
\rightarrow }N\left( 0,1\right) $.

\item[(ii) ] Under $H_{1}:e_{\ell }^{\prime }\delta _{0}\neq r$, with
probability approaching one as $n\rightarrow \infty $, the following results
hold: $\mathbb{T}_{L}\rightarrow +\infty $ \ if $e_{\ell }^{\prime }\delta
_{0}>r$; $\mathbb{T}_{L}\rightarrow -\infty $ \ if $e_{\ell }^{\prime
}\delta _{0}<r$; $\mathbb{T}_{F}\rightarrow +\infty $\ if $e_{\ell }^{\prime
}\delta _{0}>r$; and $\mathbb{T}_{F}\rightarrow -\infty $ \ if $e_{\ell
}^{\prime }\delta _{0}<r$.
\end{enumerate}

\item[b.] For Case II, where $K_{2,n}/\left( \mu _{n}^{\min }\right)
^{2}\rightarrow \infty $ but $\sqrt{K_{2,n}}/\left( \mu _{n}^{\min }\right)
^{2}\rightarrow 0$, the following results hold, for $\ell \in \left\{
1,...,d\right\} $.

\begin{enumerate}
\item[(i)] Under $H_{0}:e_{\ell }^{\prime }\delta _{0}=r$, $\mathbb{T}_{L}%
\overset{d}{\rightarrow }N\left( 0,1\right) $ and $\mathbb{T}_{F}\overset{d}{%
\rightarrow }N\left( 0,1\right) $.

\item[(ii)] Under $H_{1}:e_{\ell }^{\prime }\delta _{0}\neq r$, with
probability approaching one as $n\rightarrow \infty $, the following results
hold: $\mathbb{T}_{L}\rightarrow +\infty $ \ if $e_{\ell }^{\prime }\delta
_{0}>r$; $\mathbb{T}_{L}\rightarrow -\infty $ \ if $e_{\ell }^{\prime
}\delta _{0}<r$; $\mathbb{T}_{F}\rightarrow +\infty $\ if $e_{\ell }^{\prime
}\delta _{0}>r$; and $\mathbb{T}_{F}\rightarrow -\infty $ \ if $e_{\ell
}^{\prime }\delta _{0}<r$.
\end{enumerate}
\end{enumerate}

\medskip

Note that writing $D_{\mu }$ in the way specified in equation (\ref{Dmu})
does not really lead to any loss of generality. In fact, a seemingly more
general $D_{\mu }$ matrix, where not all of the diagonal elements grow at
the same rate, as $n\rightarrow \infty $, can always be rewritten in the
form given in equation (\ref{Dmu}), via repermutation of the rows and
columns of $D_{\mu }$. To see this, suppose that $\mu _{1n},..,\mu
_{d_{1},n},\mu _{n}^{\min }$ are not ordered as in equation (\ref{Dmu}), so
that we have some diagonal matrix $D_{\mu }^{\ast },$ whose diagonal
elements are $\mu _{1n},..,\mu _{d_{1},n},\mu _{n}^{\min },$ but in some
other order. Then, there exists some permutation matrix $P$ such that $%
D_{\mu }=PD_{\mu }^{\ast }P^{\prime }$, where $D_{\mu }$ is the diagonal
matrix given in equation (\ref{Dmu}). Moreover, let the elements of $%
\widehat{\delta }^{\ast },\delta _{0}^{\ast }$, $c^{\ast }$, and $\widehat{V}%
^{\ast }$ be ordered in a way that is conformable with $D_{\mu }^{\ast },$
and let $\widehat{\delta },\delta _{0}$, $c$, and $\widehat{V}$ be the
corresponding vectors and matrix but with elements ordered conformably with $%
D_{\mu }$. Then, it is easy to see that $\widehat{\delta }=P\widehat{\delta }%
^{\ast }$, $\delta _{0}=P\delta _{0}^{\ast }$, $c=Pc^{\ast }$, $\widehat{V}=P%
\widehat{V}^{\ast }P^{\prime }$. Hence, by making use of these relations and
of the fact that $P$ is an orthogonal matrix, we further obtain that $%
\mathbb{T}_{L}^{\ast }=c^{\ast \prime }\left( \widehat{\delta }^{\ast
}-\delta _{0}^{\ast }\right) /\sqrt{c^{\ast \prime }\widehat{V}^{\ast
}c^{\ast }}=c^{\ast \prime }P^{\prime }P\left( \widehat{\delta }^{\ast
}-\delta _{0}^{\ast }\right) /\sqrt{c^{\ast \prime }P^{\prime }P\widehat{V}%
^{\ast }P^{\prime }Pc^{\ast }}=c^{\prime }\left( \widehat{\delta }-\delta
_{0}\right) /\sqrt{c^{\prime }\widehat{V}c}=\mathbb{T}_{L}$. It follows that
the value of the t-statistic is invariant to repermutation of the order of
the elements of $\widehat{\delta }$, $\delta _{0}$, $c$, and $\widehat{V}$,
so that the asymptotic distribution which we derive for $\mathbb{T}_{L}$,
under an assumed ordering of the elements of $\widehat{\delta }$, $\delta
_{0}$, $c$, and $\widehat{V}$ that is conformable with equation (\ref{Dmu})
will still apply, even if the t-statistic computed by the empirical
researcher is based on some other ordering.

Given that the representation of $D_{\mu }$ given in equation (\ref{Dmu})
does not result in any loss of generality, the only real restriction imposed
by Assumption 10* is the condition that $e_{\ell }^{\prime }\overline{H}%
_{2\cdot }^{\prime }\overline{H}_{2\cdot }e_{\ell }\geq C_{\ast }>0$ $a.s.n.$
for some positive constant $C_{\ast }$. We show in the proof of Corollary 3
that this latter condition implies the more general conditions given in
Assumption 10 if the null hypothesis we are testing involves only one
coefficient. It follows that hypotheses involving only one coefficient can
be tested under very general assumptions about the heterogeneity of
instrument weakness since the violation of the condition $e_{\ell }^{\prime }%
\overline{H}_{2\cdot }^{\prime }\overline{H}_{2\cdot }e_{\ell }\geq C_{\ast
}>0$ $a.s.n$ will only occur if the $\ell ^{th}$ column of $\overline{H}%
_{2\cdot }$ does not have a single nonzero entry, which seems unlikely in
most practical applications.

To date, papers in the weak instrument literature have focused primarily on
size control, with little attention paid to test consistency under weak
identification. One exception is a recent paper by Mikusheva and Sun (2021),
which shows that a condition similar to $\sqrt{K_{2,n}}/\left( \mu
_{n}^{\min }\right) ^{2}\rightarrow 0$ is both necessary and sufficient for
the existence of a consistent test. Interpreted in light of their result,
the results presented in Theorems 5 as well as Corollaries 2 and 3 above
prove that t-tests based on FELIM and FEFUL are consistent as long as
instruments are strong enough so that consistency in hypothesis testing is
possible. In contrast, t-tests based on estimators such as the 2SLS
estimator will only be consistent if $K_{2,n}/\left( \mu _{n}^{\min }\right)
^{2}\rightarrow 0$ (i.e., under stronger instruments). Test statistics based
on LIML also have undesirable properties under many weak instrument
asymptotics, when there is error heteroskedasticity. In addition, note that
one advantage of t-tests is that they are particularly easy to apply if one
is interested in testing against one-sided alternatives. The results of
Theorems 5 as well as Corollaries 2 and 3 show that, when the null
hypothesis is incorrect, t-tests based on FELIM and FEFUL diverge in the
direction of the true alternative, with probability approaching one, even in
situations where identification is weaker than that typically assumed under
standard large sample theory, provided of course that $\sqrt{K_{2,n}}/\left(
\mu _{n}^{\min }\right) ^{2}\rightarrow 0$. Hence, the test statistics
proposed in this paper should be particularly useful to empirical
researchers interested in testing whether an effect in a particular
direction is statistically significant.

\section{\noindent Monte Carlo Results}

In this section, we report some Monte Carlo results based on the following
data generating process:

\begin{eqnarray*}
y_{\left( i,t\right) } &=&\underset{1\times 1}{\delta }\underset{1\times 1}{%
x_{\left( i,t\right) }}+\underset{1\times 10}{\varphi ^{\prime }}\underset{%
10\times 1}{Z_{1,\left( i,t\right) }}+\alpha _{i}+\varepsilon _{\left(
i,t\right) }\text{,} \\
x_{\left( i,t\right) } &=&\underset{1\times 10}{\Phi ^{\prime }}\underset{%
10\times 1}{Z_{1,\left( i,t\right) }}+\underset{1\times K_{2}}{\Pi }\underset%
{K_{2}\times 1}{Z_{2,\left( i,t\right) }}+\xi _{i}+u_{\left( i,t\right) }%
\text{.}
\end{eqnarray*}%
where we specify $\varphi =\iota _{10}$, $\Phi =\iota _{10}$, and $\Pi
=\left( \iota _{K_{2}}\otimes \pi \right) $ with $\iota _{10}$ and $\iota
_{K_{2}}$ being, respectively, a $10\times 1$ and a $K_{2}\times 1$ vector
of ones. Here, $\pi $ is taken to be a scalar parameter, and we choose $\pi $
so that the concentration parameter $\mu ^{2}=25,$ $35,$ $45,$ and $55$.
Moreover, in our experiments, we consider two choices of $K_{2}$: $%
K_{2}=10,30$. Additionally, we set $n=200$ and $T_{i}=3,$ for each $i\in
\left\{ 1,2,...,200\right\} $, so that $m_{n}=600$. The $\left( i,t\right)
^{th}$ observation of the vector of included exogenous regressors, or
covariates, is taken to be $Z_{1,\left( i,t\right) }=\left( 
\begin{array}{ccccccc}
z_{1,\left( i,t\right) } & z_{1,\left( i,t\right) }^{2} & z_{1,\left(
i,t\right) }^{3} & z_{1,\left( i,t\right) }^{4} & z_{1,\left( i,t\right)
}D_{\left( i,t\right) ,1} & \cdots & z_{1,\left( i,t\right) }D_{\left(
i,t\right) ,6}%
\end{array}%
\right) ^{\prime }$, where $\left\{ z_{1,\left( i,t\right) }\right\}
_{\left( i,t\right) =1}^{600}\equiv i.i.d.N\left( 0,1\right) $ and where $%
D_{\left( i,t\right) ,k}\in \left\{ 0,1\right\} $ for $k\in \left\{
1,2,..,6\right\} $ is a binary variable such that $\Pr \left( D_{\left(
i,t\right) ,k}=1\right) =1/2,$ with $\left\{ D_{\left( i,t\right)
,k}\right\} $ specified to be independent across both $\left( i,t\right) $
and $k$. We also take $\left\{ Z_{2,\left( i,t\right) }\right\} _{\left(
i,t\right) =1}^{600}\equiv i.i.d.N\left( 0,I_{K_{2}}\right) $, $\left\{
u_{\left( i,t\right) }\right\} _{\left( i,t\right) =1}^{600}\equiv
i.i.d.N\left( 0,1\right) $, $\left\{ \alpha _{i}\right\}
_{i=1}^{200}i.i.d.N\left( 0,1\right) $, and $\left\{ \xi _{i}\right\}
_{i=1}^{200}i.i.d.N\left( 0,1\right) $; with $z_{1,\left( i,t\right) }$, $%
D_{\left( i,t\right) ,k}$, $Z_{2,\left( i,t\right) }$, $u_{\left( i,t\right)
}$, $\alpha _{i}$ and $\xi _{i}$ all specified to be independent of each
other. We allow the structural disturbance, $\varepsilon _{\left( i,t\right)
},$ to exhibit conditional heteroskedasticity in a manner similar to the
design given in Hausman et al. (2012). In particular, we let%
\begin{equation}
\varepsilon _{\left( i,t\right) }=\rho u_{\left( i,t\right) }+\sqrt{\frac{%
1-\rho ^{2}}{\phi ^{2}+\left( 0.86\right) ^{2}}}\left( \phi v_{1,\left(
i,t\right) }+0.86v_{2,\left( i,t\right) }\right) ,  \label{epsilon}
\end{equation}%
where $v_{1,\left( i,t\right) }|Z_{1,\left( i,t\right) },Z_{2,\left(
i,t\right) }\sim N\left( 0,\kappa \left[ 1+\left( \iota _{10}^{\prime
}Z_{1,\left( i,t\right) }+\iota _{K_{2}}^{\prime }Z_{2,\left( i,t\right)
}\right) ^{2}\right] \right) $ and $v_{2,\left( i,t\right) }\sim N\left(
0,1\right) $. Both of these distributions are specified to be independent
across the index $\left( i,t\right) $, and $\kappa $ is a normalization
constant chosen so that the unconditional variance, $Var\left( v_{1,\left(
i,t\right) }\right) $, is equal to 1. For all experiments reported below, we
set $\rho =0.3$ and choose the parameter $\phi $, so that the R-squared for
the regression of $\varepsilon ^{2}$ on the instruments and the included
exogenous variables take the values $0$, $0.1$, and $0.2$.

Our simulation study examines the finite sample properties of our three
estimators (FEJIV, FELIM, and FEFUL) and their associated t-statistics.
Additionally, we compare the performance of our estimators with the 2SLS
estimator, the IJIVE1 estimator originally proposed in Ackerberg and
Devereux (2009), the IJIVE2 estimator introduced in Evdokimov and Koles\'{a}%
r (2018), and the UJIVE estimator originally proposed in Koles\'{a}r (2013)
and further studied in Evdokimov and Koles\'{a}r (2018). The comparison of
these point estimators is made on the basis of median bias and nine decile
range. We also evaluate the associated t-statistics for these estimators on
the basis of size control, as measured by their rejection frequencies under
the null hypothesis $H_{0}:\delta =0$.

\noindent \qquad The results of our Monte Carlo study are reported in Tables
1-6 below.

\begin{tabular}{|l|l|l|l|l|l|l|l|l|}
\hline\hline
\multicolumn{9}{|c|}{Table 1: Median Bias, $K_{2}=10$} \\ \hline\hline
$\mu ^{2}$ & $\mathcal{R}_{\varepsilon ^{2}|z_{1}^{2}}^{2}$ & 2SLS & IJIVE1
& IJIVE2 & UJIVE & FEJIV & FELIM & FEFUL \\ \hline
& \multicolumn{1}{|c|}{0} & \multicolumn{1}{|c|}{0.1092} & 
\multicolumn{1}{|c|}{0.0440} & \multicolumn{1}{|c|}{0.0441} & 
\multicolumn{1}{|c|}{0.3811} & \multicolumn{1}{|c|}{-0.0058} & 
\multicolumn{1}{|c|}{0.0042} & \multicolumn{1}{|c|}{0.0161} \\ \cline{2-9}
25 & \multicolumn{1}{|c|}{0.1} & \multicolumn{1}{|c|}{0.1080} & 
\multicolumn{1}{|c|}{0.0429} & \multicolumn{1}{|c|}{0.0427} & 
\multicolumn{1}{|c|}{0.3719} & \multicolumn{1}{|c|}{-0.0071} & 
\multicolumn{1}{|c|}{0.0052} & \multicolumn{1}{|c|}{0.0167} \\ \cline{2-9}
& \multicolumn{1}{|c|}{0.2} & \multicolumn{1}{|c|}{0.1125} & 
\multicolumn{1}{|c|}{0.0475} & \multicolumn{1}{|c|}{0.0480} & 
\multicolumn{1}{|c|}{0.3982} & \multicolumn{1}{|c|}{-0.0053} & 
\multicolumn{1}{|c|}{0.0051} & \multicolumn{1}{|c|}{0.0170} \\ \hline
& \multicolumn{1}{|c|}{0} & \multicolumn{1}{|c|}{0.0857} & 
\multicolumn{1}{|c|}{0.0290} & \multicolumn{1}{|c|}{0.0290} & 
\multicolumn{1}{|c|}{0.2562} & \multicolumn{1}{|c|}{-0.0167} & 
\multicolumn{1}{|c|}{-0.0003} & \multicolumn{1}{|c|}{0.0090} \\ \cline{2-9}
35 & \multicolumn{1}{|c|}{0.1} & \multicolumn{1}{|c|}{0.0860} & 
\multicolumn{1}{|c|}{0.0300} & \multicolumn{1}{|c|}{0.0301} & 
\multicolumn{1}{|c|}{0.2617} & \multicolumn{1}{|c|}{-0.0127} & 
\multicolumn{1}{|c|}{0.0018} & \multicolumn{1}{|c|}{0.0107} \\ \cline{2-9}
& \multicolumn{1}{|c|}{0.2} & \multicolumn{1}{|c|}{0.0879} & 
\multicolumn{1}{|c|}{0.0328} & \multicolumn{1}{|c|}{0.0327} & 
\multicolumn{1}{|c|}{0.2486} & \multicolumn{1}{|c|}{-0.0105} & 
\multicolumn{1}{|c|}{-0.0002} & \multicolumn{1}{|c|}{0.0083} \\ \hline
& \multicolumn{1}{|c|}{0} & \multicolumn{1}{|c|}{0.0733} & 
\multicolumn{1}{|c|}{0.0263} & \multicolumn{1}{|c|}{0.0263} & 
\multicolumn{1}{|c|}{0.1943} & \multicolumn{1}{|c|}{-0.0095} & 
\multicolumn{1}{|c|}{0.0009} & \multicolumn{1}{|c|}{0.0079} \\ \cline{2-9}
45 & \multicolumn{1}{|c|}{0.1} & \multicolumn{1}{|c|}{0.0738} & 
\multicolumn{1}{|c|}{0.0280} & \multicolumn{1}{|c|}{0.0281} & 
\multicolumn{1}{|c|}{0.1984} & \multicolumn{1}{|c|}{-0.0072} & 
\multicolumn{1}{|c|}{0.0025} & \multicolumn{1}{|c|}{0.0094} \\ \cline{2-9}
& \multicolumn{1}{|c|}{0.2} & \multicolumn{1}{|c|}{0.0690} & 
\multicolumn{1}{|c|}{0.0213} & \multicolumn{1}{|c|}{0.0216} & 
\multicolumn{1}{|c|}{0.1908} & \multicolumn{1}{|c|}{-0.0130} & 
\multicolumn{1}{|c|}{-0.0007} & \multicolumn{1}{|c|}{0.0057} \\ \hline
& \multicolumn{1}{|c|}{0} & \multicolumn{1}{|c|}{0.0629} & 
\multicolumn{1}{|c|}{0.0210} & \multicolumn{1}{|c|}{0.0212} & 
\multicolumn{1}{|c|}{0.1586} & \multicolumn{1}{|c|}{-0.0074} & 
\multicolumn{1}{|c|}{0.0009} & \multicolumn{1}{|c|}{0.0068} \\ \cline{2-9}
55 & \multicolumn{1}{|c|}{0.1} & \multicolumn{1}{|c|}{0.0627} & 
\multicolumn{1}{|c|}{0.0205} & \multicolumn{1}{|c|}{0.0206} & 
\multicolumn{1}{|c|}{0.1415} & \multicolumn{1}{|c|}{-0.0084} & 
\multicolumn{1}{|c|}{0.0017} & \multicolumn{1}{|c|}{0.0071} \\ \cline{2-9}
& \multicolumn{1}{|c|}{0.2} & \multicolumn{1}{|c|}{0.0583} & 
\multicolumn{1}{|c|}{0.0167} & \multicolumn{1}{|c|}{0.0165} & 
\multicolumn{1}{|c|}{0.1429} & \multicolumn{1}{|c|}{-0.0136} & 
\multicolumn{1}{|c|}{-0.0041} & \multicolumn{1}{|c|}{0.0017} \\ \hline
\end{tabular}

{\small Results based on 10,000 simulations}

\begin{tabular}{|l|l|l|l|l|l|l|l|l|}
\hline\hline
\multicolumn{9}{|c|}{Table 2: Nine Decile Range 0.05 to 0.95\QQfnmark{%
By nine decile range we mean the range between the $0.05$ and the $0.95$
quantiles. It should also be noted that the reason we compare the estimators
based on median bias and nine decile range instead of the usual criteria of
(mean) bias and variance is because it is well-known that the exact finite
sample (mean) bias and variance of LIML-type estimators do not exist under
the assumption that errors are normally distributed. However, it is also
well-known that LIML-type estimators tend to be better centered than the
2SLS estimator in terms of median bias and, in many ways, have better finite
sample properties, in spite of the fact that they have fatter tails. Hence,
the use of median bias and nine decile range allow us to conduct a broader
based Monte Carlo comparison without restricting ourselves to only those
estimators whose positive integer moments are known to exist.}, $K_{2}=10$}
\\ \hline\hline
$\mu ^{2}$ & $\mathcal{R}_{\varepsilon ^{2}|z_{1}^{2}}^{2}$ & 2SLS & IJIVE1
& IJIVE2 & UJIVE & FEJIV & FELIM & FEFUL \\ \hline
& \multicolumn{1}{|c|}{0} & \multicolumn{1}{|c|}{0.6638} & 
\multicolumn{1}{|c|}{1.0286} & \multicolumn{1}{|c|}{1.0247} & 
\multicolumn{1}{|c|}{5.9842} & \multicolumn{1}{|c|}{1.5849} & 
\multicolumn{1}{|c|}{1.2900} & \multicolumn{1}{|c|}{1.1543} \\ \cline{2-9}
25 & \multicolumn{1}{|c|}{0.1} & \multicolumn{1}{|c|}{0.6590} & 
\multicolumn{1}{|c|}{1.0475} & \multicolumn{1}{|c|}{1.0468} & 
\multicolumn{1}{|c|}{6.0422} & \multicolumn{1}{|c|}{1.6538} & 
\multicolumn{1}{|c|}{1.2810} & \multicolumn{1}{|c|}{1.1442} \\ \cline{2-9}
& \multicolumn{1}{|c|}{0.2} & \multicolumn{1}{|c|}{0.6491} & 
\multicolumn{1}{|c|}{1.0235} & \multicolumn{1}{|c|}{1.0253} & 
\multicolumn{1}{|c|}{6.1624} & \multicolumn{1}{|c|}{1.6218} & 
\multicolumn{1}{|c|}{1.2788} & \multicolumn{1}{|c|}{1.1216} \\ \hline
& \multicolumn{1}{|c|}{0} & \multicolumn{1}{|c|}{0.5936} & 
\multicolumn{1}{|c|}{0.8214} & \multicolumn{1}{|c|}{0.8200} & 
\multicolumn{1}{|c|}{5.6334} & \multicolumn{1}{|c|}{1.0861} & 
\multicolumn{1}{|c|}{0.9554} & \multicolumn{1}{|c|}{0.8921} \\ \cline{2-9}
35 & \multicolumn{1}{|c|}{0.1} & \multicolumn{1}{|c|}{0.5952} & 
\multicolumn{1}{|c|}{0.8286} & \multicolumn{1}{|c|}{0.8288} & 
\multicolumn{1}{|c|}{5.8319} & \multicolumn{1}{|c|}{1.1026} & 
\multicolumn{1}{|c|}{0.9430} & \multicolumn{1}{|c|}{0.8876} \\ \cline{2-9}
& \multicolumn{1}{|c|}{0.2} & \multicolumn{1}{|c|}{0.5755} & 
\multicolumn{1}{|c|}{0.7955} & \multicolumn{1}{|c|}{0.7939} & 
\multicolumn{1}{|c|}{5.7128} & \multicolumn{1}{|c|}{1.0402} & 
\multicolumn{1}{|c|}{0.9068} & \multicolumn{1}{|c|}{0.8561} \\ \hline
& \multicolumn{1}{|c|}{0} & \multicolumn{1}{|c|}{0.5362} & 
\multicolumn{1}{|c|}{0.6960} & \multicolumn{1}{|c|}{0.6960} & 
\multicolumn{1}{|c|}{5.1587} & \multicolumn{1}{|c|}{0.8458} & 
\multicolumn{1}{|c|}{0.7769} & \multicolumn{1}{|c|}{0.7433} \\ \cline{2-9}
45 & \multicolumn{1}{|c|}{0.1} & \multicolumn{1}{|c|}{0.5332} & 
\multicolumn{1}{|c|}{0.6876} & \multicolumn{1}{|c|}{0.6883} & 
\multicolumn{1}{|c|}{5.2225} & \multicolumn{1}{|c|}{0.8378} & 
\multicolumn{1}{|c|}{0.7665} & \multicolumn{1}{|c|}{0.7392} \\ \cline{2-9}
& \multicolumn{1}{|c|}{0.2} & \multicolumn{1}{|c|}{0.5244} & 
\multicolumn{1}{|c|}{0.6751} & \multicolumn{1}{|c|}{0.6753} & 
\multicolumn{1}{|c|}{5.2851} & \multicolumn{1}{|c|}{0.8210} & 
\multicolumn{1}{|c|}{0.7542} & \multicolumn{1}{|c|}{0.7229} \\ \hline
& \multicolumn{1}{|c|}{0} & \multicolumn{1}{|c|}{0.4929} & 
\multicolumn{1}{|c|}{0.6109} & \multicolumn{1}{|c|}{0.6114} & 
\multicolumn{1}{|c|}{4.8115} & \multicolumn{1}{|c|}{0.7132} & 
\multicolumn{1}{|c|}{0.6620} & \multicolumn{1}{|c|}{0.6418} \\ \cline{2-9}
55 & \multicolumn{1}{|c|}{0.1} & \multicolumn{1}{|c|}{0.4929} & 
\multicolumn{1}{|c|}{0.6068} & \multicolumn{1}{|c|}{0.6069} & 
\multicolumn{1}{|c|}{4.7546} & \multicolumn{1}{|c|}{0.7076} & 
\multicolumn{1}{|c|}{0.6564} & \multicolumn{1}{|c|}{0.6387} \\ \cline{2-9}
& \multicolumn{1}{|c|}{0.2} & \multicolumn{1}{|c|}{0.4857} & 
\multicolumn{1}{|c|}{0.6039} & \multicolumn{1}{|c|}{0.6029} & 
\multicolumn{1}{|c|}{4.8027} & \multicolumn{1}{|c|}{0,6972} & 
\multicolumn{1}{|c|}{0.6465} & \multicolumn{1}{|c|}{0.6279} \\ \hline
\end{tabular}%
\QQfntext{0}{
By nine decile range we mean the range between the $0.05$ and the $0.95$
quantiles. It should also be noted that the reason we compare the estimators
based on median bias and nine decile range instead of the usual criteria of
(mean) bias and variance is because it is well-known that the exact finite
sample (mean) bias and variance of LIML-type estimators do not exist under
the assumption that errors are normally distributed. However, it is also
well-known that LIML-type estimators tend to be better centered than the
2SLS estimator in terms of median bias and, in many ways, have better finite
sample properties, in spite of the fact that they have fatter tails. Hence,
the use of median bias and nine decile range allow us to conduct a broader
based Monte Carlo comparison without restricting ourselves to only those
estimators whose positive integer moments are known to exist.}

{\small Results based on 10,000 simulations}

\begin{tabular}{|l|l|l|l|l|l|l|l|l|}
\hline\hline
\multicolumn{9}{|c|}{Table 3: 0.05 Rejection Frequencies\QQfnmark{%
See Ackerberg and Devereux (2009), Koles\'{a}r (2013), and Evdokimov and
Koles\'{a}r (2018) for formulae for the estimators IJIVE1, IJIVE2, and UJIVE
as well as for the standard errors used in constructing the t-statistics for
these estimators.}, $K_{2}=10$} \\ \hline\hline
$\mu ^{2}$ & $\mathcal{R}_{\varepsilon ^{2}|z_{1}^{2}}^{2}$ & 2SLS & IJIVE1
& IJIVE2 & UJIVE & FEJIV & FELIM & FEFUL \\ \hline
& \multicolumn{1}{|c|}{0} & \multicolumn{1}{|c|}{0.1784} & 
\multicolumn{1}{|c|}{0.0951} & \multicolumn{1}{|c|}{0.0884} & 
\multicolumn{1}{|c|}{0.5215} & \multicolumn{1}{|c|}{0.0253} & 
\multicolumn{1}{|c|}{0.0518} & \multicolumn{1}{|c|}{0.0535} \\ \cline{2-9}
25 & \multicolumn{1}{|c|}{0.1} & \multicolumn{1}{|c|}{0.1842} & 
\multicolumn{1}{|c|}{0.0997} & \multicolumn{1}{|c|}{0.0937} & 
\multicolumn{1}{|c|}{0.5181} & \multicolumn{1}{|c|}{0.0326} & 
\multicolumn{1}{|c|}{0.0555} & \multicolumn{1}{|c|}{0.0561} \\ \cline{2-9}
& \multicolumn{1}{|c|}{0.2} & \multicolumn{1}{|c|}{0.1797} & 
\multicolumn{1}{|c|}{0.0958} & \multicolumn{1}{|c|}{0.0896} & 
\multicolumn{1}{|c|}{0.5334} & \multicolumn{1}{|c|}{0.0275} & 
\multicolumn{1}{|c|}{0.0538} & \multicolumn{1}{|c|}{0.0547} \\ \hline
& \multicolumn{1}{|c|}{0} & \multicolumn{1}{|c|}{0.1659} & 
\multicolumn{1}{|c|}{0.1064} & \multicolumn{1}{|c|}{0.0999} & 
\multicolumn{1}{|c|}{0.5347} & \multicolumn{1}{|c|}{0.0319} & 
\multicolumn{1}{|c|}{0.0481} & \multicolumn{1}{|c|}{0.0506} \\ \cline{2-9}
35 & \multicolumn{1}{|c|}{0.1} & \multicolumn{1}{|c|}{0.1668} & 
\multicolumn{1}{|c|}{0.1017} & \multicolumn{1}{|c|}{0.0951} & 
\multicolumn{1}{|c|}{0.5345} & \multicolumn{1}{|c|}{0.0340} & 
\multicolumn{1}{|c|}{0.0489} & \multicolumn{1}{|c|}{0.0508} \\ \cline{2-9}
& \multicolumn{1}{|c|}{0.2} & \multicolumn{1}{|c|}{0.1677} & 
\multicolumn{1}{|c|}{0.1021} & \multicolumn{1}{|c|}{0.0951} & 
\multicolumn{1}{|c|}{0.5369} & \multicolumn{1}{|c|}{0.0326} & 
\multicolumn{1}{|c|}{0.0484} & \multicolumn{1}{|c|}{0.0511} \\ \hline
& \multicolumn{1}{|c|}{0} & \multicolumn{1}{|c|}{0.1584} & 
\multicolumn{1}{|c|}{0.1098} & \multicolumn{1}{|c|}{0.1034} & 
\multicolumn{1}{|c|}{0.5601} & \multicolumn{1}{|c|}{0.0354} & 
\multicolumn{1}{|c|}{0.0503} & \multicolumn{1}{|c|}{0.0528} \\ \cline{2-9}
45 & \multicolumn{1}{|c|}{0.1} & \multicolumn{1}{|c|}{0.1592} & 
\multicolumn{1}{|c|}{0.1087} & \multicolumn{1}{|c|}{0.1023} & 
\multicolumn{1}{|c|}{0.5555} & \multicolumn{1}{|c|}{0.0351} & 
\multicolumn{1}{|c|}{0.0469} & \multicolumn{1}{|c|}{0.0493} \\ \cline{2-9}
& \multicolumn{1}{|c|}{0.2} & \multicolumn{1}{|c|}{0.1611} & 
\multicolumn{1}{|c|}{0.1100} & \multicolumn{1}{|c|}{0.1042} & 
\multicolumn{1}{|c|}{0.5606} & \multicolumn{1}{|c|}{0.0350} & 
\multicolumn{1}{|c|}{0.0483} & \multicolumn{1}{|c|}{0.0504} \\ \hline
& \multicolumn{1}{|c|}{0} & \multicolumn{1}{|c|}{0.1544} & 
\multicolumn{1}{|c|}{0.1127} & \multicolumn{1}{|c|}{0.1053} & 
\multicolumn{1}{|c|}{0.5853} & \multicolumn{1}{|c|}{0.0398} & 
\multicolumn{1}{|c|}{0.0476} & \multicolumn{1}{|c|}{0.0496} \\ \cline{2-9}
55 & \multicolumn{1}{|c|}{0.1} & \multicolumn{1}{|c|}{0.1583} & 
\multicolumn{1}{|c|}{0.1157} & \multicolumn{1}{|c|}{0.1098} & 
\multicolumn{1}{|c|}{0.5835} & \multicolumn{1}{|c|}{0.0400} & 
\multicolumn{1}{|c|}{0.0547} & \multicolumn{1}{|c|}{0.0561} \\ \cline{2-9}
& \multicolumn{1}{|c|}{0.2} & \multicolumn{1}{|c|}{0.1510} & 
\multicolumn{1}{|c|}{0.1123} & \multicolumn{1}{|c|}{0.1048} & 
\multicolumn{1}{|c|}{0.5881} & \multicolumn{1}{|c|}{0.0401} & 
\multicolumn{1}{|c|}{0.0524} & \multicolumn{1}{|c|}{0.0550} \\ \hline
\end{tabular}%
\QQfntext{0}{
See Ackerberg and Devereux (2009), Koles\'{a}r (2013), and Evdokimov and
Koles\'{a}r (2018) for formulae for the estimators IJIVE1, IJIVE2, and UJIVE
as well as for the standard errors used in constructing the t-statistics for
these estimators.}

{\small Results based on 10,000 simulations}

\begin{tabular}{|l|l|l|l|l|l|l|l|l|}
\hline\hline
\multicolumn{9}{|c|}{Table 4: Median Bias, $K_{2}=30$} \\ \hline\hline
$\mu ^{2}$ & $\mathcal{R}_{\varepsilon ^{2}|z_{1}^{2}}^{2}$ & 2SLS & IJIVE1
& IJIVE2 & UJIVE & FEJIV & FELIM & FEFUL \\ \hline
& \multicolumn{1}{|c|}{0} & \multicolumn{1}{|c|}{0.1907} & 
\multicolumn{1}{|c|}{0.1105} & \multicolumn{1}{|c|}{0.1106} & 
\multicolumn{1}{|c|}{0.5648} & \multicolumn{1}{|c|}{0.0157} & 
\multicolumn{1}{|c|}{0.0042} & \multicolumn{1}{|c|}{0.0150} \\ \cline{2-9}
25 & \multicolumn{1}{|c|}{0.1} & \multicolumn{1}{|c|}{0.1916} & 
\multicolumn{1}{|c|}{0.1136} & \multicolumn{1}{|c|}{0.1138} & 
\multicolumn{1}{|c|}{0.5828} & \multicolumn{1}{|c|}{0.0197} & 
\multicolumn{1}{|c|}{0.0085} & \multicolumn{1}{|c|}{0.0217} \\ \cline{2-9}
& \multicolumn{1}{|c|}{0.2} & \multicolumn{1}{|c|}{0.1933} & 
\multicolumn{1}{|c|}{0.1159} & \multicolumn{1}{|c|}{0.1160} & 
\multicolumn{1}{|c|}{0.5890} & \multicolumn{1}{|c|}{0.0287} & 
\multicolumn{1}{|c|}{0.0076} & \multicolumn{1}{|c|}{0.0191} \\ \hline
& \multicolumn{1}{|c|}{0} & \multicolumn{1}{|c|}{0.1702} & 
\multicolumn{1}{|c|}{0.0954} & \multicolumn{1}{|c|}{0.0954} & 
\multicolumn{1}{|c|}{0.4059} & \multicolumn{1}{|c|}{0.0069} & 
\multicolumn{1}{|c|}{0.0067} & \multicolumn{1}{|c|}{0.0150} \\ \cline{2-9}
35 & \multicolumn{1}{|c|}{0.1} & \multicolumn{1}{|c|}{0.1666} & 
\multicolumn{1}{|c|}{0.0900} & \multicolumn{1}{|c|}{0.0901} & 
\multicolumn{1}{|c|}{0.4075} & \multicolumn{1}{|c|}{-0.0097} & 
\multicolumn{1}{|c|}{-0.0023} & \multicolumn{1}{|c|}{0.0061} \\ \cline{2-9}
& \multicolumn{1}{|c|}{0.2} & \multicolumn{1}{|c|}{0.1699} & 
\multicolumn{1}{|c|}{0.0946} & \multicolumn{1}{|c|}{0.0941} & 
\multicolumn{1}{|c|}{0.4190} & \multicolumn{1}{|c|}{-0.0025} & 
\multicolumn{1}{|c|}{0.0032} & \multicolumn{1}{|c|}{0.0124} \\ \hline
& \multicolumn{1}{|c|}{0} & \multicolumn{1}{|c|}{0.1501} & 
\multicolumn{1}{|c|}{0.0764} & \multicolumn{1}{|c|}{0.0763} & 
\multicolumn{1}{|c|}{0.2939} & \multicolumn{1}{|c|}{-0.0050} & 
\multicolumn{1}{|c|}{0.0010} & \multicolumn{1}{|c|}{0.0079} \\ \cline{2-9}
45 & \multicolumn{1}{|c|}{0.1} & \multicolumn{1}{|c|}{0.1501} & 
\multicolumn{1}{|c|}{0.0789} & \multicolumn{1}{|c|}{0.0788} & 
\multicolumn{1}{|c|}{0.2928} & \multicolumn{1}{|c|}{-0.0079} & 
\multicolumn{1}{|c|}{-0.0017} & \multicolumn{1}{|c|}{0.0051} \\ \cline{2-9}
& \multicolumn{1}{|c|}{0.2} & \multicolumn{1}{|c|}{0.1502} & 
\multicolumn{1}{|c|}{0.0775} & \multicolumn{1}{|c|}{0.0778} & 
\multicolumn{1}{|c|}{0.2820} & \multicolumn{1}{|c|}{-0.0065} & 
\multicolumn{1}{|c|}{-0.0001} & \multicolumn{1}{|c|}{0.0057} \\ \hline
& \multicolumn{1}{|c|}{0} & \multicolumn{1}{|c|}{0.1357} & 
\multicolumn{1}{|c|}{0.0670} & \multicolumn{1}{|c|}{0.0672} & 
\multicolumn{1}{|c|}{0.2420} & \multicolumn{1}{|c|}{-0.0100} & 
\multicolumn{1}{|c|}{-0.0006} & \multicolumn{1}{|c|}{0.0051} \\ \cline{2-9}
55 & \multicolumn{1}{|c|}{0.1} & \multicolumn{1}{|c|}{0.1335} & 
\multicolumn{1}{|c|}{0.0641} & \multicolumn{1}{|c|}{0.0642} & 
\multicolumn{1}{|c|}{0.2202} & \multicolumn{1}{|c|}{-0.0141} & 
\multicolumn{1}{|c|}{-0.0078} & \multicolumn{1}{|c|}{-0.0026} \\ \cline{2-9}
& \multicolumn{1}{|c|}{0.2} & \multicolumn{1}{|c|}{0.1365} & 
\multicolumn{1}{|c|}{0.0679} & \multicolumn{1}{|c|}{0.0682} & 
\multicolumn{1}{|c|}{0.2246} & \multicolumn{1}{|c|}{-0.0031} & 
\multicolumn{1}{|c|}{0.0034} & \multicolumn{1}{|c|}{0.0092} \\ \hline
\end{tabular}

{\small Results based on 10,000 simulations}

\begin{tabular}{|l|l|l|l|l|l|l|l|l|}
\hline\hline
\multicolumn{9}{|c|}{Table 5: Nine Decile Range 0.05 to 0.95, $K_{2}=30$} \\ 
\hline\hline
$\mu ^{2}$ & $\mathcal{R}_{\varepsilon ^{2}|z_{1}^{2}}^{2}$ & 2SLS & IJIVE1
& IJIVE2 & UJIVE & FEJIV & FELIM & FEFUL \\ \hline
& \multicolumn{1}{|c|}{0} & \multicolumn{1}{|c|}{0.4785} & 
\multicolumn{1}{|c|}{0.9885} & \multicolumn{1}{|c|}{0.9909} & 
\multicolumn{1}{|c|}{5.7669} & \multicolumn{1}{|c|}{3.0848} & 
\multicolumn{1}{|c|}{2.1434} & \multicolumn{1}{|c|}{1.7483} \\ \cline{2-9}
25 & \multicolumn{1}{|c|}{0.1} & \multicolumn{1}{|c|}{0.4760} & 
\multicolumn{1}{|c|}{0.9629} & \multicolumn{1}{|c|}{0.9634} & 
\multicolumn{1}{|c|}{6.2210} & \multicolumn{1}{|c|}{3.0543} & 
\multicolumn{1}{|c|}{2.1265} & \multicolumn{1}{|c|}{1.7549} \\ \cline{2-9}
& \multicolumn{1}{|c|}{0.2} & \multicolumn{1}{|c|}{0.4693} & 
\multicolumn{1}{|c|}{0.9735} & \multicolumn{1}{|c|}{0.9764} & 
\multicolumn{1}{|c|}{6.0341} & \multicolumn{1}{|c|}{2.9121} & 
\multicolumn{1}{|c|}{2.1675} & \multicolumn{1}{|c|}{1.7602} \\ \hline
& \multicolumn{1}{|c|}{0} & \multicolumn{1}{|c|}{0.4501} & 
\multicolumn{1}{|c|}{0.8155} & \multicolumn{1}{|c|}{0.8175} & 
\multicolumn{1}{|c|}{6.4148} & \multicolumn{1}{|c|}{1.7734} & 
\multicolumn{1}{|c|}{1.4271} & \multicolumn{1}{|c|}{1.2895} \\ \cline{2-9}
35 & \multicolumn{1}{|c|}{0.1} & \multicolumn{1}{|c|}{0.4513} & 
\multicolumn{1}{|c|}{0.8083} & \multicolumn{1}{|c|}{0.8109} & 
\multicolumn{1}{|c|}{6.2439} & \multicolumn{1}{|c|}{1.8113} & 
\multicolumn{1}{|c|}{1.4544} & \multicolumn{1}{|c|}{1.3066} \\ \cline{2-9}
& \multicolumn{1}{|c|}{0.2} & \multicolumn{1}{|c|}{0.4427} & 
\multicolumn{1}{|c|}{0.7871} & \multicolumn{1}{|c|}{0.7899} & 
\multicolumn{1}{|c|}{6.1090} & \multicolumn{1}{|c|}{1.7457} & 
\multicolumn{1}{|c|}{1.3613} & \multicolumn{1}{|c|}{1.2405} \\ \hline
& \multicolumn{1}{|c|}{0} & \multicolumn{1}{|c|}{0.4186} & 
\multicolumn{1}{|c|}{0.6941} & \multicolumn{1}{|c|}{0.6939} & 
\multicolumn{1}{|c|}{5.8258} & \multicolumn{1}{|c|}{1.2562} & 
\multicolumn{1}{|c|}{1.0510} & \multicolumn{1}{|c|}{0.9935} \\ \cline{2-9}
45 & \multicolumn{1}{|c|}{0.1} & \multicolumn{1}{|c|}{0.4254} & 
\multicolumn{1}{|c|}{0.6958} & \multicolumn{1}{|c|}{0.6969} & 
\multicolumn{1}{|c|}{5.9272} & \multicolumn{1}{|c|}{1.2409} & 
\multicolumn{1}{|c|}{1.0471} & \multicolumn{1}{|c|}{0.9948} \\ \cline{2-9}
& \multicolumn{1}{|c|}{0.2} & \multicolumn{1}{|c|}{0.4186} & 
\multicolumn{1}{|c|}{0.6771} & \multicolumn{1}{|c|}{0.6779} & 
\multicolumn{1}{|c|}{5.9727} & \multicolumn{1}{|c|}{1.2126} & 
\multicolumn{1}{|c|}{1.0306} & \multicolumn{1}{|c|}{0.9764} \\ \hline
& \multicolumn{1}{|c|}{0} & \multicolumn{1}{|c|}{0.4008} & 
\multicolumn{1}{|c|}{0.6206} & \multicolumn{1}{|c|}{0.6211} & 
\multicolumn{1}{|c|}{5.7132} & \multicolumn{1}{|c|}{0.9825} & 
\multicolumn{1}{|c|}{0.8625} & \multicolumn{1}{|c|}{0.8287} \\ \cline{2-9}
55 & \multicolumn{1}{|c|}{0.1} & \multicolumn{1}{|c|}{0.3985} & 
\multicolumn{1}{|c|}{0.6087} & \multicolumn{1}{|c|}{0.6109} & 
\multicolumn{1}{|c|}{5.5675} & \multicolumn{1}{|c|}{0.9513} & 
\multicolumn{1}{|c|}{0.8614} & \multicolumn{1}{|c|}{0.8299} \\ \cline{2-9}
& \multicolumn{1}{|c|}{0.2} & \multicolumn{1}{|c|}{0.4028} & 
\multicolumn{1}{|c|}{0.6196} & \multicolumn{1}{|c|}{0.6214} & 
\multicolumn{1}{|c|}{5.4996} & \multicolumn{1}{|c|}{0.9661} & 
\multicolumn{1}{|c|}{0.8661} & \multicolumn{1}{|c|}{0.8354} \\ \hline
\end{tabular}

{\small Results based on 10,000 simulations}

\begin{tabular}{|l|l|l|l|l|l|l|l|l|}
\hline\hline
\multicolumn{9}{|c|}{Table 6: 0.05 Rejection Frequencies, $K_{2}=30$} \\ 
\hline\hline
$\mu ^{2}$ & $\mathcal{R}_{\varepsilon ^{2}|z_{1}^{2}}^{2}$ & 2SLS & IJIVE1
& IJIVE2 & UJIVE & FEJIV & FELIM & FEFUL \\ \hline
& \multicolumn{1}{|c|}{0} & \multicolumn{1}{|c|}{0.4113} & 
\multicolumn{1}{|c|}{0.1387} & \multicolumn{1}{|c|}{0.1214} & 
\multicolumn{1}{|c|}{0.5461} & \multicolumn{1}{|c|}{0.0249} & 
\multicolumn{1}{|c|}{0.0519} & \multicolumn{1}{|c|}{0.0534} \\ \cline{2-9}
25 & \multicolumn{1}{|c|}{0.1} & \multicolumn{1}{|c|}{0.4242} & 
\multicolumn{1}{|c|}{0.1425} & \multicolumn{1}{|c|}{0.1226} & 
\multicolumn{1}{|c|}{0.5489} & \multicolumn{1}{|c|}{0.0220} & 
\multicolumn{1}{|c|}{0.0518} & \multicolumn{1}{|c|}{0.0545} \\ \cline{2-9}
& \multicolumn{1}{|c|}{0.2} & \multicolumn{1}{|c|}{0.4350} & 
\multicolumn{1}{|c|}{0.1466} & \multicolumn{1}{|c|}{0.1266} & 
\multicolumn{1}{|c|}{0.5527} & \multicolumn{1}{|c|}{0.0251} & 
\multicolumn{1}{|c|}{0.0546} & \multicolumn{1}{|c|}{0.0565} \\ \hline
& \multicolumn{1}{|c|}{0} & \multicolumn{1}{|c|}{0.3919} & 
\multicolumn{1}{|c|}{0.1526} & \multicolumn{1}{|c|}{0.1310} & 
\multicolumn{1}{|c|}{0.5387} & \multicolumn{1}{|c|}{0.0315} & 
\multicolumn{1}{|c|}{0.0531} & \multicolumn{1}{|c|}{0.0553} \\ \cline{2-9}
35 & \multicolumn{1}{|c|}{0.1} & \multicolumn{1}{|c|}{0.3901} & 
\multicolumn{1}{|c|}{0.1527} & \multicolumn{1}{|c|}{0.1333} & 
\multicolumn{1}{|c|}{0.5355} & \multicolumn{1}{|c|}{0.0298} & 
\multicolumn{1}{|c|}{0.0577} & \multicolumn{1}{|c|}{0.0601} \\ \cline{2-9}
& \multicolumn{1}{|c|}{0.2} & \multicolumn{1}{|c|}{0.4015} & 
\multicolumn{1}{|c|}{0.1535} & \multicolumn{1}{|c|}{0.1338} & 
\multicolumn{1}{|c|}{0.5489} & \multicolumn{1}{|c|}{0.0329} & 
\multicolumn{1}{|c|}{0.0572} & \multicolumn{1}{|c|}{0.0604} \\ \hline
& \multicolumn{1}{|c|}{0} & \multicolumn{1}{|c|}{0.3624} & 
\multicolumn{1}{|c|}{0.1563} & \multicolumn{1}{|c|}{0.1362} & 
\multicolumn{1}{|c|}{0.5516} & \multicolumn{1}{|c|}{0.0339} & 
\multicolumn{1}{|c|}{0.0539} & \multicolumn{1}{|c|}{0.0559} \\ \cline{2-9}
45 & \multicolumn{1}{|c|}{0.1} & \multicolumn{1}{|c|}{0.3639} & 
\multicolumn{1}{|c|}{0.1542} & \multicolumn{1}{|c|}{0.1339} & 
\multicolumn{1}{|c|}{0.5396} & \multicolumn{1}{|c|}{0.0357} & 
\multicolumn{1}{|c|}{0.0542} & \multicolumn{1}{|c|}{0.0564} \\ \cline{2-9}
& \multicolumn{1}{|c|}{0.2} & \multicolumn{1}{|c|}{0.3764} & 
\multicolumn{1}{|c|}{0.1551} & \multicolumn{1}{|c|}{0.1344} & 
\multicolumn{1}{|c|}{0.5459} & \multicolumn{1}{|c|}{0.0370} & 
\multicolumn{1}{|c|}{0.0579} & \multicolumn{1}{|c|}{0.0601} \\ \hline
& \multicolumn{1}{|c|}{0} & \multicolumn{1}{|c|}{0.3376} & 
\multicolumn{1}{|c|}{0.1485} & \multicolumn{1}{|c|}{0.1294} & 
\multicolumn{1}{|c|}{0.5676} & \multicolumn{1}{|c|}{0.0385} & 
\multicolumn{1}{|c|}{0.0514} & \multicolumn{1}{|c|}{0.0541} \\ \cline{2-9}
55 & \multicolumn{1}{|c|}{0.1} & \multicolumn{1}{|c|}{0.3332} & 
\multicolumn{1}{|c|}{0.1455} & \multicolumn{1}{|c|}{0.1277} & 
\multicolumn{1}{|c|}{0.5638} & \multicolumn{1}{|c|}{0.0371} & 
\multicolumn{1}{|c|}{0.0534} & \multicolumn{1}{|c|}{0.0558} \\ \cline{2-9}
& \multicolumn{1}{|c|}{0.2} & \multicolumn{1}{|c|}{0.3530} & 
\multicolumn{1}{|c|}{0.1638} & \multicolumn{1}{|c|}{0.1421} & 
\multicolumn{1}{|c|}{0.5686} & \multicolumn{1}{|c|}{0.0417} & 
\multicolumn{1}{|c|}{0.0593} & \multicolumn{1}{|c|}{0.0605} \\ \hline
\end{tabular}

{\small Results based on 10,000 simulations}

Looking over the results reported in Tables 1-6, note first that, in terms
of median bias, the performance of FEJIV, FELIM, and FEFUL are uniformly
better across our experiments when compared to 2SLS, IJIVE1, IJIVE2, and
UJIVE; although our experiments do show 2SLS, IJIVE1, and IJIVE2 to be less
dispersed than the three estimators proposed in this paper. Comparing FELIM
and FEFUL in terms of nine decile range, we see that FEFUL tends to be less
dispersed than FELIM, which is in accord with the motivation behind the
original Fuller (1977) modification. Perhaps the most notable difference in
performance is that t-statistics based on FELIM\ and FEFUL have much less
size distortion than t-statistics constructed from any of the other five
estimators. Finally, note that t-statistics based on the FEJIV estimator
tend to be undersized, but the empirical rejection frequencies are still
closer to the nominal level than t-statistics based on 2SLS, IJIVE1, IJIVE2,
or UJIVE.

\section{Conclusion}

This paper considers an IV regression model with many weak instruments,
cluster specific effects, error heteroskedasticity, and possibly many
included exogenous regressors. To carry out point estimation in this setup,
we propose three new jackknife-type IV estimators, which we refer to by the
acronyms FEJIV, FELIM, and FEFUL. All three of these estimators are shown to
be robust to the effects of many weak instruments, in the sense that they
are consistent estimators within a framework broad enough to include both
the standard situation with strong instruments and situations with many weak
instruments. ~To the best of our knowledge, the estimators proposed in this
paper are the first consistent estimators which have been developed in a
many weak instrument framework when the IV regression under consideration
has both cluster specific effects and possibly many included exogenous
regressors. We establish asymptotic normality for FELIM and FEFUL under both
strong instrument and many weak instrument asymptotics. In addition, we
provide consistent standard errors for our estimators and show that, when
the null hypothesis is true, t-statistics based on these standard errors are
asymptotically normal under both strong instrument and many weak instrument
asymptotics. Finally, we show that under both strong instrument and many
weak instrument asymptotics, the t-statistics based on these standard errors
are consistent under fixed alternatives. Thus, we underscore an interesting
aspect of the many weak instrument setup. Namely, test consistency is still
possible under this framework, as has been pointed out in a recent paper by
Mikusheva and Sun (2021). In a series of Monte Carlo experiments, we find
that t-statistics based on FELIM and FEFUL control size better in finite
samples than t-statistics based on alternative jackknife-type IV estimators
that have previously been proposed in the literature. Hence, based on the
findings of this paper, we recommend that either FELIM or FEFUL be used in
settings where there are many weak instruments, cluster specific effects,
and possibly many included exogenous regressors.

\begin{thebibliography}{99}
\bibitem{} Ackerberg, D.A. and P.J. Devereux (2009): Improved JIVE
Estimators for Overidentified Linear Models with and without
Heteroskedasticity,\ Review of Economics and Statistics, 91, 351-362.

\bibitem{} Angrist, J.D., G.W. Imbens, and A. Krueger (1999): Jackknife
Instrumental Variables Estimation,\ Journal of Applied Econometrics, 14,
57-67.

\bibitem{} Antoine, B. and E. Renault (2012): Efficient Minimum Distance
Estimation with Multiple Rates of Convergence, Journal of Econometrics, 170,
350-357.

\bibitem{} Bekker, P. A. (1994): Alternative Approximations to the
Distributions of Instrumental Variables Estimators,\ Econometrica, 62,
657-681.

\bibitem{} Bekker, P. A. and F. Crudu (2015): Jackknife Instrumental
Variable Estimation with Heteroskedasticity,\ Journal of Econometrics, 185,
332-342.

\bibitem{} Blomquist, S. and M. Dahlberg (1999): Small Sample Properties of
LIML and Jackknife IV Estimators: Experiments with Weak Instruments,\
Journal of Applied Econometrics, 14, 69-88.

\bibitem{} Cattaneo, M. D., M. Jansson, and W.K. Newey (2018): Inference \
in Linear Regression Models with Many Covariates and Heteroskedasticity,
Journal of the American Statistical\textit{\ }Association, 113, 1350-1361.

\bibitem{} Chao, J. C. and N. R. Swanson (2004): Estimation and Testing
Using Jackknife IV in Heteroskedastic Regressions with Many Weak
Instruments, Unpublished Manuscript, University of Maryland and Rutgers
University.

\bibitem{} Chao, J. C. and N. R. Swanson (2005): Consistent \ Estimation
with a Large Number of Weak Instruments,\ Econometrica, 73, 1673-1692.

\bibitem{} Chao, J.C., \ N. R. Swanson, J. A. Hausman, W. K. Newey, and T.
Woutersen (2012): Asymptotic Distribution of JIVE in a Heteroskedastic IV
Regression with Many Instruments,\ Econometric Theory, 28, 42-86.

\bibitem{} Crudu, F., G. Mellace. and Z. S\'{a}ndor (2020): Inference in
Instrumental Variables Models with Heteroskedasticity and Many Instruments,\
Econometric Theory, forthcoming.

\bibitem{} Davidson, R. and J. G. MacKinnon (2006): The Case Against JIVE,
Journal of Applied Econometrics, 21, 827--833.

\bibitem{} Evdokimov, K. S. and M. Koles\'{a}r (2018): Inference in
Instrumental Variables Analysis with Heterogeneous Treatment Effects,
Unpublished Manuscript,\ Princeton University.

\bibitem{} Fuller, W. A. (1977): Some Properties of a Modification of the
Limited Information Estimator, Econometrica, 45, 939-954.

\bibitem{} Hansen, C., J. A. Hausman, and W. K. Newey (2008): Estimation
with Many Instrumental Variables,\ Journal of Business \& Economic
Statistics, 26, 398-422.

\bibitem{} Hastie, T. and R. Tibshirani (1990): Generalized Additive Models.
London: Chapman and Hall.

\bibitem{} Hausman, J. A., W. K. Newey, T. Woutersen, J. C. Chao, and N. R.
Swanson (2012): Instrumental Variable Estimation with Heteroskedasticity and
Many Instruments, Quantitative\textit{\ }Economics, 3, 211-255.

\bibitem{} Hsiao, C. and Q. Zhou (2018): JIVE for Panel Dynamic Simultaneous
Equations Models,\ Econometric Theory, 34, 1325-1369.

\bibitem{} Kleibergen, F. (2002): Pivotal Statistics for Testing Structural
Parameters in Instrumental Variables Regression, Econometrica, 70, 1781-1803.

\bibitem{} Koles\'{a}r, M. (2013): Estimation in Instrumental Variables
Models with Treatment Effect Heterogeneity, Unpublished Manuscript,\
Princeton University.

\bibitem{} Mikusheva, A. and L. Sun (2021): Inference with Many Weak
Instruments, Review of Economic Studies,\ forthcoming.

\bibitem{} Morimune, K. (1983): Approximate Distributions of k-Class
Estimators When the Degree of Overidentifiability Is Large Compared with the
Sample Size,\ Econometrica, 51, 821-841.

\bibitem{} Newey, W. K. (1997): Convergence Rates and Asymptotic Normality
for Series Estimators, Journal of Econometrics, 79, 147-168.

\bibitem{} Phillips, G. D. A. and C. Hale (1977): The Bias of Instrumental
Variable Estimators of Simultaneous Equation Systems, International Economic
Review, 18, 219-228.

\bibitem{} Phillips, P. C. B. (1983). Exact Small Sample Theory in the
Simultaneous Equations Model, in Handbook of Econometrics, Vol. I, ed. by Z.
Griliches and M.D. Intriligator. Amsterdam: North-Holland, 449-516.

\bibitem{} Rothenberg, T. (1984): Approximating the Distributions of
Econometric Estimators and Test Statistics,\ in Handbook of Econometrics,
Vol. II, ed. by Z. Griliches and M.D. Intriligator. Amsterdam:
North-Holland, 881-935.

\bibitem{} Staiger, D. and J. H. Stock (1997): Instrumental Variables
Regression with Weak Instruments, Econometrica, 65, 557-586.

\bibitem{} Stock, J. \ H. and M. Yogo (2005): Asymptotic Distributions of
Instrumental Variables Statistics with Many Instruments,\ in Identification
and Inference for Econometric Models: Essays in\textbf{\ }Honor of Thomas
Rothenberg, ed. by D.W.K. Andrews and J. H. Stock, Chapter 6. Cambridge:
Cambridge University Press.

\bibitem{} Tao, T. (2012): Topics in Random Matrix Theory. Graduate Studies
in Mathematics, 132. Providence, RI: American Mathematical Society.
\end{thebibliography}

\section{Appendix: Proofs of Main Theorems and Other Key Results}

\noindent This appendix provides the proofs for Theorem 1, Corollary 1,
Theorems 4-5, and Corollaries 2-3 of the paper. The proofs of Theorems 2 and
3 are longer and, thus, are given in Appendix S1 of a Supplemental Appendix
to this paper. This Supplemental Appendix can be viewed at the \noindent URL:

\noindent http://econweb.umd.edu/\symbol{126}chao/Research/research%
\_files/Supplemental\_Appendix\_to\_Jackknife

\noindent \_Estimation\_Cluster\_Sample\_IV\_Model\_December\_20\_2022.pdf.
In addition, the proofs provided below rely on a number of technical results
that are stated without proof in Appendix S2 of the Supplemental Appendix.
These results are designated in the derivations that follow by the use of
the prefix S. So, for example, Lemma S2-2 will refer to the second lemma in
Appendix S2 of the Supplemental Appendix. \vspace{0.08in}Proofs for these
additional supporting lemmas (more specifically, Lemmas S2-1 to S2-18) are
available in a separate online appendix which can be viewed at the URL:
\noindent 

\noindent http://econweb.umd.edu/\symbol{126}chao/Research/research%
\_files/Additional\_Online\_Appendix\_

\noindent
Jackknife\_Estimation\_Cluster\_Sample\_IV\_Model\_December\_20\_2022.pdf

\noindent \textbf{Proof of Theorem 1:}

To proceed, note first that, by parts (a) and (b) of Lemma S2-2 and by the
assumption on $\overline{\ell }_{n}$, we have $D_{\mu }^{-1}X^{\prime }\left[
A-\overline{\ell }_{n}M^{\left( Z_{1},Q\right) }\right] XD_{\mu
}^{-1}=D_{\mu }^{-1}X^{\prime }AXD_{\mu }^{-1}-\overline{\ell }_{n}D_{\mu
}^{-1}X^{\prime }M^{\left( Z_{1},Q\right) }XD_{\mu }^{-1}=H_{n}+o_{p}\left(
1\right) $, where $H_{n}=\Upsilon ^{\prime }Z_{2}^{\prime }M^{\left(
Z_{1},Q\right) }Z_{2}\Upsilon /n=O_{p}\left( 1\right) $. By Assumption
3(iii), we also have that $H_{n}$ is positive definite almost surely for $n$
sufficiently large, so that $D_{\mu }^{-1}X^{\prime }\left[ A-\overline{\ell 
}_{n}M^{\left( Z_{1},Q\right) }\right] XD_{\mu }^{-1}$ is invertible
w.p.a.1. Hence, w.p.a.1., we can write%
\begin{equation*}
\frac{1}{\mu _{n}^{\min }}D_{\mu }\left( \overline{\delta }_{n}-\delta
_{0}\right) =\left( D_{\mu }^{-1}X^{\prime }\left[ A-\overline{\ell }%
_{n}M^{\left( Z_{1},Q\right) }\right] XD_{\mu }^{-1}\right) ^{-1}\frac{1}{%
\mu _{n}^{\min }}D_{\mu }^{-1}X^{\prime }\left[ A-\overline{\ell }%
_{n}M^{\left( Z_{1},Q\right) }\right] \varepsilon .
\end{equation*}%
Applying Lemma S2-4 and Lemma S2-5, we get%
\begin{eqnarray*}
\frac{1}{\mu _{n}^{\min }}D_{\mu }^{-1}X^{\prime }\left[ A-\overline{\ell }%
_{n}M^{\left( Z_{1},Q\right) }\right] \varepsilon &=&\frac{1}{\mu _{n}^{\min
}}D_{\mu }^{-1}X^{\prime }A\varepsilon -\overline{\ell }_{n}\frac{1}{\mu
_{n}^{\min }}D_{\mu }^{-1}X^{\prime }M^{\left( Z_{1},Q\right) }\varepsilon \\
&=&O_{p}\left( \max \left\{ \frac{1}{\mu _{n}^{\min }},\frac{\sqrt{K_{2,n}}}{%
\left( \mu _{n}^{\min }\right) ^{2}}\right\} \right) +o_{p}\left( 1\right)
=o_{p}\left( 1\right) .
\end{eqnarray*}%
It follows by the Slutsky's Theorem that $\left\Vert D_{\mu }\left( 
\overline{\delta }_{n}-\delta _{0}\right) /\left( \mu _{n}^{\min }\right)
\right\Vert _{2}=o_{p}\left( 1\right) $, which gives the first result. To
show the second result, note that, by straightforward calculations, we obtain

\noindent $\left\Vert D_{\mu }\left( \overline{\delta }_{n}-\delta
_{0}\right) /\left( \mu _{n}^{\min }\right) \right\Vert _{2}\geq \sqrt{%
\left( \mu _{n}^{\min }\right) ^{2}/\left( \mu _{n}^{\min }\right) ^{2}}%
\sqrt{\left( \overline{\delta }_{n}-\delta _{0}\right) ^{\prime }\left( 
\overline{\delta }_{n}-\delta _{0}\right) }=\left\Vert \overline{\delta }%
_{n}-\delta _{0}\right\Vert _{2}$, which implies that $\left\Vert \overline{%
\delta }_{n}-\delta _{0}\right\Vert _{2}\overset{p}{\rightarrow }0$, as
required. $\square $

\bigskip

\noindent \textbf{Proof of Corollary 1:}

In light of the results given in Theorem 1, it suffices that we verify the
condition $\overline{\ell }_{n}=o_{p}\left( \left[ \mu _{n}^{\min }\right]
^{2}/n\right) =o_{p}\left( 1\right) $ for all three estimators. For the
FEJIV estimator considered in part (a), $\overline{\ell }_{n}=0$ for all $n$%
, so this condition is trivially satisfied. Now, part (b) considers the
FELIM estimator. For this estimator, the result of Lemma S2-11 has shown
that we can take $\overline{\ell }_{n}=\widehat{\ell }_{L,n}=\min_{\beta \in 
\overline{B}}\left( \beta ^{\prime }\overline{X}^{\prime }A\overline{X}\beta
\right) /\left( \beta ^{\prime }\overline{X}^{\prime }M^{\left(
Z_{1},Q\right) }\overline{X}\beta \right) $

\noindent $=\left( y-X\widehat{\delta }_{L}\right) ^{\prime }A\left( y-X%
\widehat{\delta }_{L}\right) /\left[ \left( y-X\widehat{\delta }_{L}\right)
^{\prime }M^{\left( Z_{1},Q\right) }\left( y-X\widehat{\delta }_{L}\right) %
\right] $. By part (a)\ of Lemma S2-7, we then have $\widehat{\ell }%
_{L,n}=o_{p}\left( \left[ \mu _{n}^{\min }\right] ^{2}/n\right) $, so FELIM
also satisfies the needed condition. Finally, part (c) considers the FEFUL
estimator, which takes $\overline{\ell }_{n}=\widehat{\ell }_{F,n}$

\noindent $=\left[ \widehat{\ell }_{L,n}-\left( 1-\widehat{\ell }%
_{L,n}\right) \left( C/m_{n}\right) \right] /\left[ 1-\left( 1-\widehat{\ell 
}_{L,n}\right) \left( C/m_{n}\right) \right] $. By part (b) of Lemma S2-7,
we have that $\widehat{\ell }_{F,n}=o_{p}\left( \left[ \mu _{n}^{\min }%
\right] ^{2}/n\right) $, so the needed condition is satisfied again. The
consistency results given in parts (a)-(c) of this corollary then follow as
a consequence of Theorem 1. $\square $

\medskip

\noindent \textbf{Proof of Theorem 4:}

We shall prove this theorem for the FELIM case since the proof for FEFUL is
similar. To proceed, first define $S_{L,1}=X^{\prime }AD\left( J\left[ 
\widehat{\varepsilon }_{L}\circ \widehat{\varepsilon }_{L}\right] \right) AX$%
, $S_{L,2}=\left( \widehat{\varepsilon }_{L}\circ \widehat{\varepsilon }%
_{L}\right) ^{\prime }J\left( A\circ A\right) J\left( \widehat{\varepsilon }%
_{L}\iota _{d}^{\prime }\circ M^{\left( Z,Q\right) }X\right) $, $%
S_{L,3}=\left( \widehat{\varepsilon }_{L}\circ \widehat{\varepsilon }%
_{L}\right) ^{\prime }J\left( A\circ A\right) J\left( \widehat{\varepsilon }%
_{L}\circ \widehat{\varepsilon }_{L}\right) $, $\underline{S}_{L,4}=\left( 
\widehat{\varepsilon }_{L}\iota _{d}^{\prime }\circ \widehat{\underline{U}}%
_{L}\right) ^{\prime }J\left( A\circ A\right) J\left( \widehat{\varepsilon }%
_{L}\iota _{d}^{\prime }\circ \widehat{\underline{U}}_{L}\right) $,

\noindent $\widehat{H}_{L}=X^{\prime }\left[ A-\widehat{\ell }%
_{L,n}M^{\left( Z_{1},Q\right) }\right] X$, $\Sigma _{1,n}=\Upsilon ^{\prime
}Z_{2}^{\prime }M^{\left( Z_{1},Q\right) }D_{\sigma ^{2}}M^{\left(
Z_{1},Q\right) }Z_{2}\Upsilon /n$. In addition, also define $\sigma _{\left(
i,t\right) }^{2}=E\left[ \varepsilon _{\left( i,t\right) }^{2}|\mathcal{F}%
_{n}^{Z}\right] $, $\phi _{\left( i,t\right) }=E\left[ U_{\left( i,t\right)
}\varepsilon _{\left( i,t\right) }|\mathcal{F}_{n}^{Z}\right] $, $\Psi
_{\left( i,t\right) }=E\left[ U_{\left( i,t\right) }U_{\left( i,t\right)
}^{\prime }|\mathcal{F}_{n}^{Z}\right] $, $\underline{\phi }_{\left(
i,t\right) }=E\left[ \underline{U}_{\left( i,t\right) }\varepsilon _{\left(
i,t\right) }|\mathcal{F}_{n}^{Z}\right] $, and $\underline{\Psi }_{\left(
i,t\right) }=E\left[ \underline{U}_{\left( i,t\right) }\underline{U}_{\left(
i,t\right) }^{\prime }|\mathcal{F}_{n}^{Z}\right] $ where $\underline{U}%
_{\left( i,t\right) }=U_{\left( i,t\right) }-\rho \varepsilon _{\left(
i,t\right) }$ and where for notational convenience we suppress the
dependence of $\sigma _{\left( i,t\right) }^{2}$, $\phi _{\left( i,t\right)
} $, $\Psi _{\left( i,t\right) }$, $\underline{\phi }_{\left( i,t\right) }$,
and $\underline{\Psi }_{\left( i,t\right) }$ on $\mathcal{F}_{n}^{Z}=\sigma
\left( Z\right) $.

Using these notations, to show part (a), we first write $D_{\mu }\widehat{V}%
_{L}D_{\mu }=\widehat{V}_{L,1}+\widehat{V}_{L,2}+\widehat{V}_{L,3}+\widehat{V%
}_{L,4}$, where $\widehat{V}_{L,1}=\left( D_{\mu }^{-1}\widehat{H}_{L}D_{\mu
}^{-1}\right) ^{-1}D_{\mu }^{-1}S_{L,1}D_{\mu }^{-1}\left( D_{\mu }^{-1}%
\widehat{H}_{L}D_{\mu }^{-1}\right) ^{-1}$,

\noindent $\widehat{V}_{L,2}=-\left( D_{\mu }^{-1}\widehat{H}_{L}D_{\mu
}^{-1}\right) ^{-1}D_{\mu }^{-1}\left( \widehat{\rho }_{L}S_{L,2}+S_{L,2}^{%
\prime }\widehat{\rho }_{L}^{\prime }\right) D_{\mu }^{-1}\left( D_{\mu
}^{-1}\widehat{H}_{L}D_{\mu }^{-1}\right) ^{-1}$,

\noindent $\widehat{V}_{L,3}=\left( D_{\mu }^{-1}\widehat{H}_{L}D_{\mu
}^{-1}\right) ^{-1}D_{\mu }^{-1}\widehat{\rho }_{L}S_{L,3}\widehat{\rho }%
_{L}^{\prime }D_{\mu }^{-1}\left( D_{\mu }^{-1}\widehat{H}_{L}D_{\mu
}^{-1}\right) ^{-1}$, and $\widehat{V}_{L,4}=\left( D_{\mu }^{-1}\widehat{H}%
_{L}D_{\mu }^{-1}\right) ^{-1}$

\noindent $\times D_{\mu }^{-1}\underline{S}_{L,4}D_{\mu }^{-1}\left( D_{\mu
}^{-1}\widehat{H}_{L}D_{\mu }^{-1}\right) ^{-1}$. Now, consider $\widehat{V}%
_{L,1}$ first. Note that, by Lemma S2-17, 
\begin{equation*}
D_{\mu }^{-1}X^{\prime }AD\left( \varepsilon \circ \varepsilon \right)
AXD_{\mu }^{-1}=\Sigma _{1,n}+\dsum\limits_{\substack{ \left( i,t\right)
,\left( j,s\right) =1  \\ \left( i,t\right) \neq \left( j,s\right) }}%
^{m_{n}}A_{\left( i,t\right) ,\left( j,s\right) }^{2}\sigma _{\left(
i,t\right) }^{2}D_{\mu }^{-1}\Psi _{\left( j,s\right) }D_{\mu
}^{-1}+o_{p}\left( 1\right) \text{,}
\end{equation*}%
from which we deduce that $D_{\mu }^{-1}X^{\prime }AD\left( \varepsilon
\circ \varepsilon \right) AXD_{\mu }^{-1}=O_{p}\left( 1\right) $ using
Assumptions 2(i) and 3(iii), Lemma S2-1 part (a), and the assumption that $%
K_{2,n}/\left( \mu _{n}^{\min }\right) ^{2}=O\left( 1\right) $ under Case I.

Next, note that by Lemma S2-11,

\noindent $\widehat{\ell }_{L}=\left( y-X\widehat{\delta }_{L}\right)
^{\prime }A\left( y-X\widehat{\delta }_{L}\right) /\left( y-X\widehat{\delta 
}_{L}\right) ^{\prime }M^{\left( Z_{1},Q\right) }\left( y-X\widehat{\delta }%
_{L}\right) $. Moreover, by the result given in Lemma S2-10, we have that $%
D_{\mu }^{-1}\widehat{H}_{L}D_{\mu }^{-1}=H_{n}+o_{p}\left( 1\right) $,
where, by Assumption 3(iii), $H_{n}=\Upsilon ^{\prime }Z_{2}^{\prime
}M^{\left( Z_{1},Q\right) }Z_{2}\Upsilon /n$ is positive definite $a.s.n.$
In addition, we can apply part (a) of Lemma S2-18 and Slutsky's theorem to
deduce that%
\begin{eqnarray}
\widehat{V}_{L,1} &=&\left( D_{\mu }^{-1}\widehat{H}_{L}D_{\mu }^{-1}\right)
^{-1}D_{\mu }^{-1}S_{L,1}D_{\mu }^{-1}\left( D_{\mu }^{-1}\widehat{H}%
_{L}D_{\mu }^{-1}\right) ^{-1}  \notag \\
&=&H_{n}^{-1}\Sigma _{1,n}H_{n}^{-1}+H_{n}^{-1}\left( \dsum\limits 
_{\substack{ \left( i,t\right) ,\left( j,s\right) =1  \\ \left( i,t\right)
\neq \left( j,s\right) }}^{m_{n}}A_{\left( i,t\right) ,\left( j,s\right)
}^{2}\sigma _{\left( i,t\right) }^{2}D_{\mu }^{-1}\Psi _{\left( j,s\right)
}D_{\mu }^{-1}\right) H_{n}^{-1}+o_{p}\left( 1\right) \text{.}
\label{VhatL 1}
\end{eqnarray}%
Next, consider $\widehat{V}_{L,2}$. Here, note that we can further decompose 
$\widehat{V}_{L,2}$ as $\widehat{V}_{L,2}=\widehat{V}_{L,2,1}+\widehat{V}%
_{L,2,2}$, where $\widehat{V}_{L,2,1}=-\left( D_{\mu }^{-1}\widehat{H}%
_{L}D_{\mu }^{-1}\right) ^{-1}D_{\mu }^{-1}\widehat{\rho }_{L}S_{L,2}D_{\mu
}^{-1}\left( D_{\mu }^{-1}\widehat{H}_{L}D_{\mu }^{-1}\right) ^{-1}$ and

\noindent $\widehat{V}_{L,2,2}=-\left( D_{\mu }^{-1}\widehat{H}_{L}D_{\mu
}^{-1}\right) ^{-1}D_{\mu }^{-1}S_{L,2}^{\prime }\widehat{\rho }_{L}^{\prime
}D_{\mu }^{-1}\left( D_{\mu }^{-1}\widehat{H}_{L}D_{\mu }^{-1}\right) ^{-1}$%
. Noting that $K_{2,n}/\left( \mu _{n}^{\min }\right) ^{2}=O\left( 1\right) $
under Case I and applying the result of Lemma S2-10, as well as parts (d)
and (e) of Lemma S2-18 and Slutsky's theorem, we get%
\begin{eqnarray*}
\widehat{V}_{L,2,1} &=&-H_{n}^{-1}\frac{K_{2,n}}{\left( \mu _{n}^{\min
}\right) }\left\{ D_{\mu }^{-1}\rho +D_{\mu }^{-1}\left( \widehat{\rho }%
_{L}-\rho \right) \right\} \frac{\mu _{n}^{\min }}{K_{2,n}}S_{L,2}D_{\mu
}^{-1}H_{n}^{-1}\left( 1+o_{p}\left( 1\right) \right) \\
&=&-H_{n}^{-1}D_{\mu }^{-1}\rho \dsum\limits_{\substack{ \left( i,t\right)
,\left( j,s\right) =1  \\ \left( i,t\right) \neq \left( j,s\right) }}%
^{m_{n}}A_{\left( i,t\right) ,\left( j,s\right) }^{2}\sigma _{\left(
i,t\right) }^{2}\phi _{\left( j,s\right) }^{\prime }D_{\mu
}^{-1}H_{n}^{-1}+o_{p}\left( 1\right) .
\end{eqnarray*}%
Moreover, since $\widehat{V}_{L,2,2}=\widehat{V}_{L,2,1}^{\prime }$, we also
have

\noindent $\widehat{V}_{L,2,2}=-H_{n}^{-1}\dsum\nolimits_{\left( i,t\right)
,\left( j,s\right) =1:m_{n},\left( i,t\right) \neq \left( j,s\right)
}A_{\left( i,t\right) ,\left( j,s\right) }^{2}\sigma _{\left( i,t\right)
}^{2}D_{\mu }^{-1}\phi _{\left( j,s\right) }\rho ^{\prime }D_{\mu
}^{-1}H_{n}^{-1}+o_{p}\left( 1\right) $. Given that $\widehat{V}_{L,2}=%
\widehat{V}_{L,2,1}+\widehat{V}_{L,2,2}$, it follows from these calculations
that%
\begin{equation}
\widehat{V}_{L,2}=-H_{n}^{-1}\dsum\limits_{\substack{ \left( i,t\right)
,\left( j,s\right) =1  \\ \left( i,t\right) \neq \left( j,s\right) }}%
^{m_{n}}A_{\left( i,t\right) ,\left( j,s\right) }^{2}D_{\mu }^{-1}\left(
\rho \sigma _{\left( i,t\right) }^{2}\phi _{\left( j,s\right) }^{\prime
}+\sigma _{\left( i,t\right) }^{2}\phi _{\left( j,s\right) }\rho ^{\prime
}\right) D_{\mu }^{-1}H_{n}^{-1}+o_{p}\left( 1\right)  \label{VhatL 2}
\end{equation}%
Turning our attention to $\widehat{V}_{L,3}$, note that, in this case, we
can apply Lemma S2-10, parts (b) and (e) of Lemma S2-18, and Slutsky's
theorem to obtain%
\begin{eqnarray}
\widehat{V}_{L,3} &=&K_{2,n}H_{n}^{-1}\left[ D_{\mu }^{-1}\rho +D_{\mu
}^{-1}\left( \widehat{\rho }_{L}-\rho \right) \right] \frac{S_{L,3}}{K_{2,n}}%
\rho ^{\prime }D_{\mu }^{-1}H_{n}^{-1}\left( 1+o_{p}\left( 1\right) \right) 
\notag \\
&&+K_{2,n}H_{n}^{-1}\left[ D_{\mu }^{-1}\rho +D_{\mu }^{-1}\left( \widehat{%
\rho }_{L}-\rho \right) \right] \frac{S_{L,3}}{K_{2,n}}\left( \widehat{\rho }%
_{L}-\rho \right) ^{\prime }D_{\mu }^{-1}H_{n}^{-1}\left( 1+o_{p}\left(
1\right) \right)  \notag \\
&=&H_{n}^{-1}D_{\mu }^{-1}\rho \dsum\limits_{\substack{ \left( i,t\right)
,\left( j,s\right) =1  \\ \left( i,t\right) \neq \left( j,s\right) }}%
^{m_{n}}A_{\left( i,t\right) ,\left( j,s\right) }^{2}\sigma _{\left(
i,t\right) }^{2}\sigma _{\left( j,s\right) }^{2}\rho ^{\prime }D_{\mu
}^{-1}H_{n}^{-1}+o_{p}\left( 1\right) \text{.}  \label{Vhatl 3}
\end{eqnarray}%
Lastly, we consider $\widehat{V}_{L,4}$. Here, we can apply Lemma S2-10,
part (f) of Lemma S2-18, the fact that $K_{2,n}/\left( \mu _{n}^{\min
}\right) ^{2}=O\left( 1\right) $ under Case I, as well as Slutsky's theorem
to obtain 
\begin{eqnarray}
\widehat{V}_{L,4} &=&H_{n}^{-1}D_{\mu }^{-1}\underline{S}_{L,4}D_{\mu
}^{-1}H_{n}^{-1}\left( 1+o_{p}\left( 1\right) \right)  \notag \\
&=&H_{n}^{-1}\dsum\limits_{\substack{ \left( i,t\right) ,\left( j,s\right)
=1  \\ \left( i,t\right) \neq \left( j,s\right) }}^{m_{n}}A_{\left(
i,t\right) ,\left( j,s\right) }^{2}D_{\mu }^{-1}\underline{\phi }_{\left(
i,t\right) }\underline{\phi }_{\left( j,s\right) }^{\prime }D_{\mu
}^{-1}H_{n}^{-1}+o_{p}\left( 1\right) .  \label{Vhatl 4}
\end{eqnarray}%
It follows from equations (\ref{VhatL 1}), (\ref{VhatL 2}), (\ref{Vhatl 3}),
and (\ref{Vhatl 4}) that%
\begin{eqnarray*}
D_{\mu }\widehat{V}_{L}D_{\mu } &=&H_{n}^{-1}\Sigma
_{1,n}H_{n}^{-1}+H_{n}^{-1}\dsum\limits_{\substack{ \left( i,t\right)
,\left( j,s\right) =1  \\ \left( i,t\right) \neq \left( j,s\right) }}%
^{m_{n}}A_{\left( i,t\right) ,\left( j,s\right) }^{2}\sigma _{\left(
i,t\right) }^{2}D_{\mu }^{-1}\underline{\Psi }_{\left( j,s\right) }D_{\mu
}^{-1}H_{n}^{-1} \\
&&+H_{n}^{-1}\dsum\limits_{\substack{ \left( i,t\right) ,\left( j,s\right)
=1  \\ \left( i,t\right) \neq \left( j,s\right) }}^{m_{n}}A_{\left(
i,t\right) ,\left( j,s\right) }^{2}D_{\mu }^{-1}\underline{\phi }_{\left(
i,t\right) }\underline{\phi }_{\left( j,s\right) }^{\prime }D_{\mu
}^{-1}H_{n}^{-1}+o_{p}\left( 1\right) \\
&=&H_{n}^{-1}\left( \Sigma _{1,n}+\Sigma _{2,n}\right)
H_{n}^{-1}+o_{p}\left( 1\right) =\Lambda _{I,n}+o_{p}\left( 1\right) \text{.}
\end{eqnarray*}

To show the same result for FEFUL, note that $\widehat{\delta }_{F}$
satisfies the conditions of both Lemma S2-12 and Lemma S2-18. Hence, we can
make the same argument as given above for FELIM, except that we use the
result of Lemma S2-12 in lieu of Lemma S2-10 to obtain $D_{\mu }\widehat{V}%
_{F}D_{\mu }=H_{n}^{-1}\left( \Sigma _{1,n}+\Sigma _{2,n}\right)
H_{n}^{-1}+o_{p}\left( 1\right) =\Lambda _{I,n}+o_{p}\left( 1\right) $.

To show part (b), we again only provide an explicit argument for $\widehat{V}%
_{L}$ since the proof of $\widehat{V}_{F}$ follows in a similar way. To
proceed, write $\left[ \left( \mu _{n}^{\min }\right) ^{2}/K_{2,n}\right]
D_{\mu }\widehat{V}_{L}D_{\mu }=\left[ \left( \mu _{n}^{\min }\right)
^{2}/K_{2,n}\right] \dsum\nolimits_{\ell =1}^{4}\widehat{V}_{L,\ell }$,
where $\widehat{V}_{L,1}$, $\widehat{V}_{L,2}$, $\widehat{V}_{L,3}$, and $%
\widehat{V}_{L,4}$ are as defined in the proof of part (a).

Considering $\widehat{V}_{L,1}$ first, note that, since $K_{2,n}/\left( \mu
_{n}^{\min }\right) ^{2}\rightarrow \infty $ but $\sqrt{K_{2,n}}/\left( \mu
_{n}^{\min }\right) ^{2}\rightarrow 0$ under Case II, we have, upon applying
the result of Lemma S2-10, part (a) of Lemma S2-18, and Slutsky's theorem,%
\begin{eqnarray}
\frac{\left( \mu _{n}^{\min }\right) ^{2}}{K_{2,n}}\widehat{V}_{L,1}
&=&H_{n}^{-1}\frac{\left( \mu _{n}^{\min }\right) ^{2}}{K_{2,n}}\left[
\Sigma _{1,n}+\dsum\limits_{\substack{ \left( i,t\right) ,\left( j,s\right)
=1  \\ \left( i,t\right) \neq \left( j,s\right) }}^{m_{n}}A_{\left(
i,t\right) ,\left( j,s\right) }^{2}\sigma _{\left( i,t\right) }^{2}D_{\mu
}^{-1}\Psi _{\left( j,s\right) }D_{\mu }^{-1}\right] H_{n}^{-1}\left(
1+o_{p}\left( 1\right) \right)  \notag \\
&=&H_{n}^{-1}\frac{\left( \mu _{n}^{\min }\right) ^{2}}{K_{2,n}}\dsum\limits 
_{\substack{ \left( i,t\right) ,\left( j,s\right) =1  \\ \left( i,t\right)
\neq \left( j,s\right) }}^{m_{n}}A_{\left( i,t\right) ,\left( j,s\right)
}^{2}\sigma _{\left( i,t\right) }^{2}D_{\mu }^{-1}\Psi _{\left( j,s\right)
}D_{\mu }^{-1}H_{n}^{-1}+o_{p}\left( 1\right) .  \label{VLhat 1}
\end{eqnarray}%
Now, consider $\widehat{V}_{L,2}$. Here, we write $\left[ \left( \mu
_{n}^{\min }\right) ^{2}/K_{2,n}\right] \widehat{V}_{L,2}=\left[ \left( \mu
_{n}^{\min }\right) ^{2}/K_{2,n}\right] \widehat{V}_{L,2,1}$

\noindent $+\left[ \left( \mu _{n}^{\min }\right) ^{2}/K_{2,n}\right] 
\widehat{V}_{L,2,2}$, where $\widehat{V}_{L,2,1}$ and $\widehat{V}_{L,2,2}$
are again as defined in the proof of part (a). Making use of the results of
Lemma S2-10, parts (d) and (e) of Lemma S2-18, and Slutsky's theorem while
noting that $K_{2,n}/\left( \mu _{n}^{\min }\right) ^{2}\rightarrow \infty $
under Case II, we get%
\begin{eqnarray*}
\frac{\left( \mu _{n}^{\min }\right) ^{2}}{K_{2,n}}\widehat{V}_{L,2,1}
&=&-H_{n}^{-1}\left( \mu _{n}^{\min }\right) \left\{ D_{\mu }^{-1}\rho
+D_{\mu }^{-1}\left( \widehat{\rho }_{L}-\rho \right) \right\} \frac{\mu
_{n}^{\min }}{K_{2,n}}S_{L,2}D_{\mu }^{-1}H_{n}^{-1}\left( 1+o_{p}\left(
1\right) \right) \\
&=&-H_{n}^{-1}\frac{\left( \mu _{n}^{\min }\right) ^{2}}{K_{2,n}}%
\dsum\limits _{\substack{ \left( i,t\right) ,\left( j,s\right) =1  \\ \left(
i,t\right) \neq \left( j,s\right) }}^{m_{n}}A_{\left( i,t\right) ,\left(
j,s\right) }^{2}D_{\mu }^{-1}\rho \sigma _{\left( i,t\right) }^{2}\phi
_{\left( j,s\right) }^{\prime }D_{\mu }^{-1}H_{n}^{-1}+o_{p}\left( 1\right)
\end{eqnarray*}%
Moreover, since $\widehat{V}_{L,2,2}=\widehat{V}_{L,2,1}^{\prime }$, we also
have

\noindent $\left[ \left( \mu _{n}^{\min }\right) ^{2}K_{2,n}^{-1}\right] 
\widehat{V}_{L,2,2}=-H_{n}^{-1}\left[ \left( \mu _{n}^{\min }\right)
^{2}K_{2,n}^{-1}\right] \dsum\nolimits_{\substack{ \left( i,t\right) ,\left(
j,s\right) =1  \\ \left( i,t\right) \neq \left( j,s\right) }}%
^{m_{n}}A_{\left( i,t\right) ,\left( j,s\right) }^{2}D_{\mu }^{-1}\phi
_{\left( j,s\right) }\sigma _{\left( i,t\right) }^{2}\rho ^{\prime }D_{\mu
}^{-1}H_{n}^{-1}+o_{p}\left( 1\right) $. It follows from these calculations
that%
\begin{equation}
\frac{\left( \mu _{n}^{\min }\right) ^{2}\widehat{V}_{L,2}}{K_{2,n}}%
=-H_{n}^{-1}\dsum\limits_{\substack{ \left( i,t\right) ,\left( j,s\right) =1 
\\ \left( i,t\right) \neq \left( j,s\right) }}^{m_{n}}\frac{\left( \mu
_{n}^{\min }\right) ^{2}A_{\left( i,t\right) ,\left( j,s\right) }^{2}}{%
K_{2,n}}D_{\mu }^{-1}\left( \rho \sigma _{\left( i,t\right) }^{2}\phi
_{\left( j,s\right) }^{\prime }+\phi _{\left( j,s\right) }\sigma _{\left(
i,t\right) }^{2}\rho ^{\prime }\right) D_{\mu }^{-1}H_{n}^{-1}+o_{p}\left(
1\right) .  \label{VLhat 2}
\end{equation}%
Next, consider $\widehat{V}_{L,3}$. Given that $K_{2,n}/\left( \mu
_{n}^{\min }\right) ^{2}\rightarrow \infty $ under Case II, we get, upon
applying the result given in Lemma S2-10, as well as parts (b) and (e) of
Lemma S2-18 and Slutsky's theorem,%
\begin{eqnarray}
\frac{\left( \mu _{n}^{\min }\right) ^{2}}{K_{2,n}}\widehat{V}_{L,3}
&=&\left( \mu _{n}^{\min }\right) ^{2}H_{n}^{-1}\left[ D_{\mu }^{-1}\rho
+D_{\mu }^{-1}\left( \widehat{\rho }_{L}-\rho \right) \right] \frac{S_{L,3}}{%
K_{2,n}}\rho ^{\prime }D_{\mu }^{-1}H_{n}^{-1}\left( 1+o_{p}\left( 1\right)
\right)  \notag \\
&&+\left( \mu _{n}^{\min }\right) ^{2}H_{n}^{-1}\left[ D_{\mu }^{-1}\rho
+D_{\mu }^{-1}\left( \widehat{\rho }_{L}-\rho \right) \right] \frac{S_{L,3}}{%
K_{2,n}}\left( \widehat{\rho }_{L}-\rho \right) ^{\prime }D_{\mu
}^{-1}H_{n}^{-1}\left( 1+o_{p}\left( 1\right) \right)  \notag \\
&=&H_{n}^{-1}\dsum\limits_{\substack{ \left( i,t\right) ,\left( j,s\right)
=1  \\ \left( i,t\right) \neq \left( j,s\right) }}^{m_{n}}\frac{\left( \mu
_{n}^{\min }\right) ^{2}A_{\left( i,t\right) ,\left( j,s\right) }^{2}}{%
K_{2,n}}D_{\mu }^{-1}\rho \sigma _{\left( i,t\right) }^{2}\sigma _{\left(
j,s\right) }^{2}\rho ^{\prime }D_{\mu }^{-1}H_{n}^{-1}+o_{p}\left( 1\right)
\label{VLhat 3}
\end{eqnarray}%
Finally, we consider $\widehat{V}_{L,4}$. Again, noting that $K_{2,n}/\left(
\mu _{n}^{\min }\right) ^{2}\rightarrow \infty $ under Case II, we have,
upon applying the result given in Lemma S2-10, as well as part (f) of Lemma
S2-18 and Slutsky's theorem,%
\begin{eqnarray}
\frac{\left( \mu _{n}^{\min }\right) ^{2}}{K_{2,n}}\widehat{V}_{L,4}
&=&H_{n}^{-1}\frac{\left( \mu _{n}^{\min }\right) ^{2}}{K_{2,n}}D_{\mu }^{-1}%
\underline{S}_{L,4}D_{\mu }^{-1}H_{n}^{-1}\left( 1+o_{p}\left( 1\right)
\right)  \notag \\
&=&H_{n}^{-1}\frac{\left( \mu _{n}^{\min }\right) ^{2}}{K_{2,n}}\dsum\limits 
_{\substack{ \left( i,t\right) ,\left( j,s\right) =1  \\ \left( i,t\right)
\neq \left( j,s\right) }}^{m_{n}}A_{\left( i,t\right) ,\left( j,s\right)
}^{2}D_{\mu }^{-1}\underline{\phi }_{\left( i,t\right) }\underline{\phi }%
_{\left( j,s\right) }^{\prime }D_{\mu }^{-1}H_{n}^{-1}+o_{p}\left( 1\right) .
\label{VLhat4}
\end{eqnarray}%
It follows from equations (\ref{VLhat 1}), (\ref{VLhat 2}), (\ref{VLhat 3}),
and (\ref{VLhat4}) that%
\begin{eqnarray*}
\frac{\left( \mu _{n}^{\min }\right) ^{2}D_{\mu }\widehat{V}_{L}D_{\mu }}{%
K_{2,n}} &=&H_{n}^{-1}\frac{\left( \mu _{n}^{\min }\right) ^{2}}{K_{2,n}}%
\dsum\limits_{\substack{ \left( i,t\right) ,\left( j,s\right) =1  \\ \left(
i,t\right) \neq \left( j,s\right) }}^{m_{n}}A_{\left( i,t\right) ,\left(
j,s\right) }^{2}D_{\mu }^{-1}\left( \sigma _{\left( i,t\right) }^{2}%
\underline{\Psi }_{\left( j,s\right) }+\underline{\phi }_{\left( i,t\right) }%
\underline{\phi }_{\left( j,s\right) }^{\prime }\right) D_{\mu
}^{-1}H_{n}^{-1}+o_{p}\left( 1\right) \\
&=&\frac{\left( \mu _{n}^{\min }\right) ^{2}}{K_{2,n}}H_{n}^{-1}\Sigma
_{2,n}H_{n}^{-1}+o_{p}\left( 1\right) =\Lambda _{II,n}+o_{p}\left( 1\right) 
\text{. \ }
\end{eqnarray*}%
To show the same result for FEFUL, note again that $\widehat{\delta }_{F}$
satisfies the conditions of Lemmas S2-12 and S2-18. Hence, we can make the
same argument as given above for FELIM, except using Lemma S2-12 in lieu of
Lemma S2-10 to obtain $\left[ \left( \mu _{n}^{\min }\right) ^{2}/K_{2,n}%
\right] D_{\mu }\widehat{V}_{F}D_{\mu }=$

\noindent $\left[ \left( \mu _{n}^{\min }\right) ^{2}/K_{2,n}\right]
H_{n}^{-1}\Sigma _{2,n}H_{n}^{-1}+o_{p}\left( 1\right) =\Lambda
_{II,n}+o_{p}\left( 1\right) $. $\square $

\medskip

\noindent \textbf{Proof of Theorem 5:}

To show part (a), first note that, by part (d) of Lemma S2-3 and Assumption
3(iii), $\Lambda _{I,n}$ is positive definite $a.s.n$. In addition, making
use of part (a) of Theorem 4, we have $D_{\mu }\widehat{V}_{L}D_{\mu
}=\Lambda _{I,n}+o_{p}\left( 1\right) $, so that $D_{\mu }\widehat{V}%
_{L}D_{\mu }$ is positive definite w.p.a.1. Hence, under $H_{0}:c^{\prime
}\delta _{0}=r$, we can write%
\begin{equation*}
\mathbb{T}_{L}=\frac{c^{\prime }\widehat{\delta }_{L,n}-r}{\sqrt{c^{\prime }%
\widehat{V}_{L}c}}=\frac{c^{\prime }\left( \widehat{\delta }_{L,n}-\delta
_{0}\right) }{\sqrt{c^{\prime }\widehat{V}_{L}c}}=\frac{\left( c^{\prime
}D_{\mu }^{-1}\mu _{n}^{\ast }\left( c\right) \right) \Lambda _{I,n}^{1/2}%
\left[ \Lambda _{I,n}^{-1/2}D_{\mu }\left( \widehat{\delta }_{L,n}-\delta
_{0}\right) \right] }{\sqrt{\left( c^{\prime }D_{\mu }^{-1}\mu _{n}^{\ast
}\left( c\right) \right) D_{\mu }\widehat{V}_{L}D_{\mu }\left( \mu
_{n}^{\ast }\left( c\right) D_{\mu }^{-1}c\right) }}
\end{equation*}%
Applying Theorem 2, we have $\Lambda _{I,n}^{-1/2}D_{\mu }\left( \widehat{%
\delta }_{L,n}-\delta _{0}\right) \overset{d}{\rightarrow }N\left(
0,I_{d}\right) $. It follows by the definition of $c_{\ast }$ given in
Assumption 10, as well as by applying part (a) of Theorem 4 and the
continuous mapping theorem that%
\begin{equation}
\mathbb{T}_{L}=\frac{c_{\ast }^{\prime }\Lambda _{I,n}^{1/2}\left[ \Lambda
_{I,n}^{-1/2}D_{\mu }\left( \widehat{\delta }_{L,n}-\delta _{0}\right) %
\right] }{\sqrt{c_{\ast }^{\prime }\Lambda _{I,n}c_{\ast }}}\left[
1+o_{p}\left( 1\right) \right] \overset{d}{\rightarrow }N\left( 0,1\right) 
\text{.}  \label{asy normality I}
\end{equation}%
On the other hand, under $H_{1}$, we have $c^{\prime }\delta _{0}=r+h$ for
some $h\in \mathbb{R}\backslash \left\{ 0\right\} $, and we can write $%
\mathbb{T}_{L}=\left( c^{\prime }\widehat{\delta }_{L,n}-r\right) /\sqrt{%
c^{\prime }\widehat{V}_{L}c}=c^{\prime }\left( \widehat{\delta }%
_{L,n}-\delta _{0}\right) /\sqrt{c^{\prime }\widehat{V}_{L}c}+h/\sqrt{%
c^{\prime }\widehat{V}_{L}c}$. The first term above is $O_{p}\left( 1\right)
,$ as shown in (\ref{asy normality I}) above, whereas application of part
(a) of Theorem 4, Assumption 10, and the Slutsky's theorem shows that $%
\left( \mu _{n}^{\ast }\left( c\right) \right) ^{2}c^{\prime }\widehat{V}%
_{L}c=\left( c^{\prime }D_{\mu }^{-1}\mu _{n}^{\ast }\left( c\right) \right)
D_{\mu }\widehat{V}_{L}D_{\mu }\left( \mu _{n}^{\ast }\left( c\right) D_{\mu
}^{-1}c\right) =c_{\ast }^{\prime }\Lambda _{I,n}c_{\ast }+o_{p}\left(
1\right) $, where $c_{\ast }^{\prime }\Lambda _{I,n}c_{\ast }>0$ since $%
\Lambda _{I,n}$ is positive definite in light of part (d) of Lemma S2-3 and
Assumption 3(iii) and since $c_{\ast }\neq 0$ by construction. In addition,
by parts (a) and (c) of Lemma S2-3; Assumption 3(iii); and the fact that,
under Case I, $K_{2,n}/\left( \mu _{n}^{\min }\right) ^{2}=O\left( 1\right) $%
; there exists a positive constant $C<\infty $ such that, almost surely for
all $n$ sufficiently large,%
\begin{equation}
\lambda _{\max }\left( \Lambda _{I,n}\right) \leq \frac{\lambda _{\max }%
\left[ VC\left( \Upsilon ^{\prime }Z_{2}^{\prime }M^{\left( Z_{1},Q\right)
}\varepsilon /\sqrt{n}\right) |\mathcal{F}_{n}^{Z}\right] +\frac{K_{2,n}}{%
\left( \mu _{n}^{\min }\right) ^{2}}\lambda _{\max }\left[ VC\left( 
\underline{U}^{\prime }A\varepsilon /\sqrt{K_{2,n}}\right) |\mathcal{F}%
_{n}^{Z}\right] }{\left[ \lambda _{\min }\left( H_{n}\right) \right] ^{2}}%
\leq C\text{.}  \label{maxlam bd case I}
\end{equation}%
It follows that, in this case, $h/\sqrt{c^{\prime }\widehat{V}_{L}c}=\mu
_{n}^{\ast }\left( c\right) h/\sqrt{\left( \mu _{n}^{\ast }\left( c\right)
\right) ^{2}c^{\prime }\widehat{V}_{L}c}=\left( \mu _{n}^{\ast }\left(
c\right) h/\sqrt{c_{\ast }^{\prime }\Lambda _{I,n}c_{\ast }}\right) \left[
1+o_{p}\left( 1\right) \right] $. So, w.p.a.1, $h/\sqrt{c^{\prime }\widehat{V%
}_{L}c}\rightarrow +\infty $ \ if $h>0,$ whereas $h/\sqrt{c^{\prime }%
\widehat{V}_{L}c}\rightarrow -\infty $ \ if $h<0$, from which the stated
result follows. Finally, note that the results for $\mathbb{T}_{F}$ can be
shown in the same way, so to avoid redundancy, we omit the proof.

To show part (b), we first let $\widetilde{L}_{n}=\mu _{n}^{\ast }\left(
c\right) c^{\prime }D_{\mu }^{-1}$; and note that, by Assumption 10, there
exist a constant vector $c_{\ast }\neq 0$ and a positive constant $%
\underline{C}$ such $\widetilde{L}_{n}=\mu _{n}^{\ast }\left( c\right)
c^{\prime }D_{\mu }^{-1}\rightarrow c_{\ast }^{\prime }$ and $c_{\ast
}^{\prime }\Lambda _{II,n}c_{\ast }\geq \underline{C}>0$ $a.s.n.$ It follows
that, in this case, the conditions for $\widetilde{L}_{n}$ given in Theorem
3 are trivially satisfied. Applying Theorem 3, we then obtain

\noindent $\left( \mu _{n}^{\min }/\sqrt{K_{2,n}}\right) \left[ \mu
_{n}^{\ast }\left( c\right) c^{\prime }D_{\mu }^{-1}\Lambda _{II,n}D_{\mu
}^{-1}c\mu _{n}^{\ast }\left( c\right) \right] ^{-1/2}\mu _{n}^{\ast }\left(
c\right) c^{\prime }D_{\mu }^{-1}\left[ D_{\mu }\left( \widehat{\delta }%
_{L,n}-\delta _{0}\right) \right] $

\noindent $=\left( \mu _{n}^{\min }/\sqrt{K_{2,n}}\right) \left[ c_{\ast
}^{\prime }\Lambda _{II,n}c_{\ast }\right] ^{-1/2}c_{\ast }^{\prime }\left[
D_{\mu }\left( \widehat{\delta }_{L,n}-\delta _{0}\right) \right] \left[
1+o_{p}\left( 1\right) \right] $ $\overset{d}{\rightarrow }N\left(
0,1\right) $. Moreover,

\noindent $\left[ \left( \mu _{n}^{\min }\right) ^{2}/K_{2,n}\right] D_{\mu }%
\widehat{V}_{L}D_{\mu }=\Lambda _{II,n}+o_{p}\left( 1\right) $ by part (b)
of Theorem 4. Now, under $H_{0}:c^{\prime }\delta _{0}=r$, we can write%
\begin{equation*}
\mathbb{T}_{L}=\frac{c^{\prime }\widehat{\delta }_{L,n}-r}{\sqrt{c^{\prime }%
\widehat{V}_{L}c}}=\frac{\left( \mu _{n}^{\min }/\sqrt{K_{2,n}}\right) \mu
_{n}^{\ast }\left( c\right) c^{\prime }D_{\mu }^{-1}\left[ D_{\mu }\left( 
\widehat{\delta }_{L,n}-\delta _{0}\right) \right] }{\sqrt{\left( \mu
_{n}^{\ast }\left( c\right) c^{\prime }D_{\mu }^{-1}\right) \left[ \left\{
\left( \mu _{n}^{\min }\right) ^{2}/K_{2,n}\right\} D_{\mu }\widehat{V}%
_{L}D_{\mu }\right] \left( D_{\mu }^{-1}c\mu _{n}^{\ast }\left( c\right)
\right) }}
\end{equation*}%
from which it follows that 
\begin{equation}
\mathbb{T}_{L}=\frac{\left( \mu _{n}^{\min }/\sqrt{K_{2,n}}\right) c_{\ast
}^{\prime }\left[ D_{\mu }\left( \widehat{\delta }_{L,n}-\delta _{0}\right) %
\right] }{\sqrt{c_{\ast }^{\prime }\Lambda _{II,n}c_{\ast }}}\left[
1+o_{p}\left( 1\right) \right] \overset{d}{\rightarrow }N\left( 0,1\right) 
\text{.}  \label{asy normality II}
\end{equation}%
Under $H_{1}$, we again write $c^{\prime }\delta _{0}=r+h$ for some $h\in 
\mathbb{R}\backslash \left\{ 0\right\} $, and note that, in this case, by
applying Assumption 10, part (b) of Theorem 4, and Slutsky's theorem; we have

\noindent $\left( \mu _{n}^{\ast }\left( c\right) \right) ^{2}\left\{ \left(
\mu _{n}^{\min }\right) ^{2}/K_{2,n}\right\} c^{\prime }\widehat{V}%
_{L}c=\left( \mu _{n}^{\ast }\left( c\right) c^{\prime }D_{\mu }^{-1}\right) %
\left[ \left\{ \left( \mu _{n}^{\min }\right) ^{2}/K_{2,n}\right\} D_{\mu }%
\widehat{V}_{L}D_{\mu }\right] \left( D_{\mu }^{-1}c\mu _{n}^{\ast }\left(
c\right) \right) $

\noindent $=c_{\ast }^{\prime }\Lambda _{II,n}c_{\ast }+o_{p}\left( 1\right) 
$. Moreover, there exists a positive constant $\underline{C}$ such that $%
c_{\ast }^{\prime }\Lambda _{II,n}c_{\ast }\geq \underline{C}>0$ $a.s.n.$ by
Assumption 10. In addition, by part (c) of Lemma S2-3 and Assumption 3(iii),
there exists a positive constant $C$ such that, almost surely for all $n$
sufficiently large%
\begin{equation}
\lambda _{\max }\left( \Lambda _{II,n}\right) \leq \frac{\left( \mu
_{n}^{\min }\right) ^{2}}{K_{2,n}}\frac{1}{\left[ \lambda _{\min }\left(
H_{n}\right) \right] ^{2}}\frac{K_{2,n}}{\left( \mu _{n}^{\min }\right) ^{2}}%
\lambda _{\max }\left[ VC\left( \frac{\underline{U}^{\prime }A\varepsilon }{%
\sqrt{K_{2,n}}}\right) |\mathcal{F}_{n}^{Z}\right] \leq C<\infty \text{.}
\label{maxlam bd case II}
\end{equation}%
It follows that, for this case,%
\begin{equation*}
\frac{h}{\sqrt{c^{\prime }\widehat{V}_{L}c}}=\frac{\mu _{n}^{\ast }\left(
c\right) \left( \mu _{n}^{\min }/\sqrt{K_{2,n}}\right) h}{\sqrt{\left( \mu
_{n}^{\ast }\left( c\right) \right) ^{2}\left\{ \left( \mu _{n}^{\min
}\right) ^{2}/K_{2,n}\right\} c^{\prime }\widehat{V}_{L}c}}=\frac{\mu
_{n}^{\ast }\left( c\right) \left( \mu _{n}^{\min }/\sqrt{K_{2,n}}\right) h}{%
\sqrt{c_{\ast }^{\prime }\Lambda _{II,n}c_{\ast }}}\left[ 1+o_{p}\left(
1\right) \right] \text{.}
\end{equation*}%
Hence, w.p.a.1, $h/\sqrt{c^{\prime }\widehat{V}_{L}c}\rightarrow +\infty $ \
if $h>0$ whereas $h/\sqrt{c^{\prime }\widehat{V}_{L}c}\rightarrow -\infty $
\ if $h<0$, given the condition that $\left( \mu _{n}^{\min }\right) ^{2}/%
\sqrt{K_{2,n}}\rightarrow \infty $ and given that, by construction, $\mu
_{n}^{\min }/\mu _{n}^{\ast }\left( c\right) =O\left( 1\right) $. Finally,
write 
\begin{equation*}
\mathbb{T}_{L}=\frac{c^{\prime }\widehat{\delta }_{L,n}-r}{\sqrt{c^{\prime }%
\widehat{V}_{L}c}}=\frac{c^{\prime }\left( \widehat{\delta }_{L,n}-\delta
_{0}\right) }{\sqrt{c^{\prime }\widehat{V}_{L}c}}+\frac{h}{\sqrt{c^{\prime }%
\widehat{V}_{L}c}}\text{.}
\end{equation*}%
Since the first term on the right-hand side above is $O_{p}\left( 1\right) $
as shown in (\ref{asy normality II}), we deduce that w.p.a.1, $\mathbb{T}%
_{L}\rightarrow +\infty $ \ if $h>0$ and $\mathbb{T}_{L}\rightarrow -\infty $
\ if $h<0$. The results for $\mathbb{T}_{F}$ can be shown in the same way,
so to avoid redundancy, we omit the proof. $\square $

\medskip

\noindent \textbf{Proof of Corollary 2: }Note that the assumptions and setup
of Corollary 2 is essentially the same as that of Theorem 5, except that we
do not assume the more general conditions given in Assumption 10 but rather
we assume the specialized structure where $D_{\mu }=\mu _{n}^{\min }\cdot
I_{d}$. Hence, to prove this corollary, we need to show that $D_{\mu }=\mu
_{n}^{\min }\cdot I_{d}$ implies that Assumption 10 is satisfied. To
proceed, note that, trivially in this case, $\mu _{n}^{\ast }\left( c\right)
=\mu _{n}^{\min }$ so that $\mu _{n}^{\ast }\left( c\right) D_{\mu
}^{-1}c=\mu _{n}^{\min }\left[ \left( \mu _{n}^{\min }\right) ^{-1}\cdot
I_{d}\right] c=c$ for all $n$. Thus, $c_{\ast }=c\neq 0$ in this case.
Moreover, there exists a positive constant $\underline{C}$ such that $%
c_{\ast }^{\prime }\Lambda _{II,n}c_{\ast }=c^{\prime }\Lambda
_{II,n}c=\left( \mu _{n}^{\min }\right) ^{2}c^{\prime }H_{n}^{-1}\Sigma
_{2,n}H_{n}^{-1}c/K_{2,n}=\left( \mu _{n}^{\min }\right) ^{2}c^{\prime
}H_{n}^{-1}D_{\mu }^{-1}VC\left( \underline{U}^{\prime }A\varepsilon /\sqrt{%
K_{2,n}}|\mathcal{F}_{n}^{Z}\right) D_{\mu }^{-1}H_{n}^{-1}c=c^{\prime
}H_{n}^{-1}VC\left( \underline{U}^{\prime }A\varepsilon /\sqrt{K_{2,n}}|%
\mathcal{F}_{n}^{Z}\right) H_{n}^{-1}c\geq \underline{C}>0$ $a.s.n.$ for all 
$c\neq 0$, by the almost sure positive definiteness of $VC\left( \underline{U%
}^{\prime }A\varepsilon /\sqrt{K_{2,n}}|\mathcal{F}_{n}^{Z}\right) $ as
shown in part (b) of Lemma S2-3, which completes the proof. $\square $

\medskip

\noindent \textbf{Proof of Corollary 3: }Note that the assumptions and setup
of Corollary 3 is essentially the same as that of Theorem 5, except that we
do not assume the more general conditions given in Assumption 10. Instead,
we consider the special case where $c=e_{\ell }$ for $\ell \in \left\{
1,...,d\right\} $; and, in lieu of Assumption 10, we assume the condition
that there exists a positive constant $C_{\ast }$ such that $e_{\ell
}^{\prime }\overline{H}_{2\cdot }^{\prime }\overline{H}_{2\cdot }e_{\ell
}\geq C_{\ast }>0$ $a.s.n.$ Hence, to prove this corollary, we need to show
that, in the case where the problem of interest is testing the null
hypothesis $H_{0}:c^{\prime }\delta _{0}=e_{\ell }^{\prime }\delta _{0}=r$,
the condition that $e_{\ell }^{\prime }\overline{H}_{2\cdot }^{\prime }%
\overline{H}_{2\cdot }e_{\ell }\geq C_{\ast }>0$ $a.s.n.$ implies the
conditions given in Assumption 10. To proceed, note first that since $%
c=e_{\ell }$ here, we have $\mu _{n}^{\ast }\left( c\right) =\min \left\{
\mu _{g,n}|g\in \left\{ 1,...,d\right\} \text{ and }c_{g}\neq 0\right\} =\mu
_{\ell ,n}$, so that $\mu _{n}^{\ast }\left( c\right) D_{\mu }^{-1}c=\mu
_{\ell ,n}D_{\mu }^{-1}e_{\ell }=\mu _{\ell ,n}\left( \mu _{\ell ,n}\right)
^{-1}e_{\ell }=e_{\ell }$. Thus, $c_{\ast }=e_{\ell }\neq 0$ in this case.
Moreover, note that 
\begin{equation*}
\left( \mu _{n}^{\min }\right) D_{\mu }^{-1}=\left( \mu _{n}^{\min }\right)
\left( 
\begin{array}{cc}
D_{1}^{-1} & 0 \\ 
0 & \left( \mu _{n}^{\min }\right) ^{-1}\cdot I_{d_{2}}%
\end{array}%
\right) \rightarrow \left( 
\begin{array}{cc}
0 & 0 \\ 
0 & I_{d_{2}}%
\end{array}%
\right) =D_{0}\text{, \ }\left( say\right) \text{.}
\end{equation*}%
It follows that, in this case, $c_{\ast }^{\prime }\Lambda _{II,n}c_{\ast
}=e_{\ell }^{\prime }\Lambda _{II,n}e_{\ell }=\left( \mu _{n}^{\min }\right)
^{2}e_{\ell }^{\prime }H_{n}^{-1}D_{\mu }^{-1}\Sigma _{2,n}^{\ast }D_{\mu
}^{-1}H_{n}^{-1}e_{\ell }/K_{2,n}$

\noindent $=e_{\ell }^{\prime }H_{n}^{-1}D_{0}VC\left( \underline{U}^{\prime
}A\varepsilon /\sqrt{K_{2,n}}|\mathcal{F}_{n}^{Z}\right)
D_{0}H_{n}^{-1}e_{\ell }\left[ 1+o_{a.s.}\left( 1\right) \right] \geq
Ce_{\ell }^{\prime }\overline{H}_{2\cdot }^{\prime }\overline{H}_{2\cdot
}e_{\ell }\geq CC_{\ast }=\underline{C}>0$ $a.s.n.$, by the fact that $%
VC\left( \underline{U}^{\prime }A\varepsilon /\sqrt{K_{2,n}}|\mathcal{F}%
_{n}^{Z}\right) \geq CI_{d}$ \ $a.s.n$. for some positive constant $C$, as
shown in part (b) of Lemma S2-3, and by the assumption that $e_{\ell
}^{\prime }\overline{H}_{2\cdot }^{\prime }\overline{H}_{2\cdot }e_{\ell
}\geq C_{\ast }>0$ $a.s.n.$. This completes the proof. $\square $

\medskip

\noindent

\medskip

\noindent

\end{document}
