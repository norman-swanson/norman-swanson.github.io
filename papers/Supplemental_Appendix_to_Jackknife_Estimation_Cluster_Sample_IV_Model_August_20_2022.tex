%2multibyte Version: 5.50.0.2960 CodePage: 936


\documentclass[11pt]{article}
%%%%%%%%%%%%%%%%%%%%%%%%%%%%%%%%%%%%%%%%%%%%%%%%%%%%%%%%%%%%%%%%%%%%%%%%%%%%%%%%%%%%%%%%%%%%%%%%%%%%%%%%%%%%%%%%%%%%%%%%%%%%%%%%%%%%%%%%%%%%%%%%%%%%%%%%%%%%%%%%%%%%%%%%%%%%%%%%%%%%%%%%%%%%%%%%%%%%%%%%%%%%%%%%%%%%%%%%%%%%%%%%%%%%%%%%%%%%%%%%%%%%%%%%%%%%
\usepackage{amssymb}
\usepackage{amsfonts}
\usepackage{amsmath}
\usepackage{geometry}
\usepackage[onehalfspacing]{setspace}

\setcounter{MaxMatrixCols}{10}
%TCIDATA{OutputFilter=LATEX.DLL}
%TCIDATA{Version=5.00.0.2606}
%TCIDATA{Codepage=936}
%TCIDATA{<META NAME="SaveForMode" CONTENT="1">}
%TCIDATA{BibliographyScheme=Manual}
%TCIDATA{Created=Sunday, July 18, 2004 16:10:34}
%TCIDATA{LastRevised=Sunday, August 21, 2022 13:11:31}
%TCIDATA{<META NAME="GraphicsSave" CONTENT="32">}
%TCIDATA{<META NAME="DocumentShell" CONTENT="General\Blank Document">}
%TCIDATA{CSTFile=article.cst}
%TCIDATA{PageSetup=72,72,72,72,0}
%TCIDATA{AllPages=
%F=36,\PARA{038<p type="texpara" tag="Body Text" >\hfill \thepage}
%}


\newtheorem{theorem}{Theorem}
\newtheorem{acknowledgement}[theorem]{Acknowledgement}
\newtheorem{algorithm}[theorem]{Algorithm}
\newtheorem{axiom}[theorem]{Axiom}
\newtheorem{case}[theorem]{Case}
\newtheorem{claim}[theorem]{Claim}
\newtheorem{conclusion}[theorem]{Conclusion}
\newtheorem{condition}[theorem]{Condition}
\newtheorem{conjecture}[theorem]{Conjecture}
\newtheorem{corollary}[theorem]{Corollary}
\newtheorem{criterion}[theorem]{Criterion}
\newtheorem{definition}[theorem]{Definition}
\newtheorem{example}[theorem]{Example}
\newtheorem{exercise}[theorem]{Exercise}
\newtheorem{lemma}[theorem]{Lemma}
\newtheorem{notation}[theorem]{Notation}
\newtheorem{problem}[theorem]{Problem}
\newtheorem{proposition}[theorem]{Proposition}
\newtheorem{remark}[theorem]{Remark}
\newtheorem{solution}[theorem]{Solution}
\newtheorem{summary}[theorem]{Summary}
\newenvironment{proof}[1][Proof]{\noindent\textbf{#1.} }{\ \rule{0.5em}{0.5em}}
\renewcommand{\baselinestretch}{1.3} 
\textwidth=6.8in
\textheight=8.7in
\oddsidemargin=0in
\evensidemargin=0in
\topmargin=0in
\baselineskip=10pt
\linespread{1.0}
\input{tcilatex}
\geometry{left=1in,right=1in,top=1.25in,bottom=1.25in}

\begin{document}


\noindent \noindent \setcounter{page}{1}

\begin{center}
\textbf{Supplemental Appendix for \textquotedblleft Jackknife Estimation of
a Cluster-Sample IV Regression Model with Many Weak Instruments" \footnote{%
Corresponding author: John C. Chao, Department of Economics, 7343 Preinkert
Drive, University of Maryland, chao@econ.umd.edu. Norman R. Swanson,
Department of Economics, 9500 Hamilton Street, Rutgers University,
nswanson@econ.rutgers.edu. Tiemen Woutersen, Department of Economics, 1130 E
Helen Street, University of Arizona, woutersen@arizona.edu. The authors owe
special thanks to Jerry Hausman and Whitney Newey for many discussions on
the topic of this paper over a number of years. In addition, thanks are owed
to all of the particpants of the 2019 MIT Conference Honoring Whitney Newey
for comments on and advice given on an earlier verison of this work.
Finally, the authors wish to thank Miriam Arden for excellent research
assistance. Chao thanks the University of Maryland for research support, and
Woutersen's work was supported by an Eller College of Management Research
Grant.}}

John C. Chao, Norman R. Swanson, and Tiemen Woutersen

\textit{First Draft}: September 5, 2020 \ \ \textit{This Version}: August,
20 2022

\textbf{Abstract}
\end{center}

This Supplemental Appendix is comprised of two sub-appendices. Appendix S1
provides proofs for Theorems 2 and 3 of the main paper. Appendix S2 states
additional supporting lemmas used to prove the main theorems of the paper.
Proofs for these additional lemmas are reported in a separate Online
Appendix which can be viewed at the URL:

\noindent

\noindent http://econweb.umd.edu/\symbol{126}chao/Research/research%
\_files/Additional\_

\noindent
Online\_Appendix\_Jackknife\_Estimation\_Cluster\_Sample\_IV\_Model\_August%
\_19\_2022.pdf

\section*{Appendix S1: Proof of Theorems 2 and 3}

\noindent \textbf{Proof of Theorem 2: }Define $\mathcal{Y}_{n}=\Upsilon
^{\prime }Z_{2}^{\prime }M^{\left( Z_{1},Q\right) }\varepsilon /\sqrt{n}%
+D_{\mu }^{-1}\underline{U}^{\prime }A\varepsilon $.{\small \ }Note that, by
the result of Lemma S2-9 given in Appendix S2 below, we have that $D_{\mu
}^{-1}\widehat{\Delta }\left( \delta _{0}\right) =\Upsilon ^{\prime
}Z_{2}^{\prime }M^{\left( Z_{1},Q\right) }\varepsilon /\sqrt{n}+D_{\mu }^{-1}%
\underline{U}^{\prime }A\varepsilon +o_{p}\left( 1\right) =\mathcal{Y}%
_{n}+o_{p}\left( 1\right) $.

\noindent We now establish the asymptotic normality of $\mathcal{Y}_{n}$,
upon appropriate standardization, in the case where $K_{2,n}/\left( \mu
_{n}^{\min }\right) ^{2}=O\left( 1\right) $. To proceed, let $a\in \mathbb{R}%
^{d}$ such that $\left\Vert a\right\Vert _{2}=1$ and define $b_{1n}=\Sigma
_{n}^{-1/2}a$. and $b_{2n}=\sqrt{K_{2,n}}D_{\mu }^{-1}\Sigma _{n}^{-1/2}a$,
where $\Sigma _{n}=VC\left( \mathcal{Y}_{n}|\mathcal{F}_{n}^{Z}\right)
=\Sigma _{1,n}+\Sigma _{2,n}$, with $\Sigma _{1,n}=VC\left( \Upsilon
^{\prime }Z_{2}^{\prime }M^{\left( Z_{1},Q\right) }\varepsilon /\sqrt{n}|%
\mathcal{F}_{n}^{Z}\right) $ and $\Sigma _{2,n}=VC\left( D_{\mu }^{-1}%
\underline{U}^{\prime }A\varepsilon |\mathcal{F}_{n}^{Z}\right) $. Now, let $%
\mathcal{L}_{\left( i,t\right) ,n}$

\noindent $=b_{1n}^{\prime }\Upsilon ^{\prime }Z_{2}^{\prime }M^{\left(
Z_{1},Q\right) }e_{\left( i,t\right) }\varepsilon _{\left( i,t\right) }/%
\sqrt{n}$ and $\mathcal{N}_{\left( i,t\right)
,n}=K_{2,n}^{-1/2}\dsum\nolimits_{\left( j,s\right) =1}^{\left( i,t\right)
-1}A_{\left( i,t\right) ,\left( j,s\right) }\left[ \underline{u}_{2,\left(
i,t\right) ,n}\varepsilon _{\left( j,s\right) }+\underline{u}_{2,\left(
j,s\right) ,n}\varepsilon _{\left( i,t\right) }\right] $, where $\underline{u%
}_{2,\left( i,t\right) ,n}=$ $b_{2n}^{\prime }\underline{U}_{\left(
i,t\right) }$, with $\underline{u}_{2,\left( j,s\right) ,n}$ similarly
defined, and where $e_{\left( i,t\right) }$ denotes an $m_{n}\times 1$
elementary vector whose $\left( i,t\right) ^{th}$ component is $1$ and all
other components are $0$. Using these notations, note that we can write $%
a^{\prime }\Sigma _{n}^{-1/2}\mathcal{Y}_{n}=\mathcal{L}_{\left( 1,1\right)
,n}+\dsum\nolimits_{\left( i,t\right) =2}^{m_{n}}\left\{ \mathcal{L}_{\left(
i,t\right) ,n}+\mathcal{N}_{\left( i,t\right) ,n}\right\} $. Next, observe
that, 
\begin{eqnarray*}
E\left[ \mathcal{L}_{\left( 1,1\right) ,n}^{2}|\mathcal{F}_{n}^{Z}\right]
&=&E\left[ \varepsilon _{\left( 1,1\right) }^{2}|\mathcal{F}_{n}^{Z}\right] 
\frac{\left[ a^{\prime }\Sigma _{n}^{-1/2}\Upsilon ^{\prime }Z_{2}^{\prime
}M^{\left( Z_{1},Q\right) }e_{\left( 1,1\right) }\right] ^{2}}{n} \\
&\leq &E\left[ \varepsilon _{\left( 1,1\right) }^{2}|\mathcal{F}_{n}^{Z}%
\right] a^{\prime }\Sigma _{n}^{-1}a\left( \frac{\left\Vert \Upsilon
^{\prime }Z_{2}^{\prime }M^{\left( Z_{1},Q\right) }e_{\left( 1,1\right)
}\right\Vert _{2}}{\sqrt{n}}\right) ^{2}\text{ (by CS inequality)}
\end{eqnarray*}%
\begin{eqnarray*}
&\leq &\left( \max_{1\leq \left( i,t\right) \leq m_{n}}E\left[ \varepsilon
_{\left( i,t\right) }^{2}|\mathcal{F}_{n}^{Z}\right] \right) a^{\prime
}\Sigma _{n}^{-1}a\left( \frac{\max_{1\leq \left( i,t\right) \leq
m_{n}}\left\Vert \Upsilon ^{\prime }Z_{2}^{\prime }M^{\left( Z_{1},Q\right)
}e_{\left( i,t\right) }\right\Vert _{2}}{\sqrt{n}}\right) ^{2} \\
&=&o_{p}\left( 1\right) \text{ }
\end{eqnarray*}%
in light of Assumptions 2(i) and 7 and part (d) of Lemma S2-3\footnote{%
Lemma S2-3 is stated in Appendix S2 below. A proof of this lemma is provided
in section 1 of the Additional Online Appendix which can be viewed at the
URL:
\par
\noindent http://econweb.umd.edu/\symbol{126}chao/Research/research%
\_files/Additional\_Online\_Appendix\_Jackknife\_Estimation\_
\par
\noindent Cluster\_Sample\_IV\_Model\_August\_19\_2022.pdf}. Moreover, under
Assumptions 2 and 3(iii), there exists a positive constant $C^{\ast }$ such
that 
\begin{eqnarray*}
E_{Z}\left\{ \left( E\left[ \mathcal{L}_{\left( 1,1\right) ,n}^{2}|\mathcal{F%
}_{n}^{Z}\right] \right) ^{2}\right\} &=&\frac{E_{Z}\left\{ \left[ a^{\prime
}\Sigma _{n}^{-1/2}\Upsilon ^{\prime }Z_{2}^{\prime }M^{\left(
Z_{1},Q\right) }e_{\left( 1,1\right) }\right] ^{4}\left( E\left[ \varepsilon
_{\left( 1,1\right) }^{2}|\mathcal{F}_{n}^{Z}\right] \right) ^{2}\right\} }{%
n^{2}} \\
&\leq &\frac{C}{n^{2}}E\left( \left[ a^{\prime }\Sigma _{n}^{-1/2}\Upsilon
^{\prime }Z_{2}^{\prime }M^{\left( Z_{1},Q\right) }e_{\left( 1,1\right) }%
\right] ^{4}\right) \text{ (by Assumption 2(i))} \\
&\leq &CE\left( \frac{a^{\prime }\Sigma _{n}^{-1/2}\Upsilon ^{\prime
}Z_{2}^{\prime }M^{\left( Z_{1},Q\right) }Z_{2}\Upsilon \Sigma _{n}^{-1/2}a}{%
n}\right) ^{2}\text{ (by CS inequality)} \\
&\leq &C\overline{C}=C^{\ast }<\infty \text{ (by Assumption 3(iii) and Lemma
S2-3(d))}
\end{eqnarray*}%
Since the upper bound above does not depend on $n$, we further deduce that

\noindent $\sup_{n}E_{Z}\left\{ \left( E\left[ \mathcal{L}_{\left(
1,1\right) ,n}^{2}|\mathcal{F}_{n}^{Z}\right] \right) ^{2}\right\} <\infty $%
. It follows by the law of iterated expectations and by Theorem 25.12 of
Billingsley (1995) that $E\left( \mathcal{L}_{\left( 1,1\right)
,n}^{2}\right) =E_{Z}\left( E\left[ \mathcal{L}_{\left( 1,1\right) ,n}^{2}|%
\mathcal{F}_{n}^{Z}\right] \right) \rightarrow 0$. Application of Markov's
inequality then allows us to deduce that $\mathcal{L}_{\left( 1,1\right)
,n}=b_{1n}^{\prime }\Upsilon ^{\prime }Z_{2}^{\prime }M^{\left(
Z_{1},Q\right) }e_{\left( 1,1\right) }\varepsilon _{\left( 1,1\right) }/%
\sqrt{n}$

\noindent $=o_{p}\left( 1\right) $, from which we obtain the representation $%
a^{\prime }\Sigma _{n}^{-1/2}\mathcal{Y}_{n}=\mathcal{V}_{n}+o_{p}\left(
1\right) $, where

\noindent $\mathcal{V}_{n}=\dsum\nolimits_{\left( i,t\right) =2}^{m_{n}}%
\mathcal{V}_{\left( i,t\right) ,n}$ with $\mathcal{V}_{\left( i,t\right) ,n}=%
\mathcal{L}_{\left( i,t\right) ,n}+\mathcal{N}_{\left( i,t\right) ,n}$. Note
we can also write $\mathcal{V}_{n}=\mathcal{L}_{n}+\mathcal{N}_{n}$, where $%
\mathcal{L}_{n}=\dsum\nolimits_{\left( i,t\right) =2}^{m_{n}}\mathcal{L}%
_{\left( i,t\right) ,n}$ and $\mathcal{N}_{n}=\dsum\nolimits_{\left(
i,t\right) =2}^{m_{n}}\mathcal{N}_{\left( i,t\right) ,n}$.

\noindent \qquad Next, define the $\sigma $-fields $\mathcal{F}_{\left(
i,t\right) ,n}=\sigma \left( \left\{ \varepsilon _{\left( k,\upsilon \right)
},U_{\left( k,\upsilon \right) }\right\} _{\left( k,\upsilon \right)
=1}^{\left( i,t\right) },Z\right) $ \ for $\left( i,t\right) =1,2,...,m_{n}$%
, note that by construction $\mathcal{F}_{\left( i,t\right) -1,n}\subseteq 
\mathcal{F}_{\left( i,t\right) ,n}$ for $\left( i,t\right) =2,...,m_{n}$ and 
$\mathcal{V}_{\left( i,t\right) ,n}$ is $\mathcal{F}_{\left( i,t\right) ,n}$%
-measurable. Note also that, under Assumption 1, it is easily seen that $E%
\left[ \mathcal{V}_{\left( i,t\right) ,n}|\mathcal{F}_{\left( i,t\right)
-1,n}\right] =0$. In addition, note that, by part (d)\ of Lemma S2-3 and
Lemma S2-6, and Assumption 2(i); 
\begin{eqnarray}
E\left[ \underline{u}_{2,\left( i,t\right) ,n}^{2}|\mathcal{F}_{n}^{Z}\right]
&\leq &\left( b_{2n}^{\prime }b_{2n}\right) \max_{1\leq \left( i,t\right)
\leq m_{n}}E\left[ \left\Vert \underline{U}_{\left( i,t\right) }\right\Vert
_{2}^{2}|\mathcal{F}_{n}^{Z}\right]  \notag \\
&\leq &\frac{K_{2,n}}{\left( \mu _{n}^{\min }\right) ^{2}}a^{\prime }\Sigma
_{n}^{-1}a\max_{1\leq \left( i,t\right) \leq m_{n}}E\left[ \left\Vert 
\underline{U}_{\left( i,t\right) }\right\Vert _{2}^{2}|\mathcal{F}_{n}^{Z}%
\right] =O_{a.s.}\left( 1\right)  \label{var u2 bound}
\end{eqnarray}%
since, for this theorem, we assume that $K_{2,n}/\left( \mu _{n}^{\min
}\right) ^{2}=O\left( 1\right) $. It follows then from straightforward
calculations, from applying the triangle and CS inequalities, as well as
from expression (\ref{var u2 bound}), part (d) of Lemma S2-1, part (d) of
Lemma S2-3, and Assumptions 2(i) and 3(iii) that%
\begin{eqnarray*}
&&Var\left( \mathcal{V}_{\left( i,t\right) ,n}|\mathcal{F}_{n}^{Z}\right) \\
&=&E\left[ \mathcal{L}_{\left( i,t\right) ,n}^{2}|\mathcal{F}_{n}^{Z}\right]
+E\left[ \mathcal{N}_{\left( i,t\right) ,n}^{2}|\mathcal{F}_{n}^{Z}\right] \\
&\leq &\left( \max_{1\leq \left( i,t\right) \leq m_{n}}E\left[ \varepsilon
_{\left( i,t\right) }^{2}|\mathcal{F}_{n}^{Z}\right] \right) a^{\prime
}\Sigma _{n}^{-1}a\lambda _{\max }\left( \frac{\Upsilon ^{\prime
}Z_{2}^{\prime }Z_{2}\Upsilon }{n}\right) \\
&&+\frac{4}{K_{2,n}}\left( \max_{1\leq \left( i,t\right) \leq m_{n}}E\left[
\varepsilon _{\left( i,t\right) }^{2}|\mathcal{F}_{n}^{Z}\right] \right)
\left( \max_{1\leq \left( i,t\right) \leq m_{n}}E\left[ \underline{u}%
_{2,\left( i,t\right) ,n}^{2}|\mathcal{F}_{n}^{Z}\right] \right) \left(
\max_{1\leq \left( i,t\right) \leq m_{n}}\dsum\limits_{\left( j,s\right)
=1}^{m_{n}}A_{\left( i,t\right) ,\left( j,s\right) }^{2}\right) \\
&=&O_{a.s.}\left( 1\right) +O_{a.s.}\left( \frac{K_{2,n}}{\left( \mu
_{n}^{\min }\right) ^{2}n}\right) =O_{a.s.}\left( 1\right)
\end{eqnarray*}%
By the law of iterated expectations and Theorem 16.1 of Billingsley (1995),
there exists a constant $\overline{C}$ such that $Var\left( \mathcal{V}%
_{\left( i,t\right) ,n}\right) =E\left( \mathcal{V}_{\left( i,t\right)
,n}^{2}\right) =E_{Z}\left[ E\left( \mathcal{V}_{\left( i,t\right) ,n}^{2}|%
\mathcal{F}_{n}^{Z}\right) \right] \leq \overline{C}<\infty $ for all $n$
sufficiently large. These results show that $\left\{ \mathcal{V}_{\left(
i,t\right) ,n},\mathcal{F}_{\left( i,t\right) ,n},1\leq \left( i,t\right)
\leq m_{n}\text{, }n\geq 1\right\} $ forms a square-integrable martingale
difference array.

To show the asymptotic normality of $\mathcal{V}_{n}$, we verify the
conditions of the central limit theorem for martingale difference arrays
given in Lemma S2-15. To proceed, first consider condition (\ref{MDA CLT
cond 1b}), which, as noted in the remark following Lemma S2-15, is a
sufficient condition for condition (\ref{MDA CLT cond 1}) of Lemma S2-15. We
shall verify (\ref{MDA CLT cond 1b}) for the case where $\delta =2$. Note
first that, by applying Lo\`{e}ve's $c_{r}$ inequality, we get

\begin{equation*}
\dsum\limits_{\left( i,t\right) =2}^{m_{n}}E\left[ \mathcal{V}_{\left(
i,t\right) ,n}^{4}\right] =\dsum\limits_{\left( i,t\right) =2}^{m_{n}}E\left[
\left( \mathcal{L}_{\left( i,t\right) ,n}+\mathcal{N}_{\left( i,t\right)
,n}\right) ^{4}\right] \leq 8\dsum\limits_{\left( i,t\right) =2}^{m_{n}}E%
\left[ \mathcal{L}_{\left( i,t\right) ,n}^{4}\right] +8\dsum\limits_{\left(
i,t\right) =2}^{m_{n}}E\left[ \mathcal{N}_{\left( i,t\right) ,n}^{4}\right]
\end{equation*}%
Hence, to verify condition (\ref{MDA CLT cond 1b}), it suffices to show that 
$\dsum\nolimits_{\left( i,t\right) =2}^{m_{n}}E\left[ \mathcal{L}_{\left(
i,t\right) ,n}^{4}\right] =o\left( 1\right) $ and

\noindent $\dsum\nolimits_{\left( i,t\right) =2}^{m_{n}}E\left[ \mathcal{N}%
_{\left( i,t\right) ,n}^{4}\right] =o\left( 1\right) $. To do this, we first
focus on a conditional expectation analogue of $\dsum\nolimits_{\left(
i,t\right) =2}^{m_{n}}E\left[ \mathcal{L}_{\left( i,t\right) ,n}^{4}\right] $%
. Note that%
\begin{eqnarray*}
&&\dsum\limits_{\left( i,t\right) =2}^{m_{n}}E\left[ \mathcal{L}_{\left(
i,t\right) ,n}^{4}|\mathcal{F}_{n}^{Z}\right] \\
&=&\frac{1}{n^{2}}\dsum\limits_{\left( i,t\right) =2}^{m_{n}}\left[
a^{\prime }\Sigma _{n}^{-1/2}\Upsilon ^{\prime }Z_{2}^{\prime }M^{\left(
Z_{1},Q\right) }e_{\left( i,t\right) }\right] ^{4}E\left[ \varepsilon
_{\left( i,t\right) }^{4}|\mathcal{F}_{n}^{Z}\right] \\
&\leq &a^{\prime }\Sigma _{n}^{-1}a\frac{1}{n}\dsum\limits_{\left(
i,t\right) =2}^{m_{n}}\left[ a^{\prime }\Sigma _{n}^{-1/2}\Upsilon ^{\prime
}Z_{2}^{\prime }M^{\left( Z_{1},Q\right) }e_{\left( i,t\right) }\right]
^{2}\left( \frac{\left\Vert \Upsilon ^{\prime }Z_{2}^{\prime }M^{\left(
Z_{1},Q\right) }e_{\left( i,t\right) }\right\Vert _{2}}{\sqrt{n}}\right)
^{2}E\left[ \varepsilon _{\left( i,t\right) }^{4}|\mathcal{F}_{n}^{Z}\right] 
\text{ } \\
&&\text{(by CS inequality)}
\end{eqnarray*}%
\begin{eqnarray*}
&\leq &a^{\prime }\Sigma _{n}^{-1}a\left( \max_{1\leq \left( i,t\right) \leq
m_{n}}E\left[ \varepsilon _{\left( i,t\right) }^{4}|\mathcal{F}_{n}^{Z}%
\right] \right) \left( \frac{\max_{1\leq \left( i,t\right) \leq
m_{n}}\left\Vert \Upsilon ^{\prime }Z_{2}^{\prime }M^{\left( Z_{1},Q\right)
}e_{\left( i,t\right) }\right\Vert _{2}}{\sqrt{n}}\right) ^{2} \\
&&\times \frac{1}{n}a^{\prime }\Sigma _{n}^{-1/2}\Upsilon ^{\prime
}Z_{2}^{\prime }M^{\left( Z_{1},Q\right) }\dsum\limits_{\left( i,t\right)
=1}^{m_{n}}e_{\left( i,t\right) }e_{\left( i,t\right) }^{\prime }M^{\left(
Z_{1},Q\right) }Z_{2}\Upsilon \Sigma _{n}^{-1/2}a \\
&\leq &a^{\prime }\Sigma _{n}^{-1}a\left( \max_{1\leq \left( i,t\right) \leq
m_{n}}E\left[ \varepsilon _{\left( i,t\right) }^{4}|\mathcal{F}_{n}^{Z}%
\right] \right) \left( \frac{\max_{1\leq \left( i,t\right) \leq
m_{n}}\left\Vert \Upsilon ^{\prime }Z_{2}^{\prime }M^{\left( Z_{1},Q\right)
}e_{\left( i,t\right) }\right\Vert _{2}}{\sqrt{n}}\right) ^{2} \\
&&\times \frac{a^{\prime }\Sigma _{n}^{-1/2}\Upsilon ^{\prime }Z_{2}^{\prime
}M^{\left( Z_{1},Q\right) }Z_{2}\Upsilon \Sigma _{n}^{-1/2}a}{n} \\
&\leq &\left( a^{\prime }\Sigma _{n}^{-1}a\right) ^{2}\left( \max_{1\leq
\left( i,t\right) \leq m_{n}}E\left[ \varepsilon _{\left( i,t\right) }^{4}|%
\mathcal{F}_{n}^{Z}\right] \right) \left( \frac{\max_{1\leq \left(
i,t\right) \leq m_{n}}\left\Vert \Upsilon ^{\prime }Z_{2}^{\prime }M^{\left(
Z_{1},Q\right) }e_{\left( i,t\right) }\right\Vert _{2}}{\sqrt{n}}\right)
^{2}\lambda _{\max }\left( \frac{\Upsilon ^{\prime }Z_{2}^{\prime
}Z_{2}\Upsilon }{n}\right) \\
&\leq &C\left( \frac{\max_{1\leq \left( i,t\right) \leq m_{n}}\left\Vert
\Upsilon ^{\prime }Z_{2}^{\prime }M^{\left( Z_{1},Q\right) }e_{\left(
i,t\right) }\right\Vert _{2}}{\sqrt{n}}\right) ^{2}=o_{p}\left( 1\right) 
\text{ }
\end{eqnarray*}%
where the last line above follows from Assumptions 2(i), 3(iii), and 7 and
by Lemma S2-3(d). Next, note that, under Assumptions 2 and 3(iii), there
exists a positive constant $C^{\ast }$ such that%
\begin{eqnarray*}
&&E_{Z}\left( \dsum\limits_{\left( i,t\right) =2}^{m_{n}}E\left[ \mathcal{L}%
_{\left( i,t\right) ,n}^{4}|\mathcal{F}_{n}^{Z}\right] \right) ^{2} \\
&=&\frac{1}{n^{4}}\dsum\limits_{\left( i,t\right)
=2}^{m_{n}}\dsum\limits_{\left( j,s\right) =2}^{m_{n}}E_{Z}\left( \left[
a^{\prime }\Sigma _{n}^{-1/2}\Upsilon ^{\prime }Z_{2}^{\prime }M^{\left(
Z_{1},Q\right) }e_{\left( i,t\right) }\right] ^{4}\left[ a^{\prime }\Sigma
_{n}^{-1/2}\Upsilon ^{\prime }Z_{2}^{\prime }M^{\left( Z_{1},Q\right)
}e_{\left( j,s\right) }\right] ^{4}\right. \\
&&\text{ \ \ \ \ \ \ \ \ \ \ \ \ \ \ \ \ \ \ \ \ \ \ \ \ }\left. \times E%
\left[ \varepsilon _{\left( i,t\right) }^{4}|\mathcal{F}_{n}^{Z}\right] E%
\left[ \varepsilon _{\left( j,s\right) }^{4}|\mathcal{F}_{n}^{Z}\right]
\right) \\
&\leq &\frac{C}{n^{4}}\dsum\limits_{\left( i,t\right)
=2}^{m_{n}}\dsum\limits_{\left( j,s\right) =2}^{m_{n}}E_{Z}\left( \left[
a^{\prime }\Sigma _{n}^{-1/2}\Upsilon ^{\prime }Z_{2}^{\prime }M^{\left(
Z_{1},Q\right) }e_{\left( i,t\right) }\right] ^{4}\left[ a^{\prime }\Sigma
_{n}^{-1/2}\Upsilon ^{\prime }Z_{2}^{\prime }M^{\left( Z_{1},Q\right)
}e_{\left( j,s\right) }\right] ^{4}\right)
\end{eqnarray*}%
\begin{eqnarray*}
\text{ } &\leq &\frac{C}{n^{4}}E_{Z}\left\{ a^{\prime }\Sigma
_{n}^{-1/2}\Upsilon ^{\prime }Z_{2}^{\prime }M^{\left( Z_{1},Q\right)
}\dsum\limits_{\left( i,t\right) =1}^{m_{n}}e_{\left( i,t\right) }e_{\left(
i,t\right) }^{\prime }M^{\left( Z_{1},Q\right) }Z_{2}\Upsilon \Sigma
_{n}^{-1/2}aa^{\prime }\Sigma _{n}^{-1/2}\Upsilon ^{\prime }Z_{2}^{\prime
}M^{\left( Z_{1},Q\right) }\right. \\
&&\text{ \ \ \ \ \ \ \ \ \ \ }\left. \times \text{\ }\dsum\limits_{\left(
j,s\right) =1}^{m_{n}}e_{\left( j,s\right) }e_{\left( j,s\right) }^{\prime
}M^{\left( Z_{1},Q\right) }Z_{2}\Upsilon \Sigma _{n}^{-1/2}a\left( a^{\prime
}\Sigma _{n}^{-1/2}\Upsilon ^{\prime }Z_{2}^{\prime }M^{\left(
Z_{1},Q\right) }Z_{2}\Upsilon \Sigma _{n}^{-1/2}a\right) ^{2}\right\} \text{
\ } \\
&=&CE_{Z}\left( \frac{a^{\prime }\Sigma _{n}^{-1/2}\Upsilon ^{\prime
}Z_{2}^{\prime }M^{\left( Z_{1},Q\right) }Z_{2}\Upsilon \Sigma _{n}^{-1/2}a}{%
n}\right) ^{4} \\
&\leq &C\overline{C}=C^{\ast }<\infty \text{ \ (by Assumption 3(iii) and
Lemma S2-3(d))}
\end{eqnarray*}%
where the second inequality above follows from applying the CS inequality.
Since the upper bound above does not depend on $n$, we further deduce that

\noindent $\sup_{n}E_{Z}\left( \dsum\nolimits_{\left( i,t\right) =2}^{m_{n}}E%
\left[ \mathcal{L}_{\left( i,t\right) ,n}^{4}|\mathcal{F}_{n}^{Z}\right]
\right) ^{2}<\infty $. It follows by the law of iterated expectations and by
Theorem 25.12 of Billingsley (1995) that $\dsum\nolimits_{\left( i,t\right)
=2}^{m_{n}}E\left[ \mathcal{L}_{\left( i,t\right) ,n}^{4}\right]
=\dsum\nolimits_{\left( i,t\right) =2}^{m_{n}}E_{Z}\left( E\left[ \mathcal{L}%
_{\left( i,t\right) ,n}^{4}|\mathcal{F}_{n}^{Z}\right] \right) \rightarrow 0$%
.

Turning our attention to the bilinear term, note that by Lo\`{e}ve's $c_{r}$
inequality we have

\noindent $\dsum\nolimits_{\left( i,t\right) =2}^{m_{n}}E\left[ \mathcal{N}%
_{\left( i,t\right) ,n}^{4}|\mathcal{F}_{n}^{Z}\right] \leq \mathcal{R}_{1}+%
\mathcal{R}_{2}$, where

\noindent $\mathcal{R}_{1}=\dsum\nolimits_{\left( i,t\right)
=2}^{m_{n}}\left( 8/K_{2,n}^{2}\right) E\left[ \left( \dsum\nolimits_{\left(
j,s\right) =1}^{\left( i,t\right) -1}A_{\left( i,t\right) ,\left( j,s\right)
}\underline{u}_{2,\left( i,t\right) }\varepsilon _{\left( j,s\right)
}\right) ^{4}|\mathcal{F}_{n}^{Z}\right] $ and

\noindent $\mathcal{R}_{2}=\dsum\nolimits_{\left( i,t\right)
=2}^{m_{n}}\left( 8/K_{2,n}^{2}\right) E\left[ \left( \dsum\nolimits_{\left(
j,s\right) =1}^{\left( i,t\right) -1}A_{\left( i,t\right) ,\left( j,s\right)
}\underline{u}_{2,\left( j,s\right) }\varepsilon _{\left( i,t\right)
}\right) ^{4}|\mathcal{F}_{n}^{Z}\right] $. Focusing first on the term $%
\mathcal{R}_{1}$, note that, by straightforward calculations as well as by
making use of Assumptions 2(i) and 5(ii), parts (b) and (c) of Lemma S2-1,
part (d) of Lemma S2-3, and Lemma S2-6; we deduce that, there exists a
positive constant $\overline{C}$ such that%
\begin{eqnarray*}
\frac{\left( \mu _{n}^{\min }\right) ^{4}n}{K_{2,n}^{2}}\mathcal{R}_{1}
&\leq &24n\left( a^{\prime }\Sigma _{n}^{-1}a\right) ^{2}\left( \max_{1\leq
\left( i,t\right) \leq m_{n}}E\left[ \left\Vert \underline{U}_{\left(
i,t\right) }\right\Vert _{2}^{4}|\mathcal{F}_{n}^{Z}\right] \right) \left(
\max_{1\leq \left( i,t\right) \leq m_{n}}E\left[ \varepsilon _{\left(
i,t\right) }^{4}|\mathcal{F}_{n}^{Z}\right] \right) \\
&&\times \left[ \frac{1}{K_{2,n}^{2}}\dsum\limits_{\substack{ \left(
i,t\right) ,\left( j,s\right) =1  \\ \left( i,t\right) \neq \left(
j,s\right) }}^{m_{n}}A_{\left( i,t\right) ,\left( j,s\right) }^{4}+\frac{1}{%
K_{2,n}^{2}}\dsum\limits_{\left( i,t\right) =1}^{m_{n}}\dsum\limits 
_{\substack{ \left( j,s\right) ,\left( k,\upsilon \right) =1  \\ \left(
j,s\right) \neq \left( i,t\right) ,\left( k,\upsilon \right) \neq \left(
i,t\right) }}^{m_{n}}A_{\left( i,t\right) ,\left( j,s\right) }^{2}A_{\left(
i,t\right) ,\left( k,\upsilon \right) }^{2}\right] \\
&\leq &\overline{C}n\left[ \frac{1}{K_{2,n}^{2}}\dsum\limits_{\substack{ %
\left( i,t\right) ,\left( j,s\right) =1  \\ \left( i,t\right) \neq \left(
j,s\right) }}^{m_{n}}A_{\left( i,t\right) ,\left( j,s\right) }^{4}+\frac{1}{%
K_{2,n}^{2}}\dsum\limits_{\left( i,t\right) =1}^{m_{n}}\dsum\limits 
_{\substack{ \left( j,s\right) ,\left( k,\upsilon \right) =1  \\ \left(
j,s\right) \neq \left( i,t\right) ,\left( k,\upsilon \right) \neq \left(
i,t\right) }}^{m_{n}}A_{\left( i,t\right) ,\left( j,s\right) }^{2}A_{\left(
i,t\right) ,\left( k,\upsilon \right) }^{2}\right] \\
&=&O_{a.s.}\left( \frac{K_{2,n}}{n}\right) +O_{a.s.}\left( 1\right)
=O_{a.s.}\left( 1\right) \text{.}
\end{eqnarray*}%
Applying the law of iterated expectations and Theorem 16.1 of Billingsley
(1995), we then have%
\begin{eqnarray*}
&&\frac{\left( \mu _{n}^{\min }\right) ^{4}n}{K_{2,n}^{2}}E_{Z}\left( 
\mathcal{R}_{1}\right) \\
&\leq &\overline{C}nE_{Z}\left[ \frac{1}{K_{2,n}^{2}}\dsum\limits_{\substack{
\left( i,t\right) ,\left( j,s\right) =1  \\ \left( i,t\right) \neq \left(
j,s\right) }}^{m_{n}}A_{\left( i,t\right) ,\left( j,s\right) }^{4}+\frac{1}{%
K_{2,n}^{2}}\dsum\limits_{\left( i,t\right) =1}^{m_{n}}\dsum\limits 
_{\substack{ \left( j,s\right) ,\left( k,\upsilon \right) =1  \\ \left(
j,s\right) \neq \left( i,t\right) ,\left( k,\upsilon \right) \neq \left(
i,t\right) }}^{m_{n}}A_{\left( i,t\right) ,\left( j,s\right) }^{2}A_{\left(
i,t\right) ,\left( k,\upsilon \right) }^{2}\right] \\
&=&O\left( 1\right)
\end{eqnarray*}%
from which we further deduce that%
\begin{equation*}
E_{Z}\left( \mathcal{R}_{1}\right) =\dsum\limits_{\left( i,t\right)
=2}^{m_{n}}\frac{8}{K_{2,n}^{2}}E\left[ \left( \dsum\nolimits_{\left(
j,s\right) =1}^{\left( i,t\right) -1}A_{\left( i,t\right) ,\left( j,s\right)
}\underline{u}_{2,\left( i,t\right) }\varepsilon _{\left( j,s\right)
}\right) ^{4}\right] =O\left( \frac{K_{2,n}^{2}}{\left( \mu _{n}^{\min
}\right) ^{4}n}\right) =o\left( 1\right)
\end{equation*}%
In a similar way, we can also show that

\noindent $E_{Z}\left( \mathcal{R}_{2}\right) =\left( 8/K_{2,n}^{2}\right) E%
\left[ \dsum\nolimits_{\left( i,t\right) =2}^{m_{n}}\left(
\dsum\nolimits_{\left( j,s\right) =1}^{\left( i,t\right) -1}A_{\left(
j,s\right) ,\left( i,t\right) }\underline{u}_{2,\left( j,s\right)
}\varepsilon _{\left( i,t\right) }\right) ^{4}\right] =o\left( 1\right) $.
It follows that

\noindent $\dsum\nolimits_{\left( i,t\right) =2}^{m_{n}}E\left[ \mathcal{N}%
_{\left( i,t\right) ,n}^{4}\right] \leq E_{Z}\left( \mathcal{R}_{1}\right)
+E_{Z}\left( \mathcal{R}_{2}\right) =o\left( 1\right) $. This verifies
condition (\ref{MDA CLT cond 1b}).

Next, we verify condition (\ref{MDA CLT cond 2}) of Lemma S2-15. To proceed,
first let $s_{Z}^{2}=Var\left[ \mathcal{V}_{n}|\mathcal{F}_{n}^{Z}\right]
=Var\left( \dsum\nolimits_{\left( i,t\right) =2}^{m_{n}}\mathcal{V}_{\left(
i,t\right) ,n}|\mathcal{F}_{n}^{Z}\right) $, and note that%
\begin{equation}
s_{Z}^{2}=Var\left( \frac{b_{1n}^{\prime }\Upsilon ^{\prime }Z_{2}^{\prime
}M^{\left( Z_{1},Q\right) }\varepsilon }{\sqrt{n}}+\frac{b_{2n}^{\prime }%
\underline{U}^{\prime }A\varepsilon }{\sqrt{K_{2,n}}}|\mathcal{F}%
_{n}^{Z}\right) +o_{p}\left( 1\right) =a^{\prime }\Sigma _{n}^{-1/2}\Sigma
_{n}\Sigma _{n}^{-1/2}a+o_{p}\left( 1\right) =1+o_{p}\left( 1\right)
\label{limit of s2}
\end{equation}

\noindent On the other hand, by straightforward calculation, we can write%
\begin{eqnarray}
s_{Z}^{2} &=&\frac{1}{n}\dsum\limits_{\left( i,t\right) =2}^{m_{n}}\left[
b_{1n}^{\prime }\Upsilon ^{\prime }Z_{2}^{\prime }M^{\left( Z_{1},Q\right)
}e_{\left( i,t\right) }\right] ^{2}E\left[ \varepsilon _{\left( i,t\right)
}^{2}|\mathcal{F}_{n}^{Z}\right]  \notag \\
&&+\frac{1}{K_{2,n}}\dsum\limits_{\left( i,t\right)
=2}^{m_{n}}\dsum\limits_{\left( j,s\right) =1}^{\left( i,t\right)
-1}A_{\left( i,t\right) ,\left( j,s\right) }^{2}\left\{ E\left[ \underline{u}%
_{2,\left( i,t\right) }^{2}|\mathcal{F}_{n}^{Z}\right] E\left[ \varepsilon
_{\left( j,s\right) }^{2}|\mathcal{F}_{n}^{Z}\right] +E\left[ \underline{u}%
_{2,\left( j,s\right) }^{2}|\mathcal{F}_{n}^{Z}\right] E\left[ \varepsilon
_{\left( i,t\right) }^{2}|\mathcal{F}_{n}^{Z}\right] \right\}  \notag \\
&&+\frac{2}{K_{2,n}}\dsum\limits_{\left( i,t\right)
=2}^{m_{n}}\dsum\limits_{\left( j,s\right) =1}^{\left( i,t\right)
-1}A_{\left( i,t\right) ,\left( j,s\right) }^{2}E\left[ \underline{u}%
_{2,\left( i,t\right) }\varepsilon _{\left( i,t\right) }|\mathcal{F}_{n}^{Z}%
\right] E\left[ \underline{u}_{2,\left( j,s\right) }\varepsilon _{\left(
j,s\right) }|\mathcal{F}_{n}^{Z}\right]  \label{s^2}
\end{eqnarray}%
Making use of expression (\ref{s^2}), we obtain, after some further
calculations,%
\begin{eqnarray*}
&&\dsum\limits_{\left( i,t\right) =2}^{m_{n}}E\left[ \mathcal{V}_{\left(
i,t\right) ,n}^{2}|\mathcal{F}_{\left( i,t\right) -1,n}\right] -s_{Z}^{2} \\
&=&\frac{2}{\sqrt{n}}\dsum\limits_{\left( i,t\right)
=2}^{m_{n}}\dsum\limits_{\left( j,s\right) =1}^{\left( i,t\right) -1}\left[
b_{1n}^{\prime }\Upsilon ^{\prime }Z_{2}^{\prime }M^{\left( Z_{1},Q\right)
}e_{\left( i,t\right) }\right] \frac{A_{\left( i,t\right) ,\left( j,s\right)
}}{\sqrt{K_{2,n}}}\left\{ \varepsilon _{\left( j,s\right) }E\left[
\varepsilon _{\left( i,t\right) }\underline{u}_{2,\left( i,t\right) }|%
\mathcal{F}_{n}^{Z}\right] +\underline{u}_{2,\left( j,s\right) }E\left[
\varepsilon _{\left( i,t\right) }^{2}|\mathcal{F}_{n}^{Z}\right] \right\} \\
&&+\dsum\limits_{\left( i,t\right) =2}^{m_{n}}\dsum\limits_{\left(
j,s\right) =1}^{\left( i,t\right) -1}\frac{A_{\left( i,t\right) ,\left(
j,s\right) }^{2}}{K_{2,n}}\left( \varepsilon _{\left( j,s\right) }^{2}-E%
\left[ \varepsilon _{\left( j,s\right) }^{2}|\mathcal{F}_{n}^{Z}\right]
\right) E\left[ \underline{u}_{2,\left( i,t\right) }^{2}|\mathcal{F}_{n}^{Z}%
\right] \\
&&+\dsum\limits_{\left( i,t\right) =2}^{m_{n}}\dsum\limits_{\left(
j,s\right) =1}^{\left( i,t\right) -1}\frac{A_{\left( i,t\right) ,\left(
j,s\right) }^{2}}{K_{2,n}}\left( \underline{u}_{2,\left( j,s\right) }^{2}-E%
\left[ \underline{u}_{2,\left( j,s\right) }^{2}|\mathcal{F}_{n}^{Z}\right]
\right) E\left[ \varepsilon _{\left( i,t\right) }^{2}|\mathcal{F}_{n}^{Z}%
\right] \\
&&+2\dsum\limits_{\left( i,t\right) =2}^{m_{n}}\dsum\limits_{\left(
j,s\right) =1}^{\left( i,t\right) -1}\frac{A_{\left( i,t\right) ,\left(
j,s\right) }^{2}}{K_{2,n}}\left( \varepsilon _{\left( j,s\right) }\underline{%
u}_{2,\left( j,s\right) }-E\left[ \underline{u}_{2,\left( j,s\right)
}\varepsilon _{\left( j,s\right) }|\mathcal{F}_{n}^{Z}\right] \right) E\left[
\underline{u}_{2,\left( i,t\right) }\varepsilon _{\left( i,t\right) }|%
\mathcal{F}_{n}^{Z}\right] \\
&&+2\dsum\limits_{\left( i,t\right) =3}^{m_{n}}\dsum\limits_{\left(
j,s\right) =2}^{\left( i,t\right) -1}\dsum\limits_{\left( k,\upsilon \right)
=1}^{\left( j,s\right) -1}\frac{A_{\left( i,t\right) ,\left( j,s\right)
}A_{\left( i,t\right) ,\left( k,\upsilon \right) }}{K_{2,n}}E\left[ 
\underline{u}_{2,\left( i,t\right) }\varepsilon _{\left( i,t\right) }|%
\mathcal{F}_{n}^{Z}\right] \left\{ \underline{u}_{2,\left( j,s\right)
}\varepsilon _{\left( k,\upsilon \right) }+\varepsilon _{\left( j,s\right) }%
\underline{u}_{2,\left( k,\upsilon \right) }\right\}
\end{eqnarray*}%
\begin{eqnarray*}
&&+2\dsum\limits_{\left( i,t\right) =3}^{m_{n}}\dsum\limits_{\left(
j,s\right) =2}^{\left( i,t\right) -1}\dsum\limits_{\left( k,\upsilon \right)
=1}^{\left( j,s\right) -1}\frac{A_{\left( i,t\right) ,\left( j,s\right)
}A_{\left( i,t\right) ,\left( k,\upsilon \right) }}{K_{2,n}}\varepsilon
_{\left( j,s\right) }\varepsilon _{\left( k,\upsilon \right) }E\left[ 
\underline{u}_{2,\left( i,t\right) }^{2}|\mathcal{F}_{n}^{Z}\right] \\
&&+2\dsum\limits_{\left( i,t\right) =3}^{m_{n}}\dsum\limits_{\left(
j,s\right) =2}^{\left( i,t\right) -1}\dsum\limits_{\left( k,\upsilon \right)
=1}^{\left( j,s\right) -1}\frac{A_{\left( i,t\right) ,\left( j,s\right)
}A_{\left( i,t\right) ,\left( k,\upsilon \right) }}{K_{2,n}}\underline{u}%
_{2,\left( j,s\right) }\underline{u}_{2,\left( k,\upsilon \right) }E\left[
\varepsilon _{\left( i,t\right) }^{2}|\mathcal{F}_{n}^{Z}\right] \\
&=&\mathcal{T}_{1}+\mathcal{T}_{2}+\mathcal{T}_{3}+\mathcal{T}_{4}+\mathcal{T%
}_{5}+\mathcal{T}_{6}+\mathcal{T}_{7},\text{ }\left( say\right)
\end{eqnarray*}

Note first that, by applying parts (a)-(c) of Lemma S2-14, we have $\mathcal{%
T}_{1}\overset{p}{\rightarrow }0$, $\mathcal{T}_{2}\overset{p}{\rightarrow }%
0 $, and $\mathcal{T}_{3}\overset{p}{\rightarrow }0$. Consider next the term%
\begin{equation*}
\mathcal{T}_{4}=2\dsum\limits_{\left( i,t\right)
=2}^{m_{n}}\dsum\limits_{\left( j,s\right) =1}^{\left( i,t\right) -1}\frac{%
A_{\left( i,t\right) ,\left( j,s\right) }^{2}}{K_{2,n}}\left( \underline{u}%
_{2,\left( j,s\right) }\varepsilon _{\left( j,s\right) }-E\left[ \underline{u%
}_{2,\left( j,s\right) }\varepsilon _{\left( j,s\right) }|\mathcal{F}_{n}^{Z}%
\right] \right) E\left[ \underline{u}_{2,\left( i,t\right) }\varepsilon
_{\left( i,t\right) }|\mathcal{F}_{n}^{Z}\right] .
\end{equation*}%
In this case, we apply part (a) of Lemma S2-8 with $u_{\left( j,s\right) }=%
\underline{u}_{2,\left( j,s\right) }$, $\overline{\psi }_{\left( j,s\right)
}=E\left[ \underline{u}_{2,\left( j,s\right) }\varepsilon _{\left(
j,s\right) }|\mathcal{F}_{n}^{Z}\right] $, and $\phi _{\left( i,t\right) }=E%
\left[ \underline{u}_{2,\left( i,t\right) }\varepsilon _{\left( i,t\right) }|%
\mathcal{F}_{n}^{Z}\right] $. Note that, in this case, $\left\{ \left( 
\underline{u}_{2,\left( i,t\right) },\varepsilon _{\left( i,t\right)
}\right) \right\} _{\left( i,t\right) =1}^{m_{n}}$ is independent
conditional on $\mathcal{F}_{n}^{Z}$, and $\max_{1\leq \left( i,t\right)
\leq m_{n}}E\left[ \varepsilon _{\left( i,t\right) }^{4}|\mathcal{F}_{n}^{Z}%
\right] \leq C$ $a.s.$ by Assumptions 1 and 2(i), respectively. Moreover,
note that Assumption 2, part (d) of Lemma S2-3, Lemma S2-6, and the fact
that $K_{2,n}/\left( \mu _{n}^{\min }\right) ^{2}=O\left( 1\right) $ in this
case together imply that there exists a constant $C\geq 1$ such that $E\left[
\underline{u}_{2,\left( i,t\right) }^{4}|\mathcal{F}_{n}^{Z}\right] \leq %
\left[ K_{2,n}^{2}/\left( \mu _{n}^{\min }\right) ^{4}\right] E\left[
\left\Vert \underline{U}_{\left( i,t\right) }\right\Vert _{2}^{4}|\mathcal{F}%
_{n}^{Z}\right] \left( a^{\prime }\Sigma _{n}^{-1}a\right) ^{2}\leq C<\infty 
$ \ $a.s.$ for all $\left( i,t\right) \in \left\{ 1,2,...,m_{n}\right\} $
and for all $n$ sufficiently large, so that

\noindent $\max_{1\leq \left( i,t\right) \leq m_{n}}E\left[ \underline{u}%
_{2,\left( i,t\right) }^{4}|\mathcal{F}_{n}^{Z}\right] \leq C$ \ $a.s.n.$
Finally, using the upper bound derived in expression (29) in the proof of
part (a) of Lemma S2-14\footnote{%
A proof of Lemma S2-14 is given in section 1 of the Additional Online
Appendix which can be viewed at the URL:
\par
\noindent http://econweb.umd.edu/\symbol{126}chao/Research/research%
\_files/Additional\_Online\_Appendix\_Jackknife\_Estimation\_
\par
\noindent Cluster\_Sample\_IV\_Model\_August\_19\_2022.pdf}, we obtain

\noindent $\max_{1\leq \left( i,t\right) \leq m_{n}}\left\vert \phi _{\left(
i,t\right) }\right\vert \leq \max_{1\leq \left( i,t\right) \leq m_{n}}E\left[
\left\vert \underline{u}_{2,\left( i,t\right) }\varepsilon _{\left(
i,t\right) }\right\vert |\mathcal{F}_{n}^{Z}\right] \leq C$ $a.s.n.$ and $%
\max_{1\leq \left( j,s\right) \leq m_{n}}\left\vert \overline{\psi }_{\left(
j,s\right) }\right\vert \leq \max_{1\leq \left( i,t\right) \leq m_{n}}E\left[
\left\vert \underline{u}_{2,\left( j,s\right) }\varepsilon _{\left(
j,s\right) }\right\vert |\mathcal{F}_{n}^{Z}\right] \leq C$ $a.s.n.$ It
follows by part (a) of Lemma S2-8 that $\mathcal{T}_{4}\overset{p}{%
\rightarrow }0$.

Now, consider $\mathcal{T}_{5}$. Here, we apply part (b) of Lemma S2-8 with $%
u_{\left( j,s\right) }=\underline{u}_{2,\left( j,s\right) }$ and $\phi
_{\left( i,t\right) }=E\left[ \underline{u}_{2,\left( i,t\right)
}\varepsilon _{\left( i,t\right) }|\mathcal{F}_{n}^{Z}\right] $. Note again
that $\left\{ \left( \underline{u}_{2,\left( i,t\right) },\varepsilon
_{\left( i,t\right) }\right) \right\} _{\left( i,t\right) =1}^{m_{n}}$ is
independent conditional on $\mathcal{F}_{n}^{Z}$, and $\max_{1\leq \left(
i,t\right) \leq m_{n}}E\left[ \varepsilon _{\left( i,t\right) }^{4}|\mathcal{%
F}_{n}^{Z}\right] \leq C$ $a.s.$ by Assumptions 1 and 2(i), respectively.
Moreover, previously, we have shown that $E\left[ \underline{u}_{2,\left(
i,t\right) }^{4}|\mathcal{F}_{n}^{Z}\right] \leq C$ \ $a.s.n.$ and $%
\max_{1\leq \left( i,t\right) \leq m_{n}}\left\vert \phi _{\left( i,t\right)
}\right\vert \leq C$ $a.s.n.$ Hence, applying part (b) of Lemma S2-8, we
deduce that $\mathcal{T}_{5}\overset{p}{\rightarrow }0$.

Turning our attention to $\mathcal{T}_{6}$, we note that, for this term, we
can apply part (c) of Lemma S2-8 with $\phi _{\left( i,t\right) }=E\left[ 
\underline{u}_{2,\left( i,t\right) }^{2}|\mathcal{F}_{n}^{Z}\right] $. From (%
\ref{var u2 bound}), there exists a positive constant $C$ such that $E\left[ 
\underline{u}_{2,\left( i,t\right) }^{2}|\mathcal{F}_{n}^{Z}\right] \leq
C<\infty $ \ $a.s.$ for all $\left( i,t\right) \in \left\{
1,2,...,m_{n}\right\} $ and for all $n$ sufficiently large, so that

\noindent $\max_{1\leq \left( i,t\right) \leq m_{n}}\left\vert \phi _{\left(
i,t\right) }\right\vert =\max_{1\leq \left( i,t\right) \leq m_{n}}E\left[ 
\underline{u}_{2,\left( i,t\right) }^{2}|\mathcal{F}_{n}^{Z}\right] \leq C$ $%
a.s.n.$ Hence, applying part (c) of Lemma S2-8, we obtain $\mathcal{T}_{6}%
\overset{p}{\rightarrow }0.$

Finally, consider $\mathcal{T}_{7}$. In this case, we apply part (d) of
Lemma S2-8 with $u_{\left( j,s\right) }=\underline{u}_{2,\left( j,s\right) }$%
, $u_{\left( k,\upsilon \right) }=\underline{u}_{2,\left( k,\upsilon \right)
}$, and $\phi _{\left( i,t\right) }=E\left[ \varepsilon _{\left( i,t\right)
}^{2}|\mathcal{F}_{n}^{Z}\right] $. Using a conditional version of
Liapounov's inequality and Assumption 2(i), we obtain $E\left[ \varepsilon
_{\left( i,t\right) }^{2}|\mathcal{F}_{n}^{Z}\right] \leq \left( E\left[
\varepsilon _{\left( i,t\right) }^{4}|\mathcal{F}_{n}^{Z}\right] \right)
^{1/2}\leq C<\infty $ \ $a.s.$ for all $\left( i,t\right) \in \left\{
1,2,...,m_{n}\right\} $ and for all $n$, so that $\max_{1\leq \left(
i,t\right) \leq m_{n}}\left\vert \phi _{\left( i,t\right) }\right\vert $

\noindent $=\max_{1\leq \left( i,t\right) \leq m_{n}}E\left[ \varepsilon
_{\left( i,t\right) }^{2}|\mathcal{F}_{n}^{Z}\right] \leq C$ $a.s.$
Moreover, as noted previously, Assumption 2, part (d) of Lemma S2-3, Lemma
S2-6, and the fact that $K_{2,n}/\left( \mu _{n}^{\min }\right) ^{2}=O\left(
1\right) $ together imply that

\noindent $\max_{1\leq \left( i,t\right) \leq m_{n}}E\left[ \underline{u}%
_{2,\left( i,t\right) }^{4}|\mathcal{F}_{n}^{Z}\right] \leq C$ \ $a.s.n.$ It
follows by applying part (d) of Lemma S2-8 that $\mathcal{T}_{7}\overset{p}{%
\rightarrow }0.$

The above argument shows that $\dsum\nolimits_{\left( i,t\right) =2}^{m_{n}}E%
\left[ \mathcal{V}_{\left( i,t\right) ,n}^{2}|\mathcal{F}_{\left( i,t\right)
-1,n}\right] -s_{Z}^{2}=\dsum\nolimits_{k=1}^{7}\mathcal{T}_{k}=o_{p}\left(
1\right) $. On the other hand, expression (\ref{limit of s2}) above implies
that $s_{Z}^{2}-1=o_{p}\left( 1\right) $. Putting these two results
together, we obtain $\dsum\nolimits_{\left( i,t\right) =2}^{m_{n}}E\left[ 
\mathcal{V}_{\left( i,t\right) ,n}^{2}|\mathcal{F}_{\left( i,t\right) -1,n}%
\right] -1=o_{p}\left( 1\right) $, which establishes condition (\ref{MDA CLT
cond 2}) of Lemma S2-15.

It now follows from Lemma S2-15 that

\noindent $\mathcal{V}_{n}=\dsum\nolimits_{\left( i,t\right)
=2}^{m_{n}}\left\{ b_{1n}^{\prime }\Upsilon ^{\prime }Z_{2}^{\prime
}M^{Q}e_{\left( i,t\right) }\varepsilon _{\left( i,t\right) }/\sqrt{n}%
+\dsum\nolimits_{\left( j,s\right) =1}^{\left( i,t\right) -1}A_{\left(
i,t\right) ,\left( j,s\right) }\left[ \underline{u}_{2,\left( i,t\right)
}\varepsilon _{\left( j,s\right) }+\underline{u}_{2,\left( j,s\right)
}\varepsilon _{\left( i,t\right) }\right] \right\} \overset{d}{\rightarrow }%
N\left( 0,1\right) $. Since, previously, we have shown that $a^{\prime
}\Sigma _{n}^{-1/2}\mathcal{Y}_{n}=\mathcal{V}_{n}+o_{p}\left( 1\right) $,
this further implies that $a^{\prime }\Sigma _{n}^{-1/2}\mathcal{Y}_{n}%
\overset{d}{\rightarrow }N\left( 0,1\right) $. Given that this result holds
for all $a\in \mathbb{R}^{d}$ such that $\left\Vert a\right\Vert _{2}=1$, we
can then apply the Cram\'{e}r-Wold device to obtain%
\begin{equation}
\Sigma _{n}^{-1/2}\mathcal{Y}_{n}=\Sigma _{n}^{-1/2}\left( \frac{\Upsilon
^{\prime }Z_{2}^{\prime }M^{\left( Z_{1},Q\right) }\varepsilon }{\sqrt{n}}%
+D_{\mu }^{-1}\underline{U}^{\prime }A\varepsilon \right) \overset{d}{%
\rightarrow }N\left( 0,I_{d}\right)  \label{asy normality V}
\end{equation}%
Next, let $H_{n}=\Upsilon ^{\prime }Z_{2}^{\prime }M^{\left( Z_{1},Q\right)
}Z_{2}\Upsilon /n$, $\Lambda _{I,n}=H_{n}^{-1}\Sigma _{n}H_{n}^{-1}$, and $%
\mathcal{Y}_{n}=\Upsilon ^{\prime }Z_{2}^{\prime }M^{\left( Z_{1},Q\right)
}\varepsilon /\sqrt{n}+D_{\mu }^{-1}\underline{U}^{\prime }A\varepsilon $,
as given above. Consider first $\widehat{\delta }_{L,n}$. Theorem 1 has
already shown that $\widehat{\delta }_{L,n}\overset{p}{\rightarrow }\delta
_{0}$. To show asymptotic normality of $\widehat{\delta }_{L}$, note first
that, by Lemma S2-11, $\widehat{\delta }_{L,n}$ satisfies the set of
(normalized) first-order conditions $\widehat{\Delta }\left( \widehat{\delta 
}_{L,n}\right) =0$, where

\noindent $\widehat{\Delta }\left( \delta \right) =-\left[ \left( y-X\delta
\right) ^{\prime }M^{\left( Z_{1},Q\right) }\left( y-X\delta \right) /2%
\right] \left[ \partial \widehat{Q}_{FELIM}\left( \delta \right) /\partial
\delta \right] $. Applying the mean-value theorem to each component of $%
\widehat{\Delta }\left( \delta \right) $ and expanding it around the point $%
\delta =\delta _{0}$, we obtain $0=\widehat{\Delta }\left( \widehat{\delta }%
_{L,n}\right) =\widehat{\Delta }\left( \delta _{0}\right) +\left( \partial 
\widehat{\Delta }\left( \overline{\delta }_{n}\right) /\partial \delta
^{\prime }\right) \left( \widehat{\delta }_{L,n}-\delta _{0}\right) $, with $%
\overline{\delta }_{n}$ lying on the line segment between $\widehat{\delta }%
_{L,n}$ and $\delta _{0}$. Multiplying both sides of this equation by $%
D_{\mu }^{-1}$, we further obtain%
\begin{equation}
0=D_{\mu }^{-1}\widehat{\Delta }\left( \delta _{0}\right) +D_{\mu }^{-1}%
\frac{\partial \widehat{\Delta }\left( \overline{\delta }_{n}\right) }{%
\partial \delta ^{\prime }}\left( \widehat{\delta }_{L,n}-\delta _{0}\right)
=D_{\mu }^{-1}\widehat{\Delta }\left( \delta _{0}\right) +D_{\mu }^{-1}\frac{%
\partial \widehat{\Delta }\left( \overline{\delta }_{n}\right) }{\partial
\delta ^{\prime }}D_{\mu }^{-1}D_{\mu }\left( \widehat{\delta }_{L,n}-\delta
_{0}\right)  \label{mean value approx}
\end{equation}%
From the result of Lemma S2-10, we have $-D_{\mu }^{-1}\left( \partial 
\widehat{\Delta }\left( \overline{\delta }_{n}\right) /\partial \delta
^{\prime }\right) D_{\mu }^{-1}=H_{n}+o_{p}\left( 1\right) $, where $%
H_{n}=\Upsilon ^{\prime }Z_{2}^{\prime }M^{\left( Z_{1},Q\right)
}Z_{2}\Upsilon /n$ is a positive definite matrix $a.s.n.$ by Assumption
3(iii), which, in turn, implies that $D_{\mu }^{-1}\left( \partial \widehat{%
\Delta }\left( \overline{\delta }_{n}\right) /\partial \delta ^{\prime
}\right) D_{\mu }^{-1}$ is nonsingular and, thus, invertible w.p.a.1. It
follows that, for all $n$ sufficiently large, we can solve for $D_{\mu
}\left( \widehat{\delta }_{L,n}-\delta _{0}\right) $ in (\ref{mean value
approx}) above to get%
\begin{eqnarray}
D_{\mu }\left( \widehat{\delta }_{L,n}-\delta _{0}\right) &=&-\left[ D_{\mu
}^{-1}\left( \frac{\partial \widehat{\Delta }\left( \overline{\delta }%
_{n}\right) }{\partial \delta ^{\prime }}\right) D_{\mu }^{-1}\right]
^{-1}D_{\mu }^{-1}\widehat{\Delta }\left( \delta _{0}\right)  \notag \\
&=&H_{n}^{-1}\left( \frac{\Upsilon ^{\prime }Z_{2}^{\prime }M^{\left(
Z_{1},Q\right) }\varepsilon }{\sqrt{n}}+D_{\mu }^{-1}\underline{U}^{\prime
}A\varepsilon \right) \left[ 1+o_{p}\left( 1\right) \right] \text{,}
\label{scaled FELIM}
\end{eqnarray}%
where the last equality follows by applying Lemma S2-9. By part (d) of Lemma
S2-3, $\Sigma _{n}$ is positive definite $a.s.n.$, so that $\Sigma _{n}^{-1}$
is well-defined for all $n$ sufficiently large, and both $\Sigma _{n}^{1/2}$
and $\Sigma _{n}^{-1/2}$ can be taken to be symmetric matrices. Since $H_{n}$
is also symmetric, it further follows that $\Lambda _{I,n}=H_{n}^{-1}\Sigma
_{n}H_{n}^{-1}$ is symmetric and positive definite $a.s.n.$, and both $%
\Lambda _{I,n}^{-1}=\left( H_{n}^{-1}\Sigma _{n}H_{n}^{-1}\right) ^{-1}$ and 
$\Lambda _{I,n}^{-1/2}=\left( H_{n}^{-1}\Sigma _{n}H_{n}^{-1}\right) ^{-1/2}$
are well-defined for all $n$ sufficiently large. Multiplying both sides of
the equation above by $\Lambda _{I,n}^{-1/2}$, we then get $\Lambda
_{I,n}^{-1/2}D_{\mu }\left( \widehat{\delta }_{L,n}-\delta _{0}\right)
=\left( H_{n}^{-1}\Sigma _{n}H_{n}^{-1}\right) ^{-1/2}H_{n}^{-1}\mathcal{Y}%
_{n}\left[ 1+o_{p}\left( 1\right) \right] $, where $\mathcal{Y}_{n}=\Upsilon
^{\prime }Z_{2}^{\prime }M^{\left( Z_{1},Q\right) }\varepsilon /\sqrt{n}%
+D_{\mu }^{-1}\underline{U}^{\prime }A\varepsilon $. Let $R_{W,n}=\left(
H_{n}^{-1}\Sigma _{n}H_{n}^{-1}\right) ^{-1/2}H_{n}^{-1}\Sigma _{n}^{1/2}$,
and note that $R_{W,n}R_{W,n}^{\prime }=I_{d}$ for all $n$ sufficiently
large. It, thus, follows from the result given in (\ref{asy normality V})
above and the continuous mapping theorem that $\Lambda _{I,n}^{-1/2}D_{\mu
}\left( \widehat{\delta }_{L,n}-\delta _{0}\right) \overset{d}{\rightarrow }%
N\left( 0,I_{d}\right) $, as $n\rightarrow \infty $, as required.

Turning our attention now to $\widehat{\delta }_{F,n}$, note that we can
write this estimator, appropriately standardized, as%
\begin{equation}
D_{\mu }\left( \widehat{\delta }_{F,n}-\delta _{0}\right) =\left( D_{\mu
}^{-1}X^{\prime }\left[ A-\widehat{\ell }_{F,n}M^{\left( Z_{1},Q\right) }%
\right] XD_{\mu }^{-1}\right) ^{-1}D_{\mu }^{-1}X^{\prime }\left[ A-\widehat{%
\ell }_{F,n}M^{\left( Z_{1},Q\right) }\right] \left( y-X\delta _{0}\right)
\label{FEFUL standardized}
\end{equation}%
so that, multiplying by $\Lambda _{I,n}^{-1/2}=\left( H_{n}^{-1}\Sigma
_{n}H_{n}^{-1}\right) ^{-1/2}$ and applying Lemmas S2-12 and S2-13, we
obtain $\Lambda _{I,n}^{-1/2}D_{\mu }\left( \widehat{\delta }_{F,n}-\delta
_{0}\right) =\left( H_{n}^{-1}\Sigma _{n}H_{n}^{-1}\right) ^{-1/2}H_{n}^{-1}%
\mathcal{Y}_{n}\left[ 1+o_{p}\left( 1\right) \right] $. It follows from the
result given in (\ref{asy normality V}) above and the continuous mapping
theorem that $\Lambda _{I,n}^{-1/2}D_{\mu }\left( \widehat{\delta }%
_{F,n}-\delta _{0}\right) \overset{d}{\rightarrow }N\left( 0,I_{d}\right) $,
as $n\rightarrow \infty $, as required. $\square $

\medskip

\noindent \textbf{Proof of Theorem 3: }To proceed, note that, in this case, $%
\left( \mu _{n}^{\min }\right) /\sqrt{K_{2,n}}=o\left( 1\right) $ but $\sqrt{%
K_{2,n}}/\left( \mu _{n}^{\min }\right) ^{2}$

\noindent $\rightarrow 0$, so that, by the result given in Lemma S2-9, we
have%
\begin{equation}
\frac{\mu _{n}^{\min }}{\sqrt{K_{2,n}}}D_{\mu }^{-1}\widehat{\Delta }\left(
\delta _{0}\right) =\frac{\mu _{n}^{\min }}{\sqrt{K_{2,n}}}D_{\mu }^{-1}%
\underline{U}^{\prime }A\varepsilon +o_{p}\left( 1\right)
\label{Thm3 initial approx}
\end{equation}

\noindent where $\underline{U}=U-\varepsilon \rho ^{\prime }$. Again, let $%
H_{n}=\Upsilon ^{\prime }Z_{2}^{\prime }M^{\left( Z_{1},Q\right)
}Z_{2}\Upsilon /n$, and $\Sigma _{2,n}=VC\left( D_{\mu }^{-1}\underline{U}%
^{\prime }A\varepsilon |\mathcal{F}_{n}^{Z}\right) $

\noindent $=D_{\mu }^{-1}VC\left( \underline{U}^{\prime }A\varepsilon |%
\mathcal{F}_{n}^{Z}\right) D_{\mu }^{-1}$. Now, by assumption, $\widetilde{L}%
_{n}$ can be any sequence of bounded $\left( l\times d\right) $ non-random
matrices such that $\lambda _{\min }\left( \left( \mu _{n}^{\min }\right)
^{2}\widetilde{L}_{n}H_{n}^{-1}\Sigma _{2,n}H_{n}^{-1}\widetilde{L}%
_{n}^{\prime }/K_{2,n}\right) \geq \underline{C}$ \ $a.s.n.$ for some
constant $\underline{C}$ $>0$. It follows that $\left( \mu _{n}^{\min
}\right) ^{2}\widetilde{L}_{n}H_{n}^{-1}\Sigma _{2,n}H_{n}^{-1}\widetilde{L}%
_{n}^{\prime }/K_{2,n}$ is positive definite $a.s.n.$, so that, with
probability one, $\left( \left( \mu _{n}^{\min }\right) ^{2}\widetilde{L}%
_{n}H_{n}^{-1}\Sigma _{2,n}H_{n}^{-1}\widetilde{L}_{n}^{\prime
}/K_{2,n}\right) ^{-1/2}$ is well-defined for all $n$ sufficiently large.
Hence, we can let

\noindent $\widetilde{\mathcal{N}}_{n}=\left( \left( \mu _{n}^{\min }\right)
^{2}\widetilde{L}_{n}H_{n}^{-1}\Sigma _{2,n}H_{n}^{-1}\widetilde{L}%
_{n}^{\prime }/K_{2,n}\right) ^{-1/2}\widetilde{L}_{n}H_{n}^{-1}\left( \mu
_{n}^{\min }/\sqrt{K_{2,n}}\right) D_{\mu }^{-1}\underline{U}^{\prime
}A\varepsilon $ and construct the linear combination $\mathcal{J}%
_{n}=a^{\prime }\widetilde{\mathcal{N}}_{n}$ for any $a\in \mathbb{R}^{d}$
such that $\left\Vert a\right\Vert _{2}=1$. Next, define $\underline{u}%
_{\left( i,t\right) ,n}=a^{\prime }\left( \left( \mu _{n}^{\min }\right) ^{2}%
\widetilde{L}_{n}H_{n}^{-1}\Sigma _{2,n}H_{n}^{-1}\widetilde{L}_{n}^{\prime
}/K_{2,n}\right) ^{-1/2}\widetilde{L}_{n}H_{n}^{-1}D_{\mu }^{-1}\underline{U}%
_{\left( i,t\right) }$, with $\underline{u}_{\left( j,s\right) ,n}$
similarly defined, and we can write $\mathcal{J}_{n}=\left( \mu _{n}^{\min }/%
\sqrt{K_{2,n}}\right) \dsum\nolimits_{\left( i,t\right)
=2}^{m_{n}}\dsum\nolimits_{\left( j,s\right) =1}^{\left( i,t\right)
-1}A_{\left( i,t\right) ,\left( j,s\right) }\left[ \underline{u}_{\left(
i,t\right) ,n}\varepsilon _{\left( j,s\right) }+\underline{u}_{\left(
j,s\right) ,n}\varepsilon _{\left( i,t\right) }\right] $

\noindent $=\dsum\nolimits_{\left( i,t\right) =2}^{m_{n}}\mathcal{J}_{\left(
i,t\right) ,n}$, where $\mathcal{J}_{\left( i,t\right) ,n}=\left( \mu
_{n}^{\min }/\sqrt{K_{2,n}}\right) \dsum\nolimits_{\left( j,s\right)
=1}^{\left( i,t\right) -1}A_{\left( i,t\right) ,\left( j,s\right) }\left[ 
\underline{u}_{\left( i,t\right) ,n}\varepsilon _{\left( j,s\right) }+%
\underline{u}_{\left( j,s\right) ,n}\varepsilon _{\left( i,t\right) }\right] 
$. Again, define the $\sigma $-fields $\mathcal{F}_{\left( i,t\right)
,n}=\sigma \left( \left\{ \varepsilon _{\left( k,\upsilon \right)
},U_{\left( k,\upsilon \right) }\right\} _{\left( k,\upsilon \right)
=1}^{\left( i,t\right) },Z\right) $ \ for $\left( i,t\right) =1,2,...,m_{n}$%
, noting that by construction $\mathcal{F}_{\left( i,t\right) -1,n}\subseteq 
\mathcal{F}_{\left( i,t\right) ,n}$ for $\left( i,t\right) =2,...,m_{n}$ and 
$\mathcal{J}_{\left( i,t\right) ,n}$ is $\mathcal{F}_{\left( i,t\right) ,n}-$%
measurable. In addition, note that, using Assumption 1, it is easily seen
that $E\left[ \underline{u}_{\left( i,t\right) ,n}|\mathcal{F}_{\left(
i,t\right) -1,n}\right] =0$ and $E\left[ \varepsilon _{\left( i,t\right) }|%
\mathcal{F}_{\left( i,t\right) -1,n}\right] =0$, from which it follows that $%
E\left[ \mathcal{J}_{\left( i,t\right) ,n}|\mathcal{F}_{\left( i,t\right)
-1,n}\right] =$

\noindent $\left( \mu _{n}^{\min }/\sqrt{K_{2,n}}\right)
\dsum\nolimits_{\left( j,s\right) =1}^{\left( i,t\right) -1}A_{\left(
i,t\right) ,\left( j,s\right) }\left\{ \varepsilon _{\left( j,s\right) }E%
\left[ \underline{u}_{\left( i,t\right) ,n}|\mathcal{F}_{\left( i,t\right)
-1,n}\right] +\underline{u}_{\left( j,s\right) ,n}E\left[ \varepsilon
_{\left( i,t\right) }|\mathcal{F}_{\left( i,t\right) -1,n}\right] \right\}
=0 $. Moreover, applying the CS inequality and making use of the fact that 
\begin{equation}
E\left[ \underline{u}_{\left( i,t\right) ,n}^{2}|\mathcal{F}_{n}^{Z}\right]
\leq \frac{\max_{1\leq \left( i,t\right) \leq m_{n}}E\left[ \left\Vert 
\underline{U}_{\left( i,t\right) }\right\Vert _{2}^{2}|\mathcal{F}_{n}^{Z}%
\right] \left\Vert \widetilde{L}_{n}\right\Vert _{F}^{2}}{\lambda _{\min
}\left( \left( \mu _{n}^{\min }\right) ^{2}\widetilde{L}_{n}H_{n}^{-1}\Sigma
_{2,n}H_{n}^{-1}\widetilde{L}_{n}^{\prime }/K_{2,n}\right) \left[ \lambda
_{\min }\left( H_{n}\right) \right] ^{2}}\left( \frac{1}{\mu _{n}^{\min }}%
\right) ^{2}=O_{a.s.}\left( \frac{1}{\left( \mu _{n}^{\min }\right) ^{2}}%
\right)  \label{2nd moment bd on u bar}
\end{equation}%
and that $E\left[ \varepsilon _{\left( i,t\right) }^{2}|\mathcal{F}_{n}^{Z}%
\right] \leq \overline{C}$ $a.s.$ by Assumption 2(i), we see that%
\begin{eqnarray}
&&Var\left( \mathcal{J}_{\left( i,t\right) ,n}|\mathcal{F}_{n}^{Z}\right) 
\notag \\
&\leq &\frac{\left( \mu _{n}^{\min }\right) ^{2}}{K_{2,n}}%
\dsum\limits_{\left( j,s\right) =1}^{\left( i,t\right) -1}A_{\left(
i,t\right) ,\left( j,s\right) }^{2}\left( E\left[ \underline{u}_{\left(
i,t\right) ,n}^{2}|\mathcal{F}_{n}^{Z}\right] E\left[ \varepsilon _{\left(
j,s\right) }^{2}|\mathcal{F}_{n}^{Z}\right] +E\left[ \varepsilon _{\left(
i,t\right) }^{2}|\mathcal{F}_{n}^{Z}\right] E\left[ \underline{u}_{\left(
j,s\right) ,n}^{2}|\mathcal{F}_{n}^{Z}\right] \right.  \notag \\
&&\text{ \ \ \ \ \ \ \ \ \ \ \ \ \ \ \ }\left. +2\sqrt{E\left[ \underline{u}%
_{\left( i,t\right) ,n}^{2}|\mathcal{F}_{n}^{Z}\right] E\left[ \varepsilon
_{\left( i,t\right) }^{2}|\mathcal{F}_{n}^{Z}\right] }\sqrt{E\left[ 
\underline{u}_{\left( j,s\right) ,n}^{2}|\mathcal{F}_{n}^{Z}\right] E\left[
\varepsilon _{\left( j,s\right) }^{2}|\mathcal{F}_{n}^{Z}\right] }\right) 
\notag \\
&\leq &\frac{4\overline{C}^{2}}{\left( \mu _{n}^{\min }\right) ^{2}}\frac{%
\left( \mu _{n}^{\min }\right) ^{2}}{K_{2,n}}\dsum\limits_{\left( j,s\right)
=1}^{\left( i,t\right) -1}A_{\left( i,t\right) ,\left( j,s\right) }^{2}=%
\frac{4\overline{C}^{2}}{K_{2,n}}\dsum\limits_{\left( j,s\right) =1}^{\left(
i,t\right) -1}A_{\left( i,t\right) ,\left( j,s\right) }^{2}\text{ }a.s.n.
\label{cond var of J}
\end{eqnarray}%
Hence, applying the law of iterated expectations, part (d) of Lemma S2-1,
and Theorem 16.1 of Billingsley (1995), we further deduce that $Var\left( 
\mathcal{J}_{\left( i,t\right) ,n}\right) =E_{Z}\left[ E\left( \mathcal{J}%
_{\left( i,t\right) ,n}^{2}|\mathcal{F}_{n}^{Z}\right) \right] $

\noindent $\leq \left( 4\overline{C}^{2}/K_{2,n}\right)
\dsum\nolimits_{\left( j,s\right) =1}^{\left( i,t\right) -1}E_{Z}\left[
A_{\left( i,t\right) ,\left( j,s\right) }^{2}\right] \leq C$ for some
positive constant $C$ for all $n$ sufficiently large. These results show
that $\left\{ \mathcal{J}_{\left( i,t\right) ,n},\mathcal{F}_{\left(
i,t\right) ,n},1\leq \left( i,t\right) \leq m_{n}\text{, }n\geq 1\right\} $
forms a square-integrable martingale difference array.

Next, we verify condition (\ref{MDA CLT cond 1b}) of the central limit
theorem for martingale difference arrays given in Lemma S2-15 below. By Lo%
\`{e}ve's $c_{r}$ inequality we have%
\begin{eqnarray}
&&\dsum\limits_{\left( i,t\right) =2}^{m_{n}}E\left[ \left( \frac{\mu
_{n}^{\min }}{\sqrt{K_{2,n}}}\dsum\limits_{\left( j,s\right) =1}^{\left(
i,t\right) -1}A_{\left( i,t\right) ,\left( j,s\right) }\left[ \underline{u}%
_{\left( i,t\right) ,n}\varepsilon _{\left( j,s\right) }+\underline{u}%
_{\left( j,s\right) ,n}\varepsilon _{\left( i,t\right) }\right] \right) ^{4}|%
\mathcal{F}_{n}^{Z}\right]  \notag \\
&\leq &8\dsum\limits_{\left( i,t\right) =2}^{m_{n}}\frac{\left( \mu
_{n}^{\min }\right) ^{4}}{K_{2,n}^{2}}E\left[ \left( \dsum\limits_{\left(
j,s\right) =1}^{\left( i,t\right) -1}A_{\left( i,t\right) ,\left( j,s\right)
}\underline{u}_{\left( i,t\right) ,n}\varepsilon _{\left( j,s\right)
}\right) ^{4}|\mathcal{F}_{n}^{Z}\right]  \notag \\
&&+8\dsum\limits_{\left( i,t\right) =2}^{m_{n}}\frac{\left( \mu _{n}^{\min
}\right) ^{4}}{K_{2,n}^{2}}E\left[ \left( \dsum\limits_{\left( j,s\right)
=1}^{\left( i,t\right) -1}A_{\left( i,t\right) ,\left( j,s\right) }%
\underline{u}_{\left( j,s\right) ,n}\varepsilon _{\left( i,t\right) }\right)
^{4}|\mathcal{F}_{n}^{Z}\right]  \notag \\
&=&\mathcal{E}_{1}+\mathcal{E}_{2},\text{ }\left( say\right) \text{.}
\label{4th moment bilinear form}
\end{eqnarray}%
Focusing first on $\mathcal{E}_{1}$, it is easy to see that there exists
some positive constant $C$ such that%
\begin{eqnarray*}
&&\mathcal{E}_{1} \\
&=&\frac{8\left( \mu _{n}^{\min }\right) ^{4}}{K_{2,n}^{2}}E\left[
\dsum\limits_{\left( i,t\right) =2}^{m_{n}}\left( \dsum\limits_{\left(
j,s\right) =1}^{\left( i,t\right) -1}A_{\left( i,t\right) ,\left( j,s\right)
}\underline{u}_{\left( i,t\right) ,n}\varepsilon _{\left( j,s\right)
}\right) ^{4}|\mathcal{F}_{n}^{Z}\right] \\
&\leq &\frac{8\left( \mu _{n}^{\min }\right) ^{4}}{K_{2,n}^{2}}%
\dsum\limits_{\left( i,t\right) =1}^{m_{n}}\dsum\limits_{\substack{ \left(
j,s\right) =1  \\ \left( j,s\right) \neq \left( i,t\right) }}%
^{m_{n}}A_{\left( i,t\right) ,\left( j,s\right) }^{4}E\left[ \underline{u}%
_{\left( i,t\right) ,n}^{4}|\mathcal{F}_{n}^{Z}\right] E\left[ \varepsilon
_{\left( j,s\right) }^{4}|\mathcal{F}_{n}^{Z}\right] \\
&&+\frac{24\left( \mu _{n}^{\min }\right) ^{4}}{K_{2,n}^{2}}%
\dsum\limits_{\left( i,t\right) =1}^{m_{n}}\dsum\limits_{\substack{ \left(
j,s\right) ,\left( k,\upsilon \right) =1  \\ \left( j,s\right) \neq \left(
i,t\right) ,\left( k,\upsilon \right) \neq \left( i,t\right)  \\ \left(
j,s\right) \neq \left( k,\upsilon \right) }}^{m_{n}}A_{\left( i,t\right)
,\left( j,s\right) }^{2}A_{\left( i,t\right) ,\left( k,\upsilon \right)
}^{2}E\left[ \underline{u}_{\left( i,t\right) ,n}^{4}|\mathcal{F}_{n}^{Z}%
\right] E\left[ \varepsilon _{\left( j,s\right) }^{2}|\mathcal{F}_{n}^{Z}%
\right] E\left[ \varepsilon _{\left( k,\upsilon \right) }^{2}|\mathcal{F}%
_{n}^{Z}\right] \\
&\leq &\frac{C}{K_{2,n}}\left[ \frac{1}{K_{2,n}}\dsum\limits_{\substack{ %
\left( i,t\right) ,\left( j,s\right) =1  \\ \left( j,s\right) \neq \left(
i,t\right) }}^{m_{n}}A_{\left( i,t\right) ,\left( j,s\right) }^{4}+\frac{1}{%
K_{2,n}}\dsum\limits_{\left( i,t\right) =1}^{m_{n}}\dsum\limits_{\substack{ %
\left( j,s\right) ,\left( k,\upsilon \right) =1  \\ \left( j,s\right) \neq
\left( i,t\right) ,\left( k,\upsilon \right) \neq \left( i,t\right) }}%
^{m_{n}}A_{\left( i,t\right) ,\left( j,s\right) }^{2}A_{\left( i,t\right)
,\left( k,\upsilon \right) }^{2}\right]
\end{eqnarray*}%
where the second inequality above follows from Assumption 2(i) and from an
upper bound on the conditional fourth moment of

\noindent $\underline{u}_{\left( i,t\right) ,n}=a^{\prime }\left( \left( \mu
_{n}^{\min }\right) ^{2}\widetilde{L}_{n}H_{n}^{-1}\Sigma _{2,n}H_{n}^{-1}%
\widetilde{L}_{n}^{\prime }/K_{2,n}\right) ^{-1/2}\widetilde{L}%
_{n}H_{n}^{-1}D_{\mu }^{-1}\underline{U}_{\left( i,t\right) }$ given by%
\begin{eqnarray}
E\left[ \underline{u}_{\left( i,t\right) ,n}^{4}|\mathcal{F}_{n}^{Z}\right]
&\leq &\frac{1}{\left( \mu _{n}^{\min }\right) ^{4}}\left( \max_{1\leq
\left( i,t\right) \leq m_{n}}E\left[ \left\Vert \underline{U}_{\left(
i,t\right) }\right\Vert _{2}^{4}|\mathcal{F}_{n}^{Z}\right] \right) \frac{1}{%
\left[ \lambda _{\min }\left( H_{n}\right) \right] ^{4}}  \notag \\
&&\times \left\Vert \widetilde{L}_{n}\right\Vert _{F}^{4}\left( \frac{1}{%
\lambda _{\min }\left( \left( \mu _{n}^{\min }\right) ^{2}\widetilde{L}%
_{n}H_{n}^{-1}\Sigma _{2,n}H_{n}^{-1}\widetilde{L}_{n}^{\prime
}/K_{2,n}\right) }\right) ^{2}  \notag \\
&\leq &\frac{C^{\ast }}{\left( \mu _{n}^{\min }\right) ^{4}}\text{ \ }a.s.n.%
\text{, for some constant }C^{\ast }>0\text{.}
\label{4th moment bd on u tilde}
\end{eqnarray}%
Note also that, in deriving the upper bound given in (\ref{4th moment bd on
u tilde}), we have applied Assumption 3(iii), Lemma S2-6, the boundedness of 
$\left\Vert \widetilde{L}_{n}\right\Vert _{F}^{2}$ , and the assumption that

\noindent $\lambda _{\min }\left( \left( \mu _{n}^{\min }\right) ^{2}%
\widetilde{L}_{n}H_{n}^{-1}\Sigma _{2,n}H_{n}^{-1}\widetilde{L}_{n}^{\prime
}/K_{2,n}\right) \geq \underline{C}>0$ \ $a.s.n.$ Moreover, by parts (b) and
(c) of Lemma S2-1, we have that $K_{2,n}^{-1}\dsum\nolimits_{\left(
i,t\right) ,\left( j,s\right) =1,\left( i,t\right) \neq \left( j,s\right)
}^{m_{n}}A_{\left( i,t\right) ,\left( j,s\right) }^{4}=O_{a.s.}\left(
K_{2,n}^{2}/n^{2}\right) $ and

\noindent $K_{2,n}^{-1}\dsum\nolimits_{\left( i,t\right)
=1}^{m_{n}}\dsum\nolimits_{\left( j,s\right) ,\left( k,\upsilon \right)
=1,\left( j,s\right) \neq \left( i,t\right) ,\left( k,\upsilon \right) \neq
\left( i,t\right) }^{m_{n}}A_{\left( i,t\right) ,\left( j,s\right)
}^{2}A_{\left( i,t\right) ,\left( k,\upsilon \right) }^{2}=O_{a.s.}\left(
K_{2,n}/n\right) $. from which it follows that $n\mathcal{E}%
_{1}=O_{a.s.}\left( 1\right) $ in light of Assumption 5(ii). Hence, by
applying the law of iterated expectations and Theorem 16.1 of Billingsley
(1995), we obtain, for all $n$ sufficiently large,%
\begin{eqnarray*}
&&nE_{Z}\left[ \mathcal{E}_{1}\right] \\
&=&\frac{8n\left( \mu _{n}^{\min }\right) ^{4}}{K_{2,n}^{2}}E_{Z}\left\{ E%
\left[ \dsum\limits_{\left( i,t\right) =2}^{m_{n}}\left(
\dsum\limits_{\left( j,s\right) =1}^{\left( i,t\right) -1}A_{\left(
i,t\right) ,\left( j,s\right) }\underline{u}_{\left( i,t\right)
,n}\varepsilon _{\left( j,s\right) }\right) ^{4}|\mathcal{F}_{n}^{Z}\right]
\right\} \\
&\leq &\frac{Cn}{K_{2,n}}\left\{ E_{Z}\left[ \frac{1}{K_{2,n}}\dsum\limits 
_{\substack{ \left( i,t\right) ,\left( j,s\right) =1  \\ \left( i,t\right)
\neq \left( j,s\right) }}^{m_{n}}A_{\left( i,t\right) ,\left( j,s\right)
}^{4}+\frac{1}{K_{2,n}}\dsum\limits_{\left( i,t\right)
=1}^{m_{n}}\dsum\limits_{\substack{ \left( j,s\right) ,\left( k,\upsilon
\right) =1  \\ \left( j,s\right) \neq \left( i,t\right) ,\left( k,\upsilon
\right) \neq \left( i,t\right) }}^{m_{n}}A_{\left( i,t\right) ,\left(
j,s\right) }^{2}A_{\left( i,t\right) ,\left( k,\upsilon \right) }^{2}\right]
\right\} \\
&=&O\left( 1\right) \text{,}
\end{eqnarray*}%
which shows that $E_{Z}\left[ \mathcal{E}_{1}\right] =O\left( 1/n\right)
=o\left( 1\right) $. In a similar way, we can also show that

\noindent $E_{Z}\left[ \mathcal{E}_{2}\right] =8\left[ \left( \mu _{n}^{\min
}\right) ^{4}/K_{2,n}^{2}\right] E\left[ \dsum\nolimits_{\left( i,t\right)
=2}^{m_{n}}\left( \dsum\nolimits_{\left( j,s\right) =1}^{\left( i,t\right)
-1}A_{\left( j,s\right) ,\left( i,t\right) }\underline{u}_{\left( i,t\right)
,n}\varepsilon _{\left( i,t\right) }\right) ^{4}\right] =o\left( 1\right) $.
Condition (\ref{MDA CLT cond 1b}) of Lemma S2-15 then follows from these
calculations since%
\begin{equation*}
\dsum\limits_{\left( i,t\right) =2}^{m_{n}}E\left[ \left( \frac{\mu
_{n}^{\min }}{\sqrt{K_{2,n}}}\dsum\limits_{\left( j,s\right) =1}^{\left(
i,t\right) -1}A_{\left( i,t\right) ,\left( j,s\right) }\left[ \underline{u}%
_{\left( i,t\right) ,n}\varepsilon _{\left( j,s\right) }+\underline{u}%
_{\left( j,s\right) ,n}\varepsilon _{\left( i,t\right) }\right] \right) ^{4}%
\right] \leq E_{Z}\left[ \mathcal{E}_{1}\right] +E_{Z}\left[ \mathcal{E}_{2}%
\right] =o\left( 1\right)
\end{equation*}

Next, we verify condition (\ref{MDA CLT cond 2}) of Lemma S2-15. Note first
that, by construction, $Var\left( \mathcal{J}_{n}|\mathcal{F}_{n}^{Z}\right) 
$

\noindent $=a^{\prime }\left( \widetilde{L}_{n}\Lambda _{II,n}\widetilde{L}%
_{n}^{\prime }\right) ^{-1/2}\widetilde{L}_{n}\Lambda _{II,n}\widetilde{L}%
_{n}^{\prime }\left( \widetilde{L}_{n}\Lambda _{II,n}\widetilde{L}%
_{n}^{\prime }\right) ^{-1/2}a=1$, with $\Lambda _{II,n}=\left[ \left( \mu
_{n}^{\min }\right) ^{2}/K_{2,n}\right] H_{n}^{-1}\Sigma _{2,n}H_{n}^{-1}$.
This, in turn, implies that $Var\left( \mathcal{J}_{n}\right) =E_{Z}\left[
E\left( \mathcal{J}_{n}^{2}|\mathcal{F}_{n}^{Z}\right) \right] =E_{Z}\left[
Var\left( \mathcal{J}_{n}|\mathcal{F}_{n}^{Z}\right) \right] =1$. On the
other hand, by direct calculation, we obtain%
\begin{eqnarray}
1 &=&Var\left( \mathcal{J}_{n}|\mathcal{F}_{n}^{Z}\right)  \notag \\
&=&\frac{\left( \mu _{n}^{\min }\right) ^{2}}{K_{2,n}}\dsum\limits_{\left(
i,t\right) =2}^{m_{n}}\dsum\limits_{\left( j,s\right) =1}^{\left( i,t\right)
-1}A_{\left( i,t\right) ,\left( j,s\right) }^{2}E\left[ \varepsilon _{\left(
j,s\right) }^{2}|\mathcal{F}_{n}^{Z}\right] E\left[ \underline{u}_{\left(
i,t\right) ,n}^{2}|\mathcal{F}_{n}^{Z}\right]  \notag \\
&&+\frac{\left( \mu _{n}^{\min }\right) ^{2}}{K_{2,n}}\dsum\limits_{\left(
i,t\right) =2}^{m_{n}}\dsum\limits_{\left( j,s\right) =1}^{\left( i,t\right)
-1}A_{\left( i,t\right) ,\left( j,s\right) }^{2}E\left[ \underline{u}%
_{\left( j,s\right) ,n}^{2}|\mathcal{F}_{n}^{Z}\right] E\left[ \varepsilon
_{\left( i,t\right) }^{2}|\mathcal{F}_{n}^{Z}\right]  \notag
\end{eqnarray}%
\begin{equation}
+2\frac{\left( \mu _{n}^{\min }\right) ^{2}}{K_{2,n}}\dsum\limits_{\left(
i,t\right) =2}^{m_{n}}\dsum\limits_{\left( j,s\right) =1}^{\left( i,t\right)
-1}A_{\left( i,t\right) ,\left( j,s\right) }^{2}E\left[ \underline{u}%
_{\left( j,s\right) ,n}\varepsilon _{\left( j,s\right) }|\mathcal{F}_{n}^{Z}%
\right] E\left[ \underline{u}_{\left( i,t\right) ,n}\varepsilon _{\left(
i,t\right) }|\mathcal{F}_{n}^{Z}\right]  \label{normalized cond var}
\end{equation}%
Making use of expression (\ref{normalized cond var}), we obtain, after some
further calculations,%
\begin{eqnarray}
&&\dsum\limits_{\left( i,t\right) =2}^{m_{n}}E\left[ \mathcal{J}_{\left(
i,t\right) ,n}^{2}|\mathcal{F}_{\left( i,t\right) -1,n}\right] -1  \notag \\
&=&\frac{\left( \mu _{n}^{\min }\right) ^{2}}{K_{2,n}}\dsum\limits_{\left(
i,t\right) =2}^{m_{n}}\dsum\limits_{\left( j,s\right) =1}^{\left( i,t\right)
-1}A_{\left( i,t\right) ,\left( j,s\right) }^{2}\left( \varepsilon _{\left(
j,s\right) }^{2}-E\left[ \varepsilon _{\left( j,s\right) }^{2}|\mathcal{F}%
_{n}^{Z}\right] \right) E\left[ \underline{u}_{\left( i,t\right) ,n}^{2}|%
\mathcal{F}_{n}^{Z}\right]  \notag \\
&&+\frac{\left( \mu _{n}^{\min }\right) ^{2}}{K_{2,n}}\dsum\limits_{\left(
i,t\right) =2}^{m_{n}}\dsum\limits_{\left( j,s\right) =1}^{\left( i,t\right)
-1}A_{\left( i,t\right) ,\left( j,s\right) }^{2}\left( \underline{u}_{\left(
j,s\right) ,n}^{2}-E\left[ \underline{u}_{\left( j,s\right) ,n}^{2}|\mathcal{%
F}_{n}^{Z}\right] \right) E\left[ \varepsilon _{\left( i,t\right) }^{2}|%
\mathcal{F}_{n}^{Z}\right]  \notag \\
&&+\frac{2\left( \mu _{n}^{\min }\right) ^{2}}{K_{2,n}}\dsum\limits_{\left(
i,t\right) =2}^{m_{n}}\dsum\limits_{\left( j,s\right) =1}^{\left( i,t\right)
-1}A_{\left( i,t\right) ,\left( j,s\right) }^{2}\left( \underline{u}_{\left(
j,s\right) ,n}\varepsilon _{\left( j,s\right) }-E\left[ \underline{u}%
_{\left( j,s\right) ,n}\varepsilon _{\left( j,s\right) }|\mathcal{F}_{n}^{Z}%
\right] \right) E\left[ \underline{u}_{\left( i,t\right) ,n}\varepsilon
_{\left( i,t\right) }|\mathcal{F}_{n}^{Z}\right]  \notag \\
&&+\frac{2\left( \mu _{n}^{\min }\right) ^{2}}{K_{2,n}}\dsum\limits_{\left(
i,t\right) =3}^{m_{n}}\dsum\limits_{\left( j,s\right) =2}^{\left( i,t\right)
-1}\dsum\limits_{\left( k,\upsilon \right) =1}^{\left( j,s\right)
-1}A_{\left( i,t\right) ,\left( j,s\right) }A_{\left( i,t\right) ,\left(
k,\upsilon \right) }E\left[ \underline{u}_{\left( i,t\right) ,n}\varepsilon
_{\left( i,t\right) }|\mathcal{F}_{n}^{Z}\right] \left\{ \underline{u}%
_{\left( j,s\right) ,n}\varepsilon _{\left( k,\upsilon \right) }+\varepsilon
_{\left( j,s\right) }\underline{u}_{\left( k,\upsilon \right) ,n}\right\} 
\notag \\
&&+\frac{2\left( \mu _{n}^{\min }\right) ^{2}}{K_{2,n}}\dsum\limits_{\left(
i,t\right) =3}^{m_{n}}\dsum\limits_{\left( j,s\right) =2}^{\left( i,t\right)
-1}\dsum\limits_{\left( k,\upsilon \right) =1}^{\left( j,s\right)
-1}A_{\left( i,t\right) ,\left( j,s\right) }A_{\left( i,t\right) ,\left(
k,\upsilon \right) }\varepsilon _{\left( j,s\right) }\varepsilon _{\left(
k,\upsilon \right) }E\left[ \underline{u}_{\left( i,t\right) ,n}^{2}|%
\mathcal{F}_{n}^{Z}\right]  \notag \\
&&+\frac{2\left( \mu _{n}^{\min }\right) ^{2}}{K_{2,n}}\dsum\limits_{\left(
i,t\right) =3}^{m_{n}}\dsum\limits_{\left( j,s\right) =2}^{\left( i,t\right)
-1}\dsum\limits_{\left( k,\upsilon \right) =1}^{\left( j,s\right)
-1}A_{\left( i,t\right) ,\left( j,s\right) }A_{\left( i,t\right) ,\left(
k,\upsilon \right) }\underline{u}_{\left( j,s\right) ,n}\underline{u}%
_{\left( k,\upsilon \right) ,n}E\left[ \varepsilon _{\left( i,t\right) }^{2}|%
\mathcal{F}_{n}^{Z}\right]  \notag \\
&=&\mathcal{TT}_{1}+\mathcal{TT}_{2}+\mathcal{TT}_{3}+\mathcal{TT}_{4}+%
\mathcal{TT}_{5}+\mathcal{TT}_{6}  \label{Thm 3 cond 2}
\end{eqnarray}

To analyze the terms $\mathcal{TT}_{k}$ $\left( k=1,..,6\right) $, note
first that, by applying parts (b) and (a) of Lemma S2-16, we obtain $%
\mathcal{TT}_{1}\overset{p}{\rightarrow }0$ and $\mathcal{TT}_{2}\overset{p}{%
\rightarrow }0$, respectively. Consider now the term 
\begin{equation*}
\mathcal{TT}_{3}=\frac{2\left( \mu _{n}^{\min }\right) ^{2}}{K_{2,n}}%
\dsum\limits_{\left( i,t\right) =2}^{m_{n}}\dsum\limits_{\left( j,s\right)
=1}^{\left( i,t\right) -1}A_{\left( i,t\right) ,\left( j,s\right)
}^{2}\left( \underline{u}_{\left( j,s\right) ,n}\varepsilon _{\left(
j,s\right) }-E\left[ \underline{u}_{\left( j,s\right) ,n}\varepsilon
_{\left( j,s\right) }|\mathcal{F}_{n}^{Z}\right] \right) E\left[ \underline{u%
}_{\left( i,t\right) ,n}\varepsilon _{\left( i,t\right) }|\mathcal{F}_{n}^{Z}%
\right]
\end{equation*}%
In this case, we apply part (a) of Lemma S2-8 with $u_{\left( j,s\right)
,n}=\left( \mu _{n}^{\min }\right) \underline{u}_{\left( j,s\right) ,n}$,

\noindent $\overline{\psi }_{\left( j,s\right) }=E\left[ \left( \mu
_{n}^{\min }\right) \underline{u}_{\left( j,s\right) ,n}\varepsilon _{\left(
j,s\right) }|\mathcal{F}_{n}^{Z}\right] $, and $\phi _{\left( i,t\right) }=E%
\left[ \left( \mu _{n}^{\min }\right) \underline{u}_{\left( i,t\right)
,n}\varepsilon _{\left( i,t\right) }|\mathcal{F}_{n}^{Z}\right] $. Note
that, in this case, $\left\{ \left( u_{\left( i,t\right) ,n},\varepsilon
_{\left( i,t\right) }\right) \right\} _{\left( i,t\right) =1}^{m_{n}}$ is
independent conditional on $\mathcal{F}_{n}^{Z}=\sigma \left( Z\right) $, and

\noindent $\max_{1\leq \left( i,t\right) \leq m_{n}}E\left[ \varepsilon
_{\left( i,t\right) }^{4}|\mathcal{F}_{n}^{Z}\right] \leq C$ $a.s.$ by
Assumptions 1(i) and 2(i), respectively. Moreover, the upper bound given by (%
\ref{4th moment bd on u tilde}) implies that there exists a constant $%
C^{\ast }>0$ such that

\noindent\ $\max_{1\leq \left( i,t\right) \leq m_{n}}E\left[ u_{\left(
i,t\right) ,n}^{4}|\mathcal{F}_{n}^{Z}\right] =\max_{1\leq \left( i,t\right)
\leq m_{n}}\left( \mu _{n}^{\min }\right) ^{4}E\left[ \underline{u}_{\left(
i,t\right) ,n}^{4}|\mathcal{F}_{n}^{Z}\right] \leq \left( \mu _{n}^{\min
}\right) ^{4}C^{\ast }/\left( \mu _{n}^{\min }\right) ^{4}=C^{\ast }$ \ $%
a.s.n.$ Finally, note that, by using the fact that

\noindent $\underline{u}_{\left( i,t\right) ,n}=a^{\prime }\left( \left( \mu
_{n}^{\min }\right) ^{2}\widetilde{L}_{n}H_{n}^{-1}\Sigma _{2,n}H_{n}^{-1}%
\widetilde{L}_{n}^{\prime }/K_{2,n}\right) ^{-1/2}\widetilde{L}%
_{n}H_{n}^{-1}D_{\mu }^{-1}\underline{U}_{\left( i,t\right) }$ and by
applying Assumption 2(i), Lemma S2-6, and the assumption that

\noindent $\lambda _{\min }\left( \left( \mu _{n}^{\min }\right) ^{2}%
\widetilde{L}_{n}H_{n}^{-1}\Sigma _{2,n}H_{n}^{-1}\widetilde{L}_{n}^{\prime
}/K_{2,n}\right) \geq \underline{C}>0$ \ $a.s.n.$; we can show that there
exists a constant $C>0$ such that%
\begin{eqnarray}
&&E\left[ \left\vert \left( \mu _{n}^{\min }\right) \underline{u}_{\left(
i,t\right) ,n}\varepsilon _{\left( i,t\right) }\right\vert |\mathcal{F}%
_{n}^{Z}\right]  \notag \\
&=&\left( \mu _{n}^{\min }\right) E\left[ \left\vert \varepsilon _{\left(
i,t\right) }\underline{U}_{\left( i,t\right) }^{\prime }D_{\mu
}^{-1}H_{n}^{-1}\widetilde{L}_{n}^{\prime }\left( \frac{\left( \mu
_{n}^{\min }\right) ^{2}\widetilde{L}_{n}H_{n}^{-1}\Sigma _{2,n}H_{n}^{-1}%
\widetilde{L}_{n}^{\prime }}{K_{2,n}}\right) ^{-1/2}a\right\vert |\mathcal{F}%
_{n}^{Z}\right]  \notag \\
&\leq &\left( \mu _{n}^{\min }\right) \sqrt{E\left[ \varepsilon _{\left(
i,t\right) }^{2}|\mathcal{F}_{n}^{Z}\right] }\left[ a^{\prime }\left( \frac{%
\left( \mu _{n}^{\min }\right) ^{2}\widetilde{L}_{n}H_{n}^{-1}\Sigma
_{2,n}H_{n}^{-1}\widetilde{L}_{n}^{\prime }}{K_{2,n}}\right) ^{-1/2}%
\widetilde{L}_{n}H_{n}^{-1}D_{\mu }^{-1}\right.  \notag \\
&&\left. \times E\left[ \underline{U}_{\left( i,t\right) }\underline{U}%
_{\left( i,t\right) }^{\prime }|\mathcal{F}_{n}^{Z}\right] D_{\mu
}^{-1}H_{n}^{-1}\widetilde{L}_{n}^{\prime }\left( \frac{\left( \mu
_{n}^{\min }\right) ^{2}\widetilde{L}_{n}H_{n}^{-1}\Sigma _{2,n}H_{n}^{-1}%
\widetilde{L}_{n}^{\prime }}{K_{2,n}}\right) ^{-1/2}a\right] ^{1/2}\text{(by
CS inequality)}  \notag \\
&\leq &\left( \mu _{n}^{\min }\right) \sqrt{E\left[ \varepsilon _{\left(
i,t\right) }^{2}|\mathcal{F}_{n}^{Z}\right] }\frac{1}{\left( \mu _{n}^{\min
}\right) }\left( \sqrt{\max_{1\leq \left( i,t\right) \leq m_{n}}E\left[
\left\Vert \underline{U}_{\left( i,t\right) }\right\Vert _{2}^{2}|\mathcal{F}%
_{n}^{Z}\right] }\right)  \notag \\
&&\times \frac{1}{\lambda _{\min }\left( \Upsilon ^{\prime }Z_{2}^{\prime
}M^{\left( Z_{1},Q\right) }Z_{2}\Upsilon /n\right) }\left\Vert \widetilde{L}%
_{n}\right\Vert _{F}\left( \frac{1}{\sqrt{\lambda _{\min }\left( \left( \mu
_{n}^{\min }\right) ^{2}\widetilde{L}_{n}H_{n}^{-1}\Sigma _{2,n}H_{n}^{-1}%
\widetilde{L}_{n}^{\prime }/K_{2,n}\right) }}\right)  \notag \\
&\leq &C<\infty \text{ \ }a.s.\text{ for all }\left( i,t\right) \in \left\{
1,2,...,m_{n}\right\} \text{ and for all }n\text{ sufficiently large}
\label{bd on cov of e and u tilde}
\end{eqnarray}%
from which we further deduce that $\max_{\left( i,t\right) }\left\vert \phi
_{\left( i,t\right) }\right\vert \leq \max_{\left( i,t\right) }E\left[
\left\vert \left( \mu _{n}^{\min }\right) \underline{u}_{\left( i,t\right)
,n}\varepsilon _{\left( i,t\right) }\right\vert |\mathcal{F}_{n}^{Z}\right]
\leq C$ $a.s.n.$and also that $\max_{\left( j,s\right) }\left\vert \overline{%
\psi }_{\left( j,s\right) }\right\vert \leq \max_{\left( j,s\right) }E\left[
\left\vert \left( \mu _{n}^{\min }\right) \underline{u}_{\left( j,s\right)
,n}\varepsilon _{\left( j,s\right) }\right\vert |\mathcal{F}_{n}^{Z}\right]
\leq C$ $a.s.n.$ Hence, applying part (a) of Lemma S2-8, we have $\mathcal{TT%
}_{3}\overset{p}{\rightarrow }0$.

Next, consider the term%
\begin{equation*}
\mathcal{TT}_{4}=\frac{2\left( \mu _{n}^{\min }\right) ^{2}}{K_{2,n}}%
\dsum\limits_{\left( i,t\right) =3}^{m_{n}}\dsum\limits_{\left( j,s\right)
=2}^{\left( i,t\right) -1}\dsum\limits_{\left( k,\upsilon \right)
=1}^{\left( j,s\right) -1}A_{\left( i,t\right) ,\left( j,s\right) }A_{\left(
i,t\right) ,\left( k,\upsilon \right) }E\left[ \underline{u}_{\left(
i,t\right) ,n}\varepsilon _{\left( i,t\right) }|\mathcal{F}_{n}^{Z}\right]
\left\{ \underline{u}_{\left( j,s\right) ,n}\varepsilon _{\left( k,\upsilon
\right) }+\varepsilon _{\left( j,s\right) }\underline{u}_{\left( k,\upsilon
\right) ,n}\right\}
\end{equation*}%
Here, we apply part (b) of Lemma S2-8 with $u_{\left( j,s\right) ,n}=\left(
\mu _{n}^{\min }\right) \underline{u}_{\left( j,s\right) ,n}$ and

\noindent $\phi _{\left( i,t\right) }=E\left[ \left( \mu _{n}^{\min }\right) 
\underline{u}_{\left( i,t\right) ,n}\varepsilon _{\left( i,t\right) }|%
\mathcal{F}_{n}^{Z}\right] $. Note that $\left\{ \left( u_{\left( i,t\right)
,n},\varepsilon _{\left( i,t\right) }\right) \right\} _{\left( i,t\right)
=1}^{m_{n}}$ is independent conditional on $\mathcal{F}_{n}^{Z}=\sigma
\left( Z\right) $, and

\noindent $\max_{1\leq \left( i,t\right) \leq m_{n}}E\left[ \varepsilon
_{\left( i,t\right) }^{4}|\mathcal{F}_{n}^{Z}\right] \leq C$ $a.s.$ by
Assumptions 1 and 2(i), respectively. Moreover, from calculations given
previously, we have $\max_{1\leq \left( i,t\right) \leq m_{n}}\left( \mu
_{n}^{\min }\right) ^{4}E\left[ \underline{u}_{\left( i,t\right) ,n}^{4}|%
\mathcal{F}_{n}^{Z}\right] \leq C$ \ $a.s.n.$ and

\noindent $\max_{\left( i,t\right) }\left\vert \phi _{\left( i,t\right)
}\right\vert \leq C$ $a.s.n.$ Hence, by applying part (b) of Lemma S2-8, we
deduce that $\mathcal{TT}_{4}\overset{p}{\rightarrow }0$.

Turning our attention to the term%
\begin{equation*}
\mathcal{TT}_{5}=\frac{2\left( \mu _{n}^{\min }\right) ^{2}}{K_{2,n}}%
\dsum\limits_{\left( i,t\right) =3}^{m_{n}}\dsum\limits_{\left( j,s\right)
=2}^{\left( i,t\right) -1}\dsum\limits_{\left( k,\upsilon \right)
=1}^{\left( j,s\right) -1}A_{\left( i,t\right) ,\left( j,s\right) }A_{\left(
i,t\right) ,\left( k,\upsilon \right) }\varepsilon _{\left( j,s\right)
}\varepsilon _{\left( k,\upsilon \right) }E\left[ \underline{u}_{\left(
i,t\right) ,n}^{2}|\mathcal{F}_{n}^{Z}\right]
\end{equation*}%
For this term, we apply part (c) of Lemma S2-8 with $\phi _{\left(
i,t\right) }=E\left[ u_{\left( i,t\right) ,n}^{2}|\mathcal{F}_{n}^{Z}\right] 
$ and $u_{\left( i,t\right) ,n}=\left( \mu _{n}^{\min }\right) \underline{u}%
_{\left( i,t\right) ,n}$. From (\ref{2nd moment bd on u bar}), there exists
a positive constant $C$ such that $E\left[ u_{\left( i,t\right) ,n}^{2}|%
\mathcal{F}_{n}^{Z}\right] $

\noindent $=\left( \mu _{n}^{\min }\right) ^{2}E\left[ \underline{u}_{\left(
i,t\right) ,n}^{2}|\mathcal{F}_{n}^{Z}\right] \leq C<\infty $ \ $a.s.$ for
all $\left( i,t\right) \in \left\{ 1,2,...,m_{n}\right\} $ and for all $n$
sufficiently large, so that $\max_{\left( i,t\right) }\left\vert \phi
_{\left( i,t\right) }\right\vert =\max_{1\leq \left( i,t\right) \leq m_{n}}E%
\left[ u_{\left( i,t\right) ,n}^{2}|\mathcal{F}_{n}^{Z}\right] \leq C$ $%
a.s.n.$ Hence, applying part (c) of Lemma S2-8, we obtain $\mathcal{TT}_{5}%
\overset{p}{\rightarrow }0$.

Finally, consider the term%
\begin{equation*}
\mathcal{TT}_{6}=\frac{2\left( \mu _{n}^{\min }\right) ^{2}}{K_{2,n}}%
\dsum\limits_{\left( i,t\right) =3}^{m_{n}}\dsum\limits_{\left( j,s\right)
=2}^{\left( i,t\right) -1}\dsum\limits_{\left( k,\upsilon \right)
=1}^{\left( j,s\right) -1}A_{\left( i,t\right) ,\left( j,s\right) }A_{\left(
i,t\right) ,\left( k,\upsilon \right) }\underline{u}_{\left( j,s\right) ,n}%
\underline{u}_{\left( k,\upsilon \right) ,n}E\left[ \varepsilon _{\left(
i,t\right) }^{2}|\mathcal{F}_{n}^{Z}\right]
\end{equation*}%
In this case, we apply part (d) of Lemma S2-8 with $u_{\left( j,s\right)
}=\left( \mu _{n}^{\min }\right) \underline{u}_{\left( j,s\right) ,n}$, $%
u_{\left( k,\upsilon \right) }=\left( \mu _{n}^{\min }\right) \underline{u}%
_{\left( k,\upsilon \right) ,n}$, and $\phi _{\left( i,t\right) }=E\left[
\varepsilon _{\left( i,t\right) }^{2}|\mathcal{F}_{n}^{Z}\right] $. Using a
conditional version of Liapounov's inequality and Assumption 2(i), we obtain 
$E\left[ \varepsilon _{\left( i,t\right) }^{2}|\mathcal{F}_{n}^{Z}\right]
\leq \left( E\left[ \varepsilon _{\left( i,t\right) }^{4}|\mathcal{F}_{n}^{Z}%
\right] \right) ^{1/2}\leq C<\infty $ \ $a.s.$ for all $\left( i,t\right)
\in \left\{ 1,2,...,m_{n}\right\} $ and for all $n$ sufficiently large, so
that $\max_{\left( i,t\right) }\left\vert \phi _{\left( i,t\right)
}\right\vert =\max_{\left( i,t\right) }E\left[ \varepsilon _{\left(
i,t\right) }^{2}|\mathcal{F}_{n}^{Z}\right] \leq C$ $a.s.n.$ Moreover, the
upper bound in (\ref{4th moment bd on u tilde}) implies that $\max_{1\leq
\left( i,t\right) \leq m_{n}}E\left[ u_{\left( i,t\right) ,n}^{4}|\mathcal{F}%
_{n}^{Z}\right] =\max_{1\leq \left( i,t\right) \leq m_{n}}\left( \mu
_{n}^{\min }\right) ^{4}E\left[ \underline{u}_{\left( i,t\right) ,n}^{4}|%
\mathcal{F}_{n}^{Z}\right] $

\noindent $\leq C$ \ $a.s.n.$ It follows by applying part (d) of Lemma S2-8
that $\mathcal{TT}_{6}\overset{p}{\rightarrow }0$.

It follows from the above calculations that the terms $\mathcal{TT}_{k}$ $%
\overset{p}{\rightarrow }0$ for each $k\in \left\{ 1,...,6\right\} $, which
in light of equation (\ref{Thm 3 cond 2}) implies that $\dsum\nolimits_{%
\left( i,t\right) =2}^{m_{n}}E\left[ \mathcal{J}_{\left( i,t\right) ,n}^{2}|%
\mathcal{F}_{\left( i,t\right) -1,n}\right] -1=o_{p}\left( 1\right) $. This
establishes condition (\ref{MDA CLT cond 2}) of Lemma S2-15. It now follows
from Lemma S2-15 that $\mathcal{J}_{n}$

\noindent $=\left( \mu _{n}^{\min }/\sqrt{K_{2,n}}\right) a^{\prime }\left(
\left( \mu _{n}^{\min }\right) ^{2}\widetilde{L}_{n}H_{n}^{-1}\Sigma
_{2,n}H_{n}^{-1}\widetilde{L}_{n}^{\prime }/K_{2,n}\right) ^{-1/2}\widetilde{%
L}_{n}H_{n}^{-1}D_{\mu }^{-1}\underline{U}^{\prime }A\varepsilon \overset{d}{%
\rightarrow }N\left( 0,1\right) $. Since this result holds for all $a\in 
\mathbb{R}^{d}$ such that $\left\Vert a\right\Vert _{2}=1$, applying the Cram%
\'{e}r-Wold device, we further deduce that%
\begin{equation}
\left( \mu _{n}^{\min }/\sqrt{K_{2,n}}\right) \left( \widetilde{L}%
_{n}\Lambda _{II,n}\widetilde{L}_{n}^{\prime }\right) ^{-1/2}\widetilde{L}%
_{n}H_{n}^{-1}D_{\mu }^{-1}\underline{U}^{\prime }A\varepsilon \overset{d}{%
\rightarrow }N\left( 0,I_{d}\right) ,  \label{asy normality CaseII}
\end{equation}%
where $\Lambda _{II,n}=\left( \mu _{n}^{\min }\right) ^{2}H_{n}^{-1}\Sigma
_{2,n}H_{n}^{-1}/K_{2,n}$ with $H_{n}=\Upsilon ^{\prime }Z_{2}^{\prime
}M^{\left( Z_{1},Q\right) }Z_{2}\Upsilon /n$. Next, recall that $\widehat{%
\Delta }\left( \delta \right) =-\left[ \left( y-X\delta \right) ^{\prime
}M^{\left( Z_{1},Q\right) }\left( y-X\delta \right) /2\right] \left[
\partial \widehat{Q}_{FELIM}\left( \delta \right) /\partial \delta \right] $%
; and note that, by Lemma S2-10, we have $-D_{\mu }^{-1}\left( \partial 
\widehat{\Delta }\left( \overline{\delta }_{n}\right) /\partial \delta
^{\prime }\right) D_{\mu }^{-1}=H_{n}+o_{p}\left( 1\right) $, with $H_{n}$
being positive definite in light of Assumption 3(iii), so that upon
inverting the expansion given in expression (\ref{mean value approx}) above
and multiplying by $\left( \mu _{n}^{\min }\right) /\sqrt{K_{2,n}}$, we
obtain%
\begin{equation*}
\left( \mu _{n}^{\min }/\sqrt{K_{2,n}}\right) D_{\mu }\left( \widehat{\delta 
}_{L,n}-\delta _{0}\right) =\left( \mu _{n}^{\min }/\sqrt{K_{2,n}}\right)
H_{n}^{-1}D_{\mu }^{-1}\widehat{\Delta }\left( \delta _{0}\right) \left[
1+o_{p}\left( 1\right) \right]
\end{equation*}%
\begin{equation*}
=\left( \mu _{n}^{\min }/\sqrt{K_{2,n}}\right) H_{n}^{-1}D_{\mu }^{-1}%
\underline{U}^{\prime }A\varepsilon \left[ 1+o_{p}\left( 1\right) \right] 
\text{,}
\end{equation*}

\noindent where the last equality comes from applying expression (\ref{Thm3
initial approx}). It follows by multiplying both sides of the equation above
by $\left( \widetilde{L}_{n}\Lambda _{II,n}\widetilde{L}_{n}\right) ^{-1/2}%
\widetilde{L}_{n}$ and applying the result given in expression (\ref{asy
normality CaseII}) that $\left( \mu _{n}^{\min }/\sqrt{K_{2,n}}\right)
\left( \widetilde{L}_{n}\Lambda _{II,n}\widetilde{L}_{n}\right) ^{-1/2}%
\widetilde{L}_{n}D_{\mu }\left( \widehat{\delta }_{L,n}-\delta _{0}\right) 
\overset{d}{\rightarrow }N\left( 0,I_{d}\right) $. \textit{\ }

Turning our attention now to $\widehat{\delta }_{F,n}$, note that, using
expression (\ref{FEFUL standardized}) above, we can write%
\begin{eqnarray*}
&&\frac{\left( \mu _{n}^{\min }\right) D_{\mu }\left( \widehat{\delta }%
_{F,n}-\delta _{0}\right) }{\sqrt{K_{2,n}}} \\
&=&\frac{\left( \mu _{n}^{\min }\right) \left( D_{\mu }^{-1}X^{\prime }\left[
A-\widehat{\ell }_{F,n}M^{\left( Z_{1},Q\right) }\right] XD_{\mu
}^{-1}\right) ^{-1}D_{\mu }^{-1}X^{\prime }\left[ A-\widehat{\ell }%
_{F,n}M^{\left( Z_{1},Q\right) }\right] \left( y-X\delta _{0}\right) }{\sqrt{%
K_{2,n}}}
\end{eqnarray*}%
It follows by applying Lemmas S2-12 and S2-13 that%
\begin{equation}
\frac{\mu _{n}^{\min }}{\sqrt{K_{2,n}}}D_{\mu }\left( \widehat{\delta }%
_{F,n}-\delta _{0}\right) =\frac{\mu _{n}^{\min }}{\sqrt{K_{2,n}}}%
H_{n}^{-1}D_{\mu }^{-1}\underline{U}^{\prime }A\varepsilon +o_{p}\left(
1\right) \text{,}  \label{normalized FEFUL II}
\end{equation}%
noting that, in this case, $\left( \mu _{n}^{\min }\right) /\sqrt{K_{2,n}}%
=o\left( 1\right) $ but $\sqrt{K_{2,n}}/\left( \mu _{n}^{\min }\right)
^{2}\rightarrow 0$. It follows by multiplying both sides of equation (\ref%
{normalized FEFUL II}) above by $\left( \widetilde{L}_{n}\Lambda _{II,n}%
\widetilde{L}_{n}\right) ^{-1/2}\widetilde{L}_{n}$ and applying the result
given in expression (\ref{asy normality CaseII}) that

\noindent $\left( \mu _{n}^{\min }/\sqrt{K_{2,n}}\right) \left( \widetilde{L}%
_{n}\Lambda _{II,n}\widetilde{L}_{n}\right) ^{-1/2}\widetilde{L}_{n}D_{\mu
}\left( \widehat{\delta }_{F,n}-\delta _{0}\right) \overset{d}{\rightarrow }%
N\left( 0,I_{d}\right) $. $\square $

\section*{\noindent Appendix S2: Key Lemmas Used in Proving the Main
Theorems\noindent}

\noindent \qquad In this appendix, we state a number of lemmas that are used
in the proofs of the main theorems of the paper. Proofs for these lemmas are
available in a separate online appendix which can be viewed at the URL:
http://econweb.umd.edu/\symbol{126}chao/Research/research\_files/Additional%
\_Online\_

\noindent
Appendix\_Jackknife\_Estimation\_Cluster\_Sample\_IV\_Model\_August\_19%
\_2022.pdf

\medskip

\noindent \textbf{Lemma S2-1: }Let $A=P^{\perp }-M^{\left( Z,Q\right) }D_{%
\widehat{\vartheta }}M^{\left( Z,Q\right) }$. Then, under Assumptions 2-6,
the following statements hold as $K_{2,n}$, $n\rightarrow \infty $.

\noindent (a) $\dsum\nolimits_{\left( i,t\right) ,\left( j,s\right)
=1,\left( i,t\right) \neq \left( j,s\right) }^{m_{n}}A_{\left( i,t\right)
,\left( j,s\right) }^{2}=O_{a.s.}\left( K_{2,n}\right) $.

\noindent (b) $\dsum\nolimits_{\left( i,t\right) ,\left( j,s\right)
=1,\left( i,t\right) \neq \left( j,s\right) }^{m_{n}}A_{\left( i,t\right)
,\left( j,s\right) }^{4}=O_{a.s.}\left( K_{2,n}^{3}/n^{2}\right) $.

\noindent (c) $\dsum\nolimits_{\left( j,s\right)
=1}^{m_{n}}\dsum\nolimits_{\left( i,t\right) ,\left( k,\upsilon \right)
=1,\left( i,t\right) \neq \left( j,s\right) ,\left( k,\upsilon \right) \neq
\left( j,s\right) }^{m_{n}}A_{\left( i,t\right) ,\left( j,s\right)
}^{2}A_{\left( j,s\right) ,\left( k,\upsilon \right) }^{2}=O_{a.s.}\left(
K_{2,n}^{2}/n\right) $.

\noindent (d) $\max_{1\leq \left( i,t\right) \leq m_{n}}\left(
\dsum\nolimits_{\left( j,s\right) =1}^{m_{n}}A_{\left( i,t\right) ,\left(
j,s\right) }^{2}\right) =O_{a.s.}\left( K_{2,n}/n\right) $.

\noindent (e) $\dsum\nolimits_{i_{1},i_{2}=1}^{n}\dsum\nolimits_{j=1,j\neq
i_{1},i_{2}}^{n}\dsum\nolimits_{t_{1}=1}^{T_{i_{1}}}\dsum%
\nolimits_{t_{2}=1}^{T_{i_{2}}}\dsum\nolimits_{s_{1},s_{2}=1}^{T_{j}}A_{%
\left( i_{1},t_{1}\right) ,\left( j,s_{1}\right) }^{2}A_{\left(
i_{2},t_{2}\right) ,\left( j,s_{2}\right) }^{2}=O_{a.s.}\left(
K_{2,n}^{2}/n\right) $ and $\dsum\nolimits_{i=1}^{n}\dsum%
\nolimits_{j_{1}=1,j_{1}\neq i}^{n}\dsum\nolimits_{j_{2}=1,j_{2}\neq
i}^{n}\dsum\nolimits_{t_{1},t_{2}=1}^{T_{i}}\dsum%
\nolimits_{s_{1}=1}^{T_{j_{1}}}\dsum\nolimits_{s_{2}=1}^{T_{j_{2}}}A_{\left(
i,t_{1}\right) ,\left( j_{1},s_{1}\right) }^{2}A_{\left( i,t_{2}\right)
,\left( j_{2},s_{2}\right) }^{2}=O_{a.s.}\left( K_{2,n}^{2}/n\right) $

\noindent (f) $\dsum\nolimits_{i=1}^{n}\dsum\nolimits_{t=1}^{T_{i}}\dsum%
\nolimits_{s=1}^{T_{i}}A_{\left( i,t\right) ,\left( i,s\right)
}^{2}=O_{a.s.}\left( K_{2,n}^{2}/n\right) $.

\noindent (g) $\dsum\nolimits_{i=1}^{n}\dsum\nolimits_{s_{1},t_{1}=1,s_{1}%
\neq t_{1}}^{T_{i}}\dsum\nolimits_{s_{2},t_{2}=1,s_{2}\neq
t_{2}}^{T_{i}}A_{\left( i,t_{1}\right) ,\left( i,s_{1}\right) }^{2}A_{\left(
i,t_{2}\right) ,\left( i,s_{2}\right) }^{2}=O_{a.s.}\left(
K_{2,n}^{4}/n^{3}\right) $.

\noindent \textbf{Lemma S2-2: }Let Assumptions 1-6 be satisfied. Then, the
following statements are true: (a) $D_{\mu }^{-1}X^{\prime }M^{\left(
Z_{1},Q\right) }XD_{\mu }^{-1}=O_{p}\left( n\left( \mu _{n}^{\min }\right)
^{-2}\right) $; (b) $D_{\mu }^{-1}X^{\prime }AXD_{\mu
}^{-1}=H_{n}+o_{p}\left( 1\right) $, where $H_{n}=\Upsilon ^{\prime
}Z_{2}^{\prime }M^{\left( Z_{1},Q\right) }Z_{2}\Upsilon /n=O_{p}\left(
1\right) $.

\medskip

\noindent \textbf{Lemma S2-3: }Let $\underline{U}=U-\varepsilon \rho
^{\prime }$ and $\underline{U}_{\left( i,t\right) }=U_{\left( i,t\right)
}-\rho \varepsilon _{\left( i,t\right) }$ and let $VC\left( X|\mathcal{F}%
_{n}^{Z}\right) $ denote the conditional covariance matrix of the random
vector $X$ given $\mathcal{F}_{n}^{Z}$. Under Assumptions 1-2, 5-6, and 8;
there exists positive constants $0<\underline{C}\leq \overline{C}<\infty $
such that the following statements are true.

\noindent (a) $\lambda _{\max }\left[ VC\left( \Upsilon ^{\prime
}Z_{2}^{\prime }M^{\left( Z_{1},Q\right) }\varepsilon /\sqrt{n}|\mathcal{F}%
_{n}^{Z}\right) \right] \leq \overline{C}$ \ \ $a.s.$ and $\lambda _{\min }%
\left[ VC\left( \Upsilon ^{\prime }Z_{2}^{\prime }M^{\left( Z_{1},Q\right)
}\varepsilon /\sqrt{n}|\mathcal{F}_{n}^{Z}\right) \right] \geq \underline{C}$
\ $a.s.$ for all $n$ sufficiently large.

\noindent (b) $VC\left( \underline{U}^{\prime }A\varepsilon /\sqrt{K_{2,n}}|%
\mathcal{F}_{n}^{Z}\right) \geq \underline{C}I_{d}>\underset{d\times d}{0}$
\ $a.s.$, for all $n$ sufficiently large.

\noindent (c) $\lambda _{\max }\left( VC\left[ \underline{U}^{\prime
}A\varepsilon /\sqrt{K_{2,n}}|\mathcal{F}_{n}^{Z}\right] \right) \leq 
\overline{C}$ \ $a.s.$, $\lambda _{\max }\left( VC\left[ \underline{U}%
^{\prime }A\varepsilon /\sqrt{K_{2,n}}\right] \right) \leq \overline{C}$,

\noindent $\lambda _{\max }\left( VC\left[ U^{\prime }A\varepsilon /\sqrt{%
K_{2,n}}|\mathcal{F}_{n}^{Z}\right] \right) \leq \overline{C}$ \ $a.s.$, and 
$\lambda _{\max }\left( VC\left[ U^{\prime }A\varepsilon /\sqrt{K_{2,n}}%
\right] \right) \leq \overline{C}$, for all $n$ sufficiently large.

\noindent (d) For any $a\in \mathbb{R}^{d}$ with $\left\Vert a\right\Vert
_{2}=1$ and for all $n$ sufficiently large, $\lambda _{\min }\left( \Sigma
_{n}\right) \geq \underline{C}>0$ $a.s.$ and $a^{\prime }\Sigma
_{n}^{-1}a\leq \overline{C}<\infty $ \ $a.s.$, where $\Sigma _{n}=VC\left( 
\mathcal{Y}_{n}|\mathcal{F}_{n}^{Z}\right) =\Sigma _{1,n}+\Sigma _{2,n}$, as
defined in section 4 of the main paper, and where $\mathcal{Y}_{n}=\Upsilon
^{\prime }Z_{2}^{\prime }M^{\left( Z_{1},Q\right) }\varepsilon /\sqrt{n}%
+D_{\mu }^{-1}\underline{U}^{\prime }A\varepsilon $.

\noindent \textbf{Lemma S2-4: }Under Assumptions 1-6, $D_{\mu
}^{-1}X^{\prime }A\varepsilon =\Upsilon ^{\prime }Z_{2}^{\prime }M^{\left(
Z_{1},Q\right) }\varepsilon /\sqrt{n}+D_{\mu }^{-1}U^{\prime }A\varepsilon $

\noindent $=O_{p}\left( \max \left\{ 1,\sqrt{K_{2,n}}/\left( \mu _{n}^{\min
}\right) \right\} \right) $

\noindent \textbf{Lemma S2-5: }Under Assumptions 1-6, $D_{\mu
}^{-1}X^{\prime }M^{\left( Z_{1},Q\right) }\varepsilon =O_{p}\left( n/\mu
_{n}^{\min }\right) $.

\noindent

\noindent \textbf{Lemma S2-6: }If Assumptions 2 and 8 are satisfied; then,
for $1\leq p\leq 8$ and for all $n$, there exists a positive constant $C$
such that $\max_{1\leq \left( i,t\right) \leq m_{n}}E\left[ \left\Vert 
\underline{U}_{\left( i,t\right) }\right\Vert _{2}^{p}|\mathcal{F}_{n}^{Z}%
\right] \leq C<\infty \ a.s.$, where $\underline{U}_{\left( i,t\right)
}=U_{\left( i,t\right) }-\rho \varepsilon _{\left( i,t\right) }$.

\noindent \textbf{Lemma S2-7: }Under Assumptions 1-6, the following results
hold: (a) $\widehat{\ell }_{L,n}=o_{p}\left( \left[ \mu _{n}^{\min }\right]
^{2}/n\right) $; (b) $\widehat{\ell }_{F,n}=o_{p}\left( \left[ \mu
_{n}^{\min }\right] ^{2}/n\right) $.

\noindent

\noindent \textbf{Lemma S2-8: }Let $A$ be as defined above. Assume that%
{\small \ }i){\small \ }$\left( u_{\left( 1,1\right) ,n},\varepsilon
_{\left( 1,1\right) }\right) ,...,\left( u_{\left( 1,T_{1}\right)
,n},\varepsilon _{\left( 1,T_{1}\right) }\right) ,$

\noindent $\left( u_{\left( 2,1\right) ,n},\varepsilon _{\left( 2,1\right)
,n}\right) ,...,\left( u_{\left( 2,T_{2}\right) ,n},\varepsilon _{\left(
2,T_{2}\right) ,n}\right) ,...,\left( u_{\left( n,1\right) ,n},\varepsilon
_{\left( n,1\right) ,n}\right) ....,\left( u_{\left( n,T_{n}\right)
,n},\varepsilon _{\left( n,T_{n}\right) ,n}\right) ${\small \ }are
independent conditional on $\mathcal{F}_{n}^{Z}=\sigma \left( Z\right) $;
ii)\ there exists a constant $C$\ such that, almost surely for all $n$
sufficiently large, $\max_{1\leq \left( i,t\right) \leq m_{n}}E\left(
u_{\left( i,t\right) ,n}^{4}|\mathcal{F}_{n}^{Z}\right) \leq C$, $%
\max_{1\leq \left( i,t\right) \leq m_{n}}E\left( \varepsilon _{\left(
i,t\right) ,n}^{4}|\mathcal{F}_{n}^{Z}\right) \leq C$, and

\noindent $\max_{1\leq \left( i,t\right) \leq m_{n}}\left\vert \phi _{\left(
i,t\right) ,n}\right\vert \leq C$. In addition, define $\overline{\psi }%
_{\left( j,s\right) ,n}=E\left[ u_{\left( j,s\right) ,n}\varepsilon _{\left(
j,s\right) ,n}|\mathcal{F}_{n}^{Z}\right] $ for $\left( j,s\right)
=1,...,m_{n}$. Then, under Assumptions 5 and 6, the following statements are
true:

\noindent (a) $K_{2,n}^{-1}\dsum\nolimits_{1\leq \left( j,s\right) <\left(
i,t\right) \leq m_{n}}A_{\left( i,t\right) ,\left( j,s\right) }^{2}\phi
_{\left( i,t\right) ,n}\left\{ u_{\left( j,s\right) ,n}\varepsilon _{\left(
j,s\right) ,n}-\overline{\psi }_{\left( j,s\right) ,n}\right\} \overset{p}{%
\rightarrow }0$;

\noindent (b) $K_{2,n}^{-1}\dsum\nolimits_{1\leq \text{ }\left( k,\upsilon
\right) <\left( j,s\right) <\left( i,t\right) \leq \text{ }m_{n}}A_{\left(
i,t\right) ,\left( j,s\right) }A_{\left( i,t\right) ,\left( k,\upsilon
\right) }\phi _{\left( i,t\right) ,n}\left\{ u_{\left( j,s\right)
,n}\varepsilon _{\left( k,\upsilon \right) ,n}+\varepsilon _{\left(
j,s\right) ,n}u_{\left( k,\upsilon \right) ,n}\right\} \overset{p}{%
\rightarrow }0$;

\noindent (c) $K_{2,n}^{-1}\dsum\nolimits_{1\leq \text{ }\left( k,\upsilon
\right) <\left( j,s\right) <\left( i,t\right) \leq \text{ }m_{n}}A_{\left(
i,t\right) ,\left( j,s\right) }A_{\left( i,t\right) ,\left( k,\upsilon
\right) }\phi _{\left( i,t\right) ,n}\varepsilon _{\left( j,s\right)
,n}\varepsilon _{\left( k,\upsilon \right) ,n}\overset{p}{\rightarrow }0$;

\noindent (d) $K_{2,n}^{-1}\dsum\nolimits_{1\leq \text{ }\left( k,\upsilon
\right) <\left( j,s\right) <\left( i,t\right) \leq \text{ }m_{n}}A_{\left(
i,t\right) ,\left( j,s\right) }A_{\left( i,t\right) ,\left( k,\upsilon
\right) }\phi _{\left( i,t\right) ,n}u_{\left( j,s\right) ,n}u_{\left(
k,\upsilon \right) ,n}\overset{p}{\rightarrow }0$.

\noindent \textbf{Lemma S2-9: }Let 
\begin{equation*}
\widehat{\Delta }\left( \delta _{0}\right) =-\frac{\left( y-X\delta
_{0}\right) ^{\prime }M^{\left( Z_{1},Q\right) }\left( y-X\delta _{0}\right) 
}{2}\left. \frac{\partial }{\partial \delta }\left\{ \frac{\left( y-X\delta
\right) ^{\prime }A\left( y-X\delta \right) }{\left( y-X\delta \right)
^{\prime }M^{\left( Z_{1},Q\right) }\left( y-X\delta \right) }\right\}
\right\vert _{\delta =\delta _{0}}\text{.}
\end{equation*}%
If Assumptions 1-6 and 8 are satisfied; then, $D_{\mu }^{-1}\widehat{\Delta }%
\left( \delta _{0}\right) =\Upsilon ^{\prime }Z_{2}^{\prime }M^{\left(
Z_{1},Q\right) }\varepsilon /\sqrt{n}+D_{\mu }^{-1}\underline{U}^{\prime
}A\varepsilon +o_{p}\left( 1\right) $, where $\underline{U}=U-\varepsilon
\rho ^{\prime }$ and where $\rho =\lim_{n\rightarrow \infty }E\left[
U^{\prime }M^{Q}\varepsilon \right] /E\left[ \varepsilon ^{\prime
}M^{Q}\varepsilon \right] $.

\noindent \textbf{Lemma S2-10: }Let Assumptions 1-6 be satisfied, and let $%
\overline{\delta }_{n}$ be any estimator such that, as $n\rightarrow \infty $%
, $D_{\mu }\left( \overline{\delta }_{n}-\delta _{0}\right) /\mu _{n}^{\min
}=o_{p}\left( 1\right) $. Then, $-D_{\mu }^{-1}\left( \partial \widehat{%
\Delta }\left( \overline{\delta }_{n}\right) /\partial \delta ^{\prime
}\right) D_{\mu }^{-1}=H_{n}+o_{p}\left( 1\right) $, where $H_{n}=\Upsilon
^{\prime }Z_{2}^{\prime }M^{\left( Z_{1},Q\right) }Z_{2}\Upsilon /n$ and
where

\noindent $\widehat{\Delta }\left( \delta \right) =-\left[ \left( y-X\delta
\right) ^{\prime }M^{\left( Z_{1},Q\right) }\left( y-X\delta \right) /2%
\right] \left[ \partial \widehat{Q}_{FELIM}\left( \delta \right) /\partial
\delta \right] $

\noindent $=X^{\prime }A\left( y-X\delta \right) -\widehat{\ell }\left(
\delta \right) X^{\prime }M^{\left( Z_{1},Q\right) }\left( y-X\delta \right) 
$, with

\noindent $\widehat{\ell }\left( \delta \right) =\left( y-X\delta \right)
^{\prime }A\left( y-X\delta \right) /\left[ \left( y-X\delta \right)
^{\prime }M^{\left( Z_{1},Q\right) }\left( y-X\delta \right) \right] $. In
addition, we also have%
\begin{equation}
D_{\mu }^{-1}X^{\prime }\left[ A-\widehat{\ell }\left( \overline{\delta }%
_{n}\right) M^{\left( Z_{1},Q\right) }\right] XD_{\mu
}^{-1}=H_{n}+o_{p}\left( 1\right) \text{.}  \label{LIM denom term}
\end{equation}

\medskip

\noindent \textbf{Lemma S2-11: }Let $\widehat{\ell }_{L}=Q\left( \widetilde{%
\beta }\right) =\min_{\beta \in \overline{B}}Q\left( \beta \right) $, where $%
Q\left( \beta \right) $ is as defined in Assumption 9. Then, $\widehat{\ell }%
_{L}$ is also the smallest root of the determinantal equation $\det \left[ 
\overline{X}^{\prime }A\overline{X}-\ell \overline{X}^{\prime }M^{\left(
Z_{1},Q\right) }\overline{X}\right] =0$, where $\overline{X}=\left[ y,X%
\right] $. Assume in addition that condition (13) in Assumption 9 is
satisfied; then, $\widehat{\ell }_{L}$ has the representation 
\begin{equation}
\widehat{\ell }_{L}=\frac{\left( y-X\widehat{\delta }_{L}\right) ^{\prime
}A\left( y-X\widehat{\delta }_{L}\right) }{\left( y-X\widehat{\delta }%
_{L}\right) ^{\prime }M^{\left( Z_{1},Q\right) }\left( y-X\widehat{\delta }%
_{L}\right) }\text{,}  \label{FELIM root}
\end{equation}%
where $\widehat{\delta }_{L}$ denotes the FELIM\ estimator. Moreover, $%
\overline{X}^{\prime }A\left( y-X\widehat{\delta }_{L}\right) -\widehat{\ell 
}_{L}\overline{X}^{\prime }M^{\left( Z_{1},Q\right) }\left( y-X\widehat{%
\delta }_{L}\right) $

\noindent $=0$. In particular, this implies that $\widehat{\Delta }\left( 
\widehat{\delta }_{L}\right) =0$, where

\noindent $\widehat{\Delta }\left( \delta \right) =-\left[ \left( y-X\delta
\right) ^{\prime }M^{\left( Z_{1},Q\right) }\left( y-X\delta \right) /2%
\right] \left( \partial \widehat{Q}_{FELIM}\left( \delta \right) /\partial
\delta \right) $, so that $\widehat{\delta }_{L}$ satisfies the set of
(normalized) first-order conditions for minimizing the variance ratio
objective function $\widehat{Q}_{FELIM}\left( \delta \right) =\left(
y-X\delta \right) ^{\prime }A\left( y-X\delta \right) /\left[ \left(
y-X\delta \right) ^{\prime }M^{\left( Z_{1},Q\right) }\left( y-X\delta
\right) \right] $.

\noindent

\noindent \textbf{Lemma S2-12: }If Assumptions 1-6 are satisfied; then,

\noindent $D_{\mu }^{-1}X^{\prime }\left[ A-\widehat{\ell }_{F,n}M^{\left(
Z_{1},Q\right) }\right] XD_{\mu }^{-1}=H_{n}+o_{p}\left( 1\right) $, where $%
H_{n}=\Upsilon ^{\prime }Z_{2}^{\prime }M^{\left( Z_{1},Q\right)
}Z_{2}\Upsilon /n$,

\noindent $\widehat{\ell }_{F,n}=\left[ \widehat{\ell }_{L,n}-\left( 1-%
\widehat{\ell }_{L,n}\right) \left( C/m_{n}\right) \right] /\left[ 1-\left(
1-\widehat{\ell }_{L,n}\right) \left( C/m_{n}\right) \right] $, and $%
\widehat{\ell }_{L,n}$ is smallest root of the determinantal equation $\det
\left\{ \overline{X}^{\prime }A\overline{X}-\ell \overline{X}^{\prime
}M^{\left( Z_{1},Q\right) }\overline{X}\right\} =0$, with $\overline{X}=%
\left[ 
\begin{array}{cc}
y & X%
\end{array}%
\right] $.

\noindent \noindent \textbf{Lemma S2-13: }If Assumptions 1-6 and 8-9 are
satisfied; then, $D_{\mu }^{-1}X^{\prime }\left[ A-\widehat{\ell }%
_{F,n}M^{\left( Z_{1},Q\right) }\right] \left( y-X\delta _{0}\right) $

\noindent $=\mathcal{Y}_{n}\left[ 1+o_{p}\left( 1\right) \right] $, where $%
\mathcal{Y}_{n}=\Upsilon ^{\prime }Z_{2}^{\prime }M^{\left( Z_{1},Q\right)
}\varepsilon /\sqrt{n}+D_{\mu }^{-1}\underline{U}^{\prime }A\varepsilon $
with $\underline{U}=U-\varepsilon \rho ^{\prime }$ and

\noindent $\rho =\lim_{n\rightarrow \infty }E\left[ U^{\prime
}M^{Q}\varepsilon \right] /E\left[ \varepsilon ^{\prime }M^{Q}\varepsilon %
\right] $.

\noindent

\noindent \textbf{Lemma S2-14: }For any $a\in \mathbb{R}^{d}$ such that $%
\left\Vert a\right\Vert =1$, define $b_{1n}=\Sigma _{n}^{-1/2}a$, $b_{2n}=%
\sqrt{K_{2,n}}D_{\mu }^{-1}\Sigma _{n}^{-1/2}a$, $\underline{u}_{2,\left(
i,t\right) ,n}=b_{2n}^{\prime }\underline{U}_{\left( i,t\right) }$

\noindent $=\sqrt{K_{2,n}}a^{\prime }\Sigma _{n}^{-1/2}D_{\mu }^{-1}%
\underline{U}_{\left( i,t\right) }$, $\sigma _{\left( i,t\right) ,n}^{2}=E%
\left[ \varepsilon _{\left( i,t\right) }^{2}|\mathcal{F}_{n}^{Z}\right] $, $%
\widetilde{\psi }_{\left( i,t\right) ,n}=E\left[ \underline{u}_{2,\left(
i,t\right) ,n}\varepsilon _{\left( i,t\right) }|\mathcal{F}_{n}^{Z}\right] $%
, and $\widetilde{\omega }_{\left( i,t\right) }^{2}=E\left[ \underline{u}%
_{2,\left( i,t\right) ,n}^{2}|\mathcal{F}_{n}^{Z}\right] $. If Assumptions
1-2 and 5-6 are satisfied; then, the following statements are true.

\noindent (a) $\dsum\nolimits_{\left( i,t\right)
=2}^{m_{n}}\dsum\nolimits_{\left( j,s\right) =1}^{\left( i,t\right) -1}\left[
b_{1n}^{\prime }\Upsilon ^{\prime }Z_{2}^{\prime }M^{\left( Z_{1},Q\right)
}e_{\left( i,t\right) }/\sqrt{n}\right] \left( A_{\left( i,t\right) ,\left(
j,s\right) }/\sqrt{K_{2,n}}\right) \left\{ \varepsilon _{\left( j,s\right) }%
\widetilde{\psi }_{\left( i,t\right) ,n}+\underline{u}_{2,\left( j,s\right)
}\sigma _{\left( i,t\right) ,n}^{2}\right\} $

\noindent $=O_{p}\left( K_{2,n}^{1/4}/\mu _{n}^{\min }\right) =o_{p}\left(
1\right) $.

\noindent (b) $\dsum\nolimits_{\left( i,t\right)
=2}^{m_{n}}\dsum\nolimits_{\left( j,s\right) =1}^{\left( i,t\right)
-1}\left( A_{\left( i,t\right) ,\left( j,s\right) }^{2}/K_{2,n}\right)
\left( \varepsilon _{\left( j,s\right) }^{2}-\sigma _{\left( j,s\right)
,n}^{2}\right) \widetilde{\omega }_{\left( i,t\right) ,n}^{2}=O_{p}\left(
K_{2,n}\left( \mu _{n}^{\min }\right) ^{-2}n^{-1/2}\right) =o_{p}\left(
1\right) $.

\noindent (c) $\dsum\nolimits_{\left( i,t\right)
=2}^{m_{n}}\dsum\nolimits_{\left( j,s\right) =1}^{\left( i,t\right)
-1}\left( A_{\left( i,t\right) ,\left( j,s\right) }^{2}/K_{2,n}\right)
\left( \underline{u}_{2,\left( j,s\right) ,n}^{2}-\widetilde{\omega }%
_{\left( j,s\right) ,n}^{2}\right) \sigma _{\left( i,t\right)
,n}^{2}=O_{p}\left( K_{2,n}\left( \mu _{n}^{\min }\right)
^{-2}n^{-1/2}\right) =o_{p}\left( 1\right) $.

\qquad \noindent

\noindent \textbf{Lemma S2-15 (G\"{a}nsler and Stute, 1977): }Let $\left\{
X_{i,n},\mathcal{F}_{i,n},1\leq i\leq k_{n},n\geq 1\right\} $ be a square
integrable martingale difference array. Suppose that for all $\epsilon >0$%
\begin{eqnarray}
&&\dsum\limits_{i=1}^{k_{n}}E\left[ X_{i,n}^{2}\mathbb{I}\left\{ \left\vert
X_{i,n}\right\vert >\epsilon \right\} |\mathcal{F}_{i-1,n}\right] \overset{p}%
{\rightarrow }0\text{ and}  \label{MDA CLT cond 1} \\
&&\dsum\limits_{i=1}^{k_{n}}E\left[ X_{i,n}^{2}|\mathcal{F}_{i-1,n}\right] 
\overset{p}{\rightarrow }1\text{.}  \label{MDA CLT cond 2}
\end{eqnarray}%
Then, $\dsum\nolimits_{i=1}^{k_{n}}X_{i,n}\overset{d}{\rightarrow }N\left(
0,1\right) $.

\medskip

\noindent \textbf{Remark: }Note that a sufficient condition for condition (%
\ref{MDA CLT cond 1}), which we will verify in lieu of (\ref{MDA CLT cond 1}%
) in the proof of Theorems 2 and 3\ in Appendix S1, is the following%
\begin{equation}
\dsum\limits_{i=1}^{k_{n}}E\left[ \left\vert X_{i,n}\right\vert ^{2+\delta }%
\right] \overset{p}{\rightarrow }0\text{, for some }\delta >0.
\label{MDA CLT cond 1b}
\end{equation}

\bigskip

\noindent \textbf{Lemma S2-16: }Let $\widetilde{L}_{n}$ be a sequence of $%
l\times d$ nonrandom matrices (with $l\leq d$) such that $\left\Vert 
\widetilde{L}_{n}\right\Vert _{F}^{2}\leq \overline{C}<\infty $ for some
constant $\overline{C}$, and let $\Sigma _{2,n}=VC\left( D_{\mu }^{-1}%
\underline{U}^{\prime }A\varepsilon |\mathcal{F}_{n}^{Z}\right) $

\noindent $=D_{\mu }^{-1}VC\left( \underline{U}^{\prime }A\varepsilon |%
\mathcal{F}_{n}^{Z}\right) D_{\mu }^{-1}$. Assume that there exists a
positive constant $\underline{C}$ such that

\noindent $\lambda _{\min }\left( \left( \mu _{n}^{\min }\right) ^{2}%
\widetilde{L}_{n}H_{n}^{-1}\Sigma _{2,n}H_{n}^{-1}\widetilde{L}_{n}^{\prime
}/K_{2,n}\right) \geq \underline{C}>0$ $a.s.n$. Furthermore, let $a\in 
\mathbb{R}^{d}$ such that $\left\Vert a\right\Vert _{2}=1$ and let $%
\underline{u}_{a,\left( i,t\right) ,n}=a^{\prime }\left( \left( \mu
_{n}^{\min }\right) ^{2}\widetilde{L}_{n}H_{n}^{-1}\Sigma _{2,n}H_{n}^{-1}%
\widetilde{L}_{n}^{\prime }/K_{2,n}\right) ^{-1/2}\widetilde{L}_{n}D_{\mu
}^{-1}\underline{U}_{\left( i,t\right) }$. Let Assumptions 1-2 and 5-6 be
satisfied and assume that $\left( \mu _{n}^{\min }\right)
^{2}/K_{2,n}=o\left( 1\right) $ but

\noindent $\sqrt{K_{2,n}}/\left( \mu _{n}^{\min }\right) ^{2}\rightarrow 0$.
Under these conditions, the following statements are true:

\noindent (a) $\left[ \left( \mu _{n}^{\min }\right) ^{2}/K_{2,n}\right]
\dsum\nolimits_{\left( i,t\right) =2}^{m_{n}}\dsum\nolimits_{\left(
j,s\right) =1}^{\left( i,t\right) -1}A_{\left( i,t\right) ,\left( j,s\right)
}^{2}\left( \underline{u}_{a,\left( j,s\right) ,n}^{2}-E\left[ \underline{u}%
_{a,\left( j,s\right) ,n}^{2}|\mathcal{F}_{n}^{Z}\right] \right) E\left[
\varepsilon _{\left( i,t\right) }^{2}|\mathcal{F}_{n}^{Z}\right] $

$=O_{p}\left( n^{-1/2}\right) =o_{p}\left( 1\right) $;

\noindent (b) $\left[ \left( \mu _{n}^{\min }\right) ^{2}/K_{2,n}\right]
\dsum\nolimits_{\left( i,t\right) =2}^{m_{n}}\dsum\nolimits_{\left(
j,s\right) =1}^{\left( i,t\right) -1}A_{\left( i,t\right) ,\left( j,s\right)
}^{2}\left( \varepsilon _{\left( j,s\right) }^{2}-E\left[ \varepsilon
_{\left( j,s\right) }^{2}|\mathcal{F}_{n}^{Z}\right] \right) E\left[ 
\underline{u}_{a,\left( i,t\right) ,n}^{2}|\mathcal{F}_{n}^{Z}\right] $

\noindent $=O_{p}\left( n^{-1/2}\right) =o_{p}\left( 1\right) $.

\medskip

\noindent \textbf{Lemma S2-17 \ }Under Assumptions 1-6, $D_{\mu
}^{-1}X^{\prime }AD\left( \varepsilon \circ \varepsilon \right) AXD_{\mu
}^{-1}=\Sigma _{1,n}$

\noindent $+\dsum\nolimits_{\left( i,t\right) ,\left( j,s\right) =1,\left(
i,t\right) \neq \left( j,s\right) }^{m_{n}}A_{\left( i,t\right) ,\left(
j,s\right) }^{2}\sigma _{\left( i,t\right) }^{2}D_{\mu }^{-1}\Psi _{\left(
j,s\right) }D_{\mu }^{-1}+o_{p}\left( 1\right) $, where $\Sigma
_{1,n}=\Upsilon ^{\prime }Z_{2}^{\prime }M^{\left( Z_{1},Q\right) }D_{\sigma
^{2}}M^{\left( Z_{1},Q\right) }Z_{2}\Upsilon /n$,

\noindent $\sigma _{\left( i,t\right) }^{2}=E\left[ \varepsilon _{\left(
i,t\right) }^{2}|\mathcal{F}_{n}^{Z}\right] $, $D_{\sigma ^{2}}=diag\left(
\sigma _{\left( 1,1\right) }^{2},...,\sigma _{\left( n,T_{n}\right)
}^{2}\right) $, and $\Psi _{\left( j,s\right) }=E\left[ U_{\left( j,s\right)
}U_{\left( j,s\right) }^{\prime }|\mathcal{F}_{n}^{Z}\right] $.

\medskip

\noindent \textbf{Lemma S2-18 }Let Assumptions 1-6 and 8 be satisfied, and
let $\left\{ \widehat{\delta }_{n}\right\} $ be any sequence of estimators
such that $\left\Vert \widehat{\delta }_{n}-\delta _{0}\right\Vert _{2}%
\overset{p}{\rightarrow }0$ as $n\rightarrow \infty $, as long as $\sqrt{%
K_{2,n}}/\left( \mu _{n}^{\min }\right) ^{2}\rightarrow 0$. Also, define the
following notations: let $\widehat{\varepsilon }=M^{\left( Z,Q\right)
}\left( y-X\widehat{\delta }_{n}\right) $, $J=\left[ M^{Q}\circ M^{Q}\right]
^{-1}$, $S_{1}=X^{\prime }AD\left( J\left[ \widehat{\varepsilon }\circ 
\widehat{\varepsilon }\right] \right) AX$, $S_{2}=\left( \widehat{%
\varepsilon }\circ \widehat{\varepsilon }\right) ^{\prime }J\left( A\circ
A\right) J\left( \widehat{\varepsilon }\iota _{d}^{\prime }\circ M^{\left(
Z,Q\right) }X\right) $,

\noindent $\underline{S}_{2}=\left( \widehat{\varepsilon }\circ \widehat{%
\varepsilon }\right) ^{\prime }J\left( A\circ A\right) J\left( \widehat{%
\varepsilon }\iota _{d}^{\prime }\circ \widehat{\underline{U}}\right) $ with 
$\widehat{\underline{U}}=M^{\left( Z,Q\right) }X-\widehat{\varepsilon }%
\widehat{\rho }_{n}^{\prime }$, $S_{3}=\left( \widehat{\varepsilon }\circ 
\widehat{\varepsilon }\right) ^{\prime }J\left( A\circ A\right) J\left( 
\widehat{\varepsilon }\circ \widehat{\varepsilon }\right) $,

\noindent $S_{4}=\left( \widehat{\varepsilon }\iota _{d}^{\prime }\circ
M^{\left( Z,Q\right) }X\right) ^{\prime }J\left( A\circ A\right) J\left( 
\widehat{\varepsilon }\iota _{d}^{\prime }\circ M^{\left( Z,Q\right)
}X\right) $, $\underline{S}_{4}=\left( \widehat{\varepsilon }\iota
_{d}^{\prime }\circ \widehat{\underline{U}}\right) ^{\prime }J\left( A\circ
A\right) J\left( \widehat{\varepsilon }\iota _{d}^{\prime }\circ \widehat{%
\underline{U}}\right) $, and $\Sigma _{1,n}=\Upsilon ^{\prime }Z_{2}^{\prime
}M^{\left( Z_{1},Q\right) }D_{\sigma ^{2}}M^{\left( Z_{1},Q\right)
}Z_{2}\Upsilon /n$. In addition, define $\sigma _{\left( i,t\right) }^{2}=E%
\left[ \varepsilon _{\left( i,t\right) }^{2}|\mathcal{F}_{n}^{Z}\right] $,

\noindent $D_{\sigma ^{2}}=diag\left( \sigma _{\left( 1,1\right)
}^{2},...,\sigma _{\left( n,T_{n}\right) }^{2}\right) $, $\phi _{\left(
i,t\right) }=E\left[ U_{\left( i,t\right) }\varepsilon _{\left( i,t\right) }|%
\mathcal{F}_{n}^{Z}\right] $, $\Psi _{\left( i,t\right) }=E\left[ U_{\left(
i,t\right) }U_{\left( i,t\right) }^{\prime }|\mathcal{F}_{n}^{Z}\right] $,

\noindent $\underline{\phi }_{\left( i,t\right) }=E\left[ \underline{U}%
_{\left( i,t\right) }\varepsilon _{\left( i,t\right) }|\mathcal{F}_{n}^{Z}%
\right] $, and $\underline{\Psi }_{\left( i,t\right) }=E\left[ \underline{U}%
_{\left( i,t\right) }\underline{U}_{\left( i,t\right) }^{\prime }|\mathcal{F}%
_{n}^{Z}\right] $ where $\underline{U}_{\left( i,t\right) }=U_{\left(
i,t\right) }-\rho \varepsilon _{\left( i,t\right) }$ and where for
notational convenience we suppress the dependence of $\sigma _{\left(
i,t\right) }^{2}$, $\phi _{\left( i,t\right) }$, $\Psi _{\left( i,t\right) }$%
, $\underline{\phi }_{\left( i,t\right) }$, and $\underline{\Psi }_{\left(
i,t\right) }$ on $\mathcal{F}_{n}^{Z}=\sigma \left( Z\right) $. Then, under
the above conditions, the following statements are true.

\begin{enumerate}
\item[(a)] $D_{\mu }^{-1}S_{1}D_{\mu }^{-1}=\Sigma
_{1,n}+\dsum\nolimits_{\left( i,t\right) ,\left( j,s\right) =1,\left(
i,t\right) \neq \left( j,s\right) }^{m_{n}}A_{\left( i,t\right) ,\left(
j,s\right) }^{2}\sigma _{\left( i,t\right) }^{2}D_{\mu }^{-1}\Psi _{\left(
j,s\right) }D_{\mu }^{-1}$

$+o_{p}\left( \max \left\{ 1,K_{2,n}\left( \mu _{n}^{\min }\right)
^{-2}\right\} \right) $.

\item[(b)] $S_{3}/K_{2,n}-K_{2,n}^{-1}\dsum\nolimits_{\left( i,t\right)
,\left( j,s\right) =1,\left( i,t\right) \neq \left( j,s\right)
}^{m_{n}}A_{\left( i,t\right) ,\left( j,s\right) }^{2}\sigma _{\left(
i,t\right) }^{2}\sigma _{\left( j,s\right) }^{2}=o_{p}\left( 1\right) $.

\item[(c)] $D_{\mu }^{-1}S_{4}D_{\mu }^{-1}-\dsum\nolimits_{\left(
i,t\right) ,\left( j,s\right) =1,\left( i,t\right) \neq \left( j,s\right)
}^{m_{n}}A_{\left( i,t\right) ,\left( j,s\right) }^{2}D_{\mu }^{-1}\phi
_{\left( i,t\right) }\phi _{\left( j,s\right) }^{\prime }D_{\mu
}^{-1}=o_{p}\left( K_{2,n}\left( \mu _{n}^{\min }\right) ^{-2}\right) $.

\item[(d)] $\left( \mu _{n}^{\min }/K_{2,n}\right) S_{2}D_{\mu }^{-1}-\left(
\mu _{n}^{\min }/K_{2,n}\right) \dsum\nolimits_{\left( i,t\right) ,\left(
j,s\right) =1,\left( i,t\right) \neq \left( j,s\right) }^{m_{n}}A_{\left(
i,t\right) ,\left( j,s\right) }^{2}\sigma _{\left( i,t\right) }^{2}\phi
_{\left( j,s\right) }^{\prime }D_{\mu }^{-1}=o_{p}\left( 1\right) $.

\item[(e)] $D_{\mu }^{-1}\widehat{\rho }_{n}=O_{p}\left( \left( \mu
_{n}^{\min }\right) ^{-1}\right) $ and $D_{\mu }^{-1}\left( \widehat{\rho }%
_{n}-\rho \right) =o_{p}\left( \left( \mu _{n}^{\min }\right) ^{-1}\right) $%
, where $\rho =\lim_{n\rightarrow \infty }\rho _{n}=\lim_{n\rightarrow
\infty }\left( E\left[ U^{\prime }M^{Q}\varepsilon \right] /n\right) /\left(
E\left[ \varepsilon ^{\prime }M^{Q}\varepsilon \right] /n\right) $.

\item[(f)] $D_{\mu }^{-1}\underline{S}_{4}D_{\mu
}^{-1}-\dsum\nolimits_{\left( i,t\right) ,\left( j,s\right) =1,\left(
i,t\right) \neq \left( j,s\right) }^{m_{n}}A_{\left( i,t\right) ,\left(
j,s\right) }^{2}D_{\mu }^{-1}\underline{\phi }_{\left( i,t\right) }%
\underline{\phi }_{\left( j,s\right) }^{\prime }D_{\mu }^{-1}=o_{p}\left(
K_{2,n}\left( \mu _{n}^{\min }\right) ^{-2}\right) $.

\item[(g)] $\left( \mu _{n}^{\min }/K_{2,n}\right) -\left( \mu _{n}^{\min
}/K_{2,n}\right) \dsum\nolimits_{\left( i,t\right) ,\left( j,s\right)
=1,\left( i,t\right) \neq \left( j,s\right) }^{m_{n}}A_{\left( i,t\right)
,\left( j,s\right) }^{2}\sigma _{\left( i,t\right) }^{2}\underline{\phi }%
_{\left( j,s\right) }^{\prime }D_{\mu }^{-1}=o_{p}(1)$.
\end{enumerate}

\begin{thebibliography}{9}
\bibitem{} Billingsley, P. (1995). \textit{Probability and Measure. }New
York: John Wiley \& Sons.

\bibitem{} G\"{a}nsler, P. and W. Stute (1977). \textit{%
Wahrscheinlichkeitstheorie}. New York: Springer-Verlag
\end{thebibliography}

\end{document}
