%2multibyte Version: 5.50.0.2960 CodePage: 936


\documentclass[11pt]{article}
%%%%%%%%%%%%%%%%%%%%%%%%%%%%%%%%%%%%%%%%%%%%%%%%%%%%%%%%%%%%%%%%%%%%%%%%%%%%%%%%%%%%%%%%%%%%%%%%%%%%%%%%%%%%%%%%%%%%%%%%%%%%%%%%%%%%%%%%%%%%%%%%%%%%%%%%%%%%%%%%%%%%%%%%%%%%%%%%%%%%%%%%%%%%%%%%%%%%%%%%%%%%%%%%%%%%%%%%%%%%%%%%%%%%%%%%%%%%%%%%%%%%%%%%%%%%
\usepackage{amssymb}
\usepackage{amsfonts}
\usepackage{amsmath}
\usepackage{geometry}
\usepackage[onehalfspacing]{setspace}
\usepackage{setspace}

\setcounter{MaxMatrixCols}{10}
%TCIDATA{OutputFilter=LATEX.DLL}
%TCIDATA{Version=5.00.0.2606}
%TCIDATA{Codepage=936}
%TCIDATA{<META NAME="SaveForMode" CONTENT="1">}
%TCIDATA{BibliographyScheme=Manual}
%TCIDATA{Created=Sunday, July 18, 2004 16:10:34}
%TCIDATA{LastRevised=Monday, September 08, 2025 17:51:06}
%TCIDATA{<META NAME="GraphicsSave" CONTENT="32">}
%TCIDATA{<META NAME="DocumentShell" CONTENT="General\Blank Document">}
%TCIDATA{Language=American English}
%TCIDATA{CSTFile=article.cst}
%TCIDATA{PageSetup=72,72,72,72,0}
%TCIDATA{AllPages=
%F=36,\PARA{038<p type="texpara" tag="Body Text" >\hfill \thepage}
%}


\newtheorem{theorem}{Theorem}
\newtheorem{acknowledgement}[theorem]{Acknowledgement}
\newtheorem{algorithm}[theorem]{Algorithm}
\newtheorem{axiom}[theorem]{Axiom}
\newtheorem{case}[theorem]{Case}
\newtheorem{claim}[theorem]{Claim}
\newtheorem{conclusion}[theorem]{Conclusion}
\newtheorem{condition}[theorem]{Condition}
\newtheorem{conjecture}[theorem]{Conjecture}
\newtheorem{corollary}[theorem]{Corollary}
\newtheorem{criterion}[theorem]{Criterion}
\newtheorem{definition}[theorem]{Definition}
\newtheorem{example}[theorem]{Example}
\newtheorem{exercise}[theorem]{Exercise}
\newtheorem{lemma}[theorem]{Lemma}
\newtheorem{notation}[theorem]{Notation}
\newtheorem{problem}[theorem]{Problem}
\newtheorem{proposition}[theorem]{Proposition}
\newtheorem{remark}[theorem]{Remark}
\newtheorem{solution}[theorem]{Solution}
\newtheorem{summary}[theorem]{Summary}
\newenvironment{proof}[1][Proof]{\noindent\textbf{#1.} }{\ \rule{0.5em}{0.5em}}
\renewcommand{\baselinestretch}{1.0} 
\textwidth=6.8in
\textheight=8.7in
\oddsidemargin=0in
\evensidemargin=0in
\topmargin=0in
\baselineskip=10pt
\linespread{1.3}
\input{tcilatex}
\geometry{left=1in,right=1in,top=1.25in,bottom=1.25in}

\begin{document}


\begin{center}
{\Large {Consistent Estimation, Variable Selection, and Forecasting in
Factor-Augmented VAR Models$^{\ast }$}}

\bigskip

John C. Chao$^{1}$, Yang Liu$^{2}$, Kaiwen Qiu$^{3}$, and Norman R. Swanson$%
^{3}$

\medskip

$^{1}$University of Maryland, $^{2}$Towson University, and $^{3}$Rutgers
University

\medskip

September 8, 2025

\bigskip \bigskip

Abstract
\end{center}

\begin{spacing}{1.01}
\noindent {\small We introduce a completely consistent method for variable selection with 
high dimensional datasets.
The method is presented in a framework where latent factors are estimated for the purpose of 
dimension reduction, 
and is meant to serve as a complement to extant methods. We argue that the method is of particular interest 
in empirical settings where there may be many irrelevant predictor variables. The reason for this is that situations where there are ``too many'' irrelevant variables can lead to inconsistent factor estimates. 
Interestingly, our method yields a consistent estimate of the number of such irrelevant variables, which can aid the 
applied practitioner in assessing the strength of the underlying factor structure for a particular application.
We also show that when factors constructed using our variable selection method are inputted into the forecast equations
implied by a factor augmented vector autoregressive (FAVAR) model for the purpose of forecasting, the
conditional mean forecast equations can be consistently estimated. Monte Carlo results are presented indicating that the 
variable selection method performs well in finite samples. The paper also contains two empirical illustrations, where 
we compare forecasts constructed from factor-augmented forecast equations using our variable selection procedure 
with two alternative methods for factor construction - the conventional PCA procedure, which does not pre-screen the variables, and a 
hard thresholding method that is widely used in the empirical literature. Overall, we find that our method to 
outperform both of these alternative procedures in a majority of the cases that we study across different target variables, 
forecast horizons, and data window specifications (i.e., recursive or rolling).}
\end{spacing}

\medskip

\noindent \textit{Keywords: }Factor analysis, factor augmented vector
autoregression, forecasting, moderate deviation, principal components,
self-normalization, variable selection.

\medskip

\noindent \textit{JEL Classification: }C32, C33, C38, C52, C53, C55.

\bigskip \bigskip

\begin{spacing}{1.01}
\noindent {\small $^{\ast }$John C. Chao, Department of Economics, 7343 Preinkert
Drive, University of Maryland, chao@econ.umd.edu. Kaiwen Qiu, Department of Economics, 9500 Hamilton Street, Rutgers University, kq60@economics.rutgers.edu. Norman R. Swanson, Department of Economics, 9500 Hamilton Street, Rutgers University,
nswanson@econ.rutgers.edu. Yang Liu, Department of Economics, 8000 York Rd, Towson, MD 21252, Towson University,
yliu@towson.edu. The authors are grateful to Matteo Barigozzi, Rong Chen, Harold Chiang, Domenico Giannone, Bruce Hansen, Cheng Hsiao,
Yuan Liao, Jack Porter, Xiaoxia Shi, Minchul Shin, Xiye Yang, Peter Zadrosny and participants at the 2022 International Association of Applied Econometrics conference, the 2023 AiE Conference in Honor of Joon Park, 
the 2024 Kansas Econometrics Workshop, the 2023 Rochester Conference in Econometrics, the 2024 Annual Conference of the Society for Economic Measurement, the Federal Reserve Bank of Philadelphia, and the univeristies of Glasgow, Riverside, and Wisconsin-Madison for useful comments received on earlier versions
of this paper. Chao thanks the University of Maryland for research support.}
\end{spacing}

\newpage

\noindent \noindent \setcounter{page}{2}

\section{Introduction}

In this paper, we propose a simple to implement and completely consistent
method for variable selection when estimating factors for use in dimension
reduction and factor augmented vector autoregression (FAVAR) forecasting
with high dimensional datasets. Our method is meant to add to variable
selection methods available to empirical practitioners that have been
introduced in recent papers, including those due to Bair, Hastie, Paul, and
Tibshirani (2006), Bai and Ng (2008), Giglio, Xiu, and Zhang (2023a,b),
among others. In addition, our method builds on previous supervised machine
learning approaches that are now widely used in the literature, including,
and not limited to principal components analysis (PCA), sparse, and
supervised PCA (see, e.g. Zou, Hastie, and Tibshirani (2006), Barshan,
Ghodsi, Azimifar, and Jahromi (2011), Carrasco and Rossi (2016), and Fan, Ke
and Liao (2021)); bagging, boosting, and random forest (see, e.g. Breiman
(1996), Freund and Schapire (1997), Breiman (2001), Lee and Yang (2006),
Lee, Ullah, and Wang (2020), and the references cited therein); and
regression methods such as the elastic net, garrote, and lasso (see e.g.
Breiman (1995), Tibshirani (1996), Zou and Hastie (2005), Kim and Swanson
(2014), Belloni, Chernozhukov, and Wang (2014), and the references cited
therein).

The variable selection procedure introduced here seeks to identify and
eliminate those irrelevant variables which do not load on any of the
underlying factors so that only relevant variables which contain information
about at least one of the latent factors are used in factor estimation.%
\footnote{%
Although we interpret our variable selection procedure primarily as a
procedure which selects variables on the basis of relevance, it should be
noted that in a forecasting context the procedure considered here can also
be useful for assessing whether a particular variable has predictive content
for the target variable of interest. Please see Remark 2.2 for a detailed
discussion about the close relationship between the relevance and the
predictive content of a variable in the context of a FAVAR model and how a
score statistic can be useful in assessing both.} This is of importance
because the use of irrelevant variables in extracting the latent factors
could lead to less accuracy since these variables contribute only noise but
not signal to the factor estimation process. Although variable selection
procedures for factor estimation have also been studied in the well-known
paper by Bair, Hastie, Paul, and Tibshirani (2006) on supervised principal
component methods and in some interesting recent papers by Giglio, Xiu, and
Zhang (2023a,b), a notable difference between our selection method and those
proposed in these other papers is our use of a self-normalized statistic. A
key attribute of self-normalized statistics is that their tail behavior can
be better approximated over a wider range, using moderate deviation results,
than statistics which are not self-normalized, as we will explain in greater
details in Section 2 of the paper. This, in turn, allows us to specify our
decision rule in such a way so that our procedure will be completely
consistent in the sense that the probability of Type I and Type II errors
will both go to zero as sample sizes approach infinity\footnote{%
Here, we take Type I error to be the error that an irrelevant variable is
falsely selected as a relevant variable, whereas Type II error is the error
of misclassifying a relevant variable as being irrelevant.}.

An important added value of our completely consistent variable selection
procedure is that it enables us to construct a consistent estimator $%
\widehat{N}_{1}$ of $N_{1}$ (the number of relevant variables) in the sense
that $\widehat{N}_{1}/N_{1}\overset{p}{\rightarrow }1$.\ As explained in
Section 2 of this paper, since $N_{1}$ is itself not directly observable,
having a consistent estimator $\widehat{N}_{1}$ provides empirical
researchers with a useful diagnostic statistic which can help them assess
the overall pervasiveness of the factors in empirical applications. As
discussed in Section 2, consistent estimation of $N_{1}$ will not be
possible if one does not have a selection method where the probability of a
Type II error approaches zero asymptotically. In addition, we will also not
be able to consistently estimate $N_{1}$ if the probability of a Type I
error is not controlled to vanish asymptotically, except in the special case
where $N_{2}$, the number of irrelevant variables, is negligible relative to 
$N_{1}$ ( i.e., the case where $N_{2}/N_{1}\rightarrow 0$). However, if $%
N_{2}/N_{1}\rightarrow 0$ then forecast results based on use of our
procedure should not be much different from forecast results obtained from
the use of conventional PCA (where no variable pre-screening is conducted).
This does not seem to be the case, in particular, for the FRED-MD dataset
that we examine in an empirical illustration (see Section 5 for further
details).\footnote{%
A formal proof of the consistency of $\widehat{N}_{1}$ is given in part (a)
of Lemma C-15 of the Technical Appendix. In addition, we also provide in
Section 2 of the paper some intuitive discussion about why having both the
probability of a Type I error and that of a Type II error vanish
asymptotically is important for the consistency of $\widehat{N}_{1}$. Please
see, in particular, Example 2 given in Remark 2.3.}

To properly control the probability of a Type I error in our setup, we
leverage on some important advances in moderate deviation results for weakly
dependent processes obtained recently by Chen, Shao, Wu, and Xu (2016). In
the context of the moderate deviation theory used here, a further advantage
of self-normalized statistics is that relative to their non-self-normalized
counterparts, statistics which are self-normalized are more able to
accommodate situations where the underlying distribution of the data may
have thicker tails. Hence, moderate deviation results for self-normalized
statistics require weaker moment conditions than statistics which are not
self-normalized.

In addition to proposing a new variable selection method and showing its
complete consistency, we also make a number of contributions to the
methodology of carrying out forecasting in a dynamic factor-augmented
modeling framework. More specifically, within a general FAVAR setup, which
allows time series forecasts to be made using information sets much richer
than that used in traditional VAR models, we provide an easy-to-implement
formula for the post-variable-selection principal component estimator of the
vector of factors. We then show that this post-variable-selection factor
estimator can consistently estimate the true factors up to an invertible
matrix transformation even if we do not impose the kind of normalization
conditions on the factors and the factor loadings that are typically made in
the literature; see, for example, Stock and Watson (2002a). Moreover, we
explicitly derive a closed-form representation for the system of $h$-step
ahead forecasting equations implied by a FAVAR model and show that, by
inserting our post-variable-selection factor estimates into the $h$-step
ahead forecasting equations, we can consistently estimate the conditional
mean function of the said equations; and this is true even if we do not make
strong enough identifying assumptions so that the factors can only be
consistently estimated up to an invertible matrix transformation.

Besides our theoretical results, we also present Monte Carlo results which
indicate that the finite sample performance of our variable selection
procedure is in accord with the results of our asymptotic analysis. In
particular, when the sample sizes are large, such as the case where $T=600$
and $N=1000$, then results of our simulation study show that both Type I and
Type II error rates are very close to zero. Moreover, even in the smaller
sample case where $T=100$ and $N=100$, the Type I and Type II error rates
are usually less than $0.05$, and are often much smaller than that. In
addition we carry out two small forecasting exercises to illustrate the
empirical relevance of our procedure. The first uses a variety of
macroeconomic variables from the well-known FRED-MD database and the second
employs data from an updated version of the GVAR dataset previously studied
in Dees, di Mauro, Pesaran, and Smith (2007) and Pesaran, Schuermann, and
Smith (2009). In both empirical illustrations, we compare forecasts
constructed from factor-augmented forecast equations using our variable
selection procedure with two alternative methods for factor construction
(the conventional PCA procedure, which does not pre-screen the variables,
and a hard thresholding method that is widely used in the empirical
literature). Overall, we find our method to outperform both of these
alternative procedures in a majority of the cases that we study across
different target variables, forecast horizons, and data window
specifications (i.e., recursive or rolling). We believe that our results
indicate the potential usefulness of our method in empirical applications.

The rest of the paper is organized as follows. In Section 2, we discuss the
FAVAR model and the assumptions that we impose on this model. We also
describe our variable selection procedure and provide theoretical results
establishing the complete consistency of the procedure. Section 3 provides
theoretical results on the consistent estimation of latent factors, up to an
invertible matrix transformation, as well as results on the consistent
estimation of the $h$-step ahead predictor, based on the FAVAR model.
Section 4 presents the results of a promising Monte Carlo study on the
finite sample performance of our variable selection method, and makes
recommendations regarding the calibration of the tuning parameter used in
said method. Section 5 presents the results of two empirical applications
comparing forecast results based on our method with those obtained from PCA
and the hard thresholding method of Bai and Ng (2008). Finally, Section 6
offers concluding remarks. Proofs of the main theorems as well as additional
supporting lemmas are given in a separate Technical Appendix.\footnote{%
The technical appendix is posted online at:
http://econweb.rutgers.edu/nswanson/papers/ and also at
http://econweb.umd.edu/\symbol{126}chao/Research/research.html} The
Technical Appendix is organized into three subappendices. Appendix A
provides proofs of the main theorems. Appendix B contains proofs of
supporting lemmas, used primarily in the proofs of Theorems 1 and 2, and
Appendix C contains the proofs of supporting lemmas used primarily in the
proofs of Theorems 3 and 4.

Before proceeding, we first say a few words about some of the notation used
in this paper. Throughout, let $\lambda _{\left( j\right) }\left( A\right) $%
, $\lambda _{\max }\left( A\right) $, $\lambda _{\min }\left( A\right) $,
and $tr\left( A\right) $ denote, respectively, the $j^{th}$ largest
eigenvalue, the maximal eigenvalue, the minimal eigenvalue, and the trace of
a square matrix $A$. Similarly, let $\sigma _{\left( j\right) }\left(
B\right) $, $\sigma _{\max }\left( B\right) $, and $\sigma _{\min }\left(
B\right) $ denote, respectively, the $j^{th}$ largest singular value, the
maximal singular value, and the minimal singular value of a matrix $B$,
which is not restricted to be a square matrix. In addition, let $\left\Vert
a\right\Vert _{2}$ denote the usual Euclidean norm when applied to a
(finite-dimensional) vector $a$. Also, for a matrix $A$, $\left\Vert
A\right\Vert _{2}\equiv \max \left\{ \sqrt{\lambda \left( A^{\prime
}A\right) }:\lambda \left( A^{\prime }A\right) \text{ is an eigenvalue of }%
A^{\prime }A\right\} $ denotes the matrix spectral norm, and $\left\Vert
A\right\Vert _{F}\equiv \sqrt{tr\left\{ A^{\prime }A\right\} }$ denotes the
Frobenius norm. For two random variables $X$ and $Y$, write $X\sim Y,$ if $%
X/Y=O_{p}\left( 1\right) $ and $Y/X=O_{p}\left( 1\right) $. Furthermore, let 
$\left\vert z\right\vert $ denote the absolute value or the modulus of the
number $z$; let $\left\lfloor \cdot \right\rfloor $ denote the floor
function, so that, for a real number $x$, $\left\lfloor x\right\rfloor $
gives the largest integer that is less than or equal to $x$; let $%
\left\lceil \cdot \right\rceil $ denote the ceiling function, so that, for a
real number $x$, $\left\lceil x\right\rceil $ gives the smallest integer
that is greater than or equal to $x$; and let $\iota _{p}=\left(
1,1,...,1\right) ^{\prime }$ denote a $p\times 1$ vector of ones. Finally,
the abbreviation w.p.a.1 stands for \textquotedblleft with probability
approaching one\textquotedblright .

\section{\noindent Model, Assumptions, and Variable Selection in High
Dimensions}

Consider the following $p^{th}$-order factor-augmented vector autoregression
(FAVAR):%
\begin{equation}
W_{t+1}=\mu +A_{1}W_{t}+\cdot \cdot \cdot +A_{p}W_{t-p+1}+\varepsilon _{t+1}%
\text{,}  \label{FAVAR}
\end{equation}%
where%
\begin{eqnarray*}
\underset{\left( d+K\right) \times 1}{W_{t+1}} &=&\left( 
\begin{array}{c}
\underset{d\times 1}{Y_{t+1}} \\ 
\underset{K\times 1}{F_{t+1}}%
\end{array}%
\right) \text{, }\underset{\left( d+K\right) \times 1}{\varepsilon _{t+1}}%
=\left( 
\begin{array}{c}
\underset{d\times 1}{\varepsilon _{t+1}^{Y}} \\ 
\underset{K\times 1}{\varepsilon _{t+1}^{F}}%
\end{array}%
\right) \text{, }\underset{\left( d+K\right) \times 1}{\mu }=\left( 
\begin{array}{c}
\underset{d\times 1}{\mu _{Y}} \\ 
\underset{K\times 1}{\mu _{F}}%
\end{array}%
\right) ,\text{ and} \\
\text{ }\underset{\left( d+K\right) \times \left( d+K\right) }{A_{g}}
&=&\left( 
\begin{array}{cc}
\underset{d\times d}{A_{YY,g}} & \underset{d\times K}{A_{YF,g}} \\ 
\underset{K\times d}{A_{FY,g}} & \underset{K\times K}{A_{FF,g}}%
\end{array}%
\right) ,\text{ for }g=1,...,p.
\end{eqnarray*}%
Here, $Y_{t}$ denotes the vector of observable economic variables, and $%
F_{t} $ is a vector of unobserved (latent) factors. In our analysis of this
model, it will often be convenient to rewrite the FAVAR in several
alternative forms, such as when making assumptions used in the sequel. We
thus briefly outline two alternative representations of the above model.
First, it is easy to see that the system of equations given in (\ref{FAVAR})
can be written in the form:%
\begin{eqnarray}
Y_{t+1} &=&\mu _{Y}+A_{YY}\underline{Y}_{t}+A_{YF}\underline{F}%
_{t}+\varepsilon _{t+1}^{Y},  \label{Y component FAVAR} \\
F_{t+1} &=&\mu _{F}+A_{FY}\underline{Y}_{t}+A_{FF}\underline{F}%
_{t}+\varepsilon _{t+1}^{F},  \label{F component FAVAR}
\end{eqnarray}%
where

\begin{eqnarray}
\underset{d\times dp}{A_{YY}} &=&\left( 
\begin{array}{cccc}
A_{YY,1} & A_{YY,2} & \cdots & A_{YY,p}%
\end{array}%
\right) ,\text{ }\underset{d\times Kp}{A_{YF}}=\left( 
\begin{array}{cccc}
A_{YF,1} & A_{YF,2} & \cdots & A_{YF,p}%
\end{array}%
\right) ,\text{ }  \notag \\
\underset{K\times dp}{A_{FY}} &=&\left( 
\begin{array}{cccc}
A_{FY,1} & A_{FY,2} & \cdots & A_{FY,p}%
\end{array}%
\right) ,\text{ }\underset{K\times Kp}{A_{FF}}=\left( 
\begin{array}{cccc}
A_{FF,1} & A_{FF,2} & \cdots & A_{FF,p}%
\end{array}%
\right) ,  \notag
\end{eqnarray}%
and where%
\begin{equation}
\underset{dp\times 1}{\underline{Y}_{t}}=\left( 
\begin{array}{c}
Y_{t} \\ 
Y_{t-1} \\ 
\vdots \\ 
Y_{t-p{\LARGE +}1}%
\end{array}%
\right) \text{, and}\underset{Kp\times 1}{\underline{F}_{t}}=\left( 
\begin{array}{c}
F_{t} \\ 
F_{t-1} \\ 
\vdots \\ 
F_{t-p{\LARGE +}1}%
\end{array}%
\right) \text{. }  \label{Yunderscore and Funderscore}
\end{equation}%
Another useful representation of the FAVAR model is the so-called companion
form, wherein the $p^{th}$-order model given in expression (\ref{FAVAR}) is
written in terms of a first-order model:%
\begin{equation*}
\underset{\left( d+K\right) p\times 1}{\underline{W}_{t}}=\alpha +A%
\underline{W}_{t-1}+E_{t}\text{,}
\end{equation*}%
where $\underline{W}_{t}=\left( 
\begin{array}{ccccc}
W_{t}^{\prime } & W_{t-1}^{\prime } & \cdots & W_{t-p{\LARGE +}2}^{\prime }
& W_{t-p{\LARGE +}1}^{\prime }%
\end{array}%
\right) ^{\prime }$ and where%
\begin{equation}
\alpha =\left( 
\begin{array}{c}
\mu \\ 
0 \\ 
\vdots \\ 
0 \\ 
0%
\end{array}%
\right) \text{, }A=\left( 
\begin{array}{ccccc}
A_{1} & A_{2} & \cdots & A_{p-1} & A_{p} \\ 
I_{d+K} & 0 & \cdots & 0 & 0 \\ 
0 & I_{d+K} & \ddots & \vdots & 0 \\ 
\vdots & \ddots & \ddots & 0 & \vdots \\ 
0 & \cdots & 0 & I_{d+K} & 0%
\end{array}%
\right) \text{, and }E_{t}=\left( 
\begin{array}{c}
\varepsilon _{t} \\ 
0 \\ 
\vdots \\ 
0 \\ 
0%
\end{array}%
\right) \text{.}  \label{companion form notations}
\end{equation}%
This companion form is convenient for establishing certain moment conditions
on $\underline{Y}_{t}$ and $\underline{F}_{t},$ given a moment condition on $%
\varepsilon _{t},$ and for establishing certain mixing properties of the
FAVAR model, as shown in the proofs of Lemmas B-5 and Lemma B-11 given in
Appendix B.

In addition to observations on $Y_{t}$, suppose that the data set available
to researchers includes a vector of time series variables which are related
to the unobserved factors in the following manner:%
\begin{equation}
\underset{N\times 1}{Z_{t}}\text{ }=\text{ }\Gamma \underline{F}_{t}+u_{t}%
\text{, }  \label{overspecified factor model}
\end{equation}%
where the properties of $u_{t}$ are given in Assumptions 2-3 and 2-4, below.
Now, assume that not all components of $Z_{t}$ provide useful information
for estimating the unobserved vector, $\underline{F}_{t}$, so that the $%
N\times Kp$ parameter matrix $\Gamma $ may have some rows whose elements are
all zero. More precisely, let the $1\times Kp$ vector, $\gamma _{i}^{\prime
},$ denote the $i^{th}$ row of $\Gamma $, and assume that the rows of the
matrix $\Gamma $ can be divided into two classes:%
\begin{eqnarray}
H &=&\left\{ k\in \left\{ 1,....,N\right\} :\gamma _{k}=0\right\} \text{ and}
\label{H} \\
H^{c} &=&\left\{ k\in \left\{ 1,....,N\right\} :\gamma _{k}\neq 0\right\} 
\text{.}  \label{Hc}
\end{eqnarray}%
In this case, there exists a permutation matrix $\mathcal{P}$ such that $%
\mathcal{P}Z_{t}=\left( 
\begin{array}{cc}
Z_{t}^{\left( 1\right) \prime } & Z_{t}^{\left( 2\right) \prime }%
\end{array}%
\right) ^{\prime },$ where%
\begin{eqnarray}
\underset{N_{1}\times 1}{Z_{t}^{\left( 1\right) }} &=&\Gamma _{1}\underline{F%
}_{t}+u_{t}^{\left( 1\right) }  \label{Z(1)} \\
\underset{N_{2}\times 1}{Z_{t}^{\left( 2\right) }} &=&u_{t}^{\left( 2\right)
}\text{.}  \label{Z(2)}
\end{eqnarray}%
The above representation suggests that the components of $Z_{t}^{\left(
1\right) }$ can be interpreted as some sort of \textquotedblleft
information\textquotedblright\ variables, as the information that they
supply will be helpful in estimating $\underline{F}_{t}$. On the other hand,
for the purpose of factor estimation, the components of the subvector $%
Z_{t}^{\left( 2\right) }$ are pure \textquotedblleft
noise\textquotedblright\ variables, as they do not load on the underlying
factors and only add noise if they are included in the factor estimation
process. An empirical researcher will often not have prior knowledge as to
which variables are elements of $Z_{t}^{\left( 1\right) }$ and which are
elements of $Z_{t}^{\left( 2\right) }$. \ This underscores the potential
usefulness for a variable selection procedure which will allow us to
properly identify the components of of $Z_{t}^{\left( 1\right) }$ and to use
only these variables when we try to estimate $\underline{F}_{t}$. If we
unknowingly include too many components of $Z_{t}^{\left( 2\right) }$ in the
estimation process, then inconsistency in factor estimation can result, as
shown in Theorem 2.1 of Chao, Qiu, and Swanson (2023).\footnote{%
Chao, Qiu, and Swanson (2023) is a not-for-publication working paper and can
be found at http://econweb.umd.edu/\symbol{126}chao/Research/research.html
Note that some of the results in the current paper draw on results contained
in Chao, Qiu, and Swanson (2023).}

To provide a variable selection procedure with provable guarantees, we must
first specify a number of conditions on the FAVAR model defined above.

\noindent \textbf{Assumption 2-1: }Suppose that:%
\begin{equation}
\det \left\{ I_{\left( d+K\right) }-A_{1}z-\cdot \cdot \cdot
-A_{p}z^{p}\right\} =0,\text{ implies that }\left\vert z\right\vert >1\text{.%
}  \label{stability cond}
\end{equation}

\noindent \textbf{Assumption 2-2: }Let $\varepsilon _{t}$ satisfy the
following set of conditions: (a) $\left\{ \varepsilon _{t}\right\} $ is an
independent sequence of random vectors with $E\left[ \varepsilon _{t}\right]
=0$ $\forall t$; (b) there exists a positive constant $C$ such that $%
\sup_{t}E\left\Vert \varepsilon _{t}\right\Vert _{2}^{6}\leq C<\infty $; (c) 
$\varepsilon _{t}$ admits a density $g_{\varepsilon _{t}}$ such that, for
some positive constant $M<\infty ,\sup_{t}\dint \left\vert g_{\varepsilon
_{t}}\left( \upsilon -u\right) -g_{\varepsilon _{t}}\left( \upsilon \right)
\right\vert d\varepsilon \leq M\left\vert u\right\vert $, whenever $%
\left\vert u\right\vert \leq \overline{\kappa }$ for some constant $%
\overline{\kappa }>0$; and (d) there exists a constant $\underline{C}>0$
such that $\inf_{t}\lambda _{\min }\left\{ E\left[ \varepsilon
_{t}\varepsilon _{t}^{\prime }\right] \right\} \geq \underline{C}>0$.

\noindent \textbf{Assumption 2-3: }Let $u_{i,t}$ be the $i^{th}$ element of
the error vector $u_{t}$ in expression (\ref{overspecified factor model}),
and we assume that it satisfies the following conditions: (a) $E\left[
u_{i,t}\right] =0$ for all $i$ and $t$; (b) there exists a positive constant 
$\overline{C}$ such that $\sup_{i,t}E\left\vert u_{i,t}\right\vert ^{7}\leq 
\overline{C}<\infty $, and there exists a constant $\underline{C}>0$ such
that $\inf_{i,t}E\left[ u_{i,t}^{2}\right] \geq \underline{C}$; (c) define $%
\mathcal{F}_{i,-\infty }^{t}=\sigma \left(
....,u_{i,t-2},u_{i,t-1},u_{t}\right) $, $\mathcal{F}_{i,t+m}^{\infty
}=\sigma \left( u_{i,t+m},u_{i,t+m+1},u_{i,t+m+2},....\right) $, and 
\begin{equation*}
\beta _{i}\left( m\right) =\sup_{t}E\left[ \sup \left\{ \left\vert P\left( B|%
\mathcal{F}_{i,-\infty }^{t}\right) -P\left( B\right) \right\vert :B\in 
\mathcal{F}_{i,t+m}^{\infty }\right\} \right] \text{.}
\end{equation*}%
Assume that there exist constants $a_{1}>0$ and $a_{2}>0$ such that%
\begin{equation*}
\beta _{i}\left( m\right) \leq a_{1}\exp \left\{ -a_{2}m\right\} ,\text{ for
all }i\text{;}
\end{equation*}%
and (d) there exists a positive constant $C$ such that $\sup_{t}\left( \frac{%
1}{N_{1}}\dsum\limits_{i\in H^{{\large c}}}\dsum\limits_{k\in H^{{\large c}%
}}\left\vert E\left[ u_{i,t}u_{k,t}\right] \right\vert \right) \leq C<\infty 
$ for every positive integer $N_{1}$, where $H^{{\large c}}$ is defined in
expression (\ref{Hc}) above.

\noindent \textbf{Assumption 2-4: }$\varepsilon _{t}$ and $u_{i,s}$ are
independent, for all $i,t,$ and $s$.

\noindent \textbf{Assumption 2-5: }There exists a positive constant $%
\overline{C},$ such that $\sup_{i\in H^{{\large c}}}\left\Vert \gamma
_{i}\right\Vert _{2}\leq \overline{C}<\infty $ and $\left\Vert \mu
\right\Vert _{2}\leq \overline{C}<\infty $, where $\mu =\left( \mu
_{Y}^{\prime },\mu _{F}^{\prime }\right) ^{\prime }$.

\noindent \textbf{Assumption 2-6: }There exists a positive constant $%
\overline{C},$ such that:%
\begin{equation*}
0<\frac{1}{\overline{C}}\leq \lambda _{\min }\left( \frac{\Gamma ^{\prime
}\Gamma }{N_{1}}\right) \leq \lambda _{\max }\left( \frac{\Gamma ^{\prime
}\Gamma }{N_{1}}\right) \leq \overline{C}<\infty \text{ for all }N_{1}\text{%
, }N_{2}\text{ sufficiently large,}
\end{equation*}%
where $N_{1}$ is the number of components of the subvector $Z_{t}^{\left(
1\right) }$ (i.e., the number of relevant variables) and $N_{2}$ is the
number of components of the subvector $Z_{t}^{\left( 2\right) }$ (i.e., the
number of irrelevant variables), as previously defined in expressions (\ref%
{Z(1)}) and (\ref{Z(2)}).

\noindent \textbf{Assumption 2-7: }Let $A$ be as defined in expression (\ref%
{companion form notations}) above, and let the eigenvalues of the matrix $%
I_{\left( d+K\right) p}-A$ be sorted so that:%
\begin{equation*}
\left\vert \lambda _{\left( 1\right) }\left( I_{\left( d+K\right)
p}-A\right) \right\vert \geq \left\vert \lambda _{\left( 2\right) }\left(
I_{\left( d+K\right) p}-A\right) \right\vert \geq \cdot \cdot \cdot \geq
\left\vert \lambda _{\left( \left( d+K\right) p\right) }\left( I_{\left(
d+K\right) p}-A\right) \right\vert =\overline{\phi }_{\min }\text{.}
\end{equation*}%
Suppose that there is a constant $\underline{C}>0$ such that%
\begin{equation}
\sigma _{\min }\left( I_{\left( d+K\right) p}-A\right) \geq \underline{C}%
\overline{\phi }_{\min }  \label{lower bd I-A}
\end{equation}%
In addition, there exists a positive constant $\overline{C}<\infty $ such
that, for all positive integer $j$, 
\begin{equation}
\sigma _{\max }\left( A^{j}\right) \leq \overline{C}\max \left\{ \left\vert
\lambda _{\max }\left( A^{j}\right) \right\vert ,\left\vert \lambda _{\min
}\left( A^{j}\right) \right\vert \right\} .  \label{upper bd A}
\end{equation}

\noindent \qquad Note that Assumption 2-1 is the stability condition that
one typically assumes for a stationary VAR process. One difference is that
we allow for possible heterogeneity in the distribution of $\varepsilon _{t}$
across time, so that our FAVAR process is not necessarily a strictly
stationary process. Under Assumption 2-1, there exists a vector moving
average representation for the FAVAR process. Note also that it is well
known that $\det \left\{ I_{\left( d+K\right) }-Az\right\} =\det \left\{
I_{\left( d+K\right) }-A_{1}z-\cdot \cdot \cdot -A_{p}z^{p}\right\} ,$ where 
$A$ is the coefficient matrix of the companion form given in expression (\ref%
{companion form notations}). See, for example, page 16 of L\"{u}tkepohl
(2005). It follows that Assumption 2-1 is equivalent to the condition that $%
\det \left\{ I_{\left( d+K\right) }-Az\right\} =0$ implies that $\left\vert
z\right\vert >1.$ In addition, Assumption 2-1 is equivalent to the
assumption that all eigenvalues of $A$ have modulus less than $1$. Since the
factor loading matrix $\Gamma $ is an $N\times Kp$ matrix, where $%
N=N_{1}+N_{2}$, the matrix $\Gamma ^{\prime }\Gamma $ will have order of
magnitude equal to $N$ if the factors are pervasive. Assumption 2-6 allows
for possible violations of this conventional pervasiveness assumption, which
will occur in our setup when $N_{1}/N\rightarrow 0$. Assumption 2-7 imposes
a condition whereby the extreme singular values of the matrices $A^{j}$ and $%
I_{\left( d+K\right) p}-A$ have bounds that depend on the extreme
eigenvalues of these matrices. More primitive conditions for such a
relationship between the singular values and the eigenvalues of a (not
necessarily symmetric) matrix have been studied in the linear algebra
literature. In Appendix B of this paper, we prove one such result which
extends a well-known result by Ruhe (1975). More specifically, we state and
prove the following lemma:

\noindent \textbf{Lemma 2-1}: \textit{Let }$A$\textit{\ be an }$n\times n$%
\textit{\ square matrix with (ordered) singular values given by }$\sigma
_{\left( 1\right) }\left( A\right) \geq \sigma _{\left( 2\right) }\left(
A\right) \geq \cdot \cdot \cdot \geq \sigma _{\left( n\right) }\left(
A\right) \geq 0$.\textit{\ Suppose that }$A$\textit{\ is diagonalizable,
i.e., }$A=S\Lambda S^{-1},$\textit{\ where }$\Lambda $\textit{\ is diagonal
matrix whose diagonal elements are the eigenvalues of }$A$\textit{. Let the
modulus of these eigenvalues be ordered as follows: }$\left\vert \lambda
_{\left( 1\right) }\left( A\right) \right\vert \geq \left\vert \lambda
_{\left( 2\right) }\left( A\right) \right\vert \geq \cdot \cdot \cdot \geq
\left\vert \lambda _{\left( n\right) }\left( A\right) \right\vert $.\textit{%
\ Then, for }$k\in \left\{ 1,...,n\right\} $\textit{\ and for any positive
integer }$j$\textit{, we have that:}%
\begin{equation*}
\chi \left( S\right) ^{-1}\left\vert \lambda _{\left( k\right) }\left(
A^{j}\right) \right\vert \leq \sigma _{\left( k\right) }\left( A^{j}\right)
\leq \chi \left( S\right) \left\vert \lambda _{\left( k\right) }\left(
A^{j}\right) \right\vert \text{ }
\end{equation*}%
\textit{where}%
\begin{equation*}
\chi \left( S\right) =\sigma _{\left( 1\right) }\left( S\right) \sigma
_{\left( 1\right) }\left( S^{-1}\right) \text{.\footnote{%
For expositional purposes, we have restated Lemma B-9 of the Supplementary
Materials as Lemma 2-1 here. Hence, for a proof of Lemma 2-1, please see the
proof of Lemma B-9 in Appendix B of the Supplementary Materials. }}
\end{equation*}

\smallskip

\noindent Note that in the special case where the matrices $A$ and $%
I_{\left( d+K\right) p}-A$ are diagonalizable, the inequalities given in
expressions (\ref{lower bd I-A}) and (\ref{upper bd A}) are a direct
consequence of this lemma. On the other hand, Assumption 2-7 takes into
account other situations where expressions (\ref{lower bd I-A}) and (\ref%
{upper bd A}) are valid even though the matrices $A$ and $I_{\left(
d+K\right) p}-A$ are not diagonalizable.

\noindent \qquad Assumptions 2-1, 2-2(a)-(c), and 2-7 together allow us to
show in Lemma B-11 of Appendix B that the process $\left\{ W_{t}\right\} $
generated by the FAVAR model given in expression (\ref{FAVAR}) is a $\beta $%
-mixing process with $\beta $-mixing coefficient satisfying:%
\begin{equation*}
\beta _{W}\left( m\right) \leq a_{1}\exp \left\{ -a_{2}m\right\} ,
\end{equation*}%
for some positive constants $a_{1}$ and $a_{2}$, with 
\begin{equation*}
\beta _{W}\left( m\right) =\sup_{t}E\left[ \sup \left\{ \left\vert P\left( B|%
\mathcal{A}_{-\infty }^{t}\right) -P\left( B\right) \right\vert :B\in 
\mathcal{A}_{t+m}^{\infty }\right\} \right] ,
\end{equation*}%
and with $\mathcal{A}_{-\infty }^{t}=\sigma \left(
...,W_{t-2},W_{t-1},W_{t}\right) $ and $\mathcal{A}_{t+m}^{\infty }=\sigma
\left( W_{t+m},W_{t+m+1},W_{t+m+2},....\right) $. Note that Assumption 2-2
(c) rules out situations such as that given in the famous counterexample
presented by Andrews (1984) which shows that a first-order autoregression
with errors having a discrete Bernoulli distribution is not $\alpha $%
-mixing, even if it satisfies the stability condition. Conditions similar to
Assumption 2-2(c) have also appeared in previous papers, such as Gorodetskii
(1977) and Pham and Tran (1985), which seek to provide sufficient conditions
for establishing the $\alpha $ or $\beta $ mixing properties of linear time
series processes.

Our variable selection procedure is based on a self-normalized statistic and
makes use of some pathbreaking moderate deviation results for weakly
dependent processes recently obtained by Chen, Shao, Wu, and Xu (2016). An
advantage of using a self-normalized statistic, as we will discuss a bit
more following expression (\ref{var selection decision rule}) below, is that
it allows the range of the moderate deviation approximation to be wider
relative to their non-self-normalized counterparts\textbf{.} To accommodate
data dependence, we consider self-nomalized statistics that are constructed
from observations which are first split into blocks in a manner similar to
the kind of construction one would employ in implementing a block bootstrap
or in proving a central limit theorem using the blocking technique. Two such
statistics are proposed in this paper. The first of these statistics has the
form of an $\ell _{\infty }$ norm and is given by: 
\begin{equation}
\max_{1\leq \ell \leq d}\left\vert S_{i,\ell ,T}\right\vert =\max_{1\leq
\ell \leq d}\left\vert \frac{\overline{S}_{i,\ell ,T}}{\sqrt{\overline{V}%
_{i,\ell ,T}}}\right\vert ,  \label{max statistic}
\end{equation}%
where 
\begin{eqnarray}
\overline{S}_{i,\ell ,T} &=&\dsum\limits_{r=1}^{q}\dsum\limits_{t=\left(
r-1\right) \tau +p}^{\left( r-1\right) \tau +\tau _{1}+p-1}Z_{it}y_{\ell ,t%
{\LARGE +}1}\text{ and}  \label{numerator max stat} \\
\overline{V}_{i,\ell ,T} &=&\dsum\limits_{r=1}^{q}\left[ \dsum\limits_{t=%
\left( r-1\right) \tau +p}^{\left( r-1\right) \tau +\tau
_{1}+p-1}Z_{it}y_{\ell ,t{\LARGE +}1}\right] ^{2}\text{.}
\label{denominator max stat}
\end{eqnarray}%
Here, $Z_{it}$ denotes the $i^{th}$ component of $Z_{t}$ , $y_{\ell ,t+1}$
denotes the $\ell ^{th}$ component of $Y_{t+1}$, $\tau _{1}=\left\lfloor
T_{0}^{\alpha _{{\large 1}}}\right\rfloor $, and $\tau _{2}=\left\lfloor
T_{0}^{\alpha _{{\large 2}}}\right\rfloor $, where $1>\alpha _{1}\geq \alpha
_{2}>0$, $\tau =\tau _{1}+\tau _{2}$, $q=\left\lfloor T_{0}/\tau
\right\rfloor $, and $T_{0}=T-p+1$. Note that the statistic given in
expression (\ref{max statistic}) can be interpreted as the maximum of the
(self-normalized) sample covariances between the $i^{th}$ component of $%
Z_{t} $ and the components of $Y_{t+1}$. Our second statistic has the form
of a pseudo-$L_{1}$ norm and is given by: 
\begin{equation*}
\dsum\limits_{\ell =1}^{d}\varpi _{\ell }\left\vert S_{i,\ell ,T}\right\vert
=\dsum\limits_{\ell =1}^{d}\varpi _{\ell }\left\vert \frac{\overline{S}%
_{i,\ell ,T}}{\sqrt{\overline{V}_{i,\ell ,T}}}\right\vert ,
\end{equation*}%
where $\overline{S}_{i,\ell ,T}$ and $\overline{V}_{i,\ell ,T}$ are as
defined in expressions (\ref{numerator max stat}) and (\ref{denominator max
stat}) above and where $\left\{ \varpi _{\ell }:\ell =1,..,d\right\} $
denotes pre-specified weights, such that $\varpi _{\ell }\geq 0,$ for every $%
\ell \in \left\{ 1,...,d\right\} $, and $\dsum\nolimits_{\ell =1}^{d}\varpi
_{\ell }=1$. Both of these statistics employ a blocking scheme similar to
that proposed in Chen, Shao, Wu, and Xu (2016), where, in order to keep the
effects of dependence under control, the construction of these statistics is
based only on observations in every other block. To see this, note that if
we write out the \textquotedblleft numerator\textquotedblright\ term $%
\overline{S}_{i,\ell ,T}$ in greater detail, we have that:%
\begin{eqnarray}
\overline{S}_{i,\ell ,T} &=&\dsum\limits_{t=p}^{\tau _{1}+p-1}Z_{it}y_{\ell
,t{\LARGE +}1}+\dsum\limits_{t=\tau +p}^{\tau +\tau _{1}+p-1}Z_{it}y_{\ell ,t%
{\LARGE +}1}  \notag \\
&&+\dsum\limits_{t=2\tau +p}^{2\tau +\tau _{1}+p-1}Z_{it}y_{\ell ,t{\LARGE +}%
1}+\cdot \cdot \cdot +\dsum\limits_{t=\left( q-1\right) \tau +p}^{\left(
q-1\right) \tau +\tau _{1}+p-1}Z_{it}y_{\ell ,t{\LARGE +}1}
\label{sum of Z times y}
\end{eqnarray}%
Comparing the first term and the second terms on the right-hand side of
expression (\ref{sum of Z times y}), we see that the observations $%
Z_{it}y_{\ell ,t{\LARGE +}1}$, for $t=\tau _{1}+p,...,\tau +p-1$, have not
been included in the construction of the sum. Similar observations hold when
comparing the second and the third terms, and so on.

It should also be pointed out that although Chen, Shao, Wu, and Xu \textbf{(}%
2016\textbf{)} focus their analysis on problems of testing and inference for
the mean of a scalar weakly dependent time series using self-normalized
Student-type test statistics, our paper applies the self-normalization
approach to a variable selection problem in a FAVAR setting. Namely, the
problem which we study is more akin to a classification (or model selection)
problem rather than a multiple hypothesis testing problem. In order to
consistently estimate the factors up to an invertible matrix transformation,
we develop a variable selection procedure whereby both the probability of a
false positive and the probability of a false negative converge to zero as $%
N_{1}$, $N_{2}$, $T\rightarrow \infty $\footnote{%
Here, a false positive refers to mis-classifying a variable, $Z_{it},$ as a
relevant variable for the purpose of factor estimation when its factor
loading $\gamma _{i}^{\prime }=0$, whereas a false negative refers to the
opposite case, where $\gamma _{i}^{\prime }\neq 0,$ but the variable $Z_{it}$
is mistakenly classified as irrelevant.}. This is different from the typical
multiple hypothesis testing approach whereby one tries to control the
familywise error rate (or, alternatively, the false discovery rate), so that
it is no greater than $0.05,$ say, but does not try to ensure that this
probability goes to zero as the sample size grows.

To determine whether the $i^{th}$ component of $Z_{t}$ is a relevant
variable for the purpose of factor estimation, we propose the following
procedure. Define $i\in \widehat{H}^{c}$ to indicate that the procedure has
classified $Z_{it}$ to be a relevant variable for the purpose of factor
estimation. Similarly, define $i\in \widehat{H}$ to indicate that the
procedure has classified $Z_{it}$ to be an irrelevant variable. Now, let $%
\mathbb{S}_{i,T}^{+}$ denote either the statistic $\max_{1\leq \ell \leq
d}\left\vert S_{i,\ell ,T}\right\vert $ or the statistic $%
\dsum\nolimits_{\ell =1}^{d}\varpi _{\ell }\left\vert S_{i,\ell
,T}\right\vert $. Our variable selection procedure is based on the decision
rule: 
\begin{equation}
i\in \left\{ 
\begin{array}{cc}
\widehat{H}^{c} & \text{ if }\mathbb{S}_{i,T}^{+}\geq \Phi ^{-1}\left( 1-%
\frac{\varphi }{2N}\right) \\ 
\widehat{H} & \text{if }\mathbb{S}_{i,T}^{+}<\Phi ^{-1}\left( 1-\frac{%
\varphi }{2N}\right)%
\end{array}%
\right. ,  \label{var selection decision rule}
\end{equation}%
where $\Phi ^{-1}\left( \cdot \right) $ denotes the quantile function or the
inverse of the cumulative distribution function of the standard normal
random variable, and where $\varphi $ is a tuning parameter which may depend
on $N$. Some conditions on $\varphi $ will be given in Assumptions 2-11 and
2-11* below.

To understand why using the quantile function of the standard normal as the
threshold function for our procedure is a natural choice, note first that,
by a slight modification of the arguments given in the proof of Lemma B-17
in Appendix B of the Technical Appendix, we can show that, as $T\rightarrow
\infty $%
\begin{equation}
P\left( \left\vert S_{i,\ell ,T}\right\vert \geq z\right) =2\left[ 1-\Phi
\left( z\right) \right] \left( 1+o\left( 1\right) \right) ,
\label{moderate dev result}
\end{equation}%
which holds for all $i$ and $\ell $ and for all $z$ such that

\noindent $0\leq z\leq c_{0}\min \left\{ T^{\left( 1-{\large \alpha }%
_{1}\right) /6}/L\left( T\right) ,T^{\alpha _{2}/2}\right\} $, where $%
L\left( T\right) $ denotes a slowly varying function such that $L\left(
T\right) \rightarrow \infty $ but $L\left( T\right) /T^{\left( 1-{\large %
\alpha }_{1}\right) /6}\rightarrow 0$ as $T\rightarrow \infty $. In view of
expression (\ref{moderate dev result}), we can interpret moderate deviation
as providing an asymptotic approximation of the (two-sided) tail behavior of
the self-normalized statistic, $S_{i,\ell ,T},$ based on the tails of the
standard normal distribution. An important advantage of using
self-normalized statistics in this context is that the range for which this
standard normal approximation is valid (i.e., the range $0\leq z\leq
c_{0}\min \left\{ T^{\left( 1-{\large \alpha }_{1}\right) /6}/L\left(
T\right) ,T^{\alpha _{2}/2}\right\} $) is wider for self-normalized
statistics relative to their non-self-normalized counterparts. Now, suppose
initially that we wish simply to control the probability of a Type I error
for testing the null hypothesis $H_{0}:\gamma _{i}=0$ (i.e., the $i^{th}$
variable does not load on the underlying factors) at some fixed significance
level $\alpha $. Then, expression (\ref{moderate dev result}) suggests that
a natural way to do this is to set $z=\Phi ^{-1}\left( 1-\alpha /2\right) $.
This is because, given that the quantile function $\Phi ^{-1}\left( \cdot
\right) $ is, by definition, the inverse function of the cdf $\Phi \left(
\cdot \right) $, we have that: 
\begin{equation*}
P\left( \left\vert S_{i,\ell ,T}\right\vert \geq \Phi ^{-1}\left( 1-\alpha
/2\right) \right) =2\left[ 1-\Phi \left( \Phi ^{-1}\left( 1-\alpha /2\right)
\right) \right] \left( 1+o\left( 1\right) \right) =\alpha \left( 1+o\left(
1\right) \right) ,
\end{equation*}%
so that the probability of a Type I error is controlled at the desired level 
$\alpha $ asymptotically. Note also that an advantage of moderate deviation
theory is that it gives a characterization of the relative approximation
error, as opposed to the absolute approximation error. As a result, the
approximation given is useful and meaningful even when $\alpha $ is very
small, which is of importance to us since we are interested in situations
where we might want to let $\alpha $ go to zero, as sample size approaches
infinity.

The above example provides intuition concerning the form of the threshold
function that we have specified. The variable selection problem that we
actually consider is more complicated however, since we need to control the
probability of a Type I error (or of a false positive) not just for a single
test involving the $i^{th}$ variable but for a multiple hypothesis testing
scenario involving the loading coefficient vectors for all variables $Z_{it}$
(for $i=1,...,N$). Moreover, as noted previously, we want also to design a
procedure where the probability of a false positive will go asymptotically
to zero as well. We show in Theorem 1 below that these objectives can all be
accomplished using the threshold function specified in expression (\ref{var
selection decision rule}).\footnote{%
The threshold function used here is reminiscent of the one employed in a
celebrated paper by Belloni, Chen, Chernozhukov, and Hansen (2012). More
specifically, Belloni, Chen, Chernozhukov, and Hansen (2012) use a similar
threshold function to help set the penalty level for Lasso estimation of the
first-stage equation of an IV regression model assuming $i.n.i.d$. data. In
spite of the similarity in the form of the threshold function, the problem
studied in that paper is very different from the one which we analyze here.
In consequence, the conditions we specify for setting the tuning parameter $%
\varphi $ will also be quite different from what they recommend in their
paper.}

Indeed, under appropriate conditions, the variable selection procedure
described above can be shown to be completely consistent, in the sense that
both the probability of a false positive, i.e. $P\left( i\in \widehat{H}%
^{c}|i\in H\right) $, and the probability of a false negative, i.e., $%
P\left( i\in \widehat{H}|i\in H^{c}\right) $, approach zero as $N$, $%
T\rightarrow \infty $. To show this result, we must first state a number of
additional assumptions.

\noindent \textbf{Assumption 2-8: }There exists a positive constant, $%
\underline{c},$ such that for $T$ sufficiently large: 
\begin{equation*}
\min_{1\leq \ell \leq d}\min_{i\in H}\min_{r\in \left\{ 1,...,q\right\}
}E\left\{ \left[ \frac{1}{\sqrt{\tau _{1}}}\dsum\limits_{t=\left( r-1\right)
\tau +p}^{\left( r-1\right) \tau +\tau _{1}+p-1}y_{\ell ,t{\LARGE +}1}u_{it}%
\right] ^{2}\right\} \geq \underline{c},
\end{equation*}%
where, as defined earlier,%
\begin{equation*}
\tau _{1}=\left\lfloor T_{0}^{\alpha _{{\large 1}}}\right\rfloor \text{, }%
\tau _{2}=\left\lfloor T_{0}^{\alpha _{{\large 2}}}\right\rfloor \text{ for }%
1>\alpha _{1}\geq \alpha _{2}>0\text{ and }q=\left\lfloor \frac{T_{0}}{\tau
_{1}+\tau _{2}}\right\rfloor ,
\end{equation*}%
and $T_{0}=T-p+1$.

\noindent \textbf{Assumption 2-9: }Let $i\in H^{c}=\left\{ k\in \left\{
1,....,N\right\} :\gamma _{k}\neq 0\right\} $. Suppose that there exists a
positive constant, $\underline{c},$ such that, for all $N_{1},N_{2},$and $T$
sufficiently large:

\begin{eqnarray*}
&&\min_{1\leq \ell \leq d}\min_{{\large i\in }H^{{\large c}}}\left\vert 
\frac{\mu _{i,\ell ,T}}{q\tau _{1}}\right\vert \\
&=&\min_{1\leq \ell \leq d}\min_{{\large i\in }H^{{\large c}}}\left\vert 
\frac{1}{q}\dsum\limits_{r=1}^{q}\frac{1}{\tau _{1}}\dsum\limits_{t=\left(
r-1\right) \tau +p}^{\left( r-1\right) \tau +\tau _{1}+p-1}\gamma
_{i}^{\prime }\left\{ E\left[ \underline{F}_{t}\right] \mu _{Y,\ell }+E\left[
\underline{F}_{t}\underline{Y}_{t}^{\prime }\right] \alpha _{YY,\ell }+E%
\left[ \underline{F}_{t}\underline{F}_{t}^{\prime }\right] \alpha _{YF,\ell
}\right\} \right\vert \\
&\geq &\underline{c}>0,
\end{eqnarray*}%
where $\mu _{Y,\ell }=e_{\ell ,d}^{\prime }\mu _{Y}$, $\alpha _{YY,\ell
}=A_{YY}^{\prime }e_{\ell ,d}$, and $\alpha _{YF,\ell }=A_{YF}^{\prime
}e_{\ell ,d}.$ Here, $e_{\ell ,d}$ is a $d\times 1$ elementary vector whose $%
\ell ^{th}$ component is $1$ and all other components are $0$.

\noindent \textbf{Assumption 2-10: }Suppose that, as $N_{1}$, $N_{2}$, and $%
T\rightarrow \infty $, the following rate conditions hold: $\frac{\sqrt{\ln N%
}}{T^{\min \left\{ \frac{{\large 1-\alpha }_{{\large 1}}}{{\large 6}},\frac{%
{\Large \alpha }_{{\large 2}}}{{\large 2}}\right\} }}\rightarrow 0,$ where
(a) $1>\alpha _{1}\geq \alpha _{2}>0$ and $N=N_{1}+N_{2}$, and (b) $N_{1}%
{\Large /}T^{3\alpha _{{\large 1}}}\rightarrow 0$ where $1>\alpha _{1}>0$.

\noindent \textbf{Assumption 2-11: }Let $\varphi $ satisfy the following two
conditions: (a) $\varphi \rightarrow 0$ as $N_{1},N_{2}\rightarrow \infty $,
and (b) there exists some constant $a>0,$ such that $\varphi \geq 1{\Large /}%
N^{a},$ for all $N_{1},N_{2}$ sufficiently large. Assumption 2-8 rules out
certain degenerate situations where as $T\rightarrow \infty $ 
\begin{equation*}
E\left\{ \left[ \frac{1}{\sqrt{\tau _{1}}}\dsum\limits_{t=\left( r-1\right)
\tau +p}^{\left( r-1\right) \tau +\tau _{1}+p-1}y_{\ell ,t{\LARGE +}1}u_{it}%
\right] ^{2}\right\} \rightarrow 0\text{ }
\end{equation*}%
for some $1\leq \ell \leq d$, $i\in H$, and $r\in \left\{ 1,..,q\right\} $,
since a moderate deviation result and, in fact, a central limit theorem,
would not hold in general if such degeneracies were to occur. A similar
condition is also assumed in Chen, Shao, Wu, and Xu \textbf{(}2016\textbf{)}%
. See condition (4.2) on page 1600 of that paper.

To give an intuitive interpretation for Assumption 2-9, note that the term%
\begin{equation*}
\frac{\mu _{i,\ell ,T}}{q\tau _{1}}=\frac{1}{q}\dsum\limits_{r=1}^{q}\frac{1%
}{\tau _{1}}\dsum\limits_{t=\left( r-1\right) \tau +p}^{\left( r-1\right)
\tau +\tau _{1}+p-1}\gamma _{i}^{\prime }\left\{ E\left[ \underline{F}_{t}%
\right] \mu _{Y,\ell }+E\left[ \underline{F}_{t}\underline{Y}_{t}^{\prime }%
\right] \alpha _{YY,\ell }+E\left[ \underline{F}_{t}\underline{F}%
_{t}^{\prime }\right] \alpha _{YF,\ell }\right\}
\end{equation*}%
is, in fact, a noncentrality parameter for the variable-selection{\large /}%
multiple-hypothesis-testing problem considered here. This condition allows
us to differentiate between the null hypothesis, $i\in H$ (where $\gamma
_{i}=0$) from the alternative hypothesis $i\in H^{c}$ (where $\gamma
_{i}\neq 0$) so that, under this condition, it will be possible to design
procedures, such as the one proposed here, which will have asymptotic power.
To see this more clearly, note that if $i\in H$, then it is clear that:%
\begin{equation*}
\frac{\mu _{i,\ell ,T}}{q\tau _{1}}=\frac{1}{q}\dsum\limits_{r=1}^{q}\frac{1%
}{\tau _{1}}\dsum\limits_{t=\left( r-1\right) \tau +p}^{\left( r-1\right)
\tau +\tau _{1}+p-1}\gamma _{i}^{\prime }\left\{ E\left[ \underline{F}_{t}%
\right] \mu _{Y,\ell }+E\left[ \underline{F}_{t}\underline{Y}_{t}^{\prime }%
\right] \alpha _{YY,\ell }+E\left[ \underline{F}_{t}\underline{F}%
_{t}^{\prime }\right] \alpha _{YF,\ell }\right\} =0,
\end{equation*}%
given that $\gamma _{i}=0$ in this case. On the other hand, under the
alternative hypothesis where $i\in H^{c}$, we will have $\gamma _{i}\neq 0$
so that $\mu _{i,\ell ,T}{\Large /}\left( q\tau _{1}\right) \neq 0$ under
Assumption 2-9. Now, we believe Assumption 2-9 is a fairly mild condition to
be placed on a FAVAR since, given the interconnectedness of a FAVAR, it is
unlikely to have a situation where%
\begin{equation*}
\frac{\mu _{i,\ell ,T}}{q\tau _{1}}=\frac{1}{q}\dsum\limits_{r=1}^{q}\frac{1%
}{\tau _{1}}\dsum\limits_{t=\left( r-1\right) \tau +p}^{\left( r-1\right)
\tau +\tau _{1}+p-1}\gamma _{i}^{\prime }\left\{ E\left[ \underline{F}_{t}%
\right] \mu _{Y,\ell }+E\left[ \underline{F}_{t}\underline{Y}_{t}^{\prime }%
\right] \alpha _{YY,\ell }+E\left[ \underline{F}_{t}\underline{F}%
_{t}^{\prime }\right] \alpha _{YF,\ell }\right\} =0
\end{equation*}%
in the case where $\gamma _{i}\neq 0$. It should be noted, of course, that
Assumption 2-9 does rule out certain specialized situations, such as the
case when $\mu _{Y,\ell }=0$, $\alpha _{YY,\ell }=0$, and $\alpha _{YF,\ell
}=0,$ for some $\ell \in \left\{ 1,...,d\right\} $. However, we do not
consider such cases to be of much practical interest since, for example, if $%
\mu _{Y,\ell }=0$, $\alpha _{YY,\ell }=0$, and $\alpha _{YF,\ell }=0$ for
some $\ell $ then expression (\ref{Y component FAVAR}) above implies that
the $\ell ^{th}$ component of $Y_{t{\LARGE +}1}$ will have the
representation 
\begin{eqnarray*}
y_{\ell ,t{\LARGE +}1} &=&\mu _{Y,\ell }+\underline{Y}_{t}^{\prime }\alpha
_{YY,\ell }+\underline{F}_{t}^{\prime }\alpha _{YF,\ell }+\varepsilon _{\ell
,t{\LARGE +}1}^{Y} \\
&=&\varepsilon _{\ell ,t{\LARGE +}1}^{Y},
\end{eqnarray*}%
so that, in this case, $y_{\ell ,t{\LARGE +}1}$ depends neither on $%
\underline{Y}_{t}=\left( Y_{t}^{\prime },Y_{t-1}^{\prime },...,Y_{t-p{\LARGE %
+}1}^{\prime }\right) ^{\prime }$ nor on

\noindent $\underline{F}_{t}=\left( F_{t}^{\prime },F_{t-1}^{\prime
},...,F_{t-p{\LARGE +}1}^{\prime }\right) $. This is, of course, an
unrealistic model for $y_{\ell ,t{\LARGE +}1}$ since it would not even be
dependent. Hence, we do not expect Assumption 2-9 to be violated except in
highly degenerate situations such as the one just described.

The following two theorems give our main theoretical results on the variable
selection procedure described above.

\noindent \textbf{Theorem 1: }\textit{Let\ }$H=\left\{ k\in \left\{
1,....,N\right\} :\gamma _{k}=0\right\} $\textit{. Suppose that Assumptions
2-1, 2-2(a)-(c), 2-3(a)-(c) 2-4, 2-5, 2-7, 2-8, 2-10 (a) and 2-11 hold. Let }%
$\Phi ^{-1}\left( \cdot \right) $\textit{\ denote the inverse of the
cumulative distribution function of the standard normal random variable, or,
alternatively, the quantile function of the standard normal distribution.
Then, the following statements are true:}

\begin{enumerate}
\item[(a)] \textit{Let }$\left\{ \varpi _{\ell }:\ell =1,..,d\right\} $%
\textit{\ be pre-specified weights, such that }$\varpi _{\ell }\geq 0,$%
\textit{\ for every }$\ell \in \left\{ 1,...,d\right\} $\textit{\ and }$%
\dsum\nolimits_{\ell =1}^{d}\varpi _{\ell }=1$\textit{, then:}%
\begin{equation*}
P\left( \max_{{\large i\in }H}\dsum\limits_{\ell =1}^{d}\varpi _{\ell
}\left\vert S_{i,\ell ,T}\right\vert \geq \Phi ^{-1}\left( 1-\frac{\varphi }{%
2N}\right) \right) =O\left( \frac{N_{2}\varphi }{N}\right) =o\left( 1\right) 
\text{,}
\end{equation*}%
\textit{where }$N=N_{1}+N_{2}$\textit{.}

\item[(b)] 
\begin{equation*}
P\left( \max_{{\large i\in }H}\max_{1\leq \ell \leq d}\left\vert S_{i,\ell
,T}\right\vert \geq \Phi ^{-1}\left( 1-\frac{\varphi }{2N}\right) \right)
=O\left( \frac{N_{2}\varphi }{N}\right) =o\left( 1\right) \text{. }
\end{equation*}
\end{enumerate}

\noindent \textbf{Theorem 2: }\textit{Let }$H^{c}=\left\{ k\in \left\{
1,....,N\right\} :\gamma _{k}\neq 0\right\} $\textit{. Suppose that
Assumptions 2-1, 2-2(a)-(c), 2-3(a)-(c), 2-5, 2-7, 2-9, 2-10, and 2-11 hold.
Then, the following statements are true.}

\begin{enumerate}
\item[(a)] \textit{Let }$\left\{ \varpi _{\ell }:\ell =1,..,d\right\} $%
\textit{\ be pre-specified weights, such that }$\varpi _{\ell }\geq 0,$%
\textit{\ for every }$\ell \in \left\{ 1,...,d\right\} $\textit{\ and }$%
\dsum\nolimits_{\ell =1}^{d}\varpi _{\ell }=1$\textit{, then: }%
\begin{equation*}
P\left( \min_{{\large i\in }H^{{\large c}}}\dsum\limits_{\ell =1}^{d}\varpi
_{\ell }\left\vert S_{i,\ell ,T}\right\vert \geq \Phi ^{-1}\left( 1-\frac{%
\varphi }{2N}\right) \right) \rightarrow 1\text{.}
\end{equation*}

\item[(b)] 
\begin{equation*}
P\left( \min_{{\large i\in }H^{{\large c}}}\max_{1\leq \ell \leq
d}\left\vert S_{i,\ell ,T}\right\vert \geq \Phi ^{-1}\left( 1-\frac{\varphi 
}{2N}\right) \right) \rightarrow 1\text{.}
\end{equation*}
\end{enumerate}

Theorem 1 shows that, under our procedure, the probability of a false
positive, i.e., the probability that $i\in \widehat{H}^{c},$ even though $%
\gamma _{i}=0$, approaches zero, as $N,T\rightarrow \infty $. Theorem 2
shows that the probability of a false negative, i.e., the probability that $%
i\in \widehat{H}$ even though $\gamma _{i}\neq 0$, also approaches zero, as $%
N,T\rightarrow \infty $. Together, these two theorems show that our variable
selection procedure is (completely) consistent in the sense that the
probability of committing a misclassification error vanishes as $%
N,T\rightarrow \infty $.

\medskip

\noindent \textbf{Remark 2.1: }

It should be noted that a special case of our FAVAR model which is of
particular interest is the case where $d=1$, i.e., the case where the $Y$\
variable is univariate. In this case, equation (\ref{Y component FAVAR})
reduces to 
\begin{equation}
y_{t+1}=\mu _{Y}+a_{YY,1}y_{t}+\cdots +a_{YY,p}y_{t-p{\LARGE +}1}+a_{YF}^{%
{\Large \prime }}\underline{F}_{t}+\varepsilon _{t+1}^{Y}
\label{factor aug autoreg}
\end{equation}%
where $\mu _{Y}$ is now a $1\times 1$ intercept parameter; $%
a_{YY,1}...,a_{YY,p}$ are the $p$ autoregressive parameters; $a_{YF}^{%
{\Large \prime }}$ is a $1\times Kp$ coefficient vector; and $\varepsilon
_{t+1}^{Y}$ is now a $1\times 1$ error term. Expression (\ref{factor aug
autoreg}), thus, yields a factor augmented autoregressive model which is
commonly used to forecast economic time series. Moreover, when $d=1$,\ it is
easy to see that our two statistics, $\max_{1\leq \ell \leq d}\left\vert
S_{i,\ell ,T}\right\vert $ and $\dsum\nolimits_{\ell =1}^{d}\varpi _{\ell
}\left\vert S_{i,\ell ,T}\right\vert $, reduce to the same one since%
\begin{equation*}
\max_{1\leq \ell \leq d}\left\vert S_{i,\ell ,T}\right\vert =\max_{1\leq
\ell \leq 1}\left\vert S_{i,\ell ,T}\right\vert
=S_{i,1,T}=\dsum\limits_{\ell =1}^{1}\varpi _{\ell }\left\vert S_{i,\ell
,T}\right\vert =\dsum\limits_{\ell =1}^{d}\varpi _{\ell }\left\vert
S_{i,\ell ,T}\right\vert \text{ }
\end{equation*}%
given that $1=\dsum\nolimits_{\ell =1}^{d}\varpi _{\ell }=\varpi _{1}$ in
this case. Hence, for the $d=1$ case, we can remove the subscript $1$ and
write 
\begin{equation}
S_{i,1,T}=S_{i,T}=\frac{\overline{S}_{i,T}}{\sqrt{\overline{V}_{i,T}}}
\label{stat in d=1 case exp 1}
\end{equation}%
where 
\begin{equation}
\overline{S}_{i,T}=\dsum\limits_{r=1}^{q}\dsum\limits_{t=\left( r-1\right)
\tau +p}^{\left( r-1\right) \tau +\tau _{1}+p-1}Z_{it}y_{t{\LARGE +}1}\text{
and }\overline{V}_{i,\ell ,T}=\dsum\limits_{r=1}^{q}\left[
\dsum\limits_{t=\left( r-1\right) \tau +p}^{\left( r-1\right) \tau +\tau
_{1}+p-1}Z_{it}y_{t{\LARGE +}1}\right] ^{2}  \label{stat in d=2 case exp 2}
\end{equation}%
so that our statistic can be viewed as a self-normalized sample covariance
constructed using blocked sums.

\noindent \textbf{Remark 2.2:}

It is also worth stressing at this point that although we interpret our
variable selection procedure primarily as a procedure which assesses the
relevance of a variable $Z_{it}$, it is clear from expressions (\ref{stat in
d=1 case exp 1}) and (\ref{stat in d=2 case exp 2}) above that our procedure
also contains information about the predictive content of the variable $%
Z_{it}$ for $y_{t{\LARGE +}1}$. There is a close association between the
relevance and the predictive content of a variable in the context of a FAVAR
model. This is true because if $\gamma _{i}=0$, i.e., if $Z_{it}$ does not
load on any of the underlying factors, so that $Z_{it}$ is an irrelevant
variable; then $Z_{it}$ will also not have predictive content for $y_{t%
{\LARGE +}1}$ because, within a FAVAR system, any possible correlation
between $Z_{it}$ and $y_{t{\LARGE +}1}$ only works its way through
indirectly via the factors. Hence, if $Z_{it}$ is not correlated with any of
the factors, then it will not have correlation with $y_{t{\LARGE +}1}$.
However, the reason why we choose to interpret our procedure as one which
primarily assesses the relevance of the variables $Z_{it}$ (for $i=1,...,N$)
is because a FAVAR system is quite complex, and one can find examples where
a $Z_{it}$ variable is specified to not have any predictive content for $y_{t%
{\LARGE +}1}$, but a variable selection procedure based on the score
statistic nevertheless rejects the null hypothesis with probability one as
sample sizes approach infinity. To see how this could be the case, consider
the following example.

\noindent \textbf{Example 1: } Consider a two-factor FAVAR model of the form:%
\begin{eqnarray}
y_{t{\LARGE +}1} &=&a_{YY}y_{t}+\alpha _{YF,1}f_{1,t}+\varepsilon _{t{\LARGE %
+}1}^{Y}  \notag \\
f_{1,t{\LARGE +}1}
&=&a_{FY,1}y_{t}+a_{FF,11}f_{1,t}+a_{FF,12}f_{2,t}+\varepsilon _{1,t{\LARGE +%
}1}^{F}  \notag \\
f_{2,t{\LARGE +}1}
&=&a_{FY,2}y_{t}+a_{FF,21}f_{1,t}+a_{FF,22}f_{2,t}+\varepsilon _{2,t{\LARGE +%
}1}^{F}\text{,}  \label{FAVAR Example}
\end{eqnarray}%
with the factor equation given by 
\begin{equation}
Z_{t}=\Gamma F_{t}+u_{t},  \label{factor eqn Example}
\end{equation}%
where $F_{t}=\left( 
\begin{array}{cc}
f_{1,t} & f_{2,t}%
\end{array}%
\right) ^{{\Large \prime }}$ and where $\alpha _{YF,1}\neq 0$. Note that,
under the specification given by expressions (\ref{FAVAR Example}) and (\ref%
{factor eqn Example}), the factor $f_{2,t}$ has no predictive content for
the target variable of interest $y_{t{\LARGE +}1},$ whereas the factor $%
f_{1,t}$ does have predictive content. Now, write the companion form:%
\begin{equation*}
W_{t{\LARGE +}1}=AW_{t}+\varepsilon _{t{\LARGE +}1},
\end{equation*}%
where%
\begin{equation*}
W_{t}=\left( 
\begin{array}{c}
y_{t} \\ 
f_{1,t} \\ 
f_{2,t}%
\end{array}%
\right) \text{, }\varepsilon _{t}=\left( 
\begin{array}{c}
\varepsilon _{t}^{Y} \\ 
\varepsilon _{1,t}^{F} \\ 
\varepsilon _{2,t}^{F}%
\end{array}%
\right) \text{, and }A=\left( 
\begin{array}{ccc}
a_{YY} & \alpha _{YF,1} & 0 \\ 
a_{FY,1} & a_{FF,11} & a_{FF,12} \\ 
a_{FY,2} & a_{FF,21} & a_{FF,22}%
\end{array}%
\right) ,
\end{equation*}%
and where we defne $\Sigma _{\varepsilon }=E\left[ \varepsilon
_{t}\varepsilon _{t}^{{\Large \prime }}\right] $ and assume that $\Sigma
_{\varepsilon }$ is positive definite. Here, under Assumption 2-1, we have
the vector moving-average representation:%
\begin{equation*}
W_{t{\LARGE +}1}=\dsum\limits_{j=0}^{\infty }A^{j}\varepsilon _{t{\LARGE +}%
1},
\end{equation*}%
It follows that the components of $W_{t{\LARGE +}1}$ have the univariate MA
representations: 
\begin{equation*}
y_{t{\LARGE +}1}=\dsum\limits_{j=0}^{\infty }e_{1}^{{\Large \prime }%
}A^{j}\varepsilon _{t{\LARGE +}1-j}\text{, }f_{1,t}=\dsum\limits_{k=0}^{%
\infty }e_{2}^{{\Large \prime }}A^{k}\varepsilon _{t-k}\text{, and }%
f_{2,t}=\dsum\limits_{k=0}^{\infty }e_{3}^{{\Large \prime }}A^{k}\varepsilon
_{t-k},
\end{equation*}%
with $e_{1}=\left( 
\begin{array}{ccc}
1 & 0 & 0%
\end{array}%
\right) ^{{\Large \prime }}$, $e_{2}=\left( 
\begin{array}{ccc}
0 & 1 & 0%
\end{array}%
\right) ^{{\Large \prime }}$, and $e_{3}=\left( 
\begin{array}{ccc}
0 & 0 & 1%
\end{array}%
\right) ^{{\Large \prime }}$. Let $Z_{it}$ and $Z_{jt}$ be, respectively,
the $i^{th}$ and the $j^{th}$ components of $Z_{t}$ (with $i\neq j$).
Suppose that $Z_{it}$ loads only on the second factor but not the first, so
that $\gamma _{i}^{{\large \prime }}=\left( 
\begin{array}{cc}
0 & \gamma _{i2}%
\end{array}%
\right) $, where $\gamma _{i2}\neq 0$;\ and suppose that $Z_{jt}$ loads only
on the first factor but not the second, so that $\gamma _{j}^{{\large \prime 
}}=\left( 
\begin{array}{cc}
\gamma _{j1} & 0%
\end{array}%
\right) $, where $\gamma _{j1}\neq 0.$ Hence, both $Z_{it}$ and $Z_{jt}$ are
relevant variables for factor estimation, but $Z_{jt}$ has predictive
content for $y_{t{\LARGE +}1}$ whereas $Z_{it}$ does not. Consider the score
statistics associated with $Z_{it}$ and $Z_{jt}:$ 
\begin{eqnarray*}
S_{i} &=&\dsum\limits_{t=1}^{T-1}Z_{it}y_{t{\LARGE +}1}=\dsum%
\limits_{t=1}^{T-1}\left( \gamma _{i}^{{\large \prime }}F_{t}+u_{it}\right)
y_{t{\LARGE +}1}=\dsum\limits_{t=1}^{T-1}\gamma _{i2}f_{2,t}y_{t{\LARGE +}%
1}+\dsum\limits_{t=1}^{T-1}u_{it}y_{t{\LARGE +}1}\text{ and} \\
S_{j} &=&\dsum\limits_{t=1}^{T-1}Z_{jt}y_{t{\LARGE +}1}=\dsum%
\limits_{t=1}^{T-1}\left( \gamma _{j}^{{\large \prime }}F_{t}+u_{jt}\right)
y_{t{\LARGE +}1}=\dsum\limits_{t=1}^{T-1}\gamma _{j1}f_{1,t}y_{t{\LARGE +}%
1}+\dsum\limits_{t=1}^{T-1}u_{jt}y_{t{\LARGE +}1}.
\end{eqnarray*}%
Now, suppose that $\sigma _{32}\neq 0$ ; (where $\sigma _{32}$ denotes the $%
\left( 3,2\right) ^{th}$ element of the error covariance matrix $\Sigma
_{\varepsilon }$); then, given that $\gamma _{j2}\neq 0$ and $\alpha
_{YF,1}\neq 0$, the expected value of $S_{i}$ will not, in general, be
properly centered at zero. That is, 
\begin{eqnarray*}
E\left[ S_{i}\right]  &=&\dsum\limits_{t=1}^{T-1}E\left[ Z_{it}y_{t{\LARGE +}%
1}\right] =\gamma _{i2}\dsum\limits_{t=1}^{T-1}E\left[ f_{2,t}y_{t{\LARGE +}%
1}\right] +\dsum\limits_{t=1}^{T-1}E\left[ u_{it}y_{t{\LARGE +}1}\right]  \\
&=&\gamma _{i2}\dsum\limits_{t=1}^{T-1}\dsum\limits_{k=0}^{\infty
}\dsum\limits_{\ell =0}^{\infty }e_{3}^{{\Large \prime }}A^{k}E\left[
\varepsilon _{t-k}\varepsilon _{t{\LARGE +}1-\ell }^{{\Large \prime }}\right]
\left( A^{{\Large \prime }}\right) ^{\ell }e_{1}=\gamma
_{i2}\dsum\limits_{t=1}^{T-1}\dsum\limits_{k=0}^{\infty }e_{3}^{{\Large %
\prime }}A^{k}\Sigma _{\varepsilon }\left( A^{{\Large \prime }}\right) ^{k%
{\LARGE +}1}e_{1} \\
&=&\gamma _{i2}\left( T-1\right) \left[ \sigma _{31}a_{YY}+\sigma
_{32}\alpha _{YF,1}\right] +\gamma
_{i2}\dsum\limits_{t=1}^{T-1}\dsum\limits_{k=1}^{\infty }e_{3}^{{\Large %
\prime }}A^{k}\Sigma _{\varepsilon }\left( A^{{\Large \prime }}\right) ^{k%
{\LARGE +}1}e_{1} \\
&\neq &0\text{, }
\end{eqnarray*}%
except in very specialized cases.\footnote{%
The reason why we say that in this case $E\left[ S_{i}\right] \neq 0$,
except in very specialized cases, is because although given that $\gamma
_{i2}\neq 0,$ $\sigma _{32}\neq 0,$ and $\alpha _{YF,1}\neq 0$; it is clear
that the term $\gamma _{i2}\left( T-1\right) \sigma _{32}\alpha _{YF,1}\neq 0
$; one may nevertheless argue that even in this case it is possible to have $%
E\left[ S_{i}\right] =0$ if it turns out that%
\begin{equation*}
\gamma _{i2}\left( T-1\right) \sigma _{32}\alpha _{YF,1}=-\gamma _{i2}\left(
T-1\right) \sigma _{31}a_{YY}-\gamma
_{i2}\dsum\limits_{t=1}^{T-1}\dsum\limits_{k=1}^{\infty }e_{3}^{{\Large %
\prime }}A^{k}\Sigma _{\varepsilon }\left( A^{{\Large \prime }}\right) ^{k%
{\LARGE +}1}e_{1}\text{.}
\end{equation*}%
However, note that for the above identity to hold, the elements of $A$ and $%
\Sigma _{\varepsilon }$ including $\sigma _{31}$ and $a_{YY}$ must take on
very specific values so that in general the above identity is not likely to
hold, in which case we would have $E\left[ S_{i}\right] \neq 0$.} Moreover,
given that $\gamma _{j1}\neq 0$, $\sigma _{22}>0$, and $\alpha _{YF,1}\neq 0$%
; the expected value of $S_{j}$ will also not, in general, be properly
centered at zero. That is,%
\begin{eqnarray*}
E\left[ S_{j}\right]  &=&\dsum\limits_{t=1}^{T-1}E\left[ Z_{jt}y_{t{\LARGE +}%
1}\right]  \\
&=&\gamma _{j1}\dsum\limits_{t=1}^{T-1}E\left[ f_{1,t}y_{t{\LARGE +}1}\right]
+\dsum\limits_{t=1}^{T-1}E\left[ u_{jt}y_{t{\LARGE +}1}\right]  \\
&=&\gamma _{j1}\dsum\limits_{t=1}^{T-1}\dsum\limits_{k=0}^{\infty
}\dsum\limits_{\ell =0}^{\infty }e_{2}^{{\Large \prime }}A^{k}E\left[
\varepsilon _{t-k}\varepsilon _{t{\LARGE +}1-\ell }^{{\Large \prime }}\right]
\left( A^{{\Large \prime }}\right) ^{\ell }e_{1} \\
&=&\gamma _{j1}\dsum\limits_{t=1}^{T-1}\dsum\limits_{k=0}^{\infty }e_{2}^{%
{\Large \prime }}A^{k}\Sigma _{\varepsilon }\left( A^{{\Large \prime }%
}\right) ^{k{\LARGE +}1}e_{1} \\
&=&\gamma _{j1}\left( T-1\right) \left[ \sigma _{21}a_{YY}+\sigma
_{22}\alpha _{YF,1}\right] +\gamma
_{j1}\dsum\limits_{t=1}^{T-1}\dsum\limits_{k=1}^{\infty }e_{2}^{{\Large %
\prime }}A^{k}\Sigma _{\varepsilon }\left( A^{{\Large \prime }}\right) ^{k%
{\LARGE +}1}e_{1} \\
&\neq &0\text{,}
\end{eqnarray*}%
except in very specialized cases. Hence, both statistics, when appropriately
normalized, will diverge with probability approaching one, as $T\rightarrow
\infty .$ This makes the right inference about the relevance of both of
these variables, since the divergence of these statistics implies that the
null hypothesis $H_{0}:\gamma _{i}=0$ (i.e., $Z_{it}$ is irrelevant) as well
as the null hypothesis $H_{0}:\gamma _{j}=0$ (i.e., $Z_{jt}$ is irrelevant)
will both be rejected with probability approaching one. However, if we were
to interpret these statistics as providing inference about the predictive
content of the variables $Z_{it}$ and $Z_{jt}$; then, we would have made the
wrong inference about $Z_{it}$, since it loads only on $f_{2,t}$ which is
not helpful in predicting $y_{t{\LARGE +}1}$.

On the other hand, consider the alternative scenario where $\gamma _{i2}=0$
and $\gamma _{j1}=0$ so that $\gamma _{i}^{{\large \prime }}=\left( 
\begin{array}{cc}
0 & \gamma _{i2}%
\end{array}%
\right) =\left( 
\begin{array}{cc}
0 & 0%
\end{array}%
\right) $ and $\gamma _{j}^{{\large \prime }}=\left( 
\begin{array}{cc}
\gamma _{j1} & 0%
\end{array}%
\right) =\left( 
\begin{array}{cc}
0 & 0%
\end{array}%
\right) $, and, thus, both $Z_{it}$ and $Z_{jt}$ are now irrelevant
variables. Then, under this alternative scenario, we would have%
\begin{eqnarray*}
E\left[ S_{i}\right] &=&\gamma
_{i2}\dsum\limits_{t=1}^{T-1}\dsum\limits_{j=1}^{\infty }e_{3}^{{\Large %
\prime }}A^{j-1}\Sigma _{\varepsilon }\left( A^{{\Large \prime }}\right)
^{j}e_{1}=0 \\
E\left[ S_{j}\right] &=&\gamma
_{j1}\dsum\limits_{t=1}^{T-1}\dsum\limits_{j=1}^{\infty }e_{2}^{{\Large %
\prime }}A^{j-1}\Sigma _{\varepsilon }\left( A^{{\Large \prime }}\right)
^{j}e_{1}=0
\end{eqnarray*}%
so that both statistics are now properly centered at zero, and neither will
diverge, when appropriately normalized, as $T\rightarrow \infty $. It
follows that, under this alternative scenario, given an appropriate
threshold or critical value, we will also make the correct inference
asymptotically about the fact that both $Z_{it}$ and $Z_{jt}$ are irrelevant
variables in this case. Hence, under both scenarios, we make the right
inference asymptotically about the relevance of a variable; however, under
the first scenario, we do not make the right inference about predictive
content. For ease of presentation, we have given an example based on a
simple score statistic whose construction does not involve a blocking scheme
or self-normalization. The same story holds, however, for the more
complicated score statistics discussed in this paper.

The above example shows that interpreting a score-statistic-based variable
selection procedure as a procedure which selects variables based on
predictive content as opposed to relevance leads to a situation where we
cannot, in general, interpret these type of procedures as being completely
consistent. However, leaving particular examples like this one aside, we do
believe that it is often useful to interpret our variable selection
procedure as being helpful both for assessing the predictive content and for
assessing the relevance of a variable $Z_{it}$ in a factor augmented
forecasting context, since the two objectives are closely related. In
addition, it should be noted that being able to correctly identify all of
the relevant variables and use only the relevant variables to estimate the
factors can itself be helpful from the perspective of forecasting. This is
because a forecasting equation such as the one given in expression (\ref%
{factor aug autoreg})\ above depends on the unobserved latent factors which
must be estimated. In either the case where some irrelevant variables are
employed in the estimation process or if some relevant variables are not
employed, the quality of the factor estimates could be reduced; which, in
turn, can adversely affect the quality of point forecasts.

\noindent \textbf{Remark 2.3:}

Note also that a valuable by-product of our variable selection procedure is
that it provides us with an estimate $\widehat{N}_{1}$ of the unobserved
quantity $N_{1}$, where the latter, in light of Assumption 2-6, can also be
interpreted as giving the order of magnitude of $\Gamma ^{\prime }\Gamma $
and is, thus, a measure of the overall pervasiveness of the factors in a
given application. As mentioned previously, since $N_{1}$ is itself not
directly observable, having a consistent estimator $\widehat{N}_{1}$
provides practitioners with a useful diagnostic statistic which can help
them assess the overall pervasiveness of the factors in empirical
applications. As we show in part (a) of Lemma C-15 in Appendix C of the
Technical Appendix, $\widehat{N}_{1}$ is a consistent estimator of $N_{1}$,
in the sense that $\widehat{N}_{1}/N_{1}\overset{p}{\rightarrow }1$. As can
be seen from the proof of Lemma C-15(a), this consistency result will not be
possible in general if we do not have a completely consistent variable
selection procedure where the probabilities of both Type I error and Type II
error vanish asymptotically. To give an intuitive example for why this is
the case, consider the following.

\noindent \textbf{Example 2: }Suppose that $N_{1}=\left\lceil \left(
1-\alpha \right) N\right\rceil $ and $N_{2}=\left\lfloor \alpha
N\right\rfloor $ for some fixed $\alpha $ such that $0<\alpha <1$, and
suppose that we use a variable selection procedure which, even in large
sample, results in a $5\%$ Type I error but no Type II error so that $%
\widehat{N}_{1}=\left( N_{1}+0.05N_{2}\right) \left[ 1+o_{p}\left( 1\right) %
\right] $. Then, it is easy to see that%
\begin{eqnarray*}
\frac{\widehat{N}_{1}}{N_{1}}-1 &=&\frac{\widehat{N}_{1}-N_{1}}{N_{1}} \\
&=&\frac{\left( N_{1}+0.05N_{2}\right) \left[ 1+o_{p}\left( 1\right) \right]
-N_{1}}{N_{1}} \\
&=&\frac{0.05N_{2}\left[ 1+o_{p}\left( 1\right) \right] }{N_{1}}+o_{p}\left(
1\right) \\
&=&\frac{0.05\left\lfloor \alpha N\right\rfloor }{\left\lceil \left(
1-\alpha \right) N\right\rceil }\left[ 1+o_{p}\left( 1\right) \right]
+o_{p}\left( 1\right) \\
&=&\frac{0.05\left\lfloor \alpha N\right\rfloor }{\left\lceil \left(
1-\alpha \right) N\right\rceil }+o_{p}\left( 1\right) \\
&\neq &o_{p}\left( 1\right)
\end{eqnarray*}

\noindent so that, under this scenario, $\widehat{N}_{1}$ is not a
consistent estimator of $N_{1}$. Now, one may argue at this point that if we
instead assume that $N_{2}/N_{1}\rightarrow 0$ (which corresponds to the
case where $\alpha \rightarrow 0$), then we will still have a consistency
result such that $\widehat{N}_{1}/N_{1}\overset{p}{\rightarrow }1,$ even
when we use a variable selection procedure where the probability of a Type I
error does not vanish asymptotically. \textbf{\ }However, note that if a
dataset is well approximated by the rate condition $N_{2}/N_{1}\rightarrow 0$%
, then the forecast results based on our procedure should be very similar to
the forecast results obtained under the conventional PCA procedure, given
that in this situation very few variables will be excluded from factor
estimation under our variable selection procedure. This does not seem to be
in accord with our empirical results reported in Section 5, where we find
that using FRED-MD data delivers forecast results based on our procedure
that are better than the forecast results of the conventional PCA procedure
in a vast majority of the cases considered. Moreover, our estimates of $%
N_{1} $ and $N_{2}$ using the FRED-MD dataset are not indicative of a
situation where $N_{2}$ is negligible relative to $N_{1}$.

Alternatively, suppose instead that we use a variable selection procedure
which results in a $5\%$ Type II error but no Type I error in large sample,
so that $\widehat{N}_{1}=0.95N_{1}\left[ 1+o_{p}\left( 1\right) \right] $.
For this case, we have 
\begin{eqnarray*}
\frac{\widehat{N}_{1}}{N_{1}}-1 &=&\frac{0.95N_{1}\left[ 1+o_{p}\left(
1\right) \right] -N_{1}}{N_{1}} \\
&=&-\frac{0.05N_{1}}{N_{1}}+o_{p}\left( 1\right) \\
&=&-0.05+o_{p}\left( 1\right) \\
&\neq &o_{p}\left( 1\right)
\end{eqnarray*}%
Together, the two cases described above show that, to obtain a consistent
estimate of $N_{1}$, we need in general to apply a completely consistent
variable selection method.

In their paper, Bai and Ng (2023) provide a rate condition for consistent
factor estimation (i.e., Assumption A4 in their paper) which can be restated
in our setup as the assumption that $N/\left( TN_{1}\right) \rightarrow 0$.
Now, this rate condition can provide a very useful guide for applied
researchers wishing to assess the overall pervasiveness of the factors in a
particular empirical problem of interest to them if a consistent estimator
can be developed for the unobserved quantity $N_{1}$, and this is exactly
what our procedure supplies. Viewed from this perspective, what we are
proposing here builds on the work of Bai and Ng (2023), as our procedure
helps to highlight the importance of the rate condition they have introduced
and provides additional information that will be helpful to empirical
researchers in assessing the degree of pervasiveness of the underlying
factors in a particular empirical application.

\noindent \textbf{Remark 2.4:}

Additionally, note that knowledge of the number of factors is not needed to
implement our variable selection procedure. Hence, in the case where the
number of factors needs to be determined empirically, an applied researcher
could first use our procedure to properly select the relevant variables and
then apply an information criterion such as that proposed in Bai and Ng
(2002) to estimate the number of factors.

\section{\noindent Consistent Estimation of the h-Step Ahead Predictor Based
on the FAVAR Model}

\textbf{I}n this section, we provide our main theoretical results on factor
estimation and also on the estimation of the $h$-step ahead predictor within
the FAVAR framework. This includes deriving an explicit representation of
the $h$-step ahead forecasting equation implied by the FAVAR model. The
totality of our results, as provided in this section and in the previous
section of the paper, gives a complete description of our proposed
methodology for constructing forecasts within a FAVAR framework. In
particular, our results provide explicit formulae that allow empirical
researchers to easily implement procedures for variable selection for the
purpose of factor estimation, use the selected variables to construct
estimators of the factors, and finally to estimate the $h$-step ahead
predictor.

To obtain the results of this section, we need first to impose a further
rate condition on the tuning parameter, $\varphi $ (see part (c) of
Assumption 2-11* below).

\medskip

\noindent \textbf{Assumption 2-11*: }Let $\varphi $ satisfy the following
three conditions: (a) $\varphi \rightarrow 0$ as $N_{1},N_{2}\rightarrow
\infty $; (b)\ there exists some constant $a>0,$ such that $\varphi \geq 1%
{\Large /}N^{a},$ for all $N_{1},N_{2}$ sufficiently large; and (c) 
\begin{equation*}
\max \left\{ \frac{N^{\frac{{\large 2}}{{\large 7}}}\varphi ^{\frac{{\large 5%
}}{{\large 7}}}}{N_{1}},\frac{N^{\frac{{\large 1}}{{\large 3}}}\varphi }{%
N_{1}T}\right\} \rightarrow 0\text{ as }N_{1},N_{2},T\rightarrow \infty .
\end{equation*}

\noindent Note that\textbf{\ }the rate condition given in part (c) of
Assumption 2-11* depends on $N_{1}$. However, if we choose $\varphi $ so
that $\varphi N^{\frac{{\large 2}}{{\large 5}}}=O\left( 1\right) $, then 
\begin{equation*}
\frac{N^{\frac{{\large 2}}{{\large 7}}}\varphi ^{\frac{{\large 5}}{{\large 7}%
}}}{N_{1}}=O\left( \frac{1}{N_{1}}\right) =o\left( 1\right) \text{ and }%
\frac{N^{\frac{{\large 1}}{{\large 3}}}\varphi }{N_{1}T}=O\left( \frac{1}{%
N_{1}N^{\frac{{\large 1}}{{\large 15}}}T}\right) =o\left( \frac{1}{N_{1}}%
\right) \text{.}
\end{equation*}%
Hence, with this choice of $\varphi $, Assumption 2-11* part (c) will be
satisfied as long as $N_{1}\rightarrow \infty $, and there is no need to
impose any further condition on the rate at which $N_{1}$ grows. Requiring
that $N_{1}\rightarrow \infty $ is a minimal condition, since if $%
N_{1}\nrightarrow \infty $; then consistent factor estimation, even up to an
invertible matrix transformation, is impossible. Moreover, Monte Carlo
results reported in Section 5 of this paper show that our variable selection
procedure performs very well in finite samples, under the tuning parameter
choice $\varphi =N^{-\frac{{\large 2}}{{\large 5}}}$, both in terms of
controlling the probability of a false positive (or Type I) error and in
terms of controlling the probability of a false negative (or Type II) error.

Next, consider the post-variable-selection principal component estimator of $%
\underline{F}_{t}=\left( F_{t}^{\prime },F_{t-1}^{\prime },...,F_{t-p{\LARGE %
+}1}^{\prime }\right) :$ 
\begin{equation}
\widehat{\underline{F}}_{t}=\frac{\widehat{\Gamma }^{\prime }Z_{t,N}\left( 
\widehat{H^{c}}\right) }{\widehat{N}_{1}}\text{,}  \label{factor estimator}
\end{equation}%
where%
\begin{equation*}
Z_{t,N}\left( \widehat{H^{c}}\right) =\left[ 
\begin{array}{cccc}
Z_{1,t}\mathbb{I}\left\{ 1\in \widehat{H^{c}}\right\} & Z_{2,t}\mathbb{I}%
\left\{ 2\in \widehat{H^{c}}\right\} & \cdots & Z_{N,t}\mathbb{I}\left\{
N\in \widehat{H^{c}}\right\}%
\end{array}%
\right] ^{\prime },
\end{equation*}%
with%
\begin{equation*}
\mathbb{I}\left\{ i\in \widehat{H^{c}}\right\} =\left\{ 
\begin{array}{cc}
1 & \text{if }i\in \widehat{H^{c}}\text{, i.e., if }\mathbb{S}_{i,T}^{+}\geq
\Phi ^{-1}\left( 1-\frac{\varphi }{2N}\right) \\ 
0 & \text{if }i\in \widehat{H}\text{, i.e., if }\mathbb{S}_{i,T}^{+}<\Phi
^{-1}\left( 1-\frac{\varphi }{2N}\right)%
\end{array}%
\right. ,
\end{equation*}%
and where $\widehat{N}_{1}=\#\left( \widehat{H^{c}}\right) $, i.e., the
cardinality of the set $\widehat{H^{c}}$. Here, $\widehat{\Gamma }$ denotes
the principal component estimator of the loading matrix $\Gamma $
constructed from taking $\sqrt{\widehat{N}_{1}}$ times the eigenvectors of
the post-variable-selection sample covariance matrix $\widehat{\Sigma }%
\left( \widehat{H^{c}}\right) $ associated with the $Kp$ largest eigenvalues
of this matrix, where, in this case, $\widehat{\Sigma }\left( \widehat{H^{c}}%
\right) =\frac{Z\left( \widehat{H^{c}}\right) ^{\prime }Z\left( \widehat{%
H^{c}}\right) }{\widehat{N}_{1}T_{0}}=\frac{1}{\widehat{N}_{1}T_{0}}%
\dsum\limits_{t=p}^{T}Z_{t,N}\left( \widehat{H^{c}}\right) Z_{t,N}\left( 
\widehat{H^{c}}\right) ^{\prime },$ with $T_{0}=T-p+1$. Our next result
shows that the estimator given in expression (\ref{factor estimator})
consistently estimates the unobserved factors $\underline{F}_{t},$up to an
invertible $Kp\times Kp$ matrix transformation.

\noindent \textbf{Theorem 3: }\textit{Suppose that Assumptions 2-1, 2-2,
2-3, 2-4, 2-5, 2-6, 2-7, 2-8, 2-9, and 2-10 hold. Let }$\widehat{\underline{F%
}}_{t}$\textit{\ be as defined in expression (\ref{factor estimator}).
Assume further that the specification of the tuning parameter, }$\varphi ,$%
\textit{\ in the decision rule (\ref{var selection decision rule}) satisfies
Assumption 2-11*. Then,}%
\begin{equation*}
\left\Vert \widehat{\underline{F}}_{t}-Q^{\prime }\underline{F}%
_{t}\right\Vert _{2}=o_{p}\left( 1\right) ,\text{ for all fixed }t\text{,}
\end{equation*}%
\textit{where}%
\begin{equation*}
Q=\left( \frac{\Gamma ^{\prime }\Gamma }{N_{1}}\right) ^{\frac{{\Large 1}}{%
{\Large 2}}}\Xi \widehat{V}\text{,}
\end{equation*}%
\textit{and where }$\widehat{V}$\textit{\ is the }$Kp\times Kp$\textit{\
orthogonal matrix given in Lemma C-14 part (c), and }$\Xi $\textit{\ is a }$%
Kp\times Kp$\textit{\ orthogonal matrix whose columns are the eigenvectors
of the matrix }%
\begin{equation*}
M_{FF}^{\ast }=\left( \frac{\Gamma ^{\prime }\Gamma }{N_{1}}\right)
^{1/2}M_{FF}\left( \frac{\Gamma ^{\prime }\Gamma }{N_{1}}\right)
^{1/2}=\left( \frac{\Gamma ^{\prime }\Gamma }{N_{1}}\right) ^{1/2}\frac{1}{%
T_{0}}\dsum\limits_{t=p}^{T}E\left[ \underline{F}_{t}\underline{F}%
_{t}^{\prime }\right] \left( \frac{\Gamma ^{\prime }\Gamma }{N_{1}}\right)
^{1/2}\text{.}
\end{equation*}%
\qquad \qquad

Although Theorem 3 shows that, without further identifying assumptions, we
can only estimate the factors $\underline{F}_{t}$ consistently up to an
invertible $Kp\times Kp$ matrix transformation, this result turns out to be
sufficient for us to estimate the $h$-step ahead predictor consistently.
More specifically, in Appendix C of the Technical Appendix, we show that,
for an $h$-step ahead forecast, the (infeasible) forecasting equation
implied by the FAVAR model (\ref{FAVAR}) has the form%
\begin{equation}
Y_{t{\LARGE +}h}=\beta _{0}+B_{1}^{\prime }\underline{Y}_{t}+B_{2}^{\prime }%
\underline{F}_{t}+\eta _{t{\LARGE +}h},
\label{infeasible h step forecast eqn}
\end{equation}%
where $\underline{Y}_{t}$ and $\underline{F}_{t}$ are as defined in
expression (\ref{Yunderscore and Funderscore}) above and where:%
\begin{eqnarray}
\beta _{0} &=&\dsum\limits_{j=0}^{h-1}J_{d}A^{j}\alpha \text{, }%
B_{1}^{\prime }=J_{d}A^{h}\mathcal{P}_{\left( d{\LARGE +}K\right) p}^{\prime
}S_{d}\text{, }B_{2}^{\prime }=J_{d}A^{h}\mathcal{P}_{\left( d{\LARGE +}%
K\right) p}^{\prime }S_{K}\text{ and}  \label{beta0 B1 and B2} \\
\text{ }\eta _{t{\LARGE +}h} &=&\dsum\limits_{j=0}^{h-1}J_{d}A^{j}J_{d%
{\LARGE +}K}^{\prime }\varepsilon _{t{\LARGE +}h-j}\text{.}  \notag
\end{eqnarray}%
Here, $\alpha $ and $A$ are, respectively, the intercept (vector) and the
coefficient matrix of the companion form defined in expression (\ref%
{companion form notations}) above, $\mathcal{P}_{\left( d{\LARGE +}K\right)
p}$ is a permutation matrix such that $\mathcal{P}_{\left( d{\LARGE +}%
K\right) p}\underline{W}_{t}=\left( 
\begin{array}{c}
\underline{Y}_{t} \\ 
\underline{F}_{t}%
\end{array}%
\right) $, $S_{d}=\left( 
\begin{array}{c}
I_{dp} \\ 
\underset{Kp\times dp}{0}%
\end{array}%
\right) $, $S_{K}=\left( 
\begin{array}{c}
\underset{dp\times Kp}{0} \\ 
I_{Kp}%
\end{array}%
\right) $,$\underset{d\times \left( d{\LARGE +}K\right) p}{J_{d}}=\left[ 
\begin{array}{cccc}
I_{d} & 0 & \cdots & 0%
\end{array}%
\right] $, and $\underset{\left( d{\LARGE +}K\right) \times \left( d{\LARGE +%
}K\right) p}{J_{d{\LARGE +}K}}=\left[ 
\begin{array}{cccc}
I_{d{\LARGE +}K} & 0 & \cdots & 0%
\end{array}%
\right] .$ See\ the beginning of Appendix C for a derivation of the equation
given in expression (\ref{infeasible h step forecast eqn}). The reason
expression (\ref{infeasible h step forecast eqn}) is called an infeasible
forecasting equation is because $\underline{F}_{t}$ is not observed, so to
obtain a feasible version of this forecasting equation, we must replace $%
\underline{F}_{t}$ in equation (\ref{infeasible h step forecast eqn}) with
the estimate $\underline{\widehat{F}}_{t}$ given in expression (\ref{factor
estimator}). Doing so, we arrive at a feasible $h$-step ahead forecasting
equation of the form: 
\begin{eqnarray}
Y_{t{\LARGE +}h} &=&\beta _{0}+\dsum\limits_{g=1}^{p}B_{1,g}^{\prime }Y_{t-g%
{\LARGE +}1}+\dsum\limits_{g=1}^{p}B_{2,g}^{\prime }\widehat{F}_{t-g{\LARGE +%
}1}+\widehat{\eta }_{t{\LARGE +}h}  \notag \\
&=&\beta _{0}+B_{1}^{\prime }\underline{Y}_{t}+B_{2}^{\prime }\underline{%
\widehat{F}}_{t}+\widehat{\eta }_{t{\LARGE +}h}\text{, }
\label{feasible h step forecast eqn}
\end{eqnarray}%
where $\widehat{\eta }_{t{\LARGE +}h}=\eta _{t{\LARGE +}h}-B_{2}^{\prime
}\left( \underline{\widehat{F}}_{t}-\underline{F}_{t}\right) ,$ with $\eta
_{t{\LARGE +}h}=\dsum\nolimits_{j=0}^{h-1}J_{d}A^{j}J_{d{\LARGE +}K}^{\prime
}\varepsilon _{t{\LARGE +}h-j}$.

One can interpret expression (\ref{feasible h step forecast eqn}) as a
\textquotedblleft reduced form\textquotedblright\ formulation of the
forecasting equation where the reduced form parameters $\beta _{0}$, $B_{1}$%
, and $B_{2}$ are nonlinear functions of the parameters $\left( \mu
,A_{1},....,A_{p}\right) $ of the FAVAR model, in the case where $h>1$. For
forecasting purposes, while it is possible to estimate the conditional mean
of the forecasting equation (\ref{feasible h step forecast eqn}) by
estimating the underlying parameters directly using nonlinear least squares,
here we choose instead to estimate the conditional mean by estimating the
reduced form parameters $\beta _{0}$, $B_{1}$, and $B_{2}$ via linear least
squares. An important reason why we choose this approach is due to
complications that arise both because we are forecasting with a FAVAR which
contains unobserved factors that must first be estimated and because we only
make enough identifying assumptions so that the factors can only be
estimated consistently up to an invertible $Kp\times Kp$ matrix
transformation. In fact, it turns out that estimating the underlying
parameters $\mu ,A_{1},....,A_{p}$ by nonlinear least squares and
constructing an estimator of the conditional mean of the forecasting
equation based on these estimates will not lead to a consistently estimated $%
h$-step predictor, unless further identifying assumptions are made. On the
other hand, as we show in Theorem 4 below, estimating the reduced form
parameters $\beta _{0}$, $B_{1}$, and $B_{2}$ by linear least squares allows
us to construct a consistent estimator of the conditional mean, even in the
absence of additional identifying assumptions. More precisely, let $%
\underline{\widehat{F}}_{t}$ denotes the factor estimates given in
expression (\ref{factor estimator}). Our procedure minimizes the least
squares criterion function:%
\begin{eqnarray}
Q\left( \beta _{0},B_{1},B_{2}\right) &=&\dsum\limits_{t=p}^{T-h}\left\Vert
Y_{t{\LARGE +}h}-\beta _{0}-B_{1}^{\prime }\underline{Y}_{t}-B_{2}^{\prime }%
\underline{\widehat{F}}_{t}\right\Vert _{2}^{2}  \notag \\
&=&\dsum\limits_{t=p}^{T-h}\left\Vert Y_{t{\LARGE +}h}-\beta
_{0}-\dsum\limits_{g=1}^{p}B_{1,g}^{\prime }Y_{t-g{\LARGE +}%
1}-\dsum\limits_{g=1}^{p}B_{2,g}^{\prime }\widehat{F}_{t-g{\LARGE +}%
1}\right\Vert _{2}^{2}  \label{least squares criterion}
\end{eqnarray}%
with respect to the parameters $\beta _{0}$, $B_{1}$, and $B_{2},$ and
delivers the OLS estimates $\widehat{\beta }_{0}$, $\widehat{B}_{1}$, and $%
\widehat{B}_{2}$. We then forecast $Y_{T{\LARGE +}h}$ using the $h$-step
predictor:%
\begin{equation}
\widehat{Y}_{T{\LARGE +}h}=\widehat{\beta }_{0}+\widehat{B}_{1}^{\prime }%
\underline{Y}_{T}+\widehat{B}_{2}^{\prime }\underline{\widehat{F}}_{T}\text{%
. }  \label{h step ahead predictor}
\end{equation}%
The following result shows that $\widehat{Y}_{T{\LARGE +}h}$ is a consistent
estimator of the conditional mean of the infeasible forecast equation (\ref%
{infeasible h step forecast eqn}).

\medskip

\noindent \textbf{Theorem 4: }\textit{Let }$\widehat{Y}_{T{\LARGE +}h}$%
\textit{\ be as defined in expression (\ref{h step ahead predictor}).
Suppose that Assumptions 2-1, 2-2, 2-3, 2-4, 2-5, 2-6, 2-7, 2-8, 2-9, 2-10,
and 2-11* hold. Then,}

\begin{equation*}
\widehat{Y}_{T{\LARGE +}h}-\left( \beta _{0}+B_{1}^{\prime }\underline{Y}%
_{T}+B_{2}^{\prime }\underline{F}_{T}\right) \overset{p}{\rightarrow }0\text{
as }N_{1},N_{2},T\rightarrow \infty \text{.}
\end{equation*}

\section{\noindent Monte Carlo Experiment}

In this section, we report some simulation results on the finite sample
performance of our variable selection procedure. The model used in the Monte
Carlo study is the following tri-variate FAVAR(1) process: 
\begin{eqnarray}
W_{t} &=&\mu +AW_{t-1}+\varepsilon _{t},  \label{W eqn} \\
Z_{t} &=&\gamma F_{t}+u_{t}\text{,}  \label{Z eqn}
\end{eqnarray}%
where%
\begin{equation*}
W_{t}=\left( 
\begin{array}{c}
Y_{1t} \\ 
Y_{2t} \\ 
F_{t}%
\end{array}%
\right) \text{, }\mu =\left( 
\begin{array}{c}
2 \\ 
1 \\ 
2%
\end{array}%
\right) \text{, }A=\left( 
\begin{array}{ccc}
0.9 & 0.3 & 0.5 \\ 
0 & 0.7 & 0.1 \\ 
0 & 0.6 & 0.7%
\end{array}%
\right) \text{, and }\gamma =\left( 
\begin{array}{c}
\iota _{N_{1}} \\ 
\underset{N_{2}\times 1}{0}%
\end{array}%
\right) ,
\end{equation*}%
with $\iota _{N_{1}}$ denoting an $N_{1}\times 1$ vector of ones. We
consider different configurations of $N$, $N_{1}$, and $T,$ as given below.
For the error process in equation (\ref{W eqn}), we take $\left\{
\varepsilon _{t}\right\} \equiv i.i.d.N\left( 0,\Sigma _{\varepsilon
}\right) $, where: 
\begin{equation*}
\Sigma _{\varepsilon }=\left( 
\begin{array}{ccc}
1.3 & 0.99 & 0.641 \\ 
0.99 & 0.81 & 0.009 \\ 
0.641 & 0.009 & 5.85%
\end{array}%
\right) \text{.}
\end{equation*}%
The error process, $\left\{ u_{it}\right\} ,$ in equation (\ref{Z eqn}) is
allowed to exhibit both temporal and cross-sectional dependence and also
conditional heteroskedasticity. More specifically, we let $%
u_{it}=0.8u_{it-1}+\zeta _{it}$, and following the approach for modeling
cross-sectional dependence given in the Monte Carlo design of Stock and
Watson (2002a), we specify: $\zeta _{it}=\left( 1+b^{2}\right) \eta
_{it}+b\eta _{i+1,t}+b\eta _{i-1,t}$, and set $b=1$. In addition, $\eta
_{it}=\omega _{it}\xi _{it},$ with $\left\{ \xi _{it}\right\} \equiv
i.i.d.N\left( 0,1\right) $ independent of $\left\{ \varepsilon _{t}\right\} $%
, and $\omega _{it}$ follows a GARCH(1,1) process given by: $\omega
_{it}^{2}=1+0.9\omega _{it-1}^{2}+0.05\eta _{it-1}^{2}$. To study the
effects of varying the tuning parameter, we consider specifications where $%
\varphi =\left( \ln \ln N\right) ^{-\vartheta }$ for $\vartheta =0.1,0.5,1$
and also $\varphi =N^{-\vartheta }$ for $\vartheta =0.2,0.4,0.6$.\footnote{%
We have also obtained simulation results for the cases where $\varphi
=\left( \ln N\right) ^{-\vartheta }$ for $\vartheta =0.1,0.5,1$ and where $%
\varphi =N^{-\vartheta }$ for $\vartheta =0.3,0.5$. The results obtained for
these cases are qualitatively similar to the results reported in this paper.
Hence, due to space considerations, we do not report these results here, but
they are available from the authors upon request.} We also attempt to shed
light on the effects of using blocks of different sizes on the performance
of our procedure. To do this, for $T=100$, we set $\tau _{1}=2$, $3$, $4$,
and $5$; for $T=200$, we set $\tau _{1}=5$, $6$, $8$, and $10$; and for $%
T=600$, we set $\tau _{1}=6$, $8$, $10$, and $12$. Due to space
considerations, we only report Monte Carlo results for the statistic $%
\dsum\nolimits_{\ell =1}^{d}\varpi _{\ell }\left\vert S_{i,\ell
,T}\right\vert $. Simulation results for the statistic $\max_{1\leq \ell
\leq d}\left\vert S_{i,\ell ,T}\right\vert $ have also been obtained by the
authors. The results for $\max_{1\leq \ell \leq d}\left\vert S_{i,\ell
,T}\right\vert $ are qualitatively similar to those given here for $%
\dsum\nolimits_{\ell =1}^{d}\varpi _{\ell }\left\vert S_{i,\ell
,T}\right\vert $, and they are available from the authors upon request. In
addition, since $d=2$ in our Monte Carlo setup, we set $\varpi _{1}=\varpi
_{2}=1/2$. Results are gathered in Table 1 in the back of the paper. There,
the acronym FPR denotes the \textquotedblleft False Positive
Rate\textquotedblright\ or the \textquotedblleft Type I\textquotedblright\
error rate, i.e., the proportion of cases where an irrelevant variable $%
Z_{it}$, with associated coefficient $\gamma _{i}=0$ is erroneously selected
as a relevant variable. FNR denotes the \textquotedblleft False Negative
Rate\textquotedblright\ or the \textquotedblleft Type II\textquotedblright\
error rate, i.e., the proportion of cases where a relevant variable is
erroneously identified as being irrelevant.

Looking across each row of the table, note that FPRs decrease when moving
from left to right, whereas FNRs increase. This is not surprising, because
moving from $\varphi =\left( \ln \ln N\right) ^{-0.1}$ to $\varphi =N^{-0.6}$
for a given $N$ results in smaller values of the tuning parameter $\varphi $%
, and the specified threshold $\Phi ^{-1}\left( 1-\frac{\varphi }{2N}\right) 
$ thus becomes larger. Overall, these results indicate that choosing $%
\varphi $ in the range between $\left( \ln \ln N\right) ^{-0.1}$ and $%
N^{-0.4}$\ leads to very good performance, since within this range, neither
FPR nor FNR exceeds $0.1$ in any of the cases studied here. In fact, both
are smaller than $0.05$ in a vast majority of the cases. In contrast,
choosing $\varphi =N^{-0.6}$ can lead to high FNRs, as such a choice of $%
\varphi $ can set our threshold at such a high level that our procedure ends
up having very little power.

Looking down the columns of the table, note that FPR tends to increase as $%
\tau _{1}$ increases, whereas FNR tends to decrease as $\tau _{1}$
increases. As an explanation for this result, note first that the smaller is 
$\tau _{1}$ relative to $\tau $, the larger is $\tau _{2}$ (since $\tau
=\tau _{1}+\tau _{2}$), and thus the larger is the number of observations
removed when constructing the self-normalized block sums. Intuitively, this
can lead to better accommodation of the effects of dependence and better
moderate deviation approximations under the null hypothesis, resulting in a
lower FPR. However, removal of a larger number of observations can also lead
to a reduction in power, when the alternative hypothesis is correct, so that
a negative consequence of having a smaller $\tau _{1}$ relative to $\tau $
is that FNR will tend to be higher in this case. The opposite, of course,
occurs when we try to specify a larger $\tau _{1}$ relative to $\tau $.

Our results also show that when the sample sizes are large enough such as
the cases presented in the last panel of the table, where $T=600$ and $%
N=1000 $, then both FPR and FNR are very close to zero for all of the cases
that we consider. Moreover, even in the case where $T=100$ and $N=100$, FPR
and NPR rates are usually less than 0.05, and are often much smaller than
that. This is in accord with the results of our theoretical analysis, which
shows that our variable selection procedure is completely consistent in the
sense that both the probability of a false positive and the probability of a
false negative approach zero, as the sample sizes go to infinity.

\section{Forecasting Experiments}

In this section, we carry out prediction experiments using two different
datasets and using the variable selection methodology discussed above. Our
goal is to compare different variable selection methods, and more broadly to
assess whether reducing the number of variables prior to factor estimation
using the consistent selection and estimation technique discussed above can
result in improved forecast performance.

\subsection{\noindent Empirical Illustration 1 - FRED-MD Dataset}

In this illustration, we forecast eight target variables from the monthly
real-time macroeconomic FRED-MD dataset maintained by the St. Louis Federal
Reserve Bank.\footnote{%
see https://www.stlouisfed.org/research/economists/mccracken/fred-databases}
We follow the data cleaning methods outlined on the FRED-MD data website, as
well as removing all discontinued series yielding a dataset, $\mathbf{X}$,
containing $96$ variables for the period 1975:1 to 2024:6. The full list of
all macroeconomic variables and their transformations is available upon
request from the authors.\footnote{%
We use the variable transformations recommended on the FRED-MD website (see
the FRED MD appendix on that website for complete details), with three
exceptions. First, we do not difference unemployment. Second, we do not
difference interest rate variables. Third, we do not take second differences
of any variables. Instead, we take first differences of the small number of
variables that are twice differenced in the list of recommended variable
transformations referred to above. Tabulated results based on the original
FRED-MD variable transformations are available upon request from the
authors. Also, note that the original FRED-MD dataset actually begins in
1973:3.}

Of note is that the FRED-MD dataset used in this illustration is
\textquotedblleft truly\textquotedblright\ real-time. Consider the value of
industrial production for January 2020. In February 2020, the government
reported a \textquotedblleft first release\textquotedblright\ value for
January. In March 2020, they updated their \textquotedblleft
estimate\textquotedblright\ of industrial production for January. Namely,
they reported a \textquotedblleft second release\textquotedblright\ for
January. This process of revision continues indefinitely. Namely, as the
government changes data collection and processing methodology, collects new
data and/or revises definitions of variables, new releases are reported. A
\textquotedblleft vintage\textquotedblright\ of data consists of all of the
historical data that were available, in real-time, at a particular calendar
date, say February 2020. This means that there is a unique vintage of
industrial production data available each month, and the values of the
calendar dated observations in each vintage may change over time. Using this
type of data allows the practitioner to truly simulate a forecasting
environment in which models are updated at each point in time using data
that were actually available at that time. For further discussion of the
structure of real-time datasets, as well as methods for real-time
forecasting, refer to Swanson (1996), Swanson and van Dijk (2006), and Kim
and Swanson (2018).

Our forecasting experiment is carried out as follows. All forecasts that we
construct utilize the most recent vintage of data available. Thus, at each
point in time prior to the construction of each new forecast, a new vintage
of data is used for variable selection, factor construction, and forecast
model estimation. The eight target variables for which we construct
predictions are summarized in Table 2, and include Industrial Production
(INDPRO), Civilian Unemployment Rate (UNRATE), Housing Starts: new,
privately owned (HOUST), Housing Permits: new, privately owned (PERMIT),
Real M2 Money Stock (M2REAL), 10-Year Government Treasury Bond Rate (GS10),
CPI - All Items (CPI), S\&P Common Stock Price Index - Composite (S\&P500).

We estimate the following forecasting model: 
\begin{equation}
y_{t+h}={\alpha }+{\beta _{h}}(L)y_{t}+{\gamma _{h}}(L)F\boldsymbol{_{t}}%
+\epsilon _{t+h},
\end{equation}%
where $y_{t}$ is the scalar target variable to be predicted, ${\beta _{h}}%
(L) $ and ${\gamma _{h}}(L)$ are finite order lag polynomials, $F\boldsymbol{%
_{t}}$ is a vector of estimated factors, and $\epsilon _{t}$ is a stochastic
disturbance term. Lags in this model are selected using the Schwarz
Information Criterion (SIC), and\textbf{\ }our benchmark model sets the lag
polynomial $\gamma _{h}(L)=0$. In implementing our forecast experiments, we
take the initial training sample to be the period 1975:1-1999:12; but as we
move forward in time with each new prediction, we use a new release of the
data (i.e., an updated data vintage) in constructing our forecast. The
tuning parameters as well as the number of factors are all re-estimated
based on the updated data vintage. Applying our procedure to the FRED-MD
dataset in this way yields estimates of $N_{1}$ ranging from 41 to 72 across
the different training samples that result from our updating of the data
vintage and across our 8 target variables. Since there are $N=95$ total
number of possible predictor variables excluding the target variable of
interest, our estimates of $N_{1}$\ imply that the number of irrelevant
variables $N_{2}$ is estimated to be in the range from 24 to 54 for the
range of cases described above, so that, for the FRED-MD dataset, the number
of irrelevant variables $N_{2}$ seems to be a significant proportion of the
total number of available variables. Given the non-triviality of $N_{2}$,\
having a variable selection procedure to identify those irrelevant variables
which only contribute noise to the factor estimation process can potentially
be beneficial both for estimation and for forecasting purposes. Moreover, as
we mentioned previously in section 2, to obtain a consistent estimator $%
\widehat{N}_{1}$ of $N_{1}$, we need a completely consistent variable
selection procedure, except in the case where $N_{2}/N_{1}\rightarrow 0$.
However, in light of what we have just discussed, the rate condition $%
N_{2}/N_{1}\rightarrow 0$ does not seem to be reasonable for the FRED-MD
dataset. Hence, if an applied researcher wishes to use an estimate of $N_{1}$
to assess the overall pervasiveness of the factors for a particular dataset
of interest; then, it would make sense for her/him to use a completely
consistent variable selection method since a selection method which is not
completely consistent is not guaranteed to provide a consistent estimator of 
$N_{1}$. What still remains to be seen, however, is whether our completely
consistent variable selection method will lead to improved forecast
performance.

In the sequel, we carry out variable selection and dimension reduction as
follows in order to estimate $F\boldsymbol{_{t}}$:\footnote{%
Note that all forecasting models are estimated using least squares, once the
factors are first estimated.}

\noindent \textit{Principal Components Analysis (PCA):} Utilize $\mathbf{X}$
to estimate latent factors, $F\boldsymbol{_{t}}$ using PCA, with the number
of factors determined using the $PC_{p2}$ criterion in Bai and Ng (2002).
The maximum number of the factors is set equal to both 4 and 8 (resulting in
two distinct sets of empirical results), following the findings of McCracken
and Ng (2016), who introduce and examine the dataset that we utilize in our
analysis.

\noindent \textit{Hard Thresholding (HT):} For each variable in $\mathbf{X}$%
, and forecast horizon, $h$, perform a regression of $y_{t+h}$ on lags of $%
y_{t}$ and on $X_{i,t}$, where $X_{i,t}$ is a scalar variable in $\mathbf{X}$%
, for $i=1,...,N$, and lags of $y_{t}$ are selected using the SIC. Let $%
t_{i} $ denote the $t$ statistic associated with $X_{it-h}$ in the
regression, and select variables, $X_{it}$ if $|t_{i}|$ $>1.28$. If the
number of selected variables is greater than 20, utilize PCA to estimate
factors for inclusion in the above forecasting equation, otherwise use the
AR(SIC) model. As models are re-estimated at each point in time, this
approach is a hybrid, in the sense that some models may include factors as
regressors, while others may be simple AR(SIC) models. Note that for all
variables except the S\&P500, the thresholding model was replaced with the
AR(SIC) benchmark for less than 10\% of the total number of forecasting
periods.

\noindent \textit{Chao-Swanson Variable Selection (CS)}: Use the
self-normalized statistic given in expressions (\ref{stat in d=1 case exp 1}%
) and (\ref{stat in d=2 case exp 2}) above for variable selection, and then
estimate factors for inclusion in the forecasting equation using PCA.
Consider the following sets of tuning parameter values: \{$\tau =5,\tau
_{1}=3,5\}$ and $\ \{\tau =10,\tau _{1}=6,8\},$ with 
\begin{equation*}
\varphi =\left\{ 
\begin{array}{rclrcl}
(lnlnN)^{-0.1} & (lnlnN)^{-0.6} & (lnN)^{-0.1} & (lnN)^{-0.6} & N^{-0.1} & 
N^{-0.6} \\ 
(lnlnN)^{-0.2} & (lnlnN)^{-0.7} & (lnN)^{-0.2} & (lnN)^{-0.7} & N^{-0.2} & 
N^{-0.7} \\ 
(lnlnN)^{-0.3} & (lnlnN)^{-0.8} & (lnN)^{-0.3} & (lnN)^{-0.8} & N^{-0.3} & 
N^{-0.8} \\ 
(lnlnN)^{-0.4} & (lnlnN)^{-0.9} & (lnN)^{-0.4} & (lnN)^{-0.9} & N^{-0.4} & 
N^{-0.9} \\ 
(lnlnN)^{-0.5} & (lnlnN)^{-1} & (lnN)^{-0.5} & (lnN)^{-1} & N^{-0.5} & N^{-1}%
\end{array}%
\right. .
\end{equation*}%
Tuning parameters used for each value of $h$ and target variable are
selected in real-time prior to the construction of each new forecast by
using an initial \textquotedblleft training dataset\textquotedblright\
consisting of the first 25 years of data. This training dataset is
partitioned into an in-sample period of 20 years and an out-of-sample period
of 5 years. In subsequent prediction experiments, tuning parameter is set
equal to that yielding the smallest mean square forecast error (MSFE) after
constructing real-time predictions over this 5 year period.

In summary, we construct real-time $h$-month ahead predictions using monthly
data, with $h=$1, 3, 6, and 12. Our forecasting models are called PCA, CS,
and HT, in reference to the manner in which variable selection prior to
factor estimation is carried out. The sample period used in our analysis is
1975:1-2024:6, and our ex ante forecast period is 2000:1-2024:6. Our initial
training sample period is 25 years from 1975:1 to 1999:12, as discussed
above, and both rolling and recursive windows of data are used when
re-estimating models at each point in time. Forecasting performance is
evaluated using point mean squared forecast errors (MSFEs), where MSFE=$%
\frac{1}{P}\sum_{t=1}^{T}(y_{t}-\hat{y}_{t})^{2}$, and $\hat{y}_{t}$ denotes
the real-time prediction for target variable $y_{t}$. In our tabulated
results, MSFEs, relative to those of the benchmark AR(SIC) model are
reported. Additionally, we report the results of Giacomini-White (GW) tests
(see Giacomini and White (2006)), which can be viewed as conditional
Diebold-Mariano (DM) predictive accuracy tests (see Diebold and Mariano
(1995)). Recall that the null hypothesis of the DM test when formulated
using the conditioning approach of Giacomini and White is: $H_{0}:\text{E}[L(%
\hat{\epsilon}_{t+h}^{(1)})|G_{t}]-\text{E}[L(\hat{\epsilon}%
_{t+h}^{(2)})|G_{t}]=0$, where the $\hat{\epsilon}_{t+h}^{(i)}$ are
prediction errors associated with model $i$, for $i=1,2$, and $G_{t}$
denotes the conditioning set, which includes the model and estimated
parameters. Here, $L(\cdot )$ is a quadratic loss function, and the test
statistic is $\text{DM}_{P}=P^{-1}\sum\limits_{t=1}^{P}\frac{d_{t+h}}{\hat{%
\sigma}_{\bar{d}}}$, where $d_{t+h}=[\hat{\epsilon}_{t+h}^{(1)}]^{2}-[\hat{%
\epsilon}_{t+h}^{(2)}]^{2}$, $\bar{d}$ denotes the mean of $d_{t+h}$, $\hat{%
\sigma}_{\bar{d}}$ is a heteroskedasticity and autocorrelation consistent
estimate of the standard deviation of $\bar{d}$, and $P$ denotes the number
of ex-ante predictions used to construct the test statistic.\footnote{%
In this paper, we report test results for the Wald version of this test
statistic (see Giacomini and White (2006) for further details).} If the null
hypothesis is rejected and the relative MSFE is greater than 1, then the AR
benchmark is preferred.

Our main empirical findings are gathered in Table 3 and Table 4. In these
tables, all entries are relative MSFEs, with our AR(SIC) benchmark in the
denominator. Additionally, bolded entries indicate the \textquotedblleft
MSFE-best" method for a particular target variable, forecast horizon, and
estimation window type (i.e. rolling or recursive). Starred entries denote
rejection of the null hypothesis of equal forecast accuracy when comparing
the listed model against the AR(SIC) benchmark. Note that Table 3 reports
the relative MSFE results for the case where in choosing the number of
factors using the $PC_{p2}$ criterion in Bai and Ng (2002), we set the
maximum allowable number of factors to be 4, whereas Table 4 reports the
results for the case where the maximum allowable number of factors is 8.
Moreover, the first three columns of results given in Tables 3 and 4 are for
the case where the estimation is conducted using a recursive data window,
while the last three columns of results in both tables are for the case
where estimation is conducted using a rolling data window. A number of
conclusions can be drawn from examining the results in these tables.
Focusing on Table 3, we see that when a recursive data window is used, then
we see that CS has a smaller MSFE than both PCA and HT in 16 out of the 32
possible cases, when comparing the MSFE results of PCA, HT, and CS across 8
different target variables and 4 different forecast horizons. On the other
hand, HT is the top performer in 11 out of 32 cases while PCA wins in 5 out
of 32 cases. The results are even better when we adopt a rolling data
window. In this case, CS beats both HT and PCA in 24 out of 32 cases,
whereas HT wins in 5 out of 32 cases and PCA only wins in 3 out of 32 cases.
Turning our attention to Table 4, we see that in the case where a recursive
data window is used, CS beats both HT and PCA in 18 out of 32 cases, while
HT wins in 7 out of 32 cases and PCA wins in 6 out of 32 cases, with HT and
PCA sharing first place in the one remaining case. Moreover, the results
reported in Table 4 for the case where a rolling data window is used find CS
winning in 19 out of 32 cases, HT winning in 7 out of 32 cases, and PCA
winning in 6 out of 32 cases. Finally, if we compare CS directly with PCA
across 8 different target variables, 4 forecast horizons, 2 choices of data
window specification (recursive or rolling), and 2 choices of maximum
allowable number of factors (4 or 8); we see that CS outperforms PCA in 97
out of the 128 possible cases considered here, whereas PCA only outperforms
CS in 31 cases. The latter results also provide some additional evidence
that the inclusion of the irrelevant variables actually has a non-negligible
influence on our estimation and forecasting results; since, if the number of
irrelevant variables $N_{2}$ is negligibly small so that their inclusion has
only a trivial impact; then, we would expect more balance with respect to
the number of cases for which PCA beats CS versus the number of cases for
which CS beats PCA.

To summarize, overall CS wins more frequently than either HT or PCA.
However, note that no method uniformly beats all of the competing methods
across all settings, which should not be surprising when one applies
econometric methods to real world data. The goal of this empirical
illustration is simply to provide evidence that the CS variable selection
procedure introduced in this paper can be useful for FRED-MD forecasting.
Needless to say, our results are not meant to imply that the other methods
are not also useful. Rather, we believe that the CS variable selection
method should be added to the \textquotedblleft toolbox\textquotedblright\
of methods used when constructing macroeconomic forecasts.

\subsection{Empirical Illustration 2 - Global VAR Modelling Dataset}

In this illustration, we forecast 5 target variables from the quarterly
real-time macroeconomic Global VAR Modelling (GVAR) dataset maintained by L.
Vanessa Smith.\footnote{%
see https://sites.google.com/site/gvarmodelling/home} The GVAR dataset
extends the original one used in Dees, di Mauro, Pesaran and Smith (2007),
which covers the period 1979:Q1-2003:Q4 (see also Pesaran, Schuermann and
Smith (2009)). More specifically, the dataset includes 4 vintages of
real-time data on 6 variables for 33 countries. The four vintages include
data collected in 2011 for the period 1979:Q1-2011:Q2, data collected in
2013 for the period 1979:Q1-2013:Q1, data collected in 2017 for the period
1979:Q1-2016:Q4, and data collected in 2023 for the period 1979:Q1-2023:Q3.
The 5 variables that we forecast include GDP growth, inflation, equity
returns, short-term interest rates, and long-term interest. These variables
are predicted for 6 different countries, including the USA, the UK, Germany,
France, Italy, and Japan. Additionally, the GVAR dataset to which we apply
the CS and HT variable selection methods prior to constructing factors
includes 6 variables for each of 33 countries for a total of 198 variables.
The conventional PCA procedure does not involve variable pre-screening, so
it uses this entire set of variables (again excluding the target variable of
interest) in estimating the factors.

When constructing predictions, we use a variant of the forecasting model
analyzed in the previous empirical illustration, specified as follows: 
\begin{equation}
y_{t+h}={\alpha }+{\beta _{h}}(L)y_{t}+{\gamma _{h}}(L)F\boldsymbol{_{t}}%
+\delta {_{h}}(L)W\boldsymbol{_{t}+}\epsilon _{t+h}.
\end{equation}%
All terms in the above equation are defined in the previous empirical
illustration, with the exception of ${\delta }{_{h}}(L)W\boldsymbol{_{t}}$,
where $W\boldsymbol{_{t}}$ contains the 5 target macroeconomic variables for
each of the 6 countries in our analysis, and ${\delta }{_{h}}(L)$ is a
conformably defined lag polynomial. The forecasting component of this
illustration follows the approach of Pesaran, Schuermann and Smith (2009),
where forecasts for the last 8 periods of our sample period are constructed,
and models are ranked based on comparison of MSFEs that are constructed by
averaging ranks across all 5 variables. Our sample period is
1979:Q1-2023:Q3, so that only the latest vintage of data is used in our
analysis, and the ex ante forecasting period is 2021:Q4-2023:Q3.\footnote{%
Note that Pesaran, Schuermann and Smith (2009) use an earlier vintage of
data in their forecasting analysis, and not the more recently available
vintage ending in 2023:Q3 that we use. Additionally, their analysis and ours
do not use a new vintage of data for the construction of each new forecast,
and hence are not truly real-time in the sense discussed in our previous
empirical illustration. For this reason, we also carried out a alternative
\textquotedblleft real-time\textquotedblright\ version of the experiment
reported on in this section by constructing 1 to 4 step ahead forecasts
using each of the first three vintages of data in the GVAR dataset (the 4th
vintage of data was used as the \textquotedblleft fully revised" data
against which all forecasts were compared when constructing forecast
errors). More specifically, the 1979:Q1-2011:Q2 vintage of data was used to
construct $h$=1,...,4 step ahead forecasts beginning with the period
2011:Q3, the 1979:Q1-2013:Q1 vintage of data was used to construct $h$%
=1,...,4 step ahead forecasts beginning with the period 2013:Q2, and the
1979:Q1-2016:Q4 vintage of data was used to construct $h$=1,...,4 step ahead
forecasts beginning with the period 2017:Q1. Results from this experiment
yield rankings that are identical to those reported on in the sequel, and
are available upon request from the authors.} The models that we use when
constructing include PCA, CS, HT, PCA+macro, CS+macro, and HT+macro. The
first three models (i.e., PCA, CS, and HT) are defined above. The latter
three models include lags of the 5 macroeconomic variables as explanatory
variables, rather than just lags of the target variable, with lags selected
using the SIC. All models are estimated recursively, prior to the
construction of each new forecast, using the same procedures described in
the previous section of this paper. Additionally, models are estimated using
PCA, followed by least squares, with lags selected using the SIC and the
number of factors selected using the $PC_{p2}$ criterion of Bai and Ng
(2002).

A summary of our results is presented in Table 5, where the average model
ranks are reported. These ranks are determined by comparing the MSFEs
produced by the different methods for a given country, target variable, and
forecast horizon. Model ranks are defined as \textquotedblleft 1" for the
lowest MSFE model, \textquotedblleft 2\textquotedblright\ for the second
\textquotedblleft lowest\textquotedblright\ MSFE model, etc. For example, if
a model, say PCA, yields the lowest MSFE for French GDP growth at the $h$=1
step ahead horizon, PCA is assigned a \textquotedblleft 1" as it is the
lowest MSFE model for that particular country, horizon, and variable
permutation. Just as is done in Pesaran, Schuermann and Smith (2009), the
model rankings reported in Table 5 average the individual rankings across
all 5 forecasting variables. For example, note that the first numerical
entry in the Table 5 is 2.4. This means that for the USA, on average, the
PCA model ranks 2.4, when averaging the rank of the PCA model across all
five of the variables that we forecast.

A number of conclusions can be made by examining the results provided in
Table 5. First, in the top panel of Table 5, which summarizes results for $%
h=1$, we see that the CS model is the top performer for 2 of 6 countries
(i.e., Germany and Japan). This increases to 4 of 6 countries when counting
the number of times that the CS model \textquotedblleft
wins\textquotedblright\ or comes in second place, with the USA and France
constituting the additional two countries. For the 2-step ahead horizon
(refer to Panel 2), the CS model \textquotedblleft wins\textquotedblright\
for 4 of 6 countries, and the CS+macro model wins for 1 country. For this
forecast horizon, the only country for which the top performer is not a
CS-type model is Germany, as HT outperforms all other methods in forecasting
German data in the $h=2$ case. The results are \textbf{roughly} similar when
assessing the number of CS model wins for the additional two forecast
horizons, so that overall the CS model is the top performer in terms of the
number of times it achieves the highest average ranking. Second, the model
which comes in second place in terms of the number of times that it achieves
the highest ranking is PCA. More specifically, PCA \textquotedblleft
wins\textquotedblright\ in 8 of 24 cases and also ties for first place with
HT in one other case, when comparing results across all 6 countries and 4
forecast horizons. For comparison, note that the two CS-type models (i.e.,
CS and CS+macro) together \textquotedblleft win\textquotedblright\ in 12 of
24 cases, with the CS model winning in 11 of 24 cases and the CS+macro model
winning in one additional case. Moreover, if we were to compare CS directly
with PCA across results given for the 6 different countries and 4 different
forecast horizons, we see that CS outperforms PCA in terms of average rank
in 14 out of the 24 possible cases. Third, observe that our models which
include additional macroeconomic explanatory variables do not yield superior
predictions, when compared against more parsimonious models that include
only factors and lags of the target variable being predicted. This is as
expected, given how heavily parameterized our models with additional
explanatory variables are. In summary, this experiment yields further
evidence that the CS variable selection method is useful when specifying
commonly used factor augmented (vector) autoregression type time series
models.

\section{\noindent Conclusion}

In this paper, we present a novel and completely consistent approach for
variable selection in high dimensional factor estimation problems. Our
method can be useful in contexts where not all available variables actually
load on the underlying factors so that the variables which do not load can
be considered to be irrelevant in the sense that they contribute only noise
and not signal to the factor estimation process. We show that our variable
selection procedure allows for the consistent estimation of the conditional
mean of a factor-augmented forecasting equation based on a FAVAR model, even
in cases where the number of irrelevant variables may be quite substantial.
Our new variable selection procedure is based on a self-normalized score
statistic, and it correctly identifies the set of variables which load
significantly on the underlying factors, with probability approaching one,
as the sample sizes go to infinity. Our theoretical analysis suggests that
our method for variable selection may be a useful complement to extant
methods for pre-screening variables prior to latent factor estimation. Some
support for this conclusion is also given in the form of a Monte Carlo
experiment indicating that the variables selection method has very small
false positive and false negative rates even for samples that are not very
large. In addition, we present two empirical illustrations which show the
good forecast performance of our methodology when compared to the
conventional PCA and hard thresholding procedures for variable selection.

\newpage

\bigskip

\begin{thebibliography}{99}
\bibitem{} Andrews, D.W.K. (1984): \textquotedblleft Non-strong Mixing
Autoregressive Processes,\textquotedblright\ \textit{Journal of Applied
Probability}, 21, 930-934.

\bibitem{} Bai, J. and S. Ng (2002): \textquotedblleft Determining the
Number of Factors in Approximate Factor Models,\textquotedblright\ \textit{%
Econometrica}, 70, 191-221.

\bibitem{} Bai, J. (2003): \textquotedblleft Inferential Theory for Factor
Models of Large Dimensions,\textquotedblright\ \textit{Econometrica}, 71,
135-171.

\bibitem{} Bai, J. and S. Ng (2002): \textquotedblleft Determining the
Number of Factors in Approximate Factor Models,\textquotedblright\ \textit{%
Econometrica}, 70, 191-221.

\bibitem{} Bai, J. and S. Ng (2008): \textquotedblleft Forecasting Economic
Time Series Using Targeted Predictors,\textquotedblright\ \textit{Journal of
Econometrics}, 146, 304-317.

\bibitem{} Bai, J. and S. Ng (2023): \textquotedblleft Approximate Factor
Models with Weaker Loading,\textquotedblright\ \textit{Journal of
Econometrics}, 235. 1893-1916.

\bibitem{} Bai, Z. D. and Y. Q. Yin (1993): \textquotedblleft Limit of the
Smallest Eigenvalue of a Large Dimensional Sample Covariance Matrix," 
\textit{Annals of Probability}, 21, 1275-1294.

\bibitem{} Bair, E., T. Hastie, D. Paul, and R. Tibshirani (2006):
\textquotedblleft Prediction by Supervised Principal
Components,\textquotedblright\ \textit{Journal of the American Statistical
Association}, 101, 119-137.

\bibitem{} Barshan, E., A. Ghodsi, Z. Azimifar, and M.Z. Jahromi (2011):
\textquotedblleft Supervised Principal Component Analysis: Visualization,
Classification and Regression on Subspaces and
Submanifolds,\textquotedblright\ \textit{Pattern Recognition}, 44, 1357-1371.

\bibitem{} Belloni, A., D. Chen, V. Chernozhukov, and C. Hansen (2012):
\textquotedblleft Sparse Models and Methods for Optimal Instruments with an
Application to Eminent Domain,\textquotedblright\ \textit{Econometrica}, 80,
2369-2429.

\bibitem{} Belloni, A., V. Chernozhukov, and C. Hansen (2014):
\textquotedblleft Inference on Treatment Effects After Selection Among
High-Dimensional Controls,\textquotedblright\ \textit{Review of Economic
Studies}, 81, 608-650.

\bibitem{} Belloni, A., V. Chernozhukov, and L. Wang (2014):
\textquotedblleft Pivotal Estimation via Square-Root Lasso in Nonparametric
Regression,\textquotedblright\ \textit{Annals of Statistics}, 42, 757-788.

\bibitem{} Billingsley, P. (1995): \textit{Probability and Measure}. New
York: John Wiley \& Sons.

\bibitem{} Borovkova, S., R. Burton, and H. Dehling (2001):
\textquotedblleft Limit Theorems for Functionals of Mixing Processes to
U-Statistics and Dimension Estimation,\textquotedblright\ \textit{%
Transactions of the American Mathematical Society}, 353, 4261-4318.

\bibitem{} Breiman, L. (1995). \textquotedblleft Better Subset Regression
Using the Nonnegative Garrote,\textquotedblright\ \textit{Technometrics},
37, 373-384.

\bibitem{} Breiman, L. (1996). \textquotedblleft Bagging Predictors," \ 
\textit{Machine Learning}, 26, \ 123-140.

\bibitem{} Breiman, L. (2001). \textquotedblleft Random Forests," \textit{%
Machine Learning}, 45, 5-32.

\bibitem{} Chao, J. C., K. Qiu, and N. R. Swanson (2023): \textquotedblleft
Consistent Factor Estimation and Forecasting in Factor-Augmented VAR
Models," University of Maryland and Rutgers University Working Paper.

\bibitem{} Carrasco, M. and B. Rossi (2016): \textquotedblleft In-sample
Inference and Forecasting in Misspecified Factor Models,\textquotedblright\ 
\textit{Journal of Business \& Economic Statistics,} 34, 313-338.

\bibitem{} Chen, X., Q. Shao, W. B. Wu, and L. Xu (2016): \textquotedblleft
Self-normalized Cram\'{e}r-type Moderate Deviations under
Dependence,\textquotedblright\ \textit{Annals of Statistics}, 44, 1593-1617.

\bibitem{} Davidson. J. (1994): \textit{Stochastic Limit Theory: An
Introduction for Econometricians}. New York: Oxford University Press.

\bibitem{} Davidson, K. R. and S. J. Szarek (2001): \textquotedblleft Local
Operator Theory, Random Matrices and Banach Spaces.\textquotedblright\ In 
\textit{Handbook of the Geometry of Banach Spaces}, 1, 317-366. Amsterdam:
North-Holland.

\bibitem{} Dees, S., F. Di Mauro, M.H. Pesaran, and L.V. Smith (2007):
\textquotedblleft Exploring the International Linkages of the Euro Area: A
Global VAR Analysis,\textquotedblright\ \textit{Journal of Applied
Econometrics, }22, 1-38.

\bibitem{} Diebold, F.X. and R.S. Mariano (1995): \textquotedblleft
Comparing Predictive Accuracy,\textquotedblright\ \textit{Journal of
Business and Economic Statistics}, 20, 134-144.

\bibitem{} Fan, J., Y. Ke, and Y. Liao (2021): \textquotedblleft Augmented
Factor Models with Applications to Validating Market Risk Factors and
Forecasting Bond Risk Premia,\textquotedblright\ \textit{Journal of
Econometrics}, 222, 269-294.

\bibitem{} Forni, M., M. Hallin, M. Lippi, and L. Reichlin (2005):
\textquotedblleft The Generalized Dynamic Factor Model, One-Sided Estimation
and Forecasting,\textquotedblright\ \textit{Journal of the American
Statistical Association}, 100, 830-840.

\bibitem{} Freund, Y. and R. Schapire (1997). \textquotedblleft A
Decision-Theoretic Generalization of Online Learning and an Application to
Boosting, " \textit{Journal of Computer and System Sciences}, 55, 119-139.

\bibitem{} Giacomini, R. and H. White (2006): \textquotedblleft Tests of
Conditional Predictive Ability,\textquotedblright\ \textit{Econometrica},
74, 1545-1578.

\bibitem{} Giglio, S., D. Xiu, and D. Zhang (2023a): \textquotedblleft Test
Assets and Weak Factors,\textquotedblright\ \textit{Joural of Finance},
forthcoming.

\bibitem{} Giglio, S., D. Xiu, and D. Zhang (2023b): \textquotedblleft
Prediction When Factors Are Weak,\textquotedblright\ Working Paper, Yale
School of Management and the Booth School of Business, University of Chicago.

\bibitem{} Golub, G. H. and C. F. van Loan (1996): \textit{Matrix
Computations}, 3rd Edition. Baltimore: The Johns Hopkins University Press.

\bibitem{} Goroketskii, V. V. (1977): \textquotedblleft On the Strong Mixing
Property for Linear Sequences,\textquotedblright\ \textit{Theory of
Probability and Applications}, 22, 411-413.

\bibitem{} Horn, R. and C. Johnson (1985): \textit{Matrix Analysis}.
Cambridge University Press.

\bibitem{} Johnstone, I. M. and A. Lu (2009): \textquotedblleft On
Consistency and Sparsity for Principal Components Analysis in High
Dimensions,\textquotedblright\ \textit{Journal of the American Statistical
Association}, 104, 682-697.

\bibitem{} Johnstone, I. M. and D. Paul (2018): \textquotedblleft PCA in
High Dimensions: An Orientation,\textquotedblright\ \textit{Proceedings of
the IEEE}, 106, 1277-1292.

\bibitem{} Kelly, B. and S. Pruitt (2015): \textquotedblleft The Three-Pass
Regression Filter: A New Approach to Forecasting Using Many
Predictors,\textquotedblright\ J\textit{ournal of Econometrics,} 186,
294--316.

\bibitem{} Kim, H.-H. and N.R. Swanson (2014): \textquotedblleft Forecasting
Financial And Macroeconomic Variables Using Data Reduction Methods: New
Empirical Evidence,\textquotedblright\ \textit{Journal of Econometrics,}
178, 352--367.

\bibitem{} Kim, H.-H. and N.R. Swanson (2018): \textquotedblleft Methods for
Backcasting, Nowcasting and Forecasting Using Factor-MIDAS: With an
Application to Korean GDP,\textquotedblright\ \textit{Journal of Forecasting}%
, 37, 281-302.

\bibitem{} Lee, T.-H., A. Ullah, and R. Wang 2020): \textquotedblleft
Bootstrap Aggregating and Random Forest,\textquotedblright\ in \textit{%
Macroeconomic Forecasting in the Era of Big Data Theory and Practice}, eds.
Fuleky, P., New York: Springer.

\bibitem{} Lee, T.-H. and Y. Yang, \textquotedblleft Bagging Binary and
Quantile Predictors for Time Series\textquotedblright , \textit{Journal of
Econometrics} 135, 465-497.

\bibitem{} L\"{u}tkepohl, H. (2005): \textit{New Introduction to Multiple
Time Series Analysis}. New York: Springer.

\bibitem{} McCracken, M.W. and S. Ng (2016): \textquotedblleft FRED-MD: A
Monthly Database for Macroeconomic Research,\textquotedblright\ \textit{%
Journal of Business and Economic Statistics}, 34, 574-589.

\bibitem{} Nadler, B. (2008): \textquotedblleft Finite Sample Approximation
Results for Principal Component Analysis: A Matrix Perturbation
Approach,\textquotedblright\ \textit{Annals of Statistics}, 36, 2791-2817.

\bibitem{} Paul, D. (2007): \textquotedblleft Asymptotics of Sample
Eigenstructure for a Large Dimensional Spiked Covariance
Model,\textquotedblright\ \textit{Statistica Sinica}, 17, 1617-1642.

\bibitem{} Pesaran, M.H., T. Schuermann, and L.V. Smith (2009):
\textquotedblleft Forecasting Economic and Financial Variables With Global
VARs,\textquotedblright\ \textit{International Journal of Forecasting}, 25,
642-675.

\bibitem{} Pham, T. D. and L. T. Tran (1985): \textquotedblleft Some Mixing
Properties of Time Series Models,\textquotedblright\ \textit{Stochastic
Processes and Their Applications}, 19, 297-303.

\bibitem{} Qiu, A. and Z. Qu (2021): \textquotedblleft Modeling Regime
Switching in High-Dimensional Data with Applications to U.S. Business
Cycles,\textquotedblright\ Working Paper, Boston University.

\bibitem{} Ruhe, A. (1975): \textquotedblleft On the Closeness of
Eigenvalues and Singular Values for Almost Normal
Matrices,\textquotedblright\ \textit{Linear Algebra and Its Applications},
11, 87-94.

\bibitem{} Shen, D., H. Shen, H. Zhu, J.S. Marron (2016): \textquotedblleft
The Statistics and Mathematics of High Dimension Low Sample Size
Asymptotics,\textquotedblright\ \textit{Statistica Sinica}, 26, 1747-1770.

\bibitem{} Stewart, G.W. (1973): \textquotedblleft Error and Perturbation
Bounds for Subspaces Associated with Certain Eigenvalue
Problems,\textquotedblright\ \textit{SIAM Review}, 15, 727-764.

\bibitem{} Stewart, G.W. and J. Sun (1990): \textit{Matrix Perturbation
Theory}. Boston: Academic Press.

\bibitem{} Stock, J. H. and M. W. Watson (2002a): \textquotedblleft
Forecasting Using Principal Components from a Large Number of
Predictors,\textquotedblright\ \textit{Journal of the American Statistical
Association}, 97, 1167-1179.

\bibitem{} Stock, J. H. and M. W. Watson (2002b): \textquotedblleft
Macroeconomic Forecasting Using Diffusion Indexes,\textquotedblright\ 
\textit{Journal of Business and Economic Statistics}, 20, 147-162.

\bibitem{} Swanson, N.R. (1996): \textquotedblleft Forecasting Using
First-Available Versus Fully Revised Economic Time-Series
Data,\textquotedblright\ \textit{Studies in Nonlinear Dynamics and
Econometrics}, 1, 47-64.

\bibitem{} Swanson, N.R. and D. van Dijk (2006): \textquotedblleft Are
Statistical Reporting Agencies Getting It Right? Data Rationality and
Business Cycle Asymmetry,\textquotedblright\ \textit{Journal of Business and
Economic Statistics}, 24, 24-42.

\bibitem{} Tibshirani, R. (1996): \textquotedblleft Regression Shrinkage and
Selection Via the Lasso,\textquotedblright\ \textit{Journal of the Royal
Statistical Society, Series B}, 58, 267-288.

\bibitem{} Tu, Y. and Lee, T.-H. (2019): \textquotedblleft Forecasting Using
Supervised Factor Models,\textquotedblright\ \textit{Journal of Management
Science and Engineering, }4, 12-27.

\bibitem{} Vershynin, R. (2012): \textquotedblleft Introduction to the
Non-asymptotic Analysis of Random Matrices,\textquotedblright\ In \textit{%
Compressed Sensing}, \textit{Theory and Applications, }210-268. Cambridge
University Press.

\bibitem{} Wang, W. and J. Fan (2017): \textquotedblleft Asymptotics of
Empirical Eigenstructure for High Dimensional Spiked
Covariance,\textquotedblright\ \textit{Annals of Statistics}, 45, 1342-1374.

\bibitem{} Zou, H.D., T. Hastie, and R. Tibshirani (2006): \textquotedblleft
Sparse Principal Component Analysis,\textquotedblright\ \textit{Journal of
Computational and Graphical Statistics, }15, 265-286.

\bibitem{} Zhou, Z. and X. Shao (2013): \textquotedblleft Inference for
Linear Models with Dependent Errors,\textquotedblright\ \textit{Journal of
the Royal Statistical Society Series B}, 75, 323-343.
\end{thebibliography}

\pagebreak

\begin{table}[h]
\begin{flushleft}
{{{Table 1: Monte Carlo Results for Variable Selection Using $\mathbb{S}%
_{i,T}^{+}=\dsum\nolimits_{\ell =1}^{d}\varpi _{\ell }\left\vert S_{i,\ell
,T}\right\vert $ Statistic}}{$^{\ast }$} }
\par
\vspace{0.3cm} \hspace{0.75in} \vspace{0.5in} {\small 
\begin{tabular}{|l|l|c|c|c|c|c|c|}
\hline
&  & $N=100$ & $N_{1}=50$ & $T=100$ & $\tau =5$ &  &  \\ \hline
&  & \multicolumn{1}{|l|}{{\scriptsize {$\varphi =\left( \ln \ln N\right)
^{-0.1}$}}} & \multicolumn{1}{|l|}{{\scriptsize {$\varphi =\left( \ln \ln
N\right) ^{-0.5}$}}} & \multicolumn{1}{|l|}{{\scriptsize {$\varphi =\left(
\ln \ln N\right) ^{-1}$}}} & \multicolumn{1}{|l|}{$\varphi =N^{-0.2}$} & $%
\varphi =N^{-0.4}$ & $\varphi =N^{-0.6}$ \\ \hline\hline
$\tau _{1}=2$ & FPR & 0.03916 & 0.03350 & 0.02678 & 0.01460 & 0.00382 & 
0.00076 \\ \hline
& FNR & 0.00046 & 0.00068 & 0.00104 & 0.00284 & 0.01674 & 0.09412 \\ \hline
$\tau _{1}=3$ & FPR & 0.04544 & 0.03902 & 0.03110 & 0.01810 & 0.00526 & 
0.00092 \\ \hline
& FNR & 0.00022 & 0.00032 & 0.00052 & 0.00172 & 0.01100 & 0.06942 \\ \hline
$\tau _{1}=4$ & FPR & 0.05408 & 0.04650 & 0.03756 & 0.02224 & 0.00702 & 
0.00162 \\ \hline
& FNR & 0.00016 & 0.00024 & 0.00034 & 0.00118 & 0.00828 & 0.05194 \\ \hline
$\tau _{1}=5$ & FPR & 0.06332 & 0.05462 & 0.04558 & 0.02796 & 0.00924 & 
0.00232 \\ \hline
& FNR & 0.00014 & 0.00018 & 0.00034 & 0.00084 & 0.00574 & 0.03948 \\ \hline
&  & $N=200$ & $N_{1}=100$ & $T=100$ & $\tau =5$ &  &  \\ \hline
$\tau _{1}=2$ & FPR & 0.01913 & 0.01470 & 0.01068 & 0.00486 & 0.00064 & 
0.00002 \\ \hline
& FNR & 0.00206 & 0.00282 & 0.00449 & 0.01415 & 0.09966 & 0.48356 \\ \hline
$\tau _{1}=3$ & FPR & 0.02341 & 0.01842 & 0.01365 & 0.00657 & 0.00098 & 
0.00005 \\ \hline
& FNR & 0.00143 & 0.00190 & 0.00315 & 0.00921 & 0.07372 & 0.40894 \\ \hline
$\tau _{1}=4$ & FPR & 0.02869 & 0.02306 & 0.01733 & 0.00841 & 0.00133 & 
0.00004 \\ \hline
& FNR & 0.00111 & 0.00145 & 0.00224 & 0.00661 & 0.05564 & 0.34279 \\ \hline
$\tau _{1}=5$ & FPR & 0.03506 & 0.02903 & 0.02194 & 0.01124 & 0.00213 & 
0.00017 \\ \hline
& FNR & 0.00086 & 0.00112 & 0.00172 & 0.00477 & 0.04258 & 0.28620 \\ \hline
&  & $N=400$ & $N_{1}=200$ & $T=200$ & $\tau =10$ &  &  \\ \hline
\multicolumn{1}{|c|}{$\tau _{1}=5$} & FPR & 0.00214 & 0.00148 & 0.00090 & 
0.00030 & 2.5$\times $10$^{-5}$ & 0.00000 \\ \hline
\multicolumn{1}{|c|}{} & FNR & 7.5$\times $10$^{-5}$ & 0.00016 & 0.00040 & 
0.00231 & 0.06894 & 0.67266 \\ \hline
\multicolumn{1}{|c|}{$\tau _{1}=6$} & FPR & 0.00249 & 0.00166 & 0.00104 & 
0.00034 & 0.00002 & 0.00000 \\ \hline
\multicolumn{1}{|c|}{} & FNR & 0.00004 & 0.00009 & 0.00025 & 0.00148 & 
0.05058 & 0.60968 \\ \hline
\multicolumn{1}{|c|}{$\tau _{1}=8$} & FPR & 0.00337 & 0.00235 & 0.00142 & 
0.00046 & 0.00004 & 0.00000 \\ \hline
\multicolumn{1}{|c|}{} & FNR & 0.00001 & 0.00002 & 0.00008 & 0.00068 & 
0.02712 & 0.48133 \\ \hline
\multicolumn{1}{|c|}{$\tau _{1}=10$} & FPR & 0.00484 & 0.00350 & 0.00220 & 
0.00079 & 7.5$\times $10$^{-5}$ & 5.0$\times $10$^{-6}$ \\ \hline
\multicolumn{1}{|c|}{} & FNR & 0.00001 & 0.00001 & 0.00002 & 0.00034 & 
0.01535 & 0.36382 \\ \hline
&  & $N=1000$ & $N_{1}=500$ & $T=600$ & $\tau =12$ &  &  \\ \hline
\multicolumn{1}{|c|}{$\tau _{1}=6$} & FPR & 0.00155 & 0.00121 & 0.00086 & 
0.00038 & 0.00006 & 0.00001 \\ \hline
\multicolumn{1}{|c|}{} & FNR & 0.00000 & 0.00000 & 0.00000 & 0.00000 & 
0.00000 & 0.00000 \\ \hline
\multicolumn{1}{|c|}{$\tau _{1}=8$} & FPR & 0.00201 & 0.00153 & 0.00106 & 
0.00049 & 8.2$\times $10$^{-5}$ & 1.4$\times $10$^{-5}$ \\ \hline
\multicolumn{1}{|c|}{} & FNR & 0.00000 & 0.00000 & 0.00000 & 0.00000 & 
0.00000 & 0.00000 \\ \hline
\multicolumn{1}{|c|}{$\tau _{1}=10$} & FPR & 0.00274 & 0.00216 & 0.00155 & 
0.00072 & 0.00016 & 3.2$\times $10$^{-5}$ \\ \hline
\multicolumn{1}{|c|}{} & FNR & 0.00000 & 0.00000 & 0.00000 & 0.00000 & 
0.00000 & 0.00000 \\ \hline
\multicolumn{1}{|c|}{$\tau _{1}=12$} & FPR & 0.00421 & 0.00332 & 0.00242 & 
0.00115 & 0.00028 & 6.0$\times $10$^{-5}$ \\ \hline
\multicolumn{1}{|c|}{} & FNR & 0.00000 & 0.00000 & 0.00000 & 0.00000 & 
0.00000 & 0.00000 \\ \hline
\end{tabular}
} 
\begin{tabular}{llcccccc}
\multicolumn{8}{l}{* Notes: False positive and negative rates are reported
for various values of $N, N_1,$ and $T$. Results} \\ 
\multicolumn{8}{l}{are based on 1000 simulations. See Section 3 for complete
details.} \\ 
&  &  &  &  &  &  & 
\end{tabular}%
\end{flushleft}
\end{table}
\newpage

\begin{table}[h]
{\small \noindent \textbf{Table 2: }Empirical Illustration 1 - FRED MD
Dataset: Target Forecast Variables{$^{\ast }$} }
\par
\vspace{0.3cm} {\ \ 
\begin{tabular}{lll}
\hline\hline
Target Variable & Abbreviation & Data Transformation \\ \hline
Industrial Production & INDPRO & $\Delta log(y_t)$ \\ 
Civilian Unemployment Rate & UNRATE & $y_t$ \\ 
Housing Starts (new, privately owned) & HOUST & $log(y_t)$ \\ 
Housing Permits (new, privately owned) & PERMIT & $log(y_t)$ \\ 
Real M2 Money Stock & M2REAL & $\Delta log(y_t)$ \\ 
10-Year Government Treasury Bond Rate & GS10 & $y_t$ \\ 
CPIAUCSL (all items) & CPI & $\Delta log(y_t)$ \\ 
S\&P Common Stock Price Index (composite) & S\&P500 & $\Delta log(y_t)$ \\ 
\hline\hline
\end{tabular}
}
\par
{\scriptsize \vspace{0.5cm} }
\par
{\scriptsize \noindent $^*$ Notes: This table lists the target forecast
variables that are predicted in our empirical illustration, \noindent and
associated data transformations. }
\end{table}

\newpage 
\begin{table}[h]
\renewcommand\arraystretch{1.5} {\small \noindent \textbf{Table 3: }%
Empirical Illustration 1 - FRED MD Dataset - Forecasting Results Based on
the Use of Alternative Variable Selection Methods With a Maximum of 4
Factors (Forecast Period 2000:1-2024:6){$^{\ast }$} } \vspace{0.3cm} %
\centering{\tiny \ 
\begin{tabular}{p{0.5cm}p{1.4cm}p{1.4cm}p{1.4cm}p{1.4cm}p{1.4cm}p{1.4cm}p{1.4cm}}
\hline\hline
&  & \multicolumn{3}{c}{Estimation Uses Recursive Data Window} & 
\multicolumn{3}{c}{Estimation Uses Rolling Data Window} \\ 
& \multicolumn{1}{c}{Target Variable} & \multicolumn{1}{l}{PCA} & 
\multicolumn{1}{l}{HT} & \multicolumn{1}{l}{CS} & \multicolumn{1}{l}{PCA} & 
\multicolumn{1}{l}{HT} & \multicolumn{1}{l}{CS} \\ \hline
& INDPRO & \textbf{0.924} & 0.926 & 0.926 & \textbf{0.959} & 0.990 & 0.964
\\ 
& UNRATE & 0.576 & 0.637 & \textbf{0.568} & 0.423 & \textbf{0.417} & 0.455
\\ 
& HOUST & 1.002 & \textbf{0.999} & 1.003 & 0.999 & \textbf{0.992} & 1.043 \\ 
\multicolumn{1}{l}{h=1} & PERMIT & 1.033 & 1.020 & \textbf{0.973} & 0.959 & 
0.950 & \textbf{0.873} \\ 
& M2REAL & \textbf{0.999} & 1.020 & 1.001 & 1.094 & 1.021 & \textbf{1.015}
\\ 
& GS10 & 1.102 & 1.086 & \textbf{1.065} & 1.069 & 1.074 & \textbf{1.019} \\ 
& CPI & 1.003 & 1.004 & \textbf{1.002} & 1.097 & \textbf{0.992} & 1.031 \\ 
& S\&P500 & 1.061 ** & 1.111 ** & \textbf{1.054 ***} & 1.126 & 1.453 ** & 
\textbf{1.079} \\ \hline
& INDPRO & 1.036 & 1.052 & \textbf{1.016} & 1.078 & \textbf{0.964} & 1.032
\\ 
& UNRATE & \textbf{0.564} & 0.586 & 0.580 & 0.801 & 0.801 & \textbf{0.779}
\\ 
& HOUST & 0.996 & \textbf{0.985} & 0.988 & 1.004 & 1.005 & \textbf{0.995} \\ 
\multicolumn{1}{l}{h=3} & PERMIT & 1.001 & \textbf{0.982} & 0.986 & 1.009 & 
\textbf{0.995 *} & 0.998 \\ 
& M2REAL & 1.008 & 1.006 & \textbf{0.995} & 1.075 & 1.060 & \textbf{1.037 *}
\\ 
& GS10 & 1.188 ** & \textbf{1.135 *} & 1.168 & 1.076 & 1.090 & \textbf{0.962}
\\ 
& CPI & 0.971 & 0.967 & \textbf{0.960} & 0.976 & 0.970 & \textbf{0.966 *} \\ 
& S\&P500 & 1.033 *** & \textbf{1.000} & 1.025 ** & 1.068 & 1.094 & \textbf{%
1.04} \\ \hline
& INDPRO & 1.047 * & 1.048 & \textbf{1.031} & 1.231 & 1.231 & \textbf{1.038}
\\ 
& UNRATE & 0.786 *** & 0.813 *** & \textbf{0.753 ***} & 0.733 & 0.733 & 
\textbf{0.726} \\ 
& HOUST & 1.003 & \textbf{0.976} & 0.986 & 1.014 & 0.994 & \textbf{0.980} \\ 
\multicolumn{1}{l}{h=6} & PERMIT & 1.014 & \textbf{0.975} & 0.983 & 1.019 & 
0.987 & \textbf{0.981} \\ 
& M2REAL & 1.008 & 1.007 & \textbf{1.000} & 1.023 & 1.008 & \textbf{1.002}
\\ 
& GS10 & 1.175 * & \textbf{1.085} & 1.144 & 1.092 & 1.099 & \textbf{1.044}
\\ 
& CPI & \textbf{0.995} & 1.006 & 1.003 & 1.012 & 0.989 & \textbf{0.981} \\ 
& S\&P500 & \textbf{1.026 *} & 1.038 & 1.034 * & 1.178 & 1.085 & \textbf{%
1.042} \\ \hline
& INDPRO & 1.009 & 1.011 & \textbf{1.001} & 1.003 & 1.021 & \textbf{1.001}
\\ 
& UNRATE & 0.762 *** & 0.751 *** & \textbf{0.734 ***} & 0.876 & 0.878 & 
\textbf{0.771} \\ 
& HOUST & 0.951 & \textbf{0.939 *} & 0.956 & 1.009 & 1.006 & \textbf{0.952}
\\ 
\multicolumn{1}{l}{h=12} & PERMIT & 0.979 & 0.944 & \textbf{0.94} & 1.017 & 
0.954 & \textbf{0.953} \\ 
& M2REAL & 1.002 & \textbf{0.967} & 0.988 & 1.002 & 0.967 & \textbf{0.96} \\ 
& GS10 & 1.224 & 1.152 * & \textbf{1.109} & \textbf{1.121} & 1.151 & 1.187
\\ 
& CPI & 0.949 & \textbf{0.938} & 0.955 & 0.958 & 0.948 & \textbf{0.931} \\ 
& S\&P500 & 1.002 & 1.020 & \textbf{1.000} & \textbf{0.997} & 1.031 & 1.015
\\ \hline
\end{tabular}
}
\par
{\tiny {\scriptsize \vspace{0.2cm} } }
\par
\begin{flushleft}
{\tiny {\scriptsize \noindent $^{\ast }$ Notes: Results reported in this
table summarize findings from a prediction experiment that uses real-time
data collected in the St. Louis Federal Reserve Bank's FRED-MD dataset (see
https://www.stlouisfed.org/research/economists/mccracken/fred-databases) to
construct real-time forecasts. Tabulated entries are relative MSFEs (where
the AR(SIC) MSFE is the denominator), for forecast horizons h=1,3,6, and 12
months, and for recursive and rolling data window estiamtion schemes.
Entries in bold denote lowest relative MSFEs for a given target variable,
forecast horizon, and data windowing scheme. In all cases, factors are
estimated using PCA, and the number of factors is estiamted using the $%
PC_{p2}$ criterion of Bai and Ng (2002). The variables used in factor
estimation are selected using PCA (all variables), hard thresholding (HT)
and the CS test (CS). Starred entries indicate rejection of the null
hypothesis of equal conditional predictive ability, at significance levels $%
p=0.01$ ($\ast \ast \ast $), $p=0.05$, ($\ast \ast $), and $p=0.10$ ($\ast $%
). See Section 5.1 for complete details. } }
\end{flushleft}
\end{table}

\newpage 
\begin{table}[h]
\renewcommand\arraystretch{1.5} {\small \noindent \textbf{Table 4: }%
Empirical Illustration 1 - FRED MD Dataset - Forecasting Results Based on
the Use of Alternative Variable Selection Methods With a Maximum of 8
Factors (Forecast Period 2000:1-2024:6){$^{\ast }$} } \vspace{0.3cm} %
\centering{\tiny \ 
\begin{tabular}{p{0.5cm}p{1.4cm}p{1.4cm}p{1.4cm}p{1.4cm}p{1.4cm}p{1.4cm}p{1.4cm}}
\hline\hline
&  & \multicolumn{3}{c}{Estimation Uses Recursive Data Window} & 
\multicolumn{3}{c}{Estimation Uses Rolling Data Window} \\ 
& \multicolumn{1}{c}{Target Variable} & \multicolumn{1}{l}{PCA} & 
\multicolumn{1}{l}{HT} & \multicolumn{1}{l}{CS} & \multicolumn{1}{l}{PCA} & 
\multicolumn{1}{l}{HT} & \multicolumn{1}{l}{CS} \\ \hline
& INDPRO & \textbf{0.941} & \textbf{0.941} & 0.992 & \textbf{0.906} & 0.982
& 1.009 \\ 
& UNRATE & \textbf{0.612} & 0.617 & 0.704 & 0.518 & 0.528 & \textbf{0.407}
\\ 
& HOUST & 0.998 & \textbf{0.993} & 1.032 & 1.045 & \textbf{0.989} & 1.037 \\ 
\multicolumn{1}{l}{h=1} & PERMIT & 0.971 & 1.008 & \textbf{0.941} & 0.910 & 
0.971 & \textbf{0.816} \\ 
& M2REAL & 1.027 & \textbf{1.004} & 1.008 & \textbf{1.013} & 1.021 & 1.041
\\ 
& GS10 & 1.196 & 1.128 & \textbf{1.115} & 1.096 & 1.110 & \textbf{1.070} \\ 
& CPI & 0.968 & 0.980 & \textbf{0.966} & 1.021 & \textbf{0.979} & 1.035 \\ 
& S\&P500 & 1.133 *** & 1.129 ** & \textbf{1.094 ***} & 1.295 & 1.438 & 
\textbf{1.151 **} \\ \hline
& INDPRO & 1.086 & \textbf{1.068} & 1.087 & 1.083 & \textbf{0.955} & 1.095 *
\\ 
& UNRATE & \textbf{0.574} & 0.579 & 0.575 & 0.811 & \textbf{0.799} & 0.801
\\ 
& HOUST & 0.969 * & 0.971 * & \textbf{0.968 *} & \textbf{0.991} & 1.001 & 
0.994 \\ 
\multicolumn{1}{l}{h=3} & PERMIT & 0.969 & \textbf{0.966} & 0.977 & 0.995 & 
0.991 & \textbf{0.976} \\ 
& M2REAL & \textbf{1.027} & 1.032 & 1.048 ** & \textbf{1.025} & 1.086 & 1.036
\\ 
& GS10 & 1.279 & 1.245 * & \textbf{1.128} & 1.125 & 1.126 & \textbf{1.077}
\\ 
& CPI & 0.973 & 0.945 & \textbf{0.922 *} & \textbf{0.959} & 0.984 & 0.990 \\ 
& S\&P500 & 1.065 ** & \textbf{1.028} & 1.054 ** & 1.095 & \textbf{1.041} & 
1.073 \\ \hline
& INDPRO & \textbf{1.038} & 1.045 & 1.068 ** & 1.187 & 1.189 & \textbf{1.132
*} \\ 
& UNRATE & 0.803 *** & 0.807 *** & \textbf{0.771 ***} & \textbf{0.786} & 
0.807 ** & 0.824 \\ 
& HOUST & \textbf{0.965 **} & 0.967 * & 0.966 * & 1.020 & 1.020 & \textbf{%
0.983} \\ 
\multicolumn{1}{l}{h=6} & PERMIT & \textbf{0.963 *} & 0.966 & 0.975 & 1.005
& 0.996 & \textbf{0.956} \\ 
& M2REAL & 1.031 & 1.039 * & \textbf{1.001} & 1.014 & 1.042 & \textbf{1.000}
\\ 
& GS10 & 1.305 & 1.33 * & \textbf{1.250} & 1.281 ** & 1.216 * & \textbf{1.188%
} \\ 
& CPI & 1.064 & 1.040 & \textbf{0.988} & 1.035 & 1.044 & \textbf{1.007} \\ 
& S\&P500 & 1.081 & 1.108 ** & \textbf{1.043 *} & 1.38 * & 1.24 & \textbf{%
1.132} \\ \hline
& INDPRO & 1.018 & 1.020 & \textbf{1.011} & 1.025 & \textbf{1.012} & 1.033
\\ 
& UNRATE & 0.74 *** & 0.741 *** & \textbf{0.727 ***} & 0.781 * & 0.745 * & 
\textbf{0.724} \\ 
& HOUST & 0.932 ** & 0.921 ** & \textbf{0.917 **} & 1.028 & 0.970 & \textbf{%
0.946 *} \\ 
\multicolumn{1}{l}{h=12} & PERMIT & 0.958 & \textbf{0.93} & 0.943 & 1.037 & 
1.002 & \textbf{0.931} \\ 
& M2REAL & 0.987 & 0.972 & \textbf{0.971} & 1.002 & \textbf{0.964} & 0.977
\\ 
& GS10 & 1.437 & 1.401 ** & \textbf{1.308} & 1.327 * & 1.283 & \textbf{1.277}
\\ 
& CPI & 0.972 & 0.933 & \textbf{0.914 *} & 0.959 & 0.977 & \textbf{0.957} \\ 
& S\&P500 & 1.042 & \textbf{1.022} & 1.024 & 1.018 & 1.060 & \textbf{1.015}
\\ \hline\hline
\end{tabular}
}
\par
{\tiny {\scriptsize \vspace{0.2cm} } }
\par
\begin{flushleft}
{\tiny {\scriptsize \noindent $^{\ast }$ Notes: See notes to Table 3. } }
\end{flushleft}
\end{table}

\pagebreak

\begin{table}[h]
{\small \noindent \textbf{Table 5: }Empirical Illustration 2 - Multi Country
Dataset - Average Predictive Accuracy Rank Scores by Country and Forecast
Horizon{$^{\ast }$} }
\par
\vspace{0.3cm} {\ \ 
\begin{tabular}{p{2cm}p{2cm}p{2cm}p{2cm}p{2cm}p{2cm}p{2cm}}
\hline\hline
& PCA & CS & HT & PCA+macro & CS+macro & HT+macro \\ \hline
\multicolumn{7}{c}{Average Ranks Across All Variables for Forecast Horizon
h=1} \\ \hline
USA & 2.4 & 3.6 & 3.6 & 3.8 & 3.8 & 3.8 \\ 
UK & 3.0 & 3.4 & 3.0 & 3.6 & 3.4 & 4.6 \\ 
Germany & 2.6 & 2.4 & 2.8 & 4.0 & 5.2 & 4.0 \\ 
France & 2.8 & 2.3 & 1.3 & 5.0 & 5.5 & 4.3 \\ 
Italy & 5.2 & 3.4 & 2.8 & 4.0 & 3.2 & 2.4 \\ 
Japan & 3.3 & 2.8 & 4.3 & 3.5 & 2.8 & 4.5 \\ \hline
\multicolumn{7}{c}{Average Ranks Across All Variables for Forecast Horizon
h=2} \\ \hline
USA & 3.0 & 2.4 & 3.4 & 4.2 & 3.6 & 4.4 \\ 
UK & 3.2 & 3.0 & 4.8 & 3.8 & 2.4 & 3.8 \\ 
Germany & 2.8 & 3.0 & 2.0 & 4.2 & 4.6 & 4.4 \\ 
France & 3.3 & 2.0 & 2.8 & 5.3 & 3.5 & 4.3 \\ 
Italy & 3.6 & 2.6 & 2.6 & 4.6 & 4.4 & 3.2 \\ 
Japan & 3.3 & 3.0 & 3.8 & 3.8 & 3.8 & 3.5 \\ \hline
\multicolumn{7}{c}{Average Ranks Across All Variables for Forecast Horizon
h=3} \\ \hline
USA & 2.6 & 4.2 & 2.8 & 3.4 & 5.0 & 3.0 \\ 
UK & 3.2 & 3.0 & 4.0 & 3.8 & 3.2 & 3.8 \\ 
Germany & 2.6 & 3.0 & 3.6 & 4.4 & 4.0 & 3.4 \\ 
France & 3.0 & 1.8 & 2.8 & 4.8 & 4.3 & 4.5 \\ 
Italy & 3.2 & 2.2 & 2.8 & 4.2 & 5.2 & 3.4 \\ 
Japan & 2.5 & 3.5 & 3.5 & 3.3 & 4.0 & 4.3 \\ \hline
\multicolumn{7}{c}{Average Ranks Across All Variables for Forecast Horizon
h=4} \\ \hline
USA & 2.2 & 4.0 & 3.2 & 3.6 & 4.2 & 3.8 \\ 
UK & 3.4 & 2.8 & 4.0 & 3.8 & 3.2 & 3.8 \\ 
Germany & 2.6 & 3.6 & 3.4 & 3.8 & 3.8 & 3.8 \\ 
France & 2.0 & 3.3 & 2.8 & 4.8 & 4.5 & 3.8 \\ 
Italy & 3.6 & 2.2 & 4.0 & 3.4 & 3.8 & 4.0 \\ 
Japan & 2.3 & 4.3 & 2.8 & 2.8 & 5.3 & 3.8 \\ \hline\hline
\end{tabular}
}
\par
{\scriptsize \vspace{0.5cm} }
\par
{\scriptsize \noindent $^{\ast }$ Notes: See notes to Table 3. The
experiment reported on in this table uses an updated version of the Global
VAR dataset analyzed by Dees, di Mauro, Pesaran and Smith (2007). More
specifically, quarterly forecasts are made for 5 variables including GDP ($y$%
), inflation ($p$), equity returns ($q$), short-term interest rates ($\rho
^{s}$), and long-term interest rates ($\rho ^{l}$). The ex ante forecasting
period is 2021:Q4-2023:Q3, and predictions are made for 6 different
countries and 4 different forecast horizons (i.e. $h$=1,2,3,4). Tabulated
entries are average "ranks" of our different models, constructed by
comparing MSFEs by country and forecast horizon, when averaged across all 5
variables. In our analysis, we compare 6 different models (i.e., PCA, CS,
HT, PCA+macro, CS+macro, and HT+macro). All models include an AR component,
and up to 2 factors estimated using PCA, with factors selected using the $%
PC_{p2}$ statistic. PCA+macro, CS+macro, and HT+macro include lags of the 5
macroeconomic variables as explanatory variables, rather than just lags of
the target variable, with lags selected using the SIC. Summarizing, model
ranks are calculated for each variable (excluding deq for France and dp for
Japan, due to data reliability issues for those two variables) and each
forecast horizon, and are averaged across all five variables to yield the
average ranks reported in the table. For complete details refer to Section
5.2}
\end{table}

\end{document}
