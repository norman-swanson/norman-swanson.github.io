%2multibyte Version: 5.50.0.2960 CodePage: 936


\documentclass[11pt]{article}
%%%%%%%%%%%%%%%%%%%%%%%%%%%%%%%%%%%%%%%%%%%%%%%%%%%%%%%%%%%%%%%%%%%%%%%%%%%%%%%%%%%%%%%%%%%%%%%%%%%%%%%%%%%%%%%%%%%%%%%%%%%%%%%%%%%%%%%%%%%%%%%%%%%%%%%%%%%%%%%%%%%%%%%%%%%%%%%%%%%%%%%%%%%%%%%%%%%%%%%%%%%%%%%%%%%%%%%%%%%%%%%%%%%%%%%%%%%%%%%%%%%%%%%%%%%%
\usepackage{amssymb}
\usepackage{amsfonts}
\usepackage{amsmath}
\usepackage{geometry}
\usepackage[onehalfspacing]{setspace}
\usepackage{setspace}

\setcounter{MaxMatrixCols}{10}
%TCIDATA{OutputFilter=LATEX.DLL}
%TCIDATA{Version=5.00.0.2606}
%TCIDATA{Codepage=936}
%TCIDATA{<META NAME="SaveForMode" CONTENT="1">}
%TCIDATA{BibliographyScheme=Manual}
%TCIDATA{Created=Sunday, July 18, 2004 16:10:34}
%TCIDATA{LastRevised=Monday, November 21, 2022 12:05:40}
%TCIDATA{<META NAME="GraphicsSave" CONTENT="32">}
%TCIDATA{<META NAME="DocumentShell" CONTENT="General\Blank Document">}
%TCIDATA{Language=American English}
%TCIDATA{CSTFile=article.cst}
%TCIDATA{PageSetup=72,72,72,72,0}
%TCIDATA{AllPages=
%F=36,\PARA{038<p type="texpara" tag="Body Text" >\hfill \thepage}
%}


\newtheorem{theorem}{Theorem}
\newtheorem{acknowledgement}[theorem]{Acknowledgement}
\newtheorem{algorithm}[theorem]{Algorithm}
\newtheorem{axiom}[theorem]{Axiom}
\newtheorem{case}[theorem]{Case}
\newtheorem{claim}[theorem]{Claim}
\newtheorem{conclusion}[theorem]{Conclusion}
\newtheorem{condition}[theorem]{Condition}
\newtheorem{conjecture}[theorem]{Conjecture}
\newtheorem{corollary}[theorem]{Corollary}
\newtheorem{criterion}[theorem]{Criterion}
\newtheorem{definition}[theorem]{Definition}
\newtheorem{example}[theorem]{Example}
\newtheorem{exercise}[theorem]{Exercise}
\newtheorem{lemma}[theorem]{Lemma}
\newtheorem{notation}[theorem]{Notation}
\newtheorem{problem}[theorem]{Problem}
\newtheorem{proposition}[theorem]{Proposition}
\newtheorem{remark}[theorem]{Remark}
\newtheorem{solution}[theorem]{Solution}
\newtheorem{summary}[theorem]{Summary}
\newenvironment{proof}[1][Proof]{\noindent\textbf{#1.} }{\ \rule{0.5em}{0.5em}}
\renewcommand{\baselinestretch}{1.0} 
\textwidth=6.8in
\textheight=8.7in
\oddsidemargin=0in
\evensidemargin=0in
\topmargin=0in
\baselineskip=10pt
\linespread{1.3}
\input{tcilatex}
\geometry{left=1in,right=1in,top=1.25in,bottom=1.25in}

\begin{document}


\begin{center}
{\Large {Technical Appendix: Consistent Factor Estimation and Forecasting in
Factor-Augmented VAR Models$^{\ast }$}}

\bigskip

John C. Chao$^{1},$ Yang Liu$^{2}$ and Norman R. Swanson$^{2}$

\medskip

$^{1}$University of Maryland and $^{2}$Rutgers University

\medskip

November 21, 2022

\bigskip \bigskip

Abstract
\end{center}

\begin{spacing}{1.01}
\noindent Proofs to lemmas used in Consistent Forecasting in Factor-Augmented VAR Models by Chao, Lui, and Swanson (2022) are gathered in this paper.
\end{spacing}

\bigskip \bigskip \bigskip

\noindent \textit{Keywords: }Factor analysis, factor augmented vector
autoregression, forecasting, moderate deviation, principal components,
self-normalization, variable selection.

\medskip

\noindent \textit{JEL Classification: }C32, C33, C38, C52, C53, C55.

\bigskip \bigskip

\begin{spacing}{1.01}
\noindent $^{\ast }$\textit{Corresponding Author:} John C. Chao, Department of Economics, 7343 Preinkert
Drive, University of Maryland, chao@econ.umd.edu.

\medskip

\noindent Norman R. Swanson, Department of Economics, 9500 Hamilton Street, Rutgers University,
nswanson@econ.rutgers.edu. 
\end{spacing}

\newpage

\noindent \noindent \setcounter{page}{2}

\bigskip

\section{\noindent Appendix A: Proof of the Theorem 2.1\qquad}

\noindent \textbf{Proof of Theorem 2.1:}

The proof of Theorem 1 requires a long series of calculations. Hence, we
have divided this proof into six different steps.

\textbf{\noindent }

\noindent \textbf{Step 1:}

In step 1, we shall transform the simple factor model%
\begin{equation}
\underset{N\times 1}{Z_{t}}=\underset{N\times 1}{\gamma }\underset{1\times 1}%
{f_{t}}+\underset{N\times 1}{u_{t}}\text{, }t=1,...,T
\label{one factor model Appendix}
\end{equation}
into a more convenient form. Let $\Pi $ denote an $N\times N$ orthogonal
matrix whose columns are the eigenvectors of the covariance matrix $\Sigma
_{Z}=E\left[ Z_{t}Z_{t}^{\prime }\right] $. Without loss of generality, we
can partition $\Pi $ as%
\begin{equation*}
\underset{N\times N}{\Pi }\mathbf{=}\left[ 
\begin{array}{cc}
\underset{N\times 1}{\pi _{1}} & \underset{N\times \left( N-1\right) }{\Pi
_{2}}%
\end{array}%
\right]
\end{equation*}%
where $\pi _{1}$ is the eigenvector associated with the largest eigenvalue
of $\Sigma _{Z}=E\left[ Z_{t}Z_{t}^{\prime }\right] $, i.e., $\lambda
_{\left( 1\right) }\left( \Sigma _{Z}\right) $. By the result of Lemma B-8,
we know that%
\begin{equation*}
\pi _{1}=\frac{\gamma }{\left\Vert \gamma \right\Vert _{2}}\text{ and }%
\lambda _{\left( 1\right) }\left( \Sigma _{Z}\right) =\left\Vert \gamma
\right\Vert _{2}^{2}+1\text{.}
\end{equation*}

Next, we define%
\begin{eqnarray}
W_{t} &=&\Pi ^{\prime }Z_{t}  \notag \\
&=&\Pi ^{\prime }\left( \gamma f_{t}+u_{t}\right)  \notag \\
&=&\left\Vert \gamma \right\Vert _{2}\Pi ^{\prime }\frac{\gamma }{\left\Vert
\gamma \right\Vert _{2}}f_{t}+\Pi ^{\prime }u_{t}  \notag \\
&=&\left\Vert \gamma \right\Vert _{2}f_{t}\Pi ^{\prime }\pi _{1}+\Pi
^{\prime }u_{t}\text{ \ }\left( \text{since }\pi _{1}=\frac{\gamma }{%
\left\Vert \gamma \right\Vert _{2}}\right)  \notag \\
&=&\left\Vert \gamma \right\Vert _{2}f_{t}\left( 
\begin{array}{c}
\pi _{1}^{\prime } \\ 
\Pi _{2}^{\prime }%
\end{array}%
\right) \pi _{1}+\Pi ^{\prime }u_{t}  \notag \\
&=&\left\Vert \gamma \right\Vert _{2}f_{t}\mathbf{e}_{1,N}+\eta _{t}
\label{Wt}
\end{eqnarray}%
where $\mathbf{e}_{1,N}$ is an elementary vector whose first component is $1$
and all remaining components are $0$ and where $\eta _{t}=\Pi ^{\prime
}u_{t} $. Moreover, note that $\left\{ \eta _{t}\right\} \equiv
i.i.d.N\left( 0,I_{N}\right) $ since $\Pi $ is an orthogonal matrix and $%
\eta _{t}=\Pi ^{\prime }u_{t}$ with $\left\{ u_{t}\right\} \equiv
i.i.d.N\left( 0,I_{N}\right) $. We can write out the covariance matrix of $%
W_{t}$ as 
\begin{eqnarray*}
&&\Sigma _{W} \\
&=&E\left[ W_{t}W_{t}^{\prime }\right] \\
&=&E\left[ \left( \left\Vert \gamma \right\Vert _{2}f_{t}\mathbf{e}%
_{1,N}+\eta _{t}\right) \left( \left\Vert \gamma \right\Vert _{2}f_{t}%
\mathbf{e}_{1,N}+\eta _{t}\right) ^{\prime }\right] \\
&=&\left\Vert \gamma \right\Vert _{2}^{2}E\left[ f_{t}^{2}\right] \mathbf{e}%
_{1,N}\mathbf{e}_{1,N}^{\prime }+\left\Vert \gamma \right\Vert _{2}E\left[
\eta _{t}f_{t}\right] \mathbf{e}_{1,N}^{\prime }+\left\Vert \gamma
\right\Vert _{2}\mathbf{e}_{1,N}E\left[ f_{t}\eta _{t}^{\prime }\right] +E%
\left[ \eta _{t}\eta _{t}^{\prime }\right] \\
&=&\left\Vert \gamma \right\Vert _{2}^{2}\mathbf{e}_{1,N}\mathbf{e}%
_{1,N}^{\prime }+I_{N} \\
&=&\left( 
\begin{array}{ccccc}
\left\Vert \gamma \right\Vert _{2}^{2}+1 & 0 & 0 & \cdots & 0 \\ 
0 & 1 & 0 & \cdots & 0 \\ 
0 & 0 & 1 & \ddots & \vdots \\ 
\vdots & \vdots & \ddots & \ddots & 0 \\ 
0 & 0 & \cdots & 0 & 1%
\end{array}%
\right)
\end{eqnarray*}%
from which it is easily seen that $\lambda _{\left( 1\right) }\left( \Sigma
_{W}\right) =\left\Vert \gamma \right\Vert _{2}^{2}+1$ and $\lambda _{\left(
2\right) }\left( \Sigma _{W}\right) =\lambda _{\left( 3\right) }\left(
\Sigma _{W}\right) =\cdot \cdot \cdot =\lambda _{\left( N\right) }\left(
\Sigma _{W}\right) =1$, where we let $\lambda _{\left( j\right) }\left(
\Sigma _{W}\right) $ denote the $j^{th}$ largest eigenvalue of $\Sigma _{W}$%
. In addition, the eigenvector associated with $\lambda _{\left( j\right)
}\left( \Sigma _{W}\right) $ is $\mathbf{e}_{j,N}$, an elementary vector
whose $j^{th}$ component is $1$ and all other components are $0$.

Note further that we can also write $W_{t}$ in the alternative form%
\begin{eqnarray}
W_{t} &=&\left( 
\begin{array}{c}
W_{1,t} \\ 
W_{2,t} \\ 
\vdots \\ 
W_{N,t}%
\end{array}%
\right)  \notag \\
&=&\left( 
\begin{array}{c}
\left\Vert \gamma \right\Vert _{2}f_{t} \\ 
0 \\ 
\vdots \\ 
0%
\end{array}%
\right) +\left( 
\begin{array}{c}
\eta _{1t} \\ 
\eta _{2,t} \\ 
\vdots \\ 
\eta _{N,t}%
\end{array}%
\right)  \notag \\
&=&\left( 
\begin{array}{c}
\left\Vert \gamma \right\Vert _{2}\zeta _{1,t} \\ 
\zeta _{2,t} \\ 
\vdots \\ 
\zeta _{N,t}%
\end{array}%
\right)  \notag \\
&=&\dsum\limits_{j=1}^{N}\sqrt{\ell _{j}}\zeta _{j,t}\mathbf{e}_{j,N}
\label{Wt alternative form}
\end{eqnarray}%
where $\zeta _{1,t}=f_{t}+\left\Vert \gamma \right\Vert _{2}^{-1}\eta _{1t}$
and $\zeta _{j,t}=\eta _{j,t}$ for $j=2,...,N$ and where $\ell
_{1}=\left\Vert \gamma \right\Vert _{2}^{2}$ and $\ell _{j}=1$ for $%
j=2,...,N $. In fact, this is the representation of $W_{t}$ that is given in
Lemma B-10. (See Appendix B below).

\noindent \textbf{Step 2: }

Define$\underset{N\times T}{\mathbf{W}}=\left( W_{1},...,W_{T}\right) $,
where $W_{t}$ is as defined in expression (\ref{Wt}) in step 1 above.
Partition $\mathbf{W}$ as follows%
\begin{equation*}
\underset{N\times T}{\mathbf{W}}=\left[ 
\begin{array}{c}
\underset{1\times T}{\mathbf{W}_{1}^{\prime }} \\ 
\underset{\left( N-1\right) \times T}{\mathbf{W}_{2}^{\prime }}%
\end{array}%
\right] =\left[ 
\begin{array}{c}
\underset{1\times T}{\pi _{1}^{\prime }\mathbf{Z}} \\ 
\underset{\left( N-1\right) \times T}{\Pi _{2}^{\prime }\mathbf{Z}}%
\end{array}%
\right] \text{,}
\end{equation*}%
where $\underset{N\times T}{\mathbf{Z}}=\left( Z_{1},...,Z_{T}\right) $ with 
$Z_{t}$ as defined in expression (\ref{one factor model Appendix}). Note
that the first row of $\mathbf{W}$, i.e., $\mathbf{W}_{1}^{\prime }$,
contains the \textquotedblleft signal" observations with the elevated
variance $\lambda _{1}=\left\Vert \gamma \right\Vert _{2}^{2}+1$ and where
the remaining $N-1$ rows contain the elements of the $\left( N-1\right)
\times T$ matrix $\mathbf{W}_{2}^{\prime }$ which contain only the noise
variables. Now, define the sample covariance matrix%
\begin{equation*}
\widehat{\Sigma }_{\mathbf{W}}\mathbf{=}\frac{1}{T}\mathbf{WW}^{\prime
}=\left( 
\begin{array}{cc}
T^{-1}\mathbf{W}_{1}^{\prime }\mathbf{W}_{1} & T^{-1}\mathbf{W}_{1}^{\prime }%
\mathbf{W}_{2} \\ 
T^{-1}\mathbf{W}_{2}^{\prime }\mathbf{W}_{1} & T^{-1}\mathbf{W}_{2}^{\prime }%
\mathbf{W}_{2}%
\end{array}%
\right)
\end{equation*}%
In this step, we shall further transform $\widehat{\Sigma }_{\mathbf{W}}$
into the so-called arrowhead matrix. To proceed, consider the spectral
decomposition 
\begin{equation*}
\frac{\mathbf{W}_{2}^{\prime }\mathbf{W}_{2}}{T}=\widetilde{\mathbf{B}}_{2}%
\widetilde{\Lambda }\widetilde{\mathbf{B}}_{2}^{\prime }\text{ \ }
\end{equation*}%
where $\widetilde{\Lambda }=diag\left( \widetilde{\lambda }_{\left( 2\right)
},...,\widetilde{\lambda }_{\left( N\right) }\right) $ with $\widetilde{%
\lambda }_{\left( 2\right) },...,\widetilde{\lambda }_{\left( N\right) }$
being the $N-1$ eigenvalues of $\mathbf{W}_{2}^{\prime }\mathbf{W}_{2}/T$
and $\widetilde{\mathbf{B}}_{2}$ is an $\left( N-1\right) \times \left(
N-1\right) $ orthogonal matrix whose columns are the eigenvectors of $%
\mathbf{W}_{2}^{\prime }\mathbf{W}_{2}/T$. Note that, without loss of
generality, we can assume that the eigenvalues are ordered so that $%
\widetilde{\lambda }_{\left( 2\right) }\geq \widetilde{\lambda }_{\left(
3\right) }\geq \cdot \cdot \cdot \geq \widetilde{\lambda }_{\left( N\right)
} $. Next, create the modified data matrix 
\begin{equation*}
\underset{N\times T}{\widetilde{\mathbf{W}}}=\left[ 
\begin{array}{c}
\underset{1\times T}{\mathbf{W}_{1}^{\prime }} \\ 
\underset{\left( N-1\right) \times T}{\widetilde{\mathbf{B}}_{2}^{\prime }%
\mathbf{W}_{2}^{\prime }}%
\end{array}%
\right]
\end{equation*}%
The sample covariance matrix based on the modified data matrix is then given
by%
\begin{eqnarray*}
\underset{N\times N}{\widetilde{\Sigma }_{\mathbf{W}}} &=&\frac{\widetilde{%
\mathbf{W}}\widetilde{\mathbf{W}}^{\prime }}{T} \\
&=&\left( 
\begin{array}{cc}
T^{-1}\mathbf{W}_{1}^{\prime }\mathbf{W}_{1} & T^{-1}\mathbf{W}_{1}^{\prime }%
\mathbf{W}_{2}\widetilde{\mathbf{B}}_{2} \\ 
T^{-1}\widetilde{\mathbf{B}}_{2}^{\prime }\mathbf{W}_{2}^{\prime }\mathbf{W}%
_{1} & T^{-1}\widetilde{\mathbf{B}}_{2}^{\prime }\mathbf{W}_{2}^{\prime }%
\mathbf{W}_{2}\widetilde{\mathbf{B}}_{2}%
\end{array}%
\right) \\
&=&\left( 
\begin{array}{cc}
s & \upsilon ^{\prime } \\ 
\upsilon & \widetilde{\Lambda }%
\end{array}%
\right) \\
&=&\left( 
\begin{array}{ccccc}
s & \upsilon _{2} & \upsilon _{3} & \cdots & \upsilon _{N} \\ 
\upsilon _{2} & \widetilde{\lambda }_{\left( 2\right) } & 0 & \cdots & 0 \\ 
\upsilon _{3} & 0 & \widetilde{\lambda }_{\left( 3\right) } & \ddots & \vdots
\\ 
\vdots & \vdots & \ddots & \ddots & 0 \\ 
\upsilon _{N} & 0 & \cdots & 0 & \widetilde{\lambda }_{\left( N\right) }%
\end{array}%
\right)
\end{eqnarray*}%
where $\underset{1\times 1}{s}=\mathbf{W}_{1}^{\prime }\mathbf{W}_{1}/T$ and 
\begin{equation}
\underset{\left( N-1\right) \times 1}{\upsilon }=\left( 
\begin{array}{c}
\upsilon _{2} \\ 
\vdots \\ 
\upsilon _{N}%
\end{array}%
\right) =\frac{\widetilde{\mathbf{B}}_{2}^{\prime }\mathbf{W}_{2}^{\prime }%
\mathbf{W}_{1}}{T}\text{.}  \label{upsilon}
\end{equation}%
Note that the non-zero entries of $\widetilde{\Sigma }_{\mathbf{W}}$ form
the shape of an arrow, and so such matrices have been referred to in the
linear algebra literature as an \textquotedblleft arrowhead matrix".

An advantage of this arrowhead form is that it allows us to obtain a useful
representation for the top eigenvalue of $\widetilde{\Sigma }_{\mathbf{W}}$.
This part of step 2 comes from Johnstone and Paul (2018) following an
approach originally due to Nadler (2008), but for completeness we provide
some details of the argument here. To proceed, let $\widehat{\lambda }%
_{\left( 1\right) }$ denote the largest eigenvalue of $\widetilde{\Sigma }_{%
\mathbf{W}}$ and let $\widetilde{\mathbf{v}}_{\left( 1\right) }$ be the
associated eigenvector, where, following Johnstone and Paul (2018), we will
normalize $\widetilde{\mathbf{v}}_{\left( 1\right) }$ to have the form $%
\widetilde{\mathbf{v}}_{\left( 1\right) }=\left( 
\begin{array}{cccc}
1 & \widetilde{v}_{\left( 1\right) ,2} & \cdots & \widetilde{v}_{\left(
1\right) ,N}%
\end{array}%
\right) ^{\prime }$, i.e., we normalize $\widetilde{\mathbf{v}}_{\left(
1\right) }$ so that its first component is $1$. The eigen-equation $%
\widetilde{\Sigma }_{\mathbf{W}}\widetilde{\mathbf{v}}_{\left( 1\right) }=%
\widehat{\lambda }_{\left( 1\right) }\widetilde{\mathbf{v}}_{\left( 1\right)
}$ can then be written out more explicitly as%
\begin{equation}
\left( 
\begin{array}{ccccc}
s & \upsilon _{2} & \upsilon _{3} & \cdots & \upsilon _{N} \\ 
\upsilon _{2} & \widetilde{\lambda }_{\left( 2\right) } & 0 & \cdots & 0 \\ 
\upsilon _{3} & 0 & \widetilde{\lambda }_{\left( 3\right) } & \ddots & \vdots
\\ 
\vdots & \vdots & \ddots & \ddots & 0 \\ 
\upsilon _{N} & 0 & \cdots & 0 & \widetilde{\lambda }_{\left( N\right) }%
\end{array}%
\right) \left( 
\begin{array}{c}
1 \\ 
\widetilde{v}_{\left( 1\right) ,2} \\ 
\widetilde{v}_{\left( 1\right) ,3} \\ 
\vdots \\ 
\widetilde{v}_{\left( 1\right) ,N}%
\end{array}%
\right) =\widehat{\lambda }_{\left( 1\right) }\left( 
\begin{array}{c}
1 \\ 
\widetilde{v}_{\left( 1\right) ,2} \\ 
\widetilde{v}_{\left( 1\right) ,3} \\ 
\vdots \\ 
\widetilde{v}_{\left( 1\right) ,N}%
\end{array}%
\right)  \label{eigen eqn of lambda1}
\end{equation}%
Solving this system of equations, we see that%
\begin{equation}
\widetilde{v}_{\left( 1\right) ,j}=\frac{\upsilon _{j}}{\widehat{\lambda }%
_{\left( 1\right) }-\widetilde{\lambda }_{\left( j\right) }}\text{ for }%
j=2,...,N\text{;}  \label{jth v1tilde}
\end{equation}%
where $\upsilon _{j}$ is the $j^{th}$ component of $\upsilon $ as defined in
expression (\ref{upsilon}). Hence,%
\begin{equation}
\widetilde{\mathbf{v}}_{\left( 1\right) }=\left( 
\begin{array}{c}
1 \\ 
\widetilde{v}_{\left( 1\right) ,2} \\ 
\vdots \\ 
\widetilde{v}_{\left( 1\right) ,N}%
\end{array}%
\right) =\left( 
\begin{array}{c}
1 \\ 
\upsilon _{2}/\left( \widehat{\lambda }_{\left( 1\right) }-\widetilde{%
\lambda }_{\left( 2\right) }\right) \\ 
\vdots \\ 
\upsilon _{N}/\left( \widehat{\lambda }_{\left( 1\right) }-\widetilde{%
\lambda }_{\left( N\right) }\right)%
\end{array}%
\right)  \label{v1tilde}
\end{equation}%
Moreover, since expression (\ref{eigen eqn of lambda1}) implies that%
\begin{equation*}
\widehat{\lambda }_{\left( 1\right) }=s+\upsilon _{2}\widetilde{v}_{\left(
1\right) ,2}+\cdot \cdot \cdot +\upsilon _{N}\widetilde{v}_{\left( 1\right)
,N}
\end{equation*}%
It follows from substituting the right-hand side of equation (\ref{jth
v1tilde}) for $j=2,...,N$ into the above expression that%
\begin{equation}
\widehat{\lambda }_{\left( 1\right) }=s+\dsum\limits_{j=2}^{N}\frac{\upsilon
_{j}}{\widehat{\lambda }_{\left( 1\right) }-\widetilde{\lambda }_{\left(
j\right) }}=\frac{\mathbf{W}_{1}^{\prime }\mathbf{W}_{1}}{T}%
+\dsum\limits_{j=2}^{N}\frac{\upsilon _{j}}{\widehat{\lambda }_{\left(
1\right) }-\widetilde{\lambda }_{\left( j\right) }}\text{.}
\label{lambda1 hat}
\end{equation}

Finally, in this step, we shall relate the eigenvalues and eigenvectors of $%
\widetilde{\Sigma }_{\mathbf{W}}$ to that of the pre-transformed sample
covariance matrix of our simple factor model, i.e., 
\begin{equation*}
\widehat{\Sigma }_{Z}=\frac{\mathbf{ZZ}^{\prime }}{T}=\frac{1}{T}%
\dsum\limits_{t=1}^{T}Z_{t}Z_{t}^{\prime }\text{ where }\underset{N\times T}{%
\mathbf{Z}}=\left( Z_{1},...,Z_{T}\right) \text{.}
\end{equation*}%
Understanding this relationship then allows us to derive asymptotic
properties of quantities involving the leading eigenvector of $\widehat{%
\Sigma }_{Z}$ using the explicit representation of $\widetilde{\mathbf{v}}%
_{1}$ and $\widehat{\lambda }_{1}$ given in expressions (\ref{v1tilde}) and (%
\ref{lambda1 hat}), respectively. To proceed, we first relate the
eigenvalues and eigenvectors of $\widetilde{\Sigma }_{\mathbf{W}}=\widetilde{%
\mathbf{W}}\widetilde{\mathbf{W}}^{\prime }/T$ to that of $\widehat{\Sigma }%
_{\mathbf{W}}\mathbf{=WW}^{\prime }/T$. Define%
\begin{equation*}
\underset{N\times N}{\widetilde{\mathbf{B}}}=\left( 
\begin{array}{cc}
1 & 0 \\ 
0 & \widetilde{\mathbf{B}}_{2}%
\end{array}%
\right)
\end{equation*}%
Now, since $\widetilde{\mathbf{B}}_{2}$ is an orthogonal matrix, it follows
that%
\begin{equation*}
\widetilde{\mathbf{B}}^{\prime }\widetilde{\mathbf{B}}=\left( 
\begin{array}{cc}
1 & 0 \\ 
0 & \widetilde{\mathbf{B}}_{2}^{\prime }%
\end{array}%
\right) \left( 
\begin{array}{cc}
1 & 0 \\ 
0 & \widetilde{\mathbf{B}}_{2}%
\end{array}%
\right) =\left( 
\begin{array}{cc}
1 & 0 \\ 
0 & \widetilde{\mathbf{B}}_{2}^{\prime }\widetilde{\mathbf{B}}_{2}%
\end{array}%
\right) =\left( 
\begin{array}{cc}
1 & 0 \\ 
0 & I_{N-1}%
\end{array}%
\right) =I_{N}
\end{equation*}%
and%
\begin{equation*}
\widetilde{\mathbf{B}}\widetilde{\mathbf{B}}^{\prime }=\left( 
\begin{array}{cc}
1 & 0 \\ 
0 & \widetilde{\mathbf{B}}_{2}%
\end{array}%
\right) \left( 
\begin{array}{cc}
1 & 0 \\ 
0 & \widetilde{\mathbf{B}}_{2}^{\prime }%
\end{array}%
\right) =\left( 
\begin{array}{cc}
1 & 0 \\ 
0 & \widetilde{\mathbf{B}}_{2}\widetilde{\mathbf{B}}_{2}^{\prime }%
\end{array}%
\right) =\left( 
\begin{array}{cc}
1 & 0 \\ 
0 & I_{N-1}%
\end{array}%
\right) =I_{N}
\end{equation*}%
so that $\widetilde{\mathbf{B}}$ is an orthogonal matrix as well. Next, note
that%
\begin{eqnarray*}
&&\frac{\widetilde{\mathbf{B}}^{\prime }\mathbf{WW}^{\prime }\widetilde{%
\mathbf{B}}}{T} \\
&=&\left( 
\begin{array}{cc}
1 & 0 \\ 
0 & \widetilde{\mathbf{B}}_{2}^{\prime }%
\end{array}%
\right) \left( 
\begin{array}{cc}
T^{-1}\mathbf{W}_{1}^{\prime }\mathbf{W}_{1} & T^{-1}\mathbf{W}_{1}^{\prime }%
\mathbf{W}_{2} \\ 
T^{-1}\mathbf{W}_{2}^{\prime }\mathbf{W}_{1} & T^{-1}\mathbf{W}_{2}^{\prime }%
\mathbf{W}_{2}%
\end{array}%
\right) \left( 
\begin{array}{cc}
1 & 0 \\ 
0 & \widetilde{\mathbf{B}}_{2}%
\end{array}%
\right) \\
&=&\left( 
\begin{array}{cc}
T^{-1}\mathbf{W}_{1}^{\prime }\mathbf{W}_{1} & T^{-1}\mathbf{W}_{1}^{\prime }%
\mathbf{W}_{2}\widetilde{\mathbf{B}}_{2} \\ 
T^{-1}\widetilde{\mathbf{B}}_{2}^{\prime }\mathbf{W}_{2}^{\prime }\mathbf{W}%
_{1} & T^{-1}\widetilde{\mathbf{B}}_{2}^{\prime }\mathbf{W}_{2}^{\prime }%
\mathbf{W}_{2}\widetilde{\mathbf{B}}_{2}%
\end{array}%
\right) \\
&=&\frac{\widetilde{\mathbf{W}}\widetilde{\mathbf{W}}^{\prime }}{T} \\
&=&\widetilde{\Sigma }_{\mathbf{W}}
\end{eqnarray*}%
Hence, to relate the eigenvalues and eigenvectors of $\widehat{\Sigma }_{%
\mathbf{W}}\mathbf{=WW}^{\prime }/T$ to those of $\widetilde{\Sigma }_{%
\mathbf{W}}=\widetilde{\mathbf{B}}^{\prime }\mathbf{WW}^{\prime }\widetilde{%
\mathbf{B}}/T$, we note that the eigenvalues of the $\widetilde{\Sigma }_{%
\mathbf{W}}$ are the solutions of the determinantal equation%
\begin{eqnarray*}
0 &=&\det \left\{ \frac{\widetilde{\mathbf{B}}^{\prime }\mathbf{WW}^{\prime }%
\widetilde{\mathbf{B}}}{T}-\lambda I_{N}\right\} \\
&=&\det \left\{ \widetilde{\mathbf{B}}^{\prime }\right\} \det \left\{ \frac{%
\mathbf{WW}^{\prime }}{T}-\lambda \widetilde{\mathbf{B}}\widetilde{\mathbf{B}%
}^{\prime }\right\} \det \left\{ \widetilde{\mathbf{B}}\right\} \\
&=&\det \left\{ \widetilde{\mathbf{B}}^{\prime }\right\} \det \left\{ \frac{%
\mathbf{WW}^{\prime }}{T}-\lambda I_{N}\right\} \det \left\{ \widetilde{%
\mathbf{B}}\right\} \text{ }\left( \text{since }\widetilde{\mathbf{B}}\text{%
\textbf{\ }is an orthogonal matrix}\right) \\
&=&\det \left\{ \frac{\mathbf{WW}^{\prime }}{T}-\lambda I_{N}\right\}
\end{eqnarray*}%
where the last equality holds because $\det \left\{ \widetilde{\mathbf{B}}%
^{\prime }\right\} =\det \left\{ \widetilde{\mathbf{B}}\right\} =\pm 1$
given that $\widetilde{\mathbf{B}}$\textbf{\ }is an orthogonal matrix. It
follows that $\widehat{\Sigma }_{\mathbf{W}}\mathbf{=WW}^{\prime }/T$ and $%
\widetilde{\Sigma }_{\mathbf{W}}=\widetilde{\mathbf{B}}^{\prime }\mathbf{WW}%
^{\prime }\widetilde{\mathbf{B}}/T$ have the same set of eigenvalues.
Moreover, let $\widehat{\lambda }_{\left( j\right) }$ be the $j^{th}$
largest eigenvalue of $\widehat{\Sigma }_{\mathbf{W}}\mathbf{=WW}^{\prime
}/T $, which is of course also the $j^{th}$ largest eigenvalue of $%
\widetilde{\Sigma }_{\mathbf{W}}=\widetilde{\mathbf{B}}^{\prime }\mathbf{WW}%
^{\prime }\widetilde{\mathbf{B}}/T$. Also, let $\widetilde{\mathbf{v}}%
_{\left( j\right) }$ be an eigenvector of $\widetilde{\Sigma }_{\mathbf{W}}=%
\widetilde{\mathbf{B}}^{\prime }\mathbf{WW}^{\prime }\widetilde{\mathbf{B}}%
/T $ associated with $\widehat{\lambda }_{\left( j\right) }$. Define $%
\mathbf{v}_{\left( j\right) }\equiv \widetilde{\mathbf{B}}\widetilde{\mathbf{%
v}}_{\left( j\right) }$ for $j=1,...,N$, and note that, since $\widetilde{%
\Sigma }_{\mathbf{W}}\widetilde{\mathbf{v}}_{\left( j\right) }=\widehat{%
\lambda }_{\left( j\right) }\widetilde{\mathbf{v}}_{\left( j\right) }$, we
have 
\begin{eqnarray*}
\widetilde{\mathbf{B}}^{\prime }\widehat{\Sigma }_{\mathbf{W}}\widetilde{%
\mathbf{B}}\widetilde{\mathbf{v}}_{\left( j\right) } &=&\left( \frac{%
\widetilde{\mathbf{B}}^{\prime }\mathbf{WW}^{\prime }\widetilde{\mathbf{B}}}{%
T}\right) \widetilde{\mathbf{v}}_{\left( j\right) } \\
&=&\widetilde{\Sigma }_{\mathbf{W}}\widetilde{\mathbf{v}}_{\left( j\right) }
\\
&=&\widehat{\lambda }_{\left( j\right) }\widetilde{\mathbf{v}}_{\left(
j\right) }
\end{eqnarray*}%
which implies that%
\begin{equation*}
\widehat{\Sigma }_{\mathbf{W}}\mathbf{v}_{\left( j\right) }\mathbf{=}%
\widehat{\Sigma }_{\mathbf{W}}\widetilde{\mathbf{B}}\widetilde{\mathbf{v}}%
_{\left( j\right) }=\widehat{\lambda }_{\left( j\right) }\widetilde{\mathbf{B%
}}\widetilde{\mathbf{v}}_{\left( j\right) }=\widehat{\lambda }_{\left(
j\right) }\mathbf{v}_{\left( j\right) }
\end{equation*}%
so that $\mathbf{v}_{\left( j\right) }=\widetilde{\mathbf{B}}\widetilde{%
\mathbf{v}}_{\left( j\right) }$ is an eigenvector of $\widehat{\Sigma }_{%
\mathbf{W}}\mathbf{=WW}^{\prime }/T$ associated with $\widehat{\lambda }%
_{\left( j\right) }$. Note, further that, previously, we have normalized the
first element of $\widetilde{\mathbf{v}}_{\left( 1\right) }$ to be $1$.
This, in turn, implies that the first element of $\mathbf{v}_{\left(
1\right) }$ will be $1$ as well since 
\begin{eqnarray}
\mathbf{v}_{\left( 1\right) } &=&\widetilde{\mathbf{B}}\widetilde{\mathbf{v}}%
_{\left( 1\right) }  \notag \\
&=&\left( 
\begin{array}{cc}
1 & 0 \\ 
0 & \widetilde{\mathbf{B}}_{2}%
\end{array}%
\right) \left( 
\begin{array}{c}
1 \\ 
\widetilde{\mathbf{v}}_{\left( 1\right) }^{\left( 2\right) }%
\end{array}%
\right)  \notag \\
&=&\left( 
\begin{array}{c}
1 \\ 
\widetilde{\mathbf{B}}_{2}\widetilde{\mathbf{v}}_{\left( 1\right) }^{\left(
2\right) }%
\end{array}%
\right)  \notag \\
&=&\left( 
\begin{array}{c}
1 \\ 
\mathbf{v}_{\left( 1\right) }^{\left( 2\right) }%
\end{array}%
\right)  \label{1st element normalized v1}
\end{eqnarray}%
where we let $\underset{\left( N-1\right) \times 1}{\widetilde{\mathbf{v}}%
_{\left( 1\right) }^{\left( 2\right) }}=\left( 
\begin{array}{cccc}
\widetilde{v}_{\left( 1\right) ,2} & \widetilde{v}_{\left( 1\right) ,3} & 
\cdots & \widetilde{v}_{\left( 1\right) ,N}%
\end{array}%
\right) ^{\prime }$ and $\underset{\left( N-1\right) \times 1}{\mathbf{v}%
_{\left( 1\right) }^{\left( 2\right) }}=\widetilde{\mathbf{B}}_{2}\widetilde{%
\mathbf{v}}_{\left( 1\right) }^{\left( 2\right) }=\left( 
\begin{array}{cccc}
v_{\left( 1\right) ,2} & v_{\left( 1\right) ,3} & \cdots & v_{\left(
1\right) ,N}%
\end{array}%
\right) ^{\prime }$.

In a similar manner, we can relate the eigenvalues and eigenvectors of $%
\widehat{\Sigma }_{Z}=\mathbf{ZZ}^{\prime }/T$ to those of $\widehat{\Sigma }%
_{\mathbf{W}}\mathbf{=WW}^{\prime }/T$ and, thus, also to those of $%
\widetilde{\Sigma }_{\mathbf{W}}=\widetilde{\mathbf{B}}^{\prime }\mathbf{WW}%
^{\prime }\widetilde{\mathbf{B}}/T$. In this case, note that the eigenvalues
of the $\widehat{\Sigma }_{\mathbf{W}}$ are the solutions of the
determinantal equation%
\begin{eqnarray*}
0 &=&\det \left\{ \frac{\mathbf{WW}^{\prime }}{T}-\lambda I_{N}\right\} \\
&=&\det \left\{ \frac{\Pi ^{\prime }\mathbf{ZZ}^{\prime }\Pi }{T}-\lambda
I_{N}\right\} \text{ }\left( \text{ since }\mathbf{W=}\Pi ^{\prime }\mathbf{Z%
}\right) \\
&=&\det \left\{ \mathbf{\Pi }^{\prime }\right\} \det \left\{ \frac{\mathbf{ZZ%
}^{\prime }}{T}-\lambda \mathbf{\Pi \Pi }^{\prime }\right\} \det \left\{ 
\mathbf{\Pi }\right\} \text{ \ } \\
&=&\det \left\{ \mathbf{\Pi }^{\prime }\right\} \det \left\{ \frac{\mathbf{ZZ%
}^{\prime }}{T}-\lambda I_{N}\right\} \det \left\{ \mathbf{\Pi }\right\} 
\text{ } \\
&&\left( \text{since }\mathbf{\Pi }\text{\textbf{\ }is an orthogonal matrix
whose columns are the eigenvectors of }\Sigma _{Z}=E\left[
Z_{t}Z_{t}^{\prime }\right] \right) \\
&=&\det \left\{ \frac{\mathbf{ZZ}^{\prime }}{T}-\lambda I_{N}\right\}
\end{eqnarray*}%
where the last equality holds because $\det \left\{ \mathbf{\Pi }^{\prime
}\right\} =\det \left\{ \mathbf{\Pi }\right\} =\pm 1$ given that $\mathbf{%
\Pi }$\textbf{\ }is an orthogonal matrix. It follows that $\widehat{\Sigma }%
_{Z}=\mathbf{ZZ}^{\prime }/T$ has the same set of eigenvaluses as $\widehat{%
\Sigma }_{\mathbf{W}}\mathbf{=WW}^{\prime }/T$ and, thus, also the same set
of eigenvalues as $\widetilde{\Sigma }_{\mathbf{W}}=\widetilde{\mathbf{B}}%
^{\prime }\mathbf{WW}^{\prime }\widetilde{\mathbf{B}}/T$. Using the same
notation as above, we will then also let $\widehat{\lambda }_{\left(
j\right) }$ to denote the $j^{th}$ largest eigenvalue of $\widehat{\Sigma }%
_{Z}=\mathbf{ZZ}^{\prime }/T$. Moreover, as before, let $\mathbf{v}_{j}$
denote an eigenvector of $\widehat{\Sigma }_{\mathbf{W}}\mathbf{=WW}^{\prime
}/T$ associated with $\widehat{\lambda }_{\left( j\right) }$. Now, define $%
\widehat{\pi }_{\left( j\right) }\equiv $ $\Pi \mathbf{v}_{\left( j\right) }$%
, and note that since $\widehat{\Sigma }_{\mathbf{W}}\mathbf{v}_{\left(
j\right) }=\widehat{\lambda }_{\left( j\right) }\mathbf{v}_{\left( j\right)
} $, we have, for $j=1,...,N$, 
\begin{eqnarray*}
\Pi ^{\prime }\widehat{\Sigma }_{Z}\Pi \mathbf{v}_{\left( j\right) }
&=&\left( \frac{\Pi ^{\prime }\mathbf{ZZ}^{\prime }\Pi }{T}\right) \mathbf{v}%
_{\left( j\right) } \\
&=&\widehat{\Sigma }_{\mathbf{W}}\mathbf{v}_{\left( j\right) } \\
&=&\widehat{\lambda }_{\left( j\right) }\mathbf{v}_{\left( j\right) }
\end{eqnarray*}%
which implies that%
\begin{equation*}
\widehat{\Sigma }_{Z}\widehat{\pi }_{\left( j\right) }=\widehat{\Sigma }%
_{Z}\Pi \mathbf{v}_{\left( j\right) }=\widehat{\lambda }_{\left( j\right)
}\Pi \mathbf{v}_{\left( j\right) }=\widehat{\lambda }_{\left( j\right) }%
\widehat{\pi }_{\left( j\right) }
\end{equation*}%
so that 
\begin{equation}
\widehat{\pi }_{\left( j\right) }=\Pi \mathbf{v}_{\left( j\right) }
\label{pi_j hat}
\end{equation}%
is an eigenvector of $\widehat{\Sigma }_{Z}$ associated with the eigenvalue $%
\widehat{\lambda }_{\left( j\right) }$.

\noindent \textbf{Step 3:}

For the simple factor model given in expression (\ref{one factor model
Appendix}), i.e., 
\begin{eqnarray*}
Z_{t} &=&\gamma f_{t}+u_{t} \\
&=&\left\Vert \gamma \right\Vert _{2}\pi _{\left( 1\right) }f_{t}+u_{t}\text{
for }t=1,...,T\text{;}
\end{eqnarray*}%
with $\pi _{1}=\gamma /\left\Vert \gamma \right\Vert _{2}$; the
principal-component estimator of the latent factor $f_{t}$ can be written as%
\begin{eqnarray*}
\widehat{f}_{t} &=&\frac{1}{\sqrt{N}}\left\langle \frac{\widehat{\pi }%
_{\left( 1\right) }}{\left\Vert \widehat{\pi }_{\left( 1\right) }\right\Vert
_{2}},Z_{t}\right\rangle \\
&=&\frac{\left\Vert \gamma \right\Vert _{2}f_{t}}{\sqrt{N}}\left\langle 
\frac{\widehat{\pi }_{\left( 1\right) }}{\left\Vert \widehat{\pi }_{\left(
1\right) }\right\Vert _{2}},\pi _{1}\right\rangle +\frac{1}{\sqrt{N}}%
\left\langle \frac{\widehat{\pi }_{\left( 1\right) }}{\left\Vert \widehat{%
\pi }_{\left( 1\right) }\right\Vert _{2}},u_{t}\right\rangle \\
&=&\frac{\left\Vert \gamma \right\Vert _{2}f_{t}}{\sqrt{N}}\left\langle 
\frac{\Pi \mathbf{v}_{\left( 1\right) }}{\left\Vert \Pi \mathbf{v}_{\left(
1\right) }\right\Vert _{2}},\pi _{1}\right\rangle +\frac{1}{\sqrt{N}}%
\left\langle \frac{\Pi \mathbf{v}_{\left( 1\right) }}{\left\Vert \Pi \mathbf{%
v}_{\left( 1\right) }\right\Vert _{2}},u_{t}\right\rangle \text{ }\left( 
\text{making use of expression (\ref{pi_j hat}) in step 2}\right) \\
&=&\frac{\left\Vert \gamma \right\Vert _{2}f_{t}}{\sqrt{N}}\frac{\mathbf{v}%
_{\left( 1\right) }^{\prime }\Pi ^{\prime }\pi _{1}}{\left\Vert \Pi \mathbf{v%
}_{\left( 1\right) }\right\Vert _{2}}+\frac{1}{\sqrt{N}}\frac{\mathbf{v}%
_{\left( 1\right) }^{\prime }\Pi ^{\prime }u_{t}}{\left\Vert \Pi \mathbf{v}%
_{\left( 1\right) }\right\Vert _{2}} \\
&=&\frac{\left\Vert \gamma \right\Vert _{2}f_{t}}{\sqrt{N}}\frac{\mathbf{v}%
_{1}^{\prime }\mathbf{e}_{1,N}}{\left\Vert \Pi \mathbf{v}_{\left( 1\right)
}\right\Vert _{2}}+\frac{1}{\sqrt{N}}\frac{\mathbf{v}_{\left( 1\right)
}^{\prime }\Pi ^{\prime }u_{t}}{\left\Vert \Pi \mathbf{v}_{\left( 1\right)
}\right\Vert _{2}} \\
&&\left( \text{since }\Pi ^{\prime }\pi _{\left( 1\right) }=\left( 
\begin{array}{c}
\pi _{\left( 1\right) }^{\prime } \\ 
\Pi _{\left( 2\right) }^{\prime }%
\end{array}%
\right) \pi _{\left( 1\right) }=\left( 
\begin{array}{c}
1 \\ 
\underset{\left( N-1\right) \times 1}{0}%
\end{array}%
\right) =\mathbf{e}_{1,N}\text{ given that }\Pi \text{ is an orthogonal
matrix}\right) \\
&=&\frac{\left\Vert \gamma \right\Vert _{2}f_{t}}{\sqrt{N}}\frac{\mathbf{v}%
_{\left( 1\right) }^{\prime }\mathbf{e}_{1,N}}{\left\Vert \Pi \mathbf{v}%
_{\left( 1\right) }\right\Vert _{2}}+\frac{1}{\sqrt{N}}\frac{\mathbf{v}%
_{\left( 1\right) }^{\prime }\eta _{t}}{\left\Vert \Pi \mathbf{v}%
_{1}\right\Vert _{2}}\text{ }\left( \text{since, by definition, }\eta
_{t}=\Pi ^{\prime }u_{t}\right) \\
&=&\frac{\left\Vert \gamma \right\Vert _{2}f_{t}}{\sqrt{N}}\left\langle 
\frac{\mathbf{v}_{\left( 1\right) }}{\left\Vert \mathbf{v}_{\left( 1\right)
}\right\Vert _{2}}\mathbf{,e}_{1,N}\right\rangle +\frac{1}{\sqrt{N}}%
\left\langle \frac{\mathbf{v}_{\left( 1\right) }}{\left\Vert \mathbf{v}%
_{\left( 1\right) }\right\Vert _{2}}\mathbf{,}\eta _{t}\right\rangle
\end{eqnarray*}%
where the notation $\left\langle y,x\right\rangle =y^{\prime }x$ denotes the
dot product of the vectors $y$ and $x$ and where the last equality above
follows from the fact that%
\begin{equation*}
\left\Vert \Pi \mathbf{v}_{\left( 1\right) }\right\Vert _{2}=\sqrt{\mathbf{v}%
_{\left( 1\right) }^{\prime }\Pi ^{\prime }\Pi \mathbf{v}_{\left( 1\right) }}%
=\sqrt{\mathbf{v}_{\left( 1\right) }^{\prime }\mathbf{v}_{\left( 1\right) }}%
=\left\Vert \mathbf{v}_{\left( 1\right) }\right\Vert _{2}\text{. }
\end{equation*}%
Next, given expression (\ref{1st element normalized v1}) in step 2, we see
that 
\begin{eqnarray*}
\left\langle \frac{\widetilde{\mathbf{v}}_{\left( 1\right) }}{\left\Vert 
\widetilde{\mathbf{v}}_{\left( 1\right) }\right\Vert _{2}}\mathbf{,e}%
_{1,N}\right\rangle &=&\frac{\mathbf{v}_{\left( 1\right) }^{\prime }%
\widetilde{\mathbf{B}}\mathbf{e}_{1,N}}{\left\Vert \widetilde{\mathbf{v}}%
_{\left( 1\right) }\right\Vert _{2}}\text{ }\left( \text{since }\widetilde{%
\mathbf{v}}_{\left( 1\right) }=\widetilde{\mathbf{B}}^{\prime }\mathbf{v}%
_{\left( 1\right) }\right) \\
&=&\frac{1}{\left\Vert \widetilde{\mathbf{v}}_{\left( 1\right) }\right\Vert
_{2}}\left( 
\begin{array}{cc}
1 & \mathbf{v}_{\left( 1\right) }^{\left( 2\right) \prime }%
\end{array}%
\right) \left( 
\begin{array}{cc}
1 & 0 \\ 
0 & \widetilde{\mathbf{B}}_{2}%
\end{array}%
\right) \left( 
\begin{array}{c}
1 \\ 
\underset{\left( N-1\right) \times 1}{0}%
\end{array}%
\right) \\
&=&\frac{1}{\left\Vert \widetilde{\mathbf{v}}_{\left( 1\right) }\right\Vert
_{2}}\left( 
\begin{array}{cc}
1 & \mathbf{v}_{\left( 1\right) }^{\left( 2\right) }%
\end{array}%
\right) \left( 
\begin{array}{c}
1 \\ 
\underset{\left( N-1\right) \times 1}{0}%
\end{array}%
\right) \\
&=&\frac{1}{\left\Vert \widetilde{\mathbf{v}}_{\left( 1\right) }\right\Vert
_{2}}\left\langle \mathbf{v}_{\left( 1\right) },\mathbf{e}_{1,N}\right\rangle
\\
&=&\left\langle \frac{\mathbf{v}_{\left( 1\right) }}{\left\Vert \mathbf{v}%
_{\left( 1\right) }\right\Vert _{2}}\mathbf{,e}_{1,N}\right\rangle ,
\end{eqnarray*}%
where the last line follows from the fact that%
\begin{equation*}
\left\Vert \widetilde{\mathbf{v}}_{\left( 1\right) }\right\Vert _{2}=\sqrt{%
\widetilde{\mathbf{v}}_{\left( 1\right) }^{\prime }\widetilde{\mathbf{v}}%
_{\left( 1\right) }}=\sqrt{\mathbf{v}_{\left( 1\right) }^{\prime }\widetilde{%
\mathbf{B}}\widetilde{\mathbf{B}}^{\prime }\mathbf{v}_{\left( 1\right) }}=%
\sqrt{\mathbf{v}_{\left( 1\right) }^{\prime }\mathbf{v}_{\left( 1\right) }}%
=\left\Vert \mathbf{v}_{\left( 1\right) }\right\Vert _{2}
\end{equation*}%
since $\widetilde{\mathbf{B}}\widetilde{\mathbf{B}}^{\prime }=I_{N}$. In
addition, let $\widetilde{\eta }_{t}=\widetilde{\mathbf{B}}^{\prime }\eta
_{t}$, and note that 
\begin{eqnarray*}
\left\langle \frac{\mathbf{v}_{\left( 1\right) }}{\left\Vert \mathbf{v}%
_{\left( 1\right) }\right\Vert _{2}}\mathbf{,}\eta _{t}\right\rangle &=&%
\frac{1}{\left\Vert \mathbf{v}_{\left( 1\right) }\right\Vert _{2}}\mathbf{v}%
_{\left( 1\right) }^{\prime }\eta _{t} \\
&=&\frac{1}{\left\Vert \mathbf{v}_{\left( 1\right) }\right\Vert _{2}}\mathbf{%
v}_{\left( 1\right) }^{\prime }\widetilde{\mathbf{B}}\widetilde{\mathbf{B}}%
^{\prime }\eta _{t}\text{ } \\
&=&\frac{1}{\left\Vert \mathbf{v}_{\left( 1\right) }\right\Vert _{2}}%
\widetilde{\mathbf{v}}_{\left( 1\right) }^{\prime }\widetilde{\eta }_{t} \\
&=&\left\langle \frac{\widetilde{\mathbf{v}}_{\left( 1\right) }}{\left\Vert 
\widetilde{\mathbf{v}}_{\left( 1\right) }\right\Vert _{2}}\mathbf{,}%
\widetilde{\eta }_{t}\right\rangle \text{ }\left( \text{ given that }%
\left\Vert \widetilde{\mathbf{v}}_{\left( 1\right) }\right\Vert
_{2}=\left\Vert \mathbf{v}_{\left( 1\right) }\right\Vert _{2}\right) .
\end{eqnarray*}%
Since%
\begin{equation*}
\left\{ \eta _{t}\right\} \equiv i.i.d.N\left( 0,I_{N}\right)
\end{equation*}%
and $\widetilde{\mathbf{B}}$ is an orthogonal matrix, we also have%
\begin{equation*}
\left\{ \widetilde{\eta }_{t}\right\} \equiv i.i.d.N\left( 0,I_{N}\right) 
\text{.}
\end{equation*}%
Using these calculations, we can then rewrite the expression for $\widehat{f}%
_{t}$ in terms of $\widetilde{\mathbf{v}}_{\left( 1\right) }$ and $%
\widetilde{\eta }_{t}$ as follows. 
\begin{eqnarray}
\widehat{f}_{t} &=&\frac{\left\langle \widehat{\pi }_{\left( 1\right)
},Z_{t}\right\rangle }{\sqrt{N}\left\Vert \widehat{\pi }_{\left( 1\right)
}\right\Vert _{2}}  \notag \\
&=&\frac{\left\Vert \gamma \right\Vert _{2}f_{t}}{\sqrt{N}}\left\langle 
\frac{\mathbf{v}_{\left( 1\right) }}{\left\Vert \mathbf{v}_{\left( 1\right)
}\right\Vert _{2}}\mathbf{,e}_{1,N}\right\rangle +\frac{1}{\sqrt{N}}%
\left\langle \frac{\mathbf{v}_{\left( 1\right) }}{\left\Vert \mathbf{v}%
_{\left( 1\right) }\right\Vert _{2}}\mathbf{,}\eta _{t}\right\rangle  \notag
\\
&=&\frac{\left\Vert \gamma \right\Vert _{2}}{\sqrt{N}}\left\langle \frac{%
\widetilde{\mathbf{v}}_{\left( 1\right) }}{\left\Vert \widetilde{\mathbf{v}}%
_{\left( 1\right) }\right\Vert _{2}}\mathbf{,e}_{1,N}\right\rangle f_{t}+%
\frac{1}{\sqrt{N}}\left\langle \frac{\widetilde{\mathbf{v}}_{\left( 1\right)
}}{\left\Vert \widetilde{\mathbf{v}}_{\left( 1\right) }\right\Vert _{2}}%
\mathbf{,}\widetilde{\eta }_{t}\right\rangle \text{.}  \label{f hat}
\end{eqnarray}%
Given the requirement in Assumption 2-2 that\textbf{\ } 
\begin{equation*}
\frac{N}{T\left\Vert \gamma \right\Vert _{2}^{2\left( 1+\kappa \right) }}%
=c+o\left( \frac{1}{\left\Vert \gamma \right\Vert _{2}^{2}}\right) \text{,
as }N,T\rightarrow \infty \text{,}
\end{equation*}%
for constants $c$ and $\kappa $ such that $0<c<\infty $ and $0<\kappa <1$;
it is easily seen that%
\begin{equation}
\frac{\left\Vert \gamma \right\Vert _{2}}{\sqrt{N}}=O\left( \left( \frac{1}{%
TN^{\kappa }}\right) ^{\frac{{\large 1}}{{\large 2}\left( {\large 1+\kappa }%
\right) }}\right) =o\left( 1\right) \text{.}  \label{gamma/N^(1/2)}
\end{equation}%
In the next two steps of this proof, we will show that%
\begin{equation*}
\left\langle \frac{\widetilde{\mathbf{v}}_{\left( 1\right) }}{\left\Vert 
\widetilde{\mathbf{v}}_{\left( 1\right) }\right\Vert _{2}}\mathbf{,e}%
_{1,N}\right\rangle \overset{p}{\rightarrow }0\text{ and }\frac{1}{\sqrt{N}}%
\left\langle \frac{\widetilde{\mathbf{v}}_{\left( 1\right) }}{\left\Vert 
\widetilde{\mathbf{v}}_{\left( 1\right) }\right\Vert _{2}}\mathbf{,}%
\widetilde{\eta }_{t}\right\rangle \overset{p}{\rightarrow }0\text{.}
\end{equation*}

\noindent \textbf{Step 4: }

We will first show that%
\begin{equation*}
\left\langle \frac{\widetilde{\mathbf{v}}_{\left( 1\right) }}{\left\Vert 
\widetilde{\mathbf{v}}_{1}\right\Vert _{2}}\mathbf{,e}_{1,N}\right\rangle 
\overset{p}{\rightarrow }0\text{. }
\end{equation*}%
. To proceed, note that, from expression (\ref{v1tilde}) in step 2, $%
\widetilde{\mathbf{v}}_{1}$ has the explicit form 
\begin{equation*}
\widetilde{\mathbf{v}}_{\left( 1\right) }=\left( 
\begin{array}{c}
1 \\ 
\widetilde{v}_{\left( 1\right) ,2} \\ 
\vdots \\ 
\widetilde{v}_{\left( 1\right) ,N}%
\end{array}%
\right) =\left( 
\begin{array}{c}
1 \\ 
\upsilon _{2}/\left( \widehat{\lambda }_{\left( 1\right) }-\widetilde{%
\lambda }_{\left( 2\right) }\right) \\ 
\vdots \\ 
\upsilon _{N}/\left( \widehat{\lambda }_{\left( 1\right) }-\widetilde{%
\lambda }_{\left( N\right) }\right)%
\end{array}%
\right)
\end{equation*}%
It follows that%
\begin{eqnarray*}
&&\frac{\left\langle \widetilde{\mathbf{v}}_{\left( 1\right) }\mathbf{,e}%
_{1,N}\right\rangle ^{2}}{\left\Vert \widetilde{\mathbf{v}}_{\left( 1\right)
}\right\Vert ^{2}} \\
&=&\left[ 1+\dsum\limits_{j=2}^{N}\frac{\upsilon _{j}^{2}}{\left( \widehat{%
\lambda }_{\left( 1\right) }-\widetilde{\lambda }_{\left( j\right) }\right)
^{2}}\right] ^{-1}\text{ \ } \\
&&\left( \text{since }\left\langle \widetilde{\mathbf{v}}_{\left( 1\right) }%
\mathbf{,e}_{1,N}\right\rangle =\left[ 
\begin{array}{cccc}
1 & \upsilon _{2}/\left( \widehat{\lambda }_{\left( 1\right) }-\widetilde{%
\lambda }_{\left( 2\right) }\right) & \cdots & \upsilon _{N}/\left( \widehat{%
\lambda }_{\left( 1\right) }-\widetilde{\lambda }_{\left( N\right) }\right)%
\end{array}%
\right] \left[ 
\begin{array}{c}
1 \\ 
0 \\ 
\vdots \\ 
0%
\end{array}%
\right] =1\right) \\
&=&\frac{1}{1+\tau ^{2}}
\end{eqnarray*}%
where%
\begin{equation*}
\tau ^{2}=\dsum\limits_{j=2}^{N}\frac{\upsilon _{j}^{2}}{\left( \widehat{%
\lambda }_{\left( 1\right) }-\widetilde{\lambda }_{\left( j\right) }\right)
^{2}}\text{. }
\end{equation*}%
Next, write%
\begin{eqnarray*}
\tau ^{2} &=&\dsum\limits_{j=2}^{N}\frac{\upsilon _{j}^{2}}{\left( \widehat{%
\lambda }_{\left( 1\right) }-\widetilde{\lambda }_{\left( j\right) }\right)
^{2}} \\
&=&\frac{N\left\Vert \gamma \right\Vert _{2}^{2}}{T}\frac{1}{\left\Vert
\gamma \right\Vert _{2}^{4\left( 1+\kappa \right) }}\frac{1}{\widehat{%
\lambda }_{\left( 1\right) }^{2}/\left\Vert \gamma \right\Vert _{2}^{4\left(
1+\kappa \right) }}\frac{1}{T}\dsum\limits_{j=2}^{N}\frac{T^{2}\upsilon
_{j}^{2}/\left( N\left\Vert \gamma \right\Vert _{2}^{2}\right) }{\left( 1-%
\widetilde{\lambda }_{\left( j\right) }/\widehat{\lambda }_{\left( 1\right)
}\right) ^{2}} \\
&=&\frac{N}{T\left\Vert \gamma \right\Vert _{2}^{2\left( 1+2\kappa \right) }}%
\frac{1}{\widehat{\lambda }_{\left( 1\right) }^{2}/\left\Vert \gamma
\right\Vert _{2}^{4\left( 1+\kappa \right) }}\frac{1}{T}\dsum%
\limits_{j=2}^{N}\frac{T^{2}\upsilon _{j}^{2}/\left( N\left\Vert \gamma
\right\Vert _{2}^{2}\right) }{\left( 1-\widetilde{\lambda }_{\left( j\right)
}/\widehat{\lambda }_{\left( 1\right) }\right) ^{2}}
\end{eqnarray*}%
Recall from step 2 that $\widehat{\lambda }_{\left( 1\right) }$ is the
largest eigenvalue of the sample covariance matrix 
\begin{equation*}
\widehat{\Sigma }_{\mathbf{W}}\mathbf{=}\frac{1}{T}\mathbf{WW}^{\prime
}=\left( 
\begin{array}{cc}
T^{-1}\mathbf{W}_{1}^{\prime }\mathbf{W}_{1} & T^{-1}\mathbf{W}_{1}^{\prime }%
\mathbf{W}_{2} \\ 
T^{-1}\mathbf{W}_{2}^{\prime }\mathbf{W}_{1} & T^{-1}\mathbf{W}_{2}^{\prime }%
\mathbf{W}_{2}%
\end{array}%
\right)
\end{equation*}%
while $\widetilde{\lambda }_{\left( j\right) }$ (for $j=2,...,N$) is the $%
\left( j-1\right) ^{th}$ largest eigenvalue of the submatrix $T^{-1}\mathbf{W%
}_{2}^{\prime }\mathbf{W}_{2}$. Applying Lemma B-9 and noting that $\widehat{%
\Sigma }_{\mathbf{W}}$ and $T^{-1}\mathbf{W}_{2}^{\prime }\mathbf{W}_{2}$
are positive semidefinite matrices whose elements are continuous random
variables, we see that%
\begin{equation*}
0\leq \frac{\widetilde{\lambda }_{\left( j\right) }}{\widehat{\lambda }%
_{\left( 1\right) }}<1\text{ }a.s.\text{ for }j=2,...,N\text{.}
\end{equation*}%
Note also that, by part (a) of Lemma B-5, $\widetilde{\lambda }_{\left(
j\right) }=0$ for $j=T+2,...,N$. Hence, we can further write%
\begin{equation}
\tau ^{2}\leq \frac{N}{T\left\Vert \gamma \right\Vert _{2}^{2\left(
1+2\kappa \right) }}\frac{1}{\widehat{\lambda }_{\left( 1\right)
}^{2}/\left\Vert \gamma \right\Vert _{2}^{4\left( 1+\kappa \right) }}\left(
1-\max_{2\leq j\leq T+1}\frac{\widetilde{\lambda }_{\left( j\right) }}{%
\widehat{\lambda }_{\left( 1\right) }}\right) ^{-2}\frac{1}{T}%
\dsum\limits_{j=2}^{N}\frac{T^{2}\upsilon _{j}^{2}}{N\left\Vert \gamma
\right\Vert _{2}^{2}}  \label{tau^2}
\end{equation}

To analyze the asymptotic behavior of $\tau ^{2}$, note first that we can
apply the result of Lemma B-10 in Appendix B below to obtain%
\begin{eqnarray*}
\frac{\widehat{\lambda }_{\left( 1\right) }^{2}}{\left\Vert \gamma
\right\Vert _{2}^{4\left( 1+\kappa \right) }} &=&\left[ \frac{\widehat{%
\lambda }_{\left( 1\right) }}{\left\Vert \gamma \right\Vert _{2}^{2\left(
1+\kappa \right) }}\right] ^{2} \\
&=&\left[ c+\frac{1}{\left\Vert \gamma \right\Vert _{2}^{2\kappa }}%
+o_{p}\left( \frac{1}{\left\Vert \gamma \right\Vert _{2}^{2\kappa }}\right) %
\right] ^{2} \\
&=&c^{2}\left[ 1+O_{p}\left( \frac{1}{\left\Vert \gamma \right\Vert
_{2}^{2\kappa }}\right) \right] .
\end{eqnarray*}%
from which it follows that%
\begin{equation}
\frac{1}{\widehat{\lambda }_{\left( 1\right) }^{2}/\left\Vert \gamma
\right\Vert _{2}^{4\left( 1+\kappa \right) }}=\frac{1}{c^{2}}\left[
1+O_{p}\left( \frac{1}{\left\Vert \gamma \right\Vert _{2}^{2\kappa }}\right) %
\right]  \label{gamma^4/lambda^2}
\end{equation}%
where $0<1/c^{2}<\infty $ given that $0<c<\infty $.

Next, consider $\left( 1-\max_{2\leq j\leq T+1}\left[ \widetilde{\lambda }%
_{\left( j\right) }/\widehat{\lambda }_{\left( 1\right) }\right] \right)
^{-2}$. To analyze its asymptotic behavior, we make use of Assumption 2-2,
part (b) of Lemma B-5, and Lemma B-10 to obtain%
\begin{eqnarray*}
&&\max_{2\leq j\leq T+1}\frac{\widetilde{\lambda }_{\left( j\right) }}{%
\widehat{\lambda }_{\left( 1\right) }} \\
&=&\frac{N-1}{T\left\Vert \gamma \right\Vert _{2}^{2\left( 1+\kappa \right) }%
}\frac{1}{\widehat{\lambda }_{\left( 1\right) }/\left\Vert \gamma
\right\Vert _{2}^{2\left( 1+\kappa \right) }}\frac{T}{N-1}\max_{2\leq j\leq
T+1}\widetilde{\lambda }_{\left( j\right) } \\
&=&\left[ c+o\left( \frac{1}{\left\Vert \gamma \right\Vert _{2}^{2}}\right) %
\right] \left[ c+\frac{1}{\left\Vert \gamma \right\Vert _{2}^{2\kappa }}%
+o_{p}\left( \frac{1}{\left\Vert \gamma \right\Vert _{2}^{2\kappa }}\right) %
\right] ^{-1}\left[ 1+O_{p}\left( \sqrt{\frac{T}{N}}\right) \right] \\
&=&\left[ c+o\left( \frac{1}{\left\Vert \gamma \right\Vert _{2}^{2}}\right) %
\right] \left[ c+\frac{1}{\left\Vert \gamma \right\Vert _{2}^{2\kappa }}%
\right] ^{-1}\left[ 1+o_{p}\left( \frac{1}{\left\Vert \gamma \right\Vert
_{2}^{2\kappa }}\right) \right] \left[ 1+O_{p}\left( \sqrt{\frac{T}{N}}%
\right) \right] \\
&=&\left[ c+o\left( \frac{1}{\left\Vert \gamma \right\Vert _{2}^{2}}\right) %
\right] \frac{1}{c}\left[ 1+\frac{1}{c\left\Vert \gamma \right\Vert
_{2}^{2\kappa }}\right] ^{-1}\left[ 1+o_{p}\left( \frac{1}{\left\Vert \gamma
\right\Vert _{2}^{2\kappa }}\right) \right] \left[ 1+O_{p}\left( \sqrt{\frac{%
T}{N}}\right) \right] \\
&=&\left[ 1+o\left( \frac{1}{\left\Vert \gamma \right\Vert _{2}^{2}}\right) %
\right] \left[ 1-\frac{1}{c\left\Vert \gamma \right\Vert _{2}^{2\kappa }}%
+O\left( \frac{1}{\left\Vert \gamma \right\Vert _{2}^{4\kappa }}\right) %
\right] \left[ 1+o_{p}\left( \frac{1}{\left\Vert \gamma \right\Vert
_{2}^{2\kappa }}\right) \right] \left[ 1+O_{p}\left( \sqrt{\frac{T}{N}}%
\right) \right] \\
&=&\left[ 1-\frac{1}{c\left\Vert \gamma \right\Vert _{2}^{2\kappa }}+O\left( 
\frac{1}{\left\Vert \gamma \right\Vert _{2}^{4\kappa }}\right) +o\left( 
\frac{1}{\left\Vert \gamma \right\Vert _{2}^{2}}\right) \right] \left[
1+o_{p}\left( \frac{1}{\left\Vert \gamma \right\Vert _{2}^{2\kappa }}\right) %
\right] \left[ 1+O_{p}\left( \sqrt{\frac{T}{N}}\right) \right] \\
&=&\left[ 1-\frac{1}{c\left\Vert \gamma \right\Vert _{2}^{2\kappa }}\right] %
\left[ 1+O\left( \frac{1}{\left\Vert \gamma \right\Vert _{2}^{4\kappa }}%
\right) +o\left( \frac{1}{\left\Vert \gamma \right\Vert _{2}^{2}}\right) %
\right] \left[ 1+o_{p}\left( \frac{1}{\left\Vert \gamma \right\Vert
_{2}^{2\kappa }}\right) \right] \left[ 1+O_{p}\left( \sqrt{\frac{T}{N}}%
\right) \right] \\
&=&\left[ 1-\frac{1}{c\left\Vert \gamma \right\Vert _{2}^{2\kappa }}\right] %
\left[ 1+o_{p}\left( \frac{1}{\left\Vert \gamma \right\Vert _{2}^{2\kappa }}%
\right) \right]
\end{eqnarray*}%
so that%
\begin{eqnarray*}
1-\max_{2\leq j\leq T+1}\frac{\widetilde{\lambda }_{\left( j\right) }}{%
\widehat{\lambda }_{\left( 1\right) }} &=&1-\left[ 1-\frac{1}{c\left\Vert
\gamma \right\Vert _{2}^{2\kappa }}\right] \left[ 1+o_{p}\left( \frac{1}{%
\left\Vert \gamma \right\Vert _{2}^{2\kappa }}\right) \right] \\
&=&1-1+\frac{1}{c\left\Vert \gamma \right\Vert _{2}^{2\kappa }}+o_{p}\left( 
\frac{1}{\left\Vert \gamma \right\Vert _{2}^{2\kappa }}\right) \\
&=&\frac{1}{c\left\Vert \gamma \right\Vert _{2}^{2\kappa }}+o_{p}\left( 
\frac{1}{\left\Vert \gamma \right\Vert _{2}^{2\kappa }}\right) \\
&=&\frac{1}{c\left\Vert \gamma \right\Vert _{2}^{2\kappa }}\left[
1+o_{p}\left( 1\right) \right]
\end{eqnarray*}%
and, thus,%
\begin{equation}
\left( 1-\max_{2\leq j\leq T+1}\frac{\widetilde{\lambda }_{\left( j\right) }%
}{\widehat{\lambda }_{\left( 1\right) }}\right) ^{-2}=c^{2}\left\Vert \gamma
\right\Vert _{2}^{4\kappa }\left[ 1+o_{p}\left( 1\right) \right] \text{.}
\label{1-maxlambdaratio}
\end{equation}

Now, consider $T^{-1}\dsum\nolimits_{j=2}^{N}T^{2}\upsilon _{j}^{2}/\left(
N\left\Vert \gamma \right\Vert _{2}^{2}\right) $. To proceed, note first that%
\begin{equation*}
W_{1,t}=\left\Vert \gamma \right\Vert _{2}f_{t}+\eta _{1t}=\left\Vert \gamma
\right\Vert _{2}f_{t}+\mathbf{e}_{1,N}^{\prime }\Pi ^{\prime }u_{t}
\end{equation*}%
so that, given Assumption 2-1 and given the fact that $\Pi $ is an
orthogonal matrix, we have that 
\begin{equation*}
\left\{ W_{1,t}\right\} \equiv i.i.d.\text{ }N\left( 0,\left\Vert \gamma
\right\Vert _{2}^{2}+1\right)
\end{equation*}%
from which we further deduce that%
\begin{equation*}
\frac{\mathbf{W}_{1}}{\left\Vert \gamma \right\Vert _{2}}=\left( 
\begin{array}{c}
W_{1,1}/\left\Vert \gamma \right\Vert _{2} \\ 
W_{1,2}/\left\Vert \gamma \right\Vert _{2} \\ 
\vdots \\ 
W_{1,T}/\left\Vert \gamma \right\Vert _{2}%
\end{array}%
\right) \sim N\left( 0,\left\{ 1+\frac{1}{\left\Vert \gamma \right\Vert
_{2}^{2}}\right\} I_{T}\right)
\end{equation*}%
Moreover, note that%
\begin{equation*}
\mathbf{W}_{2,t}=\left( 
\begin{array}{c}
\eta _{2t} \\ 
\vdots \\ 
\eta _{Nt}%
\end{array}%
\right) =\left( 
\begin{array}{c}
\mathbf{e}_{2,N}^{\prime }\Pi ^{\prime }u_{t} \\ 
\vdots \\ 
\mathbf{e}_{N,N}^{\prime }\Pi ^{\prime }u_{t}%
\end{array}%
\right)
\end{equation*}%
so that, under Assumption 2-1, 
\begin{equation*}
\left\{ \mathbf{W}_{2,t}\right\} \equiv i.i.d.N\left( 0,I_{N-1}\right)
\end{equation*}

\noindent By direct calculation, we have for $j=2,...,T+1$ 
\begin{eqnarray*}
E\left[ \frac{T^{2}\upsilon _{j}^{2}}{N\left\Vert \gamma \right\Vert _{2}^{2}%
}|\mathbf{W}_{2}\right] &=&T^{2}\frac{\mathbf{e}_{j-1,N-1}^{\prime }%
\widetilde{\mathbf{B}}_{2}^{\prime }\mathbf{W}_{2}^{\prime }E\left[ \mathbf{W%
}_{1}\mathbf{W}_{1}^{\prime }|\mathbf{W}_{2}\right] \mathbf{W}_{2}\widetilde{%
\mathbf{B}}_{2}\mathbf{e}_{j-1,N-1}}{N\left\Vert \gamma \right\Vert
_{2}^{2}T^{2}}\text{ } \\
&&\left( \text{ since}\underset{\left( N-1\right) \times 1}{\upsilon }=\frac{%
\widetilde{\mathbf{B}}_{2}^{\prime }\mathbf{W}_{2}^{\prime }\mathbf{W}_{1}}{T%
}\right) \\
&=&\frac{T}{N}\frac{\mathbf{e}_{j-1,N-1}^{\prime }\widetilde{\mathbf{B}}%
_{2}^{\prime }\mathbf{W}_{2}^{\prime }E\left[ \mathbf{W}_{1}\mathbf{W}%
_{1}^{\prime }\right] \mathbf{W}_{2}\widetilde{\mathbf{B}}_{2}\mathbf{e}%
_{j-1,N-1}}{\left\Vert \gamma \right\Vert _{2}^{2}T}\text{ \ } \\
&&\text{(by independence of }\mathbf{W}_{1}\text{ and }\mathbf{W}_{2}\text{)}
\\
&=&\frac{T}{N}\left( 1+\frac{1}{\left\Vert \gamma \right\Vert _{2}^{2}}%
\right) \frac{\mathbf{e}_{j-1,N-1}^{\prime }\widetilde{\mathbf{B}}%
_{2}^{\prime }\mathbf{W}_{2}^{\prime }\mathbf{W}_{2}\widetilde{\mathbf{B}}%
_{2}\mathbf{e}_{j-1,N-1}}{T} \\
&=&\frac{T}{N}\left( 1+\frac{1}{\left\Vert \gamma \right\Vert _{2}^{2}}%
\right) \mathbf{e}_{j-1,N-1}^{\prime }\widetilde{\mathbf{B}}_{2}^{\prime }%
\widetilde{\mathbf{B}}_{2}\widetilde{\Lambda }\widetilde{\mathbf{B}}%
_{2}^{\prime }\widetilde{\mathbf{B}}_{2}\mathbf{e}_{j-1,N-1}\text{ } \\
&&\left( \text{since }\frac{\mathbf{W}_{2}^{\prime }\mathbf{W}_{2}}{T}=%
\widetilde{\mathbf{B}}_{2}\widetilde{\Lambda }\widetilde{\mathbf{B}}%
_{2}^{\prime }\right) \\
&=&\left( 1+\frac{1}{\left\Vert \gamma \right\Vert _{2}^{2}}\right) \frac{T%
\widetilde{\lambda }_{\left( j\right) }}{N}
\end{eqnarray*}%
In addition, by straightforward calculation, we also get for $j=2,...,T+1$

\begin{eqnarray*}
&&E\left[ \frac{T^{4}\upsilon _{j}^{4}}{N^{2}\left\Vert \gamma \right\Vert
_{2}^{4}}|\mathbf{W}_{2}\right] \\
&=&\frac{T^{4}}{N^{2}}E\left\{ \left( \frac{\mathbf{e}_{j-1,N-1}^{\prime }%
\widetilde{\mathbf{B}}_{2}^{\prime }\mathbf{W}_{2}^{\prime }\mathbf{W}_{1}%
\mathbf{W}_{1}^{\prime }\mathbf{W}_{2}\widetilde{\mathbf{B}}_{2}\mathbf{e}%
_{j-1,N-1}}{\left\Vert \gamma \right\Vert _{2}^{2}T^{2}}\right) ^{2}|\mathbf{%
W}_{2}\right\} \\
&=&\frac{T^{4}}{N^{2}T^{4}}\dsum\limits_{r=1}^{T}\dsum\limits_{s=1}^{T}\dsum%
\limits_{t=1}^{T}\dsum\limits_{\upsilon =1}^{T}\left\{ E\left[ \frac{W_{1,r}%
}{\left\Vert \gamma \right\Vert _{2}}\frac{W_{1,s}}{\left\Vert \gamma
\right\Vert _{2}}\frac{W_{1,t}}{\left\Vert \gamma \right\Vert _{2}}\frac{%
W_{1,\upsilon }}{\left\Vert \gamma \right\Vert _{2}}|\mathbf{W}_{2}\right]
\left( \mathbf{W}_{2,r}^{\prime }\widetilde{\mathbf{B}}_{2}\mathbf{e}%
_{j-1,N-1}\right) \right. \\
&&\left. \times \left( \mathbf{W}_{2,s}^{\prime }\widetilde{\mathbf{B}}_{2}%
\mathbf{e}_{j-1,N-1}\right) \left( \mathbf{W}_{2,t}^{\prime }\widetilde{%
\mathbf{B}}_{2}\mathbf{e}_{j-1,N-1}\right) \left( \mathbf{W}_{2,\upsilon
}^{\prime }\widetilde{\mathbf{B}}_{2}\mathbf{e}_{j-1,N-1}\right) \right\} \\
&=&\frac{T^{4}}{N^{2}T^{4}}\dsum\limits_{t=1}^{T}E\left[ \frac{W_{1,t}^{4}}{%
\left\Vert \gamma \right\Vert _{2}^{4}}\right] \left( \mathbf{e}%
_{j-1,N-1}^{\prime }\widetilde{\mathbf{B}}_{2}^{\prime }\mathbf{W}_{2,t}%
\mathbf{W}_{2,t}^{\prime }\widetilde{\mathbf{B}}_{2}\mathbf{e}%
_{j-1,N-1}\right) ^{2} \\
&&+\frac{3T^{4}}{N^{2}T^{4}}\left\{ \dsum\limits_{t=1}^{T}E\left[ \frac{%
W_{1,t}^{2}}{\left\Vert \gamma \right\Vert _{2}^{2}}\right] \left( \mathbf{e}%
_{j-1,N-1}^{\prime }\widetilde{\mathbf{B}}_{2}^{\prime }\mathbf{W}_{2,t}%
\mathbf{W}_{2,t}^{\prime }\widetilde{\mathbf{B}}_{2}\mathbf{e}%
_{j-1,N-1}\right) \right. \\
&&\text{ \ \ \ \ \ \ \ \ \ \ \ \ \ }\left. \text{\ }\times
\dsum\limits_{s\neq t}E\left[ \frac{W_{1,s}^{2}}{\left\Vert \gamma
\right\Vert _{2}^{2}}\right] \left( \mathbf{e}_{j-1,N-1}^{\prime }\widetilde{%
\mathbf{B}}_{2}^{\prime }\mathbf{W}_{2,s}\mathbf{W}_{2,s}^{\prime }%
\widetilde{\mathbf{B}}_{2}\mathbf{e}_{j-1,N-1}\right) \right\} \\
&=&3\left( 1+\frac{1}{\left\Vert \gamma \right\Vert _{2}^{2}}\right) ^{2}%
\frac{T^{4}}{N^{2}T^{2}}\left( \dsum\limits_{t=1}^{T}\frac{\mathbf{e}%
_{j-1,N-1}^{\prime }\widetilde{\mathbf{B}}_{2}^{\prime }\mathbf{W}_{2,t}%
\mathbf{W}_{2,t}^{\prime }\widetilde{\mathbf{B}}_{2}\mathbf{e}_{j-1,N-1}}{T}%
\right) ^{2} \\
&&\left( \text{since }\frac{W_{1,t}}{\left\Vert \gamma \right\Vert _{2}}%
=f_{t}+\left\Vert \gamma \right\Vert _{2}^{-1}\eta _{1t}\sim N\left( 0,1+%
\frac{1}{\left\Vert \gamma \right\Vert _{2}^{2}}\right) \right) \\
&=&3\left( 1+\frac{1}{\left\Vert \gamma \right\Vert _{2}^{2}}\right)
^{2}\left( \frac{T}{N}\widetilde{\lambda }_{\left( j\right) }\right) ^{2}
\end{eqnarray*}

\noindent On the other hand, for $j=T+2,..,N-1$, we have 
\begin{equation*}
E\left[ \frac{T^{2}\upsilon _{j}^{2}}{N\left\Vert \gamma \right\Vert _{2}^{2}%
}|\mathbf{W}_{2}\right] =\left( 1+\frac{1}{\left\Vert \gamma \right\Vert
_{2}^{2}}\right) \frac{T\widetilde{\lambda }_{\left( j\right) }}{N}=0\text{ }
\end{equation*}%
and%
\begin{equation*}
E\left[ \frac{T^{4}\upsilon _{j}^{4}}{N^{2}\left\Vert \gamma \right\Vert
_{2}^{4}}|\mathbf{W}_{2}\right] =3\left( 1+\frac{1}{\left\Vert \gamma
\right\Vert _{2}^{2}}\right) ^{2}\left( \frac{T}{N}\widetilde{\lambda }%
_{\left( j\right) }\right) ^{2}=0
\end{equation*}%
since $\widetilde{\lambda }_{\left( j\right) }=0$ for $j>T+1$ by part (a) of
Lemma B-5.

Next, we show that 
\begin{equation*}
E\left\{ \left( \frac{1}{T}\dsum\limits_{j=2}^{N}\frac{T^{2}\upsilon _{j}^{2}%
}{N\left\Vert \gamma \right\Vert _{2}^{2}}-\frac{1}{T}\dsum\limits_{j=2}^{N}E%
\left[ \frac{T^{2}\upsilon _{j}^{2}}{N\left\Vert \gamma \right\Vert _{2}^{2}}%
|\mathbf{W}_{2}\right] \right) ^{2}|\mathbf{W}_{2}\right\} =O_{a.s.}\left( 
\frac{1}{T}\right)
\end{equation*}%
To proceed, write%
\begin{eqnarray}
&&E\left\{ \left( \frac{1}{T}\dsum\limits_{j=2}^{N}\frac{T^{2}\upsilon
_{j}^{2}}{N\left\Vert \gamma \right\Vert _{2}^{2}}-\frac{1}{T}%
\dsum\limits_{j=2}^{N}E\left[ \frac{T^{2}\upsilon _{j}^{2}}{N\left\Vert
\gamma \right\Vert _{2}^{2}}|\mathbf{W}_{2}\right] \right) ^{2}|\mathbf{W}%
_{2}\right\}  \notag \\
&=&E\left\{ \left( \frac{1}{T}\dsum\limits_{j=2}^{N}\left[ \frac{%
T^{2}\upsilon _{j}^{2}}{N\left\Vert \gamma \right\Vert _{2}^{2}}-\left( 1+%
\frac{1}{\left\Vert \gamma \right\Vert _{2}^{2}}\right) \frac{T\widetilde{%
\lambda }_{\left( j\right) }}{N}\right] \right) ^{2}|\mathbf{W}_{2}\right\} 
\notag \\
&=&\frac{1}{T^{2}}\dsum\limits_{j=2}^{N}E\left\{ \left[ \frac{T^{2}\upsilon
_{j}^{2}}{N\left\Vert \gamma \right\Vert _{2}^{2}}-\left( 1+\frac{1}{%
\left\Vert \gamma \right\Vert _{2}^{2}}\right) \frac{T\widetilde{\lambda }%
_{\left( j\right) }}{N}\right] ^{2}|\mathbf{W}_{2}\right\}  \notag \\
&&+\frac{1}{T^{2}}\dsum\limits_{j\neq k}E\left\{ \left[ \frac{T^{2}\upsilon
_{j}^{2}}{N\left\Vert \gamma \right\Vert _{2}^{2}}-\left( 1+\frac{1}{%
\left\Vert \gamma \right\Vert _{2}^{2}}\right) \frac{T\widetilde{\lambda }%
_{\left( j\right) }}{N}\right] \right.  \notag \\
&&\text{ \ \ \ \ \ \ \ \ \ \ \ \ \ \ \ \ \ \ }\left. \times \left[ \frac{%
T^{2}\upsilon _{k}^{2}}{N\left\Vert \gamma \right\Vert _{2}^{2}}-\left( 1+%
\frac{1}{\left\Vert \gamma \right\Vert _{2}^{2}}\right) \frac{T\widetilde{%
\lambda }_{\left( k\right) }}{N}\right] |\mathbf{W}_{2}\right\}
\label{cond MS calc}
\end{eqnarray}%
Consider the second term on the right-hand side of expression (\ref{cond MS
calc})%
\begin{eqnarray}
&&\frac{1}{T^{2}}\dsum\limits_{j\neq k}E\left\{ \left[ \frac{T^{2}\upsilon
_{j}^{2}}{N\left\Vert \gamma \right\Vert _{2}^{2}}-\left( 1+\frac{1}{%
\left\Vert \gamma \right\Vert _{2}^{2}}\right) \frac{T\widetilde{\lambda }%
_{\left( j\right) }}{N}\right] \left[ \frac{T^{2}\upsilon _{k}^{2}}{%
N\left\Vert \gamma \right\Vert _{2}^{2}}-\left( 1+\frac{1}{\left\Vert \gamma
\right\Vert _{2}^{2}}\right) \frac{T\widetilde{\lambda }_{\left( k\right) }}{%
N}\right] |\mathbf{W}_{2}\right\}  \notag \\
&=&\frac{1}{T^{2}}\dsum\limits_{j\neq k}E\left[ \frac{T^{4}\upsilon
_{j}^{2}\upsilon _{k}^{2}}{N^{2}\left\Vert \gamma \right\Vert _{2}^{4}}|%
\mathbf{W}_{2}\right] -\frac{1}{T^{2}}\left( 1+\frac{1}{\left\Vert \gamma
\right\Vert _{2}^{2}}\right) ^{2}\dsum\limits_{j\neq k}\left( \frac{T%
\widetilde{\lambda }_{\left( j\right) }}{N}\right) \left( \frac{T\widetilde{%
\lambda }_{\left( k\right) }}{N}\right)  \label{2nd term MS calc}
\end{eqnarray}%
For the first term in expression (\ref{2nd term MS calc}), note that%
\begin{eqnarray}
&&E\left[ \frac{T^{4}\upsilon _{j}^{2}\upsilon _{k}^{2}}{N^{2}\left\Vert
\gamma \right\Vert _{2}^{4}}|\mathbf{W}_{2}\right]  \notag \\
&=&\frac{T^{4}}{N^{2}}E\left\{ \left( \frac{\mathbf{e}_{j-1,N-1}^{\prime }%
\widetilde{\mathbf{B}}_{2}^{\prime }\mathbf{W}_{2}^{\prime }\mathbf{W}_{1}%
\mathbf{W}_{1}^{\prime }\mathbf{W}_{2}\widetilde{\mathbf{B}}_{2}\mathbf{e}%
_{j-1,N-1}}{\left\Vert \gamma \right\Vert _{2}^{2}T^{2}}\right) \right. 
\notag \\
&&\text{ \ \ \ \ \ \ \ \ \ }\left. \times \left( \frac{\mathbf{e}%
_{k-1,N-1}^{\prime }\widetilde{\mathbf{B}}_{2}^{\prime }\mathbf{W}%
_{2}^{\prime }\mathbf{W}_{1}\mathbf{W}_{1}^{\prime }\mathbf{W}_{2}\widetilde{%
\mathbf{B}}_{2}\mathbf{e}_{k-1,N-1}}{\left\Vert \gamma \right\Vert
_{2}^{2}T^{2}}\right) |\mathbf{W}_{2}\right\}  \notag \\
&=&\frac{T^{4}}{N^{2}T^{4}}\dsum\limits_{r=1}^{T}\dsum\limits_{s=1}^{T}\dsum%
\limits_{t=1}^{T}\dsum\limits_{\upsilon =1}^{T}\left\{ E\left[ \frac{W_{1,r}%
}{\left\Vert \gamma \right\Vert _{2}}\frac{W_{1,s}}{\left\Vert \gamma
\right\Vert _{2}}\frac{W_{1,t}}{\left\Vert \gamma \right\Vert _{2}}\frac{%
W_{1,\upsilon }}{\left\Vert \gamma \right\Vert _{2}}|\mathbf{W}_{2}\right]
\left( \mathbf{W}_{2,r}^{\prime }\widetilde{\mathbf{B}}_{2}\mathbf{e}%
_{j-1,N-1}\right) \right.  \notag \\
&&\left. \times \left( \mathbf{W}_{2,s}^{\prime }\widetilde{\mathbf{B}}_{2}%
\mathbf{e}_{j-1,N-1}\right) \left( \mathbf{W}_{2,t}^{\prime }\widetilde{%
\mathbf{B}}_{2}\mathbf{e}_{k-1,N-1}\right) \left( \mathbf{W}_{2,\upsilon
}^{\prime }\widetilde{\mathbf{B}}_{2}\mathbf{e}_{k-1,N-1}\right) \right\} 
\notag \\
&=&\frac{T^{4}}{N^{2}T^{4}}\dsum\limits_{t=1}^{T}E\left\{ \left[ \frac{%
W_{1,t}^{4}}{\left\Vert \gamma \right\Vert _{2}^{4}}\right] \left( \mathbf{e}%
_{j-1,N-1}^{\prime }\widetilde{\mathbf{B}}_{2}^{\prime }\mathbf{W}_{2,t}%
\mathbf{W}_{2,t}^{\prime }\widetilde{\mathbf{B}}_{2}\mathbf{e}%
_{j-1,N-1}\right) \right.  \notag \\
&&\text{ \ \ \ \ \ \ \ \ \ \ \ \ \ \ \ \ \ \ }\left. \times \left( \mathbf{e}%
_{k-1,N-1}^{\prime }\widetilde{\mathbf{B}}_{2}^{\prime }\mathbf{W}_{2,t}%
\mathbf{W}_{2,t}^{\prime }\widetilde{\mathbf{B}}_{2}\mathbf{e}%
_{k-1,N-1}\right) \right\}  \notag \\
&&+\frac{T^{4}}{N^{2}T^{4}}\left\{ \dsum\limits_{s=1}^{T}E\left[ \frac{%
W_{1,s}^{2}}{\left\Vert \gamma \right\Vert _{2}^{2}}\right] \left( \mathbf{e}%
_{j-1,N-1}^{\prime }\widetilde{\mathbf{B}}_{2}^{\prime }\mathbf{W}_{2,s}%
\mathbf{W}_{2,s}^{\prime }\widetilde{\mathbf{B}}_{2}\mathbf{e}%
_{j-1,N-1}\right) \right.  \notag \\
&&\text{ \ \ \ \ \ \ \ \ \ \ \ \ \ \ }\left. \text{\ }\times
\dsum\limits_{t\neq s}E\left[ \frac{W_{1,t}^{2}}{\left\Vert \gamma
\right\Vert _{2}^{2}}\right] \left( \mathbf{e}_{k-1,N-1}^{\prime }\widetilde{%
\mathbf{B}}_{2}^{\prime }\mathbf{W}_{2,t}\mathbf{W}_{2,t}^{\prime }%
\widetilde{\mathbf{B}}_{2}\mathbf{e}_{k-1,N-1}\right) \right\}  \notag \\
&&+\frac{T^{4}}{N^{2}T^{4}}\left\{ \dsum\limits_{t=1}^{T}E\left[ \frac{%
W_{1,t}^{2}}{\left\Vert \gamma \right\Vert _{2}^{2}}\right] \left( \mathbf{e}%
_{j-1,N-1}^{\prime }\widetilde{\mathbf{B}}_{2}^{\prime }\mathbf{W}_{2,t}%
\mathbf{W}_{2,t}^{\prime }\widetilde{\mathbf{B}}_{2}\mathbf{e}%
_{k-1,N-1}\right) \right.  \notag \\
&&\text{ \ \ \ \ \ \ \ \ \ \ \ \ \ }\left. \text{\ }\times
\dsum\limits_{s\neq t}E\left[ \frac{W_{1,s}^{2}}{\left\Vert \gamma
\right\Vert _{2}^{2}}\right] \left( \mathbf{e}_{j-1,N-1}^{\prime }\widetilde{%
\mathbf{B}}_{2}^{\prime }\mathbf{W}_{2,s}\mathbf{W}_{2,s}^{\prime }%
\widetilde{\mathbf{B}}_{2}\mathbf{e}_{k-1,N-1}\right) \right\}  \notag \\
&&+\frac{T^{4}}{N^{2}T^{4}}\left\{ \dsum\limits_{r=1}^{T}E\left[ \frac{%
W_{1,r}^{2}}{\left\Vert \gamma \right\Vert _{2}^{2}}\right] \left( \mathbf{e}%
_{j-1,N-1}^{\prime }\widetilde{\mathbf{B}}_{2}^{\prime }\mathbf{W}_{2,r}%
\mathbf{W}_{2,r}^{\prime }\widetilde{\mathbf{B}}_{2}\mathbf{e}%
_{k-1,N-1}\right) \right.  \notag \\
&&\text{ \ \ \ \ \ \ \ \ \ \ \ \ \ }\left. \text{\ }\times
\dsum\limits_{t\neq r}E\left[ \frac{W_{1,s}^{2}}{\left\Vert \gamma
\right\Vert _{2}^{2}}\right] \left( \mathbf{e}_{j-1,N-1}^{\prime }\widetilde{%
\mathbf{B}}_{2}^{\prime }\mathbf{W}_{2,t}\mathbf{W}_{2,t}^{\prime }%
\widetilde{\mathbf{B}}_{2}\mathbf{e}_{k-1,N-1}\right) \right\}
\label{cross term}
\end{eqnarray}%
Calculating the expectation for the first term on the right-hand side of
expression (\ref{cross term}) above, we have%
\begin{eqnarray*}
&&\frac{T^{4}}{N^{2}T^{4}}\dsum\limits_{t=1}^{T}\left\{ E\left[ \frac{%
W_{1,t}^{4}}{\left\Vert \gamma \right\Vert _{2}^{4}}\right] \left( \mathbf{e}%
_{j-1,N-1}^{\prime }\widetilde{\mathbf{B}}_{2}^{\prime }\mathbf{W}_{2,t}%
\mathbf{W}_{2,t}^{\prime }\widetilde{\mathbf{B}}_{2}\mathbf{e}%
_{j-1,N-1}\right) \right. \\
&&\text{ \ \ \ \ \ \ \ \ \ \ \ \ \ \ \ \ \ \ \ \ \ \ \ \ \ \ \ \ \ \ \ \ }%
\left. \times \left( \mathbf{e}_{k-1,N-1}^{\prime }\widetilde{\mathbf{B}}%
_{2}^{\prime }\mathbf{W}_{2,t}\mathbf{W}_{2,t}^{\prime }\widetilde{\mathbf{B}%
}_{2}\mathbf{e}_{k-1,N-1}\right) \right\} \\
&=&\frac{3T^{4}}{N^{2}T^{4}}\left( 1+\frac{1}{\left\Vert \gamma \right\Vert
_{2}^{2}}\right) ^{2}\dsum\limits_{t=1}^{T}\left\{ \left( \mathbf{e}%
_{j-1,N-1}^{\prime }\widetilde{\mathbf{B}}_{2}^{\prime }\mathbf{W}_{2,t}%
\mathbf{W}_{2,t}^{\prime }\widetilde{\mathbf{B}}_{2}\mathbf{e}%
_{j-1,N-1}\right) \right. \\
&&\text{ \ \ \ \ \ \ \ \ \ \ \ \ \ \ \ \ \ \ \ \ \ \ \ \ \ \ \ \ \ \ \ \ \ \
\ \ \ \ \ }\left. \times \left( \mathbf{e}_{k-1,N-1}^{\prime }\widetilde{%
\mathbf{B}}_{2}^{\prime }\mathbf{W}_{2,t}\mathbf{W}_{2,t}^{\prime }%
\widetilde{\mathbf{B}}_{2}\mathbf{e}_{k-1,N-1}\right) \right\}
\end{eqnarray*}%
Moreover, using the fact that%
\begin{eqnarray*}
\mathbf{e}_{j-1,N-1}^{\prime }\widetilde{\mathbf{B}}_{2}^{\prime
}\dsum\limits_{s=1}^{T}\frac{\mathbf{W}_{2,s}\mathbf{W}_{2,s}^{\prime }}{T}%
\widetilde{\mathbf{B}}_{2}\mathbf{e}_{j-1,N-1} &=&\mathbf{e}%
_{j-1,N-1}^{\prime }\widetilde{\mathbf{B}}_{2}^{\prime }\widetilde{\mathbf{B}%
}_{2}\widetilde{\Lambda }\widetilde{\mathbf{B}}_{2}^{\prime }\widetilde{%
\mathbf{B}}_{2}\mathbf{e}_{j-1,N-1} \\
&=&\mathbf{e}_{j-1,N-1}^{\prime }\widetilde{\Lambda }\mathbf{e}_{j-1,N-1} \\
&=&\widetilde{\lambda }_{\left( j\right) }
\end{eqnarray*}%
and, for $j\neq k$,%
\begin{eqnarray*}
\mathbf{e}_{j-1,N-1}^{\prime }\widetilde{\mathbf{V}}_{2}^{\prime
}\dsum\limits_{s=1}^{T}\frac{\mathbf{W}_{2,s}\mathbf{W}_{2,s}^{\prime }}{T}%
\widetilde{\mathbf{V}}_{2}\mathbf{e}_{k-1,N-1} &=&\mathbf{e}%
_{j-1,N-1}^{\prime }\widetilde{\mathbf{B}}_{2}^{\prime }\widetilde{\mathbf{B}%
}_{2}\widetilde{\Lambda }\widetilde{\mathbf{B}}_{2}^{\prime }\widetilde{%
\mathbf{B}}_{2}\mathbf{e}_{k-1,N-1} \\
&=&\mathbf{e}_{j-1,N-1}^{\prime }\widetilde{\Lambda }\mathbf{e}_{k-1,N-1} \\
&=&0
\end{eqnarray*}%
we further obtain%
\begin{eqnarray*}
&&\frac{T^{4}}{N^{2}T^{4}}\left\{ \dsum\limits_{s=1}^{T}E\left[ \frac{%
W_{1,s}^{2}}{\left\Vert \gamma \right\Vert _{2}^{2}}\right] \left( \mathbf{e}%
_{j-1,N-1}^{\prime }\widetilde{\mathbf{B}}_{2}^{\prime }\mathbf{W}_{2,s}%
\mathbf{W}_{2,s}^{\prime }\widetilde{\mathbf{B}}_{2}\mathbf{e}%
_{j-1,N-1}\right) \right. \\
&&\text{ \ \ \ \ \ \ \ \ \ \ \ \ }\left. \text{\ }\times \dsum\limits_{t\neq
s}E\left[ \frac{W_{1,t}^{2}}{\left\Vert \gamma \right\Vert _{2}^{2}}\right]
\left( \mathbf{e}_{k-1,N-1}^{\prime }\widetilde{\mathbf{B}}_{2}^{\prime }%
\mathbf{W}_{2,t}\mathbf{W}_{2,t}^{\prime }\widetilde{\mathbf{B}}_{2}\mathbf{e%
}_{k-1,N-1}\right) \right\} \\
&=&\frac{T^{4}}{N^{2}T^{2}}\left( 1+\frac{1}{\left\Vert \gamma \right\Vert
_{2}^{2}}\right) ^{2}\left\{ \left( \mathbf{e}_{j-1,N-1}^{\prime }\widetilde{%
\mathbf{B}}_{2}^{\prime }\dsum\limits_{s=1}^{T}\frac{\mathbf{W}_{2,s}\mathbf{%
W}_{2,s}^{\prime }}{T}\widetilde{\mathbf{B}}_{2}\mathbf{e}_{j-1,N-1}\right)
\right. \\
&&\left. \times \left( \mathbf{e}_{k-1,N-1}^{\prime }\widetilde{\mathbf{B}}%
_{2}^{\prime }\dsum\limits_{t=1}^{T}\frac{\mathbf{W}_{2,t}\mathbf{W}%
_{2,t}^{\prime }}{T}\widetilde{\mathbf{B}}_{2}\mathbf{e}_{k-1,N-1}\right)
\right\} \\
&&-\frac{T^{4}}{N^{2}T^{4}}\left( 1+\frac{1}{\left\Vert \gamma \right\Vert
_{2}^{2}}\right) ^{2}\dsum\limits_{t=1}^{T}\left\{ \left( \mathbf{e}%
_{j-1,N-1}^{\prime }\widetilde{\mathbf{B}}_{2}^{\prime }\mathbf{W}_{2,t}%
\mathbf{W}_{2,t}^{\prime }\widetilde{\mathbf{B}}_{2}\mathbf{e}%
_{j-1,N-1}\right) \right. \\
&&\text{ \ \ \ \ \ \ \ \ \ \ \ \ \ \ \ \ \ \ \ \ \ \ \ \ \ \ \ \ \ \ \ \ \ \
\ \ \ \ \ \ \ \ \ }\left. \text{\ }\times \left( \mathbf{e}%
_{k-1,N-1}^{\prime }\widetilde{\mathbf{B}}_{2}^{\prime }\mathbf{W}_{2,t}%
\mathbf{W}_{2,t}^{\prime }\widetilde{\mathbf{B}}_{2}\mathbf{e}%
_{k-1,N-1}\right) \right\} \\
&=&\left( 1+\frac{1}{\left\Vert \gamma \right\Vert _{2}^{2}}\right)
^{2}\left( \frac{T\widetilde{\lambda }_{\left( j\right) }}{N}\right) \left( 
\frac{T\widetilde{\lambda }_{\left( k\right) }}{N}\right) \\
&&-\frac{T^{4}}{N^{2}T^{4}}\left( 1+\frac{1}{\left\Vert \gamma \right\Vert
_{2}^{2}}\right) ^{2}\dsum\limits_{t=1}^{T}\left\{ \left( \mathbf{e}%
_{j-1,N-1}^{\prime }\widetilde{\mathbf{B}}_{2}^{\prime }\mathbf{W}_{2,t}%
\mathbf{W}_{2,t}^{\prime }\widetilde{\mathbf{B}}_{2}\mathbf{e}%
_{j-1,N-1}\right) \right. \\
&&\text{ \ \ \ \ \ \ \ \ \ \ \ \ \ \ \ \ \ \ \ \ \ \ \ \ \ \ \ \ \ \ \ \ \ \
\ \ \ \ \ \ \ \ \ }\left. \times \left( \mathbf{e}_{k-1,N-1}^{\prime }%
\widetilde{\mathbf{B}}_{2}^{\prime }\mathbf{W}_{2,t}\mathbf{W}_{2,t}^{\prime
}\widetilde{\mathbf{B}}_{2}\mathbf{e}_{k-1,N-1}\right) \right\} \text{,}
\end{eqnarray*}%
\begin{eqnarray*}
&&\frac{T^{4}}{N^{2}T^{4}}\left\{ \dsum\limits_{t=1}^{T}E\left[ \frac{%
W_{1,t}^{2}}{\left\Vert \gamma \right\Vert _{2}^{2}}\right] \left( \mathbf{e}%
_{j-1,N-1}^{\prime }\widetilde{\mathbf{B}}_{2}^{\prime }\mathbf{W}_{2,t}%
\mathbf{W}_{2,t}^{\prime }\widetilde{\mathbf{B}}_{2}\mathbf{e}%
_{k-1,N-1}\right) \right. \\
&&\text{ \ \ \ \ \ \ \ \ \ \ }\left. \text{\ }\times \dsum\limits_{s\neq t}E%
\left[ \frac{W_{1,s}^{2}}{\left\Vert \gamma \right\Vert _{2}^{2}}\right]
\left( \mathbf{e}_{j-1,N-1}^{\prime }\widetilde{\mathbf{B}}_{2}^{\prime }%
\mathbf{W}_{2,s}\mathbf{W}_{2,s}^{\prime }\widetilde{\mathbf{B}}_{2}\mathbf{e%
}_{k-1,N-1}\right) \right\} \\
&=&\frac{T^{4}}{N^{2}T^{2}}\left( 1+\frac{1}{\left\Vert \gamma \right\Vert
_{2}^{2}}\right) ^{2}\left( \mathbf{e}_{j-1,N-1}^{\prime }\widetilde{\mathbf{%
B}}_{2}^{\prime }\dsum\limits_{t=1}^{T}\frac{\mathbf{W}_{2,t}\mathbf{W}%
_{2,t}^{\prime }}{T}\widetilde{\mathbf{B}}_{2}\mathbf{e}_{k-1,N-1}\right) \\
&&\text{ \ }\times \left( \mathbf{e}_{j-1,N-1}^{\prime }\widetilde{\mathbf{B}%
}_{2}^{\prime }\dsum\limits_{s=1}^{T}\frac{\mathbf{W}_{2,s}\mathbf{W}%
_{2,s}^{\prime }}{T}\widetilde{\mathbf{B}}_{2}\mathbf{e}_{k-1,N-1}\right) \\
&&-\frac{T^{4}}{N^{2}T^{4}}\left( 1+\frac{1}{\left\Vert \gamma \right\Vert
_{2}^{2}}\right) ^{2}\dsum\limits_{t=1}^{T}\left\{ \left( \mathbf{e}%
_{j-1,N-1}^{\prime }\widetilde{\mathbf{B}}_{2}^{\prime }\mathbf{W}_{2,t}%
\mathbf{W}_{2,t}^{\prime }\widetilde{\mathbf{B}}_{2}\mathbf{e}%
_{j-1,N-1}\right) \right. \\
&&\text{ \ \ \ \ \ \ \ \ \ \ \ \ \ \ \ \ \ \ \ \ \ \ \ \ \ \ \ \ \ \ \ \ \ \
\ \ \ \ \ \ \ \ \ }\left. \times \left( \mathbf{e}_{k-1,N-1}^{\prime }%
\widetilde{\mathbf{B}}_{2}^{\prime }\mathbf{W}_{2,t}\mathbf{W}_{2,t}^{\prime
}\widetilde{\mathbf{B}}_{2}\mathbf{e}_{k-1,N-1}\right) \right\} \\
&=&-\frac{T^{4}}{N^{2}T^{4}}\left( 1+\frac{1}{\left\Vert \gamma \right\Vert
_{2}^{2}}\right) ^{2}\dsum\limits_{t=1}^{T}\left\{ \left( \mathbf{e}%
_{j-1,N-1}^{\prime }\widetilde{\mathbf{B}}_{2}^{\prime }\mathbf{W}_{2,t}%
\mathbf{W}_{2,t}^{\prime }\widetilde{\mathbf{B}}_{2}\mathbf{e}%
_{j-1,N-1}\right) \right. \\
&&\text{ \ \ \ \ \ \ \ \ \ \ \ \ \ \ \ \ \ \ \ \ \ \ \ \ \ \ \ \ \ \ \ \ \ \
\ \ \ \ \ \ \ \ }\left. \times \left( \mathbf{e}_{k-1,N-1}^{\prime }%
\widetilde{\mathbf{B}}_{2}^{\prime }\mathbf{W}_{2,t}\mathbf{W}_{2,t}^{\prime
}\widetilde{\mathbf{B}}_{2}\mathbf{e}_{k-1,N-1}\right) \right\} \text{,}
\end{eqnarray*}%
and%
\begin{eqnarray*}
&&\frac{T^{4}}{N^{2}T^{4}}\left\{ \dsum\limits_{r=1}^{T}E\left[ \frac{%
W_{1,r}^{2}}{\left\Vert \gamma \right\Vert _{2}^{2}}\right] \left( \mathbf{e}%
_{j-1,N-1}^{\prime }\widetilde{\mathbf{B}}_{2}^{\prime }\mathbf{W}_{2,r}%
\mathbf{W}_{2,r}^{\prime }\widetilde{\mathbf{B}}_{2}\mathbf{e}%
_{k-1,N-1}\right) \right. \\
&&\text{ \ \ \ \ \ \ \ \ \ \ \ \ \ }\left. \text{\ }\times
\dsum\limits_{t\neq r}E\left[ \frac{W_{1,s}^{2}}{\left\Vert \gamma
\right\Vert _{2}^{2}}\right] \left( \mathbf{e}_{j-1,N-1}^{\prime }\widetilde{%
\mathbf{B}}_{2}^{\prime }\mathbf{W}_{2,t}\mathbf{W}_{2,t}^{\prime }%
\widetilde{\mathbf{B}}_{2}\mathbf{e}_{k-1,N-1}\right) \right\} \\
&=&\frac{T^{4}}{N^{2}T^{2}}\left( 1+\frac{1}{\left\Vert \gamma \right\Vert
_{2}^{2}}\right) ^{2}\left( \mathbf{e}_{j-1,N-1}^{\prime }\widetilde{\mathbf{%
B}}_{2}^{\prime }\dsum\limits_{r=1}^{T}\frac{\mathbf{W}_{2,r}\mathbf{W}%
_{2,r}^{\prime }}{T}\widetilde{\mathbf{B}}_{2}\mathbf{e}_{k-1,N-1}\right) \\
&&\times \left( \mathbf{e}_{j-1,N-1}^{\prime }\widetilde{\mathbf{V}}%
_{2}^{\prime }\dsum\limits_{t=1}^{T}\frac{\mathbf{W}_{2,t}\mathbf{W}%
_{2,t}^{\prime }}{T}\widetilde{\mathbf{V}}_{2}\mathbf{e}_{k-1,N-1}\right) \\
&&-\frac{T^{4}}{N^{2}T^{4}}\left( 1+\frac{1}{\left\Vert \gamma \right\Vert
_{2}^{2}}\right) ^{2}\dsum\limits_{t=1}^{T}\left\{ \left( \mathbf{e}%
_{j-1,N-1}^{\prime }\widetilde{\mathbf{B}}_{2}^{\prime }\mathbf{W}_{2,t}%
\mathbf{W}_{2,t}^{\prime }\widetilde{\mathbf{B}}_{2}\mathbf{e}%
_{j-1,N-1}\right) \right. \\
&&\text{ \ \ \ \ \ \ \ \ \ \ \ \ \ \ \ \ \ \ \ \ \ \ \ \ \ \ \ \ \ \ \ \ \ \
\ \ \ \ \ \ \ \ \ }\left. \times \left( \mathbf{e}_{k-1,N-1}^{\prime }%
\widetilde{\mathbf{B}}_{2}^{\prime }\mathbf{W}_{2,t}\mathbf{W}_{2,t}^{\prime
}\widetilde{\mathbf{B}}_{2}\mathbf{e}_{k-1,N-1}\right) \right\} \\
&=&-\frac{T^{4}}{N^{2}T^{4}}\left( 1+\frac{1}{\left\Vert \gamma \right\Vert
_{2}^{2}}\right) ^{2}\dsum\limits_{t=1}^{T}\left\{ \left( \mathbf{e}%
_{j-1,N-1}^{\prime }\widetilde{\mathbf{B}}_{2}^{\prime }\mathbf{W}_{2,t}%
\mathbf{W}_{2,t}^{\prime }\widetilde{\mathbf{B}}_{2}\mathbf{e}%
_{j-1,N-1}\right) \right. \\
&&\text{ \ \ \ \ \ \ \ \ \ \ \ \ \ \ \ \ \ \ \ \ \ \ \ \ \ \ \ \ \ \ \ \ \ \
\ \ \ \ \ \ \ \ }\left. \times \left( \mathbf{e}_{k-1,N-1}^{\prime }%
\widetilde{\mathbf{B}}_{2}^{\prime }\mathbf{W}_{2,t}\mathbf{W}_{2,t}^{\prime
}\widetilde{\mathbf{B}}_{2}\mathbf{e}_{k-1,N-1}\right) \right\}
\end{eqnarray*}%
It follows from these calculations that, for $j\neq k$%
\begin{eqnarray*}
&&E\left[ \frac{T^{4}\upsilon _{j}^{2}\upsilon _{k}^{2}}{N^{2}\left\Vert
\gamma \right\Vert _{2}^{4}}|\mathbf{W}_{2}\right] \\
&=&\frac{3T^{4}}{N^{2}T^{4}}\left( 1+\frac{1}{\left\Vert \gamma \right\Vert
_{2}^{2}}\right) ^{2}\dsum\limits_{t=1}^{T}\left\{ \left( \mathbf{e}%
_{j-1,N-1}^{\prime }\widetilde{\mathbf{B}}_{2}^{\prime }\mathbf{W}_{2,t}%
\mathbf{W}_{2,t}^{\prime }\widetilde{\mathbf{B}}_{2}\mathbf{e}%
_{j-1,N-1}\right) \right. \\
&&\text{ \ \ \ \ \ \ \ \ \ \ \ \ \ \ \ \ \ \ \ \ \ \ \ \ \ \ \ \ \ \ \ \ \ \
\ \ \ \ }\left. \times \left( \mathbf{e}_{k-1,N-1}^{\prime }\widetilde{%
\mathbf{B}}_{2}^{\prime }\mathbf{W}_{2,t}\mathbf{W}_{2,t}^{\prime }%
\widetilde{\mathbf{B}}_{2}\mathbf{e}_{k-1,N-1}\right) \right\} \\
&&+\left( 1+\frac{1}{\left\Vert \gamma \right\Vert _{2}^{2}}\right)
^{2}\left( \frac{T\widetilde{\lambda }_{\left( j\right) }}{N}\right) \left( 
\frac{T\widetilde{\lambda }_{\left( k\right) }}{N}\right) \\
&&-\frac{T^{4}}{N^{2}T^{4}}\left( 1+\frac{1}{\left\Vert \gamma \right\Vert
_{2}^{2}}\right) ^{2}\dsum\limits_{t=1}^{T}\left\{ \left( \mathbf{e}%
_{j-1,N-1}^{\prime }\widetilde{\mathbf{B}}_{2}^{\prime }\mathbf{W}_{2,t}%
\mathbf{W}_{2,t}^{\prime }\widetilde{\mathbf{B}}_{2}\mathbf{e}%
_{j-1,N-1}\right) \right. \\
&&\text{ \ \ \ \ \ \ \ \ \ \ \ \ \ \ \ \ \ \ \ \ \ \ \ \ \ \ \ \ \ \ \ \ \ \
\ \ \ \ \ \ \ \ \ }\left. \times \left( \mathbf{e}_{k-1,N-1}^{\prime }%
\widetilde{\mathbf{B}}_{2}^{\prime }\mathbf{W}_{2,t}\mathbf{W}_{2,t}^{\prime
}\widetilde{\mathbf{B}}_{2}\mathbf{e}_{k-1,N-1}\right) \right\} \\
&&-\frac{T^{4}}{N^{2}T^{4}}\left( 1+\frac{1}{\left\Vert \gamma \right\Vert
_{2}^{2}}\right) ^{2}\dsum\limits_{t=1}^{T}\left\{ \left( \mathbf{e}%
_{j-1,N-1}^{\prime }\widetilde{\mathbf{B}}_{2}^{\prime }\mathbf{W}_{2,t}%
\mathbf{W}_{2,t}^{\prime }\widetilde{\mathbf{B}}_{2}\mathbf{e}%
_{j-1,N-1}\right) \right. \\
&&\text{ \ \ \ \ \ \ \ \ \ \ \ \ \ \ \ \ \ \ \ \ \ \ \ \ \ \ \ \ \ \ \ \ \ \
\ \ \ \ \ \ \ \ \ }\left. \times \left( \mathbf{e}_{k-1,N-1}^{\prime }%
\widetilde{\mathbf{B}}_{2}^{\prime }\mathbf{W}_{2,t}\mathbf{W}_{2,t}^{\prime
}\widetilde{\mathbf{B}}_{2}\mathbf{e}_{k-1,N-1}\right) \right\} \\
&&-\frac{T^{4}}{N^{2}T^{4}}\left( 1+\frac{1}{\left\Vert \gamma \right\Vert
_{2}^{2}}\right) ^{2}\dsum\limits_{t=1}^{T}\left\{ \left( \mathbf{e}%
_{j-1,N-1}^{\prime }\widetilde{\mathbf{B}}_{2}^{\prime }\mathbf{W}_{2,t}%
\mathbf{W}_{2,t}^{\prime }\widetilde{\mathbf{B}}_{2}\mathbf{e}%
_{j-1,N-1}\right) \right. \\
&&\text{ \ \ \ \ \ \ \ \ \ \ \ \ \ \ \ \ \ \ \ \ \ \ \ \ \ \ \ \ \ \ \ \ \ \
\ \ \ \ \ \ \ \ \ }\left. \times \left( \mathbf{e}_{k-1,N-1}^{\prime }%
\widetilde{\mathbf{B}}_{2}^{\prime }\mathbf{W}_{2,t}\mathbf{W}_{2,t}^{\prime
}\widetilde{\mathbf{B}}_{2}\mathbf{e}_{k-1,N-1}\right) \right\} \\
&=&\left( 1+\frac{1}{\left\Vert \gamma \right\Vert _{2}^{2}}\right)
^{2}\left( \frac{T\widetilde{\lambda }_{\left( j\right) }}{N}\right) \left( 
\frac{T\widetilde{\lambda }_{\left( k\right) }}{N}\right)
\end{eqnarray*}%
so that%
\begin{eqnarray*}
&&\frac{1}{T^{2}}\dsum\limits_{j\neq k}E\left\{ \left[ \frac{T^{2}\upsilon
_{j}^{2}}{N\left\Vert \gamma \right\Vert _{2}^{2}}-\left( 1+\frac{1}{%
\left\Vert \gamma \right\Vert _{2}^{2}}\right) \frac{T\widetilde{\lambda }%
_{\left( j\right) }}{N}\right] \left[ \frac{T^{2}\upsilon _{k}^{2}}{%
N\left\Vert \gamma \right\Vert _{2}^{2}}-\left( 1+\frac{1}{\left\Vert \gamma
\right\Vert _{2}^{2}}\right) \frac{T\widetilde{\lambda }_{\left( k\right) }}{%
N}\right] |\mathbf{W}_{2}\right\} \\
&=&\frac{1}{T^{2}}\dsum\limits_{j\neq k}E\left[ \frac{T^{4}\upsilon
_{j}^{2}\upsilon _{k}^{2}}{N^{2}\left\Vert \gamma \right\Vert _{2}^{4}}|%
\mathbf{W}_{2}\right] -\frac{1}{T^{2}}\left( 1+\frac{1}{\left\Vert \gamma
\right\Vert _{2}^{2}}\right) ^{2}\dsum\limits_{j\neq k}\left( \frac{T%
\widetilde{\lambda }_{\left( j\right) }}{N}\right) \left( \frac{T\widetilde{%
\lambda }_{\left( k\right) }}{N}\right) \\
&=&\frac{1}{T^{2}}\left( 1+\frac{1}{\left\Vert \gamma \right\Vert _{2}^{2}}%
\right) ^{2}\dsum\limits_{j\neq k}\left( \frac{T\widetilde{\lambda }_{\left(
j\right) }}{N}\right) \left( \frac{T\widetilde{\lambda }_{\left( k\right) }}{%
N}\right) -\frac{1}{T^{2}}\left( 1+\frac{1}{\left\Vert \gamma \right\Vert
_{2}^{2}}\right) ^{2}\dsum\limits_{j\neq k}\left( \frac{T\widetilde{\lambda }%
_{\left( j\right) }}{N}\right) \left( \frac{T\widetilde{\lambda }_{\left(
k\right) }}{N}\right) \\
&=&0
\end{eqnarray*}%
Hence,%
\begin{eqnarray*}
&&E\left\{ \left( \frac{1}{T}\dsum\limits_{j=2}^{N}\frac{T^{2}\upsilon
_{j}^{2}}{N\left\Vert \gamma \right\Vert _{2}^{2}}-\left( 1+\frac{1}{%
\left\Vert \gamma \right\Vert _{2}^{2}}\right) \frac{1}{T}%
\dsum\limits_{j=2}^{N}\frac{T\widetilde{\lambda }_{\left( j\right) }}{N}%
\right) ^{2}|\mathbf{W}_{2}\right\} \\
&=&\frac{1}{T^{2}}\dsum\limits_{j=2}^{N}E\left\{ \left[ \frac{T^{2}\upsilon
_{j}^{2}}{N\left\Vert \gamma \right\Vert _{2}^{2}}-\left( 1+\frac{1}{%
\left\Vert \gamma \right\Vert _{2}^{2}}\right) \frac{T\widetilde{\lambda }%
_{\left( j\right) }}{N}\right] ^{2}|\mathbf{W}_{2}\right\} \\
&&+\frac{1}{T^{2}}\dsum\limits_{j\neq k}E\left\{ \left[ \frac{T^{2}\upsilon
_{j}^{2}}{N\left\Vert \gamma \right\Vert _{2}^{2}}-\left( 1+\frac{1}{%
\left\Vert \gamma \right\Vert _{2}^{2}}\right) \frac{T\widetilde{\lambda }%
_{\left( j\right) }}{N}\right] \left[ \frac{T^{2}\upsilon _{k}^{2}}{%
N\left\Vert \gamma \right\Vert _{2}^{2}}-\left( 1+\frac{1}{\left\Vert \gamma
\right\Vert _{2}^{2}}\right) \frac{T\widetilde{\lambda }_{\left( k\right) }}{%
N}\right] |\mathbf{W}_{2}\right\} \\
&=&\frac{1}{T^{2}}\dsum\limits_{j=2}^{N}E\left\{ \left[ \frac{T^{2}\upsilon
_{j}^{2}}{N\left\Vert \gamma \right\Vert _{2}^{2}}-\left( 1+\frac{1}{%
\left\Vert \gamma \right\Vert _{2}^{2}}\right) \frac{T\widetilde{\lambda }%
_{\left( j\right) }}{N}\right] ^{2}|\mathbf{W}_{2}\right\} \\
&=&\frac{1}{T^{2}}\dsum\limits_{j=2}^{N}E\left[ \frac{T^{4}\upsilon _{j}^{4}%
}{N^{2}\left\Vert \gamma \right\Vert _{2}^{4}}|\mathbf{W}_{2}\right] -\frac{1%
}{T^{2}}\left( 1+\frac{1}{\left\Vert \gamma \right\Vert _{2}^{2}}\right)
^{2}\dsum\limits_{j=2}^{N}\left( \frac{T\widetilde{\lambda }_{\left(
j\right) }}{N}\right) ^{2} \\
&=&\frac{3}{T^{2}}\left( 1+\frac{1}{\left\Vert \gamma \right\Vert _{2}^{2}}%
\right) ^{2}\dsum\limits_{j=2}^{N}\left( \frac{T}{N}\widetilde{\lambda }%
_{\left( j\right) }\right) ^{2}-\frac{1}{T^{2}}\left( 1+\frac{1}{\left\Vert
\gamma \right\Vert _{2}^{2}}\right) ^{2}\dsum\limits_{j=2}^{N}\left( \frac{T%
\widetilde{\lambda }_{\left( j\right) }}{N}\right) ^{2} \\
&=&\frac{2}{T^{2}}\left( 1+\frac{1}{\left\Vert \gamma \right\Vert _{2}^{2}}%
\right) ^{2}\dsum\limits_{j=2}^{N}\left( \frac{T}{N}\widetilde{\lambda }%
_{\left( j\right) }\right) ^{2} \\
&=&\frac{2}{T^{2}}\left( 1+\frac{1}{\left\Vert \gamma \right\Vert _{2}^{2}}%
\right) ^{2}\dsum\limits_{j=2}^{T{\LARGE +}1}\left( \frac{T}{N}\widetilde{%
\lambda }_{\left( j\right) }\right) ^{2}\text{ \ }\left( \text{since }%
\widetilde{\lambda }_{\left( j\right) }=0\text{ for }j>T+1\right) \\
&\leq &\frac{2}{T^{2}}\left( 1+\frac{1}{\left\Vert \gamma \right\Vert
_{2}^{2}}\right) ^{2}\left( \frac{N-1}{N}\right) ^{2}T\left( \frac{T}{N-1}%
\max_{2\leq j\leq T+1}\widetilde{\lambda }_{\left( j\right) }\right) ^{2}%
\text{ \ } \\
&&\left( \text{since }\widetilde{\lambda }_{\left( j\right) }\geq 0\text{
for }j=2,..,T+1\right) \\
&=&\frac{2}{T}\left( 1+\frac{1}{\left\Vert \gamma \right\Vert _{2}^{2}}%
\right) ^{2}\left( \frac{N-1}{N}\right) ^{2}\left( \frac{T}{N-1}\max_{2\leq
j\leq T+1}\widetilde{\lambda }_{\left( j\right) }\right) ^{2} \\
&=&O_{a.s.}\left( \frac{1}{T}\right) \text{ \ }\left( \text{by Lemma B-7 and
by the fact that }\left\Vert \gamma \right\Vert _{2}^{2}\rightarrow \infty 
\text{ under Assumption 2-2}\right) \\
&=&o_{a.s.}\left( 1\right)
\end{eqnarray*}%
Applying the law of iterated expectations as well as part (i) of Theorem
16.1 of Billingsley (1995), we see that there exists a constant $\overline{C}%
<\infty $ such that for all $n$ sufficiently large%
\begin{eqnarray*}
&&E\left\{ T\left( \frac{1}{T}\dsum\limits_{j=2}^{N}\frac{T^{2}\upsilon
_{j}^{2}}{N\left\Vert \gamma \right\Vert _{2}^{2}}-\left( 1+\frac{1}{%
\left\Vert \gamma \right\Vert _{2}^{2}}\right) \frac{1}{T}%
\dsum\limits_{j=2}^{N}\frac{T\widetilde{\lambda }_{\left( j\right) }}{N}%
\right) ^{2}\right\} \\
&=&E\left\{ \left( \frac{1}{\sqrt{T}}\dsum\limits_{j=2}^{N}\frac{%
T^{2}\upsilon _{j}^{2}}{N\left\Vert \gamma \right\Vert _{2}^{2}}-\left( 1+%
\frac{1}{\left\Vert \gamma \right\Vert _{2}^{2}}\right) \frac{1}{\sqrt{T}}%
\dsum\limits_{j=2}^{N}\frac{T\widetilde{\lambda }_{\left( j\right) }}{N}%
\right) ^{2}\right\} \\
&=&E_{\mathbf{W}_{2}}\left[ E\left\{ \left( \frac{1}{\sqrt{T}}%
\dsum\limits_{j=2}^{N}\frac{T^{2}\upsilon _{j}^{2}}{N\left\Vert \gamma
\right\Vert _{2}^{2}}-\left( 1+\frac{1}{\left\Vert \gamma \right\Vert
_{2}^{2}}\right) \frac{1}{\sqrt{T}}\dsum\limits_{j=2}^{N}\frac{T\widetilde{%
\lambda }_{\left( j\right) }}{N}\right) ^{2}|\mathbf{W}_{2}\right\} \right]
\\
&\leq &\overline{C}\text{.}
\end{eqnarray*}%
Now, for any $\epsilon >0$, set $C_{\epsilon }=\sqrt{\overline{C}/\epsilon }$%
, and the Markov's inequality then implies that, for all $n$ sufficiently
large,%
\begin{eqnarray*}
&&\Pr \left\{ \sqrt{T}\left\vert \frac{1}{T}\dsum\limits_{j=2}^{N}\frac{%
T^{2}\upsilon _{j}^{2}}{N\left\Vert \gamma \right\Vert _{2}^{2}}-\left( 1+%
\frac{1}{\left\Vert \gamma \right\Vert _{2}^{2}}\right) \frac{1}{T}%
\dsum\limits_{j=2}^{N}\frac{T\widetilde{\lambda }_{\left( j\right) }}{N}%
\right\vert \geq C_{\epsilon }\right\} \\
&=&\Pr \left\{ \left( \frac{\sqrt{T}}{T}\dsum\limits_{j=2}^{N}\frac{%
T^{2}\upsilon _{j}^{2}}{N\left\Vert \gamma \right\Vert _{2}^{2}}-\left( 1+%
\frac{1}{\left\Vert \gamma \right\Vert _{2}^{2}}\right) \frac{\sqrt{T}}{T}%
\dsum\limits_{j=2}^{N}\frac{T\widetilde{\lambda }_{\left( j\right) }}{N}%
\right) ^{2}\geq C_{\epsilon }^{2}\right\} \\
&\leq &\frac{1}{C_{\epsilon }^{2}}E\left\{ \left( \frac{1}{\sqrt{T}}%
\dsum\limits_{j=2}^{N}\frac{T^{2}\upsilon _{j}^{2}}{N\left\Vert \gamma
\right\Vert _{2}^{2}}-\left( 1+\frac{1}{\left\Vert \gamma \right\Vert
_{2}^{2}}\right) \frac{1}{\sqrt{T}}\dsum\limits_{j=2}^{N}\frac{T\widetilde{%
\lambda }_{\left( j\right) }}{N}\right) ^{2}\right\} \\
&=&\frac{\epsilon }{\overline{C}}E\left\{ \left( \frac{1}{\sqrt{T}}%
\dsum\limits_{j=2}^{N}\frac{T^{2}\upsilon _{j}^{2}}{N\left\Vert \gamma
\right\Vert _{2}^{2}}-\left( 1+\frac{1}{\left\Vert \gamma \right\Vert
_{2}^{2}}\right) \frac{1}{\sqrt{T}}\dsum\limits_{j=2}^{N}\frac{T\widetilde{%
\lambda }_{\left( j\right) }}{N}\right) ^{2}\right\} \\
&\leq &\epsilon
\end{eqnarray*}%
which shows that%
\begin{eqnarray}
&&\frac{1}{T}\dsum\limits_{j=2}^{N}\frac{T^{2}\upsilon _{j}^{2}}{N\left\Vert
\gamma \right\Vert _{2}^{2}}-\left( 1+\frac{1}{\left\Vert \gamma \right\Vert
_{2}^{2}}\right) \frac{1}{T}\dsum\limits_{j=2}^{N}\frac{T\widetilde{\lambda }%
_{\left( j\right) }}{N}  \notag \\
&=&\frac{1}{T}\dsum\limits_{j=2}^{N}\frac{T^{2}\upsilon _{j}^{2}}{%
N\left\Vert \gamma \right\Vert _{2}^{2}}-\left( 1+\frac{1}{\left\Vert \gamma
\right\Vert _{2}^{2}}\right) \frac{1}{T}\dsum\limits_{j=2}^{T{\LARGE +}1}%
\frac{T\widetilde{\lambda }_{\left( j\right) }}{N}\text{ \ }\left( \text{%
since }\widetilde{\lambda }_{\left( j\right) }=0\text{ for }j>T+1\right) 
\notag \\
&=&O_{p}\left( \frac{1}{\sqrt{T}}\right) =o_{p}\left( 1\right)
\label{repre of T^2v^2/Ngamma^2}
\end{eqnarray}%
In addition, note that%
\begin{eqnarray*}
\left\vert \left( 1+\frac{1}{\left\Vert \gamma \right\Vert _{2}^{2}}\right) 
\frac{1}{T}\dsum\limits_{j=2}^{T{\LARGE +}1}\left( \frac{T}{N}\widetilde{%
\lambda }_{\left( j\right) }-1\right) \right\vert &\leq &\left( 1+\frac{1}{%
\left\Vert \gamma \right\Vert _{2}^{2}}\right) \frac{1}{T}%
\dsum\limits_{j=2}^{T{\LARGE +}1}\left\vert \frac{T}{N}\widetilde{\lambda }%
_{\left( j\right) }-1\right\vert \\
&\leq &\left( 1+\frac{1}{\left\Vert \gamma \right\Vert _{2}^{2}}\right)
\max_{2\leq j\leq T+1}\left\vert \frac{T}{N}\widetilde{\lambda }_{\left(
j\right) }-1\right\vert \overset{a.s.}{\rightarrow }0 \\
&&\left( \text{by Lemma B-7}\right)
\end{eqnarray*}%
Making use of this result and the Slutsky's theorem, we obtain%
\begin{eqnarray}
&&\left( 1+\frac{1}{\left\Vert \gamma \right\Vert _{2}^{2}}\right) \frac{1}{T%
}\dsum\limits_{j=2}^{T{\LARGE +}1}\frac{T\widetilde{\lambda }_{\left(
j\right) }}{N}  \notag \\
&=&\left( 1+\frac{1}{\left\Vert \gamma \right\Vert _{2}^{2}}\right) \frac{1}{%
T}\dsum\limits_{j=2}^{T{\LARGE +}1}\left[ \frac{T\widetilde{\lambda }%
_{\left( j\right) }}{N}-1+1\right]  \notag \\
&=&\left( 1+\frac{1}{\left\Vert \gamma \right\Vert _{2}^{2}}\right) +\left(
1+\frac{1}{\left\Vert \gamma \right\Vert _{2}^{2}}\right) \left[ \frac{1}{T}%
\dsum\limits_{j=2}^{T{\LARGE +}1}\frac{T\widetilde{\lambda }_{\left(
j\right) }}{N}-1\right]  \notag \\
&=&\left( 1+\frac{1}{\left\Vert \gamma \right\Vert _{2}^{2}}\right) +\left(
1+\frac{1}{\left\Vert \gamma \right\Vert _{2}^{2}}\right) \frac{1}{T}%
\dsum\limits_{j=2}^{T{\LARGE +}1}\left( \frac{T}{N}\widetilde{\lambda }%
_{\left( j\right) }-1\right) \overset{a.s.}{\rightarrow }1
\label{avg Tlambda/N} \\
&&\left( \text{since }\left\Vert \gamma \right\Vert _{2}\rightarrow \infty
\right)  \notag
\end{eqnarray}%
from which we further deduce, in light of expression (\ref{repre of
T^2v^2/Ngamma^2}), that%
\begin{equation}
\frac{1}{T}\dsum\limits_{j=2}^{N}\frac{T^{2}\upsilon _{j}^{2}}{N\left\Vert
\gamma \right\Vert _{2}^{2}}=\left( 1+\frac{1}{\left\Vert \gamma \right\Vert
_{2}^{2}}\right) \frac{1}{T}\dsum\limits_{j=2}^{T{\LARGE +}1}\frac{T%
\widetilde{\lambda }_{\left( j\right) }}{N}+O_{p}\left( \frac{1}{\sqrt{T}}%
\right) \overset{p}{\rightarrow }1\text{ as }N,T\rightarrow \infty \text{. }
\label{avg v^2/gamma^2}
\end{equation}

Putting together the results given in expressions (\ref{tau^2}), (\ref%
{gamma^4/lambda^2}), (\ref{1-maxlambdaratio}), and (\ref{avg v^2/gamma^2});
we see that as $N,T\rightarrow \infty $ such that $T/N\rightarrow 0$

\begin{eqnarray}
&&\tau ^{2}  \notag \\
&\leq &\frac{N}{T\left\Vert \gamma \right\Vert _{2}^{2\left( 1+2\kappa
\right) }}\frac{1}{\widehat{\lambda }_{\left( 1\right) }^{2}/\left\Vert
\gamma \right\Vert _{2}^{4\left( 1+\kappa \right) }}\left( 1-\max_{2\leq
j\leq T+1}\frac{\widetilde{\lambda }_{\left( j\right) }}{\widehat{\lambda }%
_{\left( 1\right) }}\right) ^{-2}\frac{1}{T}\dsum\limits_{j=2}^{N}\frac{%
T^{2}\upsilon _{j}^{2}}{N\left\Vert \gamma \right\Vert _{2}^{2}}  \notag \\
&=&\frac{1}{\left\Vert \gamma \right\Vert _{2}^{2\kappa }}\frac{N}{%
T\left\Vert \gamma \right\Vert _{2}^{2\left( 1+\kappa \right) }}\frac{1}{%
c^{2}}c^{2}\left\Vert \gamma \right\Vert _{2}^{4\kappa }\left( 1+\frac{1}{%
\left\Vert \gamma \right\Vert _{2}^{2}}\right) \frac{1}{T}%
\dsum\limits_{j=2}^{T{\LARGE +}1}\frac{T\widetilde{\lambda }_{\left(
j\right) }}{N}\left[ 1+o_{p}\left( 1\right) \right]  \notag \\
&=&\frac{N}{T\left\Vert \gamma \right\Vert _{2}^{2}}\left( 1+\frac{1}{%
\left\Vert \gamma \right\Vert _{2}^{2}}\right) \frac{1}{T}%
\dsum\limits_{j=2}^{T{\LARGE +}1}\frac{T\widetilde{\lambda }_{\left(
j\right) }}{N}\left[ 1+o_{p}\left( 1\right) \right]  \notag \\
&=&O_{p}\left( \frac{N}{T\left\Vert \gamma \right\Vert _{2}^{2}}\right)
\label{order of magnitude tau^2} \\
&&\left( \text{since }\left( 1+\frac{1}{\left\Vert \gamma \right\Vert
_{2}^{2}}\right) \frac{1}{T}\dsum\limits_{j=2}^{T{\LARGE +}1}\frac{T%
\widetilde{\lambda }_{\left( j\right) }}{N}\overset{p}{\rightarrow }1\text{
by expression (\ref{avg Tlambda/N})}\right)  \notag
\end{eqnarray}%
Moreover, since Assumption 2-2 implies that $N/\left( T\left\Vert \gamma
\right\Vert _{2}^{2}\right) \rightarrow \infty $ as $N,T\rightarrow \infty $
such that $T/N\rightarrow 0$, we further deduce that 
\begin{equation}
\tau ^{2}\rightarrow \infty \text{ w.p.a.1.}  \label{asy behavior tau^2}
\end{equation}%
Finally, we note that expression (\ref{asy behavior tau^2}) further implies
that%
\begin{equation}
\frac{\left\langle \widetilde{\mathbf{v}}_{\left( 1\right) }\mathbf{,e}%
_{1,N}\right\rangle ^{2}}{\left\Vert \widetilde{\mathbf{v}}_{\left( 1\right)
}\right\Vert ^{2}}=\frac{1}{1+\tau ^{2}}\overset{p}{\rightarrow }0
\label{dot prod vtilde and e1}
\end{equation}%
as $N,T\rightarrow \infty $ such that $T/N\rightarrow 0$.

\medskip

\noindent \textbf{Step 5:}

In this step, we will show that%
\begin{equation*}
\frac{1}{\sqrt{N}}\left\langle \frac{\widetilde{\mathbf{v}}_{\left( 1\right)
}}{\left\Vert \widetilde{\mathbf{v}}_{\left( 1\right) }\right\Vert _{2}}%
\mathbf{,}\widetilde{\eta }_{t}\right\rangle \overset{p}{\rightarrow }0\text{%
.}
\end{equation*}%
To proceed, write

\begin{eqnarray*}
\frac{1}{\sqrt{N}}\left\langle \frac{\widetilde{\mathbf{v}}_{\left( 1\right)
}}{\left\Vert \widetilde{\mathbf{v}}_{\left( 1\right) }\right\Vert _{2}}%
\mathbf{,}\widetilde{\eta }_{t}\right\rangle &=&\frac{1}{\sqrt{N}}\frac{%
\left\langle \widetilde{\mathbf{v}}_{\left( 1\right) }\mathbf{,}\widetilde{%
\eta }_{t}\right\rangle }{\left\Vert \widetilde{\mathbf{v}}_{\left( 1\right)
}\right\Vert _{2}} \\
&=&\frac{1}{\sqrt{N}}\left[ 1+\dsum\limits_{j=2}^{N}\frac{\upsilon _{j}^{2}}{%
\left( \widehat{\lambda }_{\left( 1\right) }-\widetilde{\lambda }_{\left(
j\right) }\right) ^{2}}\right] ^{-1/2}\left[ \widetilde{\eta }%
_{1t}+\dsum\limits_{j=2}^{N}\frac{\upsilon _{j}\widetilde{\eta }_{jt}}{%
\left( \widehat{\lambda }_{\left( 1\right) }-\widetilde{\lambda }_{\left(
j\right) }\right) }\right]
\end{eqnarray*}%
From the result given in expression (\ref{order of magnitude tau^2}) of Step
4 above, we have%
\begin{equation*}
\dsum\limits_{j=2}^{N}\frac{\upsilon _{j}^{2}}{\left( \widehat{\lambda }%
_{\left( 1\right) }-\widetilde{\lambda }_{\left( j\right) }\right) ^{2}}%
=\tau ^{2}=O_{p}\left( \frac{N}{T\left\Vert \gamma \right\Vert _{2}^{2}}%
\right)
\end{equation*}%
where $N/\left( T\left\Vert \gamma \right\Vert _{2}^{2}\right) \rightarrow
\infty $ under our Assumption 2-2. This implies that%
\begin{equation}
\frac{T\left\Vert \gamma \right\Vert _{2}^{2}}{N}\dsum\limits_{j=2}^{N}\frac{%
\upsilon _{j}^{2}}{\left( \widehat{\lambda }_{\left( 1\right) }-\widetilde{%
\lambda }_{\left( j\right) }\right) ^{2}}=O_{p}\left( 1\right) .
\label{normalized gamma*avgv^2}
\end{equation}%
Next, note that%
\begin{eqnarray}
\dsum\limits_{j=2}^{N}\upsilon _{j}\widetilde{\eta }_{jt}
&=&\dsum\limits_{j=2}^{T{\LARGE +}1}\upsilon _{j}\widetilde{\eta }%
_{jt}+\dsum\limits_{j=T{\LARGE +}2}^{N}\upsilon _{j}\widetilde{\eta }_{jt} 
\notag \\
&=&\dsum\limits_{j=2}^{T{\LARGE +}1}\frac{\mathbf{e}_{j-1,N-1}^{\prime }%
\widetilde{\mathbf{B}}_{2}^{\prime }\mathbf{W}_{2}^{\prime }\mathbf{W}_{1}%
\widetilde{\eta }_{jt}}{T}+\dsum\limits_{j=T{\LARGE +}2}^{N}\frac{\mathbf{e}%
_{j-1,N-1}^{\prime }\widetilde{\mathbf{B}}_{2}^{\prime }\mathbf{W}%
_{2}^{\prime }\mathbf{W}_{1}\widetilde{\eta }_{jt}}{T}
\label{cov of upsilon and eta}
\end{eqnarray}%
Recall that $\left\{ \widetilde{\eta }_{t}\right\} \equiv i.i.d.N\left(
0,I_{N}\right) $ so that $\left\{ \widetilde{\eta }_{j,t}\right\} \equiv
i.i.d.N\left( 0,1\right) $ across both $j$ and $t$. Recall also that $%
\left\{ f_{t}\right\} \equiv i.i.d.N\left( 0,1\right) $ and $f_{t}$ and $%
\widetilde{\eta }_{s}$ are independent for all $s$ and $t$. In addition,
since 
\begin{equation*}
\underset{T\times 1}{\mathbf{W}_{1}}=\left( 
\begin{array}{c}
\left\Vert \gamma \right\Vert _{2}\left( f_{1}+\left\Vert \gamma \right\Vert
_{2}^{-1}\eta _{1,1}\right) \\ 
\left\Vert \gamma \right\Vert _{2}\left( f_{2}+\left\Vert \gamma \right\Vert
_{2}^{-1}\eta _{1,2}\right) \\ 
\vdots \\ 
\left\Vert \gamma \right\Vert _{2}\left( f_{T}+\left\Vert \gamma \right\Vert
_{2}^{-1}\eta _{1,T}\right)%
\end{array}%
\right) \text{ and}\underset{T\times \left( N-1\right) }{\mathbf{W}_{2}}%
=\left( 
\begin{array}{cccc}
\eta _{2,1} & \eta _{3,1} & \cdots & \eta _{N-1,1} \\ 
\eta _{2,2} & \eta _{3,2} & \cdots & \eta _{N-1,2} \\ 
\vdots & \vdots &  & \vdots \\ 
\eta _{2,T} & \eta _{3,T} & \cdots & \eta _{N-1,T}%
\end{array}%
\right) \text{, }
\end{equation*}%
it follows that $\mathbf{W}_{1}$ and $\mathbf{W}_{2}$ are independent. Now,
focusing first on the term $\dsum\nolimits_{j=2}^{T{\LARGE +}1}\upsilon _{j}%
\widetilde{\eta }_{jt}$ on the right-hand side of expression (\ref{cov of
upsilon and eta}) above, note that%
\begin{eqnarray*}
&&E\left[ \left( \dsum\limits_{j=2}^{T{\LARGE +}1}\upsilon _{j}\widetilde{%
\eta }_{jt}\right) ^{2}|\mathbf{W}_{2}\right] \\
&=&\frac{1}{T^{2}}\dsum\limits_{j=2}^{T{\LARGE +}1}\dsum\limits_{k=2}^{T%
{\LARGE +}1}\widetilde{\eta }_{jt}\widetilde{\eta }_{kt}\mathbf{e}%
_{j-1,N-1}^{\prime }\widetilde{\mathbf{B}}_{2}^{\prime }\mathbf{W}%
_{2}^{\prime }E\left[ \mathbf{W}_{1}\mathbf{W}_{1}^{\prime }|\mathbf{W}_{2}%
\right] \mathbf{W}_{2}\widetilde{\mathbf{B}}_{2}\mathbf{e}_{k-1,N-1} \\
&=&\frac{1}{T^{2}}\dsum\limits_{j=2}^{T{\LARGE +}1}\dsum\limits_{k=2}^{T%
{\LARGE +}1}\widetilde{\eta }_{jt}\widetilde{\eta }_{kt}\mathbf{e}%
_{j-1,N-1}^{\prime }\widetilde{\mathbf{B}}_{2}^{\prime }\mathbf{W}%
_{2}^{\prime }E\left[ \mathbf{W}_{1}\mathbf{W}_{1}^{\prime }\right] \mathbf{W%
}_{2}\widetilde{\mathbf{B}}_{2}\mathbf{e}_{k-1,N-1} \\
&=&\frac{\left( \left\Vert \gamma \right\Vert _{2}^{2}+1\right) }{T}%
\dsum\limits_{j=2}^{T{\LARGE +}1}\dsum\limits_{k=2}^{T{\LARGE +}1}\widetilde{%
\eta }_{jt}\widetilde{\eta }_{kt}\mathbf{e}_{j-1,N-1}^{\prime }\widetilde{%
\mathbf{B}}_{2}^{\prime }\left( \frac{\mathbf{W}_{2}^{\prime }\mathbf{W}_{2}%
}{T}\right) \widetilde{\mathbf{B}}_{2}\mathbf{e}_{k-1,N-1} \\
&=&\frac{\left( \left\Vert \gamma \right\Vert _{2}^{2}+1\right) }{T}%
\dsum\limits_{j=2}^{T{\LARGE +}1}\dsum\limits_{k=2}^{T{\LARGE +}1}\widetilde{%
\eta }_{jt}\widetilde{\eta }_{kt}\mathbf{e}_{j-1,N-1}^{\prime }\widetilde{%
\mathbf{B}}_{2}^{\prime }\widetilde{\mathbf{B}}_{2}\widetilde{\Lambda }%
\widetilde{\mathbf{B}}_{2}^{\prime }\widetilde{\mathbf{B}}_{2}\mathbf{e}%
_{k-1,N-1} \\
&=&\frac{\left( \left\Vert \gamma \right\Vert _{2}^{2}+1\right) }{T}%
\dsum\limits_{j=2}^{T{\LARGE +}1}\dsum\limits_{k=2}^{T{\LARGE +}1}\widetilde{%
\eta }_{jt}\widetilde{\eta }_{kt}\mathbf{e}_{j-1,N-1}^{\prime }\widetilde{%
\Lambda }\mathbf{e}_{k-1,N-1} \\
&=&\frac{\left( \left\Vert \gamma \right\Vert _{2}^{2}+1\right) }{T}%
\dsum\limits_{j=2}^{T{\LARGE +}1}\widetilde{\eta }_{jt}^{2}\widetilde{%
\lambda }_{\left( j\right) }
\end{eqnarray*}%
This implies that%
\begin{eqnarray*}
&&E\left[ \left( \frac{1}{\left\Vert \gamma \right\Vert _{2}}\sqrt{\frac{T}{%
N-1}}\dsum\limits_{j=2}^{T{\LARGE +}1}\upsilon _{j}\widetilde{\eta }%
_{jt}\right) ^{2}|\mathbf{W}_{2}\right] \\
&=&\frac{\left( \left\Vert \gamma \right\Vert _{2}^{2}+1\right) }{%
T\left\Vert \gamma \right\Vert _{2}^{2}}\dsum\limits_{j=2}^{T{\LARGE +}1}%
\widetilde{\eta }_{jt}^{2}\frac{T}{N-1}\widetilde{\lambda }_{\left( j\right)
} \\
&\leq &\frac{\left( \left\Vert \gamma \right\Vert _{2}^{2}+1\right) }{%
\left\Vert \gamma \right\Vert _{2}^{2}}\sqrt{\frac{1}{T}\dsum\limits_{j=2}^{T%
{\LARGE +}1}\widetilde{\eta }_{jt}^{4}}\sqrt{\frac{1}{T}\dsum\limits_{j=2}^{T%
{\LARGE +}1}\left( \frac{T}{N-1}\widetilde{\lambda }_{\left( j\right)
}\right) ^{2}} \\
&=&O_{a.s.}\left( 1\right)
\end{eqnarray*}%
given that, as $N,T\rightarrow \infty $,%
\begin{equation*}
\frac{1}{T}\dsum\limits_{j=2}^{T{\LARGE +}1}\widetilde{\eta }_{jt}^{4}%
\overset{a.s.}{\rightarrow }3\text{ }
\end{equation*}%
and, by Lemma B-7,%
\begin{eqnarray*}
\frac{1}{T}\dsum\limits_{j=2}^{T{\LARGE +}1}\left( \frac{T}{N-1}\widetilde{%
\lambda }_{\left( j\right) }\right) ^{2} &\leq &\left( \frac{T}{N-1}%
\max_{2\leq j\leq T+1}\widetilde{\lambda }_{\left( j\right) }\right) ^{2}%
\text{ }\left( \text{since }\widetilde{\lambda }_{\left( j\right) }\geq 0%
\text{ for }j=2,..,T+1\right) \\
&=&\left( \frac{T}{N-1}\widetilde{\lambda }_{\left( 2\right) }\right) ^{2}%
\overset{a.s.}{\rightarrow }1\text{.}
\end{eqnarray*}%
Applying the law of iterated expectations as well as part (i) of Theorem
16.1 of Billingsley (1995), we see that there exists a constant $\overline{C}%
<\infty $ such that for all $n$ sufficiently large%
\begin{eqnarray*}
E\left\{ \left( \frac{1}{\left\Vert \gamma \right\Vert _{2}}\sqrt{\frac{T}{%
N-1}}\dsum\limits_{j=2}^{T{\LARGE +}1}\upsilon _{j}\widetilde{\eta }%
_{jt}\right) ^{2}\right\} &=&E_{\mathbf{W}_{2}}\left[ E\left\{ \left( \frac{1%
}{\left\Vert \gamma \right\Vert _{2}}\sqrt{\frac{T}{N-1}}\dsum%
\limits_{j=2}^{T{\LARGE +}1}\upsilon _{j}\widetilde{\eta }_{jt}\right) ^{2}|%
\mathbf{W}_{2}\right\} \right] \\
&\leq &\overline{C}\text{.}
\end{eqnarray*}%
Now, for any $\epsilon >0$, set $C_{\epsilon }=\sqrt{\overline{C}/\epsilon }$%
, and the Markov's inequality then implies that, for all $n$ sufficiently
large,%
\begin{eqnarray*}
\Pr \left\{ \left\vert \frac{1}{\left\Vert \gamma \right\Vert _{2}}\sqrt{%
\frac{T}{N-1}}\left\vert \dsum\limits_{j=2}^{T{\LARGE +}1}\upsilon _{j}%
\widetilde{\eta }_{jt}\right\vert \right\vert \geq C_{\epsilon }\right\}
&=&\Pr \left\{ \left( \frac{1}{\left\Vert \gamma \right\Vert _{2}}\sqrt{%
\frac{T}{N-1}}\dsum\limits_{j=2}^{T{\LARGE +}1}\upsilon _{j}\widetilde{\eta }%
_{jt}\right) ^{2}\geq C_{\epsilon }^{2}\right\} \\
&\leq &\frac{1}{C_{\epsilon }^{2}}E\left\{ \left( \frac{1}{\left\Vert \gamma
\right\Vert _{2}}\sqrt{\frac{T}{N-1}}\dsum\limits_{j=2}^{T{\LARGE +}%
1}\upsilon _{j}\widetilde{\eta }_{jt}\right) ^{2}\right\} \\
&=&\frac{\epsilon }{\overline{C}}E\left\{ \left( \frac{1}{\left\Vert \gamma
\right\Vert _{2}}\sqrt{\frac{T}{N-1}}\dsum\limits_{j=2}^{T{\LARGE +}%
1}\upsilon _{j}\widetilde{\eta }_{jt}\right) ^{2}\right\} \\
&\leq &\epsilon
\end{eqnarray*}%
which shows that%
\begin{equation}
\frac{1}{\left\Vert \gamma \right\Vert _{2}}\sqrt{\frac{T}{N-1}}\left\vert
\dsum\limits_{j=2}^{T{\LARGE +}1}\upsilon _{j}\widetilde{\eta }%
_{jt}\right\vert =O_{p}\left( 1\right) .  \label{gamma^-1*avgv*eta}
\end{equation}%
\qquad

Next, consider the second term on the right-hand side of expression (\ref%
{cov of upsilon and eta}). Define%
\begin{equation*}
\underset{T\times \left( N-1\right) }{\widetilde{D}}=\left[ 
\begin{array}{cc}
\underset{T\times T}{\widetilde{\Lambda }_{1}} & \underset{T\times \left(
N-T-1\right) }{0}%
\end{array}%
\right]
\end{equation*}%
where%
\begin{equation*}
\widetilde{\Lambda }_{1}=\left( 
\begin{array}{cccc}
\widetilde{\lambda }_{\left( 2\right) } & 0 & \cdots & 0 \\ 
0 & \widetilde{\lambda }_{\left( 3\right) } & \ddots & \vdots \\ 
\vdots & \ddots & \ddots & 0 \\ 
0 & \cdots & 0 & \widetilde{\lambda }_{\left( T+1\right) }%
\end{array}%
\right)
\end{equation*}%
Given that $N-1>T$ for $N,T$ sufficiently large and given that $\widetilde{%
\lambda }_{\left( j\right) }=0$ for $j>T+1$, we have the following
singular-value decomposition of $\mathbf{W}_{2}$:%
\begin{equation*}
\mathbf{W}_{2}=\mathbb{O}\widetilde{D}\widetilde{\mathbf{B}}_{2}^{\prime }
\end{equation*}%
where $\mathbb{O}$ is a $T\times T$ orthogonal matrix and $\widetilde{%
\mathbf{B}}_{2}$ is as defined previously. Making use of this decomposition,
we see that%
\begin{eqnarray*}
\dsum\limits_{j=T{\LARGE +}2}^{N}\upsilon _{j}\widetilde{\eta }_{jt}
&=&\dsum\limits_{j=T{\LARGE +}2}^{N}\frac{\mathbf{e}_{j-1,N-1}^{\prime }%
\widetilde{\mathbf{B}}_{2}^{\prime }\mathbf{W}_{2}^{\prime }\mathbf{W}_{1}%
\widetilde{\eta }_{jt}}{T} \\
&=&\dsum\limits_{j=T{\LARGE +}2}^{N}\frac{\widetilde{\eta }_{jt}\mathbf{W}%
_{1}^{\prime }\mathbf{W}_{2}\widetilde{\mathbf{B}}_{2}\mathbf{e}_{j-1,N-1}}{T%
} \\
&=&\dsum\limits_{j=T{\LARGE +}2}^{N}\frac{\widetilde{\eta }_{jt}\mathbf{W}%
_{1}^{\prime }\mathbb{O}\widetilde{D}\widetilde{\mathbf{B}}_{2}^{\prime }%
\widetilde{\mathbf{B}}_{2}\mathbf{e}_{j-1,N-1}}{T} \\
&=&\dsum\limits_{j=T{\LARGE +}2}^{N}\frac{\widetilde{\eta }_{jt}\mathbf{W}%
_{1}^{\prime }\mathbb{O}\widetilde{D}\mathbf{e}_{j-1,N-1}}{T} \\
&=&0
\end{eqnarray*}%
Putting things together, we have%
\begin{eqnarray*}
&&\frac{1}{\sqrt{N}}\left\vert \left\langle \frac{\widetilde{\mathbf{v}}%
_{\left( 1\right) }}{\left\Vert \widetilde{\mathbf{v}}_{\left( 1\right)
}\right\Vert _{2}}\mathbf{,}\widetilde{\eta }_{t}\right\rangle \right\vert \\
&=&\frac{1}{\sqrt{N}}\left\vert \frac{\left\langle \widetilde{\mathbf{v}}%
_{\left( 1\right) }\mathbf{,}\widetilde{\eta }_{t}\right\rangle }{\left\Vert 
\widetilde{\mathbf{v}}_{\left( 1\right) }\right\Vert _{2}}\right\vert \\
&=&\frac{1}{\sqrt{N}}\left[ 1+\dsum\limits_{j=2}^{N}\frac{\upsilon _{j}^{2}}{%
\left( \widehat{\lambda }_{\left( 1\right) }-\widetilde{\lambda }_{\left(
j\right) }\right) ^{2}}\right] ^{-1/2}\left\vert \widetilde{\eta }%
_{1t}+\dsum\limits_{j=2}^{N}\frac{\upsilon _{j}\widetilde{\eta }_{jt}}{%
\left( \widehat{\lambda }_{\left( 1\right) }-\widetilde{\lambda }_{\left(
j\right) }\right) }\right\vert \\
&=&\frac{1}{\sqrt{N}}\left[ 1+\dsum\limits_{j=2}^{N}\frac{\upsilon _{j}^{2}}{%
\left( \widehat{\lambda }_{\left( 1\right) }-\widetilde{\lambda }_{\left(
j\right) }\right) ^{2}}\right] ^{-1/2}\left\vert \widetilde{\eta }_{1t}+%
\frac{1}{\widehat{\lambda }_{\left( 1\right) }}\dsum\limits_{j=2}^{T{\LARGE +%
}1}\frac{\upsilon _{j}\widetilde{\eta }_{jt}}{\left( 1-\widetilde{\lambda }%
_{\left( j\right) }/\widehat{\lambda }_{\left( 1\right) }\right) }+\frac{1}{%
\widehat{\lambda }_{\left( 1\right) }}\dsum\limits_{j=T{\LARGE +}%
2}^{N}\upsilon _{j}\widetilde{\eta }_{jt}\right\vert \\
&&\left( \text{ noting that }\widetilde{\lambda }_{j}=0\text{ for }%
j>T+1\right) \\
&=&\frac{1}{\sqrt{N}}\left[ 1+\dsum\limits_{j=2}^{N}\frac{\upsilon _{j}^{2}}{%
\left( \widehat{\lambda }_{\left( 1\right) }-\widetilde{\lambda }_{\left(
j\right) }\right) ^{2}}\right] ^{-1/2}\left\vert \widetilde{\eta }_{1t}+%
\frac{1}{\widehat{\lambda }_{\left( 1\right) }}\dsum\limits_{j=2}^{T{\LARGE +%
}1}\frac{\upsilon _{j}\widetilde{\eta }_{jt}}{\left( 1-\widetilde{\lambda }%
_{\left( j\right) }/\widehat{\lambda }_{\left( 1\right) }\right) }\right\vert
\\
&&\left( \text{since }\dsum\limits_{j=T{\LARGE +}2}^{N}\upsilon _{j}%
\widetilde{\eta }_{jt}=0\right) \\
&=&\frac{1}{\sqrt{N}}\left[ 1+\dsum\limits_{j=2}^{N}\frac{\upsilon _{j}^{2}}{%
\left( \widehat{\lambda }_{\left( 1\right) }-\widetilde{\lambda }_{\left(
j\right) }\right) ^{2}}\right] ^{-1/2}\left\vert \widetilde{\eta }_{1t}+%
\frac{1}{\widehat{\lambda }_{\left( 1\right) }/\left\Vert \gamma \right\Vert
_{2}^{2\left( 1+\kappa \right) }}\frac{1}{\left\Vert \gamma \right\Vert
_{2}^{2\left( 1+\kappa \right) }}\dsum\limits_{j=2}^{T{\LARGE +}1}\frac{%
\upsilon _{j}\widetilde{\eta }_{jt}}{\left( 1-\widetilde{\lambda }_{\left(
j\right) }/\widehat{\lambda }_{\left( 1\right) }\right) }\right\vert \\
&\leq &\frac{1}{\sqrt{N}}\left\{ \left[ 1+\dsum\limits_{j=2}^{N}\frac{%
\upsilon _{j}^{2}}{\left( \widehat{\lambda }_{\left( 1\right) }-\widetilde{%
\lambda }_{\left( j\right) }\right) ^{2}}\right] ^{-1/2}\right. \\
&&\text{ \ \ \ \ }\left. \times \left[ \left\vert \widetilde{\eta }%
_{1t}\right\vert +\frac{1}{\widehat{\lambda }_{\left( 1\right) }/\left\Vert
\gamma \right\Vert _{2}^{2\left( 1+\kappa \right) }}\frac{1}{\left\Vert
\gamma \right\Vert _{2}^{2\left( 1+\kappa \right) }}\left( 1-\max_{2\leq
j\leq T+1}\frac{\widetilde{\lambda }_{\left( j\right) }}{\widehat{\lambda }%
_{\left( 1\right) }}\right) ^{-1}\left\vert \dsum\limits_{j=2}^{T{\LARGE +}%
1}\upsilon _{j}\widetilde{\eta }_{jt}\right\vert \right] \right\}
\end{eqnarray*}%
\begin{eqnarray}
&=&\left[ 1+\dsum\limits_{j=2}^{N}\frac{\upsilon _{j}^{2}}{\left( \widehat{%
\lambda }_{\left( 1\right) }-\widetilde{\lambda }_{\left( j\right) }\right)
^{2}}\right] ^{-1/2}  \notag \\
&&\times \left[ \frac{\left\vert \widetilde{\eta }_{1t}\right\vert }{\sqrt{N}%
}+\left( c+\frac{1}{\left\Vert \gamma \right\Vert _{2}^{2\kappa }}%
+o_{p}\left( \frac{1}{\left\Vert \gamma \right\Vert _{2}^{2\kappa }}\right)
\right) ^{-1}\frac{\left\Vert \gamma \right\Vert _{2}}{\sqrt{N}}\frac{%
c\left\Vert \gamma \right\Vert _{2}^{2\kappa }}{\left\Vert \gamma
\right\Vert _{2}\left\Vert \gamma \right\Vert _{2}^{2\left( 1+\kappa \right)
}}\left\vert \dsum\limits_{j=2}^{T{\LARGE +}1}\upsilon _{j}\widetilde{\eta }%
_{jt}\right\vert \left( 1+o_{p}\left( 1\right) \right) \right]  \notag \\
&&\left( \text{given that }\frac{\widehat{\lambda }_{\left( 1\right) }}{%
\left\Vert \gamma \right\Vert _{2}^{2\left( 1+\kappa \right) }}=c+\frac{1}{%
\left\Vert \gamma \right\Vert _{2}^{2\kappa }}+o_{p}\left( \frac{1}{%
\left\Vert \gamma \right\Vert _{2}^{2\kappa }}\right) \text{ for }0<\kappa <1%
\text{,}\right.  \notag \\
&&\left. \text{ and }\left( 1-\max_{2\leq j\leq T+1}\frac{\widetilde{\lambda 
}_{\left( j\right) }}{\widehat{\lambda }_{\left( 1\right) }}\right)
^{-1}=c\left\Vert \gamma \right\Vert _{2}^{2\kappa }\left[ 1+o_{p}\left(
1\right) \right] \right)  \notag \\
&=&\left[ 1+\dsum\limits_{j=2}^{N}\frac{\upsilon _{j}^{2}}{\left( \widehat{%
\lambda }_{\left( 1\right) }-\widetilde{\lambda }_{\left( j\right) }\right)
^{2}}\right] ^{-1/2}  \notag \\
&&\times \left[ \frac{\left\vert \widetilde{\eta }_{1t}\right\vert }{\sqrt{N}%
}+\sqrt{\frac{N-1}{T}}\frac{\left\Vert \gamma \right\Vert _{2}^{2\kappa }}{%
\left\Vert \gamma \right\Vert _{2}^{2\left( 1+\kappa \right) }}\frac{%
\left\Vert \gamma \right\Vert _{2}}{\sqrt{N}}\frac{1}{\left\Vert \gamma
\right\Vert _{2}}\sqrt{\frac{T}{N-1}}\left\vert \dsum\limits_{j=2}^{T{\LARGE %
+}1}\upsilon _{j}\widetilde{\eta }_{jt}\right\vert \left( 1+o_{p}\left(
1\right) \right) \right]  \notag \\
&=&\left[ 1+\dsum\limits_{j=2}^{N}\frac{\upsilon _{j}^{2}}{\left( \widehat{%
\lambda }_{\left( 1\right) }-\widetilde{\lambda }_{\left( j\right) }\right)
^{2}}\right] ^{-1/2}  \notag \\
&&\times \left[ \frac{\left\vert \widetilde{\eta }_{1t}\right\vert }{\sqrt{N}%
}+\sqrt{\frac{N-1}{T}}\frac{\left\Vert \gamma \right\Vert _{2}}{\left\Vert
\gamma \right\Vert _{2}^{2}\sqrt{N}}\frac{1}{\left\Vert \gamma \right\Vert
_{2}}\sqrt{\frac{T}{N-1}}\left\vert \dsum\limits_{j=2}^{T{\LARGE +}%
1}\upsilon _{j}\widetilde{\eta }_{jt}\right\vert \left( 1+o_{p}\left(
1\right) \right) \right]  \notag \\
&=&\left[ \frac{T\left\Vert \gamma \right\Vert _{2}^{2}}{N}+\frac{%
T\left\Vert \gamma \right\Vert _{2}^{2}}{N}\dsum\limits_{j=2}^{N}\frac{%
\upsilon _{j}^{2}}{\left( \widehat{\lambda }_{\left( 1\right) }-\widetilde{%
\lambda }_{\left( j\right) }\right) ^{2}}\right] ^{-1/2}  \notag \\
&&\times \left[ \sqrt{\frac{T\left\Vert \gamma \right\Vert _{2}^{2}}{N}}%
\frac{\left\vert \widetilde{\eta }_{1t}\right\vert }{N}+\sqrt{\frac{N-1}{N}}%
\frac{1}{\sqrt{N}}\frac{1}{\left\Vert \gamma \right\Vert _{2}}\sqrt{\frac{T}{%
N-1}}\left\vert \dsum\limits_{j=2}^{T{\LARGE +}1}\upsilon _{j}\widetilde{%
\eta }_{jt}\right\vert \left( 1+o_{p}\left( 1\right) \right) \right]  \notag
\\
&=&o_{p}\left( 1\right) ,  \label{dot prod vtilde and etatilde}
\end{eqnarray}%
where the last line follows from the fact that%
\begin{eqnarray*}
\frac{1}{\left\Vert \gamma \right\Vert _{2}}\sqrt{\frac{T}{N-1}}\left\vert
\dsum\limits_{j=2}^{T{\LARGE +}1}\upsilon _{j}\widetilde{\eta }%
_{jt}\right\vert &=&O_{p}\left( 1\right) \text{ }\left( \text{ by expression
(\ref{gamma^-1*avgv*eta})}\right) \\
\frac{T\left\Vert \gamma \right\Vert _{2}^{2}}{N}\dsum\limits_{j=2}^{N}\frac{%
\upsilon _{j}^{2}}{\left( \widehat{\lambda }_{\left( 1\right) }-\widetilde{%
\lambda }_{\left( j\right) }\right) ^{2}} &=&O_{p}\left( 1\right) \text{\ }%
\left( \text{by expression (\ref{normalized gamma*avgv^2})}\right)
\end{eqnarray*}%
and the fact that%
\begin{equation*}
\left\Vert \gamma \right\Vert _{2}^{2}\rightarrow \infty \text{ and }\frac{%
T\left\Vert \gamma \right\Vert _{2}^{2}}{N}\rightarrow 0\text{ }\left( \text{%
by Assumption 2-2}\right) \text{.}
\end{equation*}

\bigskip

\noindent \textbf{Step 6: }

Finally, in this last step, we bring everything together. Combining the
results given in expressions (\ref{gamma/N^(1/2)}) of step 3, (\ref{dot prod
vtilde and e1}) of step 4, and (\ref{dot prod vtilde and etatilde}) of step
5 and noting the fact that $f_{t}=O_{p}\left( 1\right) $, we can apply the
Slutsky's theorem to deduce that%
\begin{equation*}
\widehat{f}_{t}=\frac{\left\Vert \gamma \right\Vert _{2}}{\sqrt{N}}\frac{%
\left\langle \widetilde{\mathbf{v}}_{\left( 1\right) }\mathbf{,e}%
_{1,N}\right\rangle }{\left\Vert \widetilde{\mathbf{v}}_{\left( 1\right)
}\right\Vert _{2}}f_{t}+\frac{1}{\sqrt{N}}\frac{\left\langle \widetilde{%
\mathbf{v}}_{\left( 1\right) }\mathbf{,}\widetilde{\eta }_{t}\right\rangle }{%
\left\Vert \widetilde{\mathbf{v}}_{\left( 1\right) }\right\Vert _{2}}\overset%
{p}{\rightarrow }0\text{ as }N,T\rightarrow \infty
\end{equation*}%
which is the required result. $\square $

\noindent

\noindent

\section{Appendix B: Supporting Lemmas Used in the Proof of Theorem 2.1}

\noindent \qquad In this appendix, we first state and prove a number of
lemmas which are used in the proof of Theorem 2.1.

\medskip

\noindent \textbf{Lemma B-1 (Weyl's inequality):} Let $A,$ $B$ be real,
symmetric $T\times T$ matrices and let the eigenvalues $\lambda _{\left(
i\right) }\left( A\right) $, $\lambda _{\left( i\right) }\left( B\right) $,
and $\lambda _{\left( i\right) }\left( A+B\right) $ be arranged in
decreasing (or, more generally, nonincreasing) order, so that%
\begin{eqnarray*}
\lambda _{\left( 1\right) }\left( A\right) &\geq &\lambda _{\left( 2\right)
}\left( A\right) \geq \cdot \cdot \cdot \geq \lambda _{\left( T\right)
}\left( A\right) \text{,} \\
\lambda _{\left( 1\right) }\left( B\right) &\geq &\lambda _{\left( 2\right)
}\left( B\right) \geq \cdot \cdot \cdot \geq \lambda _{\left( T\right)
}\left( B\right) \text{, } \\
\lambda _{\left( 1\right) }\left( A+B\right) &\geq &\lambda _{\left(
2\right) }\left( A+B\right) \geq \cdot \cdot \cdot \geq \lambda _{\left(
T\right) }\left( A+B\right) \text{.}
\end{eqnarray*}
Then, for each $j=1,2,...,T$, we have%
\begin{equation*}
\lambda _{\left( j\right) }\left( A\right) +\lambda _{\left( T\right)
}\left( B\right) \leq \lambda _{\left( j\right) }\left( A+B\right) \leq
\lambda _{\left( j\right) }\left( A\right) +\lambda _{\left( 1\right)
}\left( B\right) \text{. }
\end{equation*}%
\textbf{Proof of Lemma B-1:} This inequality is well-known, and its proof
can be found in many linear algebra textbooks. See, for example, Theorem
4.3.1 and its proof on pages 181-182 of Horn and Johnson (1985). Hence, we
shall not provide an explicit proof here. $\square $

\noindent

\noindent \textbf{Lemma B-2: } Suppose that $\left\Vert \gamma \right\Vert
_{2}^{2}\rightarrow \infty $ as $N\rightarrow \infty $, and suppose that,
given $N$, 
\begin{equation*}
\left\{ \zeta _{1,t,N}\right\} \equiv i.i.d.N\left( 0,1+\frac{1}{\left\Vert
\gamma \right\Vert _{2}^{2}}\right) \text{ for }t=1,...,T\text{.}
\end{equation*}%
Let $\zeta _{1,N}=\left( 
\begin{array}{cccc}
\zeta _{1,1,N} & \zeta _{1,2,N} & \cdots & \zeta _{1,T,N}%
\end{array}%
\right) ^{\prime }$ and $\underset{T\times T}{A}=T^{-1}\left\Vert \gamma
\right\Vert _{2}^{2}\zeta _{1,N}\zeta _{1,N}^{\prime }$ . Then, as $%
N,T\rightarrow \infty $ such that $T/N\rightarrow 0$, we have 
\begin{equation*}
\frac{\lambda _{\left( 1\right) }\left( A\right) }{\left\Vert \gamma
\right\Vert _{2}^{2}}=1+\frac{1}{\left\Vert \gamma \right\Vert _{2}^{2}}%
+O_{p}\left( \frac{1}{\sqrt{T}}\right) \text{ }
\end{equation*}%
where $\lambda _{\left( 1\right) }\left( A\right) $ denotes the largest
eigenvalue of the matrix $A$.

\medskip

\noindent \textbf{Proof of Lemma B-2:}

Note that, since $A=\left\Vert \gamma \right\Vert _{2}^{2}\zeta _{1,N}\zeta
_{1,N}^{\prime }/T$, we can write its dual $a_{D}$ as%
\begin{equation*}
\underset{1\times 1}{a_{D}}\text{ }=\frac{1}{T}\left\Vert \gamma \right\Vert
_{2}^{2}\zeta _{1,N}^{\prime }\zeta _{1,N}
\end{equation*}%
Next, write%
\begin{equation*}
\frac{1}{T}\zeta _{1,N}^{\prime }\zeta _{1,N}=\frac{1}{T}\dsum%
\limits_{t=1}^{T}\zeta _{1,t,N}^{2}=\left( 1+\frac{1}{\left\Vert \gamma
\right\Vert _{2}^{2}}\right) \frac{1}{T}\dsum\limits_{t=1}^{T}\left( 1+\frac{%
1}{\left\Vert \gamma \right\Vert _{2}^{2}}\right) ^{-1}\zeta _{1,t,N}^{2}
\end{equation*}%
where, by assumption,%
\begin{equation*}
\left\{ \zeta _{1,t,N}\right\} \equiv i.i.d.N\left( 0,1+\frac{1}{\left\Vert
\gamma \right\Vert _{2}^{2}}\right) \text{ for each }N\text{. }
\end{equation*}%
This implies that%
\begin{eqnarray*}
\left\{ \left( 1+\frac{1}{\left\Vert \gamma \right\Vert _{2}^{2}}\right)
^{-1/2}\zeta _{1,t,N}\right\} &\equiv &i.i.d.N\left( 0,1\right) \text{ and}
\\
\left\{ \mathcal{X}_{t,N}^{\ast }\right\} &\equiv &i.i.d.\chi _{1}^{2}
\end{eqnarray*}%
where 
\begin{equation*}
\mathcal{X}_{t,N}^{\ast }=\left[ \left( 1+\frac{1}{\left\Vert \gamma
\right\Vert _{2}^{2}}\right) ^{-1/2}\zeta _{1,t,N}\right] ^{2}=\left( 1+%
\frac{1}{\left\Vert \gamma \right\Vert _{2}^{2}}\right) ^{-1}\zeta
_{1,t,N}^{2}
\end{equation*}%
and where $\chi _{1}^{2}$ denotes a chi-square random variable with one
degree of freedom. Hence, by direct calculation, we get%
\begin{eqnarray*}
&&E\left( \frac{1}{T}\zeta _{1\cdot ,N}^{\prime }\zeta _{1\cdot ,N}-\left[ 1+%
\frac{1}{\left\Vert \gamma \right\Vert _{2}^{2}}\right] \right) ^{2} \\
&=&E\left[ \left( 1+\frac{1}{\left\Vert \gamma \right\Vert _{2}^{2}}\right) 
\frac{1}{T}\dsum\limits_{t=1}^{T}\left( \left( 1+\frac{1}{\left\Vert \gamma
\right\Vert _{2}^{2}}\right) ^{-1}\zeta _{1,t,N}^{2}-1\right) \right] ^{2} \\
&=&\left( 1+\frac{1}{\left\Vert \gamma \right\Vert _{2}^{2}}\right) ^{2}%
\frac{1}{T^{2}}\dsum\limits_{t=1}^{T}\dsum\limits_{s=1}^{T}E\left\{ \left[
\left( 1+\frac{1}{\left\Vert \gamma \right\Vert _{2}^{2}}\right) ^{-1}\zeta
_{1,t,N}^{2}-1\right] \left[ \left( 1+\frac{1}{\left\Vert \gamma \right\Vert
_{2}^{2}}\right) ^{-1}\zeta _{1,s,N}^{2}-1\right] \right\} \\
&=&\left( 1+\frac{1}{\left\Vert \gamma \right\Vert _{2}^{2}}\right) ^{2}%
\frac{1}{T^{2}}\dsum\limits_{t=1}^{T}E\left\{ \left[ \left( 1+\frac{1}{%
\left\Vert \gamma \right\Vert _{2}^{2}}\right) ^{-1}\zeta _{1,t,N}^{2}-1%
\right] ^{2}\right\} \\
&=&\frac{2}{T}\left( 1+\frac{1}{\left\Vert \gamma \right\Vert _{2}^{2}}%
\right) ^{2}\text{ }\left( \text{since }E\left[ \chi _{1}^{2}\right] =1\text{
and }Var\left( \chi _{1}^{2}\right) =2\right) \\
&=&O\left( \frac{1}{T}\right)
\end{eqnarray*}%
Applying Markov's inequality, we then obtain%
\begin{equation*}
\frac{1}{T}\zeta _{1\cdot ,N}^{\prime }\zeta _{1\cdot ,N}=1+\frac{1}{%
\left\Vert \gamma \right\Vert _{2}^{2}}+O_{p}\left( \frac{1}{\sqrt{T}}\right)
\end{equation*}%
Hence, as $N,T\rightarrow \infty $%
\begin{eqnarray*}
\frac{\lambda _{\left( 1\right) }\left( A\right) }{\left\Vert \gamma
\right\Vert _{2}^{2}} &=&\frac{a_{D}}{\left\Vert \gamma \right\Vert _{2}^{2}}%
\text{ } \\
&=&\left( \frac{1}{\left\Vert \gamma \right\Vert _{2}^{2}}\right) \frac{1}{T}%
\left\Vert \gamma \right\Vert _{2}^{2}\zeta _{1\cdot ,N}^{\prime }\zeta
_{1\cdot ,N} \\
&=&\frac{1}{T}\zeta _{1\cdot ,N}^{\prime }\zeta _{1\cdot ,N} \\
&=&1+\frac{1}{\left\Vert \gamma \right\Vert _{2}^{2}}+O_{p}\left( \frac{1}{%
\sqrt{T}}\right)
\end{eqnarray*}%
where the first equality above follows from the fact that $\lambda _{\left(
1\right) }\left( A\right) =\lambda _{\max }\left( A\right) =\lambda _{\max
}\left( a_{D}\right) =a_{D}$ given that $a_{D}$ is a scalar. This proves
Lemma B-2. $\square $

\medskip

\noindent \textbf{Lemma B-3: }Let $X_{1},X_{2},...,$ $X_{N}$ be $N$
independent $T$ dimensional sub-Gaussian random vectors with zero mean
vector and identity covariance matrix and the sub-Gaussian norms bounded by
a constant $C_{0}$. Then, for every $\tau \geq 0$, with probability at least%
\begin{equation*}
1-2\exp \left\{ -c\tau ^{2}\right\} \text{,}
\end{equation*}%
one has%
\begin{eqnarray*}
\overline{w}-\max \left\{ \delta ,\delta ^{2}\right\} &\leq &\lambda
_{\left( T\right) }\left( \frac{1}{N}\dsum\limits_{i=1}^{N}w_{i}X_{i}X_{i}^{%
\prime }\right) \\
&\leq &\lambda _{\left( 1\right) }\left( \frac{1}{N}\dsum%
\limits_{i=1}^{N}w_{i}X_{i}X_{i}^{\prime }\right) \\
&=&\overline{w}+\max \left\{ \delta ,\delta ^{2}\right\}
\end{eqnarray*}%
where%
\begin{equation*}
\delta =C\sqrt{\frac{T}{N}}+\frac{\tau }{\sqrt{N}}
\end{equation*}%
for constants $C,c>0$, depending on $C_{0}$. Here, $\left\vert
w_{i}\right\vert $ is bounded for all $i$ and%
\begin{equation*}
\overline{w}=\frac{1}{N}\dsum\limits_{i=1}^{N}w_{i}\text{.}
\end{equation*}

\noindent \textbf{Remark: }Lemma B-3 is Lemma A.1 given in Appendix A of
Wang and Fan (2017), and so we state this result here without proof. As
discussed there, this lemma is an extension of the classical Davidson-Szarek
bound. See Davidson and Szarek (2001) and Vershynin (2010) for additional
discussion.

\medskip

\noindent \textbf{Lemma B-4: }Suppose that 
\begin{equation*}
\left\{ \zeta _{i,t}\right\} \equiv i.i.d.N\left( 0,1\right) \text{ for }%
i=2,...,N\text{; }t=1,...,T
\end{equation*}%
Let $\zeta _{i}=\left( 
\begin{array}{cccc}
\zeta _{i,1} & \zeta _{i,2} & \cdots & \zeta _{i,T}%
\end{array}%
\right) ^{\prime }$. Also, let%
\begin{equation*}
\underset{T\times T}{B}=\frac{1}{T}\dsum\limits_{i=2}^{N}\zeta _{i}\zeta
_{i}^{\prime }
\end{equation*}%
and let 
\begin{equation*}
\lambda _{\left( 1\right) }\left( B\right) \geq \lambda _{\left( 2\right)
}\left( B\right) \geq \cdot \cdot \cdot \cdot \geq \lambda _{\left( T\right)
}\left( B\right)
\end{equation*}%
denote the eigenvalues of $B$. Then, for $k=1,...,T$;%
\begin{equation*}
\frac{T}{N-1}\lambda _{\left( k\right) }\left( B\right) =1+O_{p}\left( \sqrt{%
\frac{T}{N}}\right) =1+o_{p}\left( 1\right) \text{,}
\end{equation*}%
as $N,T\rightarrow \infty $ such that $T/N\rightarrow 0$.

\bigskip

\noindent \textbf{Proof of Lemma B-4:}

Applying Lemma B-3 above for the case where $\tau =\sqrt{T}$ and where $%
w_{i}=1$ for all $i$, we see that, with probability at least%
\begin{equation*}
1-2\exp \left\{ -c\tau ^{2}\right\} =1-2\exp \left\{ -cT\right\} \text{,}
\end{equation*}%
the following inequality holds for any $k\in \left\{ 1,...,T\right\} $%
\begin{eqnarray*}
1-\max \left\{ \delta ,\delta ^{2}\right\} &\leq &\lambda _{\left( T\right)
}\left( \frac{1}{N-1}\dsum\limits_{j=2}^{N}\zeta _{j}\zeta _{j}^{\prime
}\right) \\
&\leq &\lambda _{\left( k\right) }\left( \frac{1}{N-1}\dsum\limits_{j=2}^{N}%
\zeta _{j}\zeta _{j}^{\prime }\right) \\
&\leq &\lambda _{\left( 1\right) }\left( \frac{1}{N-1}\dsum\limits_{j=2}^{N}%
\zeta _{j}\zeta _{j}^{\prime }\right) \\
&=&1+\max \left\{ \delta ,\delta ^{2}\right\} \text{.}
\end{eqnarray*}%
Since in this case 
\begin{equation*}
\delta =C\sqrt{\frac{T}{N}}+\frac{\tau }{\sqrt{N}}=\left( 1+C\right) \sqrt{%
\frac{T}{N}}\text{,}
\end{equation*}%
the above inequality relationship simplifies to%
\begin{equation*}
1-\left( 1+C\right) \sqrt{\frac{T}{N}}\leq \lambda _{\left( k\right) }\left( 
\frac{1}{N-1}\dsum\limits_{j=2}^{N}\zeta _{j}\zeta _{j}^{\prime }\right)
\leq 1+\left( 1+C\right) \sqrt{\frac{T}{N}}
\end{equation*}%
or%
\begin{equation*}
1-\left( 1+C\right) \sqrt{\frac{T}{N}}\leq \frac{T}{N-1}\lambda _{\left(
k\right) }\left( \frac{1}{T}\dsum\limits_{j=2}^{N}\zeta _{j\cdot }\zeta
_{j\cdot }^{\prime }\right) =\frac{T}{N-1}\lambda _{\left( k\right) }\left(
B\right) \leq 1+\left( 1+C\right) \sqrt{\frac{T}{N}}
\end{equation*}%
This shows that, as $N,T\rightarrow \infty $ such that $T/N\rightarrow 0$, 
\begin{equation*}
\frac{T}{N-1}\lambda _{\left( k\right) }\left( B\right) =1+O_{p}\left( \sqrt{%
\frac{T}{N}}\right) =1+o_{p}\left( 1\right)
\end{equation*}%
for $k=1,...,T$. $\square $

\noindent \textbf{Lemma B-5: }Suppose that $\left\{ \mathbf{W}_{2,t}\right\}
\equiv i.i.d.N\left( 0,I_{N-1}\right) $. Now, let 
\begin{equation*}
\underset{\left( N-1\right) \times T}{\mathbf{W}_{2}^{\prime }}=\left( 
\begin{array}{cccc}
\underset{\left( N-1\right) \times 1}{\mathbf{W}_{2,1}} & \underset{\left(
N-1\right) \times 1}{\mathbf{W}_{2,2}} & \cdots & \underset{\left(
N-1\right) \times 1}{\mathbf{W}_{2,T}}%
\end{array}%
\right)
\end{equation*}%
and let 
\begin{equation*}
\widetilde{\lambda }_{\left( 2\right) }\geq \widetilde{\lambda }_{\left(
3\right) }\geq \cdot \cdot \cdot \geq \widetilde{\lambda }_{\left( N\right) }
\end{equation*}%
be the $N-1$ eigenvalues of 
\begin{equation*}
\widehat{\Sigma }_{\mathbf{W}_{\mathbf{2}}}=\frac{\mathbf{W}_{2}^{\prime }%
\mathbf{W}_{2}}{T}=\frac{1}{T}\dsum\limits_{t=1}^{T}\mathbf{W}_{2,t}\mathbf{W%
}_{2,t}^{\prime }\text{.}
\end{equation*}%
Then, the following results hold as $N,T\rightarrow \infty $ such that $%
T/N\rightarrow 0$.

\begin{enumerate}
\item[(a)] 
\begin{equation*}
\widetilde{\lambda }_{\left( j\right) }=0\text{ for }j=T+2,...,N
\end{equation*}

\item[(b)] 
\begin{equation*}
\frac{T}{N-1}\max_{2\leq j\leq T+1}\widetilde{\lambda }_{\left( j\right)
}=1+O_{p}\left( \sqrt{\frac{T}{N}}\right) =1+o_{p}\left( 1\right) \text{.}
\end{equation*}%
\newline
\end{enumerate}

\noindent \textbf{Proof of Lemma B-5:}

To show part (a), note that, by assumption, for $N,T$ sufficiently large, we
have $N-1>T$, so that $\widehat{\Sigma }_{\mathbf{W}_{\mathbf{2}}}=\mathbf{W}%
_{2}^{\prime }\mathbf{W}_{2}/T$ is a $\left( N-1\right) \times \left(
N-1\right) $ matrix with rank less than or equal to $T$, from which it
follows trivially that%
\begin{equation*}
\widetilde{\lambda }_{\left( j\right) }=0\text{ for }j=T+2,...,N\text{. }
\end{equation*}

Next, to show part (b), first write 
\begin{equation*}
\underset{T\times \left( N-1\right) }{\mathbf{W}_{2}}=\left( 
\begin{array}{cccc}
\underline{W}_{2,1} & \underline{W}_{2,2} & \cdots & \underline{W}_{2,N-1}%
\end{array}%
\right)
\end{equation*}%
so that $\underline{W}_{2,i}$ denotes the $i^{th}$ column of $\mathbf{W}_{2}$
for $i=1,...,N-1$. Note that, by Sylvester's determinantal identity, the
non-zero eigenvalues of $\widehat{\Sigma }_{\mathbf{W}_{\mathbf{2}}}=\mathbf{%
W}_{2}^{\prime }\mathbf{W}_{2}/T$ $\ $(i.e., $\widetilde{\lambda }_{\left(
2\right) },....,\widetilde{\lambda }_{\left( T+1\right) }$) are the same as
those of the dual matrix%
\begin{equation*}
\underset{T\times T}{\widehat{\Sigma }_{\mathbf{W}_{\mathbf{2}},D}}\text{ }=%
\frac{\mathbf{W}_{2}\mathbf{W}_{2}^{\prime }}{T}=\frac{1}{T}%
\dsum\limits_{i=1}^{N-1}\underline{W}_{2,i}\underline{W}_{2,i}^{\prime }
\end{equation*}

\noindent Now, under our assumptions, $\left\{ \mathbf{W}_{2,t,i}\right\}
\equiv i.i.d.N\left\{ 0,1\right\} $ for $t=1,...,T$ and $i=1,...,N-1$ where $%
\mathbf{W}_{2,t,i}$ denotes the $\left( t,i\right) ^{th}$ element of $%
\mathbf{W}_{2}$. Applying Lemma B-3 above with $\tau =\sqrt{T}$, we see
that, with probability at least%
\begin{equation*}
1-2\exp \left\{ -c\tau ^{2}\right\} =1-2\exp \left\{ -cT\right\} \text{,}
\end{equation*}%
the following inequality holds for any $j\in \left\{ 2,...,T+1\right\} $%
\begin{eqnarray*}
1-\max \left\{ \delta ,\delta ^{2}\right\} &\leq &\lambda _{\left( T\right)
}\left( \frac{1}{N-1}\dsum\limits_{i=1}^{N-1}\underline{W}_{2,i}\underline{W}%
_{2,i}^{\prime }\right) \\
&\leq &\lambda _{\left( j-1\right) }\left( \frac{1}{N-1}\dsum%
\limits_{i=1}^{N-1}\underline{W}_{2,i}\underline{W}_{2,i}^{\prime }\right) \\
&\leq &\lambda _{\left( 1\right) }\left( \frac{1}{N-1}\dsum%
\limits_{i=1}^{N-1}\underline{W}_{2,i}\underline{W}_{2,i}^{\prime }\right) \\
&=&1+\max \left\{ \delta ,\delta ^{2}\right\}
\end{eqnarray*}%
where%
\begin{equation*}
\delta =C\sqrt{\frac{T}{N}}+\frac{\tau }{\sqrt{N}}=\left( 1+C\right) \sqrt{%
\frac{T}{N}}
\end{equation*}%
Moreover, by our definition,%
\begin{equation*}
\widetilde{\lambda }_{\left( j\right) }=\lambda _{\left( j-1\right) }\left( 
\frac{1}{T}\dsum\limits_{i=1}^{N-1}\underline{W}_{2,i}\underline{W}%
_{2,i}^{\prime }\right) \text{,}
\end{equation*}%
so that, by multiplying and dividing by $T$, we see that%
\begin{eqnarray*}
1-\left( 1+C\right) \sqrt{\frac{T}{N}} &\leq &\frac{T}{N-1}\lambda _{\left(
j-1\right) }\left( \frac{1}{T}\dsum\limits_{i=1}^{N-1}\underline{W}_{2,i}%
\underline{W}_{2,i}^{\prime }\right) =\frac{T}{N-1}\widetilde{\lambda }%
_{\left( j\right) } \\
&\leq &1+\left( 1+C\right) \sqrt{\frac{T}{N}}
\end{eqnarray*}%
Furthermore, since the above inequality relationship above holds for any $%
j\in \left\{ 2,...,T+1\right\} $, it must be that%
\begin{equation*}
1-\left( 1+C\right) \sqrt{\frac{T}{N}}\leq \frac{T}{N-1}\max_{2\leq j\leq
T+1}\widetilde{\lambda }_{\left( j\right) }\leq 1+\left( 1+C\right) \sqrt{%
\frac{T}{N}}
\end{equation*}%
It follows that, as $N,T\rightarrow \infty $ such that $T/N\rightarrow 0$, 
\begin{equation*}
\frac{T}{N-1}\max_{2\leq j\leq T+1}\widetilde{\lambda }_{\left( j\right)
}=1+O_{p}\left( \sqrt{\frac{T}{N}}\right) =1+o_{p}\left( 1\right) \text{. }%
\square
\end{equation*}

\medskip

\noindent \textbf{Lemma B-6: }Let $X$ be a $N\times T$ random matrix, and
let $X_{it}$ be the $\left( i,t\right) ^{th}$ element of $X$. Suppose that 
\begin{equation*}
\left\{ X_{it}\right\} \equiv i.i.d.\left( 0,1\right)
\end{equation*}%
and suppose that $E\left[ X_{it}^{4}\right] <\infty $. Moreover, let%
\begin{equation*}
B=\frac{1}{N}X^{\prime }X.
\end{equation*}%
Then, as $N,T\rightarrow \infty $ such that $T/N\rightarrow c\in \left[
0,1\right) $, 
\begin{eqnarray*}
&&\lambda _{\min }\left( B\right) \overset{a.s.}{\rightarrow }\left( 1-\sqrt{%
c}\right) ^{2}\text{,} \\
&&\lambda _{\max }\left( B\right) \overset{a.s.}{\rightarrow }\left( 1+\sqrt{%
c}\right) ^{2}\text{.}
\end{eqnarray*}%
\textbf{Remark:} Lemma B-6 is a special case of Lemma 1 given in Shen, Shen,
Zhu, and Marron (2016) and is a slightly extended version of Theorem 2 of
Bai and Yin (1993). Hence, we state this result here without proof.

\medskip

\noindent \textbf{Lemma B-7: }Suppose that $\left\{ \mathbf{W}_{2,t}\right\}
\equiv i.i.d.N\left( 0,I_{N-1}\right) $. Let 
\begin{equation*}
\widetilde{\lambda }_{\left( 2\right) }\geq \widetilde{\lambda }_{\left(
3\right) }\geq \cdot \cdot \cdot \geq \widetilde{\lambda }_{\left( N\right) }
\end{equation*}%
be the $N-1$ eigenvalues of 
\begin{equation*}
\widehat{\Sigma }_{\mathbf{W}_{\mathbf{2}}}=\frac{\mathbf{W}_{2}^{\prime }%
\mathbf{W}_{2}}{T}=\frac{1}{T}\dsum\limits_{t=1}^{T}\mathbf{W}_{2,t}\mathbf{W%
}_{2,t}^{\prime }\text{.}
\end{equation*}%
where$\underset{T\times \left( N-1\right) }{\mathbf{W}_{2}}=\left( 
\begin{array}{cccc}
\underset{\left( N-1\right) \times 1}{\mathbf{W}_{2,1}} & \underset{\left(
N-1\right) \times 1}{\mathbf{W}_{2,2}} & \cdots & \underset{\left(
N-1\right) \times 1}{\mathbf{W}_{2,T}}%
\end{array}%
\right) ^{\prime }$. Then, as $N,T\rightarrow \infty $ such that $%
T/N\rightarrow 0$, 
\begin{equation*}
\frac{T}{N-1}\widetilde{\lambda }_{\left( j\right) }\overset{a.s.}{%
\rightarrow }1\text{ for any }j\in \left\{ 2,....,T+1\right\} \text{.}
\end{equation*}%
In particular, 
\begin{equation*}
\frac{T}{N-1}\max_{2\leq j\leq T+1}\widetilde{\lambda }_{\left( j\right) }%
\overset{a.s.}{\rightarrow }1
\end{equation*}%
and%
\begin{equation*}
\max_{2\leq j\leq T+1}\left\vert \frac{T}{N}\widetilde{\lambda }_{\left(
j\right) }-1\right\vert \overset{a.s.}{\rightarrow }0\text{. }
\end{equation*}

\medskip

\noindent \textbf{Proof of Lemma B-7: }

To proceed, first define the dual matrix of $\widehat{\Sigma }_{\mathbf{W}_{%
\mathbf{2}}}$ given by%
\begin{equation*}
\underset{T\times T}{\widehat{\Sigma }_{\mathbf{W}_{\mathbf{2}},D}}\text{ }=%
\frac{\mathbf{W}_{2}\mathbf{W}_{2}^{\prime }}{T}=\frac{1}{T}%
\dsum\limits_{i=1}^{N-1}\underline{W}_{2,i}\underline{W}_{2,i}^{\prime }
\end{equation*}%
where $\underline{W}_{2,i}$ denotes the $i^{th}$ column of $\mathbf{W}_{2}$
for $i=1,...,N-1$. Now, since $T/\left( N-1\right) \rightarrow 0$ and since $%
\left\{ \mathbf{W}_{2,t,i}\right\} \equiv i.i.d.N\left\{ 0,1\right\} $ for $%
t=1,...,T$ and $i=1,...,N-1$; it follows from applying Lemma B-6 that%
\begin{eqnarray}
\frac{T}{N-1}\max_{2\leq j\leq T+1}\widetilde{\lambda }_{\left( j\right) }
&=&\frac{T}{N-1}\max_{2\leq j\leq T+1}\lambda _{\left( j-1\right) }\left( 
\widehat{\Sigma }_{\mathbf{W}_{\mathbf{2}},D}\right)  \notag \\
&=&\frac{T}{N-1}\max_{2\leq j\leq T+1}\lambda _{\left( j-1\right) }\left( 
\frac{1}{T}\dsum\limits_{i=1}^{N-1}\underline{W}_{2,i}\underline{W}%
_{2,i}^{\prime }\right)  \notag \\
&=&\max_{2\leq j\leq T+1}\lambda _{\left( j-1\right) }\left( \frac{1}{N-1}%
\dsum\limits_{i=1}^{N-1}\underline{W}_{2,i}\underline{W}_{2,i}^{\prime
}\right) \overset{a.s.}{\rightarrow }1\text{ as }N,T\rightarrow \infty \text{
}  \label{T/(N-1)*maxlambdatilde}
\end{eqnarray}%
and%
\begin{eqnarray}
\frac{T}{N-1}\min_{2\leq j\leq T+1}\widetilde{\lambda }_{\left( j\right) }
&=&\frac{T}{N-1}\min_{2\leq j\leq T+1}\lambda _{\left( j-1\right) }\left( 
\widehat{\Sigma }_{\mathbf{W}_{\mathbf{2}},D}\right)  \notag \\
&=&\frac{T}{N-1}\min_{2\leq j\leq T+1}\lambda _{\left( j-1\right) }\left( 
\frac{1}{T}\dsum\limits_{i=1}^{N-1}\underline{W}_{2,i}\underline{W}%
_{2,i}^{\prime }\right)  \notag \\
&=&\min_{2\leq j\leq T+1}\lambda _{\left( j-1\right) }\left( \frac{1}{N-1}%
\dsum\limits_{i=1}^{N-1}\underline{W}_{2,i}\underline{W}_{2,i}^{\prime
}\right) \overset{a.s.}{\rightarrow }1\text{ as }N,T\rightarrow \infty \text{%
. }  \label{T/(N-1)*minlambdatilde}
\end{eqnarray}%
Expressions (\ref{T/(N-1)*maxlambdatilde}) and (\ref{T/(N-1)*minlambdatilde}%
) then imply that, for any $j\in \left\{ 2,...,T+1\right\} $,%
\begin{equation*}
\frac{T}{N-1}\widetilde{\lambda }_{\left( j\right) }\overset{a.s.}{%
\rightarrow }1\text{ as }N,T\rightarrow \infty \text{,}
\end{equation*}%
so that%
\begin{equation*}
\frac{T}{N-1}\max_{2\leq j\leq T+1}\widetilde{\lambda }_{\left( j\right) }=%
\frac{T}{N-1}\widetilde{\lambda }_{\left( 2\right) }\overset{a.s.}{%
\rightarrow }1\text{ as }N,T\rightarrow \infty \text{.}
\end{equation*}%
In addition, note that, for any $j\in \left\{ 2,...,T+1\right\} $, 
\begin{equation*}
\frac{T}{N}\widetilde{\lambda }_{\left( j\right) }=\frac{N-1}{N}\frac{T}{N-1}%
\widetilde{\lambda }_{\left( j\right) }\overset{a.s.}{\rightarrow }1\text{
as }N,T\rightarrow \infty
\end{equation*}%
from which it further follows that 
\begin{equation*}
\max_{2\leq j\leq T+1}\left\vert \frac{T}{N}\widetilde{\lambda }_{\left(
j\right) }-1\right\vert \leq \left\vert \frac{T}{N}\widetilde{\lambda }%
_{\left( 1\right) }-1\right\vert +\left\vert \frac{T}{N}\widetilde{\lambda }%
_{\left( T+1\right) }-1\right\vert \overset{a.s.}{\rightarrow }0\text{. }%
\square
\end{equation*}

\medskip

\noindent \textbf{Lemma B-8: }Consider the simple factor model

\begin{equation*}
\underset{N\times 1}{Z_{t}}=\underset{N\times 1}{\gamma }\underset{1\times 1}%
{f_{t}}+\underset{N\times 1}{u_{t}}\text{, }t=1,...,T\text{;}
\end{equation*}%
where we assume that $\left\{ u_{t}\right\} \equiv i.i.d.N\left(
0,I_{N}\right) $, $\left\{ f_{t}\right\} \equiv i.i.d.N\left( 0,1\right) $,
and $u_{s}$ and $f_{t}$ are independent for all $t,s$. Let $\Sigma _{Z}=E%
\left[ Z_{t}Z_{t}^{\prime }\right] $; then, the eigenvalues of $\Sigma _{Z}$
are given by 
\begin{equation*}
\lambda _{\left( 1\right) }=\left\Vert \gamma \right\Vert _{2}^{2}+1\text{
and }\lambda _{\left( j\right) }=1\text{ for }j=2,...,N\text{.}
\end{equation*}%
Moreover, let $\pi _{\left( 1\right) }$ $\left( N\times 1\right) $ be the
eigenvector assocated with the top eigenvalue $\lambda _{\left( 1\right) }$;
then, 
\begin{equation*}
\pi _{\left( 1\right) }=\frac{\gamma }{\left\Vert \gamma \right\Vert _{2}}%
\text{.}
\end{equation*}

\medskip

\noindent \textbf{Proof of Lemma B-8:} To show part (a), note first that 
\begin{eqnarray*}
\Sigma _{Z} &=&E\left[ Z_{t}Z_{t}^{\prime }\right] \\
&=&E\left[ \left( \gamma f_{t}+u_{t}\right) \left( \gamma ^{\prime
}f_{t}+u_{t}^{\prime }\right) \right] \\
&=&\gamma \gamma ^{\prime }+I_{N}
\end{eqnarray*}%
Consider the determinantal equation%
\begin{eqnarray*}
0 &=&\det \left\{ \lambda I_{N}-\left( \gamma \gamma ^{\prime }+I_{N}\right)
\right\} \\
&=&\det \left\{ \left( \lambda -1\right) I_{N}-\gamma \gamma ^{\prime
}\right\} \\
&=&\det \left\{ \kappa I_{N}-\gamma \gamma ^{\prime }\right\} \text{ \ }%
\left( \text{where }\kappa =\lambda -1\right) \\
&=&\kappa ^{N}\det \left\{ I_{N}-\kappa ^{-1}\gamma \gamma ^{\prime }\right\}
\\
&=&\kappa ^{N}\left( 1-\kappa ^{-1}\gamma ^{\prime }\gamma \right) \text{ }%
\left( \text{by Sylvester's determinantal theorem}\right) \\
&=&\kappa ^{N-1}\left( \kappa -\gamma ^{\prime }\gamma \right)
\end{eqnarray*}%
so the roots of this equation are%
\begin{equation*}
\kappa _{\left( 1\right) }=\gamma ^{\prime }\gamma =\left\Vert \gamma
\right\Vert _{2}^{2},\text{ }\kappa _{\left( 2\right) }=0,....,\kappa
_{\left( N\right) }=0
\end{equation*}%
and, thus,%
\begin{equation*}
\lambda _{\left( 1\right) }=\gamma ^{\prime }\gamma +1=\left\Vert \gamma
\right\Vert _{2}^{2}+1,\lambda _{\left( 2\right) }=1,....,\lambda _{\left(
N\right) }=1\text{.}
\end{equation*}%
Next, note that%
\begin{eqnarray*}
\left( \gamma \gamma ^{\prime }+I_{N}\right) \gamma &=&\left\Vert \gamma
\right\Vert _{2}^{2}\gamma +\gamma \\
&=&\left( \left\Vert \gamma \right\Vert _{2}^{2}+1\right) \gamma
\end{eqnarray*}%
so that $\gamma $ is an (unnormalized) eigenvector of the matrix $\gamma
\gamma ^{\prime }+I_{N}$ associated with the eigenvalue $\lambda _{\left(
1\right) }=\left\Vert \gamma \right\Vert _{2}^{2}+1$. It follows that we can
take%
\begin{equation*}
\pi _{\left( 1\right) }=\gamma /\left\Vert \gamma \right\Vert _{2}
\end{equation*}%
to be the (normalized) eigenvector of $\Sigma _{Z}=E\left[
Z_{t}Z_{t}^{\prime }\right] =\gamma \gamma ^{\prime }+I_{N}$ associated with
the eigenvalue%
\begin{equation*}
\lambda _{\left( 1\right) }=\left\Vert \gamma \right\Vert _{2}^{2}+1\text{. }%
\square
\end{equation*}

\medskip

\noindent \textbf{Lemma B-9: }Let $A\in M_{n}$ be a Hermetian matrix, let $r$
be an integer with $1\leq r\leq n$, and let $A_{r}$ denote any $r\times r$
principal submatrix of $A$ (obtained by deleting $n-r$ rows and the
corresponding columns of $A$). Let the eigenvalues of $A$ and $A_{r}$ be
ordered as follows%
\begin{eqnarray*}
\lambda _{\left( 1\right) }\left( A\right) &\geq &\lambda _{\left( 2\right)
}\left( A\right) \geq \cdot \cdot \cdot \geq \lambda _{\left( n\right)
}\left( A\right) , \\
\lambda _{\left( 1\right) }\left( A_{r}\right) &\geq &\lambda _{\left(
2\right) }\left( A_{r}\right) \geq \cdot \cdot \cdot \geq \lambda _{\left(
r\right) }\left( A_{r}\right) \text{.}
\end{eqnarray*}%
Then, for each integer $k$ such that $1\leq k\leq r$, we have%
\begin{equation*}
\lambda _{\left( k\right) }\left( A\right) \geq \lambda _{\left( k\right)
}\left( A_{r}\right) \geq \lambda _{\left( n-\left[ r-k\right] \right)
}\left( A\right)
\end{equation*}%
so that for $r=n-1$, we have%
\begin{equation*}
\lambda _{\left( 1\right) }\left( A\right) \geq \lambda _{\left( 1\right)
}\left( A_{n-1}\right) \geq \lambda _{\left( 2\right) }\left( A\right) \geq
\lambda _{\left( 2\right) }\left( A_{n-1}\right) \geq \cdot \cdot \cdot \geq
\lambda _{\left( n-1\right) }\left( A\right) \geq \lambda _{\left(
n-1\right) }\left( A_{n-1}\right) \geq \lambda _{\left( n\right) }\left(
A\right)
\end{equation*}

\medskip

\noindent \textbf{Proof of Lemma B-9: }This result is essentially Theorem
4.3.15 in Horn and Johnson (1985), except that we use different notations
here. A proof of this lemma can be obtained by a slight adaptation of the
proof given in Horn and Johnson (1985) for Theorem 4.3.15 using our
notations here.

\medskip \noindent

\noindent \textbf{Lemma B-10: }Let%
\begin{equation*}
\underset{N\times 1}{W_{t}}\text{ }=\dsum\limits_{j=1}^{N}\sqrt{\ell _{j}}%
\zeta _{j,t}\mathbf{e}_{j,N}
\end{equation*}%
where $\zeta _{1,t}=f_{t}+\left\Vert \gamma \right\Vert _{2}^{-1}\eta _{1t}$
and $\zeta _{j,t}=\eta _{j,t}$ for $j=2,...,N$; where $\ell _{1}=\left\Vert
\gamma \right\Vert _{2}^{2}$ and $\ell _{j}=1$ for $j=2,...,N$; and where $%
\mathbf{e}_{j,N}$ is an $N\times 1$ elementary vector whose $j^{th}$
component is $1$ and all remaining components are $0$. Suppose that $\left\{
\eta _{t}\right\} \equiv i.i.d.N\left( 0,I_{N}\right) $, $\left\{
f_{t}\right\} \equiv i.i.d.N\left( 0,1\right) $, and $f_{t}$ and $\eta _{s}$
are independent for all $t,s$. In addition, suppose that the following
assumptions hold.

\begin{enumerate}
\item[(i)] As $N\rightarrow \infty $ 
\begin{equation*}
\left\Vert \gamma \right\Vert _{2}\rightarrow \infty \text{.}
\end{equation*}

\item[(ii)] As $N,T\rightarrow \infty $%
\begin{equation*}
\frac{N}{T\left\Vert \gamma \right\Vert _{2}^{2\left( 1+\kappa \right) }}%
=c+o\left( \frac{1}{\left\Vert \gamma \right\Vert _{2}^{2}}\right) \text{,
with }0<c<\infty
\end{equation*}%
for some $\kappa $ such that $0<\kappa <1$.
\end{enumerate}

\noindent Moreover, let $\widehat{\lambda }_{\left( 1\right) }$ denote the
largest eigenvalue of the sample covariance matrix%
\begin{equation*}
\widehat{\Sigma }_{\mathbf{W}}=\frac{1}{T}\dsum\limits_{t=1}^{T}W_{t}W_{t}^{%
\prime }\text{,}
\end{equation*}%
where $\underset{N\times T}{\mathbf{W}}=\left( W_{1},...,W_{T}\right) $.
Then, as $N,T\rightarrow \infty $ such that $T/N\rightarrow 0$; the largest
sample eigenvalue $\widehat{\lambda }_{\left( 1\right) }$ satisfy%
\begin{equation*}
\frac{\widehat{\lambda }_{\left( 1\right) }}{\left\Vert \gamma \right\Vert
_{2}^{2\left( 1+\kappa \right) }}=c+\frac{1}{\left\Vert \gamma \right\Vert
_{2}^{2\kappa }}+o_{p}\left( \frac{1}{\left\Vert \gamma \right\Vert
_{2}^{2\kappa }}\right) \text{ for }0<\kappa <1\text{. }
\end{equation*}

\noindent

\noindent \textbf{Proof of Lemma B-10: }

Following Shen, Shen, Zhu, and Marron (2016), we shall study the sample
eigenvalue properties via the dual matrix%
\begin{equation*}
\underset{T\times T}{\widehat{\Sigma }_{\mathbf{W,}D}}\text{ }=\frac{1}{T}%
\mathbf{W}^{\prime }\mathbf{W}
\end{equation*}%
which shares the same\ nonzero eigenvalues with the sample covariance matrix%
\begin{equation*}
\underset{N\times N}{\widehat{\Sigma }_{\mathbf{W}}}\text{ }=\frac{1}{T}%
\mathbf{WW}^{\prime }\text{. }
\end{equation*}%
Define $\zeta _{j\cdot }=\left( 
\begin{array}{cccc}
\zeta _{j,1} & \zeta _{j,2} & \cdots & \zeta _{j,T}%
\end{array}%
\right) ^{\prime }$. Since $W_{t}=\dsum\nolimits_{j=1}^{N}\sqrt{\ell _{j}}%
\zeta _{j,t}\mathbf{e}_{j,N}$, we can write%
\begin{equation*}
\frac{1}{T}W_{t}^{\prime }W_{s}=\dsum\limits_{k=1}^{N}\dsum\limits_{\ell
=1}^{N}\ell _{k}^{1/2}\ell _{\ell }^{1/2}\zeta _{k,t}\zeta _{\ell
,s}e_{k,N}^{T}e_{\ell ,N}=\dsum\limits_{k=1}^{N}\ell _{k}\zeta _{k,t}\zeta
_{k,s}
\end{equation*}%
where%
\begin{eqnarray*}
\ell _{1} &=&\left\Vert \gamma \right\Vert _{2}^{2},\text{ }\ell _{2}=\cdot
\cdot \cdot =\ell _{N}=1 \\
\zeta _{1,t} &=&f_{t}+\frac{1}{\left\Vert \gamma \right\Vert _{2}}\eta _{1t},%
\text{ }\zeta _{2,t}=\eta _{2t},....,\zeta _{N,t}=\eta _{Nt}\text{.}
\end{eqnarray*}%
so that%
\begin{eqnarray*}
&&\widehat{\Sigma }_{\mathbf{W,}D} \\
&=&\frac{1}{T}\underset{T\times N}{\mathbf{W}^{\prime }}\underset{N\times T}{%
\mathbf{W}}=\frac{1}{T}\left( 
\begin{array}{c}
W_{1}^{\prime } \\ 
W_{2}^{\prime } \\ 
\vdots \\ 
W_{T}^{\prime }%
\end{array}%
\right) \left( 
\begin{array}{cccc}
W_{1} & W_{2} & \cdots & W_{T}%
\end{array}%
\right) \\
&=&\frac{1}{T}\left( 
\begin{array}{cccc}
W_{1}^{\prime }W_{1} & W_{1}^{\prime }W_{2} & \cdots & W_{1}^{\prime }W_{T}
\\ 
W_{2}^{\prime }W_{1} & W_{2}^{\prime }W_{2} & \cdots & W_{2}^{\prime }W_{T}
\\ 
\vdots & \vdots &  & \vdots \\ 
W_{T}^{\prime }W_{1} & W_{T}^{\prime }W_{2} & \cdots & W_{T}^{\prime }W_{T}%
\end{array}%
\right) =\frac{1}{T}\dsum\limits_{k=1}^{N}\ell _{k}\left( 
\begin{array}{cccc}
\zeta _{k,1}^{2} & \zeta _{k,1}\zeta _{k,2} & \cdots & \zeta _{k.1}{}_{k,T}
\\ 
\zeta _{k,2}\zeta _{k,1} & \zeta _{k,2}^{2} & \cdots & \zeta _{k,2}\zeta
_{k,T} \\ 
\vdots & \vdots &  & \vdots \\ 
\zeta _{k,T}\zeta _{k,1} & \zeta _{k,T}\zeta _{k,2} & \cdots & \zeta
_{k,T}^{2}%
\end{array}%
\right) \\
&=&\frac{1}{T}\dsum\limits_{k=1}^{N}\ell _{k}\left( 
\begin{array}{c}
\zeta _{k,1} \\ 
\zeta _{k,2} \\ 
\vdots \\ 
\zeta _{k,T}%
\end{array}%
\right) \left( 
\begin{array}{cccc}
\zeta _{k,1} & \zeta _{k,2} & \cdots & \zeta _{k,T}%
\end{array}%
\right) =\frac{1}{T}\dsum\limits_{k=1}^{N}\ell _{k}\zeta _{k\cdot }\zeta
_{k\cdot }^{\prime }
\end{eqnarray*}%
which can be decomposed into sum of two matrices as follows%
\begin{equation*}
\underset{T\times T}{\widehat{\Sigma }_{\mathbf{W,}D}}=A+B
\end{equation*}%
where%
\begin{equation*}
\underset{T\times T}{A}=\frac{1}{T}\ell _{1}\underset{T\times 1}{\zeta
_{1\cdot }}\underset{1\times T}{\zeta _{1\cdot }^{\prime }}=\frac{1}{T}%
\left\Vert \gamma \right\Vert _{2}^{2}\zeta _{1\cdot }\zeta _{1\cdot
}^{\prime }\text{ and }B=\frac{1}{T}\dsum\limits_{k=2}^{N}\zeta _{k\cdot
}\zeta _{k\cdot }^{\prime }\text{.}
\end{equation*}

Next, we apply Weyl's inequality (given in Lemma B-1 above) to obtain%
\begin{equation*}
\frac{\lambda _{\left( 1\right) }\left( A\right) }{\left\Vert \gamma
\right\Vert _{2}^{2}}+\frac{\lambda _{\left( T\right) }\left( B\right) }{%
\left\Vert \gamma \right\Vert _{2}^{2}}\leq \frac{\widehat{\lambda }_{\left(
1\right) }}{\left\Vert \gamma \right\Vert _{2}^{2}}=\frac{\lambda _{\left(
1\right) }\left( A+B\right) }{\left\Vert \gamma \right\Vert _{2}^{2}}\leq 
\frac{\lambda _{\left( 1\right) }\left( A\right) }{\left\Vert \gamma
\right\Vert _{2}^{2}}+\frac{\lambda _{\left( 1\right) }\left( B\right) }{%
\left\Vert \gamma \right\Vert _{2}^{2}}\text{ }
\end{equation*}%
Moreover, as $N,T\rightarrow \infty $, $\left\Vert \gamma \right\Vert
_{2}^{2}\rightarrow \infty $ under Assumption (i); whereas Assumption (ii)
states that%
\begin{equation*}
\frac{N}{T\left\Vert \gamma \right\Vert _{2}^{2\left( 1+\kappa \right) }}%
=c+o\left( \frac{1}{\left\Vert \gamma \right\Vert _{2}^{2}}\right) \text{,
with }0<c<\infty
\end{equation*}%
from which it follows that%
\begin{eqnarray}
\frac{N-1}{T\left\Vert \gamma \right\Vert _{2}^{2\left( 1+\kappa \right) }}
&=&\frac{N}{T\left\Vert \gamma \right\Vert _{2}^{2\left( 1+\kappa \right) }}%
+O\left( \frac{1}{T\left\Vert \gamma \right\Vert _{2}^{2\left( 1+\kappa
\right) }}\right)  \notag \\
&=&c+o\left( \frac{1}{\left\Vert \gamma \right\Vert _{2}^{2}}\right)
+O\left( \frac{1}{T\left\Vert \gamma \right\Vert _{2}^{2\left( 1+\kappa
\right) }}\right)  \notag \\
&=&c+o\left( \frac{1}{\left\Vert \gamma \right\Vert _{2}^{2}}\right)
\label{(N-1)/Tgamma}
\end{eqnarray}%
In addition, recall that the result of Lemma B-4 shows that, as $%
N,T\rightarrow \infty $,

\begin{equation*}
\frac{T\lambda _{\left( 1\right) }\left( B\right) }{\left( N-1\right) }%
=1+O_{p}\left( \sqrt{\frac{T}{N}}\right) \text{ and }\frac{T\lambda _{\left(
T\right) }\left( B\right) }{N-1}=1+O_{p}\left( \sqrt{\frac{T}{N}}\right)
\end{equation*}%
Hence, applying Lemma B-4 and Assumpton (ii); we obtain, as $N,T\rightarrow
\infty $%
\begin{eqnarray*}
\frac{1}{\left\Vert \gamma \right\Vert _{2}^{2\kappa }}\frac{\lambda
_{\left( 1\right) }\left( B\right) }{\left\Vert \gamma \right\Vert _{2}^{2}}
&=&\frac{\left( N-1\right) }{T\left\Vert \gamma \right\Vert _{2}^{2\left(
1+\kappa \right) }}\frac{T\lambda _{\left( 1\right) }\left( B\right) }{%
\left( N-1\right) } \\
&=&\left[ c+o\left( \frac{1}{\left\Vert \gamma \right\Vert _{2}^{2}}\right) %
\right] \left( 1+O_{p}\left( \sqrt{\frac{T}{N}}\right) \right) \\
&=&c+O_{p}\left( \sqrt{\frac{T}{N}}\right) +o\left( \frac{1}{\left\Vert
\gamma \right\Vert _{2}^{2}}\right) \text{ } \\
\frac{1}{\left\Vert \gamma \right\Vert _{2}^{2\kappa }}\frac{\lambda
_{\left( T\right) }\left( B\right) }{\left\Vert \gamma \right\Vert _{2}^{2}}
&=&\frac{\left( N-1\right) }{T\left\Vert \gamma \right\Vert _{2}^{2\left(
1+\kappa \right) }}\frac{T\lambda _{\left( T\right) }\left( B\right) }{%
\left( N-1\right) } \\
&=&\left[ c+o\left( \frac{1}{\left\Vert \gamma \right\Vert _{2}^{2}}\right) %
\right] \left( 1+O_{p}\left( \sqrt{\frac{T}{N}}\right) \right) \\
&=&c+O_{p}\left( \sqrt{\frac{T}{N}}\right) +o\left( \frac{1}{\left\Vert
\gamma \right\Vert _{2}^{2}}\right)
\end{eqnarray*}%
which, together with the inequality relationship%
\begin{equation*}
\frac{\lambda _{\left( 1\right) }\left( A\right) }{\left\Vert \gamma
\right\Vert _{2}^{2}}+\frac{\lambda _{\left( T\right) }\left( B\right) }{%
\left\Vert \gamma \right\Vert _{2}^{2}}\leq \frac{\widehat{\lambda }_{\left(
1\right) }}{\left\Vert \gamma \right\Vert _{2}^{2}}\leq \frac{\lambda
_{\left( 1\right) }\left( A\right) }{\left\Vert \gamma \right\Vert _{2}^{2}}+%
\frac{\lambda _{\left( 1\right) }\left( B\right) }{\left\Vert \gamma
\right\Vert _{2}^{2}}\text{ }
\end{equation*}%
and the fact that, by Lemma B-2,%
\begin{equation*}
\frac{\lambda _{\left( 1\right) }\left( A\right) }{\left\Vert \gamma
\right\Vert _{2}^{2}}=1+\frac{1}{\left\Vert \gamma \right\Vert _{2}^{2}}%
+O_{p}\left( \frac{1}{\sqrt{T}}\right) \text{ }
\end{equation*}%
imply that%
\begin{eqnarray*}
\frac{1}{\left\Vert \gamma \right\Vert _{2}^{2\kappa }}\frac{\lambda
_{\left( 1\right) }\left( A\right) }{\left\Vert \gamma \right\Vert _{2}^{2}}+%
\frac{1}{\left\Vert \gamma \right\Vert _{2}^{2\kappa }}\frac{\lambda
_{\left( T\right) }\left( B\right) }{\left\Vert \gamma \right\Vert _{2}^{2}}
&=&\frac{1}{\left\Vert \gamma \right\Vert _{2}^{2\kappa }}+O\left( \frac{1}{%
\left\Vert \gamma \right\Vert _{2}^{2\left( 1+\kappa \right) }}\right)
+O_{p}\left( \frac{1}{\left\Vert \gamma \right\Vert _{2}^{2\kappa }\sqrt{T}}%
\right) +c \\
&&+O_{p}\left( \sqrt{\frac{T}{N}}\right) +o\left( \frac{1}{\left\Vert \gamma
\right\Vert _{2}^{2}}\right) \\
&=&c+\frac{1}{\left\Vert \gamma \right\Vert _{2}^{2\kappa }}+o_{p}\left( 
\frac{1}{\left\Vert \gamma \right\Vert _{2}^{2\kappa }}\right) \text{ \ } \\
\frac{1}{\left\Vert \gamma \right\Vert _{2}^{2\kappa }}\frac{\lambda
_{\left( 1\right) }\left( A\right) }{\left\Vert \gamma \right\Vert _{2}^{2}}+%
\frac{1}{\left\Vert \gamma \right\Vert _{2}^{2\kappa }}\frac{\lambda
_{\left( 1\right) }\left( B\right) }{\left\Vert \gamma \right\Vert _{2}^{2}}
&=&\frac{1}{\left\Vert \gamma \right\Vert _{2}^{2\kappa }}+O\left( \frac{1}{%
\left\Vert \gamma \right\Vert _{2}^{2\left( 1+\kappa \right) }}\right)
+O_{p}\left( \frac{1}{\left\Vert \gamma \right\Vert _{2}^{2\kappa }\sqrt{T}}%
\right) +c \\
&&+O_{p}\left( \sqrt{\frac{T}{N}}\right) +o\left( \frac{1}{\left\Vert \gamma
\right\Vert _{2}^{2}}\right) \\
&=&c+\frac{1}{\left\Vert \gamma \right\Vert _{2}^{2\kappa }}+o_{p}\left( 
\frac{1}{\left\Vert \gamma \right\Vert _{2}^{2\kappa }}\right)
\end{eqnarray*}%
so that%
\begin{equation*}
\frac{\widehat{\lambda }_{\left( 1\right) }}{\left\Vert \gamma \right\Vert
_{2}^{2\left( 1+\kappa \right) }}=\frac{1}{\left\Vert \gamma \right\Vert
_{2}^{2\kappa }}\frac{\widehat{\lambda }_{\left( 1\right) }}{\left\Vert
\gamma \right\Vert _{2}^{2}}=c+\frac{1}{\left\Vert \gamma \right\Vert
_{2}^{2\kappa }}+o_{p}\left( \frac{1}{\left\Vert \gamma \right\Vert
_{2}^{2\kappa }}\right) \text{. }
\end{equation*}

\bigskip

\section{\noindent Appendix C: Lemmas used in the proofs of Theorems 4.1 and
4.2 that have been proven in Chao and Swanson (2022b).}

\noindent \textbf{Lemma C-1: }Let $a$ and $\theta $ be real numbers such
that $a>0$ and $\theta \geq 1$. Also, let $G$ be a finite non-negative
integer. Then, \textbf{\ }%
\begin{equation*}
\dsum\limits_{m=1}^{\infty }m^{G}\exp \left\{ -am^{{\large \theta }}\right\}
<\infty
\end{equation*}

\medskip

\noindent \textbf{Lemma C-2: }Let $\left\{ V_{t}\right\} $ be a sequence of
random variables (or random vectors) defined on some probability space $%
\left( \Omega ,\mathcal{F},P\right) $, and let%
\begin{equation*}
X_{t}=g\left( V_{t},V_{t-1},...,V_{t-\varkappa }\right)
\end{equation*}%
be a measurable function for some finite positive integer $\varkappa $. In
addition, defne $\mathcal{G}_{-\infty }^{t}=\sigma \left(
....,X_{t-1},X_{t}\right) $, $\mathcal{G}_{t+m}^{\infty }=\sigma \left(
X_{t+m},X_{t+m+1},....\right) $, $\mathcal{F}_{-\infty }^{t}=\sigma \left(
....,V_{t-1},V_{t}\right) $, and

\noindent $\mathcal{F}_{t+m-\varkappa }^{\infty }=\sigma \left(
V_{t+m-\varkappa },V_{t+m+1-\varkappa },....\right) $. Under this setting,
the following results hold.

\begin{enumerate}
\item[(a)] Let 
\begin{eqnarray*}
\beta _{V,m-\varkappa } &=&\sup_{t}\beta \left( \mathcal{F}_{-\infty }^{t},%
\mathcal{F}_{t+m-\varkappa }^{\infty }\right) =\sup_{t}E\left[ \sup \left\{
\left\vert P\left( B|\mathcal{F}_{-\infty }^{t}\right) -P\left( B\right)
\right\vert :B\in \mathcal{F}_{t+m-\varkappa }^{\infty }\right\} \right] , \\
\beta _{X,m} &=&\sup_{t}\beta \left( \mathcal{G}_{-\infty }^{t},\mathcal{G}%
_{t+m}^{\infty }\right) =\sup_{t}E\left[ \sup \left\{ \left\vert P\left( H|%
\mathcal{G}_{-\infty }^{t}\right) -P\left( H\right) \right\vert :H\in 
\mathcal{G}_{t+m}^{\infty }\right\} \right] \text{.}
\end{eqnarray*}%
If $\left\{ V_{t}\right\} $ is $\beta $-mixing with 
\begin{equation*}
\beta _{V,m-\varkappa }\leq \overline{C}_{1}\exp \left\{ -C_{2}\left(
m-\varkappa \right) \right\} \text{ }
\end{equation*}%
for all $m\geq \varkappa $ and for some positive constants $\overline{C}_{1}$
and $C_{2}$; then $X_{t}$ is also $\beta $-mixing with $\beta $-mixing
coefficient satisfying%
\begin{equation*}
\beta _{X,m}\leq C_{1}\exp \left\{ -C_{2}m\right\} \text{ for all }m\geq
\varkappa \text{,}
\end{equation*}%
where $C_{1}$ is a positive constant such that $C_{1}\geq \overline{C}%
_{1}\exp \left\{ C_{2}\varkappa \right\} $.

\item[(b)] Let 
\begin{eqnarray*}
\alpha _{V,m-\varkappa } &=&\sup_{t}\alpha \left( \mathcal{F}_{-\infty }^{t},%
\mathcal{F}_{t+m-\varkappa }^{\infty }\right) =\sup_{t}\sup_{G\in \mathcal{F}%
_{-\infty }^{t},H\in \mathcal{F}_{t+m-\varkappa }^{\infty }}\left\vert
P\left( G\cap H\right) -P\left( G\right) P\left( H\right) \right\vert , \\
\alpha _{X,m} &=&\sup_{t}\alpha \left( \mathcal{G}_{-\infty }^{t},\mathcal{G}%
_{t+m}^{\infty }\right) =\sup_{t}\sup_{G\in \mathcal{G}_{-\infty }^{t},H\in 
\mathcal{G}_{t+m}^{\infty }}\left\vert P\left( G\cap H\right) -P\left(
G\right) P\left( H\right) \right\vert
\end{eqnarray*}%
If $\left\{ V_{t}\right\} $ is $\alpha $-mixing with 
\begin{equation*}
\alpha _{V,m-\varkappa }\leq \overline{C}_{1}\exp \left\{ -C_{2}\left(
m-\varkappa \right) \right\} \text{ }
\end{equation*}%
for all $m\geq \varkappa $ and for some positive constants $\overline{C}_{1}$
and $C_{2}$; then $X_{t}$ is also $\alpha $-mixing with $\alpha $-mixing
coefficient satisfying%
\begin{equation*}
\alpha _{X,m}\leq C_{1}\exp \left\{ -C_{2}m\right\} \text{ for all }m\geq
\varkappa \text{,}
\end{equation*}%
where $C_{1}$ is a positive constant such that $C_{1}\geq \overline{C}%
_{1}\exp \left\{ C_{2}\varkappa \right\} $.
\end{enumerate}

\noindent \textbf{Lemma C-3: }Let $\left\{ X_{t}\right\} $ be a sequence of
random variables that is $\alpha $-mixing. Let $p>1$ and $r\geq p/\left(
p-1\right) $, and let $q=\max \left\{ p,r\right\} $. Suppose that, for all $%
t $, 
\begin{equation*}
\left\Vert X_{t}\right\Vert _{q}=\left( E\left\vert X_{t}\right\vert
^{q}\right) ^{\frac{{\large 1}}{{\large q}}}<\infty \text{ }
\end{equation*}%
Then, 
\begin{equation*}
\left\vert Cov\left( X_{t},X_{t+m}\right) \right\vert \leq 2\left(
2^{1-1/p}+1\right) \alpha _{m}^{1-1/p-1/r}\left\Vert X_{t}\right\Vert
_{p}\left\Vert X_{t+m}\right\Vert _{r}
\end{equation*}%
where%
\begin{equation*}
\alpha _{m}=\sup_{t}\alpha \left( \mathcal{F}_{-\infty }^{t},\mathcal{F}%
_{t+m}^{\infty }\right) =\sup_{G\in \mathcal{F}_{-\infty }^{t},H\in \mathcal{%
F}_{t+m}^{\infty }}\left\vert P\left( G\cap H\right) -P\left( G\right)
P\left( H\right) \right\vert \text{.}
\end{equation*}

\noindent \textbf{Lemma C-4: }Suppose that Assumption 3-3 hold. Let $\tau
_{1}=\left\lfloor T_{0}^{\alpha _{{\large 1}}}\right\rfloor $, where $%
1>\alpha _{1}>0$ and $T_{0}=T-p+1$. Then,

\begin{enumerate}
\item[(a)] 
\begin{equation*}
\frac{1}{\tau _{1}^{2}}\dsum\limits_{\substack{ g,h=\left( r-1\right) \tau
+p  \\ g\leq h}}^{\left( r-1\right) \tau +\tau _{1}+p-1}\left\vert E\left[
u_{ig}u_{ih}\right] \right\vert =O\left( \frac{1}{\tau _{1}}\right)
\end{equation*}

\item[(b)] 
\begin{equation*}
\frac{1}{\tau _{1}^{3}}\dsum\limits_{\substack{ h,v,w=\left( r-1\right) \tau
+p  \\ h\leq v\leq w}}^{\left( r-1\right) \tau +\tau _{1}+p-1}\left\vert
E\left( u_{ih}u_{iv}u_{iw}\right) \right\vert =O\left( \frac{1}{\tau _{1}^{2}%
}\right)
\end{equation*}

\item[(c)] \textbf{\ }%
\begin{equation*}
\frac{1}{\tau _{1}^{4}}\dsum\limits_{\substack{ g,h,v,w=\left( r-1\right)
\tau +p  \\ g\leq h\leq v\leq w}}^{\left( r-1\right) \tau +\tau
_{1}+p-1}\left\vert E\left[ u_{ig}u_{ih}u_{iv}u_{iw}\right] \right\vert
=O\left( \frac{1}{\tau _{1}^{2}}\right)
\end{equation*}
\end{enumerate}

\noindent \textbf{Lemma C-5:} Suppose that Assumptions 3-1, 3-2(a)-(b), 3-5,
and 3-7 hold. Then, there exists a positve constant $\overline{C}$ such that 
\begin{equation*}
E\left\Vert \underline{W}_{t}\right\Vert _{2}^{6}\leq \overline{C}<\infty 
\text{ for all }t
\end{equation*}%
and, thus,%
\begin{equation*}
E\left\Vert \underline{Y}_{t}\right\Vert _{2}^{6}\leq \overline{C}<\infty 
\text{ and }E\left\Vert \underline{F}_{t}\right\Vert _{2}^{6}\leq \overline{C%
}<\infty \text{ for all }t\text{,}
\end{equation*}%
where%
\begin{equation*}
\underset{dp\times 1}{\underline{Y}_{t}}=\left( 
\begin{array}{c}
Y_{t} \\ 
Y_{t-1} \\ 
\vdots \\ 
Y_{t-p{\LARGE +}1}%
\end{array}%
\right) \text{, and}\underset{Kp\times 1}{\underline{F}_{t}}=\left( 
\begin{array}{c}
F_{t} \\ 
F_{t-1} \\ 
\vdots \\ 
F_{t-p{\LARGE +}1}%
\end{array}%
\right) \text{. }
\end{equation*}

\noindent \textbf{Lemma C-6: }Suppose that Assumptions 3-1, 3-2(a)-(b),
3-3(a)-(c), 3-5, 3-7, and 3-10(b) hold. Then, the following statements are
true as $N_{1},T\rightarrow \infty $

\begin{enumerate}
\item[(a)] 
\begin{equation*}
\max_{1\leq \ell \leq d}\max_{i\in H^{{\large c}}}\left\vert \frac{1}{q}%
\dsum\limits_{r=1}^{q}\frac{1}{\tau _{1}}\dsum\limits_{t=\left( r-1\right)
\tau {\LARGE +}p}^{\left( r-1\right) \tau +\tau _{1}{\LARGE +}p-1}\gamma
_{i}^{\prime }\underline{F}_{t}\varepsilon _{\ell ,t{\LARGE +}1}\right\vert 
\overset{p}{\rightarrow }0\text{. }
\end{equation*}

\item[(b)] 
\begin{equation*}
\max_{1\leq \ell \leq d}\max_{i\in H^{{\large c}}}\frac{1}{q}%
\dsum\limits_{r=1}^{q}\left( \frac{1}{\tau _{1}}\dsum\limits_{t=\left(
r-1\right) \tau +p}^{\left( r-1\right) \tau +\tau _{1}+p-1}\gamma
_{i}^{\prime }\underline{F}_{t}\varepsilon _{\ell ,t{\LARGE +}1}\right) ^{2}%
\overset{p}{\rightarrow }0
\end{equation*}

\item[(c)] 
\begin{equation*}
\max_{1\leq \ell \leq d}\max_{i\in H^{{\large c}}}\left\vert \frac{1}{q}%
\dsum\limits_{r=1}^{q}\frac{1}{\tau _{1}}\dsum\limits_{t=\left( r-1\right)
\tau {\LARGE +}p}^{\left( r-1\right) \tau +\tau _{1}{\LARGE +}p-1}y_{\ell ,t%
{\LARGE +}1}u_{it}\right\vert \overset{p}{\rightarrow }0\text{.}
\end{equation*}

\item[(d)] 
\begin{equation*}
\max_{1\leq \ell \leq d}\max_{i\in H^{{\large c}}}\frac{1}{q}%
\dsum\limits_{r=1}^{q}\left( \frac{1}{\tau _{1}}\dsum\limits_{t=\left(
r-1\right) \tau +p}^{\left( r-1\right) \tau +\tau _{1}+p-1}y_{\ell ,t{\LARGE %
+}1}u_{it}\right) ^{2}\overset{p}{\rightarrow }0
\end{equation*}

\item[(e)] 
\begin{equation*}
\max_{1\leq \ell \leq d}\max_{i\in H^{{\large c}}}\left\vert \frac{1}{q}%
\dsum\limits_{r=1}^{q}\left( \frac{1}{\tau _{1}}\dsum\limits_{t=\left(
r-1\right) \tau +p}^{\left( r-1\right) \tau +\tau _{1}+p-1}\gamma
_{i}^{\prime }\underline{F}_{t}\varepsilon _{\ell ,t{\LARGE +}1}\right)
\left( \frac{1}{\tau _{1}}\dsum\limits_{t=\left( r-1\right) \tau +p}^{\left(
r-1\right) \tau +\tau _{1}+p-1}y_{\ell ,t{\LARGE +}1}u_{it}\right)
\right\vert \overset{p}{\rightarrow }0
\end{equation*}
\end{enumerate}

\noindent \textbf{Lemma C-7:} Suppose that Assumptions 3-1 and 3-7 hold.
Then, the following statements are true.

\begin{enumerate}
\item[(a)] There exists a positive constant $C^{\dagger }$ such that 
\begin{equation*}
\left\Vert A_{YY}\right\Vert _{2}\leq C^{\dagger }\phi _{\max }
\end{equation*}%
where $\phi _{\max }=\max \left\{ \left\vert \lambda _{\max }\left( A\right)
\right\vert ,\left\vert \lambda _{\min }\left( A\right) \right\vert \right\} 
$ with $0<\phi _{\max }<1$.

\item[(b)] There exists a positive constant $C^{\dagger }$ such that 
\begin{equation*}
\left\Vert A_{YF}\right\Vert _{2}\leq C^{\dagger }\phi _{\max }
\end{equation*}%
where $\phi _{\max }$ is as defined in part (a).
\end{enumerate}

\noindent \textbf{Lemma C-8: }Consider the linear process%
\begin{equation*}
\xi _{t}=\dsum\limits_{j=0}^{\infty }\Psi _{j}\varepsilon _{t-j}
\end{equation*}%
Suppose the process satisfies the following assumptions

\begin{enumerate}
\item[(i)] Let $\left\{ \varepsilon _{t}\right\} $ is an independent
sequence of random vectors with $E\left[ \varepsilon _{t}\right] =0$ for all 
$t$. For some $\delta >0$, suppose that there exists a positive constant $K$
such that 
\begin{equation*}
E\left\Vert \varepsilon _{t}\right\Vert _{2}^{1+\delta }\leq K<\infty \text{
for all }t.
\end{equation*}

\item[(ii)] Suppose that $\varepsilon _{t}$ has p.d.f. $g_{\varepsilon _{t}}$
such that, for some positive constant $M<\infty $,%
\begin{equation*}
\sup_{t}\dint \left\vert g_{\varepsilon _{t}}\left( \upsilon -u\right)
-g_{\varepsilon _{t}}\left( \upsilon \right) \right\vert d\varepsilon \leq
M\left\vert u\right\vert
\end{equation*}%
whenever $\left\vert u\right\vert \leq \overline{\kappa }$ for some constant 
$\overline{\kappa }>0$.

\item[(iii)] Suppose that%
\begin{equation*}
\dsum\limits_{j=0}^{\infty }\left\Vert \Psi _{j}\right\Vert _{2}<\infty
\end{equation*}%
and%
\begin{equation*}
\det \left\{ \dsum\limits_{j=0}^{\infty }\Psi _{j}z^{j}\right\} \neq 0\text{
for all }z\text{ with }\left\vert z\right\vert \leq 1
\end{equation*}
\end{enumerate}

Under these conditions, suppose further that%
\begin{equation*}
\dsum\limits_{j=0}^{\infty }\left( \dsum\limits_{k=j}^{\infty }\left\Vert
\Psi _{j}\right\Vert _{2}\right) ^{\frac{{\LARGE \delta }}{{\LARGE 1+\delta }%
}}<\infty \text{;}
\end{equation*}%
then, for some positive constant $\overline{K}$, 
\begin{equation*}
\beta _{\xi }\left( m\right) \leq \overline{K}\dsum\limits_{j=m}^{\infty
}\left( \dsum\limits_{k=j}^{\infty }\left\Vert \Psi _{k}\right\Vert
_{2}\right) ^{\frac{{\LARGE \delta }}{{\LARGE 1+\delta }}}
\end{equation*}%
where%
\begin{equation*}
\beta _{\xi }\left( m\right) =\sup_{t}E\left[ \sup \left\{ \left\vert
P\left( B|\mathcal{F}_{\xi ,-\infty }^{t}\right) -P\left( B\right)
\right\vert :B\in \mathcal{F}_{\xi ,t+m}^{\infty }\right\} \right] \text{.}
\end{equation*}%
with $\mathcal{F}_{\xi ,-\infty }^{t}=\sigma \left( ....,\xi _{t-2},\xi
_{t-1},\xi _{t}\right) $ and $\mathcal{F}_{\xi ,t+m}^{\infty }=\sigma \left(
\xi _{t+m},\xi _{t+m+1},\xi _{t+m+2},....\right) $.

\noindent \textbf{Lemma C-9: }Let $A$ be an $n\times n$ square matrix with
(ordered) singular values given by%
\begin{equation*}
\sigma _{\left( 1\right) }\left( A\right) \geq \sigma _{\left( 2\right)
}\left( A\right) \geq \cdot \cdot \cdot \geq \sigma _{\left( n\right)
}\left( A\right) \geq 0\text{.}
\end{equation*}%
Suppose that $A$ is diagonalizable, i.e.,%
\begin{equation*}
A=S\Lambda S^{-1}
\end{equation*}%
where $\Lambda $ is diagonal matrix whose diagonal elements are the
eigenvalues of $A$. Let the modulus of these eigenvalues be ordered as
follows: 
\begin{equation*}
\left\vert \lambda _{\left( 1\right) }\left( A\right) \right\vert \geq
\left\vert \lambda _{\left( 2\right) }\left( A\right) \right\vert \geq \cdot
\cdot \cdot \geq \left\vert \lambda _{\left( n\right) }\left( A\right)
\right\vert \text{.}
\end{equation*}%
Then, for $k\in \left\{ 1,...,n\right\} $ and for any positive integer $j$,
we have%
\begin{equation*}
\chi \left( S\right) ^{-1}\left\vert \lambda _{\left( k\right) }\left(
A^{j}\right) \right\vert \leq \sigma _{\left( k\right) }\left( A^{j}\right)
\leq \chi \left( S\right) \left\vert \lambda _{\left( k\right) }\left(
A^{j}\right) \right\vert \text{ }
\end{equation*}%
where%
\begin{equation*}
\chi \left( S\right) =\sigma _{\left( 1\right) }\left( S\right) \sigma
_{\left( 1\right) }\left( S^{-1}\right) \text{.}
\end{equation*}

\noindent \textbf{Lemma C-10: }Let $\rho $ be such that $\left\vert \rho
\right\vert <1$. Then,%
\begin{equation*}
\dsum\limits_{j=0}^{\infty }\left( j+1\right) \rho ^{j}=\frac{1}{\left(
1-\rho \right) ^{2}}<\infty
\end{equation*}

\noindent \textbf{Lemma C-11: }Let $W_{t}=\left( Y_{t}^{\prime
},F_{t}^{\prime }\right) ^{\prime }$ be generated by the factor-augmented
VAR process%
\begin{equation*}
W_{t+1}=\mu +A_{1}W_{t}+\cdot \cdot \cdot +A_{p}W_{t-p+1}+\varepsilon _{t+1}
\end{equation*}%
described in section 3 of the main paper. Under Assumptions 3-1, 3-2(a)-(c),
and 3-7; $\left\{ W_{t}\right\} $ is a $\beta $-mixing process with $\beta $%
-mixing coefficient $\beta _{W}\left( m\right) $ such that%
\begin{equation*}
\beta _{W}\left( m\right) \leq C_{1}\exp \left\{ -C_{2}m\right\}
\end{equation*}%
for some positive constants $C_{1}$ and $C_{2}$. Here, 
\begin{equation*}
\beta _{W}\left( m\right) =\sup_{t}E\left[ \sup \left\{ \left\vert P\left( B|%
\mathcal{A}_{-\infty }^{t}\right) -P\left( B\right) \right\vert :B\in 
\mathcal{A}_{t+m}^{\infty }\right\} \right]
\end{equation*}%
with $\mathcal{A}_{-\infty }^{t}=\sigma \left(
...,W_{t-2},W_{t-1},W_{t}\right) $ and $\mathcal{A}_{t+m}^{\infty }=\sigma
\left( W_{t+m},W_{t+m+1},W_{t+m+2},....\right) $.

\noindent \noindent \textbf{Lemma C-12: }Let $\underline{Y}_{t}=\left( 
\begin{array}{ccccc}
Y_{t}^{\prime } & Y_{t-1}^{\prime } & \cdots & Y_{t-p{\LARGE +}2}^{\prime }
& Y_{t-p{\LARGE +}1}^{\prime }%
\end{array}%
\right) ^{\prime }$ and

\noindent $\underline{F}_{t}=\left( 
\begin{array}{ccccc}
F_{t}^{\prime } & F_{t-1}^{\prime } & \cdots & F_{t-p{\LARGE +}2}^{\prime }
& F_{t-p{\LARGE +}1}^{\prime }%
\end{array}%
\right) ^{\prime }$. Under Assumptions 3-1, 3-2(a)-(c), 3-5, 3-7, and
3-10(b); the following statements are true as $N,T\rightarrow \infty $

\begin{enumerate}
\item[(a)] 
\begin{equation*}
\max_{1\leq \ell \leq d}\max_{i\in H^{{\large c}}}\left\vert \frac{1}{q}%
\dsum\limits_{r=1}^{q}\frac{1}{\tau _{1}}\dsum\limits_{t=\left( r-1\right)
\tau +p}^{\left( r-1\right) \tau +\tau _{1}+p-1}\gamma _{i}^{\prime }\left( 
\underline{F}_{t}\underline{Y}_{t}^{\prime }-E\left[ \underline{F}_{t}%
\underline{Y}_{t}^{\prime }\right] \right) \alpha _{YY,\ell }\right\vert 
\overset{p}{\rightarrow }0
\end{equation*}

\item[(b)] 
\begin{equation*}
\max_{1\leq \ell \leq d}\max_{i\in H^{{\large c}}}\left\vert \frac{1}{q}%
\dsum\limits_{r=1}^{q}\frac{1}{\tau _{1}}\dsum\limits_{t=\left( r-1\right)
\tau +p}^{\left( r-1\right) \tau +\tau _{1}+p-1}\gamma _{i}^{\prime }\left( 
\underline{F}_{t}\underline{F}_{t}^{\prime }-E\left[ \underline{F}_{t}%
\underline{F}_{t}^{\prime }\right] \right) \alpha _{YF,\ell }\right\vert 
\overset{p}{\rightarrow }0
\end{equation*}

\item[(c)] 
\begin{equation*}
\max_{1\leq \ell \leq d}\max_{i\in H^{{\large c}}}\left\vert \frac{1}{q}%
\dsum\limits_{r=1}^{q}\frac{1}{\tau _{1}}\dsum\limits_{t=\left( r-1\right)
\tau +p}^{\left( r-1\right) \tau +\tau _{1}+p-1}\gamma _{i}^{\prime }\left( 
\underline{F}_{t}-E\left[ \underline{F}_{t}\right] \right) \mu _{Y,\ell
}\right\vert \overset{p}{\rightarrow }0
\end{equation*}

\item[(d)] 
\begin{eqnarray*}
&&\max_{1\leq \ell \leq d}\max_{i\in H^{{\large c}}}\frac{1}{q}%
\dsum\limits_{r=1}^{q}\left( \frac{1}{\tau _{1}}\dsum\limits_{t=\left(
r-1\right) \tau +p}^{\left( r-1\right) \tau +\tau _{1}+p-1}\gamma
_{i}^{\prime }\left\{ \left( \underline{F}_{t}-E\left[ \underline{F}_{t}%
\right] \right) \mu _{Y,\ell }+\left( \underline{F}_{t}\underline{Y}%
_{t}^{\prime }-E\left[ \underline{F}_{t}\underline{Y}_{t}^{\prime }\right]
\right) \alpha _{YY,\ell }\right. \right. \\
&&\text{ \ \ \ \ \ \ \ \ \ \ \ \ \ \ \ \ \ \ \ \ \ \ \ \ \ \ \ \ \ \ \ \ \ \
\ \ \ \ \ \ \ \ \ \ \ \ \ \ \ \ }\left. \text{\ }\left. +\left( \underline{F}%
_{t}\underline{F}_{t}^{\prime }-E\left[ \underline{F}_{t}\underline{F}%
_{t}^{\prime }\right] \right) \alpha _{YF,\ell }\right\} \right) ^{2} \\
&&\overset{p}{\rightarrow }0
\end{eqnarray*}

\item[(e)] There exists a positive constant $\overline{C}$ such that%
\begin{eqnarray*}
&&\max_{1\leq \ell \leq d}\max_{i\in H^{{\large c}}}\left( \frac{\pi
_{i,\ell ,T}}{q\tau _{1}^{2}}\right) \\
&=&\max_{1\leq \ell \leq d}\max_{i\in H^{{\large c}}}\frac{1}{q}%
\dsum\limits_{r=1}^{q}\left( \frac{1}{\tau _{1}}\dsum\limits_{t=\left(
r-1\right) \tau +p}^{\left( r-1\right) \tau +\tau _{1}+p-1}\gamma
_{i}^{\prime }\left\{ E\left[ \underline{F}_{t}\right] \mu _{Y,\ell }+E\left[
\underline{F}_{t}\underline{Y}_{t}^{\prime }\right] \alpha _{YY,\ell }+E%
\left[ \underline{F}_{t}\underline{F}_{t}^{\prime }\right] \alpha _{YF,\ell
}\right\} \right) ^{2} \\
&=&\max_{1\leq \ell \leq d}\max_{i\in H^{{\large c}}}\frac{1}{q}%
\dsum\limits_{r=1}^{q}\left( \frac{1}{\tau _{1}}\dsum\limits_{t=\left(
r-1\right) \tau +p}^{\left( r-1\right) \tau +\tau _{1}+p-1}\gamma
_{i}^{\prime }E\left[ \underline{F}_{t}y_{\ell ,t{\LARGE +}1}\right] \right)
^{2} \\
&\leq &\overline{C}<\infty
\end{eqnarray*}

\item[(f)] 
\begin{equation*}
\max_{1\leq \ell \leq d}\max_{i\in H^{{\large c}}}\frac{1}{q}%
\dsum\limits_{r=1}^{q}\left( \frac{1}{\tau _{1}}\dsum\limits_{t=\left(
r-1\right) \tau +p}^{\left( r-1\right) \tau +\tau _{1}+p-1}\gamma
_{i}^{\prime }\underline{F}_{t}\left[ \mu _{Y,\ell }+\underline{Y}%
_{t}^{\prime }\alpha _{YY,\ell }+\underline{F}_{t}^{\prime }\alpha _{YF,\ell
}\right] \right) ^{2}=O_{p}\left( 1\right) \text{.}
\end{equation*}

\item[(g)] 
\begin{eqnarray*}
&&\max_{1\leq \ell \leq d}\max_{i\in H^{{\large c}}}\left\vert \frac{1}{q}%
\dsum\limits_{r=1}^{q}\left\{ \left( \frac{1}{\tau _{1}}\dsum\limits_{t=%
\left( r-1\right) \tau +p}^{\left( r-1\right) \tau +\tau _{1}+p-1}\gamma
_{i}^{\prime }\left( \underline{F}_{t}-E\left[ \underline{F}_{t}\right]
\right) \mu _{Y,\ell }\right. \right. \right. \\
&&\text{ \ \ \ \ }\left. \left. +\frac{1}{\tau _{1}}\dsum\limits_{t=\left(
r-1\right) \tau +p}^{\left( r-1\right) \tau +\tau _{1}+p-1}\left\{ \gamma
_{i}^{\prime }\left( \underline{F}_{t}\underline{Y}_{t}^{\prime }-E\left[ 
\underline{F}_{t}\underline{Y}_{t}^{\prime }\right] \right) \alpha _{YY,\ell
}+\gamma _{i}^{\prime }\left( \underline{F}_{t}\underline{F}_{t}^{\prime }-E%
\left[ \underline{F}_{t}\underline{F}_{t}^{\prime }\right] \right) \alpha
_{YF,\ell }\right\} \right\} \right) \\
&&\text{ \ \ \ }\left. \text{\ }\left. \times \left( \frac{1}{\tau _{1}}%
\dsum\limits_{t=\left( r-1\right) \tau +p}^{\left( r-1\right) \tau +\tau
_{1}+p-1}\left\{ \gamma _{i}^{\prime }E\left[ \underline{F}_{t}\right] \mu
_{Y,\ell }+\gamma _{i}^{\prime }E\left[ \underline{F}_{t}\underline{Y}%
_{t}^{\prime }\right] \alpha _{YY,\ell }+\gamma _{i}^{\prime }E\left[ 
\underline{F}_{t}\underline{F}_{t}^{\prime }\right] \alpha _{YF,\ell
}\right\} \right) \right\} \right\vert \\
&&\overset{p}{\rightarrow }0
\end{eqnarray*}

\item[(h)] 
\begin{eqnarray*}
&&\max_{1\leq \ell \leq d}\max_{i\in H^{{\large c}}}\left\vert \frac{1}{q}%
\dsum\limits_{r=1}^{q}\left( \frac{1}{\tau _{1}}\dsum\limits_{t=\left(
r-1\right) \tau +p}^{\left( r-1\right) \tau +\tau _{1}+p-1}\gamma
_{i}^{\prime }\underline{F}_{t}\left[ \mu _{Y,\ell }+\underline{Y}%
_{t}^{\prime }\alpha _{YY,\ell }+\underline{F}_{t}^{\prime }\alpha _{YF,\ell
}\right] \right) \right. \\
&&\text{ \ \ \ \ \ \ \ \ \ \ \ \ \ \ \ \ \ \ \ \ \ \ \ }\left. \times \left( 
\frac{1}{\tau _{1}}\dsum\limits_{t=\left( r-1\right) \tau +p}^{\left(
r-1\right) \tau +\tau _{1}+p-1}y_{\ell ,t{\LARGE +}1}u_{it}\right)
\right\vert \\
&&\overset{p}{\rightarrow }0
\end{eqnarray*}

\item[(i)] 
\begin{eqnarray*}
&&\max_{1\leq \ell \leq d}\max_{i\in H^{{\large c}}}\left\vert \frac{1}{q}%
\dsum\limits_{r=1}^{q}\left( \frac{1}{\tau _{1}}\dsum\limits_{t=\left(
r-1\right) \tau +p}^{\left( r-1\right) \tau +\tau _{1}+p-1}\gamma
_{i}^{\prime }\underline{F}_{t}\left[ \mu _{Y,\ell }+\underline{Y}%
_{t}^{\prime }\alpha _{YY,\ell }+\underline{F}_{t}^{\prime }\alpha _{YF,\ell
}\right] \right) \right. \\
&&\text{ \ \ \ \ \ \ \ \ \ \ \ \ \ \ \ \ \ \ \ \ \ }\left. \text{\ \ }\times
\left( \frac{1}{\tau _{1}}\dsum\limits_{t=\left( r-1\right) \tau +p}^{\left(
r-1\right) \tau +\tau _{1}+p-1}\gamma _{i}^{\prime }\underline{F}%
_{t}\varepsilon _{\ell ,t{\LARGE +}1}\right) \right\vert \\
&&\overset{p}{\rightarrow }0
\end{eqnarray*}
\end{enumerate}

\noindent \textbf{Lemma C-13: }Suppose that Assumptions 3-1, 3-2(a)-(c),
3-3(a)-(c), 3-5, 3-7, and 3-9 hold and suppose that $N_{1},N_{2},T%
\rightarrow \infty $ such that $N_{1}/\tau _{1}^{3}=N_{1}/\left\lfloor
T_{0}^{\alpha _{{\large 1}}}\right\rfloor ^{3}\rightarrow 0$. Then, the
following statements are true.

\begin{enumerate}
\item[(a)] 
\begin{equation*}
\max_{1\leq \ell \leq d}\max_{i{\large \in }H^{{\large c}}}\left\vert \frac{%
\overline{S}_{i,\ell ,T}-\mu _{i,\ell ,T}}{\mu _{i,\ell ,T}}\right\vert 
\overset{p}{\rightarrow }0
\end{equation*}

\item[(b)] 
\begin{equation*}
\max_{1\leq \ell \leq d}\max_{i{\large \in }H^{{\large c}}}\left\vert \frac{%
\overline{V}_{i,\ell ,T}-\pi _{i,\ell ,T}}{\pi _{i,\ell ,T}}\right\vert 
\overset{p}{\rightarrow }0
\end{equation*}
\end{enumerate}

\noindent \textbf{Lemma C-14: }Let $a,b\in \mathbb{R}$ such that $a\geq 0$
and $b\geq 0$. Then,%
\begin{equation*}
\left\vert \sqrt{a}-\sqrt{b}\right\vert \leq \sqrt{\left\vert a-b\right\vert 
}
\end{equation*}

\noindent \textbf{Lemma C-15:}%
\begin{equation*}
P\left\{ \dbigcap\limits_{i=1}^{m}A_{i}\right\} \geq
\dsum\limits_{i=1}^{m}P\left( A_{i}\right) -\left( m-1\right)
\end{equation*}

\noindent \textbf{Lemma C-16: }

\begin{enumerate}
\item[(a)] For $t>0$,%
\begin{equation*}
\overline{\Phi }\left( t\right) =1-\Phi \left( t\right) \leq \frac{\phi
\left( t\right) }{t},
\end{equation*}%
where $\phi \left( t\right) $ and $\Phi \left( t\right) $ denote,
respectively, the pdf and the cdf of a standard normal random variable.

\item[(b)] Let $N=N_{1}+N_{2}$. Specify $\varphi $ such that $\varphi
\rightarrow 0$ as $N_{1},N_{2}\rightarrow \infty $ and such that, for some
constant $a>0$,%
\begin{equation*}
\varphi \geq \frac{1}{N^{a}}
\end{equation*}%
for all $N_{1},N_{2}$ sufficiently large. Then, for all $N_{1},N_{2}$
sufficiently large such that 
\begin{equation*}
1-\frac{\varphi }{2N}\geq \Phi \left( 2\right)
\end{equation*}%
we have%
\begin{equation*}
\Phi ^{-1}\left( 1-\frac{\varphi }{2N}\right) \leq \sqrt{2\left( 1+a\right) }%
\sqrt{\ln N}\text{.}
\end{equation*}
\end{enumerate}

\noindent \textbf{Lemma C-17: }Suppose that Assumptions 3-1, 3-2(a)-(c),
3-3(a)-(c) 3-4, 3-5, 3-7, and 3-8 hold. Let $\Phi \left( \cdot \right) $
denote the cumulative distribution function of the standard normal random
variable. Then, there exists a positive constant $A$ such that 
\begin{equation}
P\left( \left\vert S_{i,\ell ,T}\right\vert \geq z\right) \leq 2\left[
1-\Phi \left( z\right) \right] \left\{ 1+A\left( 1+z\right) ^{3}T^{-\left(
1-\alpha _{{\large 1}}\right) \frac{{\large 1}}{{\large 2}}}\right\}
\label{moderate deviation bd}
\end{equation}%
for%
\begin{equation*}
i\in H=\left\{ k\in \left\{ 1,....,N\right\} :\gamma _{k}=0\right\} \text{,}
\end{equation*}%
for $\ell \in \left\{ 1,...,d\right\} $, for $T$ sufficiently large, and for
all $z$ such that%
\begin{equation*}
0\leq z\leq c_{0}\min \left\{ T^{\left( 1-\alpha _{{\large 1}}\right) \frac{%
{\large 1}}{{\large 6}}},T^{\frac{{\Large \alpha }_{{\large 2}}}{{\large 2}}%
}\right\}
\end{equation*}%
with $c_{0}$ being a positive constant.

\section{\noindent Appendix D: Additional Lemmas Used in the Proofs of
Theorems 4.1 and 4.2}

\qquad \medskip \noindent \noindent \noindent

\noindent \textbf{Derivation of the }$h$\textbf{-step Ahead Forecasting
Equation Given in Expression (22) of the Main Paper:}

\textbf{\ }Consider the FAVAR process%
\begin{equation}
W_{t+1}=\mu +A_{1}W_{t}+\cdot \cdot \cdot +A_{p}W_{t-p+1}+\varepsilon _{t+1}%
\text{,}  \label{FAVAR Appendix}
\end{equation}%
where $W_{t}=\left( Y_{t}^{\prime },F_{t}^{\prime }\right) ^{\prime }$.
Suppose that equation (\ref{FAVAR Appendix}) satisfies Assumptions 3-1 and
3-2 of the main paper. Then, similar to a VAR process, we can rewrite this
model in the companion form%
\begin{equation*}
\underline{W}_{t}=\alpha +A\underline{W}_{t-1}+E_{t}
\end{equation*}%
where%
\begin{eqnarray}
\underline{W}_{t} &=&\left( 
\begin{array}{c}
W_{t} \\ 
W_{t-1} \\ 
\vdots \\ 
W_{t-p{\LARGE +}2} \\ 
W_{t-p{\large +}1}%
\end{array}%
\right) ,\text{ }W_{t}=\left( 
\begin{array}{c}
Y_{t} \\ 
F_{t}%
\end{array}%
\right) ,\text{ }E_{t}=\left( 
\begin{array}{c}
\varepsilon _{t} \\ 
0 \\ 
\vdots \\ 
0 \\ 
0%
\end{array}%
\right) \text{, }\alpha =\left( 
\begin{array}{c}
\mu \\ 
0 \\ 
\vdots \\ 
0 \\ 
0%
\end{array}%
\right) \text{, and }  \notag \\
A &=&\left( 
\begin{array}{ccccc}
A_{1} & A_{2} & \cdots & A_{p-1} & A_{p} \\ 
I_{d{\LARGE +}K} & 0 & \cdots & \cdots & 0 \\ 
0 & \ddots & \ddots &  & \vdots \\ 
\vdots & \ddots & \ddots & \ddots & \vdots \\ 
0 & \cdots & 0 & I_{d{\LARGE +}K} & 0%
\end{array}%
\right) \text{.}  \label{companion form Appendix}
\end{eqnarray}%
Successive substitution for the lagged\ $\underline{W}_{t}$\ 's gives%
\begin{equation*}
\underline{W}_{t{\LARGE +}h}=\dsum\limits_{j=0}^{h-1}A^{j}\alpha +A^{h}%
\underline{W}_{t}+\dsum\limits_{j=0}^{h-1}A^{j}E_{t{\LARGE +}h-j}
\end{equation*}%
Let%
\begin{equation*}
\underset{d\times \left( d{\LARGE +}K\right) p}{J_{d}}=\left[ 
\begin{array}{cccc}
I_{d} & 0 & \cdots & 0%
\end{array}%
\right] \text{ and }\underset{\left( d{\LARGE +}K\right) \times \left( d%
{\LARGE +}K\right) p}{J_{d{\LARGE +}K}}=\left[ 
\begin{array}{cccc}
I_{d{\LARGE +}K} & 0 & \cdots & 0%
\end{array}%
\right]
\end{equation*}%
and note that%
\begin{equation*}
J_{d}\underline{W}_{t{\LARGE +}h}=Y_{t{\LARGE +}h}\text{, \ }J_{d{\LARGE +}%
K}E_{t{\LARGE +}h-j}=\varepsilon _{t{\LARGE +}h-j}\text{,}
\end{equation*}%
and%
\begin{equation*}
J_{d{\LARGE +}K}^{\prime }J_{d{\LARGE +}K}E_{t{\LARGE +}h-j}=\left( 
\begin{array}{cccc}
I_{d{\LARGE +}K} & 0 & \cdots & 0 \\ 
0 & 0 & \cdots & 0 \\ 
\vdots & \vdots & \ddots & \vdots \\ 
0 & 0 & \cdots & 0%
\end{array}%
\right) \left( 
\begin{array}{c}
\varepsilon _{t{\LARGE +}h-j} \\ 
0 \\ 
\vdots \\ 
0 \\ 
0%
\end{array}%
\right) =\left( 
\begin{array}{c}
\varepsilon _{t{\LARGE +}h-j} \\ 
0 \\ 
\vdots \\ 
0 \\ 
0%
\end{array}%
\right)
\end{equation*}%
Hence,%
\begin{eqnarray}
Y_{t{\LARGE +}h} &=&J_{d}\underline{W}_{t{\LARGE +}h}  \notag \\
&=&\dsum\limits_{j=0}^{h-1}J_{d}A^{j}\alpha +J_{d}A^{h}\underline{W}%
_{t}+\dsum\limits_{j=0}^{h-1}J_{d}A^{j}J_{d{\LARGE +}K}^{\prime }J_{d{\LARGE %
+}K}E_{t{\LARGE +}h-j}  \notag \\
&=&\dsum\limits_{j=0}^{h-1}J_{d}A^{j}\alpha +J_{d}A^{h}\underline{W}%
_{t}+\dsum\limits_{j=0}^{h-1}J_{d}A^{j}J_{d{\LARGE +}K}^{\prime }\varepsilon
_{t{\LARGE +}h-j}  \label{h step ahead representation Yt}
\end{eqnarray}%
Furthermore, let $\mathcal{P}_{\left( d{\LARGE +}K\right) p}$ be a
permutation matrix such that%
\begin{equation}
\mathcal{P}_{\left( d{\LARGE +}K\right) p}\underline{W}_{t}=\left( 
\begin{array}{c}
\underline{Y}_{t} \\ 
\underline{F}_{t}%
\end{array}%
\right) \text{, where }\underline{Y}_{t}=\left( 
\begin{array}{c}
Y_{t} \\ 
\vdots \\ 
Y_{t-p{\LARGE +}1}%
\end{array}%
\right) \text{ and }\underline{F}_{t}=\left( \text{ }%
\begin{array}{c}
F_{t} \\ 
\vdots \\ 
F_{t-p{\large +}1}%
\end{array}%
\right) \text{.}  \label{repermutated Wbar}
\end{equation}%
and note that $\mathcal{P}_{\left( d{\LARGE +}K\right) p}$ is an orthogonal
matrix, so that $\mathcal{P}_{\left( d{\LARGE +}K\right) p}^{\prime }%
\mathcal{P}_{\left( d{\LARGE +}K\right) p}=I_{\left( d{\LARGE +}K\right) p}=%
\mathcal{P}_{\left( d{\LARGE +}K\right) p}\mathcal{P}_{\left( d{\LARGE +}%
K\right) p}^{\prime }$. Next, for $g=1,...,p$, let $e_{g,p}$ be a $p\times 1$
elementary vector whose $g^{th}$ component is $1$ and all other components
are $0$; and define%
\begin{eqnarray*}
\underset{\left( d{\LARGE +}K\right) p\times d}{S_{d,g}} &=&\left( 
\begin{array}{c}
e_{g,p}\otimes I_{d} \\ 
\underset{Kp\times d}{0}%
\end{array}%
\right) \text{, }\underset{\left( d{\LARGE +}K\right) p\times K}{S_{K,g}}%
=\left( 
\begin{array}{c}
\underset{dp\times K}{0} \\ 
e_{g,p}\otimes I_{K}%
\end{array}%
\right) \text{,} \\
\underset{\left( d{\LARGE +}K\right) p\times dp}{S_{d}} &=&\left( 
\begin{array}{cccc}
S_{d,1} & S_{d,2} & \cdots & S_{d,p}%
\end{array}%
\right) \\
&=&\left( 
\begin{array}{cccc}
e_{1,p}\otimes I_{d} & e_{2,p}\otimes I_{d} & \cdots & e_{p,p}\otimes I_{d}
\\ 
\underset{Kp\times d}{0} & \underset{Kp\times d}{0} & \cdots & \underset{%
Kp\times d}{0}%
\end{array}%
\right) \\
&=&\left( 
\begin{array}{c}
I_{p}\otimes I_{d} \\ 
\underset{Kp\times dp}{0}%
\end{array}%
\right) =\left( 
\begin{array}{c}
I_{dp} \\ 
\underset{Kp\times dp}{0}%
\end{array}%
\right) \\
\underset{\left( d{\LARGE +}K\right) p\times Kp}{S_{K}} &=&\left( 
\begin{array}{cccc}
S_{K,1} & S_{K,2} & \cdots & S_{K,p}%
\end{array}%
\right) \\
&=&\left( 
\begin{array}{cccc}
\underset{dp\times K}{0} & \underset{dp\times K}{0} & \cdots & \underset{%
dp\times K}{0} \\ 
e_{1,p}\otimes I_{K} & e_{2,p}\otimes I_{K} & \cdots & e_{p,p}\otimes I_{K}%
\end{array}%
\right) \\
&=&\left( 
\begin{array}{c}
\underset{dp\times Kp}{0} \\ 
I_{p}\otimes I_{K}%
\end{array}%
\right) =\left( 
\begin{array}{c}
\underset{dp\times Kp}{0} \\ 
I_{Kp}%
\end{array}%
\right)
\end{eqnarray*}%
It follows that 
\begin{equation}
\underset{\left( d{\LARGE +}K\right) p\times \left( d{\LARGE +}K\right) p}{S}%
=\left( 
\begin{array}{cc}
\underset{\left( d{\LARGE +}K\right) p\times dp}{S_{d}} & \underset{\left( d%
{\LARGE +}K\right) p\times Kp}{S_{K}}%
\end{array}%
\right) =\left( 
\begin{array}{cc}
I_{dp} & \underset{dp\times Kp}{0} \\ 
\underset{Kp\times dp}{0} & I_{Kp}%
\end{array}%
\right) =I_{\left( d{\LARGE +}K\right) p}  \label{selection matrix}
\end{equation}%
In addition, using these notations, it is easy to see that%
\begin{equation}
S_{d,g}^{\prime }\mathcal{P}_{\left( d{\LARGE +}K\right) p}\underline{W}%
_{t}=Y_{t-g{\LARGE +}1}\text{ for }g=1,...,p  \label{Sdg}
\end{equation}%
and, similarly,%
\begin{equation}
S_{K,g}^{\prime }\mathcal{P}_{\left( d{\LARGE +}K\right) p}\underline{W}%
_{t}=F_{t-g{\large +}1}\text{ for }g=1,...,p\text{.}  \label{SKg}
\end{equation}%
Hence, making use of expressions (\ref{h step ahead representation Yt}) and (%
\ref{selection matrix}) and the fact that $\mathcal{P}_{\left( d{\LARGE +}%
K\right) p}$ is an orthogonal matrix, we can write%
\begin{eqnarray*}
Y_{t{\LARGE +}h} &=&J_{d}\underline{W}_{t{\LARGE +}h} \\
&=&\dsum\limits_{j=0}^{h-1}J_{d}A^{j}\alpha +J_{d}A^{h}\mathcal{P}_{\left( d%
{\LARGE +}K\right) p}^{\prime }\mathcal{P}_{\left( d{\LARGE +}K\right) p}%
\underline{W}_{t}+\dsum\limits_{j=0}^{h-1}J_{d}A^{j}J_{d{\LARGE +}K}^{\prime
}\varepsilon _{t{\LARGE +}h-j} \\
&=&\dsum\limits_{j=0}^{h-1}J_{d}A^{j}\alpha +J_{d}A^{h}\mathcal{P}_{\left( d%
{\LARGE +}K\right) p}^{\prime }SS^{\prime }\mathcal{P}_{\left( d{\LARGE +}%
K\right) p}\underline{W}_{t}+\dsum\limits_{j=0}^{h-1}J_{d}A^{j}J_{d{\LARGE +}%
K}^{\prime }\varepsilon _{t{\LARGE +}h-j} \\
&=&\dsum\limits_{j=0}^{h-1}J_{d}A^{j}\alpha +\dsum\limits_{g=1}^{p}J_{d}A^{h}%
\mathcal{P}_{\left( d{\LARGE +}K\right) p}^{\prime }\left(
S_{d,g}S_{d,g}^{\prime }+S_{K,g}S_{K,g}^{\prime }\right) \mathcal{P}_{\left(
d{\LARGE +}K\right) p}\underline{W}_{t}+\dsum%
\limits_{j=0}^{h-1}J_{d}A^{j}J_{d{\LARGE +}K}^{\prime }\varepsilon _{t%
{\LARGE +}h-j}
\end{eqnarray*}%
so that, in light of expressions (\ref{Sdg}) and (\ref{SKg}), we further
deduce that 
\begin{eqnarray*}
Y_{t{\LARGE +}h} &=&J_{d}\underline{W}_{t{\LARGE +}h} \\
&=&\dsum\limits_{j=0}^{h-1}J_{d}A^{j}\alpha +\dsum\limits_{g=1}^{p}J_{d}A^{h}%
\mathcal{P}_{\left( d{\LARGE +}K\right) p}^{\prime }\left(
S_{d,g}S_{d,g}^{\prime }+S_{K,g}S_{K,g}^{\prime }\right) \mathcal{P}_{\left(
d{\LARGE +}K\right) p}\underline{W}_{t}+\dsum%
\limits_{j=0}^{h-1}J_{d}A^{j}J_{d{\LARGE +}K}^{\prime }\varepsilon _{t%
{\LARGE +}h-j} \\
&=&\dsum\limits_{j=0}^{h-1}J_{d}A^{j}\alpha +\dsum\limits_{g=1}^{p}J_{d}A^{h}%
\mathcal{P}_{\left( d{\LARGE +}K\right) p}^{\prime }S_{d,g}S_{d,g}^{\prime }%
\mathcal{P}_{\left( d{\LARGE +}K\right) p}\underline{W}_{t}+\dsum%
\limits_{g=1}^{p}J_{d}A^{h}\mathcal{P}_{\left( d{\LARGE +}K\right)
p}^{\prime }S_{K,g}S_{K,g}^{\prime }\mathcal{P}_{\left( d{\LARGE +}K\right)
p}\underline{W}_{t} \\
&&+\dsum\limits_{j=0}^{h-1}J_{d}A^{j}J_{d{\LARGE +}K}^{\prime }\varepsilon
_{t{\LARGE +}h-j} \\
&=&\dsum\limits_{j=0}^{h-1}J_{d}A^{j}\alpha +\dsum\limits_{g=1}^{p}J_{d}A^{h}%
\mathcal{P}_{\left( d{\LARGE +}K\right) p}^{\prime }S_{d,g}Y_{t-g{\LARGE +}%
1}+\dsum\limits_{g=1}^{p}J_{d}A^{h}\mathcal{P}_{\left( d{\LARGE +}K\right)
p}^{\prime }S_{K,g}F_{t-g{\LARGE +}1} \\
&&+\dsum\limits_{j=0}^{h-1}J_{d}A^{j}J_{d{\LARGE +}K}^{\prime }\varepsilon
_{t{\LARGE +}h-j} \\
&=&\beta _{0}+\dsum\limits_{g=1}^{p}B_{1,g}^{\prime }Y_{t-g{\LARGE +}%
1}+\dsum\limits_{g=1}^{p}B_{2,g}^{\prime }F_{t-g{\LARGE +}1}+\eta _{t{\LARGE %
+}h}
\end{eqnarray*}%
where%
\begin{eqnarray}
\beta _{0} &=&\dsum\limits_{j=0}^{h-1}J_{d}A^{j}\alpha \text{, }\eta _{t%
{\LARGE +}h}=\dsum\limits_{j=0}^{h-1}J_{d}A^{j}J_{d{\LARGE +}K}^{\prime
}\varepsilon _{t{\LARGE +}h-j}\text{,}  \notag \\
B_{1,g}^{\prime } &=&J_{d}A^{h}\mathcal{P}_{\left( d{\LARGE +}K\right)
p}^{\prime }S_{d,g}\text{ and }B_{2,g}^{\prime }=J_{d}A^{h}\mathcal{P}%
_{\left( d{\LARGE +}K\right) p}^{\prime }S_{K,g}\text{ for }g=1,...,p\text{.}
\label{B1g and B2g}
\end{eqnarray}%
Next, define $B_{1}^{\prime }=\left( 
\begin{array}{cccc}
B_{1,1}^{\prime } & B_{1,2}^{\prime } & \cdots & B_{1,p}^{\prime }%
\end{array}%
\right) $ and $B_{2}^{\prime }=\left( 
\begin{array}{cccc}
B_{2,1}^{\prime } & B_{2,2}^{\prime } & \cdots & B_{2,p}^{\prime }%
\end{array}%
\right) $, and note that, by expression (\ref{B1g and B2g}) above,%
\begin{eqnarray*}
B_{1}^{\prime } &=&J_{d}A^{h}\mathcal{P}_{\left( d{\LARGE +}K\right)
p}^{\prime }\left( 
\begin{array}{cccc}
S_{d,1} & S_{d,2} & \cdots & S_{d,p}%
\end{array}%
\right) =J_{d}A^{h}\mathcal{P}_{\left( d{\LARGE +}K\right) p}^{\prime }S_{d}
\\
B_{2}^{\prime } &=&J_{d}A^{h}\mathcal{P}_{\left( d{\LARGE +}K\right)
p}^{\prime }\left( 
\begin{array}{cccc}
S_{K,1} & S_{K,2} & \cdots & S_{K,p}%
\end{array}%
\right) =J_{d}A^{h}\mathcal{P}_{\left( d{\LARGE +}K\right) p}^{\prime }S_{K}%
\text{.}
\end{eqnarray*}%
Finally, let $\underline{Y}_{t}$ and $\underline{F}_{t}$ be as defined in
expression (\ref{repermutated Wbar}), and we can write the $h$-step ahead
forecast equation more succinctly as%
\begin{eqnarray*}
Y_{t{\LARGE +}h} &=&\beta _{0}+\dsum\limits_{g=1}^{p}B_{1,g}^{\prime }Y_{t-g%
{\LARGE +}1}+\dsum\limits_{g=1}^{p}B_{2,g}^{\prime }F_{t-g{\LARGE +}1}+\eta
_{t{\LARGE +}h} \\
&=&\beta _{0}+B_{1}^{\prime }\underline{Y}_{t}+B_{2}^{\prime }\underline{F}%
_{t}+\eta _{t{\LARGE +}h}\text{. }\square
\end{eqnarray*}

\medskip

\noindent \textbf{Lemma D-1:} Let $T_{h}=T-h-p+1$ where $h$ is a (fixed)
non-negative integer and $p$ is a (fixed) positive integer. Suppose that
Assumptions 3-1, 3-2(a)-(b), 3-2(d), 3-5, and 3-7 hold. Then, the following
statements are true.

\begin{enumerate}
\item[(a)] There exists a positive constant $\underline{c}$ such that%
\begin{equation*}
\lambda _{\min }\left\{ \frac{1}{T_{h}}\dsum\limits_{t=p}^{T-h}\dsum%
\limits_{j=0}^{\infty }A^{j}J_{d{\LARGE +}K}^{\prime }E\left[ \varepsilon
_{t-j}\varepsilon _{t-j}^{\prime }\right] J_{d{\LARGE +}K}\left(
A^{j}\right) ^{\prime }\right\} \geq \underline{c}>0\text{,}
\end{equation*}%
where $A$ is the coefficient matrix of the companion form given in
expression (\ref{companion form Appendix}) and where%
\begin{equation}
\underset{\left( d{\LARGE +}K\right) \times \left( d{\LARGE +}K\right) p}{%
J_{d{\LARGE +}K}}=\left[ 
\begin{array}{cccc}
I_{d{\LARGE +}K} & 0 & \cdots & 0%
\end{array}%
\right] \text{.}  \label{Jd+K}
\end{equation}

\item[(b)] The matrix%
\begin{equation*}
\frac{1}{T_{h}}\dsum\limits_{t=p}^{T-h}\left( 
\begin{array}{ccc}
1 & E\left[ \underline{Y}_{t}^{\prime }\right] & E\left[ \underline{F}%
_{t}^{\prime }\right] \\ 
E\left[ \underline{Y}_{t}\right] & E\left[ \underline{Y}_{t}\underline{Y}%
_{t}^{\prime }\right] & E\left[ \underline{Y}_{t}\underline{F}_{t}^{\prime }%
\right] \\ 
E\left[ \underline{F}_{t}\right] & E\left[ \underline{F}_{t}\underline{Y}%
_{t}^{\prime }\right] & E\left[ \underline{F}_{t}\underline{F}_{t}^{\prime }%
\right]%
\end{array}%
\right)
\end{equation*}%
is non-singular for all $T>h+p-1$.
\end{enumerate}

\medskip

\noindent \textbf{Proof of Lemma D-1:}

For part (a), we prove by contradiction. To proceed, let%
\begin{equation*}
J_{d{\LARGE +}K,r}=e_{r,p}^{\prime }\otimes I_{d{\LARGE +}K}\text{ for }r\in
\left\{ 1,....,p\right\}
\end{equation*}%
where $e_{r,p}$ is a $p\times 1$ elementary vector whose $r^{th}$ component
is equal to $1$ and all other components are equal to $0$. Note that, under
this definition, $J_{d{\LARGE +}K,1}=J_{d{\LARGE +}K}$, where $J_{d{\LARGE +}%
K}$ is as defined previously in expression (\ref{Jd+K}). Suppose that the
matrix%
\begin{equation*}
\dsum\limits_{j=0}^{\infty }A^{j}J_{d{\LARGE +}K,1}^{\prime }J_{d{\LARGE +}%
K,1}\left( A^{j}\right) ^{\prime }
\end{equation*}%
is singular; then, there exists $b\in \mathbb{R}^{\left( d{\LARGE +}K\right)
p}\backslash \left\{ 0\right\} $ such that%
\begin{equation*}
\dsum\limits_{j=0}^{\infty }b^{\prime }A^{j}J_{d{\LARGE +}K,1}^{\prime }J_{d%
{\LARGE +}K,1}\left( A^{j}\right) ^{\prime }b=0
\end{equation*}%
This, in turn, implies that $J_{d{\LARGE +}K,1}\left( A^{j}\right) ^{\prime
}b=0$ \ for all $j$. Now, partition%
\begin{equation*}
b=\left( 
\begin{array}{c}
\underset{\left( d{\LARGE +}K\right) \times 1}{b_{1}} \\ 
\underset{\left( d{\LARGE +}K\right) \times 1}{b_{2}} \\ 
\vdots \\ 
\underset{\left( d{\LARGE +}K\right) \times 1}{b_{p}}%
\end{array}%
\right)
\end{equation*}%
Note that, for $j=0$, let $L_{0}=I_{d{\LARGE +}K}$, and it is easily seen
that%
\begin{eqnarray*}
0 &=&J_{d{\LARGE +}K,1}\left( A^{0}\right) ^{\prime }b \\
&=&J_{d{\LARGE +}K,1}b \\
&=&\left[ 
\begin{array}{ccccc}
I_{d{\LARGE +}K} & 0 & \cdots & 0 & 0%
\end{array}%
\right] \left( 
\begin{array}{c}
b_{1} \\ 
b_{2} \\ 
\vdots \\ 
b_{p-1} \\ 
b_{p}%
\end{array}%
\right) \\
&=&b_{1}\text{ }\left( =L_{0}b_{1}\right)
\end{eqnarray*}%
Now, for $j=1$, define $\overline{A}=\left[ 
\begin{array}{ccccc}
A_{1} & A_{2} & \cdots & A_{p-1} & A_{p}%
\end{array}%
\right] $, and note that%
\begin{eqnarray*}
0 &=&J_{d{\LARGE +}K,1}A^{\prime }b \\
&=&\left[ 
\begin{array}{ccccc}
I_{d{\LARGE +}K} & 0 & \cdots & 0 & 0%
\end{array}%
\right] \left( 
\begin{array}{ccccc}
A_{1}^{\prime } & I_{d{\LARGE +}K} & 0 & \cdots & 0 \\ 
A_{2}^{\prime } & 0 & I_{d{\LARGE +}K} &  & \vdots \\ 
\vdots & \vdots & \ddots & \ddots & 0 \\ 
A_{p-1}^{\prime } & \vdots & 0 & \ddots & I_{d{\LARGE +}K} \\ 
A_{p}^{\prime } & 0 & \cdots & \cdots & 0%
\end{array}%
\right) \left( 
\begin{array}{c}
b_{1} \\ 
b_{2} \\ 
\vdots \\ 
b_{p-1} \\ 
b_{p}%
\end{array}%
\right) \\
&=&J_{d{\LARGE +}K,1}\left[ 
\begin{array}{ccccc}
\overline{A}^{\prime } & J_{d{\LARGE +}K,1}^{\prime } & J_{d{\LARGE +}%
K,2}^{\prime } & \cdots & J_{d{\LARGE +}K,p-1}^{\prime }%
\end{array}%
\right] b \\
&=&\left[ J_{d{\LARGE +}K,1}\overline{A}^{\prime }J_{d{\LARGE +}K,1}+J_{d%
{\LARGE +}K,2}\right] b \\
&=&\left[ L_{1}J_{d{\LARGE +}K,1}+L_{0}J_{d{\LARGE +}K,2}\right] b \\
&=&L_{1}b_{1}+L_{0}b_{2}
\end{eqnarray*}%
where $L_{1}=J_{d{\LARGE +}K,1}\overline{A}^{\prime }=A_{1}^{\prime }$.
Since previously we have shown that $b_{1}=0$, it follows that%
\begin{equation*}
b_{2}=L_{1}b_{1}+L_{0}b_{2}=0\text{.}
\end{equation*}%
Moreover, for $j=2$, using the fact that $J_{d{\LARGE +}K,r}J_{d{\LARGE +}%
K,r}^{\prime }=I_{d{\LARGE +}K}$ and $J_{d{\LARGE +}K,r}J_{d{\LARGE +}%
K,s}^{\prime }=0$ for $r\neq s$, we obtain%
\begin{eqnarray*}
0 &=&J_{d{\LARGE +}K,1}\left( A^{\prime }\right) ^{2}b \\
&=&J_{d{\LARGE +}K,1}\left[ 
\begin{array}{cccccc}
\overline{A}^{\prime } & J_{d{\LARGE +}K,1}^{\prime } & J_{d{\LARGE +}%
K,2}^{\prime } & \cdots & J_{d{\LARGE +}K,p-1}^{\prime } & J_{d{\LARGE +}%
K,p}^{\prime }%
\end{array}%
\right] ^{2}b \\
&=&\left[ L_{1}J_{d{\LARGE +}K,1}+L_{0}J_{d{\LARGE +}K,2}\right] \left[ 
\begin{array}{cccccc}
\overline{A}^{\prime } & J_{d{\LARGE +}K,1}^{\prime } & J_{d{\LARGE +}%
K,2}^{\prime } & \cdots & J_{d{\LARGE +}K,p-1}^{\prime } & J_{d{\LARGE +}%
K,p}^{\prime }%
\end{array}%
\right] b \\
&=&\left( \left[ L_{1}J_{d{\LARGE +}K,1}+L_{0}J_{d{\LARGE +}K,2}\right] 
\overline{A}^{\prime }J_{d{\LARGE +}K,1}+L_{1}J_{d{\LARGE +}K,2}+L_{0}J_{d%
{\LARGE +}K,3}\right) b \\
&=&\left( L_{2}J_{d{\LARGE +}K,1}+L_{1}J_{d{\LARGE +}K,2}+L_{0}J_{d{\LARGE +}%
K,3}\right) b \\
&=&L_{2}b_{1}+L_{1}b_{2}+L_{0}b_{3}
\end{eqnarray*}%
where%
\begin{equation*}
L_{2}=\left[ L_{1}J_{d{\LARGE +}K,1}+L_{0}J_{d{\LARGE +}K,2}\right] 
\overline{A}^{\prime }
\end{equation*}%
Given that $b_{1}=0$ and $b_{2}=0$, as we have previously shown, it then
follows that%
\begin{equation*}
b_{3}=L_{2}b_{1}+L_{1}b_{2}+L_{0}b_{3}=0\text{ }\left( \text{since }%
L_{0}=I_{d{\LARGE +}K}\right)
\end{equation*}%
We will show by mathematical induction that, in fact, $b_{r}=0$ for every $%
r\in \left\{ 1,...,p\right\} $. To proceed, suppose that $b_{1}=b_{2}=\cdot
\cdot \cdot =b_{j}=0$ and $0=J_{d{\LARGE +}K,1}\left( A^{\prime }\right)
^{j}b$. By straightforward calculations, one can show (in a manner similar
to the case where $j=0,$ $1,$ or $2$ given earlier) that $J_{d{\LARGE +}%
K,1}\left( A^{\prime }\right) ^{j}b$ has the representation 
\begin{equation*}
J_{d{\LARGE +}K,1}\left( A^{\prime }\right)
^{j}b=L_{j}b_{1}+L_{j-1}b_{2}+\cdot \cdot \cdot L_{1}b_{j}+L_{0}b_{j{\LARGE +%
}1}
\end{equation*}%
for coefficients $L_{j},L_{j-1},...,L_{1}$, and $L_{0}$ where $L_{0}=I_{d%
{\LARGE +}K}$. It follows from the induction hypotheses that%
\begin{eqnarray*}
b_{j+1} &=&L_{j}b_{1}+L_{j-1}b_{2}+\cdot \cdot \cdot L_{1}b_{j}+L_{0}b_{j%
{\LARGE +}1} \\
&=&J_{d{\LARGE +}K,1}\left( A^{\prime }\right) ^{j}b \\
&=&0\text{.}
\end{eqnarray*}%
Hence, by mathematical induction, we conclude that $b_{r}=0$ for every $r\in
\left\{ 1,...,p\right\} $, but this implies that%
\begin{equation*}
b=\left( 
\begin{array}{c}
b_{1} \\ 
b_{2} \\ 
\vdots \\ 
b_{p-1} \\ 
b_{p}%
\end{array}%
\right) =\underset{\left( d{\LARGE +}K\right) p\times 1}{0}
\end{equation*}%
which contradicts our initial assumption that $b\neq 0$. It then follows
that the matrix%
\begin{equation*}
\dsum\limits_{j=0}^{\infty }A^{j}J_{d{\LARGE +}K,1}^{\prime }J_{d{\LARGE +}%
K,1}\left( A^{j}\right) ^{\prime }
\end{equation*}%
is positive definite and, thus, also non-singular, so that there exists a
positive constant $C_{\ast }$ such that%
\begin{equation*}
\lambda _{\min }\left\{ \dsum\limits_{j=0}^{\infty }A^{j}J_{d{\LARGE +}%
K,1}^{\prime }J_{d{\LARGE +}K,1}\left( A^{j}\right) ^{\prime }\right\} \geq
C_{\ast }>0
\end{equation*}%
Moreover, in light of Assumption 3-2(d), this further implies that%
\begin{eqnarray*}
&&\lambda _{\min }\left\{ \frac{1}{T_{h}}\dsum\limits_{t=p}^{T-h}\dsum%
\limits_{j=0}^{\infty }A^{j}J_{d{\LARGE +}K}^{\prime }E\left[ \varepsilon
_{t-j}\varepsilon _{t-j}^{\prime }\right] J_{d{\LARGE +}K}\left(
A^{j}\right) ^{\prime }\right\} \\
&=&\lambda _{\min }\left\{ \dsum\limits_{j=0}^{\infty }A^{j}J_{d{\LARGE +}%
K,1}^{\prime }\frac{1}{T_{h}}\dsum\limits_{t=p}^{T-h}E\left[ \varepsilon
_{t-j}\varepsilon _{t-j}^{\prime }\right] J_{d{\LARGE +}K,1}\left(
A^{j}\right) ^{\prime }\right\} \text{ }\left( \text{since }J_{d{\LARGE +}%
K,1}=J_{d{\LARGE +}K}\right) \\
&\geq &\lambda _{\min }\left\{ \dsum\limits_{j=0}^{\infty }A^{j}J_{d{\LARGE +%
}K,1}^{\prime }J_{d{\LARGE +}K,1}\left( A^{j}\right) ^{\prime }\right\}
\lambda _{\min }\left\{ \frac{1}{T_{h}}\dsum\limits_{t=p}^{T-h}E\left[
\varepsilon _{t-j}\varepsilon _{t-j}^{\prime }\right] \right\} \\
&\geq &\lambda _{\min }\left\{ \dsum\limits_{j=0}^{\infty }A^{j}J_{d{\LARGE +%
}K,1}^{\prime }J_{d{\LARGE +}K,1}\left( A^{j}\right) ^{\prime }\right\}
\inf_{t}\lambda _{\min }\left\{ E\left[ \varepsilon _{t-j}\varepsilon
_{t-j}^{\prime }\right] \right\} \\
&\geq &C_{\ast }\underline{C} \\
&\geq &\underline{c}>0\text{ \ }\left( \text{by choosing }\underline{c}\leq
C_{\ast }\underline{C}\right) \text{. }
\end{eqnarray*}%
where the second inequality above follows from the fact that%
\begin{eqnarray*}
\lambda _{\min }\left\{ \dsum\limits_{t=p}^{T-h}\frac{E\left[ \varepsilon
_{t-j}\varepsilon _{t-j}^{\prime }\right] }{T_{h}}\right\} &\geq
&\dsum\limits_{t=p}^{T-h}\lambda _{\min }\left\{ \frac{E\left[ \varepsilon
_{t-j}\varepsilon _{t-j}^{\prime }\right] }{T_{h}}\right\} \\
&=&\frac{1}{T_{h}}\dsum\limits_{t=p}^{T-h}\lambda _{\min }\left\{ E\left[
\varepsilon _{t-j}\varepsilon _{t-j}^{\prime }\right] \right\} \\
&\geq &\inf_{t}\lambda _{\min }\left\{ E\left[ \varepsilon _{t-j}\varepsilon
_{t-j}^{\prime }\right] \right\} \text{.}
\end{eqnarray*}

Now, to show part (b), note first that extpression (\ref{VMA representation
of W}) in the proof of Lemma C-5 in Chao and Swanson (2022c) gives a vector
moving-average representation for $\underline{W}_{t}$ of the form%
\begin{equation*}
\underline{W}_{t}=\left( I_{\left( d{\LARGE +}K\right) p}-A\right) ^{-1}J_{d%
{\LARGE +}K}^{\prime }\mu +\dsum\limits_{j=0}^{\infty }A^{j}J_{d{\LARGE +}%
K}^{\prime }\varepsilon _{t-j}\text{,}
\end{equation*}%
where $J_{d{\LARGE +}K}=J_{d{\LARGE +}K,1}=\left[ 
\begin{array}{ccccc}
I_{d{\LARGE +}K} & 0 & \cdots & 0 & 0%
\end{array}%
\right] $. Now, let 
\begin{equation*}
\underset{\left( d+K\right) p\times dp}{S_{d}}=\left( 
\begin{array}{c}
I_{dp} \\ 
\underset{Kp\times dp}{0}%
\end{array}%
\right) \text{ and }\underset{\left( d+K\right) p\times Kp}{S_{K}}=\left( 
\begin{array}{c}
\underset{dp\times Kp}{0} \\ 
I_{Kp}%
\end{array}%
\right) \text{,}
\end{equation*}%
and let $\mathcal{P}_{\left( d{\LARGE +}K\right) p}$ be a permutation matrix
such that%
\begin{equation*}
\mathcal{P}_{\left( d{\LARGE +}K\right) p}\underline{W}_{t}=\left( 
\begin{array}{c}
\underline{Y}_{t} \\ 
\underline{F}_{t}%
\end{array}%
\right) \text{.}
\end{equation*}%
It follows that%
\begin{eqnarray*}
\underline{Y}_{t} &=&S_{d}^{\prime }\mathcal{P}_{\left( d{\LARGE +}K\right)
p}\underline{W}_{t} \\
&=&S_{d}^{\prime }\mathcal{P}_{\left( d{\LARGE +}K\right) p}\left( I_{\left(
d{\LARGE +}K\right) p}-A\right) ^{-1}J_{d{\LARGE +}K}^{\prime }\mu
+\dsum\limits_{j=0}^{\infty }S_{d}^{\prime }\mathcal{P}_{\left( d{\LARGE +}%
K\right) p}A^{j}J_{d{\LARGE +}K}^{\prime }\varepsilon _{t-j}
\end{eqnarray*}%
and%
\begin{eqnarray*}
\underline{F}_{t} &=&S_{K}^{\prime }\mathcal{P}_{\left( d{\LARGE +}K\right)
p}\underline{W}_{t} \\
&=&S_{K}^{\prime }\mathcal{P}_{\left( d{\LARGE +}K\right) p}\left( I_{\left(
d{\LARGE +}K\right) p}-A\right) ^{-1}J_{d{\LARGE +}K}^{\prime }\mu
+\dsum\limits_{j=0}^{\infty }S_{K}^{\prime }\mathcal{P}_{\left( d{\LARGE +}%
K\right) p}A^{j}J_{d{\LARGE +}K}^{\prime }\varepsilon _{t-j}\text{.}
\end{eqnarray*}%
Moreover,%
\begin{eqnarray*}
&&E\left[ \underline{Y}_{t}\underline{Y}_{t}^{\prime }\right] \\
&=&E\left\{ \left( S_{d}^{\prime }\mathcal{P}_{\left( d{\LARGE +}K\right)
p}\left( I_{\left( d{\LARGE +}K\right) p}-A\right) ^{-1}J_{d{\LARGE +}%
K}^{\prime }\mu +\dsum\limits_{j=0}^{\infty }S_{d}^{\prime }\mathcal{P}%
_{\left( d{\LARGE +}K\right) p}A^{j}J_{d{\LARGE +}K}^{\prime }\varepsilon
_{t-j}\right) \right. \\
&&\left. \times \left( \mu ^{\prime }J_{d{\LARGE +}K}\left( I_{\left( d%
{\LARGE +}K\right) p}-A^{\prime }\right) ^{-1}\mathcal{P}_{\left( d{\LARGE +}%
K\right) p}^{\prime }S_{d}+\dsum\limits_{k=0}^{\infty }\varepsilon
_{t-k}^{\prime }J_{d{\LARGE +}K}\left( A^{j}\right) ^{\prime }\mathcal{P}%
_{\left( d{\LARGE +}K\right) p}^{\prime }S_{d}\right) \right\} \\
&=&S_{d}^{\prime }\mathcal{P}_{\left( d{\LARGE +}K\right) p}\left( I_{\left(
d{\LARGE +}K\right) p}-A\right) ^{-1}J_{d{\LARGE +}K}^{\prime }\mu \mu
^{\prime }J_{d{\LARGE +}K}\left( I_{\left( d{\LARGE +}K\right) p}-A^{\prime
}\right) ^{-1}\mathcal{P}_{\left( d{\LARGE +}K\right) p}^{\prime }S_{d} \\
&&+\dsum\limits_{j=0}^{\infty }\dsum\limits_{k=0}^{\infty }S_{d}^{\prime }%
\mathcal{P}_{\left( d{\LARGE +}K\right) p}A^{j}J_{d{\LARGE +}K}^{\prime }E%
\left[ \varepsilon _{t-j}\varepsilon _{t-k}^{\prime }\right] J_{d{\LARGE +}%
K}\left( A^{j}\right) ^{\prime }\mathcal{P}_{\left( d{\LARGE +}K\right)
p}^{\prime }S_{d} \\
&=&S_{d}^{\prime }\mathcal{P}_{\left( d{\LARGE +}K\right) p}\left( I_{\left(
d{\LARGE +}K\right) p}-A\right) ^{-1}J_{d{\LARGE +}K}^{\prime }\mu \mu
^{\prime }J_{d{\LARGE +}K}\left( I_{\left( d{\LARGE +}K\right) p}-A^{\prime
}\right) ^{-1}\mathcal{P}_{\left( d{\LARGE +}K\right) p}^{\prime }S_{d} \\
&&+\dsum\limits_{j=0}^{\infty }S_{d}^{\prime }\mathcal{P}_{\left( d{\LARGE +}%
K\right) p}A^{j}J_{d{\LARGE +}K}^{\prime }E\left[ \varepsilon
_{t-j}\varepsilon _{t-j}^{\prime }\right] J_{d{\LARGE +}K}\left(
A^{j}\right) ^{\prime }\mathcal{P}_{\left( d{\LARGE +}K\right) p}^{\prime
}S_{d}\text{,}
\end{eqnarray*}%
\begin{eqnarray*}
&&E\left[ \underline{F}_{t}\underline{F}_{t}^{\prime }\right] \\
&=&E\left\{ \left( S_{K}^{\prime }\mathcal{P}_{\left( d{\LARGE +}K\right)
p}\left( I_{\left( d{\LARGE +}K\right) p}-A\right) ^{-1}J_{d{\LARGE +}%
K}^{\prime }\mu +\dsum\limits_{j=0}^{\infty }S_{K}^{\prime }\mathcal{P}%
_{\left( d{\LARGE +}K\right) p}A^{j}J_{d{\LARGE +}K}^{\prime }\varepsilon
_{t-j}\right) \right. \\
&&\left. \times \left( \mu ^{\prime }J_{d{\LARGE +}K}\left( I_{\left( d%
{\LARGE +}K\right) p}-A^{\prime }\right) ^{-1}\mathcal{P}_{\left( d{\LARGE +}%
K\right) p}^{\prime }S_{K}+\dsum\limits_{k=0}^{\infty }\varepsilon
_{t-k}^{\prime }J_{d{\LARGE +}K}\left( A^{j}\right) ^{\prime }\mathcal{P}%
_{\left( d{\LARGE +}K\right) p}^{\prime }S_{K}\right) \right\} \\
&=&S_{K}^{\prime }\mathcal{P}_{\left( d{\LARGE +}K\right) p}\left( I_{\left(
d{\LARGE +}K\right) p}-A\right) ^{-1}J_{d{\LARGE +}K}^{\prime }\mu \mu
^{\prime }J_{d{\LARGE +}K}\left( I_{\left( d{\LARGE +}K\right) p}-A^{\prime
}\right) ^{-1}\mathcal{P}_{\left( d{\LARGE +}K\right) p}^{\prime }S_{K} \\
&&+\dsum\limits_{j=0}^{\infty }\dsum\limits_{k=0}^{\infty }S_{K}^{\prime }%
\mathcal{P}_{\left( d{\LARGE +}K\right) p}A^{j}J_{d{\LARGE +}K}^{\prime }E%
\left[ \varepsilon _{t-j}\varepsilon _{t-k}^{\prime }\right] J_{d{\LARGE +}%
K}\left( A^{j}\right) ^{\prime }\mathcal{P}_{\left( d{\LARGE +}K\right)
p}^{\prime }S_{K} \\
&=&S_{K}^{\prime }\mathcal{P}_{\left( d{\LARGE +}K\right) p}\left( I_{\left(
d{\LARGE +}K\right) p}-A\right) ^{-1}J_{d{\LARGE +}K}^{\prime }\mu \mu
^{\prime }J_{d{\LARGE +}K}\left( I_{\left( d{\LARGE +}K\right) p}-A^{\prime
}\right) ^{-1}\mathcal{P}_{\left( d{\LARGE +}K\right) p}^{\prime }S_{K} \\
&&+\dsum\limits_{j=0}^{\infty }S_{K}^{\prime }\mathcal{P}_{\left( d{\LARGE +}%
K\right) p}A^{j}J_{d{\LARGE +}K}^{\prime }E\left[ \varepsilon
_{t-j}\varepsilon _{t-j}^{\prime }\right] J_{d{\LARGE +}K}\left(
A^{j}\right) ^{\prime }\mathcal{P}_{\left( d{\LARGE +}K\right) p}^{\prime
}S_{K}\text{,}
\end{eqnarray*}%
and%
\begin{eqnarray*}
&&E\left[ \underline{Y}_{t}\underline{F}_{t}^{\prime }\right] \\
&=&E\left\{ \left( S_{d}^{\prime }\mathcal{P}_{\left( d{\LARGE +}K\right)
p}\left( I_{\left( d{\LARGE +}K\right) p}-A\right) ^{-1}J_{d{\LARGE +}%
K}^{\prime }\mu +\dsum\limits_{j=0}^{\infty }S_{d}^{\prime }\mathcal{P}%
_{\left( d{\LARGE +}K\right) p}A^{j}J_{d{\LARGE +}K}^{\prime }\varepsilon
_{t-j}\right) \right. \\
&&\left. \times \left( \mu ^{\prime }J_{d{\LARGE +}K}\left( I_{\left( d%
{\LARGE +}K\right) p}-A^{\prime }\right) ^{-1}\mathcal{P}_{\left( d{\LARGE +}%
K\right) p}^{\prime }S_{K}+\dsum\limits_{k=0}^{\infty }\varepsilon
_{t-k}^{\prime }J_{d{\LARGE +}K}\left( A^{j}\right) ^{\prime }\mathcal{P}%
_{\left( d{\LARGE +}K\right) p}^{\prime }S_{K}\right) \right\} \\
&=&S_{d}^{\prime }\mathcal{P}_{\left( d{\LARGE +}K\right) p}\left( I_{\left(
d{\LARGE +}K\right) p}-A\right) ^{-1}J_{d{\LARGE +}K}^{\prime }\mu \mu
^{\prime }J_{d{\LARGE +}K}\left( I_{\left( d{\LARGE +}K\right) p}-A^{\prime
}\right) ^{-1}\mathcal{P}_{\left( d{\LARGE +}K\right) p}^{\prime }S_{K} \\
&&+\dsum\limits_{j=0}^{\infty }\dsum\limits_{k=0}^{\infty }S_{d}^{\prime }%
\mathcal{P}_{\left( d{\LARGE +}K\right) p}A^{j}J_{d{\LARGE +}K}^{\prime }E%
\left[ \varepsilon _{t-j}\varepsilon _{t-k}^{\prime }\right] J_{d{\LARGE +}%
K}\left( A^{j}\right) ^{\prime }\mathcal{P}_{\left( d{\LARGE +}K\right)
p}^{\prime }S_{K} \\
&=&S_{d}^{\prime }\mathcal{P}_{\left( d{\LARGE +}K\right) p}\left( I_{\left(
d{\LARGE +}K\right) p}-A\right) ^{-1}J_{d{\LARGE +}K}^{\prime }\mu \mu
^{\prime }J_{d{\LARGE +}K}\left( I_{\left( d{\LARGE +}K\right) p}-A^{\prime
}\right) ^{-1}\mathcal{P}_{\left( d{\LARGE +}K\right) p}^{\prime }S_{K} \\
&&+\dsum\limits_{j=0}^{\infty }S_{d}^{\prime }\mathcal{P}_{\left( d{\LARGE +}%
K\right) p}A^{j}J_{d{\LARGE +}K}^{\prime }E\left[ \varepsilon
_{t-j}\varepsilon _{t-j}^{\prime }\right] J_{d{\LARGE +}K}\left(
A^{j}\right) ^{\prime }\mathcal{P}_{\left( d{\LARGE +}K\right) p}^{\prime
}S_{K}\text{,}
\end{eqnarray*}%
In addition, since%
\begin{eqnarray*}
E\left[ \underline{W}_{t}\right] &=&\left( I_{\left( d{\LARGE +}K\right)
p}-A\right) ^{-1}J_{d{\LARGE +}K}^{\prime }\mu \text{ and} \\
E\left[ \underline{W}_{t}\underline{W}_{t}^{\prime }\right] &=&\left(
I_{\left( d{\LARGE +}K\right) p}-A\right) ^{-1}J_{d{\LARGE +}K}^{\prime }\mu
\mu ^{\prime }J_{d{\LARGE +}K}\left( I_{\left( d{\LARGE +}K\right)
p}-A^{\prime }\right) ^{-1} \\
&&+\dsum\limits_{j=0}^{\infty }A^{j}J_{d{\LARGE +}K}^{\prime }E\left[
\varepsilon _{t-j}\varepsilon _{t-j}^{\prime }\right] J_{d{\LARGE +}K}\left(
A^{j}\right) ^{\prime }
\end{eqnarray*}%
and since%
\begin{equation*}
\left[ 
\begin{array}{cc}
S_{d} & S_{K}%
\end{array}%
\right] =\left( 
\begin{array}{cc}
I_{dp} & \underset{dp\times Kp}{0} \\ 
\underset{Kp\times dp}{0} & I_{Kp}%
\end{array}%
\right) =I_{\left( d{\LARGE +}K\right) p}
\end{equation*}%
it is easy to see that%
\begin{eqnarray*}
&&\left( 
\begin{array}{cc}
E\left[ \underline{Y}_{t}\underline{Y}_{t}^{\prime }\right] & E\left[ 
\underline{Y}_{t}\underline{F}_{t}^{\prime }\right] \\ 
E\left[ \underline{F}_{t}\underline{Y}_{t}^{\prime }\right] & E\left[ 
\underline{F}_{t}\underline{F}_{t}^{\prime }\right]%
\end{array}%
\right) \\
&=&\left( 
\begin{array}{c}
S_{d}^{\prime } \\ 
S_{K}^{\prime }%
\end{array}%
\right) \mathcal{P}_{\left( d{\LARGE +}K\right) p}\left( I_{\left( d{\LARGE +%
}K\right) p}-A\right) ^{-1}J_{d{\LARGE +}K}^{\prime }\mu \mu ^{\prime }J_{d%
{\LARGE +}K}\left( I_{\left( d{\LARGE +}K\right) p}-A^{\prime }\right) ^{-1}%
\mathcal{P}_{\left( d{\LARGE +}K\right) p}^{\prime }\left( 
\begin{array}{cc}
S_{d} & S_{K}%
\end{array}%
\right) \\
&&+\left( 
\begin{array}{c}
S_{d}^{\prime } \\ 
S_{K}^{\prime }%
\end{array}%
\right) \dsum\limits_{j=0}^{\infty }\mathcal{P}_{\left( d{\LARGE +}K\right)
p}A^{j}J_{d{\LARGE +}K}^{\prime }E\left[ \varepsilon _{t-j}\varepsilon
_{t-j}^{\prime }\right] J_{d{\LARGE +}K}\left( A^{j}\right) ^{\prime }%
\mathcal{P}_{\left( d{\LARGE +}K\right) p}^{\prime }\left( 
\begin{array}{cc}
S_{d} & S_{K}%
\end{array}%
\right) \\
&=&\mathcal{P}_{\left( d{\LARGE +}K\right) p}\left( I_{\left( d{\LARGE +}%
K\right) p}-A\right) ^{-1}J_{d{\LARGE +}K}^{\prime }\mu \mu ^{\prime }J_{d%
{\LARGE +}K}\left( I_{\left( d{\LARGE +}K\right) p}-A^{\prime }\right) ^{-1}%
\mathcal{P}_{\left( d{\LARGE +}K\right) p}^{\prime } \\
&&+\dsum\limits_{j=0}^{\infty }\mathcal{P}_{\left( d{\LARGE +}K\right)
p}A^{j}J_{d{\LARGE +}K}^{\prime }E\left[ \varepsilon _{t-j}\varepsilon
_{t-j}^{\prime }\right] J_{d{\LARGE +}K}\left( A^{j}\right) ^{\prime }%
\mathcal{P}_{\left( d{\LARGE +}K\right) p}^{\prime } \\
&=&\mathcal{P}_{\left( d{\LARGE +}K\right) p}E\left[ \underline{W}_{t}%
\underline{W}_{t}^{\prime }\right] \mathcal{P}_{\left( d{\LARGE +}K\right)
p}^{\prime }
\end{eqnarray*}%
and%
\begin{eqnarray*}
&&\left( 
\begin{array}{cc}
E\left[ \underline{Y}_{t}^{\prime }\right] & E\left[ \underline{F}%
_{t}^{\prime }\right]%
\end{array}%
\right) \\
&=&\left( 
\begin{array}{cc}
\mu ^{\prime }J_{d{\LARGE +}K}\left( I_{\left( d{\LARGE +}K\right)
p}-A^{\prime }\right) ^{-1}\mathcal{P}_{\left( d{\LARGE +}K\right)
p}^{\prime }S_{d} & \mu ^{\prime }J_{d{\LARGE +}K}\left( I_{\left( d{\LARGE +%
}K\right) p}-A^{\prime }\right) ^{-1}\mathcal{P}_{\left( d{\LARGE +}K\right)
p}^{\prime }S_{K}%
\end{array}%
\right) \\
&=&\mu ^{\prime }J_{d{\LARGE +}K}\left( I_{\left( d{\LARGE +}K\right)
p}-A^{\prime }\right) ^{-1}\mathcal{P}_{\left( d{\LARGE +}K\right)
p}^{\prime }\left( 
\begin{array}{cc}
S_{d} & S_{K}%
\end{array}%
\right) \\
&=&\mu ^{\prime }J_{d{\LARGE +}K}\left( I_{\left( d{\LARGE +}K\right)
p}-A^{\prime }\right) ^{-1}\mathcal{P}_{\left( d{\LARGE +}K\right)
p}^{\prime } \\
&=&E\left[ \underline{W}_{t}^{\prime }\right] \mathcal{P}_{\left( d{\LARGE +}%
K\right) p}^{\prime }
\end{eqnarray*}%
Making use of these expressions, we can then write%
\begin{eqnarray*}
\left( 
\begin{array}{ccc}
1 & E\left[ \underline{Y}_{t}^{\prime }\right] & E\left[ \underline{F}%
_{t}^{\prime }\right] \\ 
E\left[ \underline{Y}_{t}\right] & E\left[ \underline{Y}_{t}\underline{Y}%
_{t}^{\prime }\right] & E\left[ \underline{Y}_{t}\underline{F}_{t}^{\prime }%
\right] \\ 
E\left[ \underline{F}_{t}\right] & E\left[ \underline{F}_{t}\underline{Y}%
_{t}^{\prime }\right] & E\left[ \underline{F}_{t}\underline{F}_{t}^{\prime }%
\right]%
\end{array}%
\right) &=&\left( 
\begin{array}{cc}
1 & E\left[ \underline{W}_{t}^{\prime }\right] \mathcal{P}_{\left( d{\LARGE +%
}K\right) p}^{\prime } \\ 
\mathcal{P}_{\left( d{\LARGE +}K\right) p}E\left[ \underline{W}_{t}\right] & 
\mathcal{P}_{\left( d{\LARGE +}K\right) p}E\left[ \underline{W}_{t}%
\underline{W}_{t}^{\prime }\right] \mathcal{P}_{\left( d{\LARGE +}K\right)
p}^{\prime }%
\end{array}%
\right) \\
&=&\left( 
\begin{array}{cc}
1 & 0 \\ 
0 & \mathcal{P}_{\left( d{\LARGE +}K\right) p}%
\end{array}%
\right) \left( 
\begin{array}{cc}
1 & E\left[ \underline{W}_{t}^{\prime }\right] \\ 
E\left[ \underline{W}_{t}\right] & E\left[ \underline{W}_{t}\underline{W}%
_{t}^{\prime }\right]%
\end{array}%
\right) \left( 
\begin{array}{cc}
1 & 0 \\ 
0 & \mathcal{P}_{\left( d{\LARGE +}K\right) p}^{\prime }%
\end{array}%
\right) \text{.}
\end{eqnarray*}%
Next, note that%
\begin{eqnarray*}
&&\det \left( 
\begin{array}{cc}
1 & E\left[ \underline{W}_{t}^{\prime }\right] \\ 
E\left[ \underline{W}_{t}\right] & E\left[ \underline{W}_{t}\underline{W}%
_{t}^{\prime }\right]%
\end{array}%
\right) \\
&=&\det \left( 1\right) \det \left\{ E\left[ \underline{W}_{t}\underline{W}%
_{t}^{\prime }\right] -E\left[ \underline{W}_{t}\right] E\left[ \underline{W}%
_{t}^{\prime }\right] \right\} \\
&=&\det \left\{ E\left[ \underline{W}_{t}\underline{W}_{t}^{\prime }\right]
-E\left[ \underline{W}_{t}\right] E\left[ \underline{W}_{t}^{\prime }\right]
\right\} \\
&=&\det \left\{ \left( I_{\left( d{\LARGE +}K\right) p}-A\right) ^{-1}J_{d%
{\LARGE +}K}^{\prime }\mu \mu ^{\prime }J_{d{\LARGE +}K}\left( I_{\left( d%
{\LARGE +}K\right) p}-A^{\prime }\right) ^{-1}\right. \\
&&+\dsum\limits_{j=0}^{\infty }A^{j}J_{d{\LARGE +}K}^{\prime }E\left[
\varepsilon _{t-j}\varepsilon _{t-j}^{\prime }\right] J_{d{\LARGE +}K}\left(
A^{j}\right) ^{\prime } \\
&&\left. -\left( I_{\left( d{\LARGE +}K\right) p}-A\right) ^{-1}J_{d{\LARGE +%
}K}^{\prime }\mu \mu ^{\prime }J_{d{\LARGE +}K}\left( I_{\left( d{\LARGE +}%
K\right) p}-A^{\prime }\right) ^{-1}\right\} \\
&=&\det \left\{ \dsum\limits_{j=0}^{\infty }A^{j}J_{d{\LARGE +}K}^{\prime }E%
\left[ \varepsilon _{t-j}\varepsilon _{t-j}^{\prime }\right] J_{d{\LARGE +}%
K}\left( A^{j}\right) ^{\prime }\right\}
\end{eqnarray*}%
Now, by Assumption 3-2(d) and by the same argument as that used to prove
part (a) above, we see that there exists a constant $\underline{c}$ such that%
\begin{eqnarray*}
&&\lambda _{\min }\left\{ \dsum\limits_{j=0}^{\infty }A^{j}J_{d{\LARGE +}%
K}^{\prime }E\left[ \varepsilon _{t-j}\varepsilon _{t-j}^{\prime }\right]
J_{d{\LARGE +}K}\left( A^{j}\right) ^{\prime }\right\} \\
&\geq &\lambda _{\min }\left\{ \dsum\limits_{j=0}^{\infty }A^{j}J_{d{\LARGE +%
}K}^{\prime }J_{d{\LARGE +}K}\left( A^{j}\right) ^{\prime }\right\}
\inf_{j}\lambda _{\min }\left\{ E\left[ \varepsilon _{t-j}\varepsilon
_{t-j}^{\prime }\right] \right\} \\
&\geq &\underline{c}>0\text{ }
\end{eqnarray*}%
for all $t$, which, in turn, implies that in this case%
\begin{eqnarray*}
\det \left( 
\begin{array}{cc}
1 & E\left[ \underline{W}_{t}^{\prime }\right] \\ 
E\left[ \underline{W}_{t}\right] & E\left[ \underline{W}_{t}\underline{W}%
_{t}^{\prime }\right]%
\end{array}%
\right) &=&\det \left\{ \dsum\limits_{j=0}^{\infty }A^{j}J_{d{\LARGE +}%
K}^{\prime }E\left[ \varepsilon _{t-j}\varepsilon _{t-j}^{\prime }\right]
J_{d{\LARGE +}K}\left( A^{j}\right) ^{\prime }\right\} \\
&\geq &\underline{c}^{\left( d{\LARGE +}K\right) p}>0
\end{eqnarray*}%
for all $t$. Furthermore, since the matrix%
\begin{equation*}
\left( 
\begin{array}{cc}
1 & 0 \\ 
0 & \mathcal{P}_{\left( d{\LARGE +}K\right) p}%
\end{array}%
\right)
\end{equation*}%
is nonsingular, it follows that the matrix%
\begin{eqnarray*}
&&\frac{1}{T_{h}}\dsum\limits_{t=p}^{T-h}\left( 
\begin{array}{ccc}
1 & E\left[ \underline{Y}_{t}^{\prime }\right] & E\left[ \underline{F}%
_{t}^{\prime }\right] \\ 
E\left[ \underline{Y}_{t}\right] & E\left[ \underline{Y}_{t}\underline{Y}%
_{t}^{\prime }\right] & E\left[ \underline{Y}_{t}\underline{F}_{t}^{\prime }%
\right] \\ 
E\left[ \underline{F}_{t}\right] & E\left[ \underline{F}_{t}\underline{Y}%
_{t}^{\prime }\right] & E\left[ \underline{F}_{t}\underline{F}_{t}^{\prime }%
\right]%
\end{array}%
\right) \\
&=&\left( 
\begin{array}{cc}
1 & 0 \\ 
0 & \mathcal{P}_{\left( d{\LARGE +}K\right) p}%
\end{array}%
\right) \frac{1}{T_{h}}\dsum\limits_{t=p}^{T-h}\left( 
\begin{array}{cc}
1 & E\left[ \underline{W}_{t}^{\prime }\right] \\ 
E\left[ \underline{W}_{t}\right] & E\left[ \underline{W}_{t}\underline{W}%
_{t}^{\prime }\right]%
\end{array}%
\right) \left( 
\begin{array}{cc}
1 & 0 \\ 
0 & \mathcal{P}_{\left( d{\LARGE +}K\right) p}^{\prime }%
\end{array}%
\right)
\end{eqnarray*}%
will be nonsingular and, thus, positive definite as required. $\square $

\medskip

\noindent

\noindent \textbf{Lemma D-2:} Let $T_{h}=T-h-p+1$ where $h$ is a (fixed)
non-negative integer and $p$ is a (fixed) positive integer. Suppose that
Assumptions 3-1, 3-2(a)-(c), 3-5, and 3-7 hold. Then, the following
statements are true.

\begin{enumerate}
\item[(a)] 
\begin{equation*}
\frac{1}{T_{h}}\dsum\limits_{t=p}^{T-h}\underline{W}_{t}\underline{W}%
_{t}^{\prime }-\frac{1}{T_{h}}\dsum\limits_{t=p}^{T-h}E\left[ \underline{W}%
_{t}\underline{W}_{t}^{\prime }\right] =O_{p}\left( \frac{1}{\sqrt{T}}\right)
\end{equation*}%
where 
\begin{equation*}
\underline{W}_{t}=\left( 
\begin{array}{c}
W_{t} \\ 
\vdots \\ 
W_{t-p{\LARGE +}1}%
\end{array}%
\right) \text{ and }W_{t}=\left[ 
\begin{array}{c}
Y_{t} \\ 
F_{t}%
\end{array}%
\right] \text{.}
\end{equation*}

\item[(b)] 
\begin{eqnarray*}
\frac{1}{T_{h}}\dsum\limits_{t=p}^{T-h}\underline{Y}_{t}\underline{Y}%
_{t}^{\prime }-\frac{1}{T_{h}}\dsum\limits_{t=p}^{T-h}E\left[ \underline{Y}%
_{t}\underline{Y}_{t}^{\prime }\right] &=&O_{p}\left( \frac{1}{\sqrt{T}}%
\right) \\
\frac{1}{T_{h}}\dsum\limits_{t=p}^{T-h}\underline{F}_{t}\underline{F}%
_{t}^{\prime }-\frac{1}{T_{h}}\dsum\limits_{t=p}^{T-h}E\left[ \underline{F}%
_{t}\underline{F}_{t}^{\prime }\right] &=&O_{p}\left( \frac{1}{\sqrt{T}}%
\right) \text{, and} \\
\frac{1}{T_{h}}\dsum\limits_{t=p}^{T-h}\underline{Y}_{t}\underline{F}%
_{t}^{\prime }-\frac{1}{T_{h}}\dsum\limits_{t=p}^{T-h}E\left[ \underline{Y}%
_{t}\underline{F}_{t}^{\prime }\right] &=&O_{p}\left( \frac{1}{\sqrt{T}}%
\right)
\end{eqnarray*}%
where $\underline{Y}_{t}$ and $\underline{F}_{t}$ are as defined in
expression (\ref{repermutated Wbar}).

\item[(c)] 
\begin{equation*}
\frac{1}{T_{h}}\dsum\limits_{t=p}^{T-h}\underline{W}_{t}=\left( I_{\left( d%
{\LARGE +}K\right) p}-A\right) ^{-1}J_{d{\LARGE +}K}^{\prime }\mu
+O_{p}\left( \frac{1}{\sqrt{T}}\right) .
\end{equation*}

\item[(d)] 
\begin{eqnarray*}
\frac{1}{T_{h}}\dsum\limits_{t=p}^{T-h}\underline{Y}_{t} &=&S_{d}^{\prime }%
\mathcal{P}_{\left( d{\LARGE +}K\right) p}\left( I_{\left( d{\LARGE +}%
K\right) p}-A\right) ^{-1}J_{d{\LARGE +}K}^{\prime }\mu +O_{p}\left( \frac{1%
}{\sqrt{T}}\right) \text{,} \\
\frac{1}{T_{h}}\dsum\limits_{t=p}^{T-h}\underline{F}_{t} &=&S_{K}^{\prime }%
\mathcal{P}_{\left( d{\LARGE +}K\right) p}\left( I_{\left( d{\LARGE +}%
K\right) p}-A\right) ^{-1}J_{d{\LARGE +}K}^{\prime }\mu +O_{p}\left( \frac{1%
}{\sqrt{T}}\right) \text{.}
\end{eqnarray*}

\item[(e)] 
\begin{equation*}
\frac{1}{T_{h}}\dsum\limits_{t=p}^{T-h}\underline{W}_{t}\eta _{t{\LARGE +}%
h}^{\prime }=O_{p}\left( \frac{1}{\sqrt{T}}\right) \text{, where }\eta _{t%
{\LARGE +}h}=\dsum\limits_{j=0}^{h-1}J_{d}A^{j}J_{d{\LARGE +}K}^{\prime
}\varepsilon _{t{\LARGE +}h-j}
\end{equation*}%
with $\underset{d\times \left( d{\LARGE +}K\right) p}{J_{d}}=\left[ 
\begin{array}{cccc}
I_{d} & 0 & \cdots & 0%
\end{array}%
\right] $ and $\underset{\left( d{\LARGE +}K\right) \times \left( d{\LARGE +}%
K\right) p}{J_{d{\LARGE +}K}}=\left[ 
\begin{array}{cccc}
I_{d{\LARGE +}K} & 0 & \cdots & 0%
\end{array}%
\right] $.

\item[(f)] 
\begin{equation*}
\frac{1}{T_{h}}\dsum\limits_{t=p}^{T-h}\underline{Y}_{t}\eta _{t+h}^{\prime
}=O_{p}\left( \frac{1}{\sqrt{T}}\right) \text{ and }\frac{1}{T_{h}}%
\dsum\limits_{t=p}^{T-h}\underline{F}_{t}\eta _{t+h}^{\prime }=O_{p}\left( 
\frac{1}{\sqrt{T}}\right) \text{,}
\end{equation*}%
where $\eta _{t{\LARGE +}h}$ is as defined in part (e) above.

\item[(g)] 
\begin{equation*}
\frac{\mathfrak{H}^{\prime }\iota _{T_{h}}}{T_{h}}=\frac{1}{T_{h}}%
\dsum\limits_{t=p}^{T-h}\eta _{t{\LARGE +}h}=O_{p}\left( \frac{1}{\sqrt{T}}%
\right) =o_{p}\left( 1\right) \text{. }
\end{equation*}

\item[(h)] 
\begin{equation*}
\frac{1}{T_{h}}\dsum\limits_{t=p}^{T-h}\eta _{t{\LARGE +}h}\eta _{t{\LARGE +}%
h}^{\prime }-\frac{1}{T_{h}}\dsum\limits_{t=p}^{T-h}E\left[ \eta _{t{\LARGE +%
}h}\eta _{t{\LARGE +}h}^{\prime }\right] =O_{p}\left( \frac{1}{\sqrt{T}}%
\right) \text{,}
\end{equation*}%
where $\eta _{t{\LARGE +}h}$ is as defined in part (e) above.
\end{enumerate}

\noindent \medskip

\noindent \textbf{Proof of Lemma D-2:}

To show part (a), we note that for $a,b\in \mathbb{R}^{\left( d+K\right) p}$
such that $\left\Vert a\right\Vert _{2}=\left\Vert b\right\Vert _{2}=1$, we
can write 
\begin{eqnarray*}
&&E\left[ \frac{1}{T_{h}}\dsum\limits_{t=p}^{T-h}\left( a^{\prime }%
\underline{W}_{t}\underline{W}_{t}^{\prime }b-E\left[ a^{\prime }\underline{W%
}_{t}\underline{W}_{t}^{\prime }b\right] \right) \right] ^{2} \\
&=&\frac{1}{T_{h}}\dsum\limits_{t=p}^{T-h}E\left[ \left( a^{\prime }%
\underline{W}_{t}\underline{W}_{t}^{\prime }b-E\left[ a^{\prime }\underline{W%
}_{t}\underline{W}_{t}^{\prime }b\right] \right) ^{2}\right] \\
&&+\frac{2}{T_{h}}\dsum\limits_{t=p}^{T-h-1}\dsum\limits_{m=1}^{T-h-t}E\left%
\{ \left( a^{\prime }\underline{W}_{t}\underline{W}_{t}^{\prime }b-E\left[
a^{\prime }\underline{W}_{t}\underline{W}_{t}^{\prime }b\right] \right)
\left( a^{\prime }\underline{W}_{t+m}\underline{W}_{t+m}^{\prime }b-E\left[
a^{\prime }\underline{W}_{t+m}\underline{W}_{t+m}^{\prime }b\right] \right)
\right\}
\end{eqnarray*}%
Note first that%
\begin{eqnarray*}
\frac{1}{T_{h}^{2}}\dsum\limits_{t=p}^{T-h}E\left[ \left( a^{\prime }%
\underline{W}_{t}\underline{W}_{t}^{\prime }b-E\left[ a^{\prime }\underline{W%
}_{t}\underline{W}_{t}^{\prime }b\right] \right) ^{2}\right] &=&\frac{1}{%
T_{h}^{2}}\dsum\limits_{t=p}^{T-h}E\left( a^{\prime }\underline{W}_{t}%
\underline{W}_{t}^{\prime }b\right) ^{2}-\frac{1}{T_{h}^{2}}%
\dsum\limits_{t=p}^{T-h}\left( E\left[ a^{\prime }\underline{W}_{t}%
\underline{W}_{t}^{\prime }b\right] \right) ^{2} \\
&\leq &\frac{1}{T_{h}^{2}}\dsum\limits_{t=p}^{T-h}E\left[ \left( a^{\prime }%
\underline{W}_{t}\underline{W}_{t}^{\prime }a\right) \left( b^{\prime }%
\underline{W}_{t}\underline{W}_{t}^{\prime }b\right) \right] \\
&\leq &\frac{1}{T_{h}^{2}}\dsum\limits_{t=p}^{T-h}\sqrt{E\left( a^{\prime }%
\underline{W}_{t}\underline{W}_{t}^{\prime }a\right) ^{2}}\sqrt{E\left(
b^{\prime }\underline{W}_{t}\underline{W}_{t}^{\prime }b\right) ^{2}} \\
&\leq &\frac{1}{T_{h}^{2}}\dsum\limits_{t=p}^{T-h}E\left\Vert \underline{W}%
_{t}\right\Vert _{2}^{4} \\
&\leq &\frac{C}{T_{h}}=O\left( \frac{1}{T}\right)
\end{eqnarray*}%
where the fourth inequality above follows from applying Liapunov's
inequality and the result given in Lemma C-5.

Next, note that by Lemma C-11, $\left\{ W_{t}\right\} $ is $\beta $-mixing
with $\beta $ mixing coefficient satisfying $\beta _{W}\left( m\right) \leq
C_{1}\exp \left\{ -C_{2}m\right\} $. Since $\alpha _{W,m}\leq \beta
_{W}\left( m\right) $, it follows that $W_{t}$ is $\alpha $-mixing as well,
with $\alpha $ mixing coefficient satisfying $\alpha _{W,m}\leq C_{1}\exp
\left\{ -C_{2}m\right\} $. Moreover, by applying part (b) of Lemma C-2, we
further deduce that $X_{t}=a^{\prime }\underline{W}_{t}\underline{W}%
_{t}^{\prime }b$ is also $\alpha $-mixing with $\alpha $ mixing coefficient
satisfying%
\begin{eqnarray*}
\alpha _{X,m} &\leq &C_{1}\exp \left\{ -C_{2}\left( m-p+1\right) \right\} \\
&\leq &C_{1}^{\ast }\exp \left\{ -C_{2}m\right\}
\end{eqnarray*}%
for some positive constant $C_{1}^{\ast }\geq C_{1}\exp \left\{ C_{2}\left(
p-1\right) \right\} $. Hence, we can apply Lemma C-3 with $p=2$ and $r=3$ to
obtain%
\begin{eqnarray*}
&&\left\vert E\left\{ \left( a^{\prime }\underline{W}_{t}\underline{W}%
_{t}^{\prime }b-E\left[ a^{\prime }\underline{W}_{t}\underline{W}%
_{t}^{\prime }b\right] \right) \left( a^{\prime }\underline{W}_{t+m}%
\underline{W}_{t+m}^{\prime }b-E\left[ a^{\prime }\underline{W}_{t{\LARGE +}%
m}\underline{W}_{t{\LARGE +}m}^{\prime }b\right] \right) \right\} \right\vert
\\
&\leq &2\left( 2^{\frac{{\large 1}}{{\large 2}}}+1\right) \alpha _{X,m}^{%
\frac{{\Large 1}}{{\Large 6}}}\sqrt{E\left( a^{\prime }\underline{W}_{t}%
\underline{W}_{t}^{\prime }b\right) ^{2}}\left( E\left\vert a^{\prime }%
\underline{W}_{t+m}\underline{W}_{t+m}^{\prime }b\right\vert ^{3}\right)
^{1/3}
\end{eqnarray*}%
where $\alpha _{X,m}$ denotes the $\alpha $ mixing coefficient for the
process $X_{t}=a^{\prime }\underline{W}_{t}\underline{W}_{t}^{\prime }b$ and
where, by our previous calculations, 
\begin{equation*}
\alpha _{X,m}^{\frac{{\Large 1}}{{\Large 6}}}\leq \left( C_{1}^{\ast
}\right) ^{\frac{{\Large 1}}{{\Large 6}}}\exp \left\{ -\frac{C_{2}m}{6}%
\right\} \text{ for all }m\text{ sufficiently large.}
\end{equation*}%
It further follows that there exists a positive constant $C_{3}$ such that%
\begin{eqnarray*}
\dsum\limits_{m=1}^{\infty }\alpha _{X,m}^{\frac{{\Large 1}}{{\Large 6}}}
&\leq &\left( C_{1}^{\ast }\right) ^{\frac{{\Large 1}}{{\Large 6}}%
}\dsum\limits_{m=1}^{\infty }\exp \left\{ -\frac{C_{2}m}{6}\right\} \\
&\leq &\left( C_{1}^{\ast }\right) ^{\frac{{\Large 1}}{{\Large 6}}%
}\dsum\limits_{m=0}^{\infty }\exp \left\{ -\frac{C_{2}m}{6}\right\} \\
&\leq &\left( C_{1}^{\ast }\right) ^{\frac{{\Large 1}}{{\Large 6}}}\left[
1-\exp \left\{ -\frac{C_{2}}{6}\right\} \right] ^{-1} \\
&\leq &C_{3}
\end{eqnarray*}%
where the last inequality stems from the fact that $\dsum\nolimits_{m=0}^{%
\infty }\exp \left\{ -\left( C_{2}m/6\right) \right\} $ is a convergent
geometric series given that $0<\exp \left\{ -\left( C_{2}/6\right) \right\}
<1$ for $C_{2}>0$. Hence, 
\begin{eqnarray*}
&&\left\vert \frac{2}{T_{h}^{2}}\dsum\limits_{t=p}^{T-h-1}\dsum%
\limits_{m=1}^{T-h-t}E\left\{ \left( a^{\prime }\underline{W}_{t}\underline{W%
}_{t}^{\prime }b-E\left[ a^{\prime }\underline{W}_{t}\underline{W}%
_{t}^{\prime }b\right] \right) \left( a^{\prime }\underline{W}_{t+m}%
\underline{W}_{t+m}^{\prime }b-E\left[ a^{\prime }\underline{W}_{t+m}%
\underline{W}_{t+m}^{\prime }b\right] \right) \right\} \right\vert \\
&\leq &\frac{2}{T_{h}^{2}}\dsum\limits_{t=p}^{T-h-1}\dsum%
\limits_{m=1}^{T-h-t}\left\vert E\left\{ \left( a^{\prime }\underline{W}_{t}%
\underline{W}_{t}^{\prime }b-E\left[ a^{\prime }\underline{W}_{t}\underline{W%
}_{t}^{\prime }b\right] \right) \left( a^{\prime }\underline{W}_{t+m}%
\underline{W}_{t+m}^{\prime }b-E\left[ a^{\prime }\underline{W}_{t+m}%
\underline{W}_{t+m}^{\prime }b\right] \right) \right\} \right\vert \\
&\leq &\frac{4}{T_{h}^{2}}\left( 2^{\frac{{\large 1}}{{\large 2}}}+1\right)
\dsum\limits_{t=p}^{T-h-1}\dsum\limits_{m=1}^{T-h-t}\alpha _{X,m}^{\frac{%
{\Large 1}}{{\Large 6}}}\sqrt{E\left( a^{\prime }\underline{W}_{t}\underline{%
W}_{t}^{\prime }b\right) ^{2}}\left( E\left\vert a^{\prime }\underline{W}%
_{t+m}\underline{W}_{t+m}^{\prime }b\right\vert ^{3}\right) ^{1/3} \\
&\leq &4\left( \sqrt{2}+1\right) \frac{1}{T_{h}^{2}}\dsum%
\limits_{t=p}^{T-h-1}\dsum\limits_{m=1}^{T-h-t}\left\{ \alpha _{X,m}^{\frac{%
{\Large 1}}{{\Large 6}}}\left[ E\left( a^{\prime }\underline{W}_{t}\right)
^{4}\right] ^{1/4}\left[ E\left( b^{\prime }\underline{W}_{t}\right) ^{4}%
\right] ^{1/4}\left[ E\left( a^{\prime }\underline{W}_{t+m}\right) ^{6}%
\right] ^{\frac{{\Large 1}}{{\Large 6}}}\right. \\
&&\text{ \ \ \ \ \ \ \ \ \ \ \ \ \ \ \ \ \ \ \ \ \ \ \ \ \ \ \ \ \ \ \ \ \ \
\ \ \ \ \ \ }\left. \times \left[ E\left( b^{\prime }\underline{W}%
_{t+m}\right) ^{6}\right] ^{\frac{{\Large 1}}{{\Large 6}}}\right\} \\
&\leq &4\left( \sqrt{2}+1\right) \left( \sup_{t}E\left[ \left\Vert 
\underline{W}_{t}\right\Vert _{2}^{4}\right] \right) ^{\frac{{\Large 1}}{%
{\Large 2}}}\left( \sup_{t}E\left[ \left\Vert \underline{W}_{t}\right\Vert
_{2}^{6}\right] \right) ^{\frac{{\Large 1}}{{\Large 3}}}\frac{1}{T_{h}^{2}}%
\dsum\limits_{t=p}^{T-h-1}\dsum\limits_{m=1}^{\infty }\alpha _{X,m}^{\frac{%
{\Large 1}}{{\Large 6}}} \\
&\leq &4\left( \sqrt{2}+1\right) \left( \sup_{t}E\left[ \left\Vert 
\underline{W}_{t}\right\Vert _{2}^{4}\right] \right) ^{\frac{{\Large 1}}{%
{\Large 2}}}\left( \sup_{t}E\left[ \left\Vert \underline{W}_{t}\right\Vert
_{2}^{6}\right] \right) ^{\frac{{\Large 1}}{{\Large 3}}}\frac{1}{T_{h}^{2}}%
\dsum\limits_{t=p}^{T-h}C_{3}\text{ } \\
&\leq &\frac{\overline{C}}{T_{h}}\text{ }=O\left( \frac{1}{T}\right) \text{ }%
\left( \text{where }\overline{C}\geq 4\left( \sqrt{2}+1\right) \left(
\sup_{t}E\left[ \left\Vert \underline{W}_{t}\right\Vert _{2}^{4}\right]
\right) ^{\frac{{\Large 1}}{{\Large 2}}}\left( \sup_{t}E\left[ \left\Vert 
\underline{W}_{t}\right\Vert _{2}^{6}\right] \right) ^{\frac{{\Large 1}}{%
{\Large 3}}}C_{3}\right)
\end{eqnarray*}%
It follows that%
\begin{eqnarray*}
&&E\left[ \frac{1}{T_{h}}\dsum\limits_{t=p}^{T-h}\left( a^{\prime }%
\underline{W}_{t}\underline{W}_{t}^{\prime }b-E\left[ a^{\prime }\underline{W%
}_{t}\underline{W}_{t}^{\prime }b\right] \right) \right] ^{2} \\
&\leq &\frac{1}{T_{h}^{2}}\dsum\limits_{t=p}^{T-h}E\left[ \left( a^{\prime }%
\underline{W}_{t}\underline{W}_{t}^{\prime }b-E\left[ a^{\prime }\underline{W%
}_{t}\underline{W}_{t}^{\prime }b\right] \right) ^{2}\right] \\
&&+\frac{2}{T_{h}^{2}}\dsum\limits_{t=p}^{T-h-1}\dsum\limits_{m=1}^{T-h-t}%
\left\vert E\left\{ \left( a^{\prime }\underline{W}_{t}\underline{W}%
_{t}^{\prime }b-E\left[ a^{\prime }\underline{W}_{t}\underline{W}%
_{t}^{\prime }b\right] \right) \left( a^{\prime }\underline{W}_{t+m}%
\underline{W}_{t+m}^{\prime }b-E\left[ a^{\prime }\underline{W}_{t+m}%
\underline{W}_{t+m}^{\prime }b\right] \right) \right\} \right\vert \\
&=&O\left( \frac{1}{T}\right)
\end{eqnarray*}%
so that, applying Markov's inequality, we get%
\begin{equation*}
\frac{1}{T_{h}}\dsum\limits_{t=p}^{T-h}a^{\prime }\underline{W}_{t}%
\underline{W}_{t}^{\prime }b-\frac{1}{T_{h}}\dsum\limits_{t=p}^{T-h}E\left[
a^{\prime }\underline{W}_{t}\underline{W}_{t}^{\prime }b\right] =O_{p}\left( 
\frac{1}{\sqrt{T}}\right)
\end{equation*}%
Since this result holds for every $a\in \mathbb{R}^{\left( d{\LARGE +}%
K\right) p}$ and $b\in \mathbb{R}^{\left( d{\LARGE +}K\right) p}$ such that $%
\left\Vert a\right\Vert _{2}=\left\Vert b\right\Vert _{2}=1$, we further
deduce that%
\begin{equation*}
\frac{1}{T_{h}}\dsum\limits_{t=p}^{T-h}\underline{W}_{t}\underline{W}%
_{t}^{\prime }-\frac{1}{T_{h}}\dsum\limits_{t=p}^{T-h}E\left[ \underline{W}%
_{t}\underline{W}_{t}^{\prime }\right] =O_{p}\left( \frac{1}{\sqrt{T}}%
\right) \text{.}
\end{equation*}

To show part (b), note first that

\begin{eqnarray*}
S_{d}^{\prime }\mathcal{P}_{\left( d{\LARGE +}K\right) p}\underline{W}_{t}
&=&\left( 
\begin{array}{cc}
I_{dp} & \underset{dp\times Kp}{0}%
\end{array}%
\right) \left( 
\begin{array}{c}
\underset{dp\times 1}{\underline{Y}_{t}} \\ 
\underset{Kp\times 1}{\underline{F}_{t}}%
\end{array}%
\right) =\underline{Y}_{t}, \\
S_{K}^{\prime }\mathcal{P}_{\left( d{\LARGE +}K\right) p}\underline{W}_{t}
&=&\left( 
\begin{array}{cc}
\underset{Kp\times dp}{0} & I_{Kp}%
\end{array}%
\right) \left( 
\begin{array}{c}
\underset{dp\times 1}{\underline{Y}_{t}} \\ 
\underset{Kp\times 1}{\underline{F}_{t}}%
\end{array}%
\right) =\underline{F}_{t}
\end{eqnarray*}%
By the result given in part (a) above, it follows from applying Slutsky's
theorem that%
\begin{eqnarray*}
&&\frac{1}{T_{h}}\dsum\limits_{t=p}^{T-h}\underline{Y}_{t}\underline{Y}%
_{t}^{\prime }-\frac{1}{T_{h}}\dsum\limits_{t=p}^{T-h}E\left[ \underline{Y}%
_{t}\underline{Y}_{t}^{\prime }\right] \\
&=&S_{d}^{\prime }\mathcal{P}_{\left( d{\LARGE +}K\right) p}\left( \frac{1}{%
T_{h}}\dsum\limits_{t=p}^{T-h}\underline{W}_{t}\underline{W}_{t}^{\prime }-%
\frac{1}{T_{h}}\dsum\limits_{t=p}^{T-h}E\left[ \underline{W}_{t}\underline{W}%
_{t}^{\prime }\right] \right) \mathcal{P}_{\left( d{\LARGE +}K\right) p}S_{d}
\\
&=&O_{p}\left( \frac{1}{\sqrt{T}}\right) \text{,}
\end{eqnarray*}%
\begin{eqnarray*}
&&\frac{1}{T_{h}}\dsum\limits_{t=p}^{T-h}\underline{F}_{t}\underline{F}%
_{t}^{\prime }-\frac{1}{T_{h}}\dsum\limits_{t=p}^{T-h}E\left[ \underline{F}%
_{t}\underline{F}_{t}^{\prime }\right] \\
&=&S_{K}^{\prime }\mathcal{P}_{\left( d{\LARGE +}K\right) p}\left( \frac{1}{%
T_{h}}\dsum\limits_{t=p}^{T-h}\underline{W}_{t}\underline{W}_{t}^{\prime }-%
\frac{1}{T_{h}}\dsum\limits_{t=p}^{T-h}E\left[ \underline{W}_{t}\underline{W}%
_{t}^{\prime }\right] \right) \mathcal{P}_{\left( d{\LARGE +}K\right) p}S_{K}
\\
&=&O_{p}\left( \frac{1}{\sqrt{T}}\right) \text{,}
\end{eqnarray*}%
and%
\begin{eqnarray*}
&&\frac{1}{T_{h}}\dsum\limits_{t=p}^{T-h}\underline{Y}_{t}\underline{F}%
_{t}^{\prime }-\frac{1}{T_{h}}\dsum\limits_{t=p}^{T-h}E\left[ \underline{Y}%
_{t}\underline{F}_{t}^{\prime }\right] \\
&=&S_{d}^{\prime }\mathcal{P}_{\left( d{\LARGE +}K\right) p}\left( \frac{1}{%
T_{h}}\dsum\limits_{t=p}^{T-h}\underline{W}_{t}\underline{W}_{t}^{\prime }-%
\frac{1}{T_{h}}\dsum\limits_{t=p}^{T-h}E\left[ \underline{W}_{t}\underline{W}%
_{t}^{\prime }\right] \right) \mathcal{P}_{\left( d{\LARGE +}K\right) p}S_{K}
\\
&=&O_{p}\left( \frac{1}{\sqrt{T}}\right) \text{.}
\end{eqnarray*}

To show part (c), let $a\in \mathbb{R}^{\left( d{\LARGE +}K\right) p}$ such
that $\left\Vert a\right\Vert _{2}=1$, and write%
\begin{eqnarray*}
\frac{1}{T_{h}}\dsum\limits_{t=p}^{T-h}a^{\prime }\underline{W}_{t} &=&\frac{%
1}{T_{h}}\dsum\limits_{t=p}^{T-h}\left\{ a^{\prime }\left( I_{\left( d%
{\LARGE +}K\right) p}-A\right) ^{-1}J_{d{\LARGE +}K}^{\prime }\mu
+\dsum\limits_{j=0}^{\infty }a^{\prime }A^{j}J_{d{\LARGE +}K}^{\prime
}\varepsilon _{t-j}\right\} \\
&=&a^{\prime }\left( I_{\left( d{\LARGE +}K\right) p}-A\right) ^{-1}J_{d%
{\LARGE +}K}^{\prime }\mu +\frac{1}{T_{h}}\dsum\limits_{t=p}^{T-h}\dsum%
\limits_{j=0}^{\infty }a^{\prime }A^{j}J_{d{\LARGE +}K}^{\prime }\varepsilon
_{t-j}
\end{eqnarray*}%
Next, note that%
\begin{eqnarray*}
E\left[ \frac{1}{T_{h}}\dsum\limits_{t=p}^{T-h}\dsum\limits_{j=0}^{\infty
}a^{\prime }A^{j}J_{d{\LARGE +}K}^{\prime }\varepsilon _{t-j}\right] ^{2} &=&%
\frac{1}{T_{h}^{2}}\dsum\limits_{t=p}^{T-h}\dsum\limits_{s=p}^{T-h}\dsum%
\limits_{j=0}^{\infty }\dsum\limits_{k=0}^{\infty }a^{\prime }A^{j}J_{d%
{\LARGE +}K}^{\prime }E\left[ \varepsilon _{t-j}\varepsilon _{s-k}^{\prime }%
\right] J_{d{\LARGE +}K}\left( A^{k}\right) ^{\prime }a \\
&=&\frac{1}{T_{h}^{2}}\dsum\limits_{t=p}^{T-h}\dsum\limits_{j=0}^{\infty
}a^{\prime }A^{j}J_{d{\LARGE +}K}^{\prime }E\left[ \varepsilon
_{t-j}\varepsilon _{t-j}^{\prime }\right] J_{d{\LARGE +}K}\left(
A^{j}\right) ^{\prime }a \\
&&+\frac{2}{T_{h}^{2}}\dsum\limits_{t=p}^{T-h-1}\dsum\limits_{m=1}^{T-h-t}%
\dsum\limits_{j=0}^{\infty }a^{\prime }A^{j}J_{d{\LARGE +}K}^{\prime }E\left[
\varepsilon _{t-j}\varepsilon _{t-j}^{\prime }\right] J_{d{\LARGE +}K}\left(
A^{m{\LARGE +}j}\right) ^{\prime }a \\
&=&\frac{1}{T_{h}^{2}}\dsum\limits_{t=p}^{T-h}\dsum\limits_{j=0}^{\infty
}a^{\prime }A^{j}J_{d{\LARGE +}K}^{\prime }E\left[ \varepsilon
_{t-j}\varepsilon _{t-j}^{\prime }\right] J_{d{\LARGE +}K}\left(
A^{j}\right) ^{\prime }a \\
&&+\frac{2}{T_{h}^{2}}\dsum\limits_{t=p}^{T-h-1}\dsum\limits_{j=0}^{\infty
}a^{\prime }A^{j}J_{d{\LARGE +}K}^{\prime }E\left[ \varepsilon
_{t-j}\varepsilon _{t-j}^{\prime }\right] J_{d{\LARGE +}K}\left(
A^{j}\right) ^{\prime }\dsum\limits_{m=1}^{T-h-t}\left( A^{m}\right)
^{\prime }a
\end{eqnarray*}%
Now,%
\begin{eqnarray*}
&&\frac{1}{T_{h}^{2}}\dsum\limits_{t=p}^{T-h}\dsum\limits_{j=0}^{\infty
}a^{\prime }A^{j}J_{d{\LARGE +}K}^{\prime }E\left[ \varepsilon
_{t-j}\varepsilon _{t-j}^{\prime }\right] J_{d{\LARGE +}K}\left(
A^{j}\right) ^{\prime }a \\
&\leq &\frac{1}{T_{h}^{2}}\dsum\limits_{t=p}^{T-h}\dsum\limits_{j=0}^{\infty
}E\left\Vert \varepsilon _{t-j}\right\Vert _{2}^{2}a^{\prime }A^{j}J_{d%
{\LARGE +}K}^{\prime }J_{d{\LARGE +}K}\left( A^{j}\right) ^{\prime }a \\
&\leq &\frac{1}{T_{h}^{2}}\dsum\limits_{t=p}^{T-h}\dsum\limits_{j=0}^{\infty
}\left( E\left\Vert \varepsilon _{t-j}\right\Vert _{2}^{6}\right) ^{\frac{%
{\Large 1}}{{\Large 3}}}a^{\prime }A^{j}\left( A^{j}\right) ^{\prime }a \\
&&\left( \text{by Liapunov's inequality and }\lambda _{\max }\left( J_{d%
{\LARGE +}K}^{\prime }J_{d{\LARGE +}K}\right) =1\right) \\
&\leq &\overline{C}^{\frac{{\Large 1}}{{\Large 3}}}\frac{1}{T_{h}^{2}}%
\dsum\limits_{t=p}^{T-h}\dsum\limits_{j=0}^{\infty }a^{\prime }A^{j}\left(
A^{j}\right) ^{\prime }a \\
&&\left( \text{where }\overline{C}\geq 1\text{ is a constant such that }%
E\left\Vert \varepsilon _{t-j}\right\Vert _{2}^{6}\leq \overline{C}<\infty 
\text{ by Assumption 3-2(b)}\right) \\
&\leq &\overline{C}^{\frac{{\Large 1}}{{\Large 3}}}\frac{1}{T_{h}^{2}}%
\dsum\limits_{t=p}^{T-h}\dsum\limits_{j=0}^{\infty }\lambda _{\max }\left\{
A^{j}\left( A^{j}\right) ^{\prime }\right\} a^{\prime }a \\
&=&\overline{C}^{\frac{{\Large 1}}{{\Large 3}}}\frac{1}{T_{h}^{2}}%
\dsum\limits_{t=p}^{T-h}\dsum\limits_{j=0}^{\infty }\lambda _{\max }\left\{
\left( A^{j}\right) ^{\prime }A^{j}\right\} \text{ } \\
&&\left( \text{since }\lambda _{\max }\left\{ A^{j}\left( A^{j}\right)
^{\prime }\right\} =\lambda _{\max }\left\{ \left( A^{j}\right) ^{\prime
}A^{j}\right\} \text{ and }a^{\prime }a=1\right) \\
&=&\overline{C}^{\frac{{\Large 1}}{{\Large 3}}}\frac{1}{T_{h}^{2}}%
\dsum\limits_{t=p}^{T-h}\dsum\limits_{j=0}^{\infty }\sigma _{\max
}^{2}\left( A^{j}\right) \\
&\leq &\overline{C}^{\frac{{\Large 1}}{{\Large 3}}}\frac{1}{T_{h}^{2}}%
\dsum\limits_{t=p}^{T-h}\dsum\limits_{j=0}^{\infty }C\max \left\{ \left\vert
\lambda _{\max }\left( A^{j}\right) \right\vert ^{2},\left\vert \lambda
_{\min }\left( A^{j}\right) \right\vert ^{2}\right\} \text{ }\left( \text{by
Assumption 3-7}\right) \\
&=&\overline{C}^{\frac{{\Large 1}}{{\Large 3}}}C\frac{1}{T_{h}^{2}}%
\dsum\limits_{t=p}^{T-h}\dsum\limits_{j=0}^{\infty }\max \left\{ \left\vert
\lambda _{\max }\left( A\right) \right\vert ^{2j},\left\vert \lambda _{\min
}\left( A\right) \right\vert ^{2j}\right\} \\
&=&\overline{C}^{\frac{{\Large 1}}{{\Large 3}}}C\frac{1}{T_{h}^{2}}%
\dsum\limits_{t=p}^{T-h}\dsum\limits_{j=0}^{\infty }\phi _{\max }^{2j}
\end{eqnarray*}%
where $\phi _{\max }=\max \left\{ \left\vert \lambda _{\max }\left( A\right)
\right\vert ,\left\vert \lambda _{\min }\left( A\right) \right\vert \right\} 
$ and where $0<\phi _{\max }<1$ since Assumption 3-1 implies that all
eigenvalues of $A$ have modulus less than $1$. It follows that%
\begin{eqnarray*}
\frac{1}{T_{h}^{2}}\dsum\limits_{t=p}^{T-h}\dsum\limits_{j=0}^{\infty
}a^{\prime }A^{j}J_{d{\LARGE +}K}^{\prime }E\left[ \varepsilon
_{t-j}\varepsilon _{t-j}^{\prime }\right] J_{d{\LARGE +}K}\left(
A^{j}\right) ^{\prime }a &\leq &\overline{C}^{\frac{{\Large 1}}{{\Large 3}}}C%
\frac{1}{T_{h}^{2}}\dsum\limits_{t=p}^{T-h}\dsum\limits_{j=0}^{\infty }\phi
_{\max }^{2j} \\
&=&\overline{C}^{\frac{{\Large 1}}{{\Large 3}}}C\frac{T-h-p+1}{T_{h}^{2}}%
\frac{1}{1-\phi _{\max }^{2}} \\
&=&\overline{C}^{\frac{{\Large 1}}{{\Large 3}}}C\frac{1}{T_{h}}\frac{1}{%
1-\phi _{\max }^{2}}\text{ \ } \\
&&\left( \text{since }T_{h}=T-h-p+1\right) \\
&=&O\left( \frac{1}{T}\right)
\end{eqnarray*}%
Moreover, write%
\begin{eqnarray*}
&&\left\vert \frac{2}{T_{h}^{2}}\dsum\limits_{t=p}^{T-h-1}\dsum%
\limits_{j=0}^{\infty }a^{\prime }A^{j}J_{d{\LARGE +}K}^{\prime }E\left[
\varepsilon _{t-j}\varepsilon _{t-j}^{\prime }\right] J_{d{\LARGE +}K}\left(
A^{j}\right) ^{\prime }\dsum\limits_{m=1}^{T-h-t}\left( A^{m}\right)
^{\prime }a\right\vert \\
&\leq &\frac{2}{T_{h}^{2}}\dsum\limits_{t=p}^{T-h-1}\left\vert
\dsum\limits_{j=0}^{\infty }a^{\prime }A^{j}J_{d{\LARGE +}K}^{\prime }E\left[
\varepsilon _{t-j}\varepsilon _{t-j}^{\prime }\right] J_{d{\LARGE +}K}\left(
A^{j}\right) ^{\prime }\dsum\limits_{m=1}^{T-h-t}\left( A^{m}\right)
^{\prime }a\right\vert \\
&\leq &\frac{2}{T_{h}^{2}}\dsum\limits_{t=p}^{T-h-1}\left\{ \sqrt{%
\dsum\limits_{j=0}^{\infty }a^{\prime }A^{j}J_{d{\LARGE +}K}^{\prime }E\left[
\varepsilon _{t-j}\varepsilon _{t-j}^{\prime }\right] E\left[ \varepsilon
_{t-j}\varepsilon _{t-j}^{\prime }\right] J_{d{\LARGE +}K}\left(
A^{j}\right) ^{\prime }a}\right. \\
&&\text{ \ \ \ \ \ \ \ \ \ \ \ }\left. \times \sqrt{\dsum\limits_{j=0}^{%
\infty }\dsum\limits_{m_{1}=1}^{T-h-t}a^{\prime }A^{m_{1}}A^{j}J_{d{\LARGE +}%
K}^{\prime }J_{d{\LARGE +}K}\left( A^{j}\right) ^{\prime
}\dsum\limits_{m_{2}=1}^{T-h-t}\left( A^{m_{2}}\right) ^{\prime }a}\right\}
\end{eqnarray*}%
Observe that%
\begin{eqnarray*}
&&\dsum\limits_{j=0}^{\infty }a^{\prime }A^{j}J_{d{\LARGE +}K}^{\prime }E%
\left[ \varepsilon _{t-j}\varepsilon _{t-j}^{\prime }\right] E\left[
\varepsilon _{t-j}\varepsilon _{t-j}^{\prime }\right] J_{d{\LARGE +}K}\left(
A^{j}\right) ^{\prime }a \\
&\leq &\dsum\limits_{j=0}^{\infty }\lambda _{\max }\left( E\left[
\varepsilon _{t-j}\varepsilon _{t-j}^{\prime }\right] E\left[ \varepsilon
_{t-j}\varepsilon _{t-j}^{\prime }\right] \right) a^{\prime }A^{j}J_{d%
{\LARGE +}K}^{\prime }J_{d{\LARGE +}K}\left( A^{j}\right) ^{\prime }a \\
&\leq &\dsum\limits_{j=0}^{\infty }\lambda _{\max }\left( E\left[
\varepsilon _{t-j}\varepsilon _{t-j}^{\prime }\right] E\left[ \varepsilon
_{t-j}\varepsilon _{t-j}^{\prime }\right] \right) C\phi _{\max }^{2j} \\
&=&C\dsum\limits_{j=0}^{\infty }\lambda _{\max }^{2}\left( E\left[
\varepsilon _{t-j}\varepsilon _{t-j}^{\prime }\right] \right) \phi _{\max
}^{2j} \\
&\leq &C\dsum\limits_{j=0}^{\infty }\left( tr\left\{ E\left[ \varepsilon
_{t-j}\varepsilon _{t-j}^{\prime }\right] \right\} \right) ^{2}\phi _{\max
}^{2j} \\
&=&C\dsum\limits_{j=0}^{\infty }\left( E\left\Vert \varepsilon
_{t-j}\right\Vert _{2}^{2}\right) ^{2}\phi _{\max }^{2j} \\
&\leq &C\dsum\limits_{j=0}^{\infty }\left( E\left\Vert \varepsilon
_{t-j}\right\Vert _{2}^{6}\right) ^{\frac{{\Large 2}}{{\Large 3}}}\phi
_{\max }^{2j}\text{ \ }\left( \text{by Liapunov's inequality}\right) \\
&\leq &\overline{C}^{\frac{{\Large 2}}{{\Large 3}}}C\frac{1}{1-\phi _{\max
}^{2}}
\end{eqnarray*}%
and%
\begin{eqnarray*}
&&\dsum\limits_{j=0}^{\infty }\dsum\limits_{m_{1}=1}^{T-h-t}a^{\prime
}A^{m_{1}}A^{j}J_{d{\LARGE +}K}^{\prime }J_{d{\LARGE +}K}\left( A^{j}\right)
^{\prime }\dsum\limits_{m_{2}=1}^{T-h-t}\left( A^{m_{2}}\right) ^{\prime }a
\\
&\leq &\dsum\limits_{j=0}^{\infty
}\dsum\limits_{m_{1}=1}^{T-h-t}\dsum\limits_{m_{2}=1}^{T-h-t}a^{\prime
}A^{m_{1}}A^{j}\left( A^{j}\right) ^{\prime }\left( A^{m_{2}}\right)
^{\prime }a \\
&\leq &C\dsum\limits_{j=0}^{\infty }\phi _{\max
}^{2j}\dsum\limits_{m_{1}=1}^{T-h-t}\dsum\limits_{m_{2}=1}^{T-h-t}\left\vert
a^{\prime }A^{m_{1}}\left( A^{m_{2}}\right) ^{\prime }a\right\vert \\
&\leq &C\dsum\limits_{j=0}^{\infty }\phi _{\max
}^{2j}\dsum\limits_{m_{1}=1}^{T-h-t}\dsum\limits_{m_{2}=1}^{T-h-t}\sqrt{%
a^{\prime }A^{m_{1}}\left( A^{m_{1}}\right) ^{\prime }a}\sqrt{a^{\prime
}A^{m_{2}}\left( A^{m_{2}}\right) ^{\prime }a} \\
&\leq &C\dsum\limits_{j=0}^{\infty }\phi _{\max
}^{2j}\dsum\limits_{m_{1}=1}^{T-h-t}\dsum\limits_{m_{2}=1}^{T-h-t}\sqrt{%
C\phi _{\max }^{2m_{1}}}\sqrt{C\phi _{\max }^{2m_{2}}} \\
&\leq &C^{2}\dsum\limits_{j=0}^{\infty }\phi _{\max
}^{2j}\dsum\limits_{m_{1}=1}^{T-h-t}\phi _{\max
}^{m_{1}}\dsum\limits_{m_{2}=1}^{T-h-t}\phi _{\max }^{m_{2}} \\
&\leq &C^{2}\frac{1}{1-\phi _{\max }^{2}}\left( \frac{1}{1-\phi _{\max }}%
\right) ^{2}
\end{eqnarray*}%
It follows that%
\begin{eqnarray*}
&&\left\vert \frac{2}{T_{h}^{2}}\dsum\limits_{t=p}^{T-h-1}\dsum%
\limits_{j=0}^{\infty }a^{\prime }A^{j}J_{d{\LARGE +}K}^{\prime }E\left[
\varepsilon _{t-j}\varepsilon _{t-j}^{\prime }\right] J_{d{\LARGE +}K}\left(
A^{j}\right) ^{\prime }\dsum\limits_{m=1}^{T-h-t}\left( A^{m}\right)
^{\prime }a\right\vert \\
&\leq &\frac{2}{T_{h}^{2}}\dsum\limits_{t=p}^{T-h-1}\left\{ \sqrt{%
\dsum\limits_{j=0}^{\infty }a^{\prime }A^{j}J_{d{\LARGE +}K}^{\prime }E\left[
\varepsilon _{t-j}\varepsilon _{t-j}^{\prime }\right] E\left[ \varepsilon
_{t-j}\varepsilon _{t-j}^{\prime }\right] J_{d{\LARGE +}K}\left(
A^{j}\right) ^{\prime }a}\right. \\
&&\text{ \ \ \ \ \ \ \ \ \ }\left. \times \sqrt{\dsum\limits_{j=0}^{\infty
}\dsum\limits_{m_{1}=1}^{T-h-t}a^{\prime }A^{m_{1}}A^{j}J_{d{\LARGE +}%
K}^{\prime }J_{d{\LARGE +}K}\left( A^{j}\right) ^{\prime
}\dsum\limits_{m_{2}=1}^{T-h-t}\left( A^{m_{2}}\right) ^{\prime }a}\right\}
\\
&\leq &\frac{2}{T_{h}^{2}}\dsum\limits_{t=p}^{T-h-1}\sqrt{\overline{C}^{%
\frac{{\Large 2}}{{\Large 3}}}C\frac{1}{1-\phi _{\max }^{2}}}\sqrt{C^{2}%
\frac{1}{1-\phi _{\max }^{2}}\left( \frac{1}{1-\phi _{\max }}\right) ^{2}} \\
&\leq &2\overline{C}^{\frac{{\Large 1}}{{\Large 3}}}C^{\frac{{\Large 3}}{%
{\Large 2}}}\frac{1}{T_{h}}\left( \frac{1}{1-\phi _{\max }^{2}}\right)
\left( \frac{1}{1-\phi _{\max }}\right) \\
&=&O\left( \frac{1}{T}\right)
\end{eqnarray*}

Putting these results together, we obtain%
\begin{eqnarray*}
&&E\left[ \frac{1}{T_{h}}\dsum\limits_{t=p}^{T-h}\dsum\limits_{j=0}^{\infty
}a^{\prime }A^{j}J_{d{\LARGE +}K}^{\prime }\varepsilon _{t-j}\right] ^{2} \\
&=&\frac{1}{T_{h}^{2}}\dsum\limits_{t=p}^{T-h}\dsum\limits_{j=0}^{\infty
}a^{\prime }A^{j}J_{d{\LARGE +}K}^{\prime }E\left[ \varepsilon
_{t-j}\varepsilon _{t-j}^{\prime }\right] J_{d{\LARGE +}K}\left(
A^{j}\right) ^{\prime }a \\
&&+\frac{2}{T_{h}^{2}}\dsum\limits_{t=p}^{T-h-1}\dsum\limits_{j=0}^{\infty
}a^{\prime }A^{j}J_{d{\LARGE +}K}^{\prime }E\left[ \varepsilon
_{t-j}\varepsilon _{t-j}^{\prime }\right] J_{d{\LARGE +}K}\left(
A^{j}\right) ^{\prime }\dsum\limits_{m=1}^{T-h-t}\left( A^{m}\right)
^{\prime }a \\
&=&O\left( \frac{1}{T}\right)
\end{eqnarray*}%
so that, upon applying Markov's inequality, we get%
\begin{equation*}
\frac{1}{T_{h}}\dsum\limits_{t=p}^{T-h}\dsum\limits_{j=0}^{\infty }a^{\prime
}A^{j}J_{d{\LARGE +}K}^{\prime }\varepsilon _{t-j}=O_{p}\left( \frac{1}{%
\sqrt{T}}\right) \text{. }
\end{equation*}%
from which we further deduce, upon applying Slutsky's theorem, that%
\begin{eqnarray*}
\frac{1}{T_{h}}\dsum\limits_{t=p}^{T-h}a^{\prime }\underline{W}_{t}
&=&a^{\prime }\left( I_{\left( d{\LARGE +}K\right) p}-A\right) ^{-1}J_{d%
{\LARGE +}K}^{\prime }\mu +\frac{1}{T_{h}}\dsum\limits_{t=p}^{T-h}\dsum%
\limits_{j=0}^{\infty }a^{\prime }A^{j}J_{d{\LARGE +}K}^{\prime }\varepsilon
_{t-j} \\
&=&a^{\prime }\left( I_{\left( d{\LARGE +}K\right) p}-A\right) ^{-1}J_{d%
{\LARGE +}K}^{\prime }\mu +O_{p}\left( \frac{1}{\sqrt{T}}\right)
\end{eqnarray*}%
Since the above result holds for all $a\in \mathbb{R}^{\left( d{\LARGE +}%
K\right) p}$ such that $\left\Vert a\right\Vert _{2}=1$, we further deduce
that%
\begin{equation*}
\frac{1}{T_{h}}\dsum\limits_{t=p}^{T-h}\underline{W}_{t}=\left( I_{\left( d%
{\LARGE +}K\right) p}-A\right) ^{-1}J_{d{\LARGE +}K}^{\prime }\mu
+O_{p}\left( \frac{1}{\sqrt{T}}\right) .
\end{equation*}

To show part (d), note again that

\begin{eqnarray*}
S_{d}^{\prime }\mathcal{P}_{\left( d{\LARGE +}K\right) p}\underline{W}_{t}
&=&\left( 
\begin{array}{cc}
I_{dp} & \underset{dp\times Kp}{0}%
\end{array}%
\right) \left( 
\begin{array}{c}
\underset{dp\times 1}{\underline{Y}_{t}} \\ 
\underset{Kp\times 1}{\underline{F}_{t}}%
\end{array}%
\right) =\underline{Y}_{t}, \\
S_{K}^{\prime }\mathcal{P}_{\left( d{\LARGE +}K\right) p}\underline{W}_{t}
&=&\left( 
\begin{array}{cc}
\underset{Kp\times dp}{0} & I_{Kp}%
\end{array}%
\right) \left( 
\begin{array}{c}
\underset{dp\times 1}{\underline{Y}_{t}} \\ 
\underset{Kp\times 1}{\underline{F}_{t}}%
\end{array}%
\right) =\underline{F}_{t}
\end{eqnarray*}%
By the result given in part (c) above, it follows by Slutsky's theorem that%
\begin{eqnarray*}
\frac{1}{T_{h}}\dsum\limits_{t=p}^{T-h}\underline{Y}_{t} &=&S_{d}^{\prime }%
\mathcal{P}_{\left( d{\LARGE +}K\right) p}\frac{1}{T_{h}}\dsum%
\limits_{t=p}^{T-h}\underline{W}_{t} \\
&=&S_{d}^{\prime }\mathcal{P}_{\left( d{\LARGE +}K\right) p}\left( I_{\left(
d{\LARGE +}K\right) p}-A\right) ^{-1}J_{d{\LARGE +}K}^{\prime }\mu
+S_{d}^{\prime }\mathcal{P}_{\left( d{\LARGE +}K\right) p}\frac{1}{T_{h}}%
\dsum\limits_{t=p}^{T-h}\dsum\limits_{j=0}^{\infty }A^{j}J_{d{\LARGE +}%
K}^{\prime }\varepsilon _{t-j} \\
&=&S_{d}^{\prime }\mathcal{P}_{\left( d{\LARGE +}K\right) p}\left( I_{\left(
d{\LARGE +}K\right) p}-A\right) ^{-1}J_{d{\LARGE +}K}^{\prime }\mu
+O_{p}\left( \frac{1}{\sqrt{T}}\right) \text{,} \\
\frac{1}{T_{h}}\dsum\limits_{t=p}^{T-h}\underline{F}_{t} &=&S_{K}^{\prime }%
\mathcal{P}_{\left( d{\LARGE +}K\right) p}\frac{1}{T_{h}}\dsum%
\limits_{t=p}^{T-h}\underline{W}_{t} \\
&=&S_{K}^{\prime }\mathcal{P}_{\left( d{\LARGE +}K\right) p}\left( I_{\left(
d{\LARGE +}K\right) p}-A\right) ^{-1}J_{d{\LARGE +}K}^{\prime }\mu
+S_{K}^{\prime }\mathcal{P}_{\left( d{\LARGE +}K\right) p}\frac{1}{T_{h}}%
\dsum\limits_{t=p}^{T-h}\dsum\limits_{j=0}^{\infty }A^{j}J_{d{\LARGE +}%
K}^{\prime }\varepsilon _{t-j} \\
&=&S_{K}^{\prime }\mathcal{P}_{\left( d{\LARGE +}K\right) p}\left( I_{\left(
d{\LARGE +}K\right) p}-A\right) ^{-1}J_{d{\LARGE +}K}^{\prime }\mu
+O_{p}\left( \frac{1}{\sqrt{T}}\right) \text{.}
\end{eqnarray*}

Turning our attention to part (e), let $a\in \mathbb{R}^{\left( d{\LARGE +}%
K\right) p}$ and $b\in \mathbb{R}^{d}$ such that $\left\Vert a\right\Vert
_{2}=1$ and $\left\Vert b\right\Vert _{2}=1$; and, by direct calculation, we
obtain 
\begin{eqnarray*}
&&E\left[ \frac{1}{T_{h}}\dsum\limits_{t=p}^{T-h}a^{\prime }\underline{W}%
_{t}\eta _{t+h}^{\prime }b\right] ^{2} \\
&=&\frac{1}{T_{h}^{2}}\dsum\limits_{t=p}^{T-h}E\left[ \left( a^{\prime }%
\underline{W}_{t}\right) ^{2}\left( \eta _{t+h}^{\prime }b\right) ^{2}\right]
+\frac{2}{T_{h}^{2}}\dsum\limits_{t=p}^{T-h-1}\dsum\limits_{m=1}^{T-h-t}E%
\left\{ \left( a^{\prime }\underline{W}_{t}\eta _{t+h}^{\prime }b\right)
\left( a^{\prime }\underline{W}_{t+m}\eta _{t+m+h}^{\prime }b\right) \right\}
\end{eqnarray*}%
Let $\sigma _{\max }\left( A^{j}\right) $ denotes the max singular value of $%
A^{j}$ and let $\phi _{\max }=\max \left\{ \left\vert \lambda _{\max }\left(
A\right) \right\vert ,\left\vert \lambda _{\min }\left( A\right) \right\vert
\right\} $, and note first that%
\begin{eqnarray}
E\left( b^{\prime }\eta _{t+h}\right) ^{4} &=&E\left(
\dsum\limits_{j=0}^{h-1}b^{\prime }J_{d}A^{j}J_{d{\LARGE +}K}^{\prime
}\varepsilon _{t{\LARGE +}h-j}\right) ^{4}\text{ }  \notag \\
&\leq &h^{3}\dsum\limits_{j=0}^{h-1}E\left[ \left( b^{\prime }J_{d}A^{j}J_{d%
{\LARGE +}K}^{\prime }\varepsilon _{t{\LARGE +}h-j}\right) ^{4}\right] \text{
\ }\left( \text{by Lo\`{e}ve's }c_{r}\text{ inequality}\right)  \notag \\
&\leq &h^{3}\dsum\limits_{j=0}^{h-1}E\left[ \left( b^{\prime }J_{d}A^{j}J_{d%
{\LARGE +}K}^{\prime }J_{d{\LARGE +}K}\left( A^{\prime }\right)
^{j}J_{d}^{\prime }b\right) ^{2}\left( \varepsilon _{t{\LARGE +}h-j}^{\prime
}\varepsilon _{t{\LARGE +}h-j}\right) ^{2}\right]  \notag \\
&=&h^{3}\dsum\limits_{j=0}^{h-1}\left( b^{\prime }J_{d}A^{j}J_{d{\LARGE +}%
K}^{\prime }J_{d{\LARGE +}K}\left( A^{j}\right) ^{\prime }J_{d}^{\prime
}b\right) ^{2}E\left\Vert \varepsilon _{t{\LARGE +}h-j}\right\Vert _{2}^{4} 
\notag \\
&\leq &h^{3}\dsum\limits_{j=0}^{h-1}\left( b^{\prime }J_{d}A^{j}\left(
A^{j}\right) ^{\prime }J_{d}^{\prime }b\right) ^{2}E\left\Vert \varepsilon
_{t{\LARGE +}h-j}\right\Vert _{2}^{4}  \notag \\
&\leq &h^{3}\dsum\limits_{j=0}^{h-1}\sigma _{\max }^{4}\left( A^{j}\right)
\left( b^{\prime }J_{d}J_{d}^{\prime }b\right) ^{2}E\left\Vert \varepsilon
_{t{\LARGE +}h-j}\right\Vert _{2}^{4}  \notag \\
&=&h^{3}\dsum\limits_{j=0}^{h-1}\sigma _{\max }^{4}\left( A^{j}\right)
E\left\Vert \varepsilon _{t{\LARGE +}h-j}\right\Vert _{2}^{4}  \notag \\
&\leq &h^{3}\dsum\limits_{j=0}^{h-1}\overline{C}\left[ \max \left\{
\left\vert \lambda _{\max }\left( A^{j}\right) \right\vert ,\left\vert
\lambda _{\min }\left( A^{j}\right) \right\vert \right\} \right]
^{4}E\left\Vert \varepsilon _{t{\LARGE +}h-j}\right\Vert _{2}^{4}\text{ }%
\left( \text{by Assumption 3-7}\right)  \notag \\
&=&h^{3}\dsum\limits_{j=0}^{h-1}\overline{C}\phi _{\max }^{4j}E\left\Vert
\varepsilon _{t{\LARGE +}h-j}\right\Vert _{2}^{4}  \notag \\
&\leq &C^{\frac{{\large 2}}{{\large 3}}}\overline{C}h^{3}\dsum%
\limits_{j=0}^{h-1}\phi _{\max }^{4j}  \notag \\
&\leq &C^{\ast }  \label{Eeta^4}
\end{eqnarray}%
where the next to last inequality follows from the fact that $E\left\Vert
\varepsilon _{t{\LARGE +}h-j}\right\Vert _{2}^{4}\leq \left(
\sup_{t}E\left\Vert \varepsilon _{t}\right\Vert ^{6}\right) ^{\frac{{\large 2%
}}{{\large 3}}}\leq C^{\frac{{\large 2}}{{\large 3}}}$ by Liapunov's
inequality and by application of Assumption 3-2(b) and where the last
inequality follows from the fact that $h$ is a fixed integer and $0<\phi
_{\max }<1$ in light of Assumption 3-1. Applying the Cauchy-Schwarz
inequality and the existence of moment result given in Lemma C-5, it then
follows that%
\begin{eqnarray*}
\frac{1}{T_{h}^{2}}\dsum\limits_{t=p}^{T-h}E\left[ \left( a^{\prime }%
\underline{W}_{t}\right) ^{2}\left( \eta _{t+h}^{\prime }b\right) ^{2}\right]
&\leq &\frac{1}{T_{h}^{2}}\dsum\limits_{t=p}^{T-h}\sqrt{E\left( a^{\prime }%
\underline{W}_{t}\underline{W}_{t}^{\prime }a\right) ^{2}}\sqrt{E\left(
b^{\prime }\eta _{t+h}\right) ^{4}} \\
&\leq &\frac{1}{T_{h}^{2}}\dsum\limits_{t=p}^{T-h}\sqrt{E\left\Vert 
\underline{W}_{t}\right\Vert _{2}^{4}}\sqrt{E\left( b^{\prime }\eta
_{t+h}\right) ^{4}} \\
&\leq &\frac{C}{T_{h}}=\frac{C}{T-h-p+1}=O\left( \frac{1}{T}\right)
\end{eqnarray*}%
Next, observe that%
\begin{eqnarray*}
&&E\left\{ \left( a^{\prime }\underline{W}_{t}\eta _{t+h}^{\prime }b\right)
\left( a^{\prime }\underline{W}_{t+m}\eta _{t+m+h}^{\prime }b\right) \right\}
\\
&=&E\left\{ \left( a^{\prime }\underline{W}_{t}\eta _{t+h}^{\prime }b\right)
\left( a^{\prime }\underline{W}_{t+m}\eta _{t+m+h}^{\prime }b\right) \right\}
\\
&=&E\left\{ a^{\prime }\underline{W}_{t}\underline{W}_{t+m}^{\prime
}a\dsum\limits_{j=0}^{h-1}b^{\prime }J_{d}A^{j}J_{d{\LARGE +}K}^{\prime
}\varepsilon _{t{\LARGE +}h-j}\dsum\limits_{k=0}^{h-1}b^{\prime
}J_{d}A^{j}J_{d{\LARGE +}K}^{\prime }\varepsilon _{t{\LARGE +}m{\LARGE +}%
h-k}\right\} \\
&=&E\left\{ a^{\prime }\underline{W}_{t}\underline{W}_{t+m}^{\prime
}a\dsum\limits_{j=0}^{h-1}\dsum\limits_{k=0}^{h-1}b^{\prime }J_{d}A^{j}J_{d%
{\LARGE +}K}^{\prime }\varepsilon _{t{\LARGE +}h-j}\varepsilon _{t{\LARGE +}m%
{\LARGE +}h-k}^{\prime }J_{d{\LARGE +}K}\left( A^{j}\right) ^{\prime
}J_{d}^{\prime }b\right\} \text{,}
\end{eqnarray*}%
so that, for $m\geq h$, we have%
\begin{eqnarray*}
&&E\left\{ \left( a^{\prime }\underline{W}_{t}\eta _{t+h}^{\prime }b\right)
\left( a^{\prime }\underline{W}_{t+m}\eta _{t+m+h}^{\prime }b\right) \right\}
\\
&=&E\left\{ a^{\prime }\underline{W}_{t}\underline{W}_{t+m}^{\prime
}a\dsum\limits_{j=0}^{h-1}\dsum\limits_{k=0}^{h-1}b^{\prime }J_{d}A^{j}J_{d%
{\LARGE +}K}^{\prime }\varepsilon _{t{\LARGE +}h-j}\varepsilon _{t{\LARGE +}m%
{\LARGE +}h-k}^{\prime }J_{d{\LARGE +}K}\left( A^{j}\right) ^{\prime
}J_{d}^{\prime }b\right\} \\
&=&E\left\{ a^{\prime }\underline{W}_{t}\underline{W}_{t+m}^{\prime
}a\dsum\limits_{j=0}^{h-1}\dsum\limits_{k=0}^{h-1}b^{\prime }J_{d}A^{j}J_{d%
{\LARGE +}K}^{\prime }\varepsilon _{t{\LARGE +}h-j}E\left[ \varepsilon _{t%
{\LARGE +}m{\LARGE +}h-k}^{\prime }|\mathcal{F}_{-\infty }^{t{\LARGE +}m}%
\right] J_{d{\LARGE +}K}\left( A^{j}\right) ^{\prime }J_{d}^{\prime
}b\right\} \\
&=&E\left\{ a^{\prime }\underline{W}_{t}\underline{W}_{t+m}^{\prime
}a\dsum\limits_{j=0}^{h-1}\dsum\limits_{k=0}^{h-1}b^{\prime }J_{d}A^{j}J_{d%
{\LARGE +}K}^{\prime }\varepsilon _{t{\LARGE +}h-j}E\left[ \varepsilon _{t%
{\LARGE +}m{\LARGE +}h-k}^{\prime }\right] J_{d{\LARGE +}K}\left(
A^{j}\right) ^{\prime }J_{d}^{\prime }b\right\} \\
&=&0
\end{eqnarray*}%
Hence, defining $\dsum\nolimits_{m=1}^{0}E\left\vert \left( a^{\prime }%
\underline{W}_{t}\eta _{t+h}^{\prime }b\right) \left( a^{\prime }\underline{W%
}_{t+m}\eta _{t+m+h}^{\prime }b\right) \right\vert =0$, we have%
\begin{eqnarray*}
&&\left\vert \frac{2}{T_{h}^{2}}\dsum\limits_{t=p}^{T-h-1}\dsum%
\limits_{m=1}^{T-h-t}E\left\{ \left( a^{\prime }\underline{W}_{t}\eta
_{t+h}^{\prime }b\right) \left( a^{\prime }\underline{W}_{t+m}\eta
_{t+m+h}^{\prime }b\right) \right\} \right\vert \\
&=&\left\vert \frac{2}{T_{h}^{2}}\dsum\limits_{t=p}^{T-h-1}\dsum%
\limits_{m=1}^{\min \left\{ h-1,T-h-t\right\} }E\left\{ \left( a^{\prime }%
\underline{W}_{t}\eta _{t+h}^{\prime }b\right) \left( a^{\prime }\underline{W%
}_{t+m}\eta _{t+m+h}^{\prime }b\right) \right\} \right\vert \\
&\leq &\frac{2}{T_{h}^{2}}\dsum\limits_{t=p}^{T-h-1}\dsum\limits_{m=1}^{\min
\left\{ h-1,T-h-t\right\} }E\left\vert \left( a^{\prime }\underline{W}%
_{t}\eta _{t+h}^{\prime }b\right) \left( a^{\prime }\underline{W}_{t+m}\eta
_{t+m+h}^{\prime }b\right) \right\vert \\
&\leq &\frac{2}{T_{h}^{2}}\dsum\limits_{t=p}^{T-h-1}\dsum\limits_{m=1}^{\min
\left\{ h-1,T-h-t\right\} }\sqrt{E\left( a^{\prime }\underline{W}_{t}%
\underline{W}_{t+m}^{\prime }a\right) ^{2}}\sqrt{E\left( b^{\prime }\eta
_{t+h}\eta _{t+m+h}^{\prime }b\right) ^{2}} \\
&=&\frac{2}{T_{h}^{2}}\dsum\limits_{t=p}^{T-h-1}\dsum\limits_{m=1}^{\min
\left\{ h-1,T-h-t\right\} }\sqrt{E\left( a^{\prime }\underline{W}_{t}%
\underline{W}_{t}^{\prime }aa^{\prime }\underline{W}_{t+m}\underline{W}%
_{t+m}^{\prime }a\right) }\sqrt{E\left\{ \left( b^{\prime }\eta
_{t+h}\right) ^{2}\left( b^{\prime }\eta _{t+m+h}\right) ^{2}\right\} } \\
&\leq &\frac{2}{T_{h}^{2}}\dsum\limits_{t=p}^{T-h-1}\dsum\limits_{m=1}^{\min
\left\{ h-1,T-h-t\right\} }\sqrt{E\left( \left\Vert \underline{W}%
_{t}\right\Vert _{2}^{2}\left\Vert \underline{W}_{t+m}\right\Vert
_{2}^{2}\right) }\sqrt{E\left\{ \left( b^{\prime }\eta _{t+h}\right)
^{2}\left( b^{\prime }\eta _{t+m+h}\right) ^{2}\right\} } \\
&\leq &\frac{2}{T_{h}^{2}}\dsum\limits_{t=p}^{T-h-1}\dsum\limits_{m=1}^{\min
\left\{ h-1,T-h-t\right\} }\left( E\left\Vert \underline{W}_{t}\right\Vert
_{2}^{4}\right) ^{\frac{{\Large 1}}{{\Large 4}}}\left( E\left\Vert 
\underline{W}_{t+m}\right\Vert _{2}^{4}\right) ^{\frac{{\Large 1}}{{\Large 4}%
}}\left( E\left( b^{\prime }\eta _{t+h}\right) ^{4}\right) ^{\frac{{\Large 1}%
}{{\Large 4}}}\left( E\left( b^{\prime }\eta _{t+m+h}\right) ^{4}\right) ^{%
\frac{{\Large 1}}{{\Large 4}}} \\
&\leq &\frac{2\left( T-h-p\right) \left( h-1\right) }{T_{h}^{2}}\overline{C}%
\text{ \ }\left( \text{applying Lemma C-5 and expression (\ref{Eeta^4}) above%
}\right) \\
&<&\frac{2\left( h-1\right) \overline{C}}{T_{h}}\text{ \ }\left( \text{since 
}T_{h}=T-h-p+1\right) \\
&=&O\left( \frac{1}{T}\right)
\end{eqnarray*}

\noindent It follows that%
\begin{eqnarray*}
&&E\left[ \frac{1}{T_{h}}\dsum\limits_{t=p}^{T-h}a^{\prime }\underline{W}%
_{t}\eta _{t+h}^{\prime }b\right] ^{2} \\
&=&\frac{1}{T_{h}^{2}}\dsum\limits_{t=p}^{T-h}E\left[ \left( a^{\prime }%
\underline{W}_{t}\right) ^{2}\left( \eta _{t+h}^{\prime }b\right) ^{2}\right]
+\frac{2}{T_{h}^{2}}\dsum\limits_{t=p}^{T-h-1}\dsum\limits_{m=1}^{T-h-t}E%
\left\{ \left( a^{\prime }\underline{W}_{t}\eta _{t+h}^{\prime }b\right)
\left( a^{\prime }\underline{W}_{t+m}\eta _{t+m+h}^{\prime }b\right) \right\}
\\
&=&O\left( \frac{1}{T}\right)
\end{eqnarray*}%
so that, applying Markov's inequality, we get%
\begin{equation*}
\frac{1}{T_{h}}\dsum\limits_{t=p}^{T-h}a^{\prime }\underline{W}_{t}\eta
_{t+h}^{\prime }b=O_{p}\left( \frac{1}{\sqrt{T}}\right)
\end{equation*}%
Since this result holds for every $a\in \mathbb{R}^{\left( d{\LARGE +}%
K\right) p}$ and $b\in \mathbb{R}^{d}$ such that $\left\Vert a\right\Vert
_{2}=1$ and $\left\Vert b\right\Vert _{2}=1$, we further deduce that%
\begin{equation*}
\frac{1}{T_{h}}\dsum\limits_{t=p}^{T-h}\underline{W}_{t}\eta _{t+h}^{\prime
}=O_{p}\left( \frac{1}{\sqrt{T}}\right) \text{.}
\end{equation*}

Now, for part (f), note that

\begin{eqnarray*}
S_{d}^{\prime }\mathcal{P}_{\left( d{\LARGE +}K\right) p}\underline{W}_{t}
&=&\left( 
\begin{array}{cc}
I_{dp} & \underset{dp\times Kp}{0}%
\end{array}%
\right) \left( 
\begin{array}{c}
\underset{dp\times 1}{\underline{Y}_{t}} \\ 
\underset{Kp\times 1}{\underline{F}_{t}}%
\end{array}%
\right) =\underline{Y}_{t}, \\
S_{K}^{\prime }\mathcal{P}_{\left( d{\LARGE +}K\right) p}\underline{W}_{t}
&=&\left( 
\begin{array}{cc}
\underset{Kp\times dp}{0} & I_{Kp}%
\end{array}%
\right) \left( 
\begin{array}{c}
\underset{dp\times 1}{\underline{Y}_{t}} \\ 
\underset{Kp\times 1}{\underline{F}_{t}}%
\end{array}%
\right) =\underline{F}_{t}
\end{eqnarray*}%
Hence, it follows by applying the result given in part (e) above and the
Slutsky's theorem that%
\begin{eqnarray*}
\frac{1}{T_{h}}\dsum\limits_{t=p}^{T-h}\underline{Y}_{t}\eta _{t+h}^{\prime
} &=&S_{d}^{\prime }\mathcal{P}_{\left( d{\LARGE +}K\right) p}\frac{1}{T_{h}}%
\dsum\limits_{t=p}^{T-h}\underline{W}_{t}\eta _{t+h}^{\prime }=O_{p}\left( 
\frac{1}{\sqrt{T}}\right) \text{ and} \\
\frac{1}{T_{h}}\dsum\limits_{t=p}^{T-h}\underline{F}_{t}\eta _{t+h}^{\prime
} &=&S_{K}^{\prime }\mathcal{P}_{\left( d{\LARGE +}K\right) p}\frac{1}{T_{h}}%
\dsum\limits_{t=p}^{T-h}\underline{W}_{t}\eta _{t+h}^{\prime }=O_{p}\left( 
\frac{1}{\sqrt{T}}\right)
\end{eqnarray*}

To show part (g), let $b\in \mathbb{R}^{d}$ such that $\left\Vert
b\right\Vert _{2}=1$ and write%
\begin{eqnarray*}
E\left( \frac{b^{\prime }\mathfrak{H}^{\prime }\iota _{T_{h}}}{\sqrt{T_{h}}}%
\right) ^{2} &=&E\left( \frac{1}{\sqrt{T_{h}}}\dsum\limits_{t=p}^{T-h}b^{%
\prime }\eta _{t{\LARGE +}h}\right) ^{2} \\
&=&\frac{1}{T_{h}}\dsum\limits_{t=p}^{T-h}\dsum\limits_{s=p}^{T-h}\dsum%
\limits_{j=0}^{h-1}\dsum\limits_{k=0}^{h-1}b^{\prime }J_{d}A^{j}J_{d{\LARGE +%
}K}^{\prime }E\left[ \varepsilon _{t{\LARGE +}h-j}\varepsilon _{s{\LARGE +}%
h-k}^{\prime }\right] J_{d{\LARGE +}K}\left( A^{k}\right) ^{\prime
}J_{d}^{\prime }b \\
&=&\frac{1}{T_{h}}\dsum\limits_{t=p}^{T-h}\dsum\limits_{j=0}^{h-1}b^{\prime
}J_{d}A^{j}J_{d{\LARGE +}K}^{\prime }E\left[ \varepsilon _{t{\LARGE +}%
h-j}\varepsilon _{t{\LARGE +}h-j}^{\prime }\right] J_{d{\LARGE +}K}\left(
A^{j}\right) ^{\prime }J_{d}^{\prime }b \\
&&+\frac{2}{T_{h}}\dsum\limits_{t=p}^{T-h-1}\dsum\limits_{m=1}^{T-h-t}\dsum%
\limits_{j=0}^{\max \left\{ 0,h-2\right\} }b^{\prime }J_{d}A^{j}J_{d{\LARGE +%
}K}^{\prime }E\left[ \varepsilon _{t{\LARGE +}h-j}\varepsilon _{t{\LARGE +}%
h-j}^{\prime }\right] J_{d{\LARGE +}K}\left( A^{m{\LARGE +}j}\right)
^{\prime }J_{d}^{\prime }b \\
&=&\frac{1}{T_{h}}\dsum\limits_{t=p}^{T-h}\dsum\limits_{j=0}^{h-1}b^{\prime
}J_{d}A^{j}J_{d{\LARGE +}K}^{\prime }E\left[ \varepsilon _{t{\LARGE +}%
h-j}\varepsilon _{t{\LARGE +}h-j}^{\prime }\right] J_{d{\LARGE +}K}\left(
A^{j}\right) ^{\prime }J_{d}^{\prime }b \\
&&+\frac{2}{T_{h}}\dsum\limits_{t=p}^{T-h-1}\dsum\limits_{j=0}^{\max \left\{
0,h-2\right\} }b^{\prime }J_{d}A^{j}J_{d{\LARGE +}K}^{\prime }E\left[
\varepsilon _{t{\LARGE +}h-j}\varepsilon _{t{\LARGE +}h-j}^{\prime }\right]
J_{d{\LARGE +}K}\left( A^{j}\right) ^{\prime
}\dsum\limits_{m=1}^{T-h-t}\left( A^{m}\right) ^{\prime }J_{d}^{\prime }b
\end{eqnarray*}%
Now,%
\begin{eqnarray*}
&&\frac{1}{T_{h}}\dsum\limits_{t=p}^{T-h}\dsum\limits_{j=0}^{h-1}b^{\prime
}J_{d}A^{j}J_{d{\LARGE +}K}^{\prime }E\left[ \varepsilon _{t{\LARGE +}%
h-j}\varepsilon _{t{\LARGE +}h-j}^{\prime }\right] J_{d{\LARGE +}K}\left(
A^{j}\right) ^{\prime }J_{d}^{\prime }b \\
&\leq &\frac{1}{T_{h}}\dsum\limits_{t=p}^{T-h}\dsum\limits_{j=0}^{h-1}E\left%
\Vert \varepsilon _{t{\LARGE +}h-j}\right\Vert _{2}^{2}b^{\prime
}J_{d}A^{j}J_{d{\LARGE +}K}^{\prime }J_{d{\LARGE +}K}\left( A^{j}\right)
^{\prime }J_{d}^{\prime }b \\
&\leq &\frac{1}{T_{h}}\dsum\limits_{t=p}^{T-h}\dsum\limits_{j=0}^{h-1}\left(
E\left\Vert \varepsilon _{t{\LARGE +}h-j}\right\Vert _{2}^{6}\right) ^{\frac{%
{\Large 1}}{{\Large 3}}}b^{\prime }J_{d}A^{j}\left( A^{j}\right) ^{\prime
}J_{d}^{\prime }b \\
&&\left( \text{by Liapunov's inequality and the fact that }\lambda _{\max
}\left( J_{d{\LARGE +}K}^{\prime }J_{d{\LARGE +}K}\right) =1\right) \\
&\leq &\overline{C}^{\frac{{\Large 1}}{{\Large 3}}}\frac{1}{T_{h}}%
\dsum\limits_{t=p}^{T-h}\dsum\limits_{j=0}^{h-1}b^{\prime }J_{d}A^{j}\left(
A^{j}\right) ^{\prime }J_{d}^{\prime }b \\
&&\left( \text{where }\overline{C}\geq 1\text{ is a constant such that }%
E\left\Vert \varepsilon _{t-j}\right\Vert _{2}^{6}\leq \overline{C}<\infty 
\text{ by Assumption 3-2(b)}\right) \\
&\leq &\overline{C}^{\frac{{\Large 1}}{{\Large 3}}}\frac{1}{T_{h}}%
\dsum\limits_{t=p}^{T-h}\dsum\limits_{j=0}^{h-1}\lambda _{\max }\left\{
A^{j}\left( A^{j}\right) ^{\prime }\right\} b^{\prime }J_{d}J_{d}^{\prime }b
\\
&=&\overline{C}^{\frac{{\Large 1}}{{\Large 3}}}\frac{1}{T_{h}}%
\dsum\limits_{t=p}^{T-h}\dsum\limits_{j=0}^{h-1}\lambda _{\max }\left\{
\left( A^{j}\right) ^{\prime }A^{j}\right\} \text{ } \\
&&\left( \text{since }\lambda _{\max }\left\{ A^{j}\left( A^{j}\right)
^{\prime }\right\} =\lambda _{\max }\left\{ \left( A^{j}\right) ^{\prime
}A^{j}\right\} \text{, }\lambda _{\max }\left( J_{d}J_{d}^{\prime }\right) =1%
\text{, and }b^{\prime }b=1\right) \\
&=&\overline{C}^{\frac{{\Large 1}}{{\Large 3}}}\frac{1}{T_{h}}%
\dsum\limits_{t=p}^{T-h}\dsum\limits_{j=0}^{h-1}\sigma _{\max }^{2}\left(
A^{j}\right) \\
&\leq &\overline{C}^{\frac{{\Large 1}}{{\Large 3}}}\frac{1}{T_{h}}%
\dsum\limits_{t=p}^{T-h}\dsum\limits_{j=0}^{h-1}C\max \left\{ \left\vert
\lambda _{\max }\left( A^{j}\right) \right\vert ^{2},\left\vert \lambda
_{\min }\left( A^{j}\right) \right\vert ^{2}\right\} \text{ }\left( \text{by
Assumption 3-7}\right) \\
&=&\overline{C}^{\frac{{\Large 1}}{{\Large 3}}}C\frac{1}{T_{h}}%
\dsum\limits_{t=p}^{T-h}\dsum\limits_{j=0}^{h-1}\max \left\{ \left\vert
\lambda _{\max }\left( A\right) \right\vert ^{2j},\left\vert \lambda _{\min
}\left( A\right) \right\vert ^{2j}\right\} \\
&=&\overline{C}^{\frac{{\Large 1}}{{\Large 3}}}C\frac{1}{T_{h}}%
\dsum\limits_{t=p}^{T-h}\dsum\limits_{j=0}^{h-1}\phi _{\max }^{2j}
\end{eqnarray*}%
where $\phi _{\max }=\max \left\{ \left\vert \lambda _{\max }\left( A\right)
\right\vert ,\left\vert \lambda _{\min }\left( A\right) \right\vert \right\} 
$ and where $0<\phi _{\max }<1$ since Assumption 3-1 implies that all
eigenvalues of $A$ have modulus less than $1$. It follows that%
\begin{eqnarray*}
\frac{1}{T_{h}}\dsum\limits_{t=p}^{T-h}\dsum\limits_{j=0}^{h-1}b^{\prime
}J_{d}A^{j}J_{d{\LARGE +}K}^{\prime }E\left[ \varepsilon _{t{\LARGE +}%
h-j}\varepsilon _{t{\LARGE +}h-j}^{\prime }\right] J_{d{\LARGE +}K}\left(
A^{j}\right) ^{\prime }J_{d}^{\prime }b &\leq &\overline{C}^{\frac{{\Large 1}%
}{{\Large 3}}}C\frac{1}{T_{h}}\dsum\limits_{t=p}^{T-h}\dsum%
\limits_{j=0}^{h-1}\phi _{\max }^{2j} \\
&\leq &\overline{C}^{\frac{{\Large 1}}{{\Large 3}}}C\frac{T-h-p+1}{T_{h}}%
\frac{1}{1-\phi _{\max }^{2}} \\
&=&\overline{C}^{\frac{{\Large 1}}{{\Large 3}}}C\frac{1}{1-\phi _{\max }^{2}}%
\text{ \ } \\
&&\left( \text{since }T_{h}=T-h-p+1\right) \\
&=&O\left( 1\right)
\end{eqnarray*}%
Moreover, write%
\begin{eqnarray*}
&&\left\vert \frac{2}{T_{h}}\dsum\limits_{t=p}^{T-h-1}\dsum\limits_{j=0}^{%
\max \left\{ 0,h-2\right\} }b^{\prime }J_{d}A^{j}J_{d{\LARGE +}K}^{\prime }E%
\left[ \varepsilon _{t{\LARGE +}h-j}\varepsilon _{t{\LARGE +}h-j}^{\prime }%
\right] J_{d{\LARGE +}K}\left( A^{j}\right) ^{\prime
}\dsum\limits_{m=1}^{T-h-t}\left( A^{m}\right) ^{\prime }J_{d}^{\prime
}b\right\vert \\
&\leq &\frac{2}{T_{h}}\dsum\limits_{t=p}^{T-h-1}\left\vert
\dsum\limits_{j=0}^{\max \left\{ 0,h-2\right\} }b^{\prime }J_{d}A^{j}J_{d%
{\LARGE +}K}^{\prime }E\left[ \varepsilon _{t{\LARGE +}h-j}\varepsilon _{t%
{\LARGE +}h-j}^{\prime }\right] J_{d{\LARGE +}K}\left( A^{j}\right) ^{\prime
}\dsum\limits_{m=1}^{T-h-t}\left( A^{m}\right) ^{\prime }J_{d}^{\prime
}b\right\vert \\
&\leq &\frac{2}{T_{h}}\dsum\limits_{t=p}^{T-h-1}\left\{ \sqrt{%
\dsum\limits_{j=0}^{\max \left\{ 0,h-2\right\} }b^{\prime }J_{d}A^{j}J_{d%
{\LARGE +}K}^{\prime }E\left[ \varepsilon _{t{\LARGE +}h-j}\varepsilon _{t%
{\LARGE +}h-j}^{\prime }\right] E\left[ \varepsilon _{t{\LARGE +}%
h-j}\varepsilon _{t{\LARGE +}h-j}^{\prime }\right] J_{d{\LARGE +}K}\left(
A^{j}\right) ^{\prime }J_{d}^{\prime }b}\right. \\
&&\left. \times \sqrt{\dsum\limits_{j=0}^{\max \left\{ 0,h-2\right\}
}\dsum\limits_{m_{1}=1}^{T-h-t}\dsum\limits_{m_{2}=1}^{T-h-t}b^{\prime
}J_{d}A^{m_{1}}A^{j}J_{d{\LARGE +}K}^{\prime }J_{d{\LARGE +}K}\left(
A^{j}\right) ^{\prime }\left( A^{m_{2}}\right) ^{\prime }J_{d}^{\prime }b}%
\right\}
\end{eqnarray*}%
Similar to the argument given previously, we have%
\begin{eqnarray*}
&&\dsum\limits_{j=0}^{\max \left\{ 0,h-2\right\} }b^{\prime }J_{d}A^{j}J_{d%
{\LARGE +}K}^{\prime }E\left[ \varepsilon _{t{\LARGE +}h-j}\varepsilon _{t%
{\LARGE +}h-j}^{\prime }\right] E\left[ \varepsilon _{t{\LARGE +}%
h-j}\varepsilon _{t{\LARGE +}h-j}^{\prime }\right] J_{d{\LARGE +}K}\left(
A^{j}\right) ^{\prime }J_{d}^{\prime }b \\
&\leq &\dsum\limits_{j=0}^{\max \left\{ 0,h-2\right\} }\lambda _{\max
}\left( E\left[ \varepsilon _{t{\LARGE +}h-j}\varepsilon _{t{\LARGE +}%
h-j}^{\prime }\right] E\left[ \varepsilon _{t{\LARGE +}h-j}\varepsilon _{t%
{\LARGE +}h-j}^{\prime }\right] \right) b^{\prime }J_{d}A^{j}J_{d{\LARGE +}%
K}^{\prime }J_{d{\LARGE +}K}\left( A^{j}\right) ^{\prime }J_{d}^{\prime }b \\
&\leq &\dsum\limits_{j=0}^{\max \left\{ 0,h-2\right\} }\lambda _{\max
}\left( E\left[ \varepsilon _{t{\LARGE +}h-j}\varepsilon _{t{\LARGE +}%
h-j}^{\prime }\right] E\left[ \varepsilon _{t{\LARGE +}h-j}\varepsilon _{t%
{\LARGE +}h-j}^{\prime }\right] \right) C\phi _{\max }^{2j} \\
&=&C\dsum\limits_{j=0}^{\max \left\{ 0,h-2\right\} }\lambda _{\max
}^{2}\left( E\left[ \varepsilon _{t{\LARGE +}h-j}\varepsilon _{t{\LARGE +}%
h-j}^{\prime }\right] \right) \phi _{\max }^{2j} \\
&\leq &C\dsum\limits_{j=0}^{\max \left\{ 0,h-2\right\} }\left( tr\left\{ E 
\left[ \varepsilon _{t{\LARGE +}h-j}\varepsilon _{t{\LARGE +}h-j}^{\prime }%
\right] \right\} \right) ^{2}\phi _{\max }^{2j} \\
&=&C\dsum\limits_{j=0}^{\max \left\{ 0,h-2\right\} }\left( E\left\Vert
\varepsilon _{t{\LARGE +}h-j}\right\Vert _{2}^{2}\right) ^{2}\phi _{\max
}^{2j} \\
&\leq &C\dsum\limits_{j=0}^{\max \left\{ 0,h-2\right\} }\left( E\left\Vert
\varepsilon _{t{\LARGE +}h-j}\right\Vert _{2}^{6}\right) ^{\frac{{\Large 2}}{%
{\Large 3}}}\phi _{\max }^{2j} \\
&\leq &\overline{C}^{\frac{{\Large 2}}{{\Large 3}}}C\frac{1}{1-\phi _{\max
}^{2}}
\end{eqnarray*}%
and%
\begin{eqnarray*}
&&\dsum\limits_{j=0}^{\max \left\{ 0,h-2\right\}
}\dsum\limits_{m_{1}=1}^{T-h-t}\dsum\limits_{m_{2}=1}^{T-h-t}b^{\prime
}J_{d}A^{m_{1}}A^{j}J_{d{\LARGE +}K}^{\prime }J_{d{\LARGE +}K}\left(
A^{j}\right) ^{\prime }\left( A^{m_{2}}\right) ^{\prime }J_{d}^{\prime }b \\
&\leq &\dsum\limits_{j=0}^{\max \left\{ 0,h-2\right\}
}\dsum\limits_{m_{1}=1}^{T-h-t}\dsum\limits_{m_{2}=1}^{T-h-t}b^{\prime
}J_{d}A^{m_{1}}A^{j}\left( A^{j}\right) ^{\prime }\left( A^{m_{2}}\right)
^{\prime }J_{d}^{\prime }b \\
&\leq &C\dsum\limits_{j=0}^{\max \left\{ 0,h-2\right\} }\phi _{\max
}^{2j}\dsum\limits_{m_{1}=1}^{T-h-t}\dsum\limits_{m_{2}=1}^{T-h-t}\left\vert
b^{\prime }J_{d}A^{m_{1}}\left( A^{m_{2}}\right) ^{\prime }J_{d}^{\prime
}b\right\vert \\
&\leq &C\dsum\limits_{j=0}^{\max \left\{ 0,h-2\right\} }\phi _{\max
}^{2j}\dsum\limits_{m_{1}=1}^{T-h-t}\dsum\limits_{m_{2}=1}^{T-h-t}\sqrt{%
b^{\prime }J_{d}A^{m_{1}}\left( A^{m_{1}}\right) ^{\prime }J_{d}^{\prime }b}%
\sqrt{b^{\prime }J_{d}A^{m_{2}}\left( A^{m_{2}}\right) ^{\prime
}J_{d}^{\prime }b} \\
&\leq &C\dsum\limits_{j=0}^{\max \left\{ 0,h-2\right\} }\phi _{\max
}^{2j}\dsum\limits_{m_{1}=1}^{T-h-t}\dsum\limits_{m_{2}=1}^{T-h-t}\sqrt{%
C\phi _{\max }^{2m_{1}}}\sqrt{C\phi _{\max }^{2m_{2}}} \\
&\leq &C^{2}\dsum\limits_{j=0}^{\max \left\{ 0,h-2\right\} }\phi _{\max
}^{2j}\dsum\limits_{m_{1}=1}^{T-h-t}\phi _{\max
}^{m_{1}}\dsum\limits_{m_{2}=1}^{T-h-t}\phi _{\max }^{m_{2}} \\
&\leq &C^{2}\frac{1}{1-\phi _{\max }^{2}}\left( \frac{1}{1-\phi _{\max }}%
\right) ^{2}
\end{eqnarray*}%
It follows that%
\begin{eqnarray*}
&&\left\vert \frac{2}{T_{h}}\dsum\limits_{t=p}^{T-h-1}\dsum%
\limits_{j=0}^{h-2}b^{\prime }J_{d}A^{j}J_{d{\LARGE +}K}^{\prime }E\left[
\varepsilon _{t{\LARGE +}h-j}\varepsilon _{t{\LARGE +}h-j}^{\prime }\right]
J_{d{\LARGE +}K}\left( A^{j}\right) ^{\prime
}\dsum\limits_{m=1}^{T-h-t}\left( A^{m}\right) ^{\prime }J_{d}^{\prime
}b\right\vert \\
&\leq &\frac{2}{T_{h}}\dsum\limits_{t=p}^{T-h-1}\left\{ \sqrt{%
\dsum\limits_{j=0}^{\max \left\{ 0,h-2\right\} }b^{\prime }J_{d}A^{j}J_{d%
{\LARGE +}K}^{\prime }E\left[ \varepsilon _{t{\LARGE +}h-j}\varepsilon _{t%
{\LARGE +}h-j}^{\prime }\right] E\left[ \varepsilon _{t{\LARGE +}%
h-j}\varepsilon _{t{\LARGE +}h-j}^{\prime }\right] J_{d{\LARGE +}K}\left(
A^{j}\right) ^{\prime }J_{d}^{\prime }b}\right. \\
&&\left. \times \sqrt{\dsum\limits_{j=0}^{\max \left\{ 0,h-2\right\}
}\dsum\limits_{m_{1}=1}^{T-h-t}\dsum\limits_{m_{2}=1}^{T-h-t}b^{\prime
}J_{d}A^{m_{1}}A^{j}J_{d{\LARGE +}K}^{\prime }J_{d{\LARGE +}K}\left(
A^{j}\right) ^{\prime }\left( A^{m_{2}}\right) ^{\prime }J_{d}^{\prime }b}%
\right\} \\
&\leq &\frac{2}{T_{h}}\dsum\limits_{t=p}^{T-h-1}\sqrt{\overline{C}^{\frac{%
{\Large 2}}{{\Large 3}}}C\frac{1}{1-\phi _{\max }^{2}}}\sqrt{C^{2}\frac{1}{%
1-\phi _{\max }^{2}}\left( \frac{1}{1-\phi _{\max }}\right) ^{2}} \\
&=&2\overline{C}^{\frac{{\Large 1}}{{\Large 3}}}C^{\frac{{\Large 3}}{{\Large %
2}}}\frac{T-h-p+1}{T_{h}}\left( \frac{1}{1-\phi _{\max }^{2}}\right) \left( 
\frac{1}{1-\phi _{\max }}\right) \\
&=&O\left( 1\right)
\end{eqnarray*}

Putting these results together, we obtain%
\begin{eqnarray*}
&&E\left( \frac{1}{\sqrt{T_{h}}}\dsum\limits_{t=p}^{T-h}b^{\prime }\eta _{t%
{\LARGE +}h}\right) ^{2} \\
&=&\frac{1}{T_{h}}\dsum\limits_{t=p}^{T-h}\dsum\limits_{j=0}^{h-1}b^{\prime
}J_{d}A^{j}J_{d{\LARGE +}K}^{\prime }E\left[ \varepsilon _{t{\LARGE +}%
h-j}\varepsilon _{t{\LARGE +}h-j}^{\prime }\right] J_{d{\LARGE +}K}\left(
A^{j}\right) ^{\prime }J_{d}^{\prime }b \\
&&+\frac{2}{T_{h}}\dsum\limits_{t=p}^{T-h-1}\dsum\limits_{j=0}^{\max \left\{
0,h-2\right\} }b^{\prime }J_{d}A^{j}J_{d{\LARGE +}K}^{\prime }E\left[
\varepsilon _{t{\LARGE +}h-j}\varepsilon _{t{\LARGE +}h-j}^{\prime }\right]
J_{d{\LARGE +}K}\left( A^{j}\right) ^{\prime
}\dsum\limits_{m=1}^{T-h-t}\left( A^{m}\right) ^{\prime }J_{d}^{\prime }b \\
&=&O\left( 1\right)
\end{eqnarray*}%
so that, upon applying Markov's inequality, we get%
\begin{equation*}
\frac{1}{\sqrt{T_{h}}}\dsum\limits_{t=p}^{T-h}b^{\prime }\eta _{t{\LARGE +}%
h}=O_{p}\left( 1\right) \text{. }
\end{equation*}%
Since the above result holds for all $b\in \mathbb{R}^{d}$ such that $%
\left\Vert b\right\Vert _{2}=1$, we further deduce that%
\begin{equation*}
\frac{1}{\sqrt{T_{h}}}\dsum\limits_{t=p}^{T-h}\eta _{t{\LARGE +}%
h}=O_{p}\left( 1\right)
\end{equation*}%
and that%
\begin{equation*}
\frac{1}{T_{h}}\dsum\limits_{t=p}^{T-h}\eta _{t{\LARGE +}h}=\frac{1}{\sqrt{%
T_{h}}}\left( \frac{1}{\sqrt{T_{h}}}\dsum\limits_{t=p}^{T-h}\eta _{t{\LARGE +%
}h}\right) =O_{p}\left( \frac{1}{\sqrt{T}}\right) =o_{p}\left( 1\right) 
\text{. }
\end{equation*}

Lastly, to show part (h), let $a,b\in \mathbb{R}^{d}$ such that $\left\Vert
a\right\Vert _{2}=\left\Vert b\right\Vert _{2}=1$; and write%
\begin{eqnarray*}
&&E\left[ \frac{1}{T_{h}}\dsum\limits_{t=p}^{T-h}\left( a^{\prime }\eta
_{t+h}\eta _{t+h}^{\prime }b-E\left[ a^{\prime }\eta _{t+h}\eta
_{t+h}^{\prime }b\right] \right) \right] ^{2} \\
&=&\frac{1}{T_{h}^{2}}\dsum\limits_{t=p}^{T-h}E\left[ \left( a^{\prime }\eta
_{t+h}\eta _{t+h}^{\prime }b-E\left[ a^{\prime }\eta _{t+h}\eta
_{t+h}^{\prime }b\right] \right) ^{2}\right] \\
&&+\frac{2}{T_{h}^{2}}\dsum\limits_{t=p}^{T-h-1}\dsum\limits_{m=1}^{T-h-t}E%
\left\{ \left( a^{\prime }\eta _{t+h}\eta _{t+h}^{\prime }b-E\left[
a^{\prime }\eta _{t+h}\eta _{t+h}^{\prime }b\right] \right) \right. \\
&&\text{ \ \ \ \ \ \ \ \ \ \ \ \ \ \ \ \ \ \ \ \ \ \ \ \ \ }\left. \times
\left( a^{\prime }\eta _{t+m+h}\eta _{t+m+h}^{\prime }b-E\left[ a^{\prime
}\eta _{t+m+h}\eta _{t+m+h}^{\prime }b\right] \right) \right\}
\end{eqnarray*}%
Making use of the Cauchy-Schwarz inequality, we then have%
\begin{eqnarray*}
&&\frac{1}{T_{h}^{2}}\dsum\limits_{t=p}^{T-h}E\left[ \left( a^{\prime }\eta
_{t+h}\eta _{t+h}^{\prime }b-E\left[ a^{\prime }\eta _{t+h}\eta
_{t+h}^{\prime }b\right] \right) ^{2}\right] \\
&=&\frac{1}{T_{h}^{2}}\dsum\limits_{t=p}^{T-h}E\left( a^{\prime }\eta
_{t+h}\eta _{t+h}^{\prime }b\right) ^{2}-\frac{1}{T_{h}^{2}}%
\dsum\limits_{t=p}^{T-h}\left( E\left[ a^{\prime }\eta _{t+h}\eta
_{t+h}^{\prime }b\right] \right) ^{2} \\
&\leq &\frac{1}{T_{h}^{2}}\dsum\limits_{t=p}^{T-h}E\left( a^{\prime }\eta
_{t+h}\eta _{t+h}^{\prime }b\right) ^{2} \\
&\leq &\frac{1}{T_{h}^{2}}\dsum\limits_{t=p}^{T-h}\sqrt{E\left( a^{\prime
}\eta _{t+h}\right) ^{4}}\sqrt{E\left( b^{\prime }\eta _{t+h}\right) ^{4}}
\end{eqnarray*}%
In the proof of part (e) of this lemma, we have shown that, given
Assumptions 3-2(b) and 3-7, there exists positive constants $C$ and $%
\overline{C}$ such that 
\begin{equation*}
E\left( b^{\prime }\eta _{t+h}\right) ^{4}\leq
h^{3}\dsum\limits_{j=0}^{h-1}C\phi _{\max }^{4j}E\left\Vert \varepsilon _{t%
{\LARGE +}h-j}\right\Vert _{2}^{4}\leq \overline{C}<\infty \text{. }
\end{equation*}%
where $\phi _{\max }=\max \left\{ \left\vert \lambda _{\max }\left( A\right)
\right\vert ,\left\vert \lambda _{\min }\left( A\right) \right\vert \right\} 
$ and where $h$ is a fixed integer and $0<\phi _{\max }<1$ in light of
Assumption 3-1. In a similar manner, we can also show that%
\begin{equation*}
E\left( a^{\prime }\eta _{t+h}\right) ^{4}\leq
h^{3}\dsum\limits_{j=0}^{h-1}C\phi _{\max }^{4j}E\left\Vert \varepsilon _{t%
{\LARGE +}h-j}\right\Vert _{2}^{4}\leq \overline{C}<\infty \text{. }
\end{equation*}%
It follows that%
\begin{eqnarray}
\frac{1}{T_{h}^{2}}\dsum\limits_{t=p}^{T-h}E\left[ \left( a^{\prime }\eta
_{t+h}\eta _{t+h}^{\prime }b-E\left[ a^{\prime }\eta _{t+h}\eta
_{t+h}^{\prime }b\right] \right) ^{2}\right] &\leq &\frac{1}{T_{h}^{2}}%
\dsum\limits_{t=p}^{T-h}\sqrt{E\left( a^{\prime }\eta _{t+h}\right) ^{4}}%
\sqrt{E\left( b^{\prime }\eta _{t+h}\right) ^{4}}  \notag \\
&\leq &\frac{1}{T_{h}^{2}}\dsum\limits_{t=p}^{T-h}\overline{C}  \notag \\
&=&\overline{C}\frac{T-h-p+1}{T_{h}^{2}}  \notag \\
&=&\frac{\overline{C}}{T_{h}}\text{ \ }\left( \text{since }%
T_{h}=T-h-p+1\right)  \notag \\
&=&O\left( \frac{1}{T}\right)  \label{lead term etaetatranspose}
\end{eqnarray}%
Next, observe that%
\begin{eqnarray*}
&&a^{\prime }\eta _{t+h}\eta _{t+h}^{\prime }b-E\left[ a^{\prime }\eta
_{t+h}\eta _{t+h}^{\prime }b\right] \\
&=&\dsum\limits_{j=0}^{h-1}\dsum\limits_{k=0}^{h-1}a^{\prime }J_{d}A^{j}J_{d%
{\LARGE +}K}^{\prime }\left( \varepsilon _{t{\LARGE +}h-j}\varepsilon _{t%
{\LARGE +}h-k}^{\prime }-E\left[ \varepsilon _{t{\LARGE +}h-j}\varepsilon _{t%
{\LARGE +}h-k}^{\prime }\right] \right) J_{d{\LARGE +}K}\left( A^{k}\right)
^{\prime }J_{d}^{\prime }b \\
&=&\dsum\limits_{j=0}^{h-1}\dsum\limits_{k=0}^{h-1}\left( b^{\prime
}J_{d}A^{k}J_{d{\LARGE +}K}^{\prime }\otimes a^{\prime }J_{d}A^{j}J_{d%
{\LARGE +}K}^{\prime }\right) \left\{ vec\left( \varepsilon _{t{\LARGE +}%
h-j}\varepsilon _{t{\LARGE +}h-k}^{\prime }\right) -vec\left( E\left[
\varepsilon _{t{\LARGE +}h-j}\varepsilon _{t{\LARGE +}h-k}^{\prime }\right]
\right) \right\} \\
&=&\dsum\limits_{j=0}^{h-1}\dsum\limits_{k=0}^{h-1}\left( b^{\prime
}J_{d}A^{k}J_{d{\LARGE +}K}^{\prime }\otimes a^{\prime }J_{d}A^{j}J_{d%
{\LARGE +}K}^{\prime }\right) \left\{ \left( \varepsilon _{t{\LARGE +}%
h-k}\otimes \varepsilon _{t{\LARGE +}h-j}\right) -E\left[ \varepsilon _{t%
{\LARGE +}h-k}\otimes \varepsilon _{t{\LARGE +}h-j}\right] \right\}
\end{eqnarray*}%
and%
\begin{eqnarray*}
&&a^{\prime }\eta _{t{\LARGE +}m{\LARGE +}h}\eta _{t{\LARGE +}m{\LARGE +}%
h}^{\prime }b-E\left[ a^{\prime }\eta _{t{\LARGE +}m{\LARGE +}h}\eta _{t%
{\LARGE +}m{\LARGE +}h}^{\prime }b\right] \\
&=&\dsum\limits_{\ell =0}^{h-1}\dsum\limits_{r=0}^{h-1}a^{\prime
}J_{d}A^{\ell }J_{d{\LARGE +}K}^{\prime }\left( \varepsilon _{t{\LARGE +}m%
{\LARGE +}h-\ell }\varepsilon _{t{\LARGE +}m{\LARGE +}h-r}^{\prime }-E\left[
\varepsilon _{t{\LARGE +}m{\LARGE +}h-\ell }\varepsilon _{t{\LARGE +}m%
{\LARGE +}h-r}^{\prime }\right] \right) J_{d{\LARGE +}K}\left( A^{r}\right)
^{\prime }J_{d}^{\prime }b \\
&=&\dsum\limits_{\ell =0}^{h-1}\dsum\limits_{r=0}^{h-1}\left( b^{\prime
}J_{d}A^{r}J_{d{\LARGE +}K}^{\prime }\otimes a^{\prime }J_{d}A^{\ell }J_{d%
{\LARGE +}K}^{\prime }\right) \left\{ vec\left( \varepsilon _{t{\LARGE +}m%
{\LARGE +}h-\ell }\varepsilon _{t{\LARGE +}m{\LARGE +}h-r}^{\prime }\right)
-vec\left( E\left[ \varepsilon _{t{\LARGE +}m{\LARGE +}h-\ell }\varepsilon
_{t{\LARGE +}m{\LARGE +}h-r}^{\prime }\right] \right) \right\} \\
&=&\dsum\limits_{\ell =0}^{h-1}\dsum\limits_{r=0}^{h-1}\left( b^{\prime
}J_{d}A^{r}J_{d{\LARGE +}K}^{\prime }\otimes a^{\prime }J_{d}A^{\ell }J_{d%
{\LARGE +}K}^{\prime }\right) \left\{ \left( \varepsilon _{t{\LARGE +}m%
{\LARGE +}h-\ell }\otimes \varepsilon _{t{\LARGE +}m{\LARGE +}h-r}\right) -E%
\left[ \varepsilon _{t{\LARGE +}m{\LARGE +}h-k}\otimes \varepsilon _{t%
{\LARGE +}m{\LARGE +}h-r}\right] \right\}
\end{eqnarray*}%
Moreover, note that, for $m\geq h$%
\begin{eqnarray*}
&&E\left\{ \left( a^{\prime }\eta _{t+h}\eta _{t+h}^{\prime }b-E\left[
a^{\prime }\eta _{t+h}\eta _{t+h}^{\prime }b\right] \right) \left( a^{\prime
}\eta _{t+m+h}\eta _{t+m+h}^{\prime }b-E\left[ a^{\prime }\eta _{t+m+h}\eta
_{t+m+h}^{\prime }b\right] \right) \right\} \\
&=&\dsum\limits_{j=0}^{h-1}\dsum\limits_{k=0}^{h-1}\dsum\limits_{\ell
=0}^{h-1}\dsum\limits_{r=0}^{h-1}\left\{ \left( b^{\prime }J_{d}A^{k}J_{d%
{\LARGE +}K}^{\prime }\otimes a^{\prime }J_{d}A^{j}J_{d{\LARGE +}K}^{\prime
}\right) \right. \\
&&\times E\left( \left[ \left( \varepsilon _{t{\LARGE +}h-k}\otimes
\varepsilon _{t{\LARGE +}h-j}\right) -E\left( \varepsilon _{t{\LARGE +}%
h-k}\otimes \varepsilon _{t{\LARGE +}h-j}\right) \right] \right. \\
&&\text{ \ \ \ \ \ }\left. \times \left[ \left( \varepsilon _{t{\LARGE +}m%
{\LARGE +}h-\ell }\otimes \varepsilon _{t{\LARGE +}m{\LARGE +}h-r}\right)
-E\left( \varepsilon _{t{\LARGE +}m{\LARGE +}h-\ell }\otimes \varepsilon _{t%
{\LARGE +}m{\LARGE +}h-r}\right) \right] ^{\prime }\right) \\
&&\left. \times \left( J_{d{\LARGE +}K}\left( A^{r}\right) ^{\prime
}J_{d}^{\prime }b\otimes J_{d{\LARGE +}K}\left( A^{\ell }\right) ^{\prime
}J_{d}^{\prime }a\right) \right\} \\
&=&0
\end{eqnarray*}%
Note further that, when $h=1$, we will always have $m\geq h$, given that by
definition $m$ is an integer $\geq 1$. This implies we need to distinguish
between the case where $h=1$ from the case where $h\geq 2$.

Consider first the case where $h=1$. In this case, we have, for all $m\geq 1$%
\begin{eqnarray*}
&&E\left\{ \left( a^{\prime }\eta _{t+1}\eta _{t+1}^{\prime }b-E\left[
a^{\prime }\eta _{t+1}\eta _{t+1}^{\prime }b\right] \right) \left( a^{\prime
}\eta _{t+m+1}\eta _{t+m+1}^{\prime }b-E\left[ a^{\prime }\eta _{t+m+1}\eta
_{t+m+1}^{\prime }b\right] \right) \right\} \\
&=&\left( b^{\prime }J_{d}A^{0}J_{d{\LARGE +}K}^{\prime }\otimes a^{\prime
}J_{d}A^{0}J_{d{\LARGE +}K}^{\prime }\right) \\
&&\times E\left( \left[ \left( \varepsilon _{t{\LARGE +}1}\otimes
\varepsilon _{t{\LARGE +}1}\right) -E\left( \varepsilon _{t{\LARGE +}%
1}\otimes \varepsilon _{t{\LARGE +}1}\right) \right] \left[ \left(
\varepsilon _{t{\LARGE +}m{\LARGE +}1}\otimes \varepsilon _{t{\LARGE +}m%
{\LARGE +}1}\right) -E\left( \varepsilon _{t{\LARGE +}m{\LARGE +}1}\otimes
\varepsilon _{t{\LARGE +}m{\LARGE +}1}\right) \right] ^{\prime }\right) \\
&&\times \left( J_{d{\LARGE +}K}\left( A^{0}\right) ^{\prime }J_{d}^{\prime
}b\otimes J_{d{\LARGE +}K}\left( A^{0}\right) ^{\prime }J_{d}^{\prime
}a\right) \\
&=&0
\end{eqnarray*}%
so that, in this case, we have%
\begin{eqnarray}
&&E\left[ \frac{1}{T_{1}}\dsum\limits_{t=p}^{T-1}\left( a^{\prime }\eta
_{t+1}\eta _{t+1}^{\prime }b-E\left[ a^{\prime }\eta _{t+1}\eta
_{t+1}^{\prime }b\right] \right) \right] ^{2}  \notag \\
&=&\frac{1}{T_{1}^{2}}\dsum\limits_{t=p}^{T-1}E\left[ \left( a^{\prime }\eta
_{t+1}\eta _{t+1}^{\prime }b-E\left[ a^{\prime }\eta _{t+1}\eta
_{t+1}^{\prime }b\right] \right) ^{2}\right]  \notag \\
&&+\frac{2}{T_{1}^{2}}\dsum\limits_{t=p}^{T-1}\dsum\limits_{m=1}^{T-1-t}E%
\left\{ \left( a^{\prime }\eta _{t+1}\eta _{t+1}^{\prime }b-E\left[
a^{\prime }\eta _{t+1}\eta _{t+1}^{\prime }b\right] \right) \right.  \notag
\\
&&\text{\ \ \ \ \ \ \ \ \ }\left. \times \left( a^{\prime }\eta _{t+m+1}\eta
_{t+m+1}^{\prime }b-E\left[ a^{\prime }\eta _{t+m+1}\eta _{t+m+1}^{\prime }b%
\right] \right) \right\}  \notag \\
&=&\frac{1}{T_{1}^{2}}\dsum\limits_{t=p}^{T-1}E\left[ \left( a^{\prime }\eta
_{t+1}\eta _{t+1}^{\prime }b-E\left[ a^{\prime }\eta _{t+1}\eta
_{t+1}^{\prime }b\right] \right) ^{2}\right]  \notag \\
&=&O\left( \frac{1}{T}\right) \text{ }\left( \text{as shown previously in
expression (\ref{lead term etaetatranspose})}\right)  \label{MSE h=1}
\end{eqnarray}

Consider next the case where $h\geq 2$. In this case, 
\begin{equation*}
E\left\{ \left( a^{\prime }\eta _{t+1}\eta _{t+1}^{\prime }b-E\left[
a^{\prime }\eta _{t+1}\eta _{t+1}^{\prime }b\right] \right) \left( a^{\prime
}\eta _{t+m+1}\eta _{t+m+1}^{\prime }b-E\left[ a^{\prime }\eta _{t+m+1}\eta
_{t+m+1}^{\prime }b\right] \right) \right\} =0
\end{equation*}%
for all $m\geq h$ as previously shown; however, for $1\leq m\leq h-1$, we
have%
\begin{eqnarray*}
&&\left\vert E\left\{ \left( a^{\prime }\eta _{t+h}\eta _{t+h}^{\prime }b-E 
\left[ a^{\prime }\eta _{t+h}\eta _{t+h}^{\prime }b\right] \right) \left(
a^{\prime }\eta _{t+m+h}\eta _{t+m+h}^{\prime }b-E\left[ a^{\prime }\eta
_{t+m+h}\eta _{t+m+h}^{\prime }b\right] \right) \right\} \right\vert \\
&=&\left\vert
\dsum\limits_{j=0}^{h-1}\dsum\limits_{k=0}^{h-1}\dsum\limits_{\ell
=0}^{h-1}\dsum\limits_{r=0}^{h-1}\left\{ \left( b^{\prime }J_{d}A^{k}J_{d%
{\LARGE +}K}^{\prime }\otimes a^{\prime }J_{d}A^{j}J_{d{\LARGE +}K}^{\prime
}\right) \right. \right. \\
&&\times E\left( \left[ \left( \varepsilon _{t{\LARGE +}h-k}\otimes
\varepsilon _{t{\LARGE +}h-j}\right) -E\left( \varepsilon _{t{\LARGE +}%
h-k}\otimes \varepsilon _{t{\LARGE +}h-j}\right) \right] \right. \\
&&\left. \times \left[ \left( \varepsilon _{t{\LARGE +}m{\LARGE +}h-\ell
}\otimes \varepsilon _{t{\LARGE +}m{\LARGE +}h-r}\right) -E\left(
\varepsilon _{t{\LARGE +}m{\LARGE +}h-\ell }\otimes \varepsilon _{t{\LARGE +}%
m{\LARGE +}h-r}\right) \right] ^{\prime }\right) \\
&&\left. \left. \times \left( J_{d{\LARGE +}K}\left( A^{r}\right) ^{\prime
}J_{d}^{\prime }b\otimes J_{d{\LARGE +}K}\left( A^{\ell }\right) ^{\prime
}J_{d}^{\prime }a\right) \right\} \right\vert \\
&=&\left\vert
\dsum\limits_{j=0}^{h-1}\dsum\limits_{k=0}^{h-1}\dsum\limits_{\ell
=0}^{h-1}\dsum\limits_{r=0}^{h-1}\left\{ \left( b^{\prime }J_{d}A^{k}J_{d%
{\LARGE +}K}^{\prime }\otimes a^{\prime }J_{d}A^{j}J_{d{\LARGE +}K}^{\prime
}\right) E\left( \varepsilon _{t{\LARGE +}h-k}\varepsilon _{t{\LARGE +}m%
{\LARGE +}h-\ell }^{\prime }\otimes \varepsilon _{t{\LARGE +}h-j}\varepsilon
_{t{\LARGE +}m{\LARGE +}h-r}^{\prime }\right) \right. \right. \\
&&\text{ \ \ \ \ \ \ \ \ \ \ \ \ \ \ \ \ \ \ \ }\left. \times \left( J_{d%
{\LARGE +}K}\left( A^{r}\right) ^{\prime }J_{d}^{\prime }b\otimes J_{d%
{\LARGE +}K}\left( A^{\ell }\right) ^{\prime }J_{d}^{\prime }a\right)
\right\} \\
&&-\dsum\limits_{j=0}^{h-1}\dsum\limits_{k=0}^{h-1}\dsum\limits_{\ell
=0}^{h-1}\dsum\limits_{r=0}^{h-1}\left\{ \left( b^{\prime }J_{d}A^{k}J_{d%
{\LARGE +}K}^{\prime }\otimes a^{\prime }J_{d}A^{j}J_{d{\LARGE +}K}^{\prime
}\right) E\left( \varepsilon _{t{\LARGE +}h-k}\otimes \varepsilon _{t{\LARGE %
+}h-j}\right) E\left( \varepsilon _{t{\LARGE +}m{\LARGE +}h-\ell }\otimes
\varepsilon _{t{\LARGE +}m{\LARGE +}h-r}\right) ^{\prime }\right. \\
&&\text{ \ \ \ \ \ \ \ \ \ \ \ \ \ \ \ \ \ \ \ \ \ \ \ }\left. \text{\ }%
\left. \times \left( J_{d{\LARGE +}K}\left( A^{r}\right) ^{\prime
}J_{d}^{\prime }b\otimes J_{d{\LARGE +}K}\left( A^{\ell }\right) ^{\prime
}J_{d}^{\prime }a\right) \right\} \right\vert \\
&\leq &\dsum\limits_{j=0}^{h-1}\left\vert \left( b^{\prime }J_{d}A^{j}J_{d%
{\LARGE +}K}^{\prime }\otimes a^{\prime }J_{d}A^{j}J_{d{\LARGE +}K}^{\prime
}\right) E\left( \varepsilon _{t{\LARGE +}h-j}\varepsilon _{t{\LARGE +}%
h-j}^{\prime }\otimes \varepsilon _{t{\LARGE +}h-j}\varepsilon _{t{\LARGE +}%
h-j}^{\prime }\right) \right. \\
&&\text{ \ \ }\left. \times \left( J_{d{\LARGE +}K}\left( A^{j}\right)
^{\prime }J_{d}^{\prime }b\otimes J_{d{\LARGE +}K}\left( A^{j}\right)
^{\prime }J_{d}^{\prime }a\right) \right\vert \\
&&+\dsum\limits_{j=0}^{h-1}\dsum\limits_{\substack{ k=0  \\ k\neq j}}%
^{h-1}\left\vert \left( b^{\prime }J_{d}A^{k}J_{d{\LARGE +}K}^{\prime
}\otimes a^{\prime }J_{d}A^{j}J_{d{\LARGE +}K}^{\prime }\right) E\left(
\varepsilon _{t{\LARGE +}h-k}\varepsilon _{t{\LARGE +}h-k}^{\prime }\otimes
\varepsilon _{t{\LARGE +}h-j}\varepsilon _{t{\LARGE +}h-j}^{\prime }\right)
\right. \\
&&\text{ \ \ \ \ \ \ \ \ \ \ \ \ }\left. \text{\ }\times \left( J_{d{\LARGE +%
}K}\left( A^{k}\right) ^{\prime }J_{d}^{\prime }b\otimes J_{d{\LARGE +}%
K}\left( A^{j}\right) ^{\prime }J_{d}^{\prime }a\right) \right\vert \\
&&+\dsum\limits_{k=0}^{h-1}\dsum\limits_{\substack{ \ell =0  \\ \ell \neq k%
{\LARGE +}m}}^{h-1}\left\vert \left( b^{\prime }J_{d}A^{k}J_{d{\LARGE +}%
K}^{\prime }\otimes a^{\prime }J_{d}A^{k}J_{d{\LARGE +}K}^{\prime }\right)
E\left( \varepsilon _{t{\LARGE +}h-k}\varepsilon _{t{\LARGE +}m{\LARGE +}%
h-\ell }^{\prime }\otimes \varepsilon _{t{\LARGE +}h-k}\varepsilon _{t%
{\LARGE +}m{\LARGE +}h-\ell }^{\prime }\right) \right. \\
&&\text{ \ \ \ \ \ \ \ \ \ \ \ \ \ \ \ \ }\left. \times \left( J_{d{\LARGE +}%
K}\left( A^{\ell }\right) ^{\prime }J_{d}^{\prime }b\otimes J_{d{\LARGE +}%
K}\left( A^{\ell }\right) ^{\prime }J_{d}^{\prime }a\right) \right\vert \\
&&+\dsum\limits_{k=0}^{h-1}\dsum\limits_{\substack{ \ell =0  \\ \ell \neq k%
{\LARGE +}m}}^{h-1}\left\vert \left( b^{\prime }J_{d}A^{k}J_{d{\LARGE +}%
K}^{\prime }\otimes a^{\prime }J_{d}A^{\ell }J_{d{\LARGE +}K}^{\prime
}\right) E\left( \varepsilon _{t{\LARGE +}h-k}\varepsilon _{t{\LARGE +}m%
{\LARGE +}h-\ell }^{\prime }\otimes \varepsilon _{t{\LARGE +}m{\LARGE +}%
h-\ell }\varepsilon _{t{\LARGE +}h-k}^{\prime }\right) \right. \\
&&\text{ \ \ \ \ \ \ \ \ \ \ \ \ \ \ \ \ }\left. \times \left( J_{d{\LARGE +}%
K}\left( A^{\ell }\right) ^{\prime }J_{d}^{\prime }b\otimes J_{d{\LARGE +}%
K}\left( A^{k}\right) ^{\prime }J_{d}^{\prime }a\right) \right\vert
\end{eqnarray*}%
\begin{eqnarray*}
&&+\dsum\limits_{j=0}^{h-1}\dsum\limits_{\ell =0}^{h-1}\left\vert \left(
b^{\prime }J_{d}A^{j}J_{d{\LARGE +}K}^{\prime }\otimes a^{\prime
}J_{d}A^{j}J_{d{\LARGE +}K}^{\prime }\right) E\left( \varepsilon _{t{\LARGE +%
}h-j}\otimes \varepsilon _{t{\LARGE +}h-j}\right) E\left( \varepsilon _{t%
{\LARGE +}m{\LARGE +}h-\ell }\otimes \varepsilon _{t{\LARGE +}m{\LARGE +}%
h-\ell }\right) ^{\prime }\right. \\
&&\text{ \ \ \ \ \ \ \ \ \ \ \ \ \ \ \ \ \ \ }\left. \times \left( J_{d%
{\LARGE +}K}\left( A^{\ell }\right) ^{\prime }J_{d}^{\prime }b\otimes J_{d%
{\LARGE +}K}\left( A^{\ell }\right) ^{\prime }J_{d}^{\prime }a\right)
\right\vert \text{.}
\end{eqnarray*}%
Analyzing each term on the majorant side of the function above, we have%
\begin{eqnarray*}
&&\dsum\limits_{j=0}^{h-1}\left\vert \left( b^{\prime }J_{d}A^{j}J_{d{\LARGE %
+}K}^{\prime }\otimes a^{\prime }J_{d}A^{j}J_{d{\LARGE +}K}^{\prime }\right)
E\left( \varepsilon _{t{\LARGE +}h-j}\varepsilon _{t{\LARGE +}h-j}^{\prime
}\otimes \varepsilon _{t{\LARGE +}h-j}\varepsilon _{t{\LARGE +}h-j}^{\prime
}\right) \right. \\
&&\text{ \ \ \ \ }\left. \times \left( J_{d{\LARGE +}K}\left( A^{j}\right)
^{\prime }J_{d}^{\prime }b\otimes J_{d{\LARGE +}K}\left( A^{j}\right)
^{\prime }J_{d}^{\prime }a\right) \right\vert \\
&\leq &\overline{C}\dsum\limits_{j=0}^{h-1}\left( b^{\prime }J_{d}A^{j}J_{d%
{\LARGE +}K}^{\prime }J_{d{\LARGE +}K}\left( A^{j}\right) ^{\prime
}J_{d}^{\prime }b\right) \left( a^{\prime }J_{d}A^{j}J_{d{\LARGE +}%
K}^{\prime }J_{d{\LARGE +}K}\left( A^{j}\right) ^{\prime }J_{d}^{\prime
}a\right) \\
&=&\overline{C}\dsum\limits_{j=0}^{h-1}\left[ b^{\prime }J_{d}A^{j}\left(
A^{j}\right) ^{\prime }J_{d}^{\prime }b\right] \left[ a^{\prime
}J_{d}A^{j}\left( A^{j}\right) ^{\prime }J_{d}^{\prime }a\right] \text{ \ }%
\left( \text{since }\lambda _{\max }\left( J_{d{\LARGE +}K}^{\prime }J_{d%
{\LARGE +}K}\right) =1\right) \\
&\leq &\overline{C}\dsum\limits_{j=0}^{h-1}\left[ \lambda _{\max }\left\{
A^{j}\left( A^{j}\right) ^{\prime }\right\} \right] ^{2}\left[ b^{\prime
}J_{d}J_{d}^{\prime }b\right] \left[ a^{\prime }J_{d}J_{d}^{\prime }a\right]
\\
&=&\overline{C}\dsum\limits_{j=0}^{h-1}\left[ \lambda _{\max }\left\{
A^{j}\left( A^{j}\right) ^{\prime }\right\} \right] ^{2}\text{ \ }\left( 
\text{since }J_{d}J_{d}^{\prime }=I_{d}\text{ and }a^{\prime }a=b^{\prime
}b=1\right) \\
&=&\overline{C}\dsum\limits_{j=0}^{h-1}\left[ \lambda _{\max }\left\{ \left(
A^{j}\right) ^{\prime }A^{j}\right\} \right] ^{2} \\
&=&\overline{C}\dsum\limits_{j=0}^{h-1}\sigma _{\max }^{4}\left( A^{j}\right)
\\
&\leq &\overline{C}\dsum\limits_{j=0}^{h-1}C^{\ast }\max \left\{ \left\vert
\lambda _{\max }\left( A^{j}\right) \right\vert ^{4},\left\vert \lambda
_{\min }\left( A^{j}\right) \right\vert ^{4}\right\} \text{ }\left( \text{by
Assumption 3-7}\right) \\
&=&\overline{C}\dsum\limits_{j=0}^{h-1}C^{\ast }\max \left\{ \left\vert
\lambda _{\max }\left( A\right) \right\vert ^{4j},\left\vert \lambda _{\min
}\left( A\right) \right\vert ^{4j}\right\} \\
&=&\overline{C}\dsum\limits_{j=0}^{h-1}C^{\ast }\phi _{\max }^{4j}\text{ }%
\left( \text{where }0<\phi _{\max }=\max \left\{ \left\vert \lambda _{\max
}\left( A\right) \right\vert ,\left\vert \lambda _{\min }\left( A\right)
\right\vert \right\} <1\right) \\
&\leq &\overline{C}hC^{\ast }\text{ \ }\left( \text{since }0<\phi _{\max }<1%
\text{ and }\phi _{\max }^{0}=1\right) \\
&\leq &C\text{ \ }\left( \text{for }\overline{C}hC^{\ast }\leq C<\infty
\right) \text{,}
\end{eqnarray*}%
\begin{eqnarray*}
&&\dsum\limits_{j=0}^{h-1}\dsum\limits_{\substack{ k=0  \\ k\neq j}}%
^{h-1}\left\vert \left( b^{\prime }J_{d}A^{k}J_{d{\LARGE +}K}^{\prime
}\otimes a^{\prime }J_{d}A^{j}J_{d{\LARGE +}K}^{\prime }\right) E\left(
\varepsilon _{t{\LARGE +}h-k}\varepsilon _{t{\LARGE +}h-k}^{\prime }\otimes
\varepsilon _{t{\LARGE +}h-j}\varepsilon _{t{\LARGE +}h-j}^{\prime }\right)
\right. \\
&&\text{ \ \ \ \ \ \ \ \ }\left. \times \left( J_{d{\LARGE +}K}\left(
A^{k}\right) ^{\prime }J_{d}^{\prime }b\otimes J_{d{\LARGE +}K}\left(
A^{j}\right) ^{\prime }J_{d}^{\prime }a\right) \right\vert \\
&\leq &\overline{C}\dsum\limits_{k=0}^{h-1}\dsum\limits_{\substack{ k=0  \\ %
k\neq j}}^{h-1}\left( b^{\prime }J_{d}A^{k}J_{d{\LARGE +}K}^{\prime }J_{d%
{\LARGE +}K}\left( A^{k}\right) ^{\prime }J_{d}^{\prime }b\right) \left(
a^{\prime }J_{d}A^{j}J_{d{\LARGE +}K}^{\prime }J_{d{\LARGE +}K}\left(
A^{j}\right) ^{\prime }J_{d}^{\prime }a\right) \\
&=&\overline{C}\dsum\limits_{k=0}^{h-1}\dsum\limits_{\substack{ k=0  \\ %
k\neq j}}^{h-1}\left[ b^{\prime }J_{d}A^{k}\left( A^{k}\right) ^{\prime
}J_{d}^{\prime }b\right] \left[ a^{\prime }J_{d}A^{j}\left( A^{j}\right)
^{\prime }J_{d}^{\prime }a\right] \text{ \ }\left( \text{since }\lambda
_{\max }\left( J_{d{\LARGE +}K}^{\prime }J_{d{\LARGE +}K}\right) =1\right) \\
&\leq &\overline{C}\dsum\limits_{k=0}^{h-1}\dsum\limits_{\substack{ k=0  \\ %
k\neq j}}^{h-1}\left[ \lambda _{\max }\left\{ A^{j}\left( A^{j}\right)
^{\prime }\right\} \right] \left[ \lambda _{\max }\left\{ A^{k}\left(
A^{k}\right) ^{\prime }\right\} \right] \left[ b^{\prime }J_{d}J_{d}^{\prime
}b\right] \left[ a^{\prime }J_{d}J_{d}^{\prime }a\right] \\
&=&\overline{C}\dsum\limits_{k=0}^{h-1}\dsum\limits_{\substack{ k=0  \\ %
k\neq j}}^{h-1}\left[ \lambda _{\max }\left\{ A^{j}\left( A^{j}\right)
^{\prime }\right\} \right] \left[ \lambda _{\max }\left\{ A^{k}\left(
A^{k}\right) ^{\prime }\right\} \right] \text{ \ }\left( \text{since }%
J_{d}J_{d}^{\prime }=I_{d}\text{ and }a^{\prime }a=b^{\prime }b=1\right) \\
&=&\overline{C}\dsum\limits_{k=0}^{h-1}\dsum\limits_{\substack{ k=0  \\ %
k\neq j}}^{h-1}\left[ \lambda _{\max }\left\{ \left( A^{j}\right) ^{\prime
}A^{j}\right\} \right] \left[ \lambda _{\max }\left\{ \left( A^{k}\right)
^{\prime }A^{k}\right\} \right] \\
&=&\overline{C}\dsum\limits_{k=0}^{h-1}\dsum\limits_{\substack{ k=0  \\ %
k\neq j}}^{h-1}\sigma _{\max }^{2}\left( A^{j}\right) \sigma _{\max
}^{2}\left( A^{k}\right) \\
&\leq &\overline{C}\dsum\limits_{k=0}^{h-1}\dsum\limits_{\substack{ k=0  \\ %
k\neq j}}^{h-1}\left( C^{\ast }\right) ^{2}\max \left\{ \left\vert \lambda
_{\max }\left( A^{j}\right) \right\vert ^{2},\left\vert \lambda _{\min
}\left( A^{j}\right) \right\vert ^{2}\right\} \max \left\{ \left\vert
\lambda _{\max }\left( A^{k}\right) \right\vert ^{2},\left\vert \lambda
_{\min }\left( A^{k}\right) \right\vert ^{2}\right\} \text{ } \\
&&\left( \text{by Assumption 3-7}\right) \\
&=&\overline{C}\dsum\limits_{k=0}^{h-1}\dsum\limits_{\substack{ k=0  \\ %
k\neq j}}^{h-1}\left( C^{\ast }\right) ^{2}\max \left\{ \left\vert \lambda
_{\max }\left( A\right) \right\vert ^{2j},\left\vert \lambda _{\min }\left(
A\right) \right\vert ^{2j}\right\} \max \left\{ \left\vert \lambda _{\max
}\left( A\right) \right\vert ^{2k},\left\vert \lambda _{\min }\left(
A\right) \right\vert ^{2k}\right\} \\
&\leq &\overline{C}\dsum\limits_{k=0}^{h-1}\dsum\limits_{\substack{ k=0  \\ %
k\neq j}}^{h-1}\left( C^{\ast }\right) ^{2}\phi _{\max }^{2j}\phi _{\max
}^{2k}\text{ }\left( \text{where }\phi _{\max }=\max \left\{ \left\vert
\lambda _{\max }\left( A\right) \right\vert ,\left\vert \lambda _{\min
}\left( A\right) \right\vert \right\} \right) \\
&=&\overline{C}h^{2}\left( C^{\ast }\right) ^{2}\text{ }\left( \text{since }%
0<\phi _{\max }<1\text{ given Assumption 3-1}\right) \\
&\leq &C\text{ \ }\left( \text{for }\overline{C}h^{2}\left( C^{\ast }\right)
^{2}\leq C<\infty \right) \text{,}
\end{eqnarray*}%
\begin{eqnarray*}
&&\dsum\limits_{k=0}^{h-1}\dsum\limits_{\substack{ \ell =0  \\ \ell \neq k%
{\LARGE +}m}}^{h-1}\left\vert \left( b^{\prime }J_{d}A^{k}J_{d{\LARGE +}%
K}^{\prime }\otimes a^{\prime }J_{d}A^{k}J_{d{\LARGE +}K}^{\prime }\right)
E\left( \varepsilon _{t{\LARGE +}h-k}\varepsilon _{t{\LARGE +}m{\LARGE +}%
h-\ell }^{\prime }\otimes \varepsilon _{t{\LARGE +}h-k}\varepsilon _{t%
{\LARGE +}m{\LARGE +}h-\ell }^{\prime }\right) \right. \\
&&\text{ \ \ \ \ \ \ \ \ \ \ \ }\left. \times \left( J_{d{\LARGE +}K}\left(
A^{\ell }\right) ^{\prime }J_{d}^{\prime }b\otimes J_{d{\LARGE +}K}\left(
A^{\ell }\right) ^{\prime }J_{d}^{\prime }a\right) \right\vert \\
&\leq &\dsum\limits_{k=0}^{h-1}\dsum\limits_{\substack{ \ell =0  \\ \ell
\neq k{\LARGE +}m}}^{h-1}\left\{ \left[ \left( b^{\prime }J_{d}A^{k}J_{d%
{\LARGE +}K}^{\prime }\otimes a^{\prime }J_{d}A^{k}J_{d{\LARGE +}K}^{\prime
}\right) E\left( \varepsilon _{t{\LARGE +}h-k}\varepsilon _{t{\LARGE +}%
h-k}^{\prime }\otimes \varepsilon _{t{\LARGE +}h-k}\varepsilon _{t{\LARGE +}%
h-k}^{\prime }\right) \right. \right. \\
&&\text{ \ \ \ \ \ \ \ \ \ \ \ \ }\left. \times \left( J_{d{\LARGE +}%
K}\left( A^{k}\right) ^{\prime }J_{d}^{\prime }b\otimes J_{d{\LARGE +}%
K}\left( A^{k}\right) ^{\prime }J_{d}^{\prime }a\right) \right] ^{1/2} \\
&&\text{ }\times \left[ \left( b^{\prime }J_{d}A^{\ell }J_{d{\LARGE +}%
K}^{\prime }\otimes a^{\prime }J_{d}A^{\ell }J_{d{\LARGE +}K}^{\prime
}\right) E\left( \varepsilon _{t{\LARGE +}m{\LARGE +}h-\ell }\varepsilon _{t%
{\LARGE +}m{\LARGE +}h-\ell }^{\prime }\otimes \varepsilon _{t{\LARGE +}m%
{\LARGE +}h-\ell }\varepsilon _{t{\LARGE +}m{\LARGE +}h-\ell }^{\prime
}\right) \right. \\
&&\text{ \ \ \ }\left. \text{\ \ }\left. \text{\ }\times \left( J_{d{\LARGE +%
}K}\left( A^{\ell }\right) ^{\prime }J_{d}^{\prime }b\otimes J_{d{\LARGE +}%
K}\left( A^{\ell }\right) ^{\prime }J_{d}^{\prime }a\right) \right]
^{1/2}\right\} \\
&\leq &\overline{C}\dsum\limits_{k=0}^{h-1}\dsum\limits_{\substack{ \ell =0 
\\ \ell \neq k{\LARGE +}m}}^{h-1}\sqrt{\left( b^{\prime }J_{d}A^{k}J_{d%
{\LARGE +}K}^{\prime }J_{d{\LARGE +}K}\left( A^{k}\right) ^{\prime
}J_{d}^{\prime }b\right) \left( a^{\prime }J_{d}A^{k}J_{d{\LARGE +}%
K}^{\prime }J_{d{\LARGE +}K}\left( A^{k}\right) ^{\prime }J_{d}^{\prime
}a\right) } \\
&&\text{ \ \ \ \ \ \ \ \ \ \ }\times \sqrt{\left( b^{\prime }J_{d}A^{\ell
}J_{d{\LARGE +}K}^{\prime }J_{d{\LARGE +}K}\left( A^{\ell }\right) ^{\prime
}J_{d}^{\prime }b\right) \left( a^{\prime }J_{d}A^{\ell }J_{d{\LARGE +}%
K}^{\prime }J_{d{\LARGE +}K}\left( A^{\ell }\right) ^{\prime }J_{d}^{\prime
}a\right) } \\
&=&\overline{C}\dsum\limits_{k=0}^{h-1}\dsum\limits_{\substack{ \ell =0  \\ %
\ell \neq k{\LARGE +}m}}^{h-1}\sqrt{\left[ b^{\prime }J_{d}A^{k}\left(
A^{k}\right) ^{\prime }J_{d}^{\prime }b\right] \left[ a^{\prime
}J_{d}A^{k}\left( A^{k}\right) ^{\prime }J_{d}^{\prime }a\right] }\sqrt{%
\left[ b^{\prime }J_{d}A^{\ell }\left( A^{\ell }\right) ^{\prime
}J_{d}^{\prime }b\right] \left[ a^{\prime }J_{d}A^{\ell }\left( A^{\ell
}\right) ^{\prime }J_{d}^{\prime }a\right] }\text{ \ } \\
&&\left( \text{since }\lambda _{\max }\left( J_{d{\LARGE +}K}^{\prime }J_{d%
{\LARGE +}K}\right) =1\right) \\
&\leq &\overline{C}\dsum\limits_{k=0}^{h-1}\dsum\limits_{\substack{ \ell =0 
\\ \ell \neq k{\LARGE +}m}}^{h-1}\left[ \lambda _{\max }\left\{ A^{k}\left(
A^{k}\right) ^{\prime }\right\} \right] \left[ \lambda _{\max }\left\{
A^{\ell }\left( A^{\ell }\right) ^{\prime }\right\} \right] \left[ b^{\prime
}J_{d}J_{d}^{\prime }b\right] \left[ a^{\prime }J_{d}J_{d}^{\prime }a\right]
\\
&=&\overline{C}\dsum\limits_{k=0}^{h-1}\dsum\limits_{\substack{ \ell =0  \\ %
\ell \neq k{\LARGE +}m}}^{h-1}\left[ \lambda _{\max }\left\{ A^{k}\left(
A^{k}\right) ^{\prime }\right\} \right] \left[ \lambda _{\max }\left\{
A^{\ell }\left( A^{\ell }\right) ^{\prime }\right\} \right] \text{ \ }\left( 
\text{since }J_{d}J_{d}^{\prime }=I_{d}\text{ and }a^{\prime }a=b^{\prime
}b=1\right) \\
&=&\overline{C}\dsum\limits_{k=0}^{h-1}\dsum\limits_{\substack{ \ell =0  \\ %
\ell \neq k{\LARGE +}m}}^{h-1}\left[ \lambda _{\max }\left\{ \left(
A^{k}\right) ^{\prime }A^{k}\right\} \right] \left[ \lambda _{\max }\left\{
\left( A^{\ell }\right) ^{\prime }A^{\ell }\right\} \right]
\end{eqnarray*}%
\begin{eqnarray*}
&=&\overline{C}\dsum\limits_{k=0}^{h-1}\dsum\limits_{\substack{ \ell =0  \\ %
\ell \neq k{\LARGE +}m}}^{h-1}\sigma _{\max }^{2}\left( A^{k}\right) \sigma
_{\max }^{2}\left( A^{\ell }\right) \\
&\leq &\overline{C}\dsum\limits_{k=0}^{h-1}\dsum\limits_{\substack{ \ell =0 
\\ \ell \neq k{\LARGE +}m}}^{h-1}\left( C^{\ast }\right) ^{2}\max \left\{
\left\vert \lambda _{\max }\left( A^{k}\right) \right\vert ^{2},\left\vert
\lambda _{\min }\left( A^{k}\right) \right\vert ^{2}\right\} \max \left\{
\left\vert \lambda _{\max }\left( A^{\ell }\right) \right\vert
^{2},\left\vert \lambda _{\min }\left( A^{\ell }\right) \right\vert
^{2}\right\} \text{ } \\
&&\left( \text{by Assumption 3-7}\right) \\
&=&\overline{C}\dsum\limits_{k=0}^{h-1}\dsum\limits_{\substack{ \ell =0  \\ %
\ell \neq k{\LARGE +}m}}^{h-1}\left( C^{\ast }\right) ^{2}\max \left\{
\left\vert \lambda _{\max }\left( A\right) \right\vert ^{2k},\left\vert
\lambda _{\min }\left( A\right) \right\vert ^{2k}\right\} \max \left\{
\left\vert \lambda _{\max }\left( A\right) \right\vert ^{2\ell },\left\vert
\lambda _{\min }\left( A\right) \right\vert ^{2\ell }\right\} \\
&\leq &\overline{C}\dsum\limits_{k=0}^{h-1}\dsum\limits_{\substack{ \ell =0 
\\ \ell \neq k{\LARGE +}m}}^{h-1}\left( C^{\ast }\right) ^{2}\phi _{\max
}^{2k}\phi _{\max }^{2\ell }\text{ }\left( \text{since }\phi _{\max }=\max
\left\{ \left\vert \lambda _{\max }\left( A\right) \right\vert ,\left\vert
\lambda _{\min }\left( A\right) \right\vert \right\} \right) \\
&=&\overline{C}h^{2}\left( C^{\ast }\right) ^{2}\text{ }\left( \text{since }%
0<\phi _{\max }<1\text{ and }\phi _{\max }^{0}=1\right) \\
&\leq &C\text{ \ }\left( \text{for }\overline{C}h^{2}\left( C^{\ast }\right)
^{2}\leq C<\infty \right) \text{,}
\end{eqnarray*}%
\begin{eqnarray*}
&&\dsum\limits_{k=0}^{h-1}\dsum\limits_{\substack{ \ell =0  \\ \ell \neq k%
{\LARGE +}m}}^{h-1}\left\vert \left( b^{\prime }J_{d}A^{k}J_{d{\LARGE +}%
K}^{\prime }\otimes a^{\prime }J_{d}A^{\ell }J_{d{\LARGE +}K}^{\prime
}\right) E\left( \varepsilon _{t{\LARGE +}h-k}\varepsilon _{t{\LARGE +}m%
{\LARGE +}h-\ell }^{\prime }\otimes \varepsilon _{t{\LARGE +}m{\LARGE +}%
h-\ell }\varepsilon _{t{\LARGE +}h-k}^{\prime }\right) \right. \\
&&\text{ \ \ \ \ \ \ \ \ \ \ \ \ }\left. \times \left( J_{d{\LARGE +}%
K}\left( A^{\ell }\right) ^{\prime }J_{d}^{\prime }b\otimes J_{d{\LARGE +}%
K}\left( A^{k}\right) ^{\prime }J_{d}^{\prime }a\right) \right\vert \\
&\leq &\dsum\limits_{k=0}^{h-1}\dsum\limits_{\substack{ \ell =0  \\ \ell
\neq k{\LARGE +}m}}^{h-1}\left\{ \left[ \left( b^{\prime }J_{d}A^{k}J_{d%
{\LARGE +}K}^{\prime }\otimes a^{\prime }J_{d}A^{\ell }J_{d{\LARGE +}%
K}^{\prime }\right) E\left( \varepsilon _{t{\LARGE +}h-k}\varepsilon _{t%
{\LARGE +}h-k}^{\prime }\otimes \varepsilon _{t{\LARGE +}m{\LARGE +}h-\ell
}\varepsilon _{t{\LARGE +}m{\LARGE +}h-\ell }^{\prime }\right) \right.
\right. \\
&&\text{ \ \ \ \ \ \ \ \ \ \ \ \ \ }\left. \times \left( J_{d{\LARGE +}%
K}\left( A^{k}\right) ^{\prime }J_{d}^{\prime }b\otimes J_{d{\LARGE +}%
K}\left( A^{\ell }\right) ^{\prime }J_{d}^{\prime }a\right) \right] ^{1/2} \\
&&\text{ \ \ \ \ \ }\times \left[ \left( b^{\prime }J_{d}A^{\ell }J_{d%
{\LARGE +}K}^{\prime }\otimes a^{\prime }J_{d}A^{k}J_{d{\LARGE +}K}^{\prime
}\right) E\left( \varepsilon _{t{\LARGE +}m{\LARGE +}h-\ell }\varepsilon _{t%
{\LARGE +}m{\LARGE +}h-\ell }^{\prime }\otimes \varepsilon _{t{\LARGE +}%
h-k}\varepsilon _{t{\LARGE +}h-k}^{\prime }\right) \right. \\
&&\text{ \ \ \ \ \ \ \ \ \ \ }\left. \text{\ }\left. \times \left( J_{d%
{\LARGE +}K}\left( A^{\ell }\right) ^{\prime }J_{d}^{\prime }b\otimes J_{d%
{\LARGE +}K}\left( A^{k}\right) ^{\prime }J_{d}^{\prime }a\right) \right]
^{1/2}\right\} \\
&\leq &\overline{C}\dsum\limits_{k=0}^{h-1}\dsum\limits_{\substack{ \ell =0 
\\ \ell \neq k{\LARGE +}m}}^{h-1}\sqrt{\left( b^{\prime }J_{d}A^{k}J_{d%
{\LARGE +}K}^{\prime }J_{d{\LARGE +}K}\left( A^{k}\right) ^{\prime
}J_{d}^{\prime }b\right) \left( a^{\prime }J_{d}A^{\ell }J_{d{\LARGE +}%
K}^{\prime }J_{d{\LARGE +}K}\left( A^{\ell }\right) ^{\prime }J_{d}^{\prime
}a\right) } \\
&&\text{ \ \ \ \ \ \ \ \ \ \ }\times \sqrt{\left( b^{\prime }J_{d}A^{\ell
}J_{d{\LARGE +}K}^{\prime }J_{d{\LARGE +}K}\left( A^{\ell }\right) ^{\prime
}J_{d}^{\prime }b\right) \left( a^{\prime }J_{d}A^{k}J_{d{\LARGE +}%
K}^{\prime }J_{d{\LARGE +}K}\left( A^{k}\right) ^{\prime }J_{d}^{\prime
}a\right) } \\
&=&\overline{C}\dsum\limits_{k=0}^{h-1}\dsum\limits_{\substack{ \ell =0  \\ %
\ell \neq k{\LARGE +}m}}^{h-1}\sqrt{\left[ b^{\prime }J_{d}A^{k}\left(
A^{k}\right) ^{\prime }J_{d}^{\prime }b\right] \left[ a^{\prime
}J_{d}A^{\ell }\left( A^{\ell }\right) ^{\prime }J_{d}^{\prime }a\right] }%
\sqrt{\left[ b^{\prime }J_{d}A^{\ell }\left( A^{\ell }\right) ^{\prime
}J_{d}^{\prime }b\right] \left[ a^{\prime }J_{d}A^{k}\left( A^{k}\right)
^{\prime }J_{d}^{\prime }a\right] }\text{ \ } \\
&&\left( \text{since }\lambda _{\max }\left( J_{d{\LARGE +}K}^{\prime }J_{d%
{\LARGE +}K}\right) =1\right) \\
&\leq &\overline{C}\dsum\limits_{k=0}^{h-1}\dsum\limits_{\substack{ \ell =0 
\\ \ell \neq k{\LARGE +}m}}^{h-1}\left[ \lambda _{\max }\left\{ A^{k}\left(
A^{k}\right) ^{\prime }\right\} \right] \left[ \lambda _{\max }\left\{
A^{\ell }\left( A^{\ell }\right) ^{\prime }\right\} \right] \left[ b^{\prime
}J_{d}J_{d}^{\prime }b\right] \left[ a^{\prime }J_{d}J_{d}^{\prime }a\right]
\\
&=&\overline{C}\dsum\limits_{k=0}^{h-1}\dsum\limits_{\substack{ \ell =0  \\ %
\ell \neq k{\LARGE +}m}}^{h-1}\left[ \lambda _{\max }\left\{ A^{k}\left(
A^{k}\right) ^{\prime }\right\} \right] \left[ \lambda _{\max }\left\{
A^{\ell }\left( A^{\ell }\right) ^{\prime }\right\} \right] \text{ \ }\left( 
\text{since }J_{d}J_{d}^{\prime }=I_{d}\text{ and }a^{\prime }a=b^{\prime
}b=1\right) \\
&=&\overline{C}\dsum\limits_{k=0}^{h-1}\dsum\limits_{\substack{ \ell =0  \\ %
\ell \neq k{\LARGE +}m}}^{h-1}\left[ \lambda _{\max }\left\{ \left(
A^{k}\right) ^{\prime }A^{k}\right\} \right] \left[ \lambda _{\max }\left\{
\left( A^{\ell }\right) ^{\prime }A^{\ell }\right\} \right]
\end{eqnarray*}%
\begin{eqnarray*}
&=&\overline{C}\dsum\limits_{k=0}^{h-1}\dsum\limits_{\substack{ \ell =0  \\ %
\ell \neq k{\LARGE +}m}}^{h-1}\sigma _{\max }^{2}\left( A^{k}\right) \sigma
_{\max }^{2}\left( A^{\ell }\right) \\
&\leq &\overline{C}\dsum\limits_{k=0}^{h-1}\dsum\limits_{\substack{ \ell =0 
\\ \ell \neq k{\LARGE +}m}}^{h-1}\left( C^{\ast }\right) ^{2}\max \left\{
\left\vert \lambda _{\max }\left( A^{k}\right) \right\vert ^{2},\left\vert
\lambda _{\min }\left( A^{k}\right) \right\vert ^{2}\right\} \max \left\{
\left\vert \lambda _{\max }\left( A^{\ell }\right) \right\vert
^{2},\left\vert \lambda _{\min }\left( A^{\ell }\right) \right\vert
^{2}\right\} \text{ } \\
&&\left( \text{by Assumption 3-7}\right) \\
&=&\overline{C}\dsum\limits_{k=0}^{h-1}\dsum\limits_{\substack{ \ell =0  \\ %
\ell \neq k{\LARGE +}m}}^{h-1}\left( C^{\ast }\right) ^{2}\max \left\{
\left\vert \lambda _{\max }\left( A\right) \right\vert ^{2k},\left\vert
\lambda _{\min }\left( A\right) \right\vert ^{2k}\right\} \max \left\{
\left\vert \lambda _{\max }\left( A\right) \right\vert ^{2\ell },\left\vert
\lambda _{\min }\left( A\right) \right\vert ^{2\ell }\right\} \\
&\leq &\overline{C}\dsum\limits_{k=0}^{h-1}\dsum\limits_{\substack{ \ell =0 
\\ \ell \neq k{\LARGE +}m}}^{h-1}\left( C^{\ast }\right) ^{2}\phi _{\max
}^{2k}\phi _{\max }^{2\ell }\text{ }\left( \text{since }\phi _{\max }=\max
\left\{ \left\vert \lambda _{\max }\left( A\right) \right\vert ,\left\vert
\lambda _{\min }\left( A\right) \right\vert \right\} \right) \\
&=&\overline{C}h^{2}\left( C^{\ast }\right) ^{2}\text{ }\left( \text{since }%
0<\phi _{\max }<1\text{ and }\phi _{\max }^{0}=1\right) \\
&\leq &C\text{ \ }\left( \text{for }\overline{C}h^{2}\left( C^{\ast }\right)
^{2}\leq C<\infty \right) \text{,}
\end{eqnarray*}%
and%
\begin{eqnarray*}
&&\dsum\limits_{j=0}^{h-1}\dsum\limits_{\ell =0}^{h-1}\left\vert \left(
b^{\prime }J_{d}A^{j}J_{d{\LARGE +}K}^{\prime }\otimes a^{\prime
}J_{d}A^{j}J_{d{\LARGE +}K}^{\prime }\right) E\left( \varepsilon _{t{\LARGE +%
}h-j}\otimes \varepsilon _{t{\LARGE +}h-j}\right) E\left( \varepsilon _{t%
{\LARGE +}m{\LARGE +}h-\ell }\otimes \varepsilon _{t{\LARGE +}m{\LARGE +}%
h-\ell }\right) ^{\prime }\right. \\
&&\text{ \ \ \ \ \ \ \ \ \ \ }\left. \times \left( J_{d{\LARGE +}K}\left(
A^{\ell }\right) ^{\prime }J_{d}^{\prime }b\otimes J_{d{\LARGE +}K}\left(
A^{\ell }\right) ^{\prime }J_{d}^{\prime }a\right) \right\vert \\
&\leq &\dsum\limits_{j=0}^{h-1}\dsum\limits_{\ell =0}^{h-1}\left\{ \left[
\left( b^{\prime }J_{d}A^{j}J_{d{\LARGE +}K}^{\prime }\otimes a^{\prime
}J_{d}A^{j}J_{d{\LARGE +}K}^{\prime }\right) E\left( \varepsilon _{t{\LARGE +%
}h-j}\otimes \varepsilon _{t{\LARGE +}h-j}\right) E\left( \varepsilon _{t%
{\LARGE +}h-j}\otimes \varepsilon _{t{\LARGE +}h-j}\right) ^{\prime }\right.
\right. \\
&&\text{ \ \ \ \ \ \ \ \ }\left. \times \left( J_{d{\LARGE +}K}\left(
A^{j}\right) ^{\prime }J_{d}^{\prime }b\otimes J_{d{\LARGE +}K}\left(
A^{j}\right) ^{\prime }J_{d}^{\prime }a\right) \right] ^{1/2} \\
&&\text{ }\times \left[ \left( b^{\prime }J_{d}A^{\ell }J_{d{\LARGE +}%
K}^{\prime }\otimes a^{\prime }J_{d}A^{\ell }J_{d{\LARGE +}K}^{\prime
}\right) E\left( \varepsilon _{t{\LARGE +}m{\LARGE +}h-\ell }\otimes
\varepsilon _{t{\LARGE +}m{\LARGE +}h-\ell }\right) E\left( \varepsilon _{t%
{\LARGE +}m{\LARGE +}h-\ell }\otimes \varepsilon _{t{\LARGE +}m{\LARGE +}%
h-\ell }\right) ^{\prime }\right. \\
&&\text{ \ \ \ \ \ \ \ \ }\left. \text{\ }\left. \times \left( J_{d{\LARGE +}%
K}\left( A^{\ell }\right) ^{\prime }J_{d}^{\prime }b\otimes J_{d{\LARGE +}%
K}\left( A^{\ell }\right) ^{\prime }J_{d}^{\prime }a\right) \right]
^{1/2}\right\} \\
&\leq &\overline{C}\dsum\limits_{j=0}^{h-1}\dsum\limits_{\ell =0}^{h-1}\sqrt{%
\left( b^{\prime }J_{d}A^{j}J_{d{\LARGE +}K}^{\prime }J_{d{\LARGE +}K}\left(
A^{j}\right) ^{\prime }J_{d}^{\prime }b\right) \left( a^{\prime
}J_{d}A^{j}J_{d{\LARGE +}K}^{\prime }J_{d{\LARGE +}K}\left( A^{j}\right)
^{\prime }J_{d}^{\prime }a\right) } \\
&&\text{ \ \ \ \ \ \ \ \ \ \ }\times \sqrt{\left( b^{\prime }J_{d}A^{\ell
}J_{d{\LARGE +}K}^{\prime }J_{d{\LARGE +}K}\left( A^{\ell }\right) ^{\prime
}J_{d}^{\prime }b\right) \left( a^{\prime }J_{d}A^{\ell }J_{d{\LARGE +}%
K}^{\prime }J_{d{\LARGE +}K}\left( A^{\ell }\right) ^{\prime }J_{d}^{\prime
}a\right) } \\
&=&\overline{C}\dsum\limits_{j=0}^{h-1}\dsum\limits_{\ell =0}^{h-1}\sqrt{%
\left[ b^{\prime }J_{d}A^{j}\left( A^{j}\right) ^{\prime }J_{d}^{\prime }b%
\right] \left[ a^{\prime }J_{d}A^{j}\left( A^{j}\right) ^{\prime
}J_{d}^{\prime }a\right] }\sqrt{\left[ b^{\prime }J_{d}A^{\ell }\left(
A^{\ell }\right) ^{\prime }J_{d}^{\prime }b\right] \left[ a^{\prime
}J_{d}A^{\ell }\left( A^{\ell }\right) ^{\prime }J_{d}^{\prime }a\right] }%
\text{ \ } \\
&&\left( \text{since }\lambda _{\max }\left( J_{d{\LARGE +}K}^{\prime }J_{d%
{\LARGE +}K}\right) =1\right) \\
&\leq &\overline{C}\dsum\limits_{j=0}^{h-1}\dsum\limits_{\ell =0}^{h-1}\left[
\lambda _{\max }\left\{ A^{j}\left( A^{j}\right) ^{\prime }\right\} \right] %
\left[ \lambda _{\max }\left\{ A^{\ell }\left( A^{\ell }\right) ^{\prime
}\right\} \right] \left[ b^{\prime }J_{d}J_{d}^{\prime }b\right] \left[
a^{\prime }J_{d}J_{d}^{\prime }a\right] \\
&=&\overline{C}\dsum\limits_{j=0}^{h-1}\dsum\limits_{\ell =0}^{h-1}\left[
\lambda _{\max }\left\{ A^{j}\left( A^{j}\right) ^{\prime }\right\} \right] %
\left[ \lambda _{\max }\left\{ A^{\ell }\left( A^{\ell }\right) ^{\prime
}\right\} \right] \text{ \ }\left( \text{since }J_{d}J_{d}^{\prime }=I_{d}%
\text{ and }a^{\prime }a=b^{\prime }b=1\right) \\
&=&\overline{C}\dsum\limits_{j=0}^{h-1}\dsum\limits_{\ell =0}^{h-1}\left[
\lambda _{\max }\left\{ \left( A^{j}\right) ^{\prime }A^{j}\right\} \right] %
\left[ \lambda _{\max }\left\{ \left( A^{\ell }\right) ^{\prime }A^{\ell
}\right\} \right] \\
&=&\overline{C}\dsum\limits_{j=0}^{h-1}\dsum\limits_{\ell =0}^{h-1}\sigma
_{\max }^{2}\left( A^{j}\right) \sigma _{\max }^{2}\left( A^{\ell }\right)
\end{eqnarray*}%
\begin{eqnarray*}
&\leq &\overline{C}\dsum\limits_{j=0}^{h-1}\dsum\limits_{\ell
=0}^{h-1}\left( C^{\ast }\right) ^{2}\max \left\{ \left\vert \lambda _{\max
}\left( A^{j}\right) \right\vert ^{2},\left\vert \lambda _{\min }\left(
A^{j}\right) \right\vert ^{2}\right\} \max \left\{ \left\vert \lambda _{\max
}\left( A^{\ell }\right) \right\vert ^{2},\left\vert \lambda _{\min }\left(
A^{\ell }\right) \right\vert ^{2}\right\} \\
&&\text{ }\left( \text{by Assumption 3-7}\right) \\
&=&\overline{C}\dsum\limits_{j=0}^{h-1}\dsum\limits_{\ell =0}^{h-1}\left(
C^{\ast }\right) ^{2}\max \left\{ \left\vert \lambda _{\max }\left( A\right)
\right\vert ^{2j},\left\vert \lambda _{\min }\left( A\right) \right\vert
^{2j}\right\} \max \left\{ \left\vert \lambda _{\max }\left( A\right)
\right\vert ^{2\ell },\left\vert \lambda _{\min }\left( A\right) \right\vert
^{2\ell }\right\} \\
&\leq &\overline{C}\dsum\limits_{j=0}^{h-1}\dsum\limits_{\ell
=0}^{h-1}\left( C^{\ast }\right) ^{2}\phi _{\max }^{2j}\phi _{\max }^{2\ell }%
\text{ }\left( \text{since }\phi _{\max }=\max \left\{ \left\vert \lambda
_{\max }\left( A\right) \right\vert ,\left\vert \lambda _{\min }\left(
A\right) \right\vert \right\} \right) \\
&=&\overline{C}h^{2}\left( C^{\ast }\right) ^{2}\text{ }\left( \text{since }%
0<\phi _{\max }<1\text{ and }\phi _{\max }^{0}=1\right) \\
&\leq &C\text{ \ }\left( \text{for }\overline{C}h^{2}\left( C^{\ast }\right)
^{2}\leq C<\infty \right) \text{,}
\end{eqnarray*}%
where upper bounds given above have made use of the fact that for all $t$
and $s$%
\begin{eqnarray*}
&&E\left[ \varepsilon _{t}\varepsilon _{t}^{\prime }\otimes \varepsilon
_{s}\varepsilon _{s}^{\prime }\right] \\
&=&E\left[ \left( \varepsilon _{t}\otimes \varepsilon _{s}\right) \left(
\varepsilon _{t}\otimes \varepsilon _{s}\right) ^{\prime }\right] \\
&\leq &tr\left\{ E\left[ \left( \varepsilon _{t}\otimes \varepsilon
_{s}\right) \left( \varepsilon _{t}\otimes \varepsilon _{s}\right) ^{\prime }%
\right] \right\} \cdot I_{\left( d{\LARGE +}K\right) ^{2}}\text{ } \\
&&\left( \text{where the inequality holds in positive semi-definite sense}%
\right) \\
&=&E\left[ tr\left\{ \left( \varepsilon _{t}\otimes \varepsilon _{s}\right)
\left( \varepsilon _{t}\otimes \varepsilon _{s}\right) ^{\prime }\right\} %
\right] \cdot I_{\left( d{\LARGE +}K\right) ^{2}} \\
&=&E\left[ tr\left\{ \left( \varepsilon _{t}\otimes \varepsilon _{s}\right)
^{\prime }\left( \varepsilon _{t}\otimes \varepsilon _{s}\right) \right\} %
\right] \cdot I_{\left( d{\LARGE +}K\right) ^{2}} \\
&=&E\left[ \varepsilon _{t}^{\prime }\varepsilon _{t}\varepsilon
_{s}^{\prime }\varepsilon _{s}\right] \cdot I_{\left( d{\LARGE +}K\right)
^{2}} \\
&=&E\left[ \left\Vert \varepsilon _{t}\right\Vert _{2}^{2}\left\Vert
\varepsilon _{s}\right\Vert _{2}^{2}\right] \cdot I_{\left( d{\LARGE +}%
K\right) ^{2}} \\
&\leq &\sup_{t}E\left[ \left\Vert \varepsilon _{t}\right\Vert _{2}^{4}\right]
\cdot I_{\left( d{\LARGE +}K\right) ^{2}} \\
&\leq &\overline{C}\cdot I_{d^{2}}\text{ }\left( \text{by Assumption 3-2(b)}%
\right)
\end{eqnarray*}%
and%
\begin{eqnarray*}
E\left( \varepsilon _{t}\otimes \varepsilon _{t}\right) E\left( \varepsilon
_{t}\otimes \varepsilon _{t}\right) ^{\prime } &\leq &tr\left\{ E\left(
\varepsilon _{t}\otimes \varepsilon _{t}\right) E\left( \varepsilon
_{t}\otimes \varepsilon _{t}\right) ^{\prime }\right\} \cdot I_{\left( d%
{\LARGE +}K\right) ^{2}} \\
&&\left( \text{where the inequality holds in positive semi-definite sense}%
\right) \\
&=&E\left( \varepsilon _{t}\otimes \varepsilon _{t}\right) ^{\prime }E\left(
\varepsilon _{t}\otimes \varepsilon _{t}\right) \cdot I_{\left( d{\LARGE +}%
K\right) ^{2}} \\
&=&\dsum\limits_{g=1}^{d}\dsum\limits_{\ell =1}^{d}\left( E\left[
\varepsilon _{gt}\varepsilon _{\ell t}\right] \right) ^{2}\cdot I_{\left( d%
{\LARGE +}K\right) ^{2}} \\
&\leq &\dsum\limits_{g=1}^{d}\dsum\limits_{\ell =1}^{d}\left( E\left\vert
\varepsilon _{gt}\varepsilon _{\ell t}\right\vert \right) ^{2}\cdot
I_{\left( d{\LARGE +}K\right) ^{2}} \\
&\leq &\dsum\limits_{g=1}^{d}\dsum\limits_{\ell =1}^{d}E\left[ \varepsilon
_{gt}^{2}\right] E\left[ \varepsilon _{\ell t}^{2}\right] \cdot I_{\left( d%
{\LARGE +}K\right) ^{2}} \\
&=&E\left[ \dsum\limits_{g=1}^{d}\varepsilon _{gt}^{2}\right] E\left[
\dsum\limits_{\ell =1}^{d}\varepsilon _{\ell t}^{2}\right] \cdot I_{\left( d%
{\LARGE +}K\right) ^{2}} \\
&=&\left( E\left\Vert \varepsilon _{t}\right\Vert _{2}^{2}\right) ^{2}\cdot
I_{\left( d{\LARGE +}K\right) ^{2}} \\
&\leq &\overline{C}\cdot I_{\left( d{\LARGE +}K\right) ^{2}}\text{ }\left( 
\text{by Assumption 3-2(b)}\right)
\end{eqnarray*}%
for some positive constant $\overline{C}$. It follows from these
calculations that, for $1\leq m\leq h-1$ where $h\geq 2$, we have%
\begin{eqnarray*}
&&\left\vert E\left\{ \left( a^{\prime }\eta _{t+h}\eta _{t+h}^{\prime }b-E 
\left[ a^{\prime }\eta _{t+h}\eta _{t+h}^{\prime }b\right] \right) \left(
a^{\prime }\eta _{t+m+h}\eta _{t+m+h}^{\prime }b-E\left[ a^{\prime }\eta
_{t+m+h}\eta _{t+m+h}^{\prime }b\right] \right) \right\} \right\vert \\
&\leq &\dsum\limits_{j=0}^{h-1}\left\vert \left( b^{\prime }J_{d}A^{j}J_{d%
{\LARGE +}K}^{\prime }\otimes a^{\prime }J_{d}A^{j}J_{d{\LARGE +}K}^{\prime
}\right) E\left( \varepsilon _{t{\LARGE +}h-j}\varepsilon _{t{\LARGE +}%
h-j}^{\prime }\otimes \varepsilon _{t{\LARGE +}h-j}\varepsilon _{t{\LARGE +}%
h-j}^{\prime }\right) \right. \\
&&\text{ \ \ \ \ \ }\left. \times \left( J_{d{\LARGE +}K}\left( A^{j}\right)
^{\prime }J_{d}^{\prime }b\otimes J_{d{\LARGE +}K}\left( A^{j}\right)
^{\prime }J_{d}^{\prime }a\right) \right\vert \\
&&+\dsum\limits_{j=0}^{h-1}\dsum\limits_{\substack{ k=0  \\ k\neq j}}%
^{h-1}\left\vert \left( b^{\prime }J_{d}A^{k}J_{d{\LARGE +}K}^{\prime
}\otimes a^{\prime }J_{d}A^{j}J_{d{\LARGE +}K}^{\prime }\right) E\left(
\varepsilon _{t{\LARGE +}h-k}\varepsilon _{t{\LARGE +}h-k}^{\prime }\otimes
\varepsilon _{t{\LARGE +}h-j}\varepsilon _{t{\LARGE +}h-j}^{\prime }\right)
\right. \\
&&\text{ \ \ \ \ \ \ \ \ \ \ \ \ \ \ }\left. \times \left( J_{d{\LARGE +}%
K}\left( A^{k}\right) ^{\prime }J_{d}^{\prime }b\otimes J_{d{\LARGE +}%
K}\left( A^{j}\right) ^{\prime }J_{d}^{\prime }a\right) \right\vert \\
&&+\dsum\limits_{k=0}^{h-1}\dsum\limits_{\substack{ \ell =0  \\ \ell \neq k%
{\LARGE +}m}}^{h-1}\left\vert \left( b^{\prime }J_{d}A^{k}J_{d{\LARGE +}%
K}^{\prime }\otimes a^{\prime }J_{d}A^{k}J_{d{\LARGE +}K}^{\prime }\right)
E\left( \varepsilon _{t{\LARGE +}h-k}\varepsilon _{t{\LARGE +}m{\LARGE +}%
h-\ell }^{\prime }\otimes \varepsilon _{t{\LARGE +}h-k}\varepsilon _{t%
{\LARGE +}m{\LARGE +}h-\ell }^{\prime }\right) \right. \\
&&\text{ \ \ \ \ \ \ \ \ \ \ \ \ \ \ \ \ }\left. \times \left( J_{d{\LARGE +}%
K}\left( A^{\ell }\right) ^{\prime }J_{d}^{\prime }b\otimes J_{d{\LARGE +}%
K}\left( A^{\ell }\right) ^{\prime }J_{d}^{\prime }a\right) \right\vert \\
&&+\dsum\limits_{k=0}^{h-1}\dsum\limits_{\substack{ \ell =0  \\ \ell \neq k%
{\LARGE +}m}}^{h-1}\left\vert \left( b^{\prime }J_{d}A^{k}J_{d{\LARGE +}%
K}^{\prime }\otimes a^{\prime }J_{d}A^{\ell }J_{d{\LARGE +}K}^{\prime
}\right) E\left( \varepsilon _{t{\LARGE +}h-k}\varepsilon _{t{\LARGE +}m%
{\LARGE +}h-\ell }^{\prime }\otimes \varepsilon _{t{\LARGE +}m{\LARGE +}%
h-\ell }\varepsilon _{t{\LARGE +}h-k}^{\prime }\right) \right. \\
&&\text{ \ \ \ \ \ \ \ \ \ \ \ \ \ \ \ \ \ }\left. \times \left( J_{d{\LARGE %
+}K}\left( A^{\ell }\right) ^{\prime }J_{d}^{\prime }b\otimes J_{d{\LARGE +}%
K}\left( A^{k}\right) ^{\prime }J_{d}^{\prime }a\right) \right\vert \\
&&+\dsum\limits_{j=0}^{h-1}\dsum\limits_{\ell =0}^{h-1}\left\vert \left(
b^{\prime }J_{d}A^{j}J_{d{\LARGE +}K}^{\prime }\otimes a^{\prime
}J_{d}A^{j}J_{d{\LARGE +}K}^{\prime }\right) E\left( \varepsilon _{t{\LARGE +%
}h-j}\otimes \varepsilon _{t{\LARGE +}h-j}\right) E\left( \varepsilon _{t%
{\LARGE +}m{\LARGE +}h-\ell }\otimes \varepsilon _{t{\LARGE +}m{\LARGE +}%
h-\ell }\right) ^{\prime }\right. \\
&&\text{ \ \ \ \ \ \ \ \ \ \ \ \ \ \ \ \ \ \ }\left. \times \left( J_{d%
{\LARGE +}K}\left( A^{\ell }\right) ^{\prime }J_{d}^{\prime }b\otimes J_{d%
{\LARGE +}K}\left( A^{\ell }\right) ^{\prime }J_{d}^{\prime }a\right)
\right\vert \\
&\leq &5C
\end{eqnarray*}%
so that, when $h\geq 2$,%
\begin{eqnarray*}
&&\left\vert \frac{2}{T_{h}^{2}}\dsum\limits_{t=p}^{T-h-1}\dsum%
\limits_{m=1}^{T-h-t}E\left\{ \left( a^{\prime }\eta _{t+h}\eta
_{t+h}^{\prime }b-E\left[ a^{\prime }\eta _{t+h}\eta _{t+h}^{\prime }b\right]
\right) \left( a^{\prime }\eta _{t+m+h}\eta _{t+m+h}^{\prime }b-E\left[
a^{\prime }\eta _{t+m+h}\eta _{t+m+h}^{\prime }b\right] \right) \right\}
\right\vert \\
&=&\left\vert \frac{2}{T_{h}^{2}}\dsum\limits_{t=p}^{T-h-1}\dsum%
\limits_{m=1}^{\min \left\{ h-1,T-h-t\right\} }E\left\{ \left( a^{\prime
}\eta _{t+h}\eta _{t+h}^{\prime }b-E\left[ a^{\prime }\eta _{t+h}\eta
_{t+h}^{\prime }b\right] \right) \right. \right. \\
&&\text{ \ \ \ \ \ \ \ \ \ \ \ \ \ \ \ \ \ \ \ \ \ \ \ \ \ \ \ \ \ }\left. 
\text{\ \ }\left. \times \left( a^{\prime }\eta _{t+m+h}\eta
_{t+m+h}^{\prime }b-E\left[ a^{\prime }\eta _{t+m+h}\eta _{t+m+h}^{\prime }b%
\right] \right) \right\} \right\vert \\
&\leq &\frac{2}{T_{h}^{2}}\dsum\limits_{t=p}^{T-h-1}\dsum\limits_{m=1}^{\min
\left\{ h-1,T-h-t\right\} }\left\vert E\left\{ \left( a^{\prime }\eta
_{t+h}\eta _{t+h}^{\prime }b-E\left[ a^{\prime }\eta _{t+h}\eta
_{t+h}^{\prime }b\right] \right) \right. \right. \\
&&\text{ \ \ \ \ \ \ \ \ \ \ \ \ \ \ \ \ \ \ \ \ \ \ \ \ \ \ \ \ \ \ \ }%
\left. \text{\ }\left. \times \left( a^{\prime }\eta _{t+m+h}\eta
_{t+m+h}^{\prime }b-E\left[ a^{\prime }\eta _{t+m+h}\eta _{t+m+h}^{\prime }b%
\right] \right) \right\} \right\vert \\
&\leq &\frac{2}{T_{h}}\frac{T-h-p}{T_{h}}\left( h-1\right) 5C \\
&<&\frac{10\left( h-1\right) C}{T_{h}}\text{ }\left( \text{ since }%
T_{h}=T-h-p+1\right) \\
&=&O\left( \frac{1}{T}\right)
\end{eqnarray*}

\noindent Putting everything together for the case where $h\geq 2$, we see
that%
\begin{eqnarray}
&&E\left[ \frac{1}{T_{h}}\dsum\limits_{t=p}^{T-h}\left( a^{\prime }\eta
_{t+h}\eta _{t+h}^{\prime }b-E\left[ a^{\prime }\eta _{t+h}\eta
_{t+h}^{\prime }b\right] \right) \right] ^{2}  \notag \\
&=&\frac{1}{T_{h}^{2}}\dsum\limits_{t=p}^{T-h}E\left[ \left( a^{\prime }\eta
_{t+h}\eta _{t+h}^{\prime }b-E\left[ a^{\prime }\eta _{t+h}\eta
_{t+h}^{\prime }b\right] \right) ^{2}\right]  \notag \\
&&+\frac{2}{T_{h}^{2}}\dsum\limits_{t=p}^{T-h-1}\dsum\limits_{m=1}^{T-h-t}E%
\left\{ \left( a^{\prime }\eta _{t+h}\eta _{t+h}^{\prime }b-E\left[
a^{\prime }\eta _{t+h}\eta _{t+h}^{\prime }b\right] \right) \left( a^{\prime
}\eta _{t+m+h}\eta _{t+m+h}^{\prime }b-E\left[ a^{\prime }\eta _{t+m+h}\eta
_{t+m+h}^{\prime }b\right] \right) \right\}  \notag \\
&=&O\left( \frac{1}{T}\right) +O\left( \frac{1}{T}\right)  \notag \\
&=&O\left( \frac{1}{T}\right)  \label{MSE h>=2}
\end{eqnarray}

\noindent \qquad In light of the results given in expressions (\ref{MSE h=1}%
) and (\ref{MSE h>=2}), we can apply Markov's inequality to show that
regardless of whether $h=1$ or $h\geq 2$%
\begin{equation*}
\frac{1}{T_{h}}\dsum\limits_{t=p}^{T-h}a^{\prime }\eta _{t+h}\eta
_{t+h}^{\prime }b-\frac{1}{T_{h}}\dsum\limits_{t=p}^{T-h}E\left[ a^{\prime
}\eta _{t+h}\eta _{t+h}^{\prime }b\right] =O_{p}\left( \frac{1}{\sqrt{T}}%
\right) \text{. }
\end{equation*}%
Moreover, since the above result holds for all $a,b\in \mathbb{R}^{d}$ such
that $\left\Vert a\right\Vert _{2}=\left\Vert b\right\Vert _{2}=1$, we
further deduce that for all (fixed) positive integer $h$%
\begin{equation*}
\frac{1}{T_{h}}\dsum\limits_{t=p}^{T-h}\eta _{t+h}\eta _{t+h}^{\prime }-%
\frac{1}{T_{h}}\dsum\limits_{t=p}^{T-h}E\left[ \eta _{t+h}\eta
_{t+h}^{\prime }\right] =O_{p}\left( \frac{1}{\sqrt{T}}\right) \text{. }%
\square
\end{equation*}

\medskip

\noindent \textbf{Lemma D-3:} Suppose that $A$ is an $N\times N$ symmetric
matrix which we can partition as%
\begin{equation*}
A=\left( 
\begin{array}{cc}
\underset{r\times r}{A_{11}} & \underset{r\times \left( N{\large -r}\right) }%
{A_{12}} \\ 
\underset{\left( N{\large -r}\right) \times r}{A_{21}} & \underset{\left( N%
{\large -r}\right) \times \left( N{\large -r}\right) }{A_{22}}%
\end{array}%
\right)
\end{equation*}%
Then, 
\begin{equation*}
\left\Vert A_{21}\right\Vert _{2}\leq \left\Vert A\right\Vert _{2}\text{.}
\end{equation*}%
\textbf{Proof of Lemma D-3: }Define 
\begin{equation*}
\underset{N\times r}{B_{1}}=\left( 
\begin{array}{c}
I_{r} \\ 
0%
\end{array}%
\right) \text{.}
\end{equation*}%
Let $\overline{\upsilon }\in \mathbb{R}^{r}$ be such that $\left\Vert 
\overline{\upsilon }\right\Vert _{2}=1$ and%
\begin{equation*}
\overline{\upsilon }^{\prime }A_{21}^{\prime }A_{21}\overline{\upsilon }%
=\max_{\left\Vert \upsilon \right\Vert _{2}=1}\upsilon ^{\prime
}A_{21}^{\prime }A_{21}\upsilon
\end{equation*}%
It follows that%
\begin{eqnarray*}
\left\Vert A_{21}\right\Vert _{2} &=&\sqrt{\overline{\upsilon }^{\prime
}A_{21}^{\prime }A_{21}\overline{\upsilon }} \\
&\leq &\sqrt{\overline{\upsilon }^{\prime }A_{11}^{\prime }A_{11}\overline{%
\upsilon }+\overline{\upsilon }^{\prime }A_{21}^{\prime }A_{21}\overline{%
\upsilon }} \\
&=&\sqrt{\overline{\upsilon }^{\prime }B_{1}^{\prime }A^{\prime }AB_{1}%
\overline{\upsilon }} \\
&\leq &\sqrt{\max_{\left\Vert \upsilon \right\Vert _{2}=1}\upsilon ^{\prime
}A^{\prime }A\upsilon }\text{ }\left( \text{noting that }\left\Vert B_{1}%
\overline{\upsilon }\right\Vert _{2}=\sqrt{\overline{\upsilon }^{\prime
}B_{1}^{\prime }B_{1}\overline{\upsilon }}=\sqrt{\overline{\upsilon }%
^{\prime }\overline{\upsilon }}=1\right) \\
&=&\left\Vert A\right\Vert _{2}\text{. }\square
\end{eqnarray*}

\noindent \textbf{Remark: }This is a well-known linear algebraic result. A
similar result has also been given in the beginning of section 6 of
Johnstone and Lu (2009).

\medskip

\noindent \textbf{Lemma D-4: }Let 
\begin{equation}
M_{FF}=\frac{1}{T_{0}}\dsum\limits_{t=p}^{T}E\left[ \underline{F}_{t}%
\underline{F}_{t}^{\prime }\right]  \label{MFF def}
\end{equation}%
where $T_{0}=T-p+1$. Then, under Assumptions 3-1, 3-2(a)-(b), 3-2(d), 3-5,
and 3-7; there exists a positive constant $\underline{C}$ such that 
\begin{equation*}
\lambda _{\min }\left\{ M_{FF}\right\} \geq \underline{C}>0
\end{equation*}%
for all $T>p-1$.

\medskip

\noindent \textbf{Proof of Lemma D-4: }

To proceed, note that we can write%
\begin{equation*}
\frac{1}{T_{0}}\dsum\limits_{t=p}^{T}\left( 
\begin{array}{cc}
E\left[ \underline{Y}_{t}\underline{Y}_{t}^{\prime }\right] & E\left[ 
\underline{Y}_{t}\underline{F}_{t}^{\prime }\right] \\ 
E\left[ \underline{F}_{t}\underline{Y}_{t}^{\prime }\right] & E\left[ 
\underline{F}_{t}\underline{F}_{t}^{\prime }\right]%
\end{array}%
\right) =\mathcal{P}_{\left( d{\LARGE +}K\right) p}\frac{1}{T_{0}}%
\dsum\limits_{t=p}^{T}E\left[ \underline{W}_{t}\underline{W}_{t}^{\prime }%
\right] \mathcal{P}_{\left( d{\LARGE +}K\right) p}^{\prime }
\end{equation*}%
from which it follows that%
\begin{eqnarray*}
\lambda _{\min }\left\{ \frac{1}{T_{0}}\dsum\limits_{t=p}^{T}\left( 
\begin{array}{cc}
E\left[ \underline{Y}_{t}\underline{Y}_{t}^{\prime }\right] & E\left[ 
\underline{Y}_{t}\underline{F}_{t}^{\prime }\right] \\ 
E\left[ \underline{F}_{t}\underline{Y}_{t}^{\prime }\right] & E\left[ 
\underline{F}_{t}\underline{F}_{t}^{\prime }\right]%
\end{array}%
\right) \right\} &=&\lambda _{\min }\left\{ \mathcal{P}_{\left( d{\LARGE +}%
K\right) p}\frac{1}{T_{0}}\dsum\limits_{t=p}^{T}E\left[ \underline{W}_{t}%
\underline{W}_{t}^{\prime }\right] \mathcal{P}_{\left( d{\LARGE +}K\right)
p}^{\prime }\right\} \\
&\geq &\lambda _{\min }\left\{ \frac{1}{T_{0}}\dsum\limits_{t=p}^{T}E\left[ 
\underline{W}_{t}\underline{W}_{t}^{\prime }\right] \right\} \lambda _{\min
}\left\{ \mathcal{P}_{\left( d{\LARGE +}K\right) p}\mathcal{P}_{\left( d%
{\LARGE +}K\right) p}^{\prime }\right\} \\
&=&\lambda _{\min }\left\{ \frac{1}{T_{0}}\dsum\limits_{t=p}^{T}E\left[ 
\underline{W}_{t}\underline{W}_{t}^{\prime }\right] \right\} \lambda _{\min
}\left\{ I_{\left( d{\LARGE +}K\right) p}\right\} \\
&&\left( \text{since }\mathcal{P}_{\left( d{\LARGE +}K\right) p}\text{ is an
orthogonal matrix}\right) \\
&=&\lambda _{\min }\left\{ \frac{1}{T_{0}}\dsum\limits_{t=p}^{T}E\left[ 
\underline{W}_{t}\underline{W}_{t}^{\prime }\right] \right\}
\end{eqnarray*}%
Next, note that%
\begin{eqnarray*}
\frac{1}{T_{0}}\dsum\limits_{t=p}^{T}E\left[ \underline{W}_{t}\underline{W}%
_{t}^{\prime }\right] &=&\frac{1}{T_{0}}\dsum\limits_{t=p}^{T}\left(
I_{\left( d{\LARGE +}K\right) p}-A\right) ^{-1}J_{d{\LARGE +}K}^{\prime }\mu
\mu ^{\prime }J_{d{\LARGE +}K}\left( I_{\left( d{\LARGE +}K\right)
p}-A^{\prime }\right) ^{-1} \\
&&+\frac{1}{T_{0}}\dsum\limits_{t=p}^{T}\dsum\limits_{j=0}^{\infty }A^{j}J_{d%
{\LARGE +}K}^{\prime }E\left[ \varepsilon _{t-j}\varepsilon _{t-j}^{\prime }%
\right] J_{d{\LARGE +}K}\left( A^{j}\right) ^{\prime } \\
&=&\left( I_{\left( d{\LARGE +}K\right) p}-A\right) ^{-1}J_{d{\LARGE +}%
K}^{\prime }\mu \mu ^{\prime }J_{d{\LARGE +}K}\left( I_{\left( d{\LARGE +}%
K\right) p}-A^{\prime }\right) ^{-1} \\
&&+\dsum\limits_{j=0}^{\infty }A^{j}J_{d{\LARGE +}K}^{\prime }\frac{1}{T_{0}}%
\dsum\limits_{t=p}^{T}E\left[ \varepsilon _{t-j}\varepsilon _{t-j}^{\prime }%
\right] J_{d{\LARGE +}K}\left( A^{j}\right) ^{\prime }
\end{eqnarray*}%
so that there exists a positive constant $\underline{C}$ such that%
\begin{eqnarray*}
&&\lambda _{\min }\left\{ \frac{1}{T_{0}}\dsum\limits_{t=p}^{T}E\left[ 
\underline{W}_{t}\underline{W}_{t}^{\prime }\right] \right\} \\
&\geq &\lambda _{\min }\left\{ \left( I_{\left( d{\LARGE +}K\right)
p}-A\right) ^{-1}J_{d{\LARGE +}K}^{\prime }\mu \mu ^{\prime }J_{d{\LARGE +}%
K}\left( I_{\left( d{\LARGE +}K\right) p}-A^{\prime }\right) ^{-1}\right\} \\
&&+\lambda _{\min }\left\{ \dsum\limits_{j=0}^{\infty }A^{j}J_{d{\LARGE +}%
K}^{\prime }\frac{1}{T_{0}}\dsum\limits_{t=p}^{T}E\left[ \varepsilon
_{t-j}\varepsilon _{t-j}^{\prime }\right] J_{d{\LARGE +}K}\left(
A^{j}\right) ^{\prime }\right\} \\
&&\left( \text{by Weyl's Theorem (see Theorem 4.3.1 of Horn and Johnson,
1985)}\right) \\
&\geq &\lambda _{\min }\left\{ \dsum\limits_{j=0}^{\infty }A^{j}J_{d{\LARGE +%
}K}^{\prime }\frac{1}{T_{0}}\dsum\limits_{t=p}^{T}E\left[ \varepsilon
_{t-j}\varepsilon _{t-j}^{\prime }\right] J_{d{\LARGE +}K}\left(
A^{j}\right) ^{\prime }\right\} \\
&\geq &\underline{c}>0\text{ for all }T>p-1\text{ }\left( \text{by the
result given in part (a) of Lemma D-1}\right)
\end{eqnarray*}%
It then follows that%
\begin{eqnarray*}
&&\lambda _{\min }\left\{ M_{FF}\right\} \\
&=&\lambda _{\min }\left\{ \frac{1}{T_{0}}\dsum\limits_{t=p}^{T}E\left[ 
\underline{F}_{t}\underline{F}_{t}^{\prime }\right] \right\} \\
&\geq &\lambda _{\min }\left\{ \frac{1}{T_{0}}\dsum\limits_{t=p}^{T}\left( 
\begin{array}{cc}
E\left[ \underline{Y}_{t}\underline{Y}_{t}^{\prime }\right] & E\left[ 
\underline{Y}_{t}\underline{F}_{t}^{\prime }\right] \\ 
E\left[ \underline{F}_{t}\underline{Y}_{t}^{\prime }\right] & E\left[ 
\underline{F}_{t}\underline{F}_{t}^{\prime }\right]%
\end{array}%
\right) \right\} \\
&&\left( \text{ by the Poincar\'{e} separation theorem (see Corollary 4.3.16
of Horn and Johnson, 1985)\ }\right) \\
&\geq &\lambda _{\min }\left\{ \frac{1}{T_{0}}\dsum\limits_{t=p}^{T}E\left[ 
\underline{W}_{t}\underline{W}_{t}^{\prime }\right] \right\} \\
&\geq &\underline{C}>0\text{ for all }T>p-1\text{,}
\end{eqnarray*}%
as required. $\square $

\medskip

\noindent \textbf{Lemma D-5: }Let $T_{h}=T-h-p+1$ where $h$ is a (fixed)
non-negative integer and $p$ is a (fixed) positive integer. Suppose that
Assumption 3-3 hold. Then,

\begin{enumerate}
\item[(a)] 
\begin{equation*}
\frac{1}{T_{h}}\dsum\limits_{\substack{ v,w=p  \\ v\leq w}}^{T-h}\left\vert E%
\left[ u_{iv}u_{iw}\right] \right\vert =O\left( 1\right)
\end{equation*}

\item[(b)] 
\begin{equation*}
\frac{1}{T_{h}}\dsum\limits_{\substack{ t,s,g=p  \\ t\leq s\leq g}}%
^{T-h}\left\vert E\left( u_{it}u_{is}u_{ig}\right) \right\vert =O\left(
1\right)
\end{equation*}

\item[(c)] \textbf{\ }%
\begin{equation*}
\frac{1}{T_{h}^{2}}\dsum\limits_{\substack{ t,s,g,v=p  \\ t\leq s\leq g\leq
v }}^{T-h}\left\vert E\left( u_{it}u_{is}u_{ig}u_{iv}\right) \right\vert
=O\left( 1\right)
\end{equation*}
\end{enumerate}

\medskip

\noindent \textbf{Proof of Lemma D-5: }

To show part (a), first write%
\begin{equation}
\frac{1}{T_{h}}\dsum\limits_{\substack{ v,w=p  \\ v\leq w}}^{T-h}\left\vert E%
\left[ u_{iv}u_{iw}\right] \right\vert =\frac{1}{T_{h}}\dsum%
\limits_{v=p}^{T-h}E\left[ u_{iv}^{2}\right] +\frac{1}{T_{h}}\dsum\limits 
_{\substack{ v,w=p  \\ v<w}}^{T-h}\left\vert E\left[ u_{iv}u_{iw}\right]
\right\vert  \label{T1Euguh}
\end{equation}%
Consider now the first term on the right-hand side of expression (\ref%
{T1Euguh}). Note that, trivially, by Assumption 3-3(b),%
\begin{equation}
\frac{1}{T_{h}}\dsum\limits_{v=p}^{T-h}E\left[ u_{iv}^{2}\right] \leq
C=O\left( 1\right)  \label{T1term2 1}
\end{equation}%
For the second term on the right-hand side of expression (\ref{T1Euguh}),
note that by Assumption 3-3(c), $\left\{ u_{it}\right\} _{t=-\infty
}^{\infty }$ is $\beta $-mixing with $\beta $ mixing coefficient satisfying 
\begin{equation*}
\beta _{i}\left( m\right) \leq a_{1}\exp \left\{ -a_{2}m\right\} \text{.}
\end{equation*}%
for every $i$. Since $\alpha _{i,m}\leq \beta _{i}\left( m\right) $, it
follows that $\left\{ u_{it}\right\} _{t=-\infty }^{\infty }$ is $\alpha $%
-mixing as well, with $\alpha $ mixing coefficient satisfying%
\begin{equation*}
\alpha _{i,m}\leq a_{1}\exp \left\{ -a_{2}m\right\} \text{ for every }i\text{%
. }
\end{equation*}

\noindent Hence, in this case, we can apply Lemma C-3 with $p=6$ and $r=5/4$
to obtain%
\begin{eqnarray*}
&&\frac{1}{T_{h}}\dsum\limits_{\substack{ v,w=p  \\ v<w}}^{T-h}\left\vert E%
\left[ u_{iv}u_{iw}\right] \right\vert \\
&\leq &\frac{1}{T_{h}}\dsum\limits_{\substack{ v,w=p  \\ v<w}}^{T-h}2\left(
2^{1-\frac{{\large 1}}{{\large 6}}}+1\right) \left[ a_{1}\exp \left\{
-a_{2}\left( w-v\right) \right\} \right] ^{1-\frac{{\large 1}}{{\large 6}}-%
\frac{{\large 4}}{{\large 5}}}\left( E\left\vert u_{iv}\right\vert
^{6}\right) ^{\frac{{\large 1}}{{\large 6}}}\left( E\left\vert
u_{iw}\right\vert ^{\frac{{\large 5}}{{\large 4}}}\right) ^{\frac{{\large 4}%
}{{\large 5}}}
\end{eqnarray*}%
Application of Liapunov's inequality then gives us%
\begin{eqnarray*}
\left( E\left\vert u_{iv}\right\vert ^{6}\right) ^{\frac{{\large 1}}{{\large %
6}}}\left( E\left\vert u_{iw}\right\vert ^{\frac{{\large 5}}{{\large 4}}%
}\right) ^{\frac{{\large 4}}{{\large 5}}} &\leq &\left( E\left\vert
u_{iv}\right\vert ^{6}\right) ^{\frac{{\large 1}}{{\large 6}}}\left(
E\left\vert u_{iw}\right\vert ^{6}\right) ^{\frac{{\large 1}}{{\large 6}}} \\
&\leq &\left( \sup_{t}E\left\vert u_{it}\right\vert ^{6}\right) ^{\frac{%
{\large 1}}{{\large 3}}}\text{ \ } \\
&=&C^{\frac{{\large 1}}{{\large 3}}}<\infty \text{ \ }\left( \text{by
Assumption 3-3(b)}\right)
\end{eqnarray*}%
Moreover, let $\varrho =w-v$, so that $w=v+\varrho $. Using these notations
and the boundedness of $\left( E\left\vert u_{iv}\right\vert ^{6}\right) ^{%
\frac{{\large 1}}{{\large 6}}}\left( E\left\vert u_{iw}\right\vert ^{\frac{%
{\large 5}}{{\large 4}}}\right) ^{\frac{{\large 4}}{{\large 5}}}$ as shown
above, we can further write%
\begin{eqnarray}
&&\frac{1}{T_{h}}\dsum\limits_{\substack{ v,w=p  \\ v<w}}^{T-h}\left\vert E%
\left[ u_{iv}u_{iw}\right] \right\vert  \notag \\
&&\frac{1}{T_{h}}\dsum\limits_{\substack{ v,w=p  \\ v<w}}^{T-h}2\left( 2^{1-%
\frac{{\large 1}}{{\large 6}}}+1\right) \left[ a_{1}\exp \left\{
-a_{2}\left( w-v\right) \right\} \right] ^{1-\frac{{\large 1}}{{\large 6}}-%
\frac{{\large 4}}{{\large 5}}}\left( E\left\vert u_{iv}\right\vert
^{6}\right) ^{\frac{{\large 1}}{{\large 6}}}\left( E\left\vert
u_{iw}\right\vert ^{\frac{{\large 5}}{{\large 4}}}\right) ^{\frac{{\large 4}%
}{{\large 5}}}  \notag \\
&\leq &\frac{C^{\frac{{\large 1}}{{\large 3}}}}{T_{h}}\dsum\limits 
_{\substack{ v,w=p  \\ v<w}}^{T-h}2\left( 2^{\frac{{\large 5}}{{\large 6}}%
}+1\right) \left[ a_{1}\exp \left\{ -a_{2}\left( w-v\right) \right\} \right]
^{\frac{{\large 1}}{{\large 30}}}  \notag \\
&\leq &\frac{C^{\ast }}{T_{h}}\dsum\limits_{\substack{ v,w=p  \\ v<w}}%
^{T-h}\exp \left\{ -\frac{a_{2}}{30}\varrho \right\} \text{ }  \notag \\
&&\left( \text{for some constant }C^{\ast }\text{ such that }2\left( 2^{%
\frac{{\large 5}}{{\large 6}}}+1\right) C^{\frac{{\large 1}}{{\large 3}}%
}a_{1}^{\frac{{\large 1}}{{\large 30}}}\leq C^{\ast }<\infty \right)  \notag
\\
&\leq &\frac{C^{\ast }}{T_{h}}\dsum\limits_{\substack{ v=p}}%
^{T-h}\dsum\limits_{\varrho =1}^{\infty }\exp \left\{ -\frac{a_{2}}{30}%
\varrho \right\}  \notag \\
&=&C^{\ast }\dsum\limits_{\substack{ \varrho _{{\large 1}}=1}}^{\infty }\exp
\left\{ -\frac{a_{2}}{30}\varrho \right\} \text{ }  \notag \\
&=&O\left( 1\right) \text{ \ }\left( \text{given Lemma C-1}\right)
\label{T1term2 2}
\end{eqnarray}%
It follows from expressions (\ref{T1Euguh}), (\ref{T1term2 1}), and (\ref%
{T1term2 2}) that%
\begin{eqnarray*}
\frac{1}{T_{h}}\dsum\limits_{\substack{ v,w=p  \\ v\leq w}}^{T-h}\left\vert E%
\left[ u_{iv}u_{iw}\right] \right\vert &=&\frac{1}{T_{h}}\dsum%
\limits_{v=p}^{T-h}E\left[ u_{iv}^{2}\right] +\frac{1}{T_{h}}\dsum\limits 
_{\substack{ v,w=p  \\ v<w}}^{T-h}\left\vert E\left[ u_{ig}u_{ih}\right]
\right\vert \\
&=&O\left( 1\right) +O\left( 1\right) \\
&=&O\left( 1\right) \text{.}
\end{eqnarray*}

To show part (b), first write%
\begin{eqnarray}
\frac{1}{T_{h}}\dsum\limits_{\substack{ t,s,g=p  \\ t\leq s\leq g}}%
^{T-h}\left\vert E\left( u_{it}u_{is}u_{ig}\right) \right\vert &=&\frac{1}{%
T_{h}}\dsum\limits_{\substack{ t=p}}^{T-h}E\left\vert u_{it}\right\vert ^{3}+%
\frac{1}{T_{h}}\dsum\limits_{\substack{ t,s,g=p  \\ t\leq s\leq g  \\ %
s-t>g-s,s-t>0}}^{T-h}\left\vert E\left( u_{it}u_{is}u_{ig}\right) \right\vert
\notag \\
&&+\frac{1}{T_{h}}\dsum\limits_{\substack{ t,s,g=p  \\ t\leq s\leq g  \\ %
g-s\geq s-t,\text{ }g-s>0}}^{T-h}\left\vert E\left(
u_{it}u_{is}u_{ig}\right) \right\vert  \label{T1Euhuvuw}
\end{eqnarray}%
For the first term on the right-hand side of expression (\ref{T1Euhuvuw})
above, note that, trivially, we can apply Assumption 3-3(b) to obtain%
\begin{equation}
\frac{1}{T_{h}}\dsum\limits_{\substack{ t=p}}^{T-h}E\left\vert
u_{it}\right\vert ^{3}\leq C=O\left( 1\right) \text{.}  \label{T1term3 1}
\end{equation}

Next, note that, for the second term on the right-hand side of expression (%
\ref{T1Euhuvuw}) above, we can apply Lemma C-3 with $p=6$ and $r=5/4$ to
obtain%
\begin{eqnarray*}
&&\frac{1}{T_{h}}\dsum\limits_{\substack{ t,s,g=p  \\ t\leq s\leq g  \\ %
s-t>g-s,s-t>0}}^{T-h}\left\vert E\left( u_{it}u_{is}u_{ig}\right) \right\vert
\\
&\leq &\frac{1}{T_{h}}\dsum\limits_{\substack{ t,s,g=p  \\ t\leq s\leq g  \\ %
s-t>g-s,s-t>0}}^{T-h}2\left( 2^{1-\frac{{\large 1}}{{\large 6}}}+1\right) %
\left[ a_{1}\exp \left\{ -a_{2}\left( s-t\right) \right\} \right] ^{1-\frac{%
{\large 1}}{{\large 6}}-\frac{{\large 4}}{{\large 5}}}\left( E\left\vert
u_{it}\right\vert ^{6}\right) ^{\frac{{\large 1}}{{\large 6}}}\left(
E\left\vert u_{is}u_{ig}\right\vert ^{\frac{{\large 5}}{{\large 4}}}\right)
^{\frac{{\large 4}}{{\large 5}}}
\end{eqnarray*}%
Next, applying H\"{o}lder's inequality, we have%
\begin{eqnarray*}
\left( E\left\vert u_{it}\right\vert ^{6}\right) ^{\frac{{\large 1}}{{\large %
6}}}\left( E\left\vert u_{is}u_{ig}\right\vert ^{\frac{{\large 5}}{{\large 4}%
}}\right) ^{\frac{{\large 4}}{{\large 5}}} &\leq &\left( E\left\vert
u_{it}\right\vert ^{6}\right) ^{\frac{{\large 1}}{{\large 6}}}\left( \left(
E\left\vert u_{is}\right\vert ^{\frac{{\large 5}}{{\large 2}}}\right) ^{%
\frac{{\large 1}}{{\large 2}}}\left( E\left\vert u_{ig}\right\vert ^{\frac{%
{\large 5}}{{\large 2}}}\right) ^{\frac{{\large 1}}{{\large 2}}}\right) ^{%
\frac{{\large 4}}{{\large 5}}} \\
&=&\left( E\left\vert u_{it}\right\vert ^{6}\right) ^{\frac{{\large 1}}{%
{\large 6}}}\left( E\left\vert u_{is}\right\vert ^{\frac{{\large 5}}{{\large %
2}}}\right) ^{\frac{{\large 2}}{{\large 5}}}\left( E\left\vert
u_{ig}\right\vert ^{\frac{{\large 5}}{{\large 2}}}\right) ^{\frac{{\large 2}%
}{{\large 5}}} \\
&\leq &\left( E\left\vert u_{it}\right\vert ^{6}\right) ^{\frac{{\large 1}}{%
{\large 6}}}\left( E\left\vert u_{is}\right\vert ^{6}\right) ^{\frac{{\large %
1}}{{\large 6}}}\left( E\left\vert u_{ig}\right\vert ^{6}\right) ^{\frac{%
{\large 1}}{{\large 6}}} \\
&&\left( \text{by Liapunov's inequality}\right) \\
&=&C^{\frac{{\large 1}}{{\large 2}}}<\infty \text{ }\left( \text{by
Assumption 3-3(b)}\right)
\end{eqnarray*}%
Moreover, let $\varrho _{1}=s-t$ and $\varrho _{2}=g-s$, so that $%
s=t+\varrho _{1}$ and $g=s+$ $\varrho _{2}=t+\varrho _{1}+$ $\varrho _{2}$.
Using these notations and the boundedness of $\left( E\left\vert
u_{it}\right\vert ^{6}\right) ^{\frac{{\large 1}}{{\large 6}}}\left(
E\left\vert u_{is}u_{ig}\right\vert ^{\frac{{\large 5}}{{\large 4}}}\right)
^{\frac{{\large 4}}{{\large 5}}}$ as shown above, we can further write%
\begin{eqnarray}
&&\frac{1}{T_{h}}\dsum\limits_{\substack{ t,s,g=p  \\ t\leq s\leq g  \\ %
s-t>g-s,s-t>0}}^{T-h}\left\vert E\left( u_{it}u_{is}u_{ig}\right) \right\vert
\notag \\
&\leq &\frac{1}{T_{h}}\dsum\limits_{\substack{ t,s,g=p  \\ t\leq s\leq g  \\ %
s-t>g-s,s-t>0}}^{T-h}2\left( 2^{1-\frac{{\large 1}}{{\large 6}}}+1\right) %
\left[ a_{1}\exp \left\{ -a_{2}\left( s-t\right) \right\} \right] ^{1-\frac{%
{\large 1}}{{\large 6}}-\frac{{\large 4}}{{\large 5}}}\left( E\left\vert
u_{it}\right\vert ^{6}\right) ^{\frac{{\large 1}}{{\large 6}}}\left(
E\left\vert u_{is}u_{ig}\right\vert ^{\frac{{\large 5}}{{\large 4}}}\right)
^{\frac{{\large 4}}{{\large 5}}}  \notag \\
&\leq &\frac{C^{\frac{{\large 1}}{{\large 2}}}}{T_{h}}\dsum\limits 
_{\substack{ t,s,g=p  \\ t\leq s\leq g  \\ s-t>g-s,s-t>0}}^{T-h}2\left( 2^{%
\frac{{\large 5}}{{\large 6}}}+1\right) \left[ a_{1}\exp \left\{
-a_{2}\left( s-t\right) \right\} \right] ^{\frac{{\large 1}}{{\large 30}}} 
\notag \\
&\leq &\frac{C^{\ast }}{T_{h}}\dsum\limits_{\substack{ t,s,g=p  \\ t\leq
s\leq g  \\ s-t>g-s,s-t>0}}^{T-h}\exp \left\{ -\frac{a_{2}}{30}\varrho
_{1}\right\} \text{ }  \notag \\
&&\left( \text{for some constant }C^{\ast }\text{ such that }2\left( 2^{%
\frac{{\large 5}}{{\large 6}}}+1\right) C^{\frac{{\large 1}}{{\large 2}}%
}a_{1}^{\frac{{\large 1}}{{\large 30}}}\leq C^{\ast }<\infty \right)  \notag
\\
&\leq &\frac{C^{\ast }}{T_{h}}\dsum\limits_{\substack{ t=p}}%
^{T-h}\dsum\limits_{\varrho _{{\large 1}}=1}^{\infty }\dsum\limits_{\varrho
_{{\large 2}}=0}^{\varrho _{{\large 1}}-1}\exp \left\{ -\frac{a_{2}}{30}%
\varrho _{1}\right\}  \notag \\
&\leq &\frac{C^{\ast }}{T_{h}}\dsum\limits_{\substack{ t=p}}%
^{T-h}\dsum\limits_{\substack{ \varrho _{{\large 1}}=1}}^{\infty }\varrho
_{1}\exp \left\{ -\frac{a_{2}}{30}\varrho _{1}\right\}  \notag \\
&=&C^{\ast }\dsum\limits_{\substack{ \varrho _{{\large 1}}=1}}^{\infty
}\varrho _{1}\exp \left\{ -\frac{a_{2}}{30}\varrho _{1}^{{\large \backslash }%
}\right\} \text{ }  \notag \\
&=&O\left( 1\right) \text{ \ }\left( \text{given Lemma C-1}\right)
\label{T1term3 2}
\end{eqnarray}

Similarly, for the third term on the right-hand side of expression (\ref%
{T1Euhuvuw}), we can apply Lemma C-3 with $p=6$ and $r=5/4$, we have%
\begin{eqnarray*}
&&\frac{1}{T_{h}}\dsum\limits_{\substack{ t,s,g=p  \\ t\leq s\leq g  \\ %
g-s\geq s-t,\text{ }g-s>0}}^{T-h}\left\vert E\left(
u_{it}u_{is}u_{ig}\right) \right\vert \\
&\leq &\frac{1}{T_{h}}\dsum\limits_{\substack{ t,s,g=p  \\ t\leq s\leq g  \\ %
g-s\geq s-t,\text{ }g-s>0}}^{T-h}2\left( 2^{1-\frac{{\large 1}}{{\large 6}}%
}+1\right) \left[ a_{1}\exp \left\{ -a_{2}\left( g-s\right) \right\} \right]
^{1-\frac{{\large 4}}{{\large 5}}-\frac{{\large 1}}{{\large 6}}}\left(
E\left\vert u_{it}u_{is}\right\vert ^{\frac{{\large 5}}{{\large 4}}}\right)
^{\frac{{\large 4}}{{\large 5}}}\left( E\left\vert u_{ig}\right\vert
^{6}\right) ^{\frac{{\large 1}}{{\large 6}}}
\end{eqnarray*}%
Next, applying H\"{o}lder's inequality, we have%
\begin{eqnarray*}
\left( E\left\vert u_{it}u_{is}\right\vert ^{\frac{{\large 5}}{{\large 4}}%
}\right) ^{\frac{{\large 4}}{{\large 5}}}\left( E\left\vert
u_{ig}\right\vert ^{6}\right) ^{\frac{{\large 1}}{{\large 6}}} &\leq &\left(
\left( E\left\vert u_{it}\right\vert ^{\frac{{\large 5}}{{\large 2}}}\right)
^{\frac{{\large 1}}{{\large 2}}}\left( E\left\vert u_{is}\right\vert ^{\frac{%
{\large 5}}{{\large 2}}}\right) ^{\frac{{\large 1}}{{\large 2}}}\right) ^{%
\frac{{\large 4}}{{\large 5}}}\left( E\left\vert u_{ig}\right\vert
^{6}\right) ^{\frac{{\large 1}}{{\large 6}}} \\
&=&\left( E\left\vert u_{it}\right\vert ^{\frac{{\large 5}}{{\large 2}}%
}\right) ^{\frac{{\large 2}}{{\large 5}}}\left( E\left\vert
u_{is}\right\vert ^{\frac{{\large 5}}{{\large 2}}}\right) ^{\frac{{\large 2}%
}{{\large 5}}}\left( E\left\vert u_{ig}\right\vert ^{6}\right) ^{\frac{%
{\large 1}}{{\large 6}}} \\
&\leq &\left( E\left\vert u_{it}\right\vert ^{6}\right) ^{\frac{{\large 1}}{%
{\large 6}}}\left( E\left\vert u_{is}\right\vert ^{6}\right) ^{\frac{{\large %
1}}{{\large 6}}}\left( E\left\vert u_{ig}\right\vert ^{6}\right) ^{\frac{%
{\large 1}}{{\large 6}}} \\
&&\left( \text{by Liapunov's inequality}\right) \text{\ } \\
&=&C^{\frac{{\large 1}}{{\large 2}}}<\infty \text{ }\left( \text{by
Assumption 3-3(b)}\right)
\end{eqnarray*}%
Moreover, let $\varrho _{1}=s-t$ and $\varrho _{2}=g-s$, so that $%
s=t+\varrho _{1}$ and $g=s+$ $\varrho _{2}=t+\varrho _{1}+$ $\varrho _{2}$.
Using these notations and the boundedness of $\left( E\left\vert
u_{it}u_{is}\right\vert ^{\frac{{\large 5}}{{\large 4}}}\right) ^{\frac{%
{\large 4}}{{\large 5}}}\left( E\left\vert u_{ig}\right\vert ^{6}\right) ^{%
\frac{{\large 1}}{{\large 6}}}$ as shown above, we can further write%
\begin{eqnarray}
&&\frac{1}{T_{h}}\dsum\limits_{\substack{ t,s,g=p  \\ t\leq s\leq g  \\ %
g-s\geq s-t,\text{ }g-s>0}}^{T-h}\left\vert E\left(
u_{it}u_{is}u_{ig}\right) \right\vert  \notag \\
&\leq &\frac{1}{T_{h}}\dsum\limits_{\substack{ t,s,g=p  \\ t\leq s\leq g  \\ %
g-s\geq s-t,\text{ }g-s>0}}^{T-h}2\left( 2^{1-\frac{{\large 1}}{{\large 6}}%
}+1\right) \left[ a_{1}\exp \left\{ -a_{2}\left( g-s\right) \right\} \right]
^{1-\frac{{\large 4}}{{\large 5}}-\frac{{\large 1}}{{\large 6}}}\left(
E\left\vert u_{it}u_{is}\right\vert ^{\frac{{\large 5}}{{\large 4}}}\right)
^{\frac{{\large 4}}{{\large 5}}}\left( E\left\vert u_{ig}\right\vert
^{6}\right) ^{\frac{{\large 1}}{{\large 6}}}  \notag \\
&\leq &\frac{C^{\frac{{\large 1}}{{\large 2}}}}{T_{h}}\dsum\limits 
_{\substack{ t,s,g=p  \\ t\leq s\leq g  \\ g-s\geq s-t,\text{ }g-s>0}}%
^{T-h}2\left( 2^{\frac{{\large 5}}{{\large 6}}}+1\right) \left[ a_{1}\exp
\left\{ -a_{2}\left( g-s\right) \right\} \right] ^{\frac{{\large 1}}{{\large %
30}}}  \notag \\
&\leq &\frac{C^{\ast }}{T_{h}}\dsum\limits_{\substack{ t,s,g=p  \\ t\leq
s\leq g  \\ g-s\geq s-t,\text{ }g-s>0}}^{T-h}\exp \left\{ -\frac{a_{2}}{30}%
\varrho _{2}\right\} \text{ }  \notag \\
&&\left( \text{for some constant }C^{\ast }\text{ such that }2\left( 2^{%
\frac{{\large 5}}{{\large 6}}}+1\right) C^{\frac{{\large 1}}{{\large 2}}%
}a_{1}^{\frac{{\large 1}}{{\large 30}}}\leq C^{\ast }<\infty \right)  \notag
\\
&\leq &\frac{C^{\ast }}{T_{h}}\dsum\limits_{t=p}^{T-h}\dsum\limits 
_{\substack{ \varrho _{{\large 2}}=1}}^{\infty }\dsum\limits_{\varrho _{%
{\large 1}}=0}^{\varrho _{{\large 2}}}\exp \left\{ -\frac{a_{2}}{30}\varrho
_{2}\right\}  \notag \\
&=&\frac{C^{\ast }}{T_{h}}\dsum\limits_{t=p}^{T-h}\dsum\limits_{\substack{ %
\varrho _{{\large 2}}=1}}^{\infty }\left( \varrho _{2}+1\right) \exp \left\{
-\frac{a_{2}}{30}\varrho _{2}\right\}  \notag \\
&=&C^{\ast }\left[ \dsum\limits_{\substack{ \varrho _{{\large 2}}=1}}%
^{\infty }\varrho _{2}\exp \left\{ -\frac{a_{2}}{30}\varrho _{2}\right\}
+\dsum\limits_{\substack{ \varrho _{{\large 2}}=1}}^{\infty }\exp \left\{ -%
\frac{a_{2}}{30}\varrho _{2}\right\} \right]  \notag \\
&=&O\left( 1\right) \text{ \ }\left( \text{given Lemma C-1}\right)
\label{T1term3 3}
\end{eqnarray}

It follows from expressions (\ref{Euhuvuw}), (\ref{term3 1}), (\ref{term3 2}%
), and (\ref{term3 3}) that%
\begin{eqnarray*}
\frac{1}{T_{h}}\dsum\limits_{\substack{ t,s,g=p  \\ t\leq s\leq g}}%
^{T-h}\left\vert E\left( u_{it}u_{is}u_{ig}\right) \right\vert &=&\frac{1}{%
T_{h}}\dsum\limits_{\substack{ t=p}}^{T-h}E\left\vert u_{it}\right\vert ^{3}+%
\frac{1}{T_{h}}\dsum\limits_{\substack{ t,s,g=p  \\ t\leq s\leq g  \\ %
s-t>g-s,s-t>0}}^{T-h}\left\vert E\left( u_{it}u_{is}u_{ig}\right) \right\vert
\\
&&+\frac{1}{T_{h}}\dsum\limits_{\substack{ t,s,g=p  \\ t\leq s\leq g  \\ %
g-s\geq s-t,\text{ }g-s>0}}^{T-h}\left\vert E\left(
u_{it}u_{is}u_{ig}\right) \right\vert \\
&=&O\left( 1\right) +O\left( 1\right) +O\left( 1\right) \\
&=&O\left( 1\right) \text{. }
\end{eqnarray*}

Finally, to show part (c), we first write%
\begin{eqnarray}
&&\frac{1}{T_{h}^{2}}\dsum\limits_{\substack{ t,s,g,v=p  \\ t\leq s\leq
g\leq v}}^{T-h}\left\vert E\left( u_{it}u_{is}u_{ig}u_{iv}\right) \right\vert
\notag \\
&=&\frac{1}{T_{h}^{2}}\dsum\limits_{\substack{ t,s=p  \\ t\leq s}}%
^{T-h}\left\vert E\left[ u_{it}u_{is}^{3}\right] \right\vert +\frac{1}{%
T_{h}^{2}}\dsum\limits_{\substack{ t,s,g,v=p  \\ t\leq s\leq g\leq v  \\ %
v-g>g-s,\text{ }v-g>0}}^{T-h}\left\vert E\left(
u_{it}u_{is}u_{ig}u_{iv}\right) \right\vert +\frac{1}{T_{h}^{2}}\dsum\limits 
_{\substack{ t,s,g,v=p  \\ t\leq s\leq g\leq v  \\ v-g\leq g-s,\text{ }g-s>0 
}}^{T-h}\left\vert E\left( u_{it}u_{is}u_{ig}u_{iv}\right) \right\vert 
\notag \\
&=&\frac{1}{T_{h}^{2}}\dsum\limits_{\substack{ t,s=p  \\ t\leq s}}%
^{T-h}\left\vert E\left[ u_{it}u_{is}^{3}\right] \right\vert +\frac{1}{%
T_{h}^{2}}\dsum\limits_{\substack{ t,s,g,v=p  \\ t\leq s\leq g\leq v  \\ %
v-g>g-s,\text{ }v-g>0}}^{T-h}\left\vert E\left[ \left\{ u_{it}u_{is}-E\left(
u_{it}u_{is}\right) +E\left( u_{it}u_{is}\right) \right\} u_{ig}u_{iv}\right]
\right\vert  \notag \\
&&+\frac{1}{T_{h}^{2}}\dsum\limits_{\substack{ t,s,g,v=p  \\ t\leq s\leq
g\leq v  \\ v-g\leq g-s,\text{ }g-s>0}}^{T-h}\left\vert E\left[ \left\{
u_{it}u_{is}-E\left( u_{it}u_{is}\right) +E\left( u_{it}u_{is}\right)
\right\} u_{ig}u_{iv}\right] \right\vert  \notag \\
&\leq &\frac{1}{T_{h}^{2}}\dsum\limits_{\substack{ t,s=p  \\ t\leq s}}%
^{T-h}\left\vert E\left[ u_{it}u_{is}^{3}\right] \right\vert +\frac{1}{%
T_{h}^{2}}\dsum\limits_{\substack{ t,s,g,v=p  \\ t\leq s\leq g\leq v  \\ %
v-g>g-s,\text{ }v-g>0}}^{T-h}\left\vert E\left[ \left\{ u_{it}u_{is}-E\left(
u_{it}u_{is}\right) \right\} u_{ig}u_{iv}\right] \right\vert  \notag \\
&&+\frac{1}{T_{h}^{2}}\dsum\limits_{\substack{ t,s,g,v=p  \\ t\leq s\leq
g\leq v  \\ v-g\leq g-s,\text{ }g-s>0}}^{T-h}\left\vert E\left[ \left\{
u_{it}u_{is}-E\left( u_{it}u_{is}\right) \right\} u_{ig}u_{iv}\right]
\right\vert +\frac{1}{T_{h}^{2}}\dsum\limits_{\substack{ t,s,g,v=p  \\ t\leq
s\leq g\leq v  \\ v-s>0}}^{T-h}\left\vert E\left( u_{it}u_{is}\right)
\right\vert \left\vert E\left( u_{ig}u_{iv}\right) \right\vert
\label{T1Eutusuguv}
\end{eqnarray}%
For the first term on the right-hand side of expression (\ref{T1Eutusuguv})
above, note that, by Jensen's inequality, the Cauchy-Schwarz inequality, and
Assumption 3-3(b); we have%
\begin{eqnarray}
\frac{1}{T_{h}^{2}}\dsum\limits_{\substack{ t,s=p  \\ t\leq s}}%
^{T-h}\left\vert E\left[ u_{it}u_{is}^{3}\right] \right\vert &\leq &\frac{1}{%
T_{h}^{2}}\dsum\limits_{\substack{ t,s=p  \\ t\leq s}}^{T-h}E\left[
\left\vert u_{it}u_{is}^{3}\right\vert \right]  \notag \\
&\leq &\frac{1}{T_{h}^{2}}\dsum\limits_{\substack{ t,s=p  \\ t\leq s}}^{T-h}%
\sqrt{E\left\vert u_{it}\right\vert ^{2}}\sqrt{E\left\vert u_{is}\right\vert
^{6}}  \notag \\
&\leq &\frac{1}{T_{h}^{2}}\dsum\limits_{\substack{ t,s=p  \\ t\leq s}}%
^{T-h}\left( E\left\vert u_{it}\right\vert ^{6}\right) ^{\frac{{\large 1}}{%
{\large 6}}}\sqrt{E\left\vert u_{is}\right\vert ^{6}}  \notag \\
&&\left( \text{by Liapunov's inequality}\right)  \notag \\
&\leq &\text{ }\frac{C^{\frac{{\large 2}}{{\large 3}}}T_{h}^{2}}{T_{h}^{2}}%
\text{\ }\left( \text{by Assumption 3-3(b)}\right)  \notag \\
&=&O\left( 1\right)  \label{T1term4 1}
\end{eqnarray}

Next, for the second term on the right-hand side of expression (\ref%
{T1Eutusuguv}), we can apply Lemma C-3 with $p=4/3$ and $r=6$ to obtain%
\begin{eqnarray*}
&&\frac{1}{T_{h}^{2}}\dsum\limits_{\substack{ t,s,g,v=p  \\ t\leq s\leq
g\leq v  \\ v-g>g-s,\text{ }v-g>0}}^{T-h}\left\vert E\left[ \left\{
u_{it}u_{is}-E\left( u_{it}u_{is}\right) \right\} u_{ig}u_{iv}\right]
\right\vert \\
&\leq &\frac{1}{T_{h}^{2}}\dsum\limits_{\substack{ t,s,g,v=p  \\ t\leq s\leq
g\leq v  \\ v-g>g-s,\text{ }v-g>0}}^{T-h}\left\{ 2\left( 2^{1-\frac{{\large 3%
}}{{\large 4}}}+1\right) \left[ a_{1}\exp \left\{ -a_{2}\left( v-g\right)
\right\} \right] ^{1-\frac{{\large 3}}{{\large 4}}-\frac{{\large 1}}{{\large %
6}}}\right. \\
&&\text{ \ \ \ \ \ \ \ \ \ \ \ \ \ \ \ \ \ \ \ \ \ }\left. \times \left(
E\left\vert \left\{ u_{it}u_{is}-E\left( u_{it}u_{is}\right) \right\}
u_{ig}\right\vert ^{\frac{{\large 4}}{{\large 3}}}\right) ^{\frac{{\large 3}%
}{{\large 4}}}\left( E\left\vert u_{iv}\right\vert ^{6}\right) ^{\frac{%
{\large 1}}{{\large 6}}}\right\}
\end{eqnarray*}%
Next, by repeated application of H\"{o}lder's inequality, 
\begin{eqnarray*}
E\left\vert \left\{ u_{it}u_{is}-E\left( u_{it}u_{is}\right) \right\}
u_{ig}\right\vert ^{\frac{{\large 4}}{{\large 3}}} &\leq &\left[ E\left(
u_{it}u_{is}-E\left( u_{it}u_{is}\right) \right) ^{\frac{{\large 12}}{%
{\large 7}}}\right] ^{\frac{{\large 7}}{{\large 9}}}\left[ E\left\vert
u_{ig}\right\vert ^{6}\right] ^{\frac{{\large 2}}{{\large 9}}} \\
&\leq &\left[ 2^{\frac{{\large 5}}{{\large 7}}}\left( E\left\vert
u_{it}u_{is}\right\vert ^{\frac{{\large 12}}{{\large 7}}}+\left\vert E\left[
u_{it}u_{is}\right] \right\vert ^{\frac{{\large 12}}{{\large 7}}}\right) %
\right] ^{\frac{{\large 7}}{{\large 9}}}\left[ E\left\vert u_{ig}\right\vert
^{6}\right] ^{\frac{{\large 2}}{{\large 9}}} \\
&&\left( \text{by Lo\`{e}ve's }c_{r}\text{ inequality}\right) \\
&\leq &\left[ 2^{\frac{{\large 5}}{{\large 7}}}\left( E\left\vert
u_{it}u_{is}\right\vert ^{\frac{{\large 12}}{{\large 7}}}+E\left\vert
u_{it}u_{is}\right\vert ^{\frac{{\large 12}}{{\large 7}}}\right) \right] ^{%
\frac{{\large 7}}{{\large 9}}}\left[ E\left\vert u_{ig}\right\vert ^{6}%
\right] ^{\frac{{\large 2}}{{\large 9}}} \\
&&\left( \text{by Jensen's inequality}\right) \\
&=&\left[ 2^{\frac{{\large 12}}{{\large 7}}}E\left\vert
u_{it}u_{is}\right\vert ^{\frac{{\large 12}}{{\large 7}}}\right] ^{\frac{%
{\large 7}}{{\large 9}}}\left[ E\left\vert u_{ig}\right\vert ^{6}\right] ^{%
\frac{{\large 2}}{{\large 9}}} \\
&\leq &2^{\frac{{\large 4}}{{\large 3}}}\left[ \left( E\left\vert
u_{it}\right\vert ^{\frac{{\large 24}}{{\large 7}}}\right) ^{\frac{{\large 1}%
}{{\large 2}}}\left( E\left\vert u_{is}\right\vert ^{\frac{{\large 24}}{%
{\large 7}}}\right) ^{\frac{{\large 1}}{{\large 2}}}\right] ^{\frac{{\large 7%
}}{{\large 9}}}\left[ E\left\vert u_{ig}\right\vert ^{6}\right] ^{\frac{%
{\large 2}}{{\large 9}}} \\
&=&2^{\frac{{\large 4}}{{\large 3}}}\left[ \left( E\left\vert
u_{it}\right\vert ^{\frac{{\large 24}}{{\large 7}}}\right) ^{\frac{{\large 7}%
}{{\large 24}}}\left( E\left\vert u_{is}\right\vert ^{\frac{{\large 24}}{%
{\large 7}}}\right) ^{\frac{{\large 7}}{{\large 24}}}\right] ^{\frac{{\large %
4}}{{\large 3}}}\left[ E\left\vert u_{ig}\right\vert ^{6}\right] ^{\frac{%
{\large 2}}{{\large 9}}} \\
&\leq &2^{\frac{{\large 4}}{{\large 3}}}\left[ \left( E\left\vert
u_{it}\right\vert ^{6}\right) ^{\frac{{\large 1}}{{\large 6}}}\left(
E\left\vert u_{is}\right\vert ^{6}\right) ^{\frac{{\large 1}}{{\large 6}}}%
\right] ^{\frac{{\large 4}}{{\large 3}}}\left[ E\left\vert u_{ig}\right\vert
^{6}\right] ^{\frac{{\large 2}}{{\large 9}}} \\
&\leq &2^{\frac{{\large 4}}{{\large 3}}}\left( C\right) ^{\frac{{\large 2}}{%
{\large 9}}}\left( C\right) ^{\frac{{\large 2}}{{\large 9}}}\left( C\right)
^{\frac{{\large 2}}{{\large 9}}}\text{ \ }\left( \text{by Assumption 3-3(b) }%
\right) \\
&=&2^{\frac{{\large 4}}{{\large 3}}}C^{\frac{{\large 2}}{{\large 3}}}
\end{eqnarray*}%
Moreover, let $\varrho _{1}=g-s$ and $\varrho _{2}=v-g$ so that $g=s+$ $%
\varrho _{1}$ and $v=g+\varrho _{2}=s+\varrho _{1}+\varrho _{2}$. Using
these notations and the boundedness of $E\left\vert \left\{
u_{it}u_{is}-E\left( u_{it}u_{is}\right) \right\} u_{ig}\right\vert ^{\frac{%
{\large 4}}{{\large 3}}}$ as shown above, we can further write%
\begin{eqnarray}
&&\frac{1}{T_{h}^{2}}\dsum\limits_{\substack{ t,s,g,v=p  \\ t\leq s\leq
g\leq v  \\ v-g>g-s,\text{ }v-g>0}}^{T-h}\left\vert E\left[ \left\{
u_{it}u_{is}-E\left( u_{it}u_{is}\right) \right\} u_{ig}u_{iv}\right]
\right\vert  \notag \\
&\leq &\frac{1}{T_{h}^{2}}\dsum\limits_{\substack{ t,s,g,v=p  \\ t\leq s\leq
g\leq v  \\ v-g>g-s,\text{ }v-g>0}}^{T-h}\left\{ 2\left( 2^{1-\frac{{\large 3%
}}{{\large 4}}}+1\right) \left[ a_{1}\exp \left\{ -a_{2}\left( v-g\right)
\right\} \right] ^{1-\frac{{\large 3}}{{\large 4}}-\frac{{\large 1}}{{\large %
6}}}\right.  \notag \\
&&\text{ \ \ \ \ \ \ \ \ \ \ \ \ \ \ \ \ \ \ \ \ \ }\left. \times \left(
E\left\vert \left\{ u_{it}u_{is}-E\left( u_{it}u_{is}\right) \right\}
u_{ig}\right\vert ^{\frac{{\large 4}}{{\large 3}}}\right) ^{\frac{{\large 3}%
}{{\large 4}}}\left( E\left\vert u_{iv}\right\vert ^{6}\right) ^{\frac{%
{\large 1}}{{\large 6}}}\right\}  \notag \\
&\leq &\frac{1}{T_{h}^{2}}\dsum\limits_{\substack{ t,s,g,v=p  \\ t\leq s\leq
g\leq v  \\ v-g>g-s,\text{ }v-g>0}}^{T-h}2\left( 2^{\frac{{\large 1}}{%
{\large 4}}}+1\right) \left[ a_{1}\exp \left\{ -a_{2}\left( v-g\right)
\right\} \right] ^{\frac{{\large 1}}{{\large 12}}}\left( 2^{\frac{{\large 4}%
}{{\large 3}}}C^{\frac{{\large 2}}{{\large 3}}}\right) ^{\frac{{\large 3}}{%
{\large 4}}}\left( C\right) ^{\frac{{\large 1}}{{\large 6}}}  \notag \\
&\leq &\text{ }\frac{C^{\ast }}{T_{h}^{2}}\dsum\limits_{\substack{ t,s,g,v=p 
\\ t\leq s\leq g\leq v  \\ v-g>g-s,\text{ }v-g>0}}^{T-h}\exp \left\{ -\frac{%
a_{2}}{12}\varrho _{2}\right\}  \notag \\
&&\left( \text{for some constant }C^{\ast }\text{ such that }4\left( 2^{%
\frac{{\large 1}}{{\large 4}}}+1\right) C^{\frac{{\large 2}}{{\large 3}}%
}a_{1}^{\frac{{\large 1}}{{\large 12}}}\leq C^{\ast }<\infty \right)  \notag
\\
&\leq &\frac{C^{\ast }}{T_{h}^{2}}\dsum\limits_{\substack{ t=p}}%
^{T-h}\dsum\limits_{\substack{ s=p}}^{T-h}\dsum\limits_{\varrho _{{\large 2}%
}=1}^{\infty }\dsum\limits_{\varrho _{{\large 1}}=0}^{\varrho _{{\large 2}%
}-1}\exp \left\{ -\frac{a_{2}}{12}\varrho _{2}\right\}  \notag \\
&=&C^{\ast }\dsum\limits_{\substack{ \rho _{{\large 2}}=1}}^{\infty }\varrho
_{2}\exp \left\{ -\frac{a_{2}}{12}\varrho _{2}\right\}  \notag \\
&=&O\left( 1\right) \text{ \ }\left( \text{given Lemma C-1}\right)
\label{T1term4 2}
\end{eqnarray}

Similarly, for the third term on the right-hand side of expression (\ref%
{T1Eutusuguv}) above, we can apply Lemma C-3 with $p=2$ and $r=3$ to obtain%
\begin{eqnarray*}
&&\frac{1}{T_{h}^{2}}\dsum\limits_{\substack{ t,s,g,v=p  \\ t\leq s\leq
g\leq v  \\ v-g\leq g-s,\text{ }g-s>0}}^{T-h}\left\vert E\left[ \left\{
u_{it}u_{is}-E\left( u_{it}u_{is}\right) \right\} u_{ig}u_{iv}\right]
\right\vert \\
&\leq &\frac{1}{T_{h}^{2}}\dsum\limits_{\substack{ t,s,g,v=p  \\ t\leq s\leq
g\leq v  \\ v-g\leq g-s,\text{ }g-s>0}}^{T-h}\left\{ 2\left( 2^{1-\frac{%
{\large 1}}{{\large 2}}}+1\right) \left[ a_{1}\exp \left\{ -a_{2}\left(
g-s\right) \right\} \right] ^{1-\frac{{\large 1}}{{\large 2}}-\frac{{\large 1%
}}{{\large 3}}}\right. \\
&&\text{ \ \ \ \ \ \ \ \ \ \ \ \ \ \ \ \ \ \ \ \ \ \ \ }\left. \times \left(
E\left\vert \left\{ u_{it}u_{is}-E\left( u_{it}u_{is}\right) \right\}
\right\vert ^{2}\right) ^{\frac{{\large 1}}{{\large 2}}}\left( E\left\vert
u_{ig}u_{iv}\right\vert ^{3}\right) ^{\frac{{\large 1}}{{\large 3}}}\right\}
\end{eqnarray*}%
Next, applications of H\"{o}lder's inequality yield%
\begin{eqnarray*}
E\left\vert u_{ig}u_{iv}\right\vert ^{3} &\leq &\left( E\left\vert
u_{ig}\right\vert ^{6}\right) ^{\frac{{\large 1}}{{\large 2}}}\left(
E\left\vert u_{iv}\right\vert ^{6}\right) ^{\frac{{\large 1}}{{\large 2}}} \\
&\leq &\left( C\right) ^{\frac{{\large 1}}{{\large 2}}}\left( C\right) ^{%
\frac{{\large 1}}{{\large 2}}}\text{ \ }\left( \text{by Assumption 3-3(b)}%
\right) \\
&=&C<\infty
\end{eqnarray*}%
and%
\begin{eqnarray*}
E\left\vert \left\{ u_{it}u_{is}-E\left( u_{it}u_{is}\right) \right\}
\right\vert ^{2} &\leq &2\left( E\left\vert u_{it}u_{is}\right\vert
^{2}+\left\vert E\left[ u_{it}u_{is}\right] \right\vert ^{2}\right) \text{ }%
\left( \text{by Lo\`{e}ve's }c_{r}\text{ inequality}\right) \\
&\leq &2\left( E\left\vert u_{it}u_{is}\right\vert ^{2}+E\left\vert
u_{it}u_{is}\right\vert ^{2}\right) \text{ }\left( \text{by Jensen's
inequality}\right) \\
&=&4E\left\vert u_{it}u_{is}\right\vert ^{2} \\
&\leq &4\left[ \left( E\left\vert u_{it}\right\vert ^{4}\right) ^{\frac{%
{\large 1}}{{\large 4}}}\left( E\left\vert u_{is}\right\vert ^{4}\right) ^{%
\frac{{\large 1}}{{\large 4}}}\right] ^{2} \\
&\leq &4\left[ \left( E\left\vert u_{it}\right\vert ^{6}\right) ^{\frac{%
{\large 1}}{{\large 6}}}\left( E\left\vert u_{is}\right\vert ^{6}\right) ^{%
\frac{{\large 1}}{{\large 6}}}\right] ^{2}\text{ \ }\left( \text{by
Liapunov's inequality}\right) \\
&\leq &4\left( \sup_{t}E\left\vert u_{it}\right\vert ^{6}\right) ^{\frac{%
{\large 2}}{{\large 3}}} \\
&\leq &4\left( C\right) ^{\frac{{\large 2}}{{\large 3}}}<\infty \text{ \ }%
\left( \text{by Assumption 3-3(b) }\right) \text{ }
\end{eqnarray*}%
Moreover, let $\varrho _{1}=g-s$ and $\varrho _{2}=v-g$ so that $g=s+$ $%
\varrho _{1}$ and $v=g+\varrho _{2}=s+\varrho _{1}+\varrho _{2}$. Using
these notations and the boundedness of $E\left\vert u_{ig}u_{iv}\right\vert
^{3}$ and $E\left\vert \left\{ u_{it}u_{is}-E\left( u_{it}u_{is}\right)
\right\} \right\vert ^{2}$ as shown above, we can further write%
\begin{eqnarray}
&&\frac{1}{T_{h}^{2}}\dsum\limits_{\substack{ t,s,g,v=p  \\ t\leq s\leq
g\leq v  \\ v-g\leq g-s,\text{ }g-s>0}}^{T-h}\left\vert E\left[ \left\{
u_{it}u_{is}-E\left( u_{it}u_{is}\right) \right\} u_{ig}u_{iv}\right]
\right\vert  \notag \\
&\leq &\frac{1}{T_{h}^{2}}\dsum\limits_{\substack{ t,s,g,v=p  \\ t\leq s\leq
g\leq v  \\ v-g\leq g-s,\text{ }g-s>0}}^{T-h}\left\{ 2\left( 2^{1-\frac{%
{\large 1}}{{\large 2}}}+1\right) \left[ a_{1}\exp \left\{ -a_{2}\left(
g-s\right) \right\} \right] ^{1-\frac{{\large 1}}{{\large 2}}-\frac{{\large 1%
}}{{\large 3}}}\right.  \notag \\
&&\text{ \ \ \ \ \ \ \ \ \ \ \ \ \ \ \ \ \ \ \ \ \ \ \ }\left. \times \left(
E\left\vert \left\{ u_{it}u_{is}-E\left( u_{it}u_{is}\right) \right\}
\right\vert ^{2}\right) ^{\frac{{\large 1}}{{\large 2}}}\left( E\left\vert
u_{ig}u_{iv}\right\vert ^{3}\right) ^{\frac{{\large 1}}{{\large 3}}}\right\}
\notag \\
&\leq &\frac{1}{T_{h}^{2}}\dsum\limits_{\substack{ t,s,g,v=p  \\ t\leq s\leq
g\leq v  \\ v-g\leq g-s,\text{ }g-s>0}}^{T-h}2\left( 2^{\frac{{\large 1}}{%
{\large 2}}}+1\right) \left[ a_{1}\exp \left\{ -a_{2}\left( g-s\right)
\right\} \right] ^{\frac{{\large 1}}{{\large 6}}}\left( 4C^{\frac{{\large 2}%
}{{\large 3}}}\right) ^{\frac{{\large 1}}{{\large 2}}}\left( C\right) ^{%
\frac{{\large 1}}{{\large 3}}}\text{ }  \notag \\
&\leq &\frac{C^{\ast }}{T_{h}^{2}}\dsum\limits_{\substack{ t,s,g,v=p  \\ %
t\leq s\leq g\leq v  \\ v-g\leq g-s,\text{ }g-s>0}}^{T-h}\exp \left\{ -\frac{%
a_{2}}{6}\varrho _{1}\right\}  \notag \\
&&\left( \text{for some constant }C^{\ast }\text{ such that }4\left( 2^{%
\frac{{\large 1}}{{\large 2}}}+1\right) C^{\frac{{\large 2}}{{\large 3}}%
}a_{1}^{\frac{{\large 1}}{{\large 6}}}\leq C^{\ast }<\infty \right)  \notag
\\
&\leq &\frac{C^{\ast }}{T_{h}^{2}}\dsum\limits_{\substack{ t=p}}%
^{T-h}\dsum\limits_{\substack{ s=p}}^{T-h}\dsum\limits_{\varrho _{{\large 1}%
}=1}^{\infty }\dsum\limits_{\varrho _{{\large 2}}=0}^{\varrho _{{\large 1}%
}}\exp \left\{ -\frac{a_{2}}{6}\varrho _{1}\right\}  \notag \\
&=&C^{\ast }\dsum\limits_{\substack{ \rho _{{\large 1}}=1}}^{\infty }\left(
\varrho _{{\large 1}}+1\right) \exp \left\{ -\frac{a_{2}}{6}\varrho
_{1}\right\}  \notag \\
&=&O\left( 1\right) \text{ \ }\left( \text{given Lemma C-1}\right)
\label{T1term4 3}
\end{eqnarray}

Finally, consider the fourth term on the right-hand side of expression (\ref%
{T1Eutusuguv}) above. For this term, we apply the result given in part (a)
to obtain 
\begin{eqnarray}
\frac{1}{T_{h}^{2}}\dsum\limits_{\substack{ t,s,g,v=p  \\ t\leq s\leq g\leq
v  \\ v-s>0}}^{T-h}\left\vert E\left( u_{it}u_{is}\right) \right\vert
\left\vert E\left( u_{ig}u_{iv}\right) \right\vert &\leq &\left( \frac{1}{%
T_{h}}\dsum\limits_{\substack{ t,s=p  \\ t\leq s}}^{T-h}\left\vert E\left(
u_{it}u_{is}\right) \right\vert \right) \left( \frac{1}{T_{h}}\dsum\limits 
_{\substack{ g,v=p  \\ g\leq v}}^{T-h}\left\vert E\left( u_{ig}u_{iv}\right)
\right\vert \right)  \notag \\
&=&O\left( 1\right) \text{.}  \label{T1term4 4}
\end{eqnarray}

\noindent It follows from expressions (\ref{T1Eutusuguv})-(\ref{T1term4 4})
that%
\begin{eqnarray*}
&&\frac{1}{T_{h}^{2}}\dsum\limits_{\substack{ t,s,g,v=p  \\ t\leq s\leq
g\leq v}}^{T-h}\left\vert E\left( u_{it}u_{is}u_{ig}u_{iv}\right) \right\vert
\\
&\leq &\frac{1}{T_{h}^{2}}\dsum\limits_{\substack{ t,s=p  \\ t\leq s}}%
^{T-h}\left\vert E\left[ u_{it}u_{is}^{3}\right] \right\vert +\frac{1}{%
T_{h}^{2}}\dsum\limits_{\substack{ t,s,g,v=p  \\ t\leq s\leq g\leq v  \\ %
v-g>g-s,\text{ }v-g>0}}^{T-h}\left\vert E\left[ \left\{ u_{it}u_{is}-E\left(
u_{it}u_{is}\right) \right\} u_{ig}u_{iv}\right] \right\vert \\
&&+\frac{1}{T_{h}^{2}}\dsum\limits_{\substack{ t,s,g,v=p  \\ t\leq s\leq
g\leq v  \\ v-g\leq g-s,\text{ }g-s>0}}^{T-h}\left\vert E\left[ \left\{
u_{it}u_{is}-E\left( u_{it}u_{is}\right) \right\} u_{ig}u_{iv}\right]
\right\vert +\frac{1}{T_{h}^{2}}\dsum\limits_{\substack{ t,s,g,v=p  \\ t\leq
s\leq g\leq v  \\ v-s>0}}^{T-h}\left\vert E\left( u_{it}u_{is}\right)
\right\vert \left\vert E\left( u_{ig}u_{iv}\right) \right\vert \\
&=&O\left( 1\right) \text{. }\square
\end{eqnarray*}

\medskip

\medskip

\noindent \textbf{Lemma D-6: }Let $T_{h}=T-h-p+1$ where $h$ is a (fixed)
non-negative integer and $p$ is a (fixed) positive integer. Suppose that
Assumptions 3-1, 3-2(a)-(b), 3-5, and 3-7 hold. Then, as $%
N_{1},N_{2},T\rightarrow \infty $, 
\begin{equation*}
\max_{i\in H}\frac{1}{N_{1}}\dsum\limits_{k\in H^{{\large c}}}\left( \frac{%
\gamma _{k}^{\prime }\underline{F}^{\prime }u_{i\cdot }}{\sqrt{N_{1}}T_{h}}%
\right) ^{2}=O_{p}\left( \frac{N_{2}^{\frac{{\large 1}}{{\large 3}}}}{N_{1}T}%
\right) \text{.}
\end{equation*}

\medskip

\noindent \textbf{Proof of Lemma D-6:}

To proceed, we first show the boundedness of the quantity%
\begin{equation*}
\frac{1}{N_{1}}\dsum\limits_{k\in H^{{\large c}}}\dsum\limits_{i\in H}\frac{1%
}{N_{2}T_{h}^{3}}E\left( \dsum\limits_{t=p}^{T-h}\gamma _{k}^{\prime }%
\underline{F}_{t}u_{i,t}\right) ^{6}
\end{equation*}%
Note first that there exist a constant $C_{1}>1$ such that%
\begin{eqnarray*}
&&\frac{1}{N_{1}}\dsum\limits_{k\in H^{{\large c}}}\dsum\limits_{i\in H}%
\frac{1}{N_{2}T_{h}^{3}}E\left( \dsum\limits_{t=p}^{T-h}\gamma _{k}^{\prime }%
\underline{F}_{t}u_{i,t}\right) ^{6} \\
&\leq &\frac{C_{1}}{N_{1}N_{2}T_{h}^{3}}\dsum\limits_{k\in H^{{\large c}%
}}\dsum\limits_{i\in H}\dsum\limits_{\substack{ t,s,g,\ell ,v,w=p  \\ t\leq
s\leq g\leq \ell \leq v\leq w}}^{T-h}\left\{ \left\vert E\left[
u_{it}u_{is}u_{ig}u_{i\ell }u_{iv}u_{iw}\right] \right\vert \right. \\
&&\text{ \ \ \ \ \ \ \ \ \ \ \ \ \ \ \ \ \ \ \ \ \ \ \ \ \ \ \ \ \ \ }\left.
\times \left\vert E\left[ \left( \gamma _{k}^{\prime }\underline{F}%
_{t}\right) \left( \gamma _{k}^{\prime }\underline{F}_{s}\right) \left(
\gamma _{k}^{\prime }\underline{F}_{g}\right) \left( \gamma _{k}^{\prime }%
\underline{F}_{\ell }\right) \left( \gamma _{k}^{\prime }\underline{F}%
_{v}\right) \left( \gamma _{k}^{\prime }\underline{F}_{w}\right) \right]
\right\vert \right\}
\end{eqnarray*}%
Next, note that, by repeated application of H\"{o}lder's inequality, we have
by Assumption 3-5 and Lemma C-5 that there exists a positive constant $C$
such that%
\begin{eqnarray*}
&&\left\vert E\left[ \left( \gamma _{k}^{\prime }\underline{F}_{t}\right)
\left( \gamma _{k}^{\prime }\underline{F}_{s}\right) \left( \gamma
_{k}^{\prime }\underline{F}_{g}\right) \left( \gamma _{k}^{\prime }%
\underline{F}_{\ell }\right) \left( \gamma _{k}^{\prime }\underline{F}%
_{v}\right) \left( \gamma _{k}^{\prime }\underline{F}_{w}\right) \right]
\right\vert \\
&\leq &E\left[ \left\vert \gamma _{k}^{\prime }\underline{F}_{t}\right\vert
\left\vert \gamma _{k}^{\prime }\underline{F}_{s}\right\vert \left\vert
\gamma _{k}^{\prime }\underline{F}_{g}\right\vert \left\vert \gamma
_{k}^{\prime }\underline{F}_{\ell }\right\vert \left\vert \gamma
_{k}^{\prime }\underline{F}_{v}\right\vert \left\vert \gamma _{k}^{\prime }%
\underline{F}_{w}\right\vert \right] \\
&\leq &\left\Vert \gamma _{k}\right\Vert _{2}^{6}E\left[ \left\Vert 
\underline{F}_{t}\right\Vert _{2}\left\Vert \underline{F}_{s}\right\Vert
_{2}\left\Vert \underline{F}_{g}\right\Vert _{2}\left\Vert \underline{F}%
_{\ell }\right\Vert _{2}\left\Vert \underline{F}_{v}\right\Vert
_{2}\left\Vert \underline{F}_{w}\right\Vert _{2}\right] \\
&\leq &\left\Vert \gamma _{k}\right\Vert _{2}^{6}\left( E\left[ \left\Vert 
\underline{F}_{t}\right\Vert _{2}^{2}\left\Vert \underline{F}_{s}\right\Vert
_{2}^{2}\left\Vert \underline{F}_{g}\right\Vert _{2}^{2}\right] \right) ^{%
\frac{{\large 1}}{{\large 2}}}\left( E\left[ \left\Vert \underline{F}_{\ell
}\right\Vert _{2}^{2}\left\Vert \underline{F}_{v}\right\Vert
_{2}^{2}\left\Vert \underline{F}_{w}\right\Vert _{2}^{2}\right] \right) ^{%
\frac{{\large 1}}{{\large 2}}} \\
&\leq &\left\Vert \gamma _{k}\right\Vert _{2}^{6}\left( \left\{ E\left[
\left\Vert \underline{F}_{t}\right\Vert _{2}^{6}\right] \right\} ^{\frac{%
{\large 1}}{{\large 3}}}\left( E\left[ \left\Vert \underline{F}%
_{s}\right\Vert _{2}^{3}\left\Vert \underline{F}_{g}\right\Vert _{2}^{3}%
\right] \right) ^{\frac{{\large 2}}{{\large 3}}}\right) ^{\frac{{\large 1}}{%
{\large 2}}} \\
&&\times \left( \left\{ E\left[ \left\Vert \underline{F}_{\ell }\right\Vert
_{2}^{6}\right] \right\} ^{\frac{{\large 1}}{{\large 3}}}\left( E\left[
\left\Vert \underline{F}_{v}\right\Vert _{2}^{3}\left\Vert \underline{F}%
_{w}\right\Vert _{2}^{3}\right] \right) ^{\frac{{\large 2}}{{\large 3}}%
}\right) ^{\frac{{\large 1}}{{\large 2}}} \\
&\leq &\left\Vert \gamma _{k}\right\Vert _{2}^{6}\left( \left\{ E\left[
\left\Vert \underline{F}_{t}\right\Vert _{2}^{6}\right] \right\} ^{\frac{%
{\large 1}}{{\large 3}}}\left\{ E\left[ \left\Vert \underline{F}%
_{s}\right\Vert _{2}^{6}\right] \right\} ^{\frac{{\large 1}}{{\large 3}}%
}\left\{ E\left[ \left\Vert \underline{F}_{g}\right\Vert _{2}^{6}\right]
\right\} ^{\frac{{\large 1}}{{\large 3}}}\right) ^{\frac{{\large 1}}{{\large %
2}}} \\
&&\times \left( \left\{ E\left[ \left\Vert \underline{F}_{\ell }\right\Vert
_{2}^{6}\right] \right\} ^{\frac{{\large 1}}{{\large 3}}}\left\{ E\left[
\left\Vert \underline{F}_{v}\right\Vert _{2}^{6}\right] \right\} ^{\frac{%
{\large 1}}{{\large 3}}}\left\{ E\left[ \left\Vert \underline{F}%
_{w}\right\Vert _{2}^{6}\right] \right\} ^{\frac{{\large 1}}{{\large 3}}%
}\right) ^{\frac{{\large 1}}{{\large 2}}} \\
&\leq &\left\Vert \gamma _{k}\right\Vert _{2}^{6}\left\{ E\left[ \left\Vert 
\underline{F}_{t}\right\Vert _{2}^{6}\right] \right\} ^{\frac{{\large 1}}{%
{\large 6}}}\left\{ E\left[ \left\Vert \underline{F}_{s}\right\Vert _{2}^{6}%
\right] \right\} ^{\frac{{\large 1}}{{\large 6}}}\left\{ E\left[ \left\Vert 
\underline{F}_{g}\right\Vert _{2}^{6}\right] \right\} ^{\frac{{\large 1}}{%
{\large 6}}} \\
&&\times \left\{ E\left[ \left\Vert \underline{F}_{\ell }\right\Vert _{2}^{6}%
\right] \right\} ^{\frac{{\large 1}}{{\large 6}}}\left\{ E\left[ \left\Vert 
\underline{F}_{v}\right\Vert _{2}^{6}\right] \right\} ^{\frac{{\large 1}}{%
{\large 6}}}\left\{ E\left[ \left\Vert \underline{F}_{w}\right\Vert _{2}^{6}%
\right] \right\} ^{\frac{{\large 1}}{{\large 6}}} \\
&\leq &\left\Vert \gamma _{k}\right\Vert _{2}^{6}\sup_{t}E\left[ \left\Vert 
\underline{F}_{t}\right\Vert _{2}^{6}\right] \\
&\leq &C<\infty
\end{eqnarray*}%
Hence, we can write%
\begin{eqnarray*}
&&\frac{1}{N_{1}}\dsum\limits_{k\in H^{{\large c}}}\dsum\limits_{i\in H}%
\frac{1}{N_{2}T_{h}^{3}}E\left( \dsum\limits_{t=p}^{T-h}\gamma _{k}^{\prime }%
\underline{F}_{t}u_{i,t}\right) ^{6} \\
&\leq &\frac{C_{1}C}{N_{1}N_{2}T_{h}^{3}}\dsum\limits_{k\in H^{{\large c}%
}}\dsum\limits_{i\in H}\dsum\limits_{\substack{ t,s,g,\ell ,v,w=p  \\ t\leq
s\leq g\leq \ell \leq v\leq w}}^{T-h}\left\vert E\left[
u_{it}u_{is}u_{ig}u_{i\ell }u_{iv}u_{iw}\right] \right\vert \\
&\leq &\frac{C_{1}C}{N_{1}N_{2}T_{h}^{3}}\dsum\limits_{k\in H^{{\large c}%
}}\dsum\limits_{i\in H}\dsum\limits_{\substack{ t,s,g=p  \\ t\leq s\leq g}}%
^{T-h}\left\vert E\left[ u_{it}u_{is}u_{ig}^{4}\right] \right\vert \\
&&+\frac{C_{1}C}{N_{1}N_{2}T_{h}^{3}}\dsum\limits_{k\in H^{{\large c}%
}}\dsum\limits_{i\in H}\dsum\limits_{\substack{ t,s,g,\ell ,v,w=p  \\ t\leq
s\leq g\leq \ell \leq v\leq w  \\ w-v\geq \max \left\{ v-\ell ,\ell
-g\right\} ,w-v>0}}^{T-h}\left\vert E\left[ u_{it}u_{is}u_{ig}u_{i\ell
}u_{iv}u_{iw}\right] \right\vert \\
&&+\frac{C_{1}C}{N_{1}N_{2}T_{h}^{3}}\dsum\limits_{k\in H^{{\large c}%
}}\dsum\limits_{i\in H}\dsum\limits_{\substack{ t,s,g,\ell ,v,w=p  \\ t\leq
s\leq g\leq \ell \leq v\leq w  \\ v-\ell \geq \max \left\{ w-v,\ell
-g\right\} ,v-\ell >0}}^{T-h}\left\vert E\left[ u_{it}u_{is}u_{ig}u_{i\ell
}u_{iv}u_{iw}\right] \right\vert \\
&&+\frac{C_{1}C}{N_{1}N_{2}T_{h}^{3}}\dsum\limits_{k\in H^{{\large c}%
}}\dsum\limits_{i\in H}\dsum\limits_{\substack{ t,s,g,\ell ,v,w=p  \\ t\leq
s\leq g\leq \ell \leq v\leq w  \\ \ell -g\geq \max \left\{ w-v,v-\ell
\right\} ,\ell -g>0}}^{T-h}\left\vert E\left[ u_{it}u_{is}u_{ig}u_{i\ell
}u_{iv}u_{iw}\right] \right\vert \\
&\leq &\frac{C_{1}C}{N_{1}N_{2}T_{h}^{3}}\dsum\limits_{k\in H^{{\large c}%
}}\dsum\limits_{i\in H}\dsum\limits_{\substack{ t,s,g=p  \\ t\leq s\leq g}}%
^{T-h}\left\vert E\left[ u_{it}u_{is}u_{ig}^{4}\right] \right\vert \\
&&+\frac{C_{1}C}{N_{1}N_{2}T_{h}^{3}}\dsum\limits_{k\in H^{{\large c}%
}}\dsum\limits_{i\in H}\dsum\limits_{\substack{ t,s,g,\ell ,v,w=p  \\ t\leq
s\leq g\leq \ell \leq v\leq w  \\ w-v\geq \max \left\{ v-\ell ,\ell
-g\right\} ,w-v>0}}^{T-h}\left\vert E\left[ u_{it}u_{is}u_{ig}u_{i\ell
}u_{iv}u_{iw}\right] \right\vert \\
&&+\frac{C_{1}C}{N_{1}N_{2}T_{h}^{3}}\dsum\limits_{k\in H^{{\large c}%
}}\dsum\limits_{i\in H}\dsum\limits_{\substack{ t,s,g,\ell ,v,w=p  \\ t\leq
s\leq g\leq \ell \leq v\leq w  \\ v-\ell \geq \max \left\{ w-v,\ell
-g\right\} ,v-\ell >0}}^{T-h}\left\vert E\left[ \left\{
u_{it}u_{is}u_{ig}u_{i\ell }-E\left( u_{it}u_{is}u_{ig}u_{i\ell }\right)
\right\} u_{iv}u_{iw}\right] \right\vert \\
&&+\frac{C_{1}C}{N_{1}N_{2}T_{h}^{3}}\dsum\limits_{k\in H^{{\large c}%
}}\dsum\limits_{i\in H}\dsum\limits_{\substack{ t,s,g,\ell ,v,w=p  \\ t\leq
s\leq g\leq \ell \leq v\leq w  \\ v-\ell \geq \max \left\{ w-v,\ell
-g\right\} ,v-\ell >0}}^{T-h}\left\vert E\left( u_{it}u_{is}u_{ig}u_{i\ell
}\right) \right\vert \left\vert E\left( u_{iv}u_{iw}\right) \right\vert
\end{eqnarray*}

\begin{eqnarray*}
&&+\frac{C_{1}C}{N_{1}N_{2}T_{h}^{3}}\dsum\limits_{k\in H^{{\large c}%
}}\dsum\limits_{i\in H}\dsum\limits_{\substack{ t,s,g,\ell ,v,w=p  \\ t\leq
s\leq g\leq \ell \leq v\leq w  \\ \ell -g\geq \max \left\{ w-v,v-\ell
\right\} ,\ell -g>0}}^{T-h}\left\vert E\left[ \left\{
u_{it}u_{is}u_{ig}-E\left( u_{it}u_{is}u_{ig}\right) \right\} u_{i\ell
}u_{iv}u_{iw}\right] \right\vert \\
&&+\frac{C_{1}C}{N_{1}N_{2}T_{h}^{3}}\dsum\limits_{k\in H^{{\large c}%
}}\dsum\limits_{i\in H}\dsum\limits_{\substack{ t,s,g,\ell ,v,w=p  \\ t\leq
s\leq g\leq \ell \leq v\leq w  \\ \ell -g\geq \max \left\{ w-v,v-\ell
\right\} ,\ell -g>0}}^{T-h}\left\vert E\left( u_{it}u_{is}u_{ig}\right)
\right\vert \left\vert E\left( u_{i\ell }u_{iv}u_{iw}\right) \right\vert \\
&=&\mathcal{TT}_{1}+\mathcal{TT}_{2}+\mathcal{TT}_{3}+\mathcal{TT}_{4}+%
\mathcal{TT}_{5}+\mathcal{TT}_{6},\text{ }\left( say\right) \text{.}
\end{eqnarray*}%
Consider first $\mathcal{TT}_{1}$. Note that%
\begin{eqnarray}
\mathcal{TT}_{1} &=&\frac{C_{1}C}{N_{1}N_{2}T_{h}^{3}}\dsum\limits_{k\in H^{%
{\large c}}}\dsum\limits_{i\in H}\dsum\limits_{\substack{ t,s,g=p  \\ t\leq
s\leq g}}^{T-h}\left\vert E\left[ u_{it}u_{is}u_{ig}^{4}\right] \right\vert 
\notag \\
&\leq &\frac{C_{1}C}{N_{1}N_{2}T_{h}^{3}}\dsum\limits_{k\in H^{{\large c}%
}}\dsum\limits_{i\in H}\dsum\limits_{\substack{ t,s,g=p  \\ t\leq s\leq g}}%
^{T-h}E\left[ \left\vert u_{it}u_{is}u_{ig}^{4}\right\vert \right]  \notag \\
&\leq &\text{ }\frac{C_{1}C}{N_{1}N_{2}T_{h}^{3}}\dsum\limits_{k\in H^{%
{\large c}}}\dsum\limits_{i\in H}\dsum\limits_{\substack{ t,s,g=p  \\ t\leq
s\leq g}}^{T-h}\left( E\left[ \left\vert u_{it}u_{is}\right\vert ^{3}\right]
\right) ^{\frac{{\large 1}}{{\large 3}}}\left( E\left[ \left\vert
u_{ig}\right\vert ^{6}\right] \right) ^{\frac{{\large 2}}{{\large 3}}}\left( 
\text{by H\"{o}lder's inequality}\right)  \notag \\
&\leq &\frac{C_{1}C}{N_{1}N_{2}T_{h}^{3}}\dsum\limits_{k\in H^{{\large c}%
}}\dsum\limits_{i\in H}\dsum\limits_{\substack{ t,s,g=p  \\ t\leq s\leq g}}%
^{T-h}\left( \left[ E\left\{ \left\vert u_{it}\right\vert ^{6}\right\} %
\right] ^{\frac{{\large 1}}{{\large 2}}}\left[ E\left\{ \left\vert
u_{is}\right\vert ^{6}\right\} \right] ^{\frac{{\large 1}}{{\large 2}}%
}\right) ^{\frac{{\large 1}}{{\large 3}}}\left( E\left[ \left\vert
u_{ig}\right\vert ^{6}\right] \right) ^{\frac{{\large 2}}{{\large 3}}} 
\notag \\
&&\left( \text{by further application of H\"{o}lder's inequality}\right) 
\notag \\
&=&\frac{C_{1}C}{N_{1}N_{2}T_{h}^{3}}\dsum\limits_{k\in H^{{\large c}%
}}\dsum\limits_{i\in H}\dsum\limits_{\substack{ t,s,g=p  \\ t\leq s\leq g}}%
^{T-h}\left( E\left\{ \left\vert u_{it}\right\vert ^{6}\right\} \right) ^{%
\frac{{\large 1}}{{\large 6}}}\left( E\left\{ \left\vert u_{is}\right\vert
^{6}\right\} \right) ^{\frac{{\large 1}}{{\large 6}}}\left( E\left[
\left\vert u_{ig}\right\vert ^{6}\right] \right) ^{\frac{{\large 2}}{{\large %
3}}}  \notag \\
&\leq &\text{ }\frac{C_{1}C}{N_{1}N_{2}T_{h}^{3}}\dsum\limits_{k\in H^{%
{\large c}}}\dsum\limits_{i\in H}\dsum\limits_{\substack{ t,s,g=p  \\ t\leq
s\leq g}}^{T-h}\left( \sup_{t}E\left\{ \left\vert u_{it}\right\vert
^{7}\right\} \right) ^{\frac{{\large 6}}{{\large 7}}}  \notag \\
&&\left( \text{using Liapunov's inequality}\right)  \notag \\
&\leq &\text{ }\frac{C_{1}C}{N_{1}N_{2}T_{h}^{3}}\dsum\limits_{k\in H^{%
{\large c}}}\dsum\limits_{i\in H}\dsum\limits_{\substack{ t,s,g=p  \\ t\leq
s\leq g}}^{T-h}\overline{C}^{\frac{{\large 6}}{{\large 7}}}\text{\ }\left( 
\text{by Assumption \ 3-3(b)}\right)  \notag \\
&\leq &C_{1}C\overline{C}^{\frac{{\large 6}}{{\large 7}}}\frac{%
N_{1}N_{2}T_{h}^{3}}{N_{1}N_{2}T_{h}^{3}}  \notag \\
&=&C_{1}C\overline{C}^{\frac{{\large 6}}{{\large 7}}}=O\left( 1\right)
\label{TT1}
\end{eqnarray}

Next, consider $\mathcal{TT}_{2}$. For this term, note first that by
Assumption 3-3(c), $\left\{ u_{it}\right\} _{t=-\infty }^{\infty }$ is $%
\beta $-mixing with $\beta $ mixing coefficient satisfying 
\begin{equation*}
\beta _{i}\left( m\right) \leq a_{1}\exp \left\{ -a_{2}m\right\} \text{ }
\end{equation*}%
for every $i$. Since $\alpha _{i,m}\leq \beta _{i}\left( m\right) $, it
follows that $\left\{ u_{it}\right\} _{t=-\infty }^{\infty }$ is $\alpha $%
-mixing as well, with $\alpha $ mixing coefficient satisfying%
\begin{equation*}
\alpha _{i,m}\leq a_{1}\exp \left\{ -a_{2}m\right\} \text{ for every }i\text{%
.}
\end{equation*}

\noindent Hence, in this case, we can apply Lemma C-3 with $p=5/4$ and $r=6$
to obtain%
\begin{eqnarray*}
&&\mathcal{TT}_{2} \\
&=&\frac{C_{1}C}{N_{1}N_{2}T_{h}^{3}}\dsum\limits_{k\in H^{{\large c}%
}}\dsum\limits_{i\in H}\dsum\limits_{\substack{ t,s,g,\ell ,v,w=p  \\ t\leq
s\leq g\leq \ell \leq v\leq w  \\ w-v\geq \max \left\{ v-\ell ,\ell
-g\right\} ,w-v>0}}^{T-h}\left\vert E\left[ u_{it}u_{is}u_{ig}u_{i\ell
}u_{iv}u_{iw}\right] \right\vert \\
&\leq &\frac{C_{1}C}{N_{1}N_{2}T_{h}^{3}}\dsum\limits_{k\in H^{{\large c}%
}}\dsum\limits_{i\in H}\dsum\limits_{\substack{ t,s,g,\ell ,v,w=p  \\ t\leq
s\leq g\leq \ell \leq v\leq w  \\ w-v\geq \max \left\{ v-\ell ,\ell
-g\right\} ,w-v>0}}^{T-h}\left\{ 2\left( 2^{1-\frac{{\large 4}}{{\large 5}}%
}+1\right) \left[ a_{1}\exp \left\{ -a_{2}\left( w-v\right) \right\} \right]
^{1-\frac{{\large 4}}{{\large 5}}-\frac{{\large 1}}{{\large 6}}}\right. \\
&&\text{ \ \ \ \ \ \ \ \ \ \ \ \ \ \ \ \ \ \ \ \ \ \ \ \ \ \ \ \ \ \ \ \ \ \
\ \ \ \ \ \ \ \ \ \ \ \ \ \ \ \ \ \ \ \ \ \ \ \ \ }\left. \times \left(
E\left\vert u_{it}u_{is}u_{ig}u_{i\ell }u_{iv}\right\vert ^{\frac{{\large 5}%
}{{\large 4}}}\right) ^{\frac{{\large 4}}{{\large 5}}}\left( E\left\vert
u_{iw}\right\vert ^{6}\right) ^{\frac{{\large 1}}{{\large 6}}}\right\} \\
&=&\frac{C_{1}C}{N_{1}N_{2}T_{h}^{3}}\dsum\limits_{k\in H^{{\large c}%
}}\dsum\limits_{i\in H}\dsum\limits_{\substack{ t,s,g,\ell ,v,w=p  \\ t\leq
s\leq g\leq \ell \leq v\leq w  \\ w-v\geq \max \left\{ v-\ell ,\ell
-g\right\} ,w-v>0}}^{T-h}\left\{ 2\left( 2^{\frac{{\large 1}}{{\large 5}}%
}+1\right) \left[ a_{1}\exp \left\{ -a_{2}\left( w-v\right) \right\} \right]
^{\frac{{\large 1}}{{\large 30}}}\right. \\
&&\text{ \ \ \ \ \ \ \ \ \ \ \ \ \ \ \ \ \ \ \ \ \ \ \ \ \ \ \ \ \ \ \ \ \ \
\ \ \ \ \ \ \ \ \ \ \ \ \ \ \ \ \ \ \ \ \ \ \ \ \ }\left. \times \left(
E\left\vert u_{it}u_{is}u_{ig}u_{i\ell }u_{iv}\right\vert ^{\frac{{\large 5}%
}{{\large 4}}}\right) ^{\frac{{\large 4}}{{\large 5}}}\left( E\left\vert
u_{iw}\right\vert ^{6}\right) ^{\frac{{\large 1}}{{\large 6}}}\right\}
\end{eqnarray*}%
Next, by repeated application of H\"{o}lder's inequality, we have 
\begin{eqnarray*}
&&E\left\vert u_{it}u_{is}u_{ig}u_{i\ell }u_{iv}\right\vert ^{\frac{{\large 5%
}}{{\large 4}}} \\
&\leq &\left[ E\left\vert u_{it}u_{is}u_{ig}\right\vert ^{\frac{{\large 25}}{%
{\large 12}}}\right] ^{\frac{{\large 3}}{{\large 5}}}\left[ E\left\vert
u_{i\ell }u_{iv}\right\vert ^{\frac{{\large 25}}{{\large 8}}}\right] ^{\frac{%
{\large 2}}{{\large 5}}} \\
&\leq &\left[ \left( E\left\vert u_{it}u_{is}\right\vert ^{\frac{{\large 150}%
}{{\large 47}}}\right) ^{\frac{{\large 47}}{{\large 72}}}\left( E\left\vert
u_{ig}\right\vert ^{6}\right) ^{\frac{{\large 25}}{{\large 72}}}\right] ^{%
\frac{{\large 3}}{{\large 5}}}\left[ \left( E\left\vert u_{i\ell
}\right\vert ^{\frac{{\large 25}}{{\large 4}}}\right) ^{\frac{{\large 1}}{%
{\large 2}}}\left( E\left\vert u_{iv}\right\vert ^{\frac{{\large 25}}{%
{\large 4}}}\right) ^{\frac{{\large 1}}{{\large 2}}}\right] ^{\frac{{\large 2%
}}{{\large 5}}} \\
&\leq &\left[ \left( \sqrt{E\left\vert u_{it}\right\vert ^{\frac{{\large 300}%
}{{\large 47}}}}\sqrt{E\left\vert u_{is}\right\vert ^{\frac{{\large 300}}{%
{\large 47}}}}\right) ^{\frac{{\large 47}}{{\large 72}}}\left( E\left\vert
u_{ig}\right\vert ^{6}\right) ^{\frac{{\large 25}}{{\large 72}}}\right] ^{%
\frac{{\large 3}}{{\large 5}}}\left[ \left( E\left\vert u_{i\ell
}\right\vert ^{\frac{{\large 25}}{{\large 4}}}\right) ^{\frac{{\large 1}}{%
{\large 2}}}\left( E\left\vert u_{iv}\right\vert ^{\frac{{\large 25}}{%
{\large 4}}}\right) ^{\frac{{\large 1}}{{\large 2}}}\right] ^{\frac{{\large 2%
}}{{\large 5}}} \\
&\leq &\left( E\left\vert u_{it}\right\vert ^{\frac{{\large 300}}{{\large 47}%
}}\right) ^{\frac{{\large 141}}{{\large 720}}}\left( E\left\vert
u_{is}\right\vert ^{\frac{{\large 300}}{{\large 47}}}\right) ^{\frac{{\large %
141}}{{\large 720}}}\left( E\left\vert u_{i\ell }\right\vert ^{6}\right) ^{%
\frac{{\large 15}}{{\large 72}}}\left( E\left\vert u_{iv}\right\vert ^{\frac{%
{\large 25}}{{\large 4}}}\right) ^{\frac{{\large 1}}{{\large 5}}}\left(
E\left\vert u_{iw}\right\vert ^{\frac{{\large 25}}{{\large 4}}}\right) ^{%
\frac{{\large 1}}{{\large 5}}} \\
&=&\left[ \left( E\left\vert u_{it}\right\vert ^{\frac{{\large 300}}{{\large %
47}}}\right) ^{\frac{{\large 47}}{{\large 300}}}\left( E\left\vert
u_{is}\right\vert ^{\frac{{\large 300}}{{\large 47}}}\right) ^{\frac{{\large %
47}}{{\large 300}}}\right] ^{\frac{{\large 5}}{{\large 4}}}\left[ \left(
E\left\vert u_{i\ell }\right\vert ^{6}\right) ^{\frac{{\large 1}}{{\large 6}}%
}\right] ^{\frac{{\large 5}}{{\large 4}}}\left[ \left( E\left\vert
u_{iv}\right\vert ^{\frac{{\large 25}}{{\large 4}}}\right) ^{\frac{{\large 4}%
}{{\large 25}}}\right] ^{\frac{{\large 5}}{{\large 4}}}\left[ \left(
E\left\vert u_{iw}\right\vert ^{\frac{{\large 25}}{{\large 4}}}\right) ^{%
\frac{{\large 4}}{{\large 25}}}\right] ^{\frac{{\large 5}}{{\large 4}}} \\
&\leq &\left[ \left( E\left\vert u_{it}\right\vert ^{7}\right) ^{\frac{%
{\large 1}}{{\large 7}}}\left( E\left\vert u_{is}\right\vert ^{7}\right) ^{%
\frac{{\large 1}}{{\large 7}}}\right] ^{\frac{{\large 5}}{{\large 4}}}\left[
\left( E\left\vert u_{i\ell }\right\vert ^{7}\right) ^{\frac{{\large 1}}{%
{\large 7}}}\right] ^{\frac{{\large 5}}{{\large 4}}}\left[ \left(
E\left\vert u_{iv}\right\vert ^{7}\right) ^{\frac{{\large 1}}{{\large 7}}}%
\right] ^{\frac{{\large 5}}{{\large 4}}}\left[ \left( E\left\vert
u_{iw}\right\vert ^{7}\right) ^{\frac{{\large 1}}{{\large 7}}}\right] ^{%
\frac{{\large 5}}{{\large 4}}} \\
&&\left( \text{by Liapunov's inequality}\right) \\
&\leq &\left( \overline{C}\right) ^{\frac{{\large 5}}{{\large 28}}}\left( 
\overline{C}\right) ^{\frac{{\large 5}}{{\large 28}}}\left( \overline{C}%
\right) ^{\frac{{\large 5}}{{\large 28}}}\left( \overline{C}\right) ^{\frac{%
{\large 5}}{{\large 28}}}\left( \overline{C}\right) ^{\frac{{\large 5}}{%
{\large 28}}}\text{ \ }\left( \text{by Assumption 3-3(b)}\right) \\
&=&\overline{C}^{\frac{{\large 25}}{{\large 28}}}
\end{eqnarray*}%
By Liapunov's inequality and Assumption 3-3(b), we also obtain%
\begin{equation*}
\left( E\left\vert u_{iw}\right\vert ^{6}\right) ^{\frac{{\large 1}}{{\large %
6}}}\leq \left( E\left\vert u_{iw}\right\vert ^{7}\right) ^{\frac{{\large 1}%
}{{\large 7}}}\leq \overline{C}^{\frac{{\large 1}}{{\large 7}}}\text{.}
\end{equation*}%
Moreover, let $\rho _{1}=\ell -g$, $\rho _{2}=v-\ell $, and $\rho _{3}=w-v$,
so that $\ell =g+\rho _{1}$, $v=\ell +$ $\rho _{2}=g+\rho _{1}+$ $\rho _{2}$%
, $w=v+\rho _{3}=g+\rho _{1}+$ $\rho _{2}+\rho _{3}$.. Using these notations
and the boundedness of $E\left\vert u_{it}u_{is}u_{ig}u_{i\ell
}u_{iv}\right\vert ^{\frac{{\large 5}}{{\large 4}}}$ as shown above, we can
further write%
\begin{eqnarray}
&&\mathcal{TT}_{2}  \notag \\
&\leq &\frac{C_{1}C}{N_{1}N_{2}T_{h}^{3}}\dsum\limits_{k\in H^{{\large c}%
}}\dsum\limits_{i\in H}\dsum\limits_{\substack{ t,s,g,\ell ,v,w=p  \\ t\leq
s\leq g\leq \ell \leq v\leq w  \\ w-v\geq \max \left\{ v-\ell ,\ell
-g\right\} ,w-v>0}}^{T-h}\left\{ 2\left( 2^{\frac{{\large 1}}{{\large 5}}%
}+1\right) \left[ a_{1}\exp \left\{ -a_{2}\left( w-v\right) \right\} \right]
^{\frac{{\large 1}}{{\large 30}}}\right.  \notag \\
&&\text{ \ \ \ \ \ \ \ \ \ \ \ \ \ \ \ \ \ \ \ \ \ \ \ \ \ \ \ \ \ \ \ \ \ \
\ \ \ \ \ \ \ \ \ \ \ \ \ \ \ \ \ \ \ \ \ \ \ \ }\left. \times \left(
E\left\vert u_{it}u_{is}u_{ig}u_{i\ell }u_{iv}\right\vert ^{\frac{{\large 5}%
}{{\large 4}}}\right) ^{\frac{{\large 4}}{{\large 5}}}\left( E\left\vert
u_{iw}\right\vert ^{6}\right) ^{\frac{{\large 1}}{{\large 6}}}\right\} 
\notag \\
&\leq &\frac{C_{1}C}{N_{1}N_{2}T_{h}^{3}}\dsum\limits_{k\in H^{{\large c}%
}}\dsum\limits_{i\in H}\dsum\limits_{\substack{ t,s,g,\ell ,v,w=p  \\ t\leq
s\leq g\leq \ell \leq v\leq w  \\ w-v\geq \max \left\{ v-\ell ,\ell
-g\right\} ,w-v>0}}^{T-h}2\left( 2^{\frac{{\large 1}}{{\large 5}}}+1\right) %
\left[ a_{1}\exp \left\{ -a_{2}\left( w-v\right) \right\} \right] ^{\frac{%
{\large 1}}{{\large 30}}}\left( \overline{C}^{\frac{{\large 25}}{{\large 28}}%
}\right) ^{\frac{{\large 4}}{{\large 5}}}\overline{C}^{\frac{{\large 1}}{%
{\large 7}}}  \notag \\
&\leq &\frac{C_{1}C\overline{C}^{\frac{{\large 6}}{{\large 7}}}}{%
N_{1}N_{2}T_{h}^{3}}\dsum\limits_{k\in H^{{\large c}}}\dsum\limits_{i\in
H}\dsum\limits_{\substack{ t,s,g,\ell ,v,w=p  \\ t\leq s\leq g\leq \ell \leq
v\leq w  \\ w-v\geq \max \left\{ v-\ell ,\ell -g\right\} ,w-v>0}}%
^{T-h}2\left( 2^{\frac{{\large 1}}{{\large 5}}}+1\right) \left[ a_{1}\exp
\left\{ -a_{2}\left( w-v\right) \right\} \right] ^{\frac{{\large 1}}{{\large %
30}}}  \notag \\
&\leq &\text{ }\frac{C_{1}^{\ast }}{N_{1}N_{2}T_{h}^{3}}\dsum\limits_{k\in
H^{{\large c}}}\dsum\limits_{i\in H}\dsum\limits_{\substack{ t,s,g,\ell
,v,w=p  \\ t\leq s\leq g\leq \ell \leq v\leq w  \\ w-v\geq \max \left\{
v-\ell ,\ell -g\right\} ,w-v>0}}^{T-h}\exp \left\{ -\frac{a_{2}}{30}\rho
_{3}\right\}  \notag \\
&&\left( \text{for some constant }C_{1}^{\ast }\text{ such that }2\left( 2^{%
\frac{{\large 1}}{{\large 5}}}+1\right) C_{1}C\overline{C}^{\frac{{\large 6}%
}{{\large 7}}}a_{1}^{\frac{{\large 1}}{{\large 30}}}\leq C_{1}^{\ast
}<\infty \right)  \notag \\
&\leq &\frac{C_{1}^{\ast }}{N_{1}N_{2}T_{h}^{3}}\dsum\limits_{k\in H^{%
{\large c}}}\dsum\limits_{i\in H}\dsum\limits_{\substack{ t=p}}%
^{T-h}\dsum\limits_{\substack{ s=p}}^{T-h}\dsum\limits_{\substack{ g=p}}%
^{T-h}\dsum\limits_{\substack{ \rho _{{\large 3}}=1}}^{\infty
}\dsum\limits_{\rho _{{\large 1}}=0}^{\rho _{{\large 3}}}\dsum\limits_{\rho
_{{\large 2}}=0}^{\rho _{{\large 3}}}\exp \left\{ -\frac{a_{2}}{30}\rho
_{3}\right\}  \notag \\
&\leq &\frac{C_{1}^{\ast }}{N_{1}N_{2}T_{h}^{3}}\dsum\limits_{k\in H^{%
{\large c}}}\dsum\limits_{i\in H}\dsum\limits_{\substack{ t=p}}%
^{T-h}\dsum\limits_{\substack{ s=p}}^{T-h}\dsum\limits_{\substack{ g=p}}%
^{T-h}\dsum\limits_{\substack{ \rho _{{\large 3}}=1}}^{\infty }\left( \rho
_{3}+1\right) ^{2}\exp \left\{ -\frac{a_{2}}{30}\rho _{3}\right\}  \notag \\
&=&C_{1}^{\ast }\frac{N_{1}N_{2}T_{h}^{3}}{N_{1}N_{2}T_{h}^{3}}\left[
\dsum\limits_{\substack{ \rho _{{\large 3}}=1}}^{\infty }\rho _{3\backslash
}^{2}\exp \left\{ -\frac{a_{2}}{30}\rho _{3}\right\} +2\dsum\limits 
_{\substack{ \rho _{{\large 3}}=1}}^{\infty }\rho _{3\backslash }\exp
\left\{ -\frac{a_{2}}{30}\rho _{3}\right\} +\dsum\limits_{\substack{ \rho _{%
{\large 3}}=1}}^{\infty }\exp \left\{ -\frac{a_{2}}{30}\rho _{3}\right\} %
\right]  \notag \\
&\leq &C_{1}^{\ast }\overline{\overline{C}}_{1}\text{ \ }  \label{TT2}
\end{eqnarray}%
for some positive constant 
\begin{equation*}
\overline{\overline{C}}_{1}\geq \dsum\limits_{\substack{ \rho _{{\large 3}%
}=1 }}^{\infty }\rho _{3\backslash }^{2}\exp \left\{ -\frac{a_{2}}{30}\rho
_{3}\right\} +2\dsum\limits_{\substack{ \rho _{{\large 3}}=1}}^{\infty }\rho
_{3\backslash }\exp \left\{ -\frac{a_{2}}{30}\rho _{3}\right\} +\dsum\limits 
_{\substack{ \rho _{{\large 3}}=1}}^{\infty }\exp \left\{ -\frac{a_{2}}{30}%
\rho _{3}\right\} \text{.}
\end{equation*}%
which exists in light of Lemma C-1.

Now, consider $\mathcal{TT}_{3}$. Here, we apply Lemma C-3 with $p=3/2$ and $%
r=7/2$ to obtain%
\begin{eqnarray*}
\mathcal{TT}_{3} &=&\frac{C_{1}C}{N_{1}N_{2}T_{h}^{3}}\dsum\limits_{k\in H^{%
{\large c}}}\dsum\limits_{i\in H}\dsum\limits_{\substack{ t,s,g,\ell ,v,w=p 
\\ t\leq s\leq g\leq \ell \leq v\leq w  \\ v-\ell \geq \max \left\{ w-v,\ell
-g\right\} ,v-\ell >0}}^{T-h}\left\vert E\left[ \left\{
u_{it}u_{is}u_{ig}u_{i\ell }-E\left( u_{it}u_{is}u_{ig}u_{i\ell }\right)
\right\} u_{iv}u_{iw}\right] \right\vert \\
&\leq &\frac{C_{1}C}{N_{1}N_{2}T_{h}^{3}}\dsum\limits_{k\in H^{{\large c}%
}}\dsum\limits_{i\in H}\dsum\limits_{\substack{ t,s,g,\ell ,v,w=p  \\ t\leq
s\leq g\leq \ell \leq v\leq w  \\ v-\ell \geq \max \left\{ w-v,\ell
-g\right\} ,v-\ell >0}}^{T-h}\left\{ 2\left( 2^{1-\frac{{\large 2}}{{\large 3%
}}}+1\right) \left[ a_{1}\exp \left\{ -a_{2}\left( v-\ell \right) \right\} %
\right] ^{1-\frac{{\large 2}}{{\large 3}}-\frac{{\large 2}}{{\large 7}}%
}\right. \\
&&\text{ \ \ \ \ \ \ \ \ \ \ \ \ \ \ \ \ \ \ \ \ \ \ \ \ \ \ \ \ }\left.
\times \left( E\left\vert \left\{ u_{it}u_{is}u_{ig}u_{i\ell }-E\left(
u_{it}u_{is}u_{ig}u_{i\ell }\right) \right\} \right\vert ^{\frac{{\large 3}}{%
{\large 2}}}\right) ^{\frac{{\large 2}}{{\large 3}}}\left( E\left\vert
u_{iv}u_{iw}\right\vert ^{\frac{{\large 7}}{{\large 2}}}\right) ^{\frac{%
{\large 2}}{{\large 7}}}\right\} \\
&=&\frac{C_{1}C}{N_{1}N_{2}T_{h}^{3}}\dsum\limits_{k\in H^{{\large c}%
}}\dsum\limits_{i\in H}\dsum\limits_{\substack{ t,s,g,\ell ,v,w=p  \\ t\leq
s\leq g\leq \ell \leq v\leq w  \\ v-\ell \geq \max \left\{ w-v,\ell
-g\right\} ,v-\ell >0}}^{T-h}\left\{ 2\left( 2^{\frac{{\large 1}}{{\large 3}}%
}+1\right) \left[ a_{1}\exp \left\{ -a_{2}\left( v-\ell \right) \right\} %
\right] ^{\frac{{\large 1}}{{\large 21}}}\right. \\
&&\text{\ \ \ \ \ \ \ \ \ \ \ \ \ \ \ \ \ \ \ \ \ \ \ \ \ \ \ \ \ \ \ }%
\left. \times \left( E\left\vert \left\{ u_{it}u_{is}u_{ig}u_{i\ell
}-E\left( u_{it}u_{is}u_{ig}u_{i\ell }\right) \right\} \right\vert ^{\frac{%
{\large 3}}{{\large 2}}}\right) ^{\frac{{\large 2}}{{\large 3}}}\left(
E\left\vert u_{iv}u_{iw}\right\vert ^{\frac{{\large 7}}{{\large 2}}}\right)
^{\frac{{\large 2}}{{\large 7}}}\right\}
\end{eqnarray*}%
Next, observe that by applying of H\"{o}lder's inequality, we have%
\begin{eqnarray*}
E\left\vert u_{iv}u_{iw}\right\vert ^{\frac{{\large 7}}{{\large 2}}} &\leq
&\left( E\left\vert u_{iv}\right\vert ^{7}\right) ^{\frac{{\large 1}}{%
{\large 2}}}\left( E\left\vert u_{iw}\right\vert ^{7}\right) ^{\frac{{\large %
1}}{{\large 2}}} \\
&\leq &\left( \overline{C}\right) ^{\frac{{\large 1}}{{\large 2}}}\left( 
\overline{C}\right) ^{\frac{{\large 1}}{{\large 2}}}\text{ \ }\left( \text{%
by Assumption 3-3(b)}\right) \\
&=&\overline{C}<\infty \text{,}
\end{eqnarray*}%
and%
\begin{eqnarray*}
E\left\vert \left\{ u_{it}u_{is}u_{ig}u_{i\ell }-E\left(
u_{it}u_{is}u_{ig}u_{i\ell }\right) \right\} \right\vert ^{\frac{{\large 3}}{%
{\large 2}}} &\leq &2^{\frac{{\large 1}}{{\large 2}}}\left( E\left\vert
u_{it}u_{is}u_{ig}u_{i\ell }\right\vert ^{\frac{{\large 3}}{{\large 2}}%
}+\left\vert E\left[ u_{it}u_{is}u_{ig}u_{i\ell }\right] \right\vert ^{\frac{%
{\large 3}}{{\large 2}}}\right) \\
&&\left( \text{by Lo\`{e}ve's }c_{r}\text{ inequality}\right) \\
&\leq &2^{\frac{{\large 1}}{{\large 2}}}\left( E\left\vert
u_{it}u_{is}u_{ig}u_{i\ell }\right\vert ^{\frac{{\large 3}}{{\large 2}}%
}+E\left\vert u_{it}u_{is}u_{ig}u_{i\ell }\right\vert ^{\frac{{\large 3}}{%
{\large 2}}}\right) \text{ } \\
&&\left( \text{by Jensen's inequality}\right) \\
&\leq &2^{\frac{{\large 3}}{{\large 2}}}E\left\vert
u_{it}u_{is}u_{ig}u_{i\ell }\right\vert ^{\frac{{\large 3}}{{\large 2}}} \\
&\leq &2^{\frac{{\large 3}}{{\large 2}}}\left( E\left\vert
u_{it}u_{is}\right\vert ^{3}\right) ^{\frac{{\large 1}}{{\large 2}}}\left(
E\left\vert u_{ig}u_{i\ell }\right\vert ^{3}\right) ^{\frac{{\large 1}}{%
{\large 2}}} \\
&\leq &2^{\frac{{\large 3}}{{\large 2}}}\left( \left( E\left\vert
u_{it}\right\vert ^{6}\right) ^{\frac{{\large 1}}{{\large 2}}}\left(
E\left\vert u_{is}\right\vert ^{6}\right) ^{\frac{{\large 1}}{{\large 2}}%
}\right) ^{\frac{{\large 1}}{{\large 2}}}\left( \left( E\left\vert
u_{ig}\right\vert ^{6}\right) ^{\frac{{\large 1}}{{\large 2}}}\left(
E\left\vert u_{i\ell }\right\vert ^{6}\right) ^{\frac{{\large 1}}{{\large 2}}%
}\right) ^{\frac{{\large 1}}{{\large 2}}} \\
&=&2^{\frac{{\large 3}}{{\large 2}}}\left[ \left( E\left\vert
u_{it}\right\vert ^{6}\right) ^{\frac{{\large 1}}{{\large 6}}}\left(
E\left\vert u_{is}\right\vert ^{6}\right) ^{\frac{{\large 1}}{{\large 6}}%
}\left( E\left\vert u_{ig}\right\vert ^{6}\right) ^{\frac{{\large 1}}{%
{\large 6}}}\left( E\left\vert u_{i\ell }\right\vert ^{6}\right) ^{\frac{%
{\large 1}}{{\large 6}}}\right] ^{\frac{{\large 3}}{{\large 2}}} \\
&\leq &2^{\frac{{\large 3}}{{\large 2}}}\left[ \left( E\left\vert
u_{it}\right\vert ^{7}\right) ^{\frac{{\large 1}}{{\large 7}}}\left(
E\left\vert u_{is}\right\vert ^{7}\right) ^{\frac{{\large 1}}{{\large 7}}%
}\left( E\left\vert u_{ig}\right\vert ^{7}\right) ^{\frac{{\large 1}}{%
{\large 7}}}\left( E\left\vert u_{i\ell }\right\vert ^{7}\right) ^{\frac{%
{\large 1}}{{\large 7}}}\right] ^{\frac{{\large 3}}{{\large 2}}} \\
&&\left( \text{by Liapunov's inequality}\right) \\
&\leq &2^{\frac{{\large 3}}{{\large 2}}}\left[ \left( \sup_{t}E\left\vert
u_{it}\right\vert ^{7}\right) ^{\frac{{\large 4}}{{\large 7}}}\right] ^{%
\frac{{\large 3}}{{\large 2}}}\text{ \ } \\
&=&2^{\frac{{\large 3}}{{\large 2}}}\overline{C}^{\frac{{\large 6}}{{\large 7%
}}}\text{ \ }\left( \text{by Assumption 3-3(b)}\right)
\end{eqnarray*}%
Again, let $\rho _{1}=\ell -g$, $\rho _{2}=v-\ell $, and $\rho _{3}=w-v$, so
that $\ell =g+\rho _{1}$, $v=\ell +$ $\rho _{2}=g+\rho _{1}+$ $\rho _{2}$, $%
w=v+\rho _{3}=g+\rho _{1}+$ $\rho _{2}+\rho _{3}$. Using these notations and
the boundedness of $E\left\vert u_{iv}u_{iw}\right\vert ^{\frac{{\large 7}}{%
{\large 2}}}$ and $E\left\vert \left\{ u_{it}u_{is}u_{ig}u_{i\ell }-E\left(
u_{it}u_{is}u_{ig}u_{i\ell }\right) \right\} \right\vert ^{\frac{{\large 3}}{%
{\large 2}}}$ as shown above, we can further write%
\begin{eqnarray}
&&\mathcal{TT}_{3}  \notag \\
&=&\frac{C_{1}C}{N_{1}N_{2}T_{h}^{3}}\dsum\limits_{k\in H^{{\large c}%
}}\dsum\limits_{i\in H}\dsum\limits_{\substack{ t,s,g,\ell ,v,w=p  \\ t\leq
s\leq g\leq \ell \leq v\leq w  \\ v-\ell \geq \max \left\{ w-v,\ell
-g\right\} ,v-\ell >0}}^{T-h}\left\vert E\left[ \left\{
u_{it}u_{is}u_{ig}u_{i\ell }-E\left( u_{it}u_{is}u_{ig}u_{i\ell }\right)
\right\} u_{iv}u_{iw}\right] \right\vert  \notag \\
&\leq &\frac{C_{1}C}{N_{1}N_{2}T_{h}^{3}}\dsum\limits_{k\in H^{{\large c}%
}}\dsum\limits_{i\in H}\dsum\limits_{\substack{ t,s,g,\ell ,v,w=p  \\ t\leq
s\leq g\leq \ell \leq v\leq w  \\ v-\ell \geq \max \left\{ w-v,\ell
-g\right\} ,v-\ell >0}}^{T-h}\left\{ 2\left( 2^{\frac{{\large 1}}{{\large 3}}%
}+1\right) \left[ a_{1}\exp \left\{ -a_{2}\left( v-\ell \right) \right\} %
\right] ^{\frac{{\large 1}}{{\large 21}}}\right.  \notag \\
&&\text{ \ \ \ \ \ \ \ \ \ \ \ \ \ \ \ \ \ \ \ \ \ \ \ \ \ \ \ \ \ \ \ \ }%
\left. \times \left( E\left\vert \left\{ u_{it}u_{is}u_{ig}u_{i\ell
}-E\left( u_{it}u_{is}u_{ig}u_{i\ell }\right) \right\} \right\vert ^{\frac{%
{\large 3}}{{\large 2}}}\right) ^{\frac{{\large 2}}{{\large 3}}}\left(
E\left\vert u_{iv}u_{iw}\right\vert ^{\frac{{\large 7}}{{\large 2}}}\right)
^{\frac{{\large 2}}{{\large 7}}}\right\}  \notag \\
&\leq &\frac{C_{1}C}{N_{1}N_{2}T_{h}^{3}}\dsum\limits_{k\in H^{{\large c}%
}}\dsum\limits_{i\in H}\dsum\limits_{\substack{ t,s,g,\ell ,v,w=p  \\ t\leq
s\leq g\leq \ell \leq v\leq w  \\ v-\ell \geq \max \left\{ w-v,\ell
-g\right\} ,v-\ell >0}}^{T-h}2\left( 2^{\frac{{\large 1}}{{\large 3}}%
}+1\right) \left[ a_{1}\exp \left\{ -a_{2}\left( v-\ell \right) \right\} %
\right] ^{\frac{{\large 1}}{{\large 21}}}\left( 2^{\frac{{\large 3}}{{\large %
2}}}\overline{C}^{\frac{{\large 6}}{{\large 7}}}\right) ^{\frac{{\large 2}}{%
{\large 3}}}\left( \overline{C}\right) ^{\frac{{\large 2}}{{\large 7}}} 
\notag \\
&\leq &\frac{C_{2}^{\ast }}{N_{1}N_{2}T_{h}^{3}}\dsum\limits_{k\in H^{%
{\large c}}}\dsum\limits_{i\in H}\dsum\limits_{\substack{ t,s,g,\ell ,v,w=p 
\\ t\leq s\leq g\leq \ell \leq v\leq w  \\ v-\ell \geq \max \left\{ w-v,\ell
-g\right\} ,v-\ell >0}}^{T-h}\exp \left\{ -\frac{a_{2}}{21}\varrho
_{2}\right\}  \notag \\
&&\left( \text{for some constant }C_{2}^{\ast }\text{ such that }4\left( 2^{%
\frac{{\large 1}}{{\large 3}}}+1\right) C_{1}C\overline{C}^{\frac{{\large 6}%
}{{\large 7}}}a_{1}^{\frac{{\large 1}}{{\large 21}}}\leq C_{2}^{\ast
}<\infty \right)  \notag \\
&\leq &\frac{C_{2}^{\ast }}{N_{1}N_{2}T_{h}^{3}}\dsum\limits_{k\in H^{%
{\large c}}}\dsum\limits_{i\in H}\dsum\limits_{\substack{ t=p}}%
^{T-h}\dsum\limits_{\substack{ s=p}}^{T-h}\dsum\limits_{\substack{ g=p}}%
^{T-h}\dsum\limits_{\varrho _{{\large 2}}=1}^{\infty }\dsum\limits_{\varrho
_{{\large 1}}=0}^{\varrho _{{\large 2}}}\dsum\limits_{\varrho _{{\large 3}%
}=0}^{\varrho _{{\large 2}}}\exp \left\{ -\frac{a_{2}}{21}\varrho
_{2}\right\}  \notag \\
&=&C_{2}^{\ast }\frac{N_{1}N_{2}T_{h}^{3}}{N_{1}N_{2}T_{h}^{3}}%
\dsum\limits_{\varrho _{{\large 2}}=1}^{\infty }\left( \varrho _{2}+1\right)
^{2}\exp \left\{ -\frac{a_{2}}{21}\varrho _{2}\right\}  \notag \\
&=&C_{2}^{\ast }\left[ \dsum\limits_{\varrho _{{\large 2}}=1}^{\infty
}\varrho _{2}^{2}\exp \left\{ -\frac{a_{2}}{21}\varrho _{2}\right\}
+2\dsum\limits_{\varrho _{{\large 2}}=1}^{\infty }\varrho _{2}\exp \left\{ -%
\frac{a_{2}}{21}\varrho _{2}\right\} +\dsum\limits_{\varrho _{{\large 2}%
}=1}^{\infty }\exp \left\{ -\frac{a_{2}}{21}\varrho _{2}\right\} \right] 
\notag \\
&\leq &C_{2}^{\ast }\overline{\overline{C}}_{2}  \label{TT3}
\end{eqnarray}%
for some positive constant 
\begin{equation*}
\overline{\overline{C}}_{2}\geq \dsum\limits_{\varrho _{{\large 2}%
}=1}^{\infty }\varrho _{2}^{2}\exp \left\{ -\frac{a_{2}}{21}\varrho
_{2}\right\} +2\dsum\limits_{\varrho _{{\large 2}}=1}^{\infty }\varrho
_{2}\exp \left\{ -\frac{a_{2}}{21}\varrho _{2}\right\}
+\dsum\limits_{\varrho _{{\large 2}}=1}^{\infty }\exp \left\{ -\frac{a_{2}}{%
21}\varrho _{2}\right\}
\end{equation*}%
which exists in light of Lemma C-1.

Turning our attention to the term $\mathcal{TT}_{4}$, note that, from the
upper bounds given in the proofs of parts (a) and (c) of Lemma D-5, it is
clear that there exists a positive constant $C^{\ast \ast }$ such that, for
all $i$ and for all $T$ sufficiently large,

\begin{equation*}
\frac{1}{T_{h}}\dsum\limits_{\substack{ v,w=p  \\ v\leq w}}^{T-h}\left\vert
E\left( u_{iv}u_{iw}\right) \right\vert \leq C_{1}^{\ast \ast }
\end{equation*}%
and%
\begin{equation*}
\frac{1}{T_{h}^{2}}\dsum\limits_{\substack{ t,s,g,\ell =p  \\ t\leq s\leq
g\leq \ell }}^{T-h}\left\vert E\left( u_{it}u_{is}u_{ig}u_{i\ell }\right)
\right\vert \leq C_{1}^{\ast \ast }
\end{equation*}%
from which it follows that%
\begin{eqnarray}
\mathcal{TT}_{4} &=&\text{\ \ \ }\frac{C_{1}C}{N_{1}N_{2}T_{h}^{3}}%
\dsum\limits_{k\in H^{{\large c}}}\dsum\limits_{i\in H}\dsum\limits 
_{\substack{ t,s,g,\ell ,v,w=p  \\ t\leq s\leq g\leq \ell \leq v\leq w  \\ %
v-\ell \geq \max \left\{ w-v,\ell -g\right\} ,v-\ell >0}}^{T-h}\left\vert
E\left( u_{it}u_{is}u_{ig}u_{i\ell }\right) \right\vert \left\vert E\left(
u_{iv}u_{iw}\right) \right\vert  \notag \\
&\leq &\frac{C_{1}C}{N_{1}N_{2}}\dsum\limits_{k\in H^{{\large c}%
}}\dsum\limits_{i\in H}\left( \frac{1}{T_{h}^{2}}\dsum\limits_{\substack{ %
t,s,g,\ell =p  \\ t\leq s\leq g\leq \ell }}^{T-h}\left\vert E\left(
u_{it}u_{is}u_{ig}u_{i\ell }\right) \right\vert \right) \left( \frac{1}{T_{h}%
}\dsum\limits_{\substack{ v,w=p  \\ v\leq w}}^{T-h}\left\vert E\left(
u_{iv}u_{iw}\right) \right\vert \right)  \notag \\
&\leq &\frac{C_{1}C}{N_{1}N_{2}}\dsum\limits_{k\in H^{{\large c}%
}}\dsum\limits_{i\in H}\left( C_{1}^{\ast \ast }\right) ^{2}  \notag \\
&=&C_{1}C\left( C_{1}^{\ast \ast }\right) ^{2}\frac{N_{1}N_{2}}{N_{1}N_{2}} 
\notag \\
&=&C_{1}C\left( C_{1}^{\ast \ast }\right) ^{2}  \label{TT4}
\end{eqnarray}

Consider now $\mathcal{TT}_{5}$. In this case, we apply Lemma C-3 with $p=2$
and $r=9/4$ to obtain%
\begin{eqnarray*}
\mathcal{TT}_{5} &=&\frac{C_{1}C}{N_{1}N_{2}T_{h}^{3}}\dsum\limits_{k\in H^{%
{\large c}}}\dsum\limits_{i\in H}\dsum\limits_{\substack{ t,s,g,\ell ,v,w=p 
\\ t\leq s\leq g\leq \ell \leq v\leq w  \\ \ell -g\geq \max \left\{
w-v,v-\ell \right\} ,\ell -g>0}}^{T-h}\left\vert E\left[ \left\{
u_{it}u_{is}u_{ig}-E\left( u_{it}u_{is}u_{ig}\right) \right\} u_{i\ell
}u_{iv}u_{iw}\right] \right\vert \\
&\leq &\frac{C_{1}C}{N_{1}N_{2}T_{h}^{3}}\dsum\limits_{k\in H^{{\large c}%
}}\dsum\limits_{i\in H}\dsum\limits_{\substack{ t,s,g,\ell ,v,w=p  \\ t\leq
s\leq g\leq \ell \leq v\leq w  \\ \ell -g\geq \max \left\{ w-v,v-\ell
\right\} ,\ell -g>0}}^{T-h}\left\{ 2\left( 2^{1-\frac{{\large 1}}{{\large 2}}%
}+1\right) \left[ a_{1}\exp \left\{ -a_{2}\left( \ell -g\right) \right\} %
\right] ^{1-\frac{{\large 1}}{{\large 2}}-\frac{{\large 4}}{{\large 9}}%
}\right. \\
&&\text{\ \ \ \ \ \ \ \ \ \ \ \ \ \ \ \ \ \ \ \ \ \ \ \ \ \ \ \ \ \ \ \ \ }%
\left. \times \left( E\left\vert \left\{ u_{it}u_{is}u_{ig}-E\left(
u_{it}u_{is}u_{ig}\right) \right\} \right\vert ^{2}\right) ^{\frac{{\large 1}%
}{{\large 2}}}\left( E\left\vert u_{i\ell }u_{iv}u_{iw}\right\vert ^{\frac{%
{\large 9}}{{\large 4}}}\right) ^{\frac{{\large 4}}{{\large 9}}}\right\} \\
&=&\frac{C_{1}C}{N_{1}N_{2}T_{h}^{3}}\dsum\limits_{k\in H^{{\large c}%
}}\dsum\limits_{i\in H}\dsum\limits_{\substack{ t,s,g,\ell ,v,w=p  \\ t\leq
s\leq g\leq \ell \leq v\leq w  \\ \ell -g\geq \max \left\{ w-v,v-\ell
\right\} ,\ell -g>0}}^{T-h}\left\{ 2\left( 2^{\frac{{\large 1}}{{\large 2}}%
}+1\right) \left[ a_{1}\exp \left\{ -a_{2}\left( \ell -g\right) \right\} %
\right] ^{\frac{{\large 1}}{{\large 18}}}\right. \\
&&\text{ \ \ \ \ \ \ \ \ \ \ \ \ \ \ \ \ \ \ \ \ \ \ \ \ \ \ \ \ \ \ \ }%
\left. \times \left( E\left\vert \left\{ u_{it}u_{is}u_{ig}-E\left(
u_{it}u_{is}u_{ig}\right) \right\} \right\vert ^{2}\right) ^{\frac{{\large 1}%
}{{\large 2}}}\left( E\left\vert u_{i\ell }u_{iv}u_{iw}\right\vert ^{\frac{%
{\large 9}}{{\large 4}}}\right) ^{\frac{{\large 4}}{{\large 9}}}\right\}
\end{eqnarray*}%
Next, by repeated application of H\"{o}lder's inequality, we obtain%
\begin{eqnarray*}
&&E\left\vert u_{i\ell }u_{iv}u_{iw}\right\vert ^{\frac{{\large 9}}{{\large 4%
}}} \\
&\leq &\left[ E\left\vert u_{i\ell }\right\vert ^{7}\right] ^{\frac{{\large 9%
}}{{\large 28}}}\left[ E\left\vert u_{iv}u_{iw}\right\vert ^{\frac{{\large 63%
}}{{\large 19}}}\right] ^{\frac{{\large 19}}{{\large 28}}} \\
&\leq &\left[ E\left\vert u_{i\ell }\right\vert ^{7}\right] ^{\frac{{\large 9%
}}{{\large 28}}}\left[ \left( E\left\vert u_{iv}\right\vert ^{\frac{{\large %
126}}{{\large 19}}}\right) ^{\frac{{\large 1}}{{\large 2}}}\left(
E\left\vert u_{iw}\right\vert ^{\frac{{\large 126}}{{\large 19}}}\right) ^{%
\frac{{\large 1}}{{\large 2}}}\right] ^{\frac{{\large 19}}{{\large 28}}} \\
&=&\left[ E\left\vert u_{i\ell }\right\vert ^{7}\right] ^{\frac{{\large 9}}{%
{\large 28}}}\left( E\left\vert u_{iv}\right\vert ^{\frac{{\large 126}}{%
{\large 19}}}\right) ^{\frac{{\large 19}}{{\large 56}}}\left( E\left\vert
u_{iw}\right\vert ^{\frac{{\large 126}}{{\large 19}}}\right) ^{\frac{{\large %
19}}{{\large 56}}} \\
&=&\left[ E\left\vert u_{i\ell }\right\vert ^{7}\right] ^{\frac{{\large 9}}{%
{\large 28}}}\left[ \left( E\left\vert u_{iv}\right\vert ^{\frac{{\large 126}%
}{{\large 19}}}\right) ^{\frac{{\large 19}}{{\large 126}}}\left( E\left\vert
u_{iw}\right\vert ^{\frac{{\large 126}}{{\large 19}}}\right) ^{\frac{{\large %
19}}{{\large 126}}}\right] ^{\frac{{\large 9}}{{\large 4}}} \\
&\leq &\left[ E\left\vert u_{i\ell }\right\vert ^{7}\right] ^{\frac{{\large 9%
}}{{\large 28}}}\left[ \left( E\left\vert u_{iv}\right\vert ^{7}\right) ^{%
\frac{{\large 1}}{{\large 7}}}\left( E\left\vert u_{iw}\right\vert
^{7}\right) ^{\frac{{\large 1}}{{\large 7}}}\right] ^{\frac{{\large 9}}{%
{\large 4}}}\text{ }\left( \text{by Liapunov's inequality}\right) \\
&\leq &\left( \sup_{t}E\left\vert u_{it}\right\vert ^{7}\right) ^{\frac{%
{\large 27}}{{\large 28}}}\text{ } \\
&\leq &\overline{C}^{\frac{{\large 27}}{{\large 28}}}\text{ \ \ }\left( 
\text{by Assumption 3-3(b)}\right)
\end{eqnarray*}%
and%
\begin{eqnarray*}
E\left\vert \left\{ u_{it}u_{is}u_{ig}-E\left( u_{it}u_{is}u_{ig}\right)
\right\} \right\vert ^{2} &\leq &2\left( E\left\vert
u_{it}u_{is}u_{ig}\right\vert ^{2}+\left\vert E\left[ u_{it}u_{is}u_{ig}%
\right] \right\vert ^{2}\right) \\
&&\left( \text{by Lo\`{e}ve's }c_{r}\text{ inequality}\right) \\
&\leq &2\left( E\left\vert u_{it}u_{is}u_{ig}\right\vert ^{2}+E\left\vert
u_{it}u_{is}u_{ig}\right\vert ^{2}\right) \text{ } \\
&&\left( \text{by Jensen's inequality}\right) \\
&\leq &4E\left\vert u_{it}u_{is}u_{ig}\right\vert ^{2} \\
&\leq &4\left( E\left\vert u_{it}\right\vert ^{6}\right) ^{\frac{{\large 1}}{%
{\large 3}}}\left( E\left\vert u_{is}u_{ig}\right\vert ^{3}\right) ^{\frac{%
{\large 2}}{{\large 3}}} \\
&\leq &4\left( E\left\vert u_{it}\right\vert ^{6}\right) ^{\frac{{\large 1}}{%
{\large 3}}}\left( \sqrt{E\left\vert u_{is}\right\vert ^{6}}\sqrt{%
E\left\vert u_{ig}\right\vert ^{6}}\right) ^{\frac{{\large 2}}{{\large 3}}}
\\
&=&4\left[ \left( E\left\vert u_{it}\right\vert ^{6}\right) ^{\frac{{\large 1%
}}{{\large 6}}}\right] ^{2}\left[ \left( E\left\vert u_{is}\right\vert
^{6}\right) ^{\frac{{\large 1}}{{\large 6}}}\left( E\left\vert
u_{ig}\right\vert ^{6}\right) ^{\frac{{\large 1}}{{\large 6}}}\right] ^{2} \\
&\leq &4\left[ \left( E\left\vert u_{it}\right\vert ^{7}\right) ^{\frac{%
{\large 1}}{{\large 7}}}\right] ^{2}\left[ \left( E\left\vert
u_{is}\right\vert ^{7}\right) ^{\frac{{\large 1}}{{\large 7}}}\left(
E\left\vert u_{ig}\right\vert ^{7}\right) ^{\frac{{\large 1}}{{\large 7}}}%
\right] ^{2} \\
&&\left( \text{by Liapunov's inequality}\right) \\
&\leq &4\left[ \left( \sup_{t}E\left\vert u_{it}\right\vert ^{7}\right) ^{%
\frac{{\large 1}}{{\large 7}}}\right] ^{6} \\
&\leq &4\overline{C}^{\frac{{\large 6}}{{\large 7}}}\text{ }\left( \text{by
Assumption 3-3(b)}\right)
\end{eqnarray*}%
Define again $\rho _{1}=\ell -g$, $\rho _{2}=v-\ell $, and $\rho _{3}=w-v$,
so that $\ell =g+\rho _{1}$, $v=\ell +$ $\rho _{2}=g+\rho _{1}+$ $\rho _{2}$%
, $w=v+\rho _{3}=g+\rho _{1}+$ $\rho _{2}+\rho _{3}$. Using these notations
and the boundedness of $E\left\vert u_{i\ell }u_{iv}u_{iw}\right\vert ^{%
\frac{{\large 9}}{{\large 4}}}$ and $E\left\vert \left\{
u_{it}u_{is}u_{ig}-E\left( u_{it}u_{is}u_{ig}\right) \right\} \right\vert
^{2}$ as shown above, we can further write%
\begin{eqnarray}
&&\mathcal{TT}_{5}  \notag \\
&\leq &\frac{C_{1}C}{N_{1}N_{2}T_{h}^{3}}\dsum\limits_{k\in H^{{\large c}%
}}\dsum\limits_{i\in H}\dsum\limits_{\substack{ t,s,g,\ell ,v,w=p  \\ t\leq
s\leq g\leq \ell \leq v\leq w  \\ \ell -g\geq \max \left\{ w-v,v-\ell
\right\} ,\ell -g>0}}^{T-h}\left\{ 2\left( 2^{\frac{{\large 1}}{{\large 2}}%
}+1\right) \left[ a_{1}\exp \left\{ -a_{2}\left( \ell -g\right) \right\} %
\right] ^{\frac{{\large 1}}{{\large 18}}}\right.  \notag \\
&&\text{ \ \ \ \ \ \ \ \ \ \ \ \ \ \ \ \ \ \ \ \ \ \ \ \ \ \ \ \ \ \ \ \ \ \
\ \ \ \ \ \ \ \ \ \ }\left. \times \left( E\left\vert \left\{
u_{it}u_{is}u_{ig}-E\left( u_{it}u_{is}u_{ig}\right) \right\} \right\vert
^{2}\right) ^{\frac{{\large 1}}{{\large 2}}}\left( E\left\vert u_{i\ell
}u_{iv}u_{iw}\right\vert ^{\frac{{\large 9}}{{\large 4}}}\right) ^{\frac{%
{\large 4}}{{\large 9}}}\right\}  \notag \\
&\leq &\frac{C_{1}C}{N_{1}N_{2}T_{h}^{3}}\dsum\limits_{k\in H^{{\large c}%
}}\dsum\limits_{i\in H}\dsum\limits_{\substack{ t,s,g,\ell ,v,w=p  \\ t\leq
s\leq g\leq \ell \leq v\leq w  \\ \ell -g\geq \max \left\{ w-v,v-\ell
\right\} ,\ell -g>0}}^{T-h}2\left( 2^{\frac{{\large 1}}{{\large 2}}%
}+1\right) \left[ a_{1}\exp \left\{ -a_{2}\left( \ell -g\right) \right\} %
\right] ^{\frac{{\large 1}}{{\large 18}}}\left( 4\overline{C}^{\frac{{\large %
6}}{{\large 7}}}\right) ^{\frac{{\large 1}}{{\large 2}}}\left( \overline{C}^{%
\frac{{\large 27}}{{\large 28}}}\right) ^{\frac{{\large 4}}{{\large 9}}} 
\notag \\
&\leq &\frac{C_{3}^{\ast }}{N_{1}N_{2}T_{h}^{3}}\dsum\limits_{k\in H^{%
{\large c}}}\dsum\limits_{i\in H}\dsum\limits_{\substack{ t,s,g,\ell ,v,w=p 
\\ t\leq s\leq g\leq \ell \leq v\leq w  \\ \ell -g\geq \max \left\{
w-v,v-\ell \right\} ,\ell -g>0}}^{T-h}\exp \left\{ -\frac{a_{2}}{18}\varrho
_{1}\right\}  \notag \\
&&\left( \text{for some constant }C_{3}^{\ast }\text{ such that }4\left( 2^{%
\frac{{\large 1}}{{\large 2}}}+1\right) C_{1}C\overline{C}^{\frac{{\large 6}%
}{{\large 7}}}a_{1}^{\frac{{\large 1}}{{\large 18}}}\leq C_{3}^{\ast
}<\infty \right)  \notag \\
&\leq &\frac{C_{3}^{\ast }}{N_{1}N_{2}T_{h}^{3}}\dsum\limits_{k\in H^{%
{\large c}}}\dsum\limits_{i\in H}\dsum\limits_{\substack{ t=p}}%
^{T-h}\dsum\limits_{\substack{ s=p}}^{T-h}\dsum\limits_{\substack{ g=p}}%
^{T-h}\dsum\limits_{\varrho _{{\large 1}}=1}^{\infty }\dsum\limits_{\varrho
_{{\large 2}}=0}^{\varrho _{{\large 1}}}\dsum\limits_{\varrho _{{\large 3}%
}=0}^{\varrho _{{\large 1}}}\exp \left\{ -\frac{a_{2}}{18}\varrho
_{1}\right\}  \notag \\
&\leq &\frac{C_{3}^{\ast }N_{1}N_{2}T_{h}^{3}}{N_{1}N_{2}T_{h}^{3}}%
\dsum\limits_{\varrho _{{\large 1}}=1}^{\infty }\left( \varrho _{{\large 1}%
}+1\right) ^{2}\exp \left\{ -\frac{a_{2}}{18}\varrho _{1}\right\}  \notag \\
&\leq &C_{3}^{\ast }\left[ \dsum\limits_{\varrho _{{\large 1}}=1}^{\infty
}\varrho _{1}^{2}\exp \left\{ -\frac{a_{2}}{18}\varrho _{1}\right\}
+2\dsum\limits_{\varrho _{{\large 1}}=1}^{\infty }\varrho _{1}\exp \left\{ -%
\frac{a_{2}}{18}\varrho _{1}\right\} +\dsum\limits_{\varrho _{{\large 1}%
}=1}^{\infty }\exp \left\{ -\frac{a_{2}}{18}\varrho _{1}\right\} \right] 
\notag \\
&\leq &C_{3}^{\ast }\overline{\overline{C}}_{3}  \label{TT5}
\end{eqnarray}%
for some positive constant 
\begin{equation*}
\overline{\overline{C}}_{3}\geq \dsum\limits_{\varrho _{{\large 1}%
}=1}^{\infty }\varrho _{1}^{2}\exp \left\{ -\frac{a_{2}}{18}\varrho
_{1}\right\} +2\dsum\limits_{\varrho _{{\large 1}}=1}^{\infty }\varrho
_{1}\exp \left\{ -\frac{a_{2}}{18}\varrho _{1}\right\}
+\dsum\limits_{\varrho _{{\large 1}}=1}^{\infty }\exp \left\{ -\frac{a_{2}}{%
18}\varrho _{1}\right\}
\end{equation*}%
which exists in light of Lemma C-1.

Finally, consider $\mathcal{TT}_{6}$. Note that, from the upper bounds given
in the proofs of part (b) of Lemma D-5, it is clear that there exists a
positive constant $C_{2}^{\ast \ast }$ such that, for all $i$ and for all $T$
sufficiently large,%
\begin{equation*}
\frac{1}{T_{h}}\dsum\limits_{\substack{ t,s,g=p  \\ t\leq s\leq g}}%
^{T-h}\left\vert E\left( u_{it}u_{is}u_{ig}\right) \right\vert \leq
C_{2}^{\ast \ast }
\end{equation*}%
and%
\begin{equation*}
\frac{1}{T_{h}}\dsum\limits_{\substack{ \ell ,v,w=p  \\ \ell \leq v\leq w}}%
^{T-h}\left\vert E\left( u_{i\ell }u_{iv}u_{iw}\right) \right\vert \leq
C_{2}^{\ast \ast }
\end{equation*}%
from which it follows that%
\begin{eqnarray}
\mathcal{TT}_{6} &=&\frac{C_{1}C}{N_{1}N_{2}T_{h}^{3}}\dsum\limits_{k\in H^{%
{\large c}}}\dsum\limits_{i\in H}\dsum\limits_{\substack{ t,s,g,\ell ,v,w=p 
\\ t\leq s\leq g\leq \ell \leq v\leq w  \\ \ell -g\geq \max \left\{
w-v,v-\ell \right\} ,\ell -g>0}}^{T-h}\left\vert E\left(
u_{it}u_{is}u_{ig}\right) \right\vert \left\vert E\left( u_{i\ell
}u_{iv}u_{iw}\right) \right\vert  \notag \\
&\leq &\frac{C_{1}C}{N_{1}N_{2}T_{h}}\dsum\limits_{k\in H^{{\large c}%
}}\dsum\limits_{i\in H}\left( \frac{1}{T_{h}}\dsum\limits_{\substack{ %
t,s,g=p  \\ t\leq s\leq g}}^{T-h}\left\vert E\left(
u_{it}u_{is}u_{ig}\right) \right\vert \right) \left( \frac{1}{T_{h}}%
\dsum\limits_{\substack{ \ell ,v,w=p  \\ \ell \leq v\leq w}}^{T-h}\left\vert
E\left( u_{i\ell }u_{iv}u_{iw}\right) \right\vert \right)  \notag \\
&\leq &\frac{C_{1}C}{N_{1}N_{2}T_{h}}\dsum\limits_{k\in H^{{\large c}%
}}\dsum\limits_{i\in H}\left( C^{\ast \ast }\right) ^{2}  \notag \\
&=&C_{1}C\left( C_{2}^{\ast \ast }\right) ^{2}\frac{N_{1}N_{2}}{%
N_{1}N_{2}T_{h}}  \notag \\
&=&\frac{C_{1}C\left( C_{2}^{\ast \ast }\right) ^{2}}{T_{h}}=O\left( \frac{1%
}{T}\right) \text{. }  \label{TT6}
\end{eqnarray}

It follows from expressions (\ref{TT1})-(\ref{TT6}) that, for all $%
N_{1},N_{2}$, and $T$ sufficiently large,

\begin{eqnarray*}
&&\frac{1}{N_{1}}\dsum\limits_{k\in H^{{\large c}}}\dsum\limits_{i\in H}%
\frac{1}{N_{1}^{3}T_{h}^{3}}E\left( \dsum\limits_{t=p}^{T-h}\gamma
_{k}^{\prime }\underline{F}_{t}u_{i,t}\right) ^{6} \\
&\leq &\mathcal{TT}_{1}+\mathcal{TT}_{2}+\mathcal{TT}_{3}+\mathcal{TT}_{4}+%
\mathcal{TT}_{5}+\mathcal{TT}_{6} \\
&\leq &C_{1}C\overline{C}^{\frac{{\large 6}}{{\large 7}}}+C_{1}^{\ast }%
\overline{\overline{C}}_{1}+C_{2}^{\ast }\overline{\overline{C}}%
_{2}+C_{1}C\left( C_{1}^{\ast \ast }\right) ^{2}+C_{3}^{\ast }\overline{%
\overline{C}}_{3}+\frac{C_{1}C\left( C_{2}^{\ast \ast }\right) ^{2}}{T_{h}}
\\
&\leq &\widetilde{C}
\end{eqnarray*}%
for some positive constant $\widetilde{C}$ such that%
\begin{equation*}
\widetilde{C}\geq C_{1}C\overline{C}^{\frac{{\large 6}}{{\large 7}}%
}+C_{1}^{\ast }\overline{\overline{C}}_{1}+C_{2}^{\ast }\overline{\overline{C%
}}_{2}+C_{1}C\left( C_{1}^{\ast \ast }\right) ^{2}+C_{3}^{\ast }\overline{%
\overline{C}}_{3}+\frac{C_{1}C\left( C_{2}^{\ast \ast }\right) ^{2}}{T_{h}}%
\text{.}
\end{equation*}%
Hence, for any $\epsilon >0$, set $C_{\epsilon }=\left( \widetilde{C}%
/\epsilon \right) ^{\frac{{\large 1}}{{\large 3}}}$, and note that 
\begin{eqnarray*}
&&\Pr \left\{ \frac{N_{1}T_{h}}{N_{2}^{\frac{{\large 1}}{{\large 3}}}}%
\max_{i\in H}\frac{1}{N_{1}}\dsum\limits_{k\in H^{{\large c}}}\left( \frac{%
\gamma _{k}^{\prime }\underline{F}^{\prime }u_{i\cdot }}{\sqrt{N_{1}}T_{h}}%
\right) ^{2}\geq C_{\epsilon }\right\} \\
&=&\Pr \left\{ \max_{i\in H}\frac{1}{N_{1}}\dsum\limits_{k\in H^{{\large c}%
}}\left( \frac{1}{N_{2}^{\frac{{\large 1}}{{\large 6}}}\sqrt{T_{h}}}%
\dsum\limits_{t=p}^{T-h}\gamma _{k}^{\prime }\underline{F}_{t}u_{i,t}\right)
^{2}\geq C_{\epsilon }\right\} \\
&=&\Pr \left\{ \max_{i\in H}\left[ \frac{1}{N_{1}}\dsum\limits_{k\in H^{%
{\large c}}}\left( \frac{1}{N_{2}^{\frac{{\large 1}}{{\large 6}}}\sqrt{T_{h}}%
}\dsum\limits_{t=p}^{T-h}\gamma _{k}^{\prime }\underline{F}%
_{t}u_{i,t}\right) ^{2}\right] ^{3}\geq C_{\epsilon }^{3}\right\} \\
&\leq &\Pr \left\{ \max_{i\in H}\frac{1}{N_{1}}\dsum\limits_{k\in H^{{\large %
c}}}\left( \frac{1}{N_{2}^{\frac{{\large 1}}{{\large 6}}}\sqrt{T_{h}}}%
\dsum\limits_{t=p}^{T-h}\gamma _{k}^{\prime }\underline{F}_{t}u_{i,t}\right)
^{6}\geq C_{\epsilon }^{3}\right\} \text{ (by Jensen's inequality)} \\
&\leq &\Pr \left\{ \frac{1}{N_{1}}\dsum\limits_{k\in H^{{\large c}%
}}\dsum\limits_{i\in H}\left( \frac{1}{N_{2}^{\frac{{\large 1}}{{\large 6}}}%
\sqrt{T_{h}}}\dsum\limits_{t=p}^{T-h}\gamma _{k}^{\prime }\underline{F}%
_{t}u_{i,t}\right) ^{6}\geq C_{\epsilon }^{3}\right\} \\
&\leq &\frac{\epsilon }{\overline{C}}\frac{1}{N_{1}}\dsum\limits_{k\in H^{%
{\large c}}}\dsum\limits_{i\in H}\frac{1}{N_{2}T_{h}^{3}}E\left(
\dsum\limits_{t=p}^{T-h}\gamma _{k}^{\prime }\underline{F}_{t}u_{i,t}\right)
^{6} \\
&\leq &\frac{\epsilon }{\widetilde{C}}\widetilde{C} \\
&=&\epsilon
\end{eqnarray*}%
This shows that%
\begin{equation*}
\max_{i\in H}\frac{1}{N_{1}}\dsum\limits_{k\in H^{{\large c}}}\left( \frac{%
\gamma _{k}^{\prime }\underline{F}^{\prime }u_{i\cdot }}{\sqrt{N_{1}}T_{h}}%
\right) ^{2}=O_{p}\left( \frac{N_{2}^{\frac{{\large 1}}{{\large 3}}}}{%
N_{1}T_{h}}\right) =O_{p}\left( \frac{N_{2}^{\frac{{\large 1}}{{\large 3}}}}{%
N_{1}T}\right) \text{. }\square
\end{equation*}

\medskip

Before stating the next lemma, we first introduce some more notations. Let $%
\mathbb{S}_{i,T}^{+}$ denote either the statistic $\dsum\nolimits_{\ell
=1}^{d}\varpi _{\ell }\left\vert S_{i,\ell ,T}\right\vert $ or the statistic 
$\max_{1\leq \ell \leq d}\left\vert S_{i,\ell ,T}\right\vert $, and define%
\begin{eqnarray}
\widehat{H^{c}} &=&\left\{ i\in \left\{ 1,....,N\right\} :\mathbb{S}%
_{i,T}^{+}\geq \Phi ^{-1}\left( 1-\frac{\varphi }{2N}\right) \right\} \text{,%
}  \label{Hchat} \\
\widehat{H} &=&\left\{ i\in \left\{ 1,....,N\right\} :\mathbb{S}%
_{i,T}^{+}<\Phi ^{-1}\left( 1-\frac{\varphi }{2N}\right) \right\} \text{,}
\label{Hhat} \\
\widehat{N}_{1} &=&\#\left( \widehat{H^{c}}\right) \text{, i.e., the
cardinality of the set }\widehat{H^{c}}\text{,}  \label{N1hat} \\
\Gamma \left( \widehat{H^{c}}\right) &=&\left( 
\begin{array}{c}
\gamma _{1}\left( \widehat{H^{c}}\right) ^{\prime } \\ 
\gamma _{2}\left( \widehat{H^{c}}\right) ^{\prime } \\ 
\vdots \\ 
\gamma _{N}\left( \widehat{H^{c}}\right) ^{\prime }%
\end{array}%
\right) =\left( 
\begin{array}{c}
\mathbb{I}\left\{ 1\in \widehat{H^{c}}\right\} \gamma _{1}^{\prime } \\ 
\mathbb{I}\left\{ 2\in \widehat{H^{c}}\right\} \gamma _{2}^{\prime } \\ 
\vdots \\ 
\mathbb{I}\left\{ N\in \widehat{H^{c}}\right\} \gamma _{N}^{\prime }%
\end{array}%
\right) \text{, and}  \notag \\
U\left( \widehat{H^{c}}\right) &=&\left( 
\begin{array}{c}
u_{1\cdot }\left( \widehat{H^{c}}\right) ^{\prime } \\ 
u_{2\cdot }\left( \widehat{H^{c}}\right) ^{\prime } \\ 
\vdots \\ 
u_{N\cdot }\left( \widehat{H^{c}}\right) ^{\prime }%
\end{array}%
\right) =\left( 
\begin{array}{c}
\mathbb{I}\left\{ 1\in \widehat{H^{c}}\right\} u_{1\cdot }^{\prime } \\ 
\mathbb{I}\left\{ 2\in \widehat{H^{c}}\right\} u_{2\cdot }^{\prime } \\ 
\vdots \\ 
\mathbb{I}\left\{ N\in \widehat{H^{c}}\right\} u_{N\cdot }^{\prime }%
\end{array}%
\right) \text{,}  \label{variable selection notations}
\end{eqnarray}%
where $u_{i\cdot }=\left( u_{i,p},u_{i,p+1},...,u_{i,T-h}\right) ^{\prime }$
for $i=1,...,N$.

\medskip

\noindent \textbf{Lemma D-7: } Let $T_{h}=T-h-p+1$ where $h$ is a (fixed)
non-negative integer and $p$ is a (fixed) positive integer. Suppose that
Assumptions 3-1, 3-2(a)-(c), 3-3(a)-(c), 3-4, 3-5, 3-7, 3-8, 3-10(a) and
3-11 hold. Then, as $N_{1},N_{2},T\rightarrow \infty $, the following
statements are true.

\begin{enumerate}
\item[(a)] 
\begin{equation*}
\dsum\limits_{i\in H}\mathbb{I}\left\{ i\in \widehat{H^{c}}\right\}
=O_{p}\left( \varphi \right)
\end{equation*}

\item[(b)] 
\begin{equation*}
\dsum\limits_{i\in H}\mathbb{I}\left\{ i\in \widehat{H^{c}}\right\} \frac{1}{%
N_{1}}\dsum\limits_{k\in H^{{\large c}}}\left( \frac{\gamma _{k}^{\prime }%
\underline{F}^{\prime }u_{i\cdot }}{\sqrt{N_{1}}T_{h}}\right)
^{2}=O_{p}\left( \frac{N_{2}^{\frac{{\large 1}}{{\large 3}}}\varphi }{N_{1}T}%
\right) .
\end{equation*}

\item[(c)] 
\begin{equation*}
\frac{1}{N_{1}}\dsum\limits_{i\in H^{{\large c}}}\mathbb{I}\left\{ i\in 
\widehat{H^{c}}\right\} \dsum\limits_{k\in H^{{\large c}}}\left( \frac{%
\gamma _{k}^{\prime }\underline{F}^{\prime }u_{i\cdot }}{\sqrt{N_{1}}T_{h}}%
\right) ^{2}=O_{p}\left( \frac{1}{T}\right)
\end{equation*}
\end{enumerate}

\medskip

\noindent \textbf{Proof of Lemma D-7: }

To show part (a), let $\mathbb{S}_{i,T}^{+}$ denote either the statistic $%
\dsum\nolimits_{\ell =1}^{d}\varpi _{\ell }\left\vert S_{i,\ell
,T}\right\vert $ or the statistic $\max_{1\leq \ell \leq d}\left\vert
S_{i,\ell ,T}\right\vert $. Following arguments similar to that given in the
proof of part (a) of Theorem 1 in Chao and Swanson (2022a), we see that
there exists a constant $C>2d$ such that%
\begin{eqnarray*}
\dsum\limits_{i\in H}E\left[ \mathbb{I}\left\{ i\in \widehat{H^{c}}\right\} %
\right] &=&\dsum\limits_{i\in H}\Pr \left( i\in \widehat{H^{c}}\right) \\
&=&\dsum\limits_{i\in H}\Pr \left\{ \mathbb{S}_{i,T}^{+}\geq \Phi
^{-1}\left( 1-\frac{\varphi }{2N}\right) \right\} \\
&\leq &C\frac{N_{2}\varphi }{N} \\
&\leq &C\varphi
\end{eqnarray*}%
for all $N_{1},N_{2}$, and $T$ sufficiently large. Hence, for any $\epsilon
>0$, set $C_{\epsilon }=C/\epsilon $, and note that 
\begin{eqnarray*}
\Pr \left\{ \frac{1}{\varphi }\dsum\limits_{i\in H}\mathbb{I}\left\{ i\in 
\widehat{H^{c}}\right\} \geq C_{\epsilon }\right\} &\leq &\frac{1}{%
C_{\epsilon }\varphi }\dsum\limits_{i\in H}E\left[ \mathbb{I}\left\{ i\in 
\widehat{H^{c}}\right\} \right] \text{ \ (by Markov's inequality)} \\
&\leq &\frac{\epsilon }{C\varphi }C\varphi \\
&=&\epsilon
\end{eqnarray*}%
which shows that%
\begin{equation*}
\dsum\limits_{i\in H}\mathbb{I}\left\{ i\in \widehat{H^{c}}\right\}
=O_{p}\left( \varphi \right)
\end{equation*}

Next, to show part (b), we combine the result given in part (a) of this
lemma with the result of Lemma D-6 to obtain%
\begin{eqnarray*}
&&\dsum\limits_{i\in H}\mathbb{I}\left\{ i\in \widehat{H^{c}}\right\} \frac{1%
}{N_{1}}\dsum\limits_{k\in H^{{\large c}}}\left( \frac{\gamma _{k}^{\prime }%
\underline{F}^{\prime }u_{i\cdot }}{\sqrt{N_{1}}T_{h}}\right) ^{2} \\
&\leq &\max_{i\in H}\frac{1}{N_{1}}\dsum\limits_{k\in H^{{\large c}}}\left( 
\frac{\gamma _{k}^{\prime }\underline{F}^{\prime }u_{i\cdot }}{\sqrt{N_{1}}%
T_{h}}\right) ^{2}\left[ \dsum\limits_{i\in H}\mathbb{I}\left\{ i\in 
\widehat{H^{c}}\right\} \right] \text{ (by H\"{o}lder's inequality)} \\
&=&O_{p}\left( \frac{N_{2}^{\frac{{\large 1}}{{\large 3}}}}{N_{1}T}\right)
O_{p}\left( \varphi \right) \\
&=&O_{p}\left( \frac{N_{2}^{\frac{{\large 1}}{{\large 3}}}\varphi }{N_{1}T}%
\right) \text{. }
\end{eqnarray*}

Finally, to show part (c), note first that 
\begin{equation*}
\frac{1}{N_{1}}\dsum\limits_{i\in H^{{\large c}}}\mathbb{I}\left\{ i\in 
\widehat{H^{c}}\right\} \dsum\limits_{k\in H^{{\large c}}}\left( \frac{%
\gamma _{k}^{\prime }\underline{F}^{\prime }u_{i\cdot }}{\sqrt{N_{1}}T_{h}}%
\right) ^{2}\leq \frac{1}{N_{1}^{2}}\dsum\limits_{i\in H^{{\large c}%
}}\dsum\limits_{k\in H^{{\large c}}}\left( \frac{\gamma _{k}^{\prime }%
\underline{F}^{\prime }u_{i\cdot }}{T_{h}}\right) ^{2}
\end{equation*}%
Moreover, write%
\begin{eqnarray*}
0 &\leq &\frac{1}{N_{1}^{2}}\dsum\limits_{i\in H^{{\large c}%
}}\dsum\limits_{k\in H^{{\large c}}}E\left( \frac{\gamma _{k}^{\prime }%
\underline{F}^{\prime }u_{i\cdot }}{T_{h}}\right) ^{2} \\
&=&\frac{1}{N_{1}^{2}}\dsum\limits_{i\in H^{{\large c}}}\dsum\limits_{k\in
H^{{\large c}}}E\left( \frac{1}{T_{h}}\dsum\limits_{t=p}^{T-h}\gamma
_{k}^{\prime }\underline{F}_{t}u_{i,t}\right) ^{2} \\
&=&\frac{1}{N_{1}^{2}T_{h}^{2}}\dsum\limits_{i\in H^{{\large c}%
}}\dsum\limits_{k\in H^{{\large c}}}E\left( \dsum\limits_{t=p}^{T-h}\gamma
_{k}^{\prime }\underline{F}_{t}u_{i,t}\right) ^{2} \\
&=&\frac{1}{N_{1}^{2}T_{h}^{2}}\dsum\limits_{i\in H^{{\large c}%
}}\dsum\limits_{k\in H^{{\large c}}}\dsum\limits_{t=p}^{T-h}\dsum%
\limits_{s=p}^{T-h}E\left\{ \gamma _{k}^{\prime }\underline{F}%
_{s}u_{i,s}u_{i,t}\underline{F}_{t}^{\prime }\gamma _{k}\right\} \\
&=&\frac{1}{N_{1}^{2}T_{h}^{2}}\dsum\limits_{i\in H^{{\large c}%
}}\dsum\limits_{k\in H^{{\large c}}}\dsum\limits_{t=p}^{T-h}\gamma
_{k}^{\prime }E_{F}\left[ \underline{F}_{t}E\left( u_{i,t}^{2}\right) 
\underline{F}_{t}^{\prime }\right] \gamma _{k} \\
&&+\frac{2}{N_{1}^{2}T_{h}^{2}}\dsum\limits_{i\in H^{{\large c}%
}}\dsum\limits_{k\in H^{{\large c}}}\dsum\limits_{t=p}^{T-h-1}\dsum%
\limits_{m=1}^{T-h-t}E_{F}\left[ \gamma _{k}^{\prime }\underline{F}%
_{t}E\left( u_{i,t}u_{i,t{\LARGE +}m}\right) \underline{F}_{t{\LARGE +}%
m}^{\prime }\gamma _{k}\right] \\
&=&\frac{1}{N_{1}^{2}T_{h}^{2}}\dsum\limits_{i\in H^{{\large c}%
}}\dsum\limits_{k\in H^{{\large c}}}\dsum\limits_{t=p}^{T-h}\gamma
_{k}^{\prime }E_{F}\left[ \underline{F}_{t}E\left( u_{i,t}^{2}\right) 
\underline{F}_{t}^{\prime }\right] \gamma _{k} \\
&&+\frac{2}{N_{1}^{2}T_{h}^{2}}\dsum\limits_{i\in H^{{\large c}%
}}\dsum\limits_{k\in H^{{\large c}}}\dsum\limits_{t=p}^{T-h-1}\dsum%
\limits_{m=1}^{T-h-t}E\left( u_{i,t}u_{i,t{\LARGE +}m}\right) E_{F}\left[
\gamma _{k}^{\prime }\underline{F}_{t}\underline{F}_{t{\LARGE +}m}^{\prime
}\gamma _{k}\right] \\
&\leq &\frac{1}{N_{1}^{2}T_{h}^{2}}\dsum\limits_{i\in H^{{\large c}%
}}\dsum\limits_{k\in H^{{\large c}}}\dsum\limits_{t=p}^{T-h}\gamma
_{k}^{\prime }E_{F}\left[ \underline{F}_{t}E\left( u_{i,t}^{2}\right) 
\underline{F}_{t}^{\prime }\right] \gamma _{k} \\
&&+\frac{2}{N_{1}^{2}T_{h}^{2}}\dsum\limits_{i\in H^{{\large c}%
}}\dsum\limits_{k\in H^{{\large c}}}\dsum\limits_{t=p}^{T-h-1}\dsum%
\limits_{m=1}^{T-h-t}\left\vert E\left( u_{i,t}u_{i,t{\LARGE +}m}\right)
\right\vert \left\vert \gamma _{k}^{\prime }E_{F}\left[ \underline{F}_{t}%
\underline{F}_{t{\LARGE +}m}^{\prime }\right] \gamma _{k}\right\vert
\end{eqnarray*}%
Note that by Assumption 3-3(c), $\left\{ u_{i,t}\right\} _{t=-\infty
}^{\infty }$ is $\beta $-mixing with $\beta $ mixing coefficient satisfying 
\begin{equation*}
\beta _{i}\left( m\right) \leq a_{1}\exp \left\{ -a_{2}m\right\}
\end{equation*}%
for every $i$. Since $\alpha _{i,m}\leq \beta _{i}\left( m\right) $, it
follows that $\left\{ u_{it}\right\} _{t=-\infty }^{\infty }$ is $\alpha $%
-mixing as well, with $\alpha $ mixing coefficient satisfying%
\begin{equation*}
\alpha _{i,m}\leq a_{1}\exp \left\{ -a_{2}m\right\} \text{ for every }i\text{%
.}
\end{equation*}

\noindent Hence, applying Lemma C-3 with $p=3$ and $r=3$ as well as
Assumptions 3-3(b) and 3-5 and Lemma C-5; we get%
\begin{eqnarray*}
&&\frac{1}{N_{1}^{2}T_{h}^{2}}\dsum\limits_{i\in H^{{\large c}%
}}\dsum\limits_{k\in H^{{\large c}}}E\left[ \left(
\dsum\limits_{t=p}^{T-h}\gamma _{k}^{\prime }\underline{F}_{t}u_{i,t}\right)
^{2}\right] \\
&\leq &\frac{1}{N_{1}^{2}T_{h}^{2}}\dsum\limits_{i\in H^{{\large c}%
}}\dsum\limits_{k\in H^{{\large c}}}\dsum\limits_{t=p}^{T-h}\gamma
_{k}^{\prime }E_{F}\left[ \underline{F}_{t}E\left( u_{i,t}^{2}\right) 
\underline{F}_{t}^{\prime }\right] \gamma _{k} \\
&&+\frac{2}{N_{1}^{2}T_{h}^{2}}\dsum\limits_{i\in H^{{\large c}%
}}\dsum\limits_{k\in H^{{\large c}}}\dsum\limits_{t=p}^{T-h-1}\dsum%
\limits_{m=1}^{T-h-t}\left\vert E\left( u_{i,t}u_{i,t{\LARGE +}m}\right)
\right\vert \left\vert \gamma _{k}^{\prime }E_{F}\left[ \underline{F}_{t}%
\underline{F}_{t{\LARGE +}m}^{\prime }\right] \gamma _{k}\right\vert \\
&\leq &\frac{1}{N_{1}^{2}T_{h}^{2}}\dsum\limits_{i\in H^{{\large c}%
}}\dsum\limits_{k\in H^{{\large c}}}\dsum\limits_{t=p}^{T-h}\gamma
_{k}^{\prime }E_{F}\left[ \underline{F}_{t}E\left( u_{i,t}^{2}\right) 
\underline{F}_{t}^{\prime }\right] \gamma _{k} \\
&&+\frac{2}{N_{1}^{2}T_{h}^{2}}\dsum\limits_{i\in H^{{\large c}%
}}\dsum\limits_{k\in H^{{\large c}}}\dsum\limits_{t=p}^{T-h-1}\dsum%
\limits_{m=1}^{T-h-t}\left\vert E\left( u_{i,t}u_{i,t{\LARGE +}m}\right)
\right\vert E\left\vert \gamma _{k}^{\prime }\underline{F}_{t}\underline{F}%
_{t{\LARGE +}m}^{\prime }\gamma _{k}\right\vert \\
&\leq &\frac{1}{N_{1}^{2}T_{h}^{2}}\dsum\limits_{i\in H^{{\large c}%
}}\dsum\limits_{k\in H^{{\large c}}}\dsum\limits_{t=p}^{T-h}E\left(
u_{i,t}^{2}\right) \left\Vert \gamma _{k}\right\Vert _{2}^{2}E\left[
\left\Vert \underline{F}_{t}\right\Vert _{2}^{2}\right] \\
&&+\frac{1}{N_{1}^{2}T_{h}^{2}}\dsum\limits_{i\in H^{{\large c}%
}}\dsum\limits_{k\in H^{{\large c}}}2\left( 2^{\frac{{\large 2}}{{\large 3}}%
}+1\right) 2\dsum\limits_{t=p}^{T-h-1}\dsum\limits_{m=1}^{T-h-t}\left\{
\alpha _{m}^{\frac{{\large 1}}{{\large 3}}}\left( E\left\vert
u_{i,t}\right\vert ^{3}\right) ^{\frac{{\large 1}}{{\large 3}}}\left(
E\left\vert u_{i,t{\LARGE +}m}\right\vert ^{3}\right) ^{\frac{{\large 1}}{%
{\large 3}}}\right. \\
&&\text{ \ \ \ \ \ \ \ \ \ \ \ \ \ \ \ \ \ \ \ \ \ \ \ \ \ \ \ \ \ \ \ \ \ \
\ \ \ \ \ \ \ \ \ \ \ \ \ \ \ \ \ }\left. \times \sqrt{\gamma _{k}^{\prime }E%
\left[ \underline{F}_{t}\underline{F}_{t}^{\prime }\right] \gamma _{k}}\sqrt{%
\gamma _{k}^{\prime }E\left[ \underline{F}_{t{\LARGE +}m}\underline{F}_{t%
{\LARGE +}m}^{\prime }\right] \gamma _{k}}\right\} \\
&\leq &\frac{1}{N_{1}^{2}T_{h}^{2}}\dsum\limits_{i\in H^{{\large c}%
}}\dsum\limits_{k\in H^{{\large c}}}\dsum\limits_{t=p}^{T-h}E\left(
u_{i,t}^{2}\right) \left\Vert \gamma _{k}\right\Vert _{2}^{2}E\left[
\left\Vert \underline{F}_{t}\right\Vert _{2}^{2}\right] \\
&&+\frac{1}{N_{1}^{2}T_{h}^{2}}\dsum\limits_{i\in H^{{\large c}%
}}\dsum\limits_{k\in H^{{\large c}}}4\left( 2^{\frac{{\large 2}}{{\large 3}}%
}+1\right) \dsum\limits_{t=p}^{T-h-1}\dsum\limits_{m=1}^{T-h-t}\left\{
\alpha _{m}^{\frac{{\large 1}}{{\large 3}}}\left( E\left\vert
u_{i,t}\right\vert ^{3}\right) ^{\frac{{\large 1}}{{\large 3}}}\left(
E\left\vert u_{i,t{\LARGE +}m}\right\vert ^{3}\right) ^{\frac{{\large 1}}{%
{\large 3}}}\right. \\
&&\text{ \ \ \ \ \ \ \ \ \ \ \ \ \ \ \ \ \ \ \ \ \ \ \ \ \ \ \ \ \ \ \ \ \ \
\ \ \ \ \ \ \ \ \ \ \ \ \ \ \ \ \ \ \ \ \ \ \ \ \ \ \ \ \ }\left. \times
\left\Vert \gamma _{k}\right\Vert _{2}^{2}\sqrt{E\left\Vert \underline{F}%
_{t}\right\Vert _{2}^{2}}\sqrt{E\left\Vert \underline{F}_{t{\LARGE +}%
m}\right\Vert _{2}^{2}}\right\} \\
&\leq &\frac{C_{1}}{T_{h}}+C_{2}\frac{1}{T_{h}^{2}}\dsum%
\limits_{t=p}^{T-h-1}\dsum\limits_{m=1}^{T-h-t}a_{1}^{\frac{{\large 1}}{%
{\large 3}}}\exp \left\{ -\frac{a_{2}}{3}m\right\} \\
&\leq &\frac{C_{1}}{T_{h}}+C_{2}a_{1}^{\frac{{\large 1}}{{\large 3}}}\frac{1%
}{T_{h}}\dsum\limits_{m=1}^{\infty }\exp \left\{ -\frac{a_{2}}{3}m\right\} \\
&\leq &\frac{\overline{C}}{T_{h}}
\end{eqnarray*}%
for some positive constant 
\begin{equation*}
\overline{C}\geq C_{1}+C_{2}a_{1}^{\frac{{\large 1}}{{\large 3}}%
}\dsum\limits_{m=1}^{\infty }\exp \left\{ -\frac{a_{2}}{3}m\right\}
\end{equation*}%
which exists in light of Lemma C-1. Hence, for any $\epsilon >0$, set $%
C_{\epsilon }=\overline{C}/\epsilon $, and note that%
\begin{eqnarray*}
&&\Pr \left\{ \frac{T_{h}}{N_{1}}\dsum\limits_{i\in H^{{\large c}}}\mathbb{I}%
\left\{ i\in \widehat{H^{c}}\right\} \dsum\limits_{k\in H^{{\large c}%
}}\left( \frac{\gamma _{k}^{\prime }\underline{F}^{\prime }u_{i\cdot }}{%
\sqrt{N_{1}}T_{h}}\right) ^{2}\geq C_{\epsilon }\right\} \\
&\leq &\Pr \left\{ \frac{T_{h}}{N_{1}^{2}}\dsum\limits_{i\in H^{{\large c}%
}}\dsum\limits_{k\in H^{{\large c}}}\left( \frac{\gamma _{k}^{\prime }%
\underline{F}^{\prime }u_{i\cdot }}{T_{h}}\right) ^{2}\geq C_{\epsilon
}\right\} \\
&\leq &\frac{T_{h}}{C_{\epsilon }}\frac{1}{N_{1}^{2}T_{h}^{2}}%
\dsum\limits_{i\in H^{{\large c}}}\dsum\limits_{k\in H^{{\large c}}}E\left[
\left( \dsum\limits_{t=p}^{T-h}\gamma _{k}^{\prime }\underline{F}%
_{t}u_{i,t}\right) ^{2}\right] \\
&\leq &\frac{\epsilon }{\overline{C}}T_{h}\frac{\overline{C}}{T_{h}} \\
&=&\epsilon
\end{eqnarray*}%
which shows that%
\begin{equation*}
\frac{1}{N_{1}}\dsum\limits_{i\in H^{{\large c}}}\mathbb{I}\left\{ i\in 
\widehat{H^{c}}\right\} \dsum\limits_{k\in H^{{\large c}}}\left( \frac{%
\gamma _{k}^{\prime }\underline{F}^{\prime }u_{i\cdot }}{\sqrt{N_{1}}T_{h}}%
\right) ^{2}=O_{p}\left( \frac{1}{T_{h}}\right) =O_{p}\left( \frac{1}{T}%
\right) \text{. }\square
\end{equation*}

\bigskip

\noindent \textbf{Lemma D-8: }Let $T_{h}=T-h-p+1$ where $h$ is a (fixed)
non-negative integer and $p$ is a (fixed) positive integer. Suppose that
Assumptions 3-1, 3-2(a)-(c), 3-3(a)-(c), 3-4, 3-5, 3-7, 3-8, 3-10(a) and
3-11* hold. Then, the following statements are true.

\begin{enumerate}
\item[(a)] 
\begin{equation*}
\frac{1}{N_{1}}\dsum\limits_{i\in H^{{\large c}}}\mathbb{I}\left\{ i\in 
\widehat{H^{c}}\right\} \left( \frac{u_{i\cdot }^{\prime }u_{i\cdot }}{T_{h}}%
\right) =O_{p}\left( 1\right) \text{. }
\end{equation*}

\item[(b)] 
\begin{equation*}
\frac{1}{N_{1}}\dsum\limits_{i\in H}\mathbb{I}\left\{ i\in \widehat{H^{c}}%
\right\} \left( \frac{u_{i\cdot }^{\prime }u_{i\cdot }}{T_{h}}\right)
=O_{p}\left( \frac{N^{\frac{{\large 2}}{{\large 7}}}\varphi ^{\frac{{\large 5%
}}{{\large 7}}}}{N_{1}}\right) =o_{p}\left( 1\right) \text{.}
\end{equation*}
\end{enumerate}

\medskip

\noindent \textbf{Proof of Lemma D-8:}

To show part (a), note first that%
\begin{eqnarray*}
\frac{1}{N_{1}}\dsum\limits_{i\in H^{{\large c}}}E\left[ \mathbb{I}\left\{
i\in \widehat{H^{c}}\right\} \left( \frac{u_{i\cdot }^{\prime }u_{i\cdot }}{%
T_{h}}\right) \right] &\leq &\frac{1}{N_{1}}\dsum\limits_{i\in H^{{\large c}%
}}E\left[ \left( \frac{u_{i\cdot }^{\prime }u_{i\cdot }}{T_{h}}\right) %
\right] \\
&=&\frac{1}{N_{1}}\dsum\limits_{i\in H^{{\large c}}}E\left[ \frac{1}{T_{h}}%
\dsum\limits_{t=p}^{T-h}u_{i,t}^{2}\right] \\
&\leq &\frac{1}{N_{1}}\dsum\limits_{i\in H^{{\large c}}}\frac{1}{T_{h}}%
\dsum\limits_{t=p}^{T-h}\sup_{i,t}E\left[ u_{i,t}^{2}\right] \\
&\leq &\frac{1}{N_{1}}\dsum\limits_{i\in H^{{\large c}}}\frac{1}{T_{h}}%
\dsum\limits_{t=p}^{T-h}C \\
&=&C
\end{eqnarray*}%
for some positive constant $C\geq \sup_{i,t}E\left[ u_{i,t}^{2}\right] $
which exists in light of Assumption 3-3(b). Hence, for any $\epsilon >0$,
set $C_{\epsilon }=C/\epsilon $, and note that 
\begin{eqnarray*}
&&\Pr \left\{ \frac{1}{N_{1}}\dsum\limits_{i\in H^{{\large c}}}\mathbb{I}%
\left\{ i\in \widehat{H^{c}}\right\} \left( \frac{u_{i\cdot }^{\prime
}u_{i\cdot }}{T_{h}}\right) \geq C_{\epsilon }\right\} \\
&\leq &\frac{1}{C_{\epsilon }}\frac{1}{N_{1}}\dsum\limits_{i\in H^{{\large c}%
}}E\left[ \mathbb{I}\left\{ i\in \widehat{H^{c}}\right\} \left( \frac{%
u_{i\cdot }^{\prime }u_{i\cdot }}{T_{h}}\right) \right] \text{ \ (by
Markov's inequality)} \\
&\leq &\frac{\epsilon }{C}C \\
&=&\epsilon
\end{eqnarray*}%
which shows that%
\begin{equation*}
\frac{1}{N_{1}}\dsum\limits_{i\in H^{{\large c}}}\mathbb{I}\left\{ i\in 
\widehat{H^{c}}\right\} \left( \frac{u_{i\cdot }^{\prime }u_{i\cdot }}{T_{h}}%
\right) =O_{p}\left( 1\right) \text{. }
\end{equation*}

Next, to show part (b), note that%
\begin{eqnarray*}
&&\frac{1}{N_{1}}\dsum\limits_{i\in H}E\left[ \mathbb{I}\left\{ i\in 
\widehat{H^{c}}\right\} \left( \frac{u_{i\cdot }^{\prime }u_{i\cdot }}{T_{h}}%
\right) \right] \\
&\leq &\frac{1}{N_{1}}\dsum\limits_{i\in H}\left( \Pr \left\{ i\in \widehat{%
H^{c}}\right\} \right) ^{\frac{{\large 5}}{{\large 7}}}\left( E\left[ \left( 
\frac{u_{i\cdot }^{\prime }u_{i\cdot }}{T_{h}}\right) ^{\frac{{\large 7}}{%
{\large 2}}}\right] \right) ^{\frac{{\large 2}}{{\large 7}}}\text{ (by H\"{o}%
lder's inequality)} \\
&=&\frac{1}{N_{1}}\dsum\limits_{i\in H}\left( \Pr \left\{ i\in \widehat{H^{c}%
}\right\} \right) ^{\frac{{\large 5}}{{\large 7}}}\left( E\left[ \left( 
\frac{1}{T_{h}}\dsum\limits_{t=p}^{T-h}u_{i,t}^{2}\right) ^{\frac{{\large 7}%
}{{\large 2}}}\right] \right) ^{\frac{{\large 2}}{{\large 7}}} \\
&\leq &\frac{1}{N_{1}}\dsum\limits_{i\in H}\left( \Pr \left\{ \mathbb{S}%
_{i,T}^{+}\geq \Phi ^{-1}\left( 1-\frac{\varphi }{2N}\right) \right\}
\right) ^{\frac{{\large 5}}{{\large 7}}}\left( \frac{1}{T_{h}}%
\dsum\limits_{t=p}^{T-h}\sup_{i,t}E\left[ \left\vert u_{i,t}\right\vert ^{7}%
\right] \right) ^{\frac{{\large 2}}{{\large 7}}} \\
&\leq &C_{1}^{\frac{{\large 2}}{{\large 7}}}\frac{1}{N_{1}}%
\dsum\limits_{i\in H}\left( \Pr \left\{ \mathbb{S}_{i,T}^{+}\geq \Phi
^{-1}\left( 1-\frac{\varphi }{2N}\right) \right\} \right) ^{\frac{{\large 5}%
}{{\large 7}}}
\end{eqnarray*}%
for some positive constant $C_{1}\geq \sup_{i,t}E\left[ \left\vert
u_{i,t}\right\vert ^{7}\right] $ which exists in light of Assumption 3-3(b).
Now, let $\mathbb{S}_{i,T}^{+}$ denote either the statistic $%
\dsum\nolimits_{\ell =1}^{d}\varpi _{\ell }\left\vert S_{i,\ell
,T}\right\vert $ or the statistic $\max_{1\leq \ell \leq d}\left\vert
S_{i,\ell ,T}\right\vert $; and, following arguments similar to that given
in the proof of part (a) of Theorem 1 in Chao and Swanson (2022a), we see
that, for any $i\in H$, there exists a constant $C_{2}>2d$ such that%
\begin{equation*}
\Pr \left\{ \mathbb{S}_{i,T}^{+}\geq \Phi ^{-1}\left( 1-\frac{\varphi }{2N}%
\right) \right\} \leq C_{2}\frac{\varphi }{N}
\end{equation*}%
for all $N_{1},N_{2}$, and $T$ sufficiently large, from which it follows that%
\begin{eqnarray*}
\frac{1}{N_{1}}\dsum\limits_{i\in H}E\left[ \mathbb{I}\left\{ i\in \widehat{%
H^{c}}\right\} \left( \frac{u_{i\cdot }^{\prime }u_{i\cdot }}{T_{h}}\right) %
\right] &\leq &C_{1}^{\frac{{\large 2}}{{\large 7}}}\frac{1}{N_{1}}%
\dsum\limits_{i\in H}\left( \Pr \left\{ \mathbb{S}_{i,T}^{+}\geq \Phi
^{-1}\left( 1-\frac{\varphi }{2N}\right) \right\} \right) ^{\frac{{\large 5}%
}{{\large 7}}} \\
&\leq &C_{1}^{\frac{{\large 2}}{{\large 7}}}\frac{1}{N_{1}}%
\dsum\limits_{i\in H}C_{2}^{\frac{{\large 5}}{{\large 7}}}\left( \frac{%
\varphi }{N}\right) ^{\frac{{\large 5}}{{\large 7}}} \\
&=&C_{1}^{\frac{{\large 2}}{{\large 7}}}C_{2}^{\frac{{\large 5}}{{\large 7}}}%
\frac{N_{2}\varphi ^{\frac{{\large 5}}{{\large 7}}}}{N_{1}N^{\frac{{\large 5}%
}{{\large 7}}}} \\
&\leq &C_{3}\frac{N^{\frac{{\large 2}}{{\large 7}}}\varphi ^{\frac{{\large 5}%
}{{\large 7}}}}{N_{1}}
\end{eqnarray*}%
for all $N_{1},N_{2}$, and $T$ sufficiently large and for some positive
constant $C_{3}\geq C_{1}^{\frac{{\large 2}}{{\large 7}}}C_{2}^{\frac{%
{\large 5}}{{\large 7}}}$. Hence, for any $\epsilon >0$, set $C_{\epsilon
}=C_{3}/\epsilon $, and note that 
\begin{eqnarray*}
&&\Pr \left\{ \frac{N_{1}}{N^{\frac{{\large 2}}{{\large 7}}}\varphi ^{\frac{%
{\large 5}}{{\large 7}}}}\frac{1}{N_{1}}\dsum\limits_{i\in H}\mathbb{I}%
\left\{ i\in \widehat{H^{c}}\right\} \left( \frac{u_{i\cdot }^{\prime
}u_{i\cdot }}{T_{h}}\right) \geq C_{\epsilon }\right\} \\
&\leq &\frac{N_{1}}{N^{\frac{{\large 2}}{{\large 7}}}\varphi ^{\frac{{\large %
5}}{{\large 7}}}}\frac{1}{C_{\epsilon }}\frac{1}{N_{1}}\dsum\limits_{i\in H}E%
\left[ \mathbb{I}\left\{ i\in \widehat{H^{c}}\right\} \left( \frac{u_{i\cdot
}^{\prime }u_{i\cdot }}{T_{h}}\right) \right] \text{ \ (by Markov's
inequality)} \\
&\leq &\frac{N_{1}}{N^{\frac{{\large 2}}{{\large 7}}}\varphi ^{\frac{{\large %
5}}{{\large 7}}}}\frac{\epsilon }{C_{3}}C_{3}\frac{N^{\frac{{\large 2}}{%
{\large 7}}}\varphi ^{\frac{{\large 5}}{{\large 7}}}}{N_{1}} \\
&=&\epsilon
\end{eqnarray*}%
which shows that%
\begin{equation*}
\frac{1}{N_{1}}\dsum\limits_{i\in H}\mathbb{I}\left\{ i\in \widehat{H^{c}}%
\right\} \left( \frac{u_{i\cdot }^{\prime }u_{i\cdot }}{T_{h}}\right)
=O_{p}\left( \frac{N^{\frac{{\large 2}}{{\large 7}}}\varphi ^{\frac{{\large 5%
}}{{\large 7}}}}{N_{1}}\right) =o_{p}\left( 1\right) \text{. }\square
\end{equation*}

\medskip

\noindent \textbf{Lemma D-9: }Let $T_{h}=T-h-p+1$ where $h$ is a (fixed)
non-negative integer and $p$ is a (fixed) positive integer. Suppose that
Assumptions 3-1, 3-2(a)-(c), 3-3, 3-4, 3-5, 3-7, 3-8, 3-10(a) and 3-11*
hold. Then, the following statements are true.

\medskip

\begin{enumerate}
\item[(a)] 
\begin{equation*}
\mathcal{T}_{1}=\frac{1}{N_{1}}\dsum\limits_{i\in H^{{\large c}}}\mathbb{I}%
\left\{ i\in \widehat{H^{c}}\right\} \frac{1}{N_{1}}\dsum\limits_{k\in H^{%
{\large c}}}\mathbb{I}\left\{ k\in \widehat{H^{c}}\right\} \left( \frac{%
u_{i\cdot }^{\prime }u_{k\cdot }}{T_{h}}\right) ^{2}=O_{p}\left( \max
\left\{ \frac{1}{N_{1}},\frac{1}{T}\right\} \right) =o_{p}\left( 1\right) 
\text{.}
\end{equation*}

\item[(b)] 
\begin{equation*}
\mathcal{T}_{2}=\frac{1}{N_{1}}\dsum\limits_{i\in H^{{\large c}}}\mathbb{I}%
\left\{ i\in \widehat{H^{c}}\right\} \frac{1}{N_{1}}\dsum\limits_{k\in H}%
\mathbb{I}\left\{ k\in \widehat{H^{c}}\right\} \left( \frac{u_{i\cdot
}^{\prime }u_{k\cdot }}{T_{h}}\right) ^{2}=O_{p}\left( \frac{N^{\frac{%
{\large 2}}{{\large 7}}}\varphi ^{\frac{{\large 5}}{{\large 7}}}}{N_{1}}%
\right) =o_{p}\left( 1\right) \text{.}
\end{equation*}

\item[(c)] 
\begin{equation*}
\mathcal{T}_{3}=\frac{1}{N_{1}}\dsum\limits_{i\in H}\mathbb{I}\left\{ i\in 
\widehat{H^{c}}\right\} \frac{1}{N_{1}}\dsum\limits_{k\in H^{{\large c}}}%
\mathbb{I}\left\{ k\in \widehat{H^{c}}\right\} \left( \frac{u_{i\cdot
}^{\prime }u_{k\cdot }}{T_{h}}\right) ^{2}=O_{p}\left( \frac{N^{\frac{%
{\large 2}}{{\large 7}}}\varphi ^{\frac{{\large 5}}{{\large 7}}}}{N_{1}}%
\right) =o_{p}\left( 1\right) \text{.}
\end{equation*}

\item[(d)] 
\begin{equation*}
\mathcal{T}_{4}=\frac{1}{N_{1}}\dsum\limits_{i\in H}\mathbb{I}\left\{ i\in 
\widehat{H^{c}}\right\} \frac{1}{N_{1}}\dsum\limits_{k\in H}\mathbb{I}%
\left\{ k\in \widehat{H^{c}}\right\} \left( \frac{u_{i\cdot }^{\prime
}u_{k\cdot }}{T_{h}}\right) ^{2}=O_{p}\left( \frac{N^{\frac{{\large 4}}{%
{\large 7}}}\varphi ^{\frac{{\large 10}}{{\large 7}}}}{N_{1}^{2}}\right)
=o_{p}\left( 1\right) \text{.}
\end{equation*}
\end{enumerate}

\noindent \medskip

\noindent \textbf{Proof of Lemma D-9:}

To show part (a), note that%
\begin{eqnarray*}
0 &\leq &\mathcal{T}_{1} \\
&=&\frac{1}{N_{1}}\dsum\limits_{i\in H^{{\large c}}}\mathbb{I}\left\{ i\in 
\widehat{H^{c}}\right\} \frac{1}{N_{1}}\dsum\limits_{k\in H^{{\large c}}}%
\mathbb{I}\left\{ k\in \widehat{H^{c}}\right\} \left( \frac{u_{i\cdot
}^{\prime }u_{k\cdot }}{T_{h}}\right) ^{2} \\
&\leq &\frac{1}{N_{1}^{2}}\dsum\limits_{i\in H^{{\large c}%
}}\dsum\limits_{k\in H^{{\large c}}}\left( \frac{u_{i\cdot }^{\prime
}u_{k\cdot }}{T_{h}}\right) ^{2} \\
&=&\frac{1}{N_{1}^{2}}\dsum\limits_{i\in H^{{\large c}}}\dsum\limits_{k\in
H^{{\large c}}}\frac{1}{T_{h}^{2}}\dsum\limits_{t=p}^{T-h}\dsum%
\limits_{s=p}^{T-h}u_{i,t}u_{k,t}u_{i,s}u_{k,s} \\
&=&\frac{1}{N_{1}^{2}}\dsum\limits_{i\in H^{{\large c}}}\dsum\limits_{k\in
H^{{\large c}}}\frac{1}{T_{h}^{2}}\dsum%
\limits_{t=p}^{T-h}u_{i,t}^{2}u_{k,t}^{2} \\
&&+\frac{1}{N_{1}^{2}}\dsum\limits_{i\in H^{{\large c}}}\dsum\limits_{k\in
H^{{\large c}}}\frac{2}{T_{h}^{2}}\dsum\limits_{t=p}^{T-h-1}\dsum%
\limits_{m=1}^{T-h-t}u_{i,t}u_{k,t}u_{i,t+m}u_{k,t+m}
\end{eqnarray*}%
From the non-negativity of $\mathcal{T}_{1}$, we get%
\begin{eqnarray*}
E\left\vert \mathcal{T}_{1}\right\vert &=&E\left[ \mathcal{T}_{1}\right] \\
&\leq &\frac{1}{N_{1}^{2}}\dsum\limits_{i\in H^{{\large c}%
}}\dsum\limits_{k\in H^{{\large c}}}\frac{1}{T_{h}^{2}}\dsum%
\limits_{t=p}^{T-h}E\left[ u_{i,t}^{2}u_{k,t}^{2}\right] \\
&&+\frac{1}{N_{1}^{2}}\dsum\limits_{i\in H^{{\large c}}}\dsum\limits_{k\in
H^{{\large c}}}\frac{2}{T_{h}^{2}}\dsum\limits_{t=p}^{T-h-1}\dsum%
\limits_{m=1}^{T-h-t}E\left[ u_{i,t}u_{k,t}u_{i,t+m}u_{k,t+m}\right]
\end{eqnarray*}%
Now,%
\begin{eqnarray*}
\frac{1}{N_{1}^{2}}\dsum\limits_{i\in H^{{\large c}}}\dsum\limits_{k\in H^{%
{\large c}}}\frac{1}{T_{h}^{2}}\dsum\limits_{t=p}^{T-h}E\left[
u_{i,t}^{2}u_{k,t}^{2}\right] &\leq &\frac{1}{N_{1}^{2}}\dsum\limits_{i\in
H^{{\large c}}}\dsum\limits_{k\in H^{{\large c}}}\frac{1}{T_{h}^{2}}%
\dsum\limits_{t=p}^{T-h}\sqrt{E\left[ u_{i,t}^{4}\right] }\sqrt{E\left[
u_{k,t}^{4}\right] } \\
&\leq &\left( \sup_{i,t}E\left[ u_{i,t}^{4}\right] \right) \frac{1}{T_{h}} \\
&\leq &\frac{C_{1}}{T_{h}}
\end{eqnarray*}%
for some positive constant $C_{1}\geq \sup_{i,t}E\left[ u_{i,t}^{4}\right] $
which exists in light of Assumption 3-3(b). Moreover,%
\begin{eqnarray*}
&&\frac{1}{N_{1}^{2}}\dsum\limits_{i\in H^{{\large c}}}\dsum\limits_{k\in H^{%
{\large c}}}\frac{2}{T_{h}^{2}}\dsum\limits_{t=p}^{T-h-1}\dsum%
\limits_{m=1}^{T-h-t}E\left[ u_{i,t}u_{k,t}u_{i,t+m}u_{k,t+m}\right] \\
&=&\frac{1}{N_{1}^{2}}\dsum\limits_{i\in H^{{\large c}}}\dsum\limits_{k\in
H^{{\large c}}}\frac{2}{T_{h}^{2}}\dsum\limits_{t=p}^{T-h-1}\dsum%
\limits_{m=1}^{T-h-t}E\left[ \left( u_{i,t}u_{k,t}-E\left[ u_{i,t}u_{k,t}%
\right] \right) \left( u_{i,t+m}u_{k,t+m}-E\left[ u_{i,t+m}u_{k,t+m}\right]
\right) \right] \\
&&+\frac{1}{N_{1}^{2}}\dsum\limits_{i\in H^{{\large c}}}\dsum\limits_{k\in
H^{{\large c}}}\frac{2}{T_{h}^{2}}\dsum\limits_{t=p}^{T-h-1}\dsum%
\limits_{m=1}^{T-h-t}E\left[ u_{i,t}u_{k,t}\right] E\left[ u_{i,t+m}u_{k,t+m}%
\right] \\
&\leq &\frac{1}{N_{1}^{2}}\dsum\limits_{i\in H^{{\large c}%
}}\dsum\limits_{k\in H^{{\large c}}}\frac{2}{T_{h}^{2}}\dsum%
\limits_{t=p}^{T-h-1}\dsum\limits_{m=1}^{T-h-t}\left\vert E\left[ \left(
u_{i,t}u_{k,t}-E\left[ u_{i,t}u_{k,t}\right] \right) \left(
u_{i,t+m}u_{k,t+m}-E\left[ u_{i,t+m}u_{k,t+m}\right] \right) \right]
\right\vert \\
&&+\frac{2}{T_{h}^{2}}\dsum\limits_{t=p}^{T-h-1}\dsum\limits_{m=1}^{T-h-t}%
\frac{1}{N_{1}^{2}}\dsum\limits_{i\in H^{{\large c}}}\dsum\limits_{k\in H^{%
{\large c}}}\left\vert E\left[ u_{i,t}u_{k,t}\right] \right\vert \left\vert E%
\left[ u_{i,t+m}u_{k,t+m}\right] \right\vert
\end{eqnarray*}%
Consider the first term on the right-hand side above. Note that by
Assumption 3-3(c), $\left\{ u_{it}\right\} _{t=-\infty }^{\infty }$ is $%
\beta $-mixing with $\beta $ mixing coefficient satisfying 
\begin{equation*}
\beta _{i}\left( m\right) \leq a_{1}\exp \left\{ -a_{2}m\right\}
\end{equation*}%
for every $i$. Since $\alpha _{i,m}\leq \beta _{i}\left( m\right) $, it
follows that $\left\{ u_{it}\right\} _{t=-\infty }^{\infty }$ is $\alpha $%
-mixing as well, with $\alpha $ mixing coefficient satisfying%
\begin{equation*}
\alpha _{i,m}\leq a_{1}\exp \left\{ -a_{2}m\right\} \text{ for every }i\text{%
.}
\end{equation*}

\noindent Hence, we can apply Lemma C-3 with $p=2$ and $r=3$ to obtain%
\begin{eqnarray*}
&&\frac{1}{N_{1}^{2}}\dsum\limits_{i\in H^{{\large c}}}\dsum\limits_{k\in H^{%
{\large c}}}\frac{2}{T_{h}^{2}}\dsum\limits_{t=p}^{T-h-1}\dsum%
\limits_{m=1}^{T-h-t}\left\vert E\left[ \left( u_{i,t}u_{k,t}-E\left[
u_{i,t}u_{k,t}\right] \right) \left( u_{i,t+m}u_{k,t+m}-E\left[
u_{i,t+m}u_{k,t+m}\right] \right) \right] \right\vert \\
&\leq &\frac{1}{N_{1}^{2}}\dsum\limits_{i\in H^{{\large c}%
}}\dsum\limits_{k\in H^{{\large c}}}\frac{2}{T_{h}^{2}}\dsum%
\limits_{t=p}^{T-h-1}\dsum\limits_{m=1}^{T-h-t}2\left( \sqrt{2}+1\right)
\alpha _{m}^{\frac{{\large 1}}{{\large 6}}}\sqrt{E\left[
u_{i,t}^{2}u_{k,t}^{2}\right] }\left( E\left\vert
u_{i,t+m}u_{k,t+m}\right\vert ^{3}\right) ^{\frac{{\large 1}}{{\large 3}}} \\
&\leq &\frac{1}{N_{1}^{2}}\dsum\limits_{i\in H^{{\large c}%
}}\dsum\limits_{k\in H^{{\large c}}}\frac{4\left( \sqrt{2}+1\right) }{%
T_{h}^{2}}\dsum\limits_{t=p}^{T-h-1}\dsum\limits_{m=1}^{T-h-t}a_{1}^{\frac{%
{\large 1}}{{\large 6}}}\exp \left\{ -\frac{a_{2}}{6}m\right\} \sqrt{E\left[
u_{i,t}^{2}u_{k,t}^{2}\right] }\left( E\left\vert
u_{i,t+m}u_{k,t+m}\right\vert ^{3}\right) ^{\frac{{\large 1}}{{\large 3}}} \\
&\leq &\dsum\limits_{i\in H^{{\large c}}}\dsum\limits_{k\in H^{{\large c}%
}}\dsum\limits_{t=p}^{T-h-1}\dsum\limits_{m=1}^{\infty }\exp \left\{ -\frac{%
a_{2}}{6}m\right\} \frac{4a_{1}^{\frac{{\large 1}}{{\large 6}}}\left( \sqrt{2%
}+1\right) \left( E\left[ u_{i,t}^{4}\right] \right) ^{\frac{{\large 1}}{%
{\large 4}}}\left( E\left[ u_{k,t}^{4}\right] \right) ^{\frac{{\large 1}}{%
{\large 4}}}\left( E\left[ u_{i,t+m}^{6}\right] E\left[ u_{k,t+m}^{6}\right]
\right) ^{\frac{{\large 1}}{{\large 6}}}}{N_{1}^{2}T_{h}^{2}} \\
&\leq &\dsum\limits_{i\in H^{{\large c}}}\dsum\limits_{k\in H^{{\large c}%
}}\dsum\limits_{t=p}^{T-h-1}\dsum\limits_{m=1}^{\infty }\exp \left\{ -\frac{%
a_{2}}{6}m\right\} \frac{4a_{1}^{\frac{{\large 1}}{{\large 6}}}\left( \sqrt{2%
}+1\right) \left( E\left[ u_{i,t}^{6}\right] \right) ^{\frac{{\large 1}}{%
{\large 6}}}\left( E\left[ u_{k,t}^{6}\right] \right) ^{\frac{{\large 1}}{%
{\large 6}}}\left( E\left[ u_{i,t+m}^{6}\right] E\left[ u_{k,t+m}^{6}\right]
\right) ^{\frac{{\large 1}}{{\large 6}}}}{N_{1}^{2}T_{h}^{2}} \\
&\leq &\frac{4\overline{C}\left( \sqrt{2}+1\right) a_{1}^{\frac{{\large 1}}{%
{\large 6}}}\left( \sup_{i,t}E\left[ u_{i,t}^{6}\right] \right) ^{\frac{%
{\large 2}}{{\large 3}}}}{T_{h}}\text{ } \\
&&\left( \text{for some positive constant }\overline{C}\text{ such that }%
\overline{C}\text{ }\geq \dsum\limits_{m=1}^{\infty }\exp \left\{ -\frac{%
a_{2}}{6}m\right\} \right) \\
&\leq &\frac{4\overline{C}\left( \sqrt{2}+1\right) a_{1}^{\frac{{\large 1}}{%
{\large 6}}}C^{\frac{{\large 2}}{{\large 3}}}}{T_{h}}\text{ } \\
&&\left( \text{by Assumption 3-3(b), there exists positive constant }C\text{
such that }\sup_{i,t}E\left\vert u_{i,t}\right\vert ^{6}\leq C<\infty \right)
\\
&\leq &\frac{C_{2}}{T_{h}}\text{ \ }\left( \text{setting }C_{2}\geq 4%
\overline{C}\left( \sqrt{2}+1\right) a_{1}^{\frac{{\large 1}}{{\large 6}}}C^{%
\frac{{\large 2}}{{\large 3}}}\right) \\
&=&O\left( \frac{1}{T}\right)
\end{eqnarray*}%
Moreover, 
\begin{eqnarray*}
&&\frac{2}{T_{h}^{2}}\dsum\limits_{t=p}^{T-h-1}\dsum\limits_{m=1}^{T-h-t}%
\frac{1}{N_{1}^{2}}\dsum\limits_{i\in H^{{\large c}}}\dsum\limits_{k\in H^{%
{\large c}}}\left\vert E\left[ u_{i,t}u_{k,t}\right] \right\vert \left\vert E%
\left[ u_{i,t+m}u_{k,t+m}\right] \right\vert \\
&\leq &\frac{2}{T_{h}^{2}}\dsum\limits_{t=p}^{T-h-1}\dsum%
\limits_{m=1}^{T-h-t}\frac{1}{N_{1}^{2}}\dsum\limits_{i\in H^{{\large c}%
}}\dsum\limits_{k\in H^{{\large c}}}\left\vert E\left[ u_{i,t}u_{k,t}\right]
\right\vert \sqrt{E\left[ u_{i,t+m}^{2}\right] }\sqrt{E\left[ u_{k,t+m}^{2}%
\right] } \\
&\leq &\left( \sup_{i,t}E\left[ u_{i,t}^{2}\right] \right) \frac{2}{T_{h}^{2}%
}\dsum\limits_{t=p}^{T-h-1}\dsum\limits_{m=1}^{T-h-t}\frac{1}{N_{1}^{2}}%
\dsum\limits_{i\in H^{{\large c}}}\dsum\limits_{k\in H^{{\large c}%
}}\left\vert E\left[ u_{i,t}u_{k,t}\right] \right\vert \\
&\leq &\frac{2}{N_{1}}\left( \sup_{i,t}E\left[ u_{i,t}^{2}\right] \right)
\sup_{t}\left( \frac{1}{N_{1}}\dsum\limits_{i\in H^{{\large c}%
}}\dsum\limits_{k\in H^{{\large c}}}\left\vert E\left[ u_{i,t}u_{k,t}\right]
\right\vert \right) \\
&\leq &\frac{C_{3}}{N_{1}}\text{.}
\end{eqnarray*}%
for some positive constant $C_{3}$ such that 
\begin{equation*}
2\left( \sup_{i,t}E\left[ u_{i,t}^{2}\right] \right) \sup_{t}\left( \frac{1}{%
N_{1}}\dsum\limits_{i\in H^{{\large c}}}\dsum\limits_{k\in H^{{\large c}%
}}\left\vert E\left[ u_{i,t}u_{k,t}\right] \right\vert \right) \leq
C_{3}<\infty
\end{equation*}%
which exists in light of Assumptions 3-3(b) and 3-3(d). It follows from
these results that%
\begin{eqnarray*}
&&E\left\vert \mathcal{T}_{1}\right\vert \\
&=&E\left[ \frac{1}{N_{1}}\dsum\limits_{i\in H^{{\large c}}}\mathbb{I}%
\left\{ i\in \widehat{H^{c}}\right\} \frac{1}{N_{1}}\dsum\limits_{k\in H^{%
{\large c}}}\mathbb{I}\left\{ k\in \widehat{H^{c}}\right\} \left( \frac{%
u_{i\cdot }^{\prime }u_{k\cdot }}{T_{h}}\right) ^{2}\right] \\
&\leq &\frac{1}{N_{1}^{2}}\dsum\limits_{i\in H^{{\large c}%
}}\dsum\limits_{k\in H^{{\large c}}}\frac{1}{T_{h}^{2}}\dsum%
\limits_{t=p}^{T-h}E\left[ u_{i,t}^{2}u_{k,t}^{2}\right] +\frac{1}{N_{1}^{2}}%
\dsum\limits_{i\in H^{{\large c}}}\dsum\limits_{k\in H^{{\large c}}}\frac{2}{%
T_{h}^{2}}\dsum\limits_{t=p}^{T-h-1}\dsum\limits_{m=1}^{T-h-t}E\left[
u_{i,t}u_{k,t}u_{i,t+m}u_{k,t+m}\right] \\
&\leq &\frac{1}{N_{1}^{2}}\dsum\limits_{i\in H^{{\large c}%
}}\dsum\limits_{k\in H^{{\large c}}}\frac{1}{T_{h}^{2}}\dsum%
\limits_{t=p}^{T-h}E\left[ u_{i,t}^{2}u_{k,t}^{2}\right] \\
&&+\frac{1}{N_{1}^{2}}\dsum\limits_{i\in H^{{\large c}}}\dsum\limits_{k\in
H^{{\large c}}}\frac{2}{T_{h}^{2}}\dsum\limits_{t=p}^{T-h-1}\dsum%
\limits_{m=1}^{T-h-t}\left\vert E\left[ \left( u_{i,t}u_{k,t}-E\left[
u_{i,t}u_{k,t}\right] \right) \left( u_{i,t+m}u_{k,t+m}-E\left[
u_{i,t+m}u_{k,t+m}\right] \right) \right] \right\vert \\
&&+\frac{2}{T_{h}^{2}}\dsum\limits_{t=p}^{T-h-1}\dsum\limits_{m=1}^{T-h-t}%
\frac{1}{N_{1}^{2}}\dsum\limits_{i\in H^{{\large c}}}\dsum\limits_{k\in H^{%
{\large c}}}\left\vert E\left[ u_{i,t}u_{k,t}\right] \right\vert \left\vert E%
\left[ u_{i,t+m}u_{k,t+m}\right] \right\vert \\
&\leq &\frac{C_{1}}{T_{h}}+\frac{C_{2}}{T_{h}}+\frac{C_{3}}{N_{1}} \\
&\leq &\frac{\overline{C}}{\min \left\{ N_{1},T_{h}\right\} }
\end{eqnarray*}%
for some positive constant $\overline{C}\geq C_{1}+C_{2}+C_{3}$. Hence, for
any $\epsilon >0$, set $C_{\epsilon }=\overline{C}/\epsilon $, and applying
Markov's inequality, we obtain 
\begin{eqnarray*}
&&\Pr \left( \min \left\{ N_{1},T_{h}\right\} \left\vert \mathcal{T}%
_{1}\right\vert \geq C_{\epsilon }\right) \\
&=&\Pr \left( \min \left\{ N_{1},T_{h}\right\} \left\vert \frac{1}{N_{1}}%
\dsum\limits_{i\in H^{{\large c}}}\mathbb{I}\left\{ i\in \widehat{H^{c}}%
\right\} \frac{1}{N_{1}}\dsum\limits_{k\in H^{{\large c}}}\mathbb{I}\left\{
k\in \widehat{H^{c}}\right\} \left( \frac{u_{i\cdot }^{\prime }u_{k\cdot }}{%
T_{h}}\right) ^{2}\right\vert \geq C_{\epsilon }\right) \\
&\leq &\frac{\min \left\{ N_{1},T_{h}\right\} }{C_{\epsilon }}E\left\{ \frac{%
1}{N_{1}}\dsum\limits_{i\in H^{{\large c}}}\mathbb{I}\left\{ i\in \widehat{%
H^{c}}\right\} \frac{1}{N_{1}}\dsum\limits_{k\in H^{{\large c}}}\mathbb{I}%
\left\{ k\in \widehat{H^{c}}\right\} \left( \frac{u_{i\cdot }^{\prime
}u_{k\cdot }}{T_{h}}\right) ^{2}\right\} \\
&\leq &\min \left\{ N_{1},T_{h}\right\} \frac{\epsilon }{\overline{C}}\frac{%
\overline{C}}{\min \left\{ N_{1},T_{h}\right\} } \\
&=&\epsilon
\end{eqnarray*}%
so that%
\begin{eqnarray*}
\mathcal{T}_{1} &=&\frac{1}{N_{1}}\dsum\limits_{i\in H^{{\large c}}}\mathbb{I%
}\left\{ i\in \widehat{H^{c}}\right\} \frac{1}{N_{1}}\dsum\limits_{k\in H^{%
{\large c}}}\mathbb{I}\left\{ k\in \widehat{H^{c}}\right\} \left( \frac{%
u_{i\cdot }^{\prime }u_{k\cdot }}{T_{h}}\right) ^{2} \\
&=&O_{p}\left( \frac{1}{\min \left\{ N_{1},T_{h}\right\} }\right)
=O_{p}\left( \frac{1}{\min \left\{ N_{1},T\right\} }\right) =O_{p}\left(
\max \left\{ \frac{1}{N_{1}},\frac{1}{T}\right\} \right) .
\end{eqnarray*}

Next, to show part (b), we apply parts (a) and (b) of Lemma D-8 to obtain%
\begin{eqnarray*}
\mathcal{T}_{2} &=&\frac{1}{N_{1}}\dsum\limits_{i\in H^{{\large c}}}\mathbb{I%
}\left\{ i\in \widehat{H^{c}}\right\} \frac{1}{N_{1}}\dsum\limits_{k\in H}%
\mathbb{I}\left\{ k\in \widehat{H^{c}}\right\} \left( \frac{u_{i\cdot
}^{\prime }u_{k\cdot }}{T_{h}}\right) ^{2} \\
&=&\frac{1}{N_{1}^{2}}\dsum\limits_{i\in H^{{\large c}}}\dsum\limits_{k\in H}%
\mathbb{I}\left\{ i\in \widehat{H^{c}}\right\} \mathbb{I}\left\{ k\in 
\widehat{H^{c}}\right\} \left( \frac{u_{i\cdot }^{\prime }u_{k\cdot }}{T_{h}}%
\right) ^{2} \\
&\leq &\frac{1}{N_{1}^{2}}\dsum\limits_{i\in H^{{\large c}%
}}\dsum\limits_{k\in H}\mathbb{I}\left\{ i\in \widehat{H^{c}}\right\} 
\mathbb{I}\left\{ k\in \widehat{H^{c}}\right\} \left( \frac{u_{i\cdot
}^{\prime }u_{i\cdot }}{T_{h}}\right) \left( \frac{u_{k\cdot }^{\prime
}u_{k\cdot }}{T_{h}}\right) \text{ (by CS inequality)} \\
&=&\left[ \frac{1}{N_{1}}\dsum\limits_{i\in H^{{\large c}}}\mathbb{I}\left\{
i\in \widehat{H^{c}}\right\} \left( \frac{u_{i\cdot }^{\prime }u_{i\cdot }}{%
T_{h}}\right) \right] \left[ \frac{1}{N_{1}}\dsum\limits_{k\in H}\mathbb{I}%
\left\{ k\in \widehat{H^{c}}\right\} \left( \frac{u_{k\cdot }^{\prime
}u_{k\cdot }}{T_{h}}\right) \right] \\
&=&O_{p}\left( 1\right) O_{p}\left( \frac{N^{\frac{{\large 2}}{{\large 7}}%
}\varphi ^{\frac{{\large 5}}{{\large 7}}}}{N_{1}}\right) \\
&=&O_{p}\left( \frac{N^{\frac{{\large 2}}{{\large 7}}}\varphi ^{\frac{%
{\large 5}}{{\large 7}}}}{N_{1}}\right) =o_{p}\left( 1\right)
\end{eqnarray*}

Part (c) can be shown in the same way as part (b) above. Hence, to avoid
redundancy, we do not give an explicit proof here.

Finally, to show part (d), we apply part (b) of Lemma D-8 to obtain%
\begin{eqnarray*}
\mathcal{T}_{4} &=&\frac{1}{N_{1}}\dsum\limits_{i\in H}\mathbb{I}\left\{
i\in \widehat{H^{c}}\right\} \frac{1}{N_{1}}\dsum\limits_{k\in H}\mathbb{I}%
\left\{ k\in \widehat{H^{c}}\right\} \left( \frac{u_{i\cdot }^{\prime
}u_{k\cdot }}{T_{h}}\right) ^{2} \\
&=&\frac{1}{N_{1}^{2}}\dsum\limits_{i\in H}\dsum\limits_{k\in H}\mathbb{I}%
\left\{ i\in \widehat{H^{c}}\right\} \mathbb{I}\left\{ k\in \widehat{H^{c}}%
\right\} \left( \frac{u_{i\cdot }^{\prime }u_{k\cdot }}{T_{h}}\right) ^{2} \\
&\leq &\frac{1}{N_{1}^{2}}\dsum\limits_{i\in H}\dsum\limits_{k\in H}\mathbb{I%
}\left\{ i\in \widehat{H^{c}}\right\} \mathbb{I}\left\{ k\in \widehat{H^{c}}%
\right\} \left( \frac{u_{i\cdot }^{\prime }u_{i\cdot }}{T_{h}}\right) \left( 
\frac{u_{k\cdot }^{\prime }u_{k\cdot }}{T_{h}}\right) \text{ (by CS
inequality)} \\
&=&\left[ \frac{1}{N_{1}}\dsum\limits_{i\in H}\mathbb{I}\left\{ i\in 
\widehat{H^{c}}\right\} \left( \frac{u_{i\cdot }^{\prime }u_{i\cdot }}{T_{h}}%
\right) \right] ^{2} \\
&=&O_{p}\left( \frac{N^{\frac{{\large 4}}{{\large 7}}}\varphi ^{\frac{%
{\large 10}}{{\large 7}}}}{N_{1}^{2}}\right) =o_{p}\left( 1\right) \text{. }%
\square
\end{eqnarray*}

\medskip

\noindent \textbf{Lemma D-10: }Let 
\begin{equation}
\widehat{\Sigma }\left( \widehat{H^{c}}\right) =\frac{Z\left( \widehat{H^{c}}%
\right) ^{\prime }Z\left( \widehat{H^{c}}\right) }{\widehat{N}_{1}T_{0}}
\label{post-selection sample cov matrix}
\end{equation}%
where $T_{0}=T-p+1$, where $\widehat{H^{c}}$ and $\widehat{N}_{1}$ are as
defined, respectively, in expressions (\ref{Hchat}) and (\ref{N1hat}) above,
and where%
\begin{equation}
\underset{T_{0}\times N}{Z\left( \widehat{H^{c}}\right) }=\left[ 
\begin{array}{cccc}
Z_{1\cdot }\mathbb{I}\left\{ 1\in \widehat{H^{c}}\right\} & Z_{2\cdot }%
\mathbb{I}\left\{ 2\in \widehat{H^{c}}\right\} & \cdots & Z_{N\cdot }\mathbb{%
I}\left\{ N\in \widehat{H^{c}}\right\}%
\end{array}%
\right]  \label{ZHhat}
\end{equation}%
with $Z_{i\cdot }=\left( Z_{i,p},Z_{i,p+1},...,Z_{i,T}\right) ^{\prime }$
for $i=1,...,N$. Suppose that Assumptions 3-1, 3-2(a)-(c), 3-3, 3-4, 3-5,
3-7, 3-8, 3-10, and 3-11* hold.

Under the assumed conditions, 
\begin{equation*}
\left\Vert \widehat{\Sigma }\left( \widehat{H^{c}}\right) -\frac{\Gamma
M_{FF}\Gamma ^{\prime }}{N_{1}}\right\Vert _{2}=o_{p}\left( 1\right) \text{
as }N_{1},\text{ }N_{2},\text{ }T\rightarrow \infty \text{,}
\end{equation*}%
where%
\begin{equation*}
M_{FF}=\frac{1}{T_{0}}\dsum\limits_{t=p}^{T}E\left[ \underline{F}_{t}%
\underline{F}_{t}^{\prime }\right] \text{.}
\end{equation*}

\bigskip

\noindent \textbf{Proof of Lemma D-10:}

To proceed, note that we can write%
\begin{equation*}
Z\left( \widehat{H^{c}}\right) =\underline{F}\Gamma \left( \widehat{H^{c}}%
\right) ^{{\Large \prime }}+U\left( \widehat{H^{c}}\right) ,
\end{equation*}%
so that%
\begin{eqnarray}
&&\widehat{\Sigma }\left( \widehat{H^{c}}\right) -\frac{\Gamma M_{FF}\Gamma
^{\prime }}{N_{1}}  \notag \\
&=&\frac{Z\left( \widehat{H^{c}}\right) ^{\prime }Z\left( \widehat{H^{c}}%
\right) }{\widehat{N}_{1}T_{0}}-\frac{\Gamma M_{FF}\Gamma ^{\prime }}{N_{1}}
\notag \\
&=&\left( 1+\frac{\widehat{N}_{1}-N_{1}}{N_{1}}\right) ^{-1}\frac{Z\left( 
\widehat{H^{c}}\right) ^{\prime }Z\left( \widehat{H^{c}}\right) }{N_{1}T_{0}}%
-\frac{\Gamma M_{FF}\Gamma ^{\prime }}{N_{1}}  \notag \\
&=&\left( 1+\frac{\widehat{N}_{1}-N_{1}}{N_{1}}\right) ^{-1}\left\{ \frac{%
\Gamma \left( \widehat{H^{c}}\right) \underline{F}^{\prime }\underline{F}%
\Gamma \left( \widehat{H^{c}}\right) ^{\prime }}{N_{1}T_{0}}+\frac{U\left( 
\widehat{H^{c}}\right) ^{\prime }\underline{F}\Gamma \left( \widehat{H^{c}}%
\right) ^{\prime }}{N_{1}T_{0}}\right.  \notag \\
&&\text{ \ \ \ \ \ \ \ \ \ \ \ \ \ \ \ \ \ \ \ \ \ \ \ \ }\left. \text{\ }+%
\frac{\Gamma \left( \widehat{H^{c}}\right) \underline{F}^{\prime }U\left( 
\widehat{H^{c}}\right) }{N_{1}T_{0}}+\frac{U\left( \widehat{H^{c}}\right)
^{\prime }U\left( \widehat{H^{c}}\right) }{N_{1}T_{0}}\right\} -\frac{\Gamma
M_{FF}\Gamma ^{\prime }}{N_{1}}  \notag \\
&=&-\left( \frac{\widehat{N}_{1}-N_{1}}{\widehat{N}_{1}}\right) \frac{\Gamma
M_{FF}\Gamma ^{\prime }}{N_{1}}+\left( 1+\frac{\widehat{N}_{1}-N_{1}}{N_{1}}%
\right) ^{-1}\left\{ \frac{1}{N_{1}}\Gamma \left( \widehat{H^{c}}\right) %
\left[ \frac{\underline{F}^{\prime }\underline{F}}{T_{0}}-M_{FF}\right]
\Gamma \left( \widehat{H^{c}}\right) \right.  \notag \\
&&+\frac{1}{N_{1}}\left( \Gamma \left( \widehat{H^{c}}\right) M_{FF}\Gamma
\left( \widehat{H^{c}}\right) ^{\prime }-\Gamma M_{FF}\Gamma ^{\prime
}\right) +\frac{U\left( \widehat{H^{c}}\right) ^{\prime }U\left( \widehat{%
H^{c}}\right) }{N_{1}T_{0}}  \notag \\
&&\left. +\frac{U\left( \widehat{H^{c}}\right) ^{\prime }\underline{F}\Gamma
\left( \widehat{H^{c}}\right) ^{\prime }}{N_{1}T_{0}}+\frac{\Gamma \left( 
\widehat{H^{c}}\right) \underline{F}^{\prime }U\left( \widehat{H^{c}}\right) 
}{N_{1}T_{0}}\right\}  \label{sigmahat minus matrix mean}
\end{eqnarray}%
where $M_{FF}$ is as defined in (\ref{MFF def}), where $\Gamma \left( 
\widehat{H^{c}}\right) $ and $U\left( \widehat{H^{c}}\right) $ are as
defined in (\ref{variable selection notations}), and where $Z\left( \widehat{%
H^{c}}\right) $ is as defined in expression (\ref{ZHhat}).

Consider first the term $-\left[ \left( \widehat{N}_{1}-N_{1}\right) /%
\widehat{N}_{1}\right] \left( \Gamma M_{FF}\Gamma ^{\prime }/N_{1}\right) $.
Note that, for some positive constant $\overline{C}$ such that 
\begin{eqnarray}
\left\Vert M_{FF}\right\Vert _{F} &=&\left\Vert \frac{1}{T_{0}}%
\dsum\limits_{t=p}^{T}E\left[ \underline{F}_{t}\underline{F}_{t}^{\prime }%
\right] \right\Vert _{F}  \notag \\
&\leq &\frac{1}{T_{0}}\dsum\limits_{t=p}^{T}\left\Vert E\left[ \underline{F}%
_{t}\underline{F}_{t}^{\prime }\right] \right\Vert _{F}\text{ }  \notag \\
&&\text{(by the homogeneity of matrix norm and the triangle inequality)} 
\notag \\
&\leq &\frac{1}{T_{0}}\dsum\limits_{t=p}^{T}E\left\Vert \underline{F}_{t}%
\underline{F}_{t}^{\prime }\right\Vert _{F}\text{ (by the Jensen's
inequality)}  \notag \\
&=&\frac{1}{T_{0}}\dsum\limits_{t=p}^{T}E\left[ \sqrt{tr\left\{ \underline{F}%
_{t}\underline{F}_{t}^{\prime }\underline{F}_{t}\underline{F}_{t}^{\prime
}\right\} }\right]  \notag \\
&=&\frac{1}{T_{0}}\dsum\limits_{t=p}^{T}E\sqrt{\left\Vert \underline{F}%
_{t}\right\Vert _{2}^{4}}  \notag \\
&=&\frac{1}{T_{0}}\dsum\limits_{t=p}^{T}E\left[ \left\Vert \underline{F}%
_{t}\right\Vert _{2}^{2}\right]  \notag \\
&\leq &\frac{1}{T_{0}}\dsum\limits_{t=p}^{T}\left( E\left[ \left\Vert 
\underline{F}_{t}\right\Vert _{2}^{6}\right] \right) ^{\frac{{\large 1}}{%
{\large 3}}}\text{ (by Liapunov's inequality)}  \notag \\
&\leq &\overline{C}^{\frac{{\large 1}}{{\large 3}}}\text{ \ (by Lemma C-5)}
\label{bd on MFF} \\
&<&\infty  \notag
\end{eqnarray}%
from which it follows that%
\begin{eqnarray*}
\left\Vert \frac{\Gamma M_{FF}\Gamma ^{\prime }}{N_{1}}\right\Vert _{F} &=&%
\sqrt{tr\left\{ \frac{\Gamma M_{FF}\Gamma ^{\prime }}{N_{1}}\frac{\Gamma
M_{FF}\Gamma ^{\prime }}{N_{1}}\right\} } \\
&\leq &\sqrt{\lambda _{\max }\left( \frac{\Gamma ^{\prime }\Gamma }{N_{1}}%
\right) tr\left\{ \frac{\Gamma M_{FF}^{2}\Gamma ^{\prime }}{N_{1}}\right\} }
\\
&=&\sqrt{\lambda _{\max }\left( \frac{\Gamma ^{\prime }\Gamma }{N_{1}}%
\right) tr\left\{ \frac{M_{FF}\Gamma ^{\prime }\Gamma M_{FF}}{N_{1}}\right\} 
} \\
&\leq &\sqrt{\lambda _{\max }^{2}\left( \frac{\Gamma ^{\prime }\Gamma }{N_{1}%
}\right) tr\left\{ M_{FF}^{2}\right\} } \\
&=&\lambda _{\max }\left( \frac{\Gamma ^{\prime }\Gamma }{N_{1}}\right)
\left\Vert M_{FF}\right\Vert _{F} \\
&\leq &C^{\ast }\overline{C}^{\frac{{\large 1}}{{\large 3}}}<\infty \text{
for all }N_{1}\text{, }N_{2}\text{ sufficiently large,}
\end{eqnarray*}%
since, by Assumption 3-6, there exists some positive constant $C^{\ast }$
such that $\lambda _{\max }\left( \Gamma ^{\prime }\Gamma /N_{1}\right) \leq
C^{\ast }<\infty $ for all $N_{1}$, $N_{2}$ sufficiently large. Moreover,
applying part (a) of Lemma D-15 and the Slutsky's theorem, we have%
\begin{equation*}
\left\vert -\left( \frac{\widehat{N}_{1}-N_{1}}{\widehat{N}_{1}}\right)
\right\vert =\left\vert \frac{\widehat{N}_{1}-N_{1}}{\widehat{N}_{1}}%
\right\vert =\left\vert \frac{\widehat{N}_{1}-N_{1}}{N_{1}}\right\vert
\left\vert \frac{1}{\left( \widehat{N}_{1}-N_{1}\right) /N_{1}+1}\right\vert 
\overset{p}{\rightarrow }0
\end{equation*}%
so that by a further application of the Slutsky's theorem, we can deduce that%
\begin{equation}
\left\Vert -\left( \frac{\widehat{N}_{1}-N_{1}}{\widehat{N}_{1}}\right) 
\frac{\Gamma M_{FF}\Gamma ^{\prime }}{N_{1}}\right\Vert _{F}=\left\vert 
\frac{\widehat{N}_{1}-N_{1}}{\widehat{N}_{1}}\right\vert \left\Vert \frac{%
\Gamma M_{FF}\Gamma ^{\prime }}{N_{1}}\right\Vert _{F}\overset{p}{%
\rightarrow }0\text{.}  \label{first term D-10}
\end{equation}

Consider now the other terms on the right-hand side of expression (\ref%
{sigmahat minus matrix mean}). To proceed, we first note that, by applying
part (a) of Lemma D-15 and the Slutsky's theorem, we have%
\begin{equation*}
\left\vert \left( 1+\frac{\widehat{N}_{1}-N_{1}}{N_{1}}\right)
^{-1}\right\vert =\left\vert 1+\frac{\widehat{N}_{1}-N_{1}}{N_{1}}%
\right\vert ^{-1}\overset{p}{\rightarrow }1\text{.}
\end{equation*}%
Next, note that%
\begin{eqnarray*}
&&\left\Vert \frac{\Gamma \left( \widehat{H^{c}}\right) M_{FF}\Gamma \left( 
\widehat{H^{c}}\right) ^{\prime }}{N_{1}}-\frac{\Gamma M_{FF}\Gamma ^{\prime
}}{N_{1}}\right\Vert _{F}^{2} \\
&=&\dsum\limits_{i=1}^{N}\dsum\limits_{k=1}^{N}\left( \mathbb{I}\left\{ i\in 
\widehat{H^{c}}\right\} \mathbb{I}\left\{ k\in \widehat{H^{c}}\right\}
\gamma _{i}^{\prime }M_{FF}\gamma _{k}-\gamma _{i}^{\prime }M_{FF}\gamma
_{k}\right) ^{2} \\
&=&\dsum\limits_{i\in H^{{\large c}}}\dsum\limits_{k\in H^{{\large c}%
}}\left( \mathbb{I}\left\{ i\in \widehat{H^{c}}\right\} \mathbb{I}\left\{
k\in \widehat{H^{c}}\right\} \gamma _{i}^{\prime }M_{FF}\gamma _{k}-\gamma
_{i}^{\prime }M_{FF}\gamma _{k}\right) ^{2}
\end{eqnarray*}%
where $H^{c}=\left\{ k\in \left\{ 1,....,N\right\} :\gamma _{k}\neq
0\right\} $, where $\widehat{H^{c}}=\left\{ i\in \left\{ 1,....,N\right\} :%
\mathbb{S}_{i,T}^{+}\geq \Phi ^{-1}\left( 1-\frac{\varphi }{2N}\right)
\right\} $, and where $\mathbb{S}_{i,T}^{+}$ denotes either the statistic $%
\dsum\nolimits_{\ell =1}^{d}\varpi _{\ell }\left\vert S_{i,\ell
,T}\right\vert $ or the statistic $\max_{1\leq \ell \leq d}\left\vert
S_{i,\ell ,T}\right\vert $. Note that%
\begin{equation*}
\dsum\limits_{i\in H^{{\large c}}}\dsum\limits_{k\in H^{{\large c}}}\left( 
\mathbb{I}\left\{ i\in \widehat{H^{c}}\right\} \mathbb{I}\left\{ k\in 
\widehat{H^{c}}\right\} \gamma _{i}^{\prime }M_{FF}\gamma _{k}-\gamma
_{i}^{\prime }M_{FF}\gamma _{k}\right) ^{2}=0\text{ if }\mathbb{I}\left\{
i\in \widehat{H^{c}}\right\} =1\text{ for every }i\in H^{c}\text{,}
\end{equation*}%
so that, for any $\epsilon >0$,%
\begin{eqnarray*}
\left\{ \left\Vert \frac{\Gamma \left( \widehat{H^{c}}\right) M_{FF}\Gamma
\left( \widehat{H^{c}}\right) ^{\prime }}{N_{1}}-\frac{\Gamma M_{FF}\Gamma
^{\prime }}{N_{1}}\right\Vert _{F}\geq \epsilon \right\} &\subseteq &\left\{
i\notin \widehat{H^{c}}\text{ for at least one }i\in H^{c}\right\} \\
&=&\dbigcup\limits_{i\in H^{{\large c}}}\left\{ i\notin \widehat{H^{c}}%
\right\} \\
&=&\dbigcup\limits_{i\in H^{{\large c}}}\left\{ \mathbb{S}_{i,T}^{+}<\Phi
^{-1}\left( 1-\frac{\varphi }{2N}\right) \right\} \\
&=&\left\{ \dbigcap\limits_{i\in H^{{\large c}}}\left\{ \mathbb{S}%
_{i,T}^{+}\geq \Phi ^{-1}\left( 1-\frac{\varphi }{2N}\right) \right\}
\right\} ^{c}
\end{eqnarray*}%
Hence, applying either part (a) or part (b) of Theorem 2 in Chao and Swanson
(2022a) depending on whether $\mathbb{S}_{i,T}^{+}=\dsum\nolimits_{\ell
=1}^{d}\varpi _{\ell }\left\vert S_{i,\ell ,T}\right\vert $ or $\mathbb{S}%
_{i,T}^{+}=\max_{1\leq \ell \leq d}\left\vert S_{i,\ell ,T}\right\vert $, we
obtain 
\begin{eqnarray*}
&&\Pr \left( \left\Vert \frac{\Gamma \left( \widehat{H^{c}}\right)
M_{FF}\Gamma \left( \widehat{H^{c}}\right) ^{\prime }}{N_{1}}-\frac{\Gamma
M_{FF}\Gamma ^{\prime }}{N_{1}}\right\Vert _{F}\geq \epsilon \right) \\
&\leq &1-\Pr \left( \dbigcap\limits_{i\in H^{{\large c}}}\left\{ \mathbb{S}%
_{i,T}^{+}\geq \Phi ^{-1}\left( 1-\frac{\varphi }{2N}\right) \right\} \right)
\\
&=&1-\Pr \left( \min_{{\large i\in }H^{{\large c}}}\mathbb{S}_{i,T}^{+}\geq
\Phi ^{-1}\left( 1-\frac{\varphi }{2N}\right) \right) \\
&\rightarrow &1-1=0\text{ as }N_{1},N_{2},T\rightarrow \infty \text{,}
\end{eqnarray*}%
so that%
\begin{equation}
\left\Vert \frac{\Gamma \left( \widehat{H^{c}}\right) M_{FF}\Gamma \left( 
\widehat{H^{c}}\right) ^{\prime }}{N_{1}}-\frac{\Gamma M_{FF}\Gamma ^{\prime
}}{N_{1}}\right\Vert _{F}=o_{p}\left( 1\right)  \label{dev GammaF^2 asy 1}
\end{equation}

Now, consider the term $\Gamma \left( \widehat{H^{c}}\right) \left[ \frac{%
\underline{F}^{\prime }\underline{F}}{T_{0}}-M_{FF}\right] \Gamma \left( 
\widehat{H^{c}}\right) ^{\prime }/N_{1}$. For this term, note first that, by
sub-multiplicativity of matrix norms, we have that%
\begin{eqnarray*}
\left\Vert \frac{\Gamma \left( \widehat{H^{c}}\right) \left[ \frac{%
\underline{F}^{\prime }\underline{F}}{T_{0}}-M_{FF}\right] \Gamma \left( 
\widehat{H^{c}}\right) ^{\prime }}{N_{1}}\right\Vert _{F} &\leq &\left\Vert 
\frac{\Gamma \left( \widehat{H^{c}}\right) }{\sqrt{N_{1}}}\right\Vert
_{F}\left\Vert \frac{\underline{F}^{\prime }\underline{F}}{T_{0}}%
-M_{FF}\right\Vert _{F}\left\Vert \frac{\Gamma \left( \widehat{H^{c}}\right)
^{\prime }}{\sqrt{N_{1}}}\right\Vert _{F} \\
&=&\left\Vert \frac{\Gamma \left( \widehat{H^{c}}\right) }{\sqrt{N_{1}}}%
\right\Vert _{F}^{2}\left\Vert \frac{\underline{F}^{\prime }\underline{F}}{%
T_{0}}-M_{FF}\right\Vert _{F}
\end{eqnarray*}%
Note that%
\begin{eqnarray*}
\left\Vert \frac{\Gamma \left( \widehat{H^{c}}\right) }{\sqrt{N_{1}}}%
\right\Vert _{F}^{2} &=&tr\left\{ \frac{\Gamma \left( \widehat{H^{c}}\right)
^{\prime }\Gamma \left( \widehat{H^{c}}\right) }{N_{1}}\right\} \\
&=&tr\left\{ \frac{1}{N_{1}}\dsum\limits_{i=1}^{N}\mathbb{I}\left\{ i\in 
\widehat{H^{c}}\right\} \gamma _{i}\gamma _{i}^{\prime }\right\} \\
&=&\frac{1}{N_{1}}\dsum\limits_{i=1}^{N}\mathbb{I}\left\{ i\in \widehat{H^{c}%
}\right\} tr\left\{ \gamma _{i}\gamma _{i}^{\prime }\right\} \\
&=&\frac{1}{N_{1}}\dsum\limits_{i=1}^{N}\left\Vert \gamma _{i}\right\Vert
_{2}^{2}\mathbb{I}\left\{ i\in \widehat{H^{c}}\right\} \\
&\leq &\sup_{i}\left\Vert \gamma _{i}\right\Vert _{2}^{2}\frac{1}{N_{1}}%
\dsum\limits_{i=1}^{N}\mathbb{I}\left\{ i\in \widehat{H^{c}}\right\} \\
&=&\sup_{i\in \in H^{{\large c}}}\left\Vert \gamma _{i}\right\Vert _{2}^{2}%
\frac{1}{N_{1}}\dsum\limits_{i=1}^{N}\mathbb{I}\left\{ i\in \widehat{H^{c}}%
\right\} \\
&&\left( \text{since }\gamma _{i}=0\text{ for all }\gamma _{i}\in H\right) \\
&\leq &C_{1}\frac{1}{N_{1}}\dsum\limits_{i=1}^{N}\mathbb{I}\left\{ i\in 
\widehat{H^{c}}\right\}
\end{eqnarray*}%
for some positive constanct $C_{1}\geq \sup_{i}\left\Vert \gamma
_{i}\right\Vert _{2}^{2}=\sup_{i\in \in H^{{\large c}}}\left\Vert \gamma
_{i}\right\Vert _{2}^{2}$ which exists in light of Assumption 3-5. Moreover,
write%
\begin{equation}
\frac{1}{N_{1}}\dsum\limits_{i=1}^{N}\mathbb{I}\left\{ i\in \widehat{H^{c}}%
\right\} =\frac{1}{N_{1}}\dsum\limits_{i\in H}\mathbb{I}\left\{ i\in 
\widehat{H^{c}}\right\} +\frac{1}{N_{1}}\dsum\limits_{i\in H^{{\large c}}}%
\mathbb{I}\left\{ i\in \widehat{H^{c}}\right\}  \label{avg of identity Hc}
\end{equation}%
For the first term on the right-hand side of expression (\ref{avg of
identity Hc}) above, we can apply part (a) of Lemma D-7 to obtain%
\begin{equation*}
\frac{1}{N_{1}}\dsum\limits_{i\in H}\mathbb{I}\left\{ i\in \widehat{H^{c}}%
\right\} =O_{p}\left( \frac{\varphi }{N_{1}}\right) =o_{p}\left( 1\right) 
\text{.}
\end{equation*}%
With regard to the second term on the right-hand side of expression (\ref%
{avg of identity Hc}), note that%
\begin{equation*}
\frac{1}{N_{1}}\dsum\limits_{i\in H^{{\large c}}}E\left[ \mathbb{I}\left\{
i\in \widehat{H^{c}}\right\} \right] \leq 1
\end{equation*}%
since, by definition, $N_{1}$ is the cardinality of the set$\left\{ i\in
\left\{ 1,...,N\right\} :i\in H^{c}\right\} $. Hence, for any $\epsilon >0$,
set $C_{\epsilon }=C/\epsilon $ for any positive constant $C\geq 1$, and
note that 
\begin{eqnarray*}
\Pr \left\{ \frac{1}{N_{1}}\dsum\limits_{i\in H^{{\large c}}}\mathbb{I}%
\left\{ i\in \widehat{H^{c}}\right\} \geq C_{\epsilon }\right\} &\leq &\frac{%
1}{C_{\epsilon }}\frac{1}{N_{1}}\dsum\limits_{i\in H^{{\large c}}}E\left[ 
\mathbb{I}\left\{ i\in \widehat{H^{c}}\right\} \right] \text{ \ (by Markov's
inequality)} \\
&\leq &\frac{\epsilon }{C}C \\
&=&\epsilon
\end{eqnarray*}%
which shows that%
\begin{equation*}
\frac{1}{N_{1}}\dsum\limits_{i\in H^{{\large c}}}\mathbb{I}\left\{ i\in 
\widehat{H^{c}}\right\} =O_{p}\left( 1\right) \text{.}
\end{equation*}%
It follows that%
\begin{eqnarray*}
\left\Vert \frac{\Gamma \left( \widehat{H^{c}}\right) }{\sqrt{N_{1}}}%
\right\Vert _{F}^{2} &\leq &C_{1}\frac{1}{N_{1}}\dsum\limits_{i=1}^{N}%
\mathbb{I}\left\{ i\in \widehat{H^{c}}\right\} \\
&=&\frac{C_{1}}{N_{1}}\dsum\limits_{i\in H}\mathbb{I}\left\{ i\in \widehat{%
H^{c}}\right\} +\frac{C_{1}}{N_{1}}\dsum\limits_{i\in H^{{\large c}}}\mathbb{%
I}\left\{ i\in \widehat{H^{c}}\right\} \\
&=&O_{p}\left( \frac{\varphi }{N_{1}}\right) +O_{p}\left( 1\right) \\
&=&O_{p}\left( 1\right) \text{.}
\end{eqnarray*}%
In addition, applying the result of part (b) of Lemma D-2, we have that%
\begin{equation*}
\left\Vert \frac{\underline{F}^{\prime }\underline{F}}{T_{0}}%
-M_{FF}\right\Vert _{F}=O_{p}\left( \frac{1}{\sqrt{T}}\right) =o_{p}\left(
1\right)
\end{equation*}%
from which we further deduce that%
\begin{eqnarray}
\left\Vert \frac{\Gamma \left( \widehat{H^{c}}\right) \left[ \frac{%
\underline{F}^{\prime }\underline{F}}{T_{0}}-M_{FF}\right] \Gamma \left( 
\widehat{H^{c}}\right) ^{\prime }}{N_{1}}\right\Vert _{F} &\leq &\left\Vert 
\frac{\Gamma \left( \widehat{H^{c}}\right) }{\sqrt{N_{1}}}\right\Vert
_{F}^{2}\left\Vert \frac{\underline{F}^{\prime }\underline{F}}{T_{0}}%
-M_{FF}\right\Vert _{F}  \notag \\
&=&O_{p}\left( 1\right) O_{p}\left( \frac{1}{\sqrt{T}}\right)  \notag \\
&=&O_{p}\left( \frac{1}{\sqrt{T}}\right) \text{.}  \label{dev GammaF^2 asy 2}
\end{eqnarray}

Turning our attention to the term $U\left( \widehat{H^{c}}\right) ^{\prime }%
\underline{F}\Gamma \left( \widehat{H^{c}}\right) ^{\prime }/\left(
N_{1}T_{0}\right) $, we first write

\begin{eqnarray*}
&&\left\Vert \frac{U\left( \widehat{H^{c}}\right) ^{\prime }\underline{F}%
\Gamma \left( \widehat{H^{c}}\right) ^{\prime }}{N_{1}T_{0}}\right\Vert
_{F}^{2} \\
&=&\dsum\limits_{i=1}^{N}\dsum\limits_{k=1}^{N}\left( \mathbb{I}\left\{ i\in 
\widehat{H^{c}}\right\} \mathbb{I}\left\{ k\in \widehat{H^{c}}\right\} \frac{%
u_{i\cdot }^{\prime }\underline{F}\gamma _{k}}{N_{1}T_{0}}\right) ^{2} \\
&=&\frac{1}{N_{1}^{2}T_{0}^{2}}\dsum\limits_{i=1}^{N}\dsum\limits_{k=1}^{N}%
\mathbb{I}\left\{ i\in \widehat{H^{c}}\right\} \mathbb{I}\left\{ k\in 
\widehat{H^{c}}\right\} \gamma _{k}^{\prime }\underline{F}^{\prime
}u_{i\cdot }u_{i\cdot }^{\prime }\underline{F}\gamma _{k} \\
&=&\frac{1}{N_{1}^{2}T_{0}^{2}}\dsum\limits_{i=1}^{N}\dsum\limits_{k\in H^{%
{\large c}}}\mathbb{I}\left\{ i\in \widehat{H^{c}}\right\} \mathbb{I}\left\{
k\in \widehat{H^{c}}\right\} \gamma _{k}^{\prime }\underline{F}^{\prime
}u_{i\cdot }u_{i\cdot }^{\prime }\underline{F}\gamma _{k} \\
&=&\frac{1}{N_{1}}\dsum\limits_{k\in H^{{\large c}}}\mathbb{I}\left\{ k\in 
\widehat{H^{c}}\right\} \frac{1}{N_{1}T_{0}^{2}}\dsum\limits_{i=1}^{N}%
\mathbb{I}\left\{ i\in \widehat{H^{c}}\right\} \left( \gamma _{k}^{\prime }%
\underline{F}^{\prime }u_{i\cdot }\right) ^{2} \\
&\leq &\frac{1}{N_{1}}\dsum\limits_{k\in H^{{\large c}}}\frac{1}{%
N_{1}T_{0}^{2}}\dsum\limits_{i=1}^{N}\mathbb{I}\left\{ i\in \widehat{H^{c}}%
\right\} \left( \gamma _{k}^{\prime }\underline{F}^{\prime }u_{i\cdot
}\right) ^{2} \\
&=&\frac{1}{N_{1}}\dsum\limits_{k\in H^{{\large c}}}\frac{1}{N_{1}T_{0}^{2}}%
\dsum\limits_{i\in H}\mathbb{I}\left\{ i\in \widehat{H^{c}}\right\} \left(
\gamma _{k}^{\prime }\underline{F}^{\prime }u_{i\cdot }\right) ^{2}+\frac{1}{%
N_{1}}\dsum\limits_{k\in H^{{\large c}}}\frac{1}{N_{1}T_{0}^{2}}%
\dsum\limits_{i\in H^{{\large c}}}\mathbb{I}\left\{ i\in \widehat{H^{c}}%
\right\} \left( \gamma _{k}^{\prime }\underline{F}^{\prime }u_{i\cdot
}\right) ^{2} \\
&=&\dsum\limits_{i\in H}\mathbb{I}\left\{ i\in \widehat{H^{c}}\right\} \frac{%
1}{N_{1}}\dsum\limits_{k\in H^{{\large c}}}\left( \frac{\gamma _{k}^{\prime }%
\underline{F}^{\prime }u_{i\cdot }}{\sqrt{N_{1}}T_{0}}\right) ^{2}+\frac{1}{%
N_{1}}\dsum\limits_{i\in H^{{\large c}}}\mathbb{I}\left\{ i\in \widehat{H^{c}%
}\right\} \dsum\limits_{k\in H^{{\large c}}}\left( \frac{\gamma _{k}^{\prime
}\underline{F}^{\prime }u_{i\cdot }}{\sqrt{N_{1}}T_{0}}\right) ^{2}
\end{eqnarray*}%
Applying parts (b) and (c) of Lemma D-7, we obtain 
\begin{eqnarray*}
&&\left\Vert \frac{U\left( \widehat{H^{c}}\right) ^{\prime }\underline{F}%
\Gamma \left( \widehat{H^{c}}\right) ^{\prime }}{N_{1}T_{0}}\right\Vert
_{F}^{2} \\
&\leq &\dsum\limits_{i\in H}\mathbb{I}\left\{ i\in \widehat{H^{c}}\right\} 
\frac{1}{N_{1}}\dsum\limits_{k\in H^{{\large c}}}\left( \frac{\gamma
_{k}^{\prime }\underline{F}^{\prime }u_{i\cdot }}{\sqrt{N_{1}}T_{0}}\right)
^{2}+\frac{1}{N_{1}}\dsum\limits_{i\in H^{{\large c}}}\mathbb{I}\left\{ i\in 
\widehat{H^{c}}\right\} \dsum\limits_{k\in H^{{\large c}}}\left( \frac{%
\gamma _{k}^{\prime }\underline{F}^{\prime }u_{i\cdot }}{\sqrt{N_{1}}T_{0}}%
\right) ^{2} \\
&=&O_{p}\left( \frac{N_{2}^{\frac{{\large 1}}{{\large 3}}}\varphi }{N_{1}T}%
\right) +O_{p}\left( \frac{1}{T}\right) \\
&=&O_{p}\left( \max \left\{ \frac{N_{2}^{\frac{{\large 1}}{{\large 3}}%
}\varphi }{N_{1}T},\frac{1}{T}\right\} \right) \\
&=&o_{p}\left( 1\right) \text{ }\left( \text{by Assumption 3-11*}\right)
\end{eqnarray*}%
so that%
\begin{equation}
\left\Vert \frac{U\left( \widehat{H^{c}}\right) ^{\prime }\underline{F}%
\Gamma \left( \widehat{H^{c}}\right) ^{\prime }}{N_{1}T_{0}}\right\Vert
_{F}=O_{p}\left( \max \left\{ N_{2}^{\frac{{\large 1}}{{\large 6}}}\sqrt{%
\frac{\varphi }{N_{1}T}},\frac{1}{\sqrt{T}}\right\} \right) =o_{p}\left(
1\right) \text{.}  \label{UFGamma asy result 1}
\end{equation}%
Since%
\begin{equation*}
\left\Vert \frac{\Gamma \left( \widehat{H^{c}}\right) \underline{F}^{\prime
}U\left( \widehat{H^{c}}\right) }{N_{1}T_{0}}\right\Vert _{F}=\left\Vert 
\frac{U\left( \widehat{H^{c}}\right) ^{\prime }\underline{F}\Gamma \left( 
\widehat{H^{c}}\right) ^{\prime }}{N_{1}T_{0}}\right\Vert _{F}
\end{equation*}%
it follows immediately also that%
\begin{equation}
\left\Vert \frac{U\left( \widehat{H^{c}}\right) ^{\prime }\underline{F}%
\Gamma \left( \widehat{H^{c}}\right) ^{\prime }}{N_{1}T_{0}}\right\Vert
_{F}=O_{p}\left( \max \left\{ N_{2}^{\frac{{\large 1}}{{\large 6}}}\sqrt{%
\frac{\varphi }{N_{1}T}},\frac{1}{\sqrt{T}}\right\} \right) =o_{p}\left(
1\right) \text{.}  \label{UFGamma asy result 2}
\end{equation}

Finally, consider the term $\left\Vert U\left( \widehat{H^{c}}\right)
^{\prime }U\left( \widehat{H^{c}}\right) /N_{1}T_{0}\right\Vert _{F}^{2}$,
where%
\begin{equation*}
U\left( \widehat{H^{c}}\right) =\left[ 
\begin{array}{cccc}
u_{1\cdot }\mathbb{I}\left\{ 1\in \widehat{H^{c}}\right\} & u_{2\cdot }%
\mathbb{I}\left\{ 2\in \widehat{H^{c}}\right\} & \cdots & u_{N\cdot }\mathbb{%
I}\left\{ N\in \widehat{H^{c}}\right\}%
\end{array}%
\right] \text{.}
\end{equation*}%
Given that%
\begin{eqnarray*}
&&U\left( \widehat{H^{c}}\right) ^{\prime }U\left( \widehat{H^{c}}\right) \\
&=&\left( 
\begin{array}{ccc}
u_{1\cdot }^{\prime }u_{1\cdot }\mathbb{I}\left\{ 1\in \widehat{H^{c}}%
\right\} & \cdots & u_{1\cdot }^{\prime }u_{N\cdot }\mathbb{I}\left\{ 1\in 
\widehat{H^{c}}\right\} \mathbb{I}\left\{ N\in \widehat{H^{c}}\right\} \\ 
u_{1\cdot }^{\prime }u_{2\cdot }\mathbb{I}\left\{ 1\in \widehat{H^{c}}%
\right\} \mathbb{I}\left\{ 2\in \widehat{H^{c}}\right\} & \cdots & u_{2\cdot
}^{\prime }u_{N\cdot }\mathbb{I}\left\{ 2\in \widehat{H^{c}}\right\} \mathbb{%
I}\left\{ N\in \widehat{H^{c}}\right\} \\ 
\vdots &  & \vdots \\ 
u_{1\cdot }^{\prime }u_{N\cdot }\mathbb{I}\left\{ 1\in \widehat{H^{c}}%
\right\} \mathbb{I}\left\{ N\in \widehat{H^{c}}\right\} & \cdots & u_{N\cdot
}^{\prime }u_{N\cdot }\mathbb{I}\left\{ N\in \widehat{H^{c}}\right\}%
\end{array}%
\right) \text{,}
\end{eqnarray*}%
we can write 
\begin{eqnarray*}
\left\Vert \frac{U\left( \widehat{H^{c}}\right) ^{\prime }U\left( \widehat{%
H^{c}}\right) }{N_{1}T_{0}}\right\Vert _{F}^{2}
&=&\dsum\limits_{i=1}^{N}\dsum\limits_{k=1}^{N}\left( \mathbb{I}\left\{ i\in 
\widehat{H^{c}}\right\} \mathbb{I}\left\{ k\in \widehat{H^{c}}\right\} \frac{%
u_{i\cdot }^{\prime }u_{k\cdot }}{N_{1}T_{0}}\right) ^{2} \\
&=&\frac{1}{N_{1}^{2}T_{0}^{2}}\dsum\limits_{i=1}^{N}\dsum\limits_{k=1}^{N}%
\mathbb{I}\left\{ i\in \widehat{H^{c}}\right\} \mathbb{I}\left\{ k\in 
\widehat{H^{c}}\right\} \left( u_{i\cdot }^{\prime }u_{k\cdot }\right) ^{2}
\\
&=&\frac{1}{N_{1}}\dsum\limits_{i=1}^{N}\mathbb{I}\left\{ i\in \widehat{H^{c}%
}\right\} \frac{1}{N_{1}}\dsum\limits_{k=1}^{N}\mathbb{I}\left\{ k\in 
\widehat{H^{c}}\right\} \left( \frac{u_{i\cdot }^{\prime }u_{k\cdot }}{T_{0}}%
\right) ^{2} \\
&=&\frac{1}{N_{1}}\dsum\limits_{i\in H^{{\large c}}}\mathbb{I}\left\{ i\in 
\widehat{H^{c}}\right\} \frac{1}{N_{1}}\dsum\limits_{k\in H^{{\large c}}}%
\mathbb{I}\left\{ k\in \widehat{H^{c}}\right\} \left( \frac{u_{i\cdot
}^{\prime }u_{k\cdot }}{T_{0}}\right) ^{2} \\
&&+\frac{1}{N_{1}}\dsum\limits_{i\in H^{{\large c}}}\mathbb{I}\left\{ i\in 
\widehat{H^{c}}\right\} \frac{1}{N_{1}}\dsum\limits_{k\in H}\mathbb{I}%
\left\{ k\in \widehat{H^{c}}\right\} \left( \frac{u_{i\cdot }^{\prime
}u_{k\cdot }}{T_{0}}\right) ^{2} \\
&&+\frac{1}{N_{1}}\dsum\limits_{i\in H}\mathbb{I}\left\{ i\in \widehat{H^{c}}%
\right\} \frac{1}{N_{1}}\dsum\limits_{k\in H^{{\large c}}}\mathbb{I}\left\{
k\in \widehat{H^{c}}\right\} \left( \frac{u_{i\cdot }^{\prime }u_{k\cdot }}{%
T_{0}}\right) ^{2} \\
&&+\frac{1}{N_{1}}\dsum\limits_{i\in H}\mathbb{I}\left\{ i\in \widehat{H^{c}}%
\right\} \frac{1}{N_{1}}\dsum\limits_{k\in H}\mathbb{I}\left\{ k\in \widehat{%
H^{c}}\right\} \left( \frac{u_{i\cdot }^{\prime }u_{k\cdot }}{T_{0}}\right)
^{2} \\
&=&\mathcal{T}_{1}+\mathcal{T}_{2}+\mathcal{T}_{3}+\mathcal{T}_{4}\text{ }%
\left( say\right) \text{,}
\end{eqnarray*}%
where the order of magnitude in probability of the terms $\mathcal{T}_{1}$, $%
\mathcal{T}_{2}$, $\mathcal{T}_{3}$, and $\mathcal{T}_{4}$ are given in
parts (a)-(d) of Lemma D-9. It, thus, follows by applying parts (a)-(d) of
Lemma D-9 with $h=0$ that

\begin{eqnarray*}
&&\left\Vert \frac{U\left( \widehat{H^{c}}\right) ^{\prime }U\left( \widehat{%
H^{c}}\right) }{N_{1}T_{0}}\right\Vert _{F}^{2} \\
&=&\frac{1}{N_{1}^{2}T_{0}^{2}}\dsum\limits_{i=1}^{N}\dsum\limits_{k=1}^{N}%
\mathbb{I}\left\{ i\in \widehat{H^{c}}\right\} \mathbb{I}\left\{ k\in 
\widehat{H^{c}}\right\} \left( u_{i\cdot }^{\prime }u_{k\cdot }\right) ^{2}
\\
&=&\mathcal{T}_{1}+\mathcal{T}_{2}+\mathcal{T}_{3}+\mathcal{T}_{4} \\
&=&O_{p}\left( \max \left\{ \frac{1}{N_{1}},\frac{1}{T}\right\} \right)
+O_{p}\left( \frac{N^{\frac{{\large 2}}{{\large 7}}}\varphi ^{\frac{{\large 5%
}}{{\large 7}}}}{N_{1}}\right) +O_{p}\left( \frac{N^{\frac{{\large 2}}{%
{\large 7}}}\varphi ^{\frac{{\large 5}}{{\large 7}}}}{N_{1}}\right)
+O_{p}\left( \frac{N^{\frac{{\large 4}}{{\large 7}}}\varphi ^{\frac{{\large %
10}}{{\large 7}}}}{N_{1}^{2}}\right) \\
&=&O_{p}\left( \max \left\{ \frac{1}{N_{1}},\frac{1}{T},\frac{N^{\frac{%
{\large 2}}{{\large 7}}}\varphi ^{\frac{{\large 5}}{{\large 7}}}}{N_{1}}%
\right\} \right) \\
&=&o_{p}\left( 1\right) \text{. }
\end{eqnarray*}

\noindent from which we further deduce that%
\begin{equation}
\left\Vert \frac{U\left( \widehat{H^{c}}\right) ^{\prime }U\left( \widehat{%
H^{c}}\right) }{N_{1}T_{0}}\right\Vert _{F}=O_{p}\left( \max \left\{ \frac{1%
}{\sqrt{N_{1}}},\frac{1}{\sqrt{T}},\frac{N^{\frac{{\large 1}}{{\large 7}}%
}\varphi ^{\frac{{\large 5}}{{\large 14}}}}{\sqrt{N_{1}}}\right\} \right)
=o_{p}\left( 1\right) \text{ as }N_{1},N_{2},T\rightarrow \infty \text{. }
\label{UU asy result}
\end{equation}

Expressions (\ref{first term D-10})-(\ref{UU asy result}) together imply that%
\begin{eqnarray*}
&&\left\Vert \widehat{\Sigma }\left( \widehat{H^{c}}\right) -\frac{\Gamma
M_{FF}\Gamma ^{\prime }}{N_{1}}\right\Vert _{F} \\
&=&\left\Vert \left( 1+\frac{\widehat{N}_{1}-N_{1}}{N_{1}}\right) ^{-1}\frac{%
Z\left( \widehat{H^{c}}\right) ^{\prime }Z\left( \widehat{H^{c}}\right) }{%
N_{1}T_{0}}-\frac{\Gamma M_{FF}\Gamma ^{\prime }}{N_{1}}\right\Vert _{F} \\
&\leq &\left\vert \frac{\widehat{N}_{1}-N_{1}}{\widehat{N}_{1}}\right\vert
\left\Vert \frac{\Gamma M_{FF}\Gamma ^{\prime }}{N_{1}}\right\Vert
_{F}+\left\vert 1+\frac{\widehat{N}_{1}-N_{1}}{N_{1}}\right\vert
^{-1}\left\Vert \frac{\Gamma \left( \widehat{H^{c}}\right) M_{FF}\Gamma
\left( \widehat{H^{c}}\right) ^{\prime }}{N_{1}}-\frac{\Gamma M_{FF}\Gamma
^{\prime }}{N_{1}}\right\Vert _{F} \\
&&+\left\vert 1+\frac{\widehat{N}_{1}-N_{1}}{N_{1}}\right\vert
^{-1}\left\Vert \frac{1}{N_{1}}\Gamma \left( \widehat{H^{c}}\right) \left[ 
\frac{\underline{F}^{\prime }\underline{F}}{T_{0}}-M_{FF}\right] \Gamma
\left( \widehat{H^{c}}\right) ^{\prime }\right\Vert _{F} \\
&&+\left\vert 1+\frac{\widehat{N}_{1}-N_{1}}{N_{1}}\right\vert
^{-1}\left\Vert \frac{U\left( \widehat{H^{c}}\right) ^{\prime }\underline{F}%
\Gamma \left( \widehat{H^{c}}\right) ^{\prime }}{N_{1}T_{0}}\right\Vert _{F}
\\
&&+\left\vert 1+\frac{\widehat{N}_{1}-N_{1}}{N_{1}}\right\vert
^{-1}\left\Vert \frac{\Gamma \left( \widehat{H^{c}}\right) \underline{F}%
^{\prime }U\left( \widehat{H^{c}}\right) }{N_{1}T_{0}}\right\Vert
_{F}+\left\vert 1+\frac{\widehat{N}_{1}-N_{1}}{N_{1}}\right\vert
^{-1}\left\Vert \frac{U\left( \widehat{H^{c}}\right) ^{\prime }U\left( 
\widehat{H^{c}}\right) }{N_{1}T_{0}}\right\Vert _{F} \\
&=&o_{p}\left( 1\right) \text{ as }N_{1},N_{2},T\rightarrow \infty \text{.}
\end{eqnarray*}%
Since $\left\Vert A\right\Vert _{2}\leq \left\Vert A\right\Vert _{F}$, we
also have%
\begin{eqnarray*}
&&\left\Vert E\right\Vert _{2} \\
&=&\left\Vert \widehat{\Sigma }\left( \widehat{H^{c}}\right) -\frac{\Gamma
M_{FF}\Gamma ^{\prime }}{N_{1}}\right\Vert _{2} \\
&=&\left\Vert \left( 1+\frac{\widehat{N}_{1}-N_{1}}{N_{1}}\right) ^{-1}\frac{%
Z\left( \widehat{H^{c}}\right) ^{\prime }Z\left( \widehat{H^{c}}\right) }{%
\widehat{N}_{1}T_{0}}-\frac{\Gamma M_{FF}\Gamma ^{\prime }}{N_{1}}%
\right\Vert _{2} \\
&\leq &\left\vert \frac{\widehat{N}_{1}-N_{1}}{\widehat{N}_{1}}\right\vert
\left\Vert \frac{\Gamma M_{FF}\Gamma ^{\prime }}{N_{1}}\right\Vert
_{2}+\left\vert 1+\frac{\widehat{N}_{1}-N_{1}}{N_{1}}\right\vert
^{-1}\left\Vert \frac{\Gamma \left( \widehat{H^{c}}\right) M_{FF}\Gamma
\left( \widehat{H^{c}}\right) ^{\prime }}{N_{1}}-\frac{\Gamma M_{FF}\Gamma
^{\prime }}{N_{1}}\right\Vert _{2} \\
&&+\left\vert 1+\frac{\widehat{N}_{1}-N_{1}}{N_{1}}\right\vert
^{-1}\left\Vert \frac{1}{N_{1}}\Gamma \left( \widehat{H^{c}}\right) \left[ 
\frac{\underline{F}^{\prime }\underline{F}}{T_{0}}-M_{FF}\right] \Gamma
\left( \widehat{H^{c}}\right) ^{\prime }\right\Vert _{2} \\
&&+\left\vert 1+\frac{\widehat{N}_{1}-N_{1}}{N_{1}}\right\vert
^{-1}\left\Vert \frac{U\left( \widehat{H^{c}}\right) ^{\prime }\underline{F}%
\Gamma \left( \widehat{H^{c}}\right) ^{\prime }}{N_{1}T_{0}}\right\Vert _{2}
\\
&&+\left\vert 1+\frac{\widehat{N}_{1}-N_{1}}{N_{1}}\right\vert
^{-1}\left\Vert \frac{\Gamma \left( \widehat{H^{c}}\right) \underline{F}%
^{\prime }U\left( \widehat{H^{c}}\right) }{N_{1}T_{0}}\right\Vert
_{2}+\left\vert 1+\frac{\widehat{N}_{1}-N_{1}}{N_{1}}\right\vert
^{-1}\left\Vert \frac{U\left( \widehat{H^{c}}\right) ^{\prime }U\left( 
\widehat{H^{c}}\right) }{N_{1}T_{0}}\right\Vert _{2} \\
&\leq &\left\vert \frac{\widehat{N}_{1}-N_{1}}{\widehat{N}_{1}}\right\vert
^{-1}\left\Vert \frac{\Gamma M_{FF}\Gamma ^{\prime }}{N_{1}}\right\Vert
_{F}+\left\vert 1+\frac{\widehat{N}_{1}-N_{1}}{N_{1}}\right\vert
^{-1}\left\Vert \frac{\Gamma \left( \widehat{H^{c}}\right) M_{FF}\Gamma
\left( \widehat{H^{c}}\right) ^{\prime }}{N_{1}}-\frac{\Gamma M_{FF}\Gamma
^{\prime }}{N_{1}}\right\Vert _{F} \\
&&+\left\vert 1+\frac{\widehat{N}_{1}-N_{1}}{N_{1}}\right\vert
^{-1}\left\Vert \frac{1}{N_{1}}\Gamma \left( \widehat{H^{c}}\right) \left[ 
\frac{\underline{F}^{\prime }\underline{F}}{T_{0}}-M_{FF}\right] \Gamma
\left( \widehat{H^{c}}\right) ^{\prime }\right\Vert _{F} \\
&&+\left\vert 1+\frac{\widehat{N}_{1}-N_{1}}{N_{1}}\right\vert
^{-1}\left\Vert \frac{U\left( \widehat{H^{c}}\right) ^{\prime }\underline{F}%
\Gamma \left( \widehat{H^{c}}\right) ^{\prime }}{N_{1}T_{0}}\right\Vert _{F}
\\
&&+\left\vert 1+\frac{\widehat{N}_{1}-N_{1}}{N_{1}}\right\vert
^{-1}\left\Vert \frac{\Gamma \left( \widehat{H^{c}}\right) \underline{F}%
^{\prime }U\left( \widehat{H^{c}}\right) }{N_{1}T_{0}}\right\Vert
_{F}+\left\vert 1+\frac{\widehat{N}_{1}-N_{1}}{N_{1}}\right\vert
^{-1}\left\Vert \frac{U\left( \widehat{H^{c}}\right) ^{\prime }U\left( 
\widehat{H^{c}}\right) }{N_{1}T_{0}}\right\Vert _{F} \\
&=&o_{p}\left( 1\right) \text{ as }N_{1},N_{2},T\rightarrow \infty \text{. }%
\square
\end{eqnarray*}

\noindent \textbf{Lemma D-11: }Let%
\begin{equation*}
\underset{N\times N}{A}=\text{ }\frac{\Gamma M_{FF}\Gamma ^{\prime }}{N_{1}}%
\text{ }
\end{equation*}%
where%
\begin{equation*}
M_{FF}=\frac{1}{T_{0}}\dsum\limits_{t=p}^{T}E\left[ \underline{F}_{t}%
\underline{F}_{t}^{\prime }\right] \text{ with }T_{0}=T-p+1\text{.}
\end{equation*}%
Suppose that Assumptions 3-1, 3-2(a)-(b), 3-2(d), 3-5, 3-6 and 3-7 hold; and
let $G$ be an $N\times N$ orthogonal matrix whose columns are the
eigenvectors of $A$. Under the assumed conditions, the following statements
are true.

\begin{enumerate}
\item[(a)] $Rank\left( A\right) =Kp$ for all $N_{1}$, $N_{2}$ sufficiently
large, and, hence, $0$ is an eigenvlaue of $A$ with algebraic multiplicity
equaling $N-Kp$.

\item[(b)] Partition $G$ as follows:%
\begin{equation*}
\underset{N\times N}{G}=\left[ 
\begin{array}{cc}
\underset{N\times Kp}{G_{1}} & \underset{N\times \left( N-Kp\right) }{G_{2}}%
\end{array}%
\right]
\end{equation*}%
Without loss of generality, suppose that the columns of $G_{1}$ are
eigenvectors associated with the non-zero eigenvalues of $A$, whereas $G_{2}$
contains the eigenvectors associated with the zero eigenvalue. Then, the
matrix $G^{\prime }AG$ can be partitioned as follows:%
\begin{equation}
G^{\prime }AG=\left( 
\begin{array}{cc}
\underset{Kp\times Kp}{\Lambda _{1}} & \underset{Kp\times \left( N-Kp\right) 
}{0} \\ 
\underset{\left( N-Kp\right) \times Kp}{0} & \underset{\left( N-Kp\right)
\times \left( N-Kp\right) }{\Lambda _{2}}%
\end{array}%
\right) =\left( 
\begin{array}{cc}
\underset{Kp\times Kp}{\Lambda _{1}} & \underset{Kp\times \left( N-Kp\right) 
}{0} \\ 
\underset{\left( N-Kp\right) \times Kp}{0} & \underset{\left( N-Kp\right)
\times \left( N-Kp\right) }{0}%
\end{array}%
\right) \text{. }  \label{spectral decomp}
\end{equation}%
where $\Lambda _{1}$ is a diagonal matrix whose diagonal elements are the
non-zero eigenvalues of $A$ and where $\Lambda _{2}=0$.

\item[(c)] Define the separation measure%
\begin{equation*}
\text{sep}\left( \Lambda _{1},\Lambda _{2}\right) =\min_{X{\Large \neq }0}%
\frac{\left\Vert \Lambda _{1}X-X\Lambda _{2}\right\Vert _{F}}{\left\Vert
X\right\Vert _{F}}\text{;}
\end{equation*}%
then, there exists a positive constant $\underline{c}$ such that%
\begin{equation*}
\text{sep}\left( \Lambda _{1},\Lambda _{2}\right) =\text{sep}\left( \Lambda
_{1},0\right) =\min_{X{\Large \neq }0}\frac{\left\Vert \Lambda
_{1}X\right\Vert _{F}}{\left\Vert X\right\Vert _{F}}\geq \lambda _{\min
}\left( \frac{M_{FF}^{1/2}\Gamma ^{\prime }\Gamma M_{FF}^{1/2}}{N_{1}}%
\right) \geq \underline{c}>0\text{.}
\end{equation*}
\end{enumerate}

\noindent \noindent

\noindent \textbf{Proof of Lemma D-11: }To show part (a), note first that,
by the result of Lemma D-4 above, there exists a positive constant $%
\underline{C}$ such that 
\begin{equation*}
\lambda _{\min }\left\{ M_{FF}\right\} \geq \underline{C}>0
\end{equation*}%
for all $T>p-1$; and, by Assumption 3-6, we have, 
\begin{equation*}
\lambda _{\min }\left( \frac{\Gamma ^{\prime }\Gamma }{N_{1}}\right) \geq 
\frac{1}{\overline{C}}\text{ for }N_{1},N_{2}\text{ sufficiently large}.
\end{equation*}%
for some constant $\overline{C}$ such that $0<\overline{C}<\infty $.
Combining these two inequalities, we see that%
\begin{eqnarray*}
\lambda _{\min }\left\{ \frac{M_{FF}^{1/2}\Gamma ^{\prime }\Gamma
M_{FF}^{1/2}}{N_{1}}\right\} &\geq &\lambda _{\min }\left( \frac{\Gamma
^{\prime }\Gamma }{N_{1}}\right) \lambda _{\min }\left\{ M_{FF}\right\} \\
&\geq &\frac{\underline{C}}{\overline{C}}>0\text{ for all }N_{1}\text{, }%
N_{2}\text{, and }T\text{ sufficiently large.}
\end{eqnarray*}%
This implies that the $Kp\times Kp$ matrix%
\begin{equation*}
\frac{M_{FF}^{1/2}\Gamma ^{\prime }\Gamma M_{FF}^{1/2}}{N_{1}}
\end{equation*}%
is a positive definite (and, therefore, also non-singular) for $N_{1}$, $%
N_{2}$, and $T$ sufficiently large. Moreover, observe that%
\begin{eqnarray}
&&\det \left\{ \lambda I_{N}-\frac{\Gamma M_{FF}\Gamma ^{\prime }}{N_{1}}%
\right\}  \notag \\
&=&\lambda ^{N}\det \left\{ I_{N}-\lambda ^{-1}\frac{\Gamma M_{FF}\Gamma
^{\prime }}{N_{1}}\right\}  \notag \\
&=&\lambda ^{N}\det \left\{ I_{Kp}-\lambda ^{-1}\text{ }\frac{%
M_{FF}^{1/2}\Gamma ^{\prime }\Gamma M_{FF}^{1/2}}{N_{1}}\right\} \text{\ }%
\left( \text{by Sylvester's determinantal theorem}\right)  \notag \\
&=&\lambda ^{N-Kp}\det \left\{ \lambda I_{Kp}-\text{ }\frac{%
M_{FF}^{1/2}\Gamma ^{\prime }\Gamma M_{FF}^{1/2}}{N_{1}}\right\}
\label{det identity}
\end{eqnarray}%
Hence, the non-zero eigenvalues of the matrix $\Gamma M_{FF}\Gamma ^{\prime
}/N_{1}$ correspond exactly to the eigenvalues of the positive definite
matrix $M_{FF}^{1/2}\Gamma ^{\prime }\Gamma M_{FF}^{1/2}/N_{1}$, from which
we further deduce that the matrix%
\begin{equation*}
A=\frac{\Gamma M_{FF}\Gamma ^{\prime }}{N_{1}}
\end{equation*}%
must be of rank $Kp$ for $N_{1}$, $N_{2}$, $T$ sufficiently large. Since $A$
is an $N\times N$ matrix with $N=N_{1}+N_{2}$, it follows immediately that $%
0 $ is an eigenvalue of $A$ with algebraic multiplicity equaling $N-Kp$ for $%
N_{1}$, $N_{2}$, $T$ sufficiently large.

To show part (b), let $\Lambda _{1}=diag\left( \lambda _{1,1},....,\lambda
_{1,Kp}\right) $, whose diagonal elements $\lambda _{1,i}>0$, for $i\in
\left\{ 1,...,Kp\right\} $, denote the non-zero eigenvalues of $A$ (which
must all be positive given that they correspond to the eigenvalues of the
positive definite matrix $M_{FF}^{1/2}\Gamma ^{\prime }\Gamma
M_{FF}^{1/2}/N_{1}$ as shown in the proof of part (a)). Moreover, let 
\begin{equation*}
\Lambda _{2}=\underset{\left( N-Kp\right) \times \left( N-Kp\right) }{0}
\end{equation*}%
whose diagonal elements are the $N-Kp$ zero eigenvalues of $A.$Since $A$ is
a symmetric matrix, the representation given in expression (\ref{spectral
decomp}) follows immediately from the usual spectral decomposition.

Finally, to show part (c), note that for any $Kp\times \left( N-Kp\right) $
matrix $X\neq 0$, we have\textbf{\ }%
\begin{eqnarray*}
\left\Vert \Lambda _{1}X-X\Lambda _{2}\right\Vert _{F} &=&\left\Vert \Lambda
_{1}X\right\Vert _{F}\text{ \ }\left( \text{since }\Lambda _{2}=0\right) \\
&=&\sqrt{tr\left\{ X^{\prime }\Lambda _{1}^{\prime }\Lambda _{1}X\right\} }
\\
&\geq &\lambda _{\min }\left( \Lambda _{1}\right) \sqrt{tr\left\{ X^{\prime
}X\right\} } \\
&=&\lambda _{\min }\left( \Lambda _{1}\right) \left\Vert X\right\Vert _{F}
\end{eqnarray*}%
It follows that%
\begin{eqnarray*}
\text{sep}\left( \Lambda _{1},\Lambda _{2}\right) &=&\min_{X{\Large \neq }0}%
\frac{\left\Vert \Lambda _{1}X-X\Lambda _{2}\right\Vert _{F}}{\left\Vert
X\right\Vert _{F}} \\
&=&\min_{X{\Large \neq }0}\frac{\left\Vert \Lambda _{1}X\right\Vert _{F}}{%
\left\Vert X\right\Vert _{F}}\text{ \ }\left( \text{since }\Lambda _{2}=0%
\text{ in this case}\right) \\
&\geq &\frac{\lambda _{\min }\left( \Lambda _{1}\right) \left\Vert
X\right\Vert _{F}}{\left\Vert X\right\Vert _{F}} \\
&=&\lambda _{\min }\left( \Lambda _{1}\right)
\end{eqnarray*}%
Furthermore, in light of expression (\ref{det identity}), the diagonal
elements of $\Lambda _{1}$, being the non-zero eigenvalues of $A$, must all
be the solutions of the determinantal equation%
\begin{equation*}
\det \left\{ \lambda I_{Kp}-\text{ }\frac{M_{FF}^{1/2}\Gamma ^{\prime
}\Gamma M_{FF}^{1/2}}{N_{1}}\right\} =0
\end{equation*}%
so that, as noted in the proof of part (a) above, they are also the
eigenvalues of the dual matrix $M_{FF}^{1/2}\Gamma ^{\prime }\Gamma
M_{FF}^{1/2}/N_{1}$. It follows from the proof of part (a) that there exists
a positive constant $\underline{c}$ such that for all $N_{1}$, $N_{2}$, and $%
T$ sufficiently large. 
\begin{eqnarray*}
\text{sep}\left( \Lambda _{1},\Lambda _{2}\right) &=&\text{sep}\left(
\Lambda _{1},0\right) \\
&\geq &\lambda _{\min }\left( \Lambda _{1}\right) \\
&=&\lambda _{\min }\left( \frac{M_{FF}^{1/2}\Gamma ^{\prime }\Gamma
M_{FF}^{1/2}}{N_{1}}\right) \\
&\geq &\underline{c}>0\text{. }\square
\end{eqnarray*}%
\textbf{\ }

\medskip

\noindent \textbf{Lemma D-12: }Suppose that $A$ and $E$ are both $n\times n$
symmetric matrices and that%
\begin{equation*}
G=\left[ 
\begin{array}{cc}
\underset{n{\large \times r}}{G_{1}} & \underset{{\large n}\times \left( n%
{\large -r}\right) }{G_{2}}%
\end{array}%
\right]
\end{equation*}%
is an orthogonal matrix such that%
\begin{equation*}
\text{ran}\left( G_{1}\right) =\left\{ y\in \mathbb{R}^{n}:y=G_{1}x\text{
for some }x\in \mathbb{R}^{r}\right\}
\end{equation*}%
is an invariant subspace for $A$, i.e., for any $\widetilde{q}\in $ ran$%
\left( G_{1}\right) $ and let $q^{\ast }=A\widetilde{q}$; then $q^{\ast }\in 
$ ran$\left( G_{1}\right) $. Partition the matrices $G^{\prime }AG$ and $%
G^{\prime }EG$ as follows:%
\begin{equation*}
G^{\prime }AG=\left( 
\begin{array}{cc}
\underset{r\times r}{\Lambda _{1}} & \underset{r\times \left( n-r\right) }{0}
\\ 
\underset{\left( N-r\right) \times r}{0} & \underset{\left( n-r\right)
\times \left( n-r\right) }{\Lambda _{2}}%
\end{array}%
\right) \text{ and }G^{\prime }EG=\left( 
\begin{array}{cc}
\underset{r\times r}{E_{11}} & \underset{r\times \left( n-r\right) }{%
E_{21}^{\prime }} \\ 
\underset{\left( n-r\right) \times r}{E_{21}} & \underset{\left( n-r\right)
\times \left( n-r\right) }{E_{22}}%
\end{array}%
\right) \text{. }
\end{equation*}%
If%
\begin{equation}
\text{sep}\left( \Lambda _{1},\Lambda _{2}\right) =\min_{X{\Large \neq }0}%
\frac{\left\Vert \Lambda _{1}X-X\Lambda _{2}\right\Vert _{F}}{\left\Vert
X\right\Vert _{F}}>0  \label{sep bd from zero}
\end{equation}%
and if 
\begin{eqnarray}
\left\Vert E\right\Vert _{2} &\leq &\frac{\text{sep}\left( \Lambda
_{1},\Lambda _{2}\right) }{5}  \notag \\
&=&\frac{1}{5}\min_{X{\Large \neq }0}\frac{\left\Vert \Lambda _{1}X-X\Lambda
_{2}\right\Vert _{F}}{\left\Vert X\right\Vert _{F}}\text{,}
\label{upper bd E}
\end{eqnarray}%
then, there exists a matrix $R\in \mathbb{R}^{\left( {\large n-r}\right)
\times {\large r}}$ satisfying%
\begin{eqnarray*}
\left\Vert R\right\Vert _{2} &\leq &\frac{4}{\text{sep}\left( \Lambda
_{1},\Lambda _{2}\right) }\left\Vert E_{21}\right\Vert _{2} \\
&=&4\left( \min_{X{\Large \neq }0}\frac{\left\Vert \Lambda _{1}X-X\Lambda
_{2}\right\Vert _{F}}{\left\Vert X\right\Vert _{F}}\right) ^{-1}\left\Vert
E_{21}\right\Vert _{2}
\end{eqnarray*}%
such that the columns of 
\begin{equation*}
\widehat{G}_{1}=\left( G_{1}+G_{2}R\right) \left( I_{r}+R^{\prime }R\right)
^{-1/2}
\end{equation*}%
define an orthonormal basis for a subspace that is invariant for $A+E$.

\medskip

\noindent \textbf{Remark:} Lemma D-12 is a well-known result in linear
algebra restated here in our notations. It is given in Golub and van Loan
(1996) as Theorem 8.1.10. As noted in Golub and van Loan (1996), this result
is also a slight adaptation of Theorem 4.11 in Stewart (1973), which could
be consulted for proof details.

\medskip

\noindent \textbf{Lemma D-13: }Let $\mathcal{X}$ be an invariant subspace of 
$A$, and let the columns of $X$ form a basis for $\mathcal{X}$. Then, there
is a unique matrix $L$ such that%
\begin{equation*}
AX=XL\text{.}
\end{equation*}%
The matrix $L$ is the representation of $A$ on $\mathcal{X}$ with respect to
the basis $X$. In particular, $\left( \upsilon ,\lambda \right) $ is an
eigenpair of $L$ if and only if $\left( X\upsilon ,\lambda \right) $ is an
eigenpair of $A$.

\noindent

\noindent \textbf{Proof of Lemma D-13: }This is Theorem 3.9 of Stewart and
Sun (1990). For a proof of this theorem, see Stewart and Sun (1990).

\medskip

A straightforward application of Lemma D-12 (or Theorem 8.1.10 of Golub and
van Loan, 1996) to our setting here leads to the following lemma.

\medskip

\noindent \textbf{Lemma D-14: }Let $\widehat{\Sigma }\left( \widehat{H^{c}}%
\right) $ be the post-variable-selection sample covariance matrix as defined
in expression (\ref{post-selection sample cov matrix}) in Lemma D-10.
Decompose $\widehat{\Sigma }\left( \widehat{H^{c}}\right) $ as follows:%
\begin{equation*}
\widehat{\Sigma }\left( \widehat{H^{c}}\right) =A+E\text{,}
\end{equation*}%
where 
\begin{equation}
A=\frac{\Gamma M_{FF}\Gamma ^{\prime }}{N_{1}}  \label{A matrix}
\end{equation}%
and where 
\begin{eqnarray}
E &=&\widehat{\Sigma }\left( \widehat{H^{c}}\right) -\frac{\Gamma
M_{FF}\Gamma ^{\prime }}{N_{1}}  \notag \\
&=&\left( \frac{\Gamma \left( \widehat{H^{c}}\right) M_{FF}\Gamma \left( 
\widehat{H^{c}}\right) ^{\prime }}{\widehat{N}_{1}}-\frac{\Gamma
M_{FF}\Gamma ^{\prime }}{N_{1}}\right) +\frac{1}{\widehat{N}_{1}}\Gamma
\left( \widehat{H^{c}}\right) \left[ \frac{\underline{F}^{\prime }\underline{%
F}}{T_{0}}-M_{FF}\right] \Gamma \left( \widehat{H^{c}}\right) ^{\prime } 
\notag \\
&&+\frac{U\left( \widehat{H^{c}}\right) ^{\prime }\underline{F}\Gamma \left( 
\widehat{H^{c}}\right) ^{\prime }}{\widehat{N}_{1}T_{0}}+\frac{\Gamma \left( 
\widehat{H^{c}}\right) \underline{F}^{\prime }U\left( \widehat{H^{c}}\right) 
}{\widehat{N}_{1}T_{0}}+\frac{U\left( \widehat{H^{c}}\right) ^{\prime
}U\left( \widehat{H^{c}}\right) }{\widehat{N}_{1}T_{0}},  \label{E matrix}
\end{eqnarray}%
with $T_{0}=T-p+1$ and%
\begin{equation*}
M_{FF}=\frac{1}{T_{0}}\dsum\limits_{t=p}^{T}E\left[ \underline{F}_{t}%
\underline{F}_{t}^{\prime }\right] \text{.}
\end{equation*}%
Suppose that Assumptions 3-1, 3-2, 3-3, 3-4 3-5, 3-6, 3-7, 3-8, 3-10, and
3-11* hold, and define%
\begin{equation*}
\underset{N\times N}{G}=\left[ 
\begin{array}{cc}
\underset{N\times Kp}{G_{1}} & \underset{N\times \left( N-Kp\right) }{G_{2}}%
\end{array}%
\right]
\end{equation*}%
to be an orthogonal matrix whose columns are the eigenvectors of the matrix $%
A$. Without loss of generality, suppose that the columns of $G_{1}$ are the
eigenvectors associated with the non-zero eigenvalues of $A$, whereas $G_{2}$
contains the eigenvectors associated with the zero eigenvalue which has an
algebraic multiplicity of $N-Kp$ in this case\footnote{%
That $0$ is an eigenvalue of the matrix%
\begin{equation*}
A=\frac{\Gamma M_{FF}\Gamma ^{\prime }}{N_{1}}
\end{equation*}%
with algebraic multiplicity equaling $N-Kp$ has already been shown
previously in Lemma D-11.}. Partition the matrices $G^{\prime }AG$ and $%
G^{\prime }EG$ as follows:%
\begin{eqnarray*}
G^{\prime }AG &=&\left( 
\begin{array}{cc}
\underset{Kp\times Kp}{\Lambda _{1}} & \underset{Kp\times \left( N-Kp\right) 
}{0} \\ 
\underset{\left( N-Kp\right) \times Kp}{0} & \underset{\left( N-Kp\right)
\times \left( N-Kp\right) }{\Lambda _{2}}%
\end{array}%
\right) =\left( 
\begin{array}{cc}
\underset{Kp\times Kp}{\Lambda _{1}} & \underset{Kp\times \left( N-Kp\right) 
}{0} \\ 
\underset{\left( N-Kp\right) \times Kp}{0} & \underset{\left( N-Kp\right)
\times \left( N-Kp\right) }{0}%
\end{array}%
\right) \text{ and} \\
G^{\prime }EG &=&\left( 
\begin{array}{cc}
\underset{Kp\times Kp}{E_{11}} & \underset{Kp\times \left( N-Kp\right) }{%
E_{21}^{\prime }} \\ 
\underset{\left( N-Kp\right) \times Kp}{E_{21}} & \underset{\left(
N-Kp\right) \times \left( N-Kp\right) }{E_{22}}%
\end{array}%
\right) \text{,}
\end{eqnarray*}%
where $\Lambda _{1}$ is a diagonal matrix whose diagonal elements are the $%
Kp $ largest eigevalues of the matrix $A$.\footnote{%
We have also previously shown in Lemma D-11 that $G^{\prime }AG$ can be
partitioned in the manner given here.}\qquad\ 

Under the assumed conditions, the following statements are true.

\begin{enumerate}
\item[(a)] There exists a $\left( N-Kp\right) \times Kp$ matrix $R$ such
that the columns of the matrix%
\begin{equation*}
\widehat{G}_{1}=\left( G_{1}+G_{2}R\right) \left( I_{Kp}+R^{\prime }R\right)
^{-1/2}
\end{equation*}%
define an orthonormal basis for a subspace that is invariant for $\widehat{%
\Sigma }\left( \widehat{H^{c}}\right) =A+E$. Moreover,%
\begin{equation*}
\left\Vert R\right\Vert _{2}=o_{p}\left( 1\right) \text{ as }N_{1},N_{2},%
\text{ and}T\rightarrow \infty
\end{equation*}

\item[(b)] $\left\Vert \widehat{G}_{1}-G_{1}\right\Vert _{2}=o_{p}\left(
1\right) $ as $N_{1},$ $N_{2}$, and $T\rightarrow \infty $

\item[(c)] The exists a unique symmetric matrix $L$ such that%
\begin{equation*}
\left( A+E\right) \widehat{G}_{1}=\widehat{G}_{1}L\text{. }
\end{equation*}%
Moreover, let 
\begin{equation}
\widehat{\Lambda }=diag\left( \widehat{\lambda }_{1},...,\widehat{\lambda }%
_{Kp}\right)  \label{LambdaL}
\end{equation}%
denote a diagonal matrix whose diagonal elements are the eigenvalues of the
matrix $L$, and let%
\begin{equation}
\widehat{V}=\left( 
\begin{array}{cccc}
\widehat{v}_{1} & \widehat{v}_{2} & \cdots & \widehat{v}_{Kp}%
\end{array}%
\right)  \label{V eigenvector matrix}
\end{equation}%
be a $Kp\times Kp$ matrix whose $\ell ^{th}$ column (i.e., $\widehat{v}%
_{\ell }$) is an eigenvector of $L$ associated with the eigenvalue $\widehat{%
\lambda }_{\ell }$ for $\ell =1,...,Kp$. Then, $\widehat{V}$ is an
orthogonal matrix and $\left( \widehat{G}_{1}\widehat{v}_{\ell },\widehat{%
\lambda }_{\ell }\right) $ is an eigenpair for the matrix $A+E$ for $\ell
=1,...,Kp$.

\item[(d)] The columns of the matrix%
\begin{equation*}
\widehat{G}_{1}\widehat{V}=\widehat{G}_{1}\left( 
\begin{array}{cccc}
\widehat{v}_{1} & \widehat{v}_{2} & \cdots & \widehat{v}_{Kp}%
\end{array}%
\right) =\left( 
\begin{array}{cccc}
\widehat{G}_{1}\widehat{v}_{1} & \widehat{G}_{1}\widehat{v}_{2} & \cdots & 
\widehat{G}_{1}\widehat{v}_{Kp}%
\end{array}%
\right)
\end{equation*}%
are the eigenvectors associated with the $Kp$ largest eigenvalues of the
post-variable-selection sample covariance matrix%
\begin{equation*}
A+E=\widehat{\Sigma }\left( \widehat{H^{c}}\right) \text{.}
\end{equation*}
\end{enumerate}

\medskip

\noindent \textbf{Proof of Lemm D-14: }

To show part (a), we first verify that the conditions (\ref{sep bd from zero}%
) and (\ref{upper bd E}) of Lemma D-12 are satisfied here. To proceed, let
ran$\left( G_{1}\right) $ denote the range space of $G_{1}$, i.e., 
\begin{equation*}
\text{ran}\left( G_{1}\right) =\left\{ g\in \mathbb{R}^{N}:g=G_{1}b\text{
for some }b\in \mathbb{R}^{Kp}\right\}
\end{equation*}%
and, by definition, $\Lambda _{1}$ is a $Kp\times Kp$ diagonal matrix whose
diagonal elements are the non-zero eigenvalues of the matrix $A=\Gamma
M_{FF}\Gamma ^{\prime }/N_{1}$. Now, for any $\widetilde{g}\in $ ran$\left(
G_{1}\right) $, note that%
\begin{eqnarray*}
g^{\ast } &=&A\widetilde{g} \\
&=&\left( \frac{\Gamma M_{FF}\Gamma ^{\prime }}{N_{1}}\right) G_{1}b \\
&=&G_{1}\Lambda _{1}b \\
&=&G_{1}b^{\ast }\text{ where }b^{\ast }=\Lambda _{1}b\text{.}
\end{eqnarray*}%
from which it follows that $g^{\ast }\in $ ran$\left( G_{1}\right) $, so
that ran$\left( G_{1}\right) $ is an invariant subspace of $A$. Next, by
applying the result of Lemma D-11, we have%
\begin{eqnarray*}
\text{sep}\left( \Lambda _{1},\Lambda _{2}\right) &=&\text{sep}\left(
\Lambda _{1},0\right) \\
&=&\min_{X{\Large \neq }0}\frac{\left\Vert \Lambda _{1}X\right\Vert _{F}}{%
\left\Vert X\right\Vert _{F}} \\
&\geq &\lambda _{\min }\left( \Lambda _{1}\right) \\
&=&\lambda _{\min }\left( \frac{M_{FF}^{1/2}\Gamma ^{\prime }\Gamma
M_{FF}^{1/2}}{N_{1}}\right) \\
&\geq &\underline{c}>0\text{ for }N_{1}\text{ and }N_{2}\text{ sufficiently
large,}
\end{eqnarray*}%
so that condition (\ref{sep bd from zero}) of Lemma D-12 is fulfilled. Next,
note that, from the result of Lemma D-10, we have%
\begin{equation*}
\left\Vert E\right\Vert _{2}=\left\Vert \widehat{\Sigma }\left( \widehat{%
H^{c}}\right) -\frac{\Gamma M_{FF}\Gamma ^{\prime }}{N_{1}}\right\Vert
_{2}=o_{p}\left( 1\right) \text{ \ as }N_{1}\text{, }N_{2}\text{, and }%
T\rightarrow 0\text{;}
\end{equation*}%
from which it follows that%
\begin{equation*}
\left\Vert E\right\Vert _{2}\leq \frac{\text{sep}\left( \Lambda
_{1},0\right) }{5}\text{ w.p.a.1 as }N_{1}\text{, }N_{2}\text{, and }%
T\rightarrow 0\text{.}
\end{equation*}%
so that condition (\ref{upper bd E}) of Lemma D-12 is also satisfied here
w.p.a.1. Hence, application of Lemma D-12 allows us to conclude that there
exists a $\left( N-Kp\right) \times Kp$ matrix $R$ such that the columns of
the matrix%
\begin{equation*}
\widehat{G}_{1}=\left( G_{1}+G_{2}R\right) \left( I_{Kp}+R^{\prime }R\right)
^{-1/2}
\end{equation*}%
define an orthonormal basis for a subspace that is invariant for $A+E$. In
addition,%
\begin{eqnarray*}
\left\Vert R\right\Vert _{2} &\leq &\frac{4}{\text{sep}\left( \Lambda
_{1},0\right) }\left\Vert E\right\Vert _{2} \\
&=&4\left[ \lambda _{\min }\left( \frac{M_{FF}^{1/2}\Gamma ^{\prime }\Gamma
M_{FF}^{1/2}}{N_{1}}\right) \right] ^{-1}\left\Vert E\right\Vert _{2} \\
&\leq &\frac{4}{\underline{c}}\left\Vert E\right\Vert _{2}\text{ \ }\left( 
\text{for some }\underline{c}>0\text{ by Assumption 3-6 and Lemma D-4}\right)
\\
&=&o_{p}\left( 1\right) \text{,}
\end{eqnarray*}%
which shows result (a).

To show that $\left\Vert \widehat{G}_{1}-G_{1}\right\Vert _{2}=o_{p}\left(
1\right) $, we first show that an explicit representation for $G_{1}$ can be
given as%
\begin{equation*}
G_{1}=\frac{\Gamma }{\sqrt{N_{1}}}\left( \frac{\Gamma ^{\prime }\Gamma }{%
N_{1}}\right) ^{-1/2}\Xi =\Gamma \left( \Gamma ^{\prime }\Gamma \right)
^{-1/2}\Xi \text{ }
\end{equation*}%
where $\Xi $ is an orthogonal matrix whose columns are eigenvectors of the
matrix%
\begin{equation*}
\underset{Kp\times Kp}{M_{FF}^{\ast }}=\left( \frac{\Gamma ^{\prime }\Gamma 
}{N_{1}}\right) ^{1/2}M_{FF}\left( \frac{\Gamma ^{\prime }\Gamma }{N_{1}}%
\right) ^{1/2}
\end{equation*}%
To see that this representation satisfies the various properties we require
of $G_{1}$, note first that%
\begin{equation*}
G_{1}^{\prime }G_{1}=\Xi ^{\prime }\left( \frac{\Gamma ^{\prime }\Gamma }{%
N_{1}}\right) ^{-1/2}\frac{\Gamma ^{\prime }\Gamma }{N_{1}}\left( \frac{%
\Gamma ^{\prime }\Gamma }{N_{1}}\right) ^{-1/2}\Xi \text{ }=I_{Kp}\text{;}
\end{equation*}%
hence, $G_{1}$ so represented does have orthonormal columns. Moreover, note
that%
\begin{eqnarray}
\frac{\Gamma M_{FF}\Gamma ^{\prime }}{N_{1}}G_{1} &=&\frac{\Gamma }{\sqrt{%
N_{1}}}M_{FF}\frac{\Gamma ^{\prime }\Gamma }{\sqrt{N_{1}}}\left( \Gamma
^{\prime }\Gamma \right) ^{-1/2}\Xi  \notag \\
&=&\frac{\Gamma }{\sqrt{N_{1}}}M_{FF}\frac{\Gamma ^{\prime }\Gamma }{N_{1}}%
\left( \frac{\Gamma ^{\prime }\Gamma }{N_{1}}\right) ^{-1/2}\Xi  \notag \\
&=&\frac{\Gamma }{\sqrt{N_{1}}}\left( \frac{\Gamma ^{\prime }\Gamma }{N_{1}}%
\right) ^{-1/2}\left( \frac{\Gamma ^{\prime }\Gamma }{N_{1}}\right)
^{1/2}M_{FF}\left( \frac{\Gamma ^{\prime }\Gamma }{N_{1}}\right)
^{1/2}\left( \frac{\Gamma ^{\prime }\Gamma }{N_{1}}\right) ^{-1/2}\frac{%
\Gamma ^{\prime }\Gamma }{N_{1}}\left( \frac{\Gamma ^{\prime }\Gamma }{N_{1}}%
\right) ^{-1/2}\Xi  \notag \\
&=&\frac{\Gamma }{\sqrt{N_{1}}}\left( \frac{\Gamma ^{\prime }\Gamma }{N_{1}}%
\right) ^{-1/2}M_{FF}^{\ast }\Xi \text{ \ }  \notag \\
&=&\Gamma \left( \Gamma ^{\prime }\Gamma \right) ^{-1/2}\Xi \Lambda _{1} 
\notag \\
&=&G_{1}\Lambda _{1}  \label{eigen eqn for A}
\end{eqnarray}%
where $\Lambda _{1}$ is a $Kp\times Kp$ diagonal matrix whose diagonal
elements are the eigenvalues of the matrix $M_{FF}^{\ast }$, which also
happen to be the non-zero eigenvalues of the matrix $A=\Gamma M_{FF}\Gamma
^{\prime }/N_{1}$. Pre-multiplying the above equation by $G_{1}^{\prime }$,
we obtain 
\begin{equation*}
G_{1}^{\prime }\frac{\Gamma M_{FF}\Gamma ^{\prime }}{N_{1}}%
G_{1}=G_{1}^{\prime }G_{1}\Lambda _{1}=\Lambda _{1}\text{.}
\end{equation*}%
Since equation (\ref{eigen eqn for A}) shows that the columns of $\Gamma
\left( \Gamma ^{\prime }\Gamma \right) ^{-1/2}\Xi $ are indeed the
eigenvectors of the matrix $A=\Gamma M_{FF}\Gamma ^{\prime }/N_{1}$, by the
argument given previously in the proof of part (a) above, we can then deduce
that ran$\left( G_{1}\right) $, the range space of $G_{1}$ with $%
G_{1}=\Gamma \left( \Gamma ^{\prime }\Gamma \right) ^{-1/2}\Xi $, is an
invariant subspace of $A$. It follows that setting

\begin{equation*}
G_{1}=\Gamma \left( \Gamma ^{\prime }\Gamma \right) ^{-1/2}\Xi \text{ }
\end{equation*}%
fulfills all the required properties of $G_{1}$ specified in Lemma D-12
above.

Next, write%
\begin{eqnarray*}
\widehat{G}_{1}-G_{1} &=&\left( G_{1}+G_{2}R\right) \left( I_{Kp}+R^{\prime
}R\right) ^{-1/2}-G_{1} \\
&=&G_{1}\left[ \left( I_{Kp}+R^{\prime }R\right) ^{-1/2}-I_{Kp}\right]
+G_{2}R\left( I_{Kp}+R^{\prime }R\right) ^{-1/2} \\
&=&\frac{\Gamma }{\sqrt{N_{1}}}\left( \frac{\Gamma ^{\prime }\Gamma }{N_{1}}%
\right) ^{-1/2}\Xi \left[ \left( I_{Kp}+R^{\prime }R\right) ^{-1/2}-I_{Kp}%
\right] +G_{2}R\left( I_{Kp}+R^{\prime }R\right) ^{-1/2}
\end{eqnarray*}%
Applying the submultiplicative property of matrix norms and the triangle
inequality, we obtain%
\begin{eqnarray*}
&&\left\Vert \widehat{G}_{1}-G_{1}\right\Vert _{2} \\
&\leq &\left\Vert \frac{\Gamma }{\sqrt{N_{1}}}\left( \frac{\Gamma ^{\prime
}\Gamma }{N_{1}}\right) ^{-1/2}\right\Vert _{2}\left\Vert \Xi \right\Vert
_{2}\left\Vert \left( I_{Kp}+R^{\prime }R\right) ^{-1/2}-I_{Kp}\right\Vert
_{2} \\
&&+\left\Vert G_{2}\right\Vert _{2}\left\Vert R\right\Vert _{2}\left\Vert
\left( I_{Kp}+R^{\prime }R\right) ^{-1/2}\right\Vert _{2} \\
&=&\left\Vert \left( I_{Kp}+R^{\prime }R\right) ^{-1/2}-I_{Kp}\right\Vert
_{2}+\left\Vert R\right\Vert _{2}\left\Vert \left( I_{Kp}+R^{\prime
}R\right) ^{-1/2}\right\Vert _{2}
\end{eqnarray*}%
where the last equality follows from the fact that%
\begin{eqnarray*}
\text{ }\left\Vert \Xi \right\Vert _{2} &=&\sqrt{\lambda _{\max }\left( \Xi
^{\prime }\Xi \right) }=\sqrt{\lambda _{\max }\left( I_{Kp}\right) }=1\text{,%
} \\
\left\Vert G_{2}^{\prime }\right\Vert _{2} &=&\sqrt{\lambda _{\max }\left(
G_{2}G_{2}^{\prime }\right) }=\sqrt{\lambda _{\max }\left( G_{2}^{\prime
}G_{2}\right) }=\sqrt{\lambda _{\max }\left( I_{N-Kp}\right) }=1\text{, and}
\\
\left\Vert \frac{\Gamma }{\sqrt{N_{1}}}\left( \frac{\Gamma ^{\prime }\Gamma 
}{N_{1}}\right) ^{-1/2}\right\Vert _{2} &=&\sqrt{\lambda _{\max }\left\{
\left( \frac{\Gamma ^{\prime }\Gamma }{N_{1}}\right) ^{-1/2}\frac{\Gamma
^{\prime }\Gamma }{N_{1}}\left( \frac{\Gamma ^{\prime }\Gamma }{N_{1}}%
\right) ^{-1/2}\right\} }=\sqrt{\lambda _{\max }\left\{ I_{Kp}\right\} }=1%
\text{.}
\end{eqnarray*}%
Now, if $\left( \lambda ,\rho \right) $ is an eigen-pair of $R^{\prime }R$
so that%
\begin{equation*}
R^{\prime }R\rho =\lambda \rho \text{ with }\lambda \geq 0\text{ given that }%
R^{\prime }R\text{ is positive semidefinite;}
\end{equation*}%
then,%
\begin{eqnarray*}
\left( I_{Kp}+R^{\prime }R\right) \rho &=&\left( 1+\lambda \right) \rho , \\
\left( I_{Kp}+R^{\prime }R\right) ^{-1/2}\rho &=&\frac{1}{\sqrt[\backslash ]{%
1+\lambda }}\rho \text{, and} \\
\left[ I_{Kp}-\left( I_{Kp}+R^{\prime }R\right) ^{-1/2}\right] \rho
&=&\left( I_{Kp}-\frac{1}{\sqrt[\backslash ]{1+\lambda }}I_{Kp}\right) \rho
\\
&=&\frac{\sqrt{1+\lambda }-1}{\sqrt[\backslash ]{1+\lambda }}\rho
\end{eqnarray*}%
since 
\begin{equation*}
\frac{1}{\sqrt[\backslash ]{1+\lambda }}\text{ is an eigenvalue of }\left(
I_{Kp}+R^{\prime }R\right) ^{-1/2}\text{ associated with the eigenvector }%
\rho
\end{equation*}%
and%
\begin{equation*}
\frac{\sqrt{1+\lambda }-1}{\sqrt[\backslash ]{1+\lambda }}\text{ is an
eigenvalue of }I_{Kp}-\left( I_{Kp}+R^{\prime }R\right) ^{-1/2}\text{
associated with the eigenvector }\rho
\end{equation*}%
Moreover, let%
\begin{equation*}
g\left( \lambda \right) =\frac{\sqrt{1+\lambda }-1}{\sqrt[\backslash ]{%
1+\lambda }}
\end{equation*}%
and note that%
\begin{eqnarray*}
g^{\prime }\left( \lambda \right) &=&\frac{1}{2}\frac{1}{1+\lambda }-\frac{1%
}{2}\frac{\sqrt{1+\lambda }-1}{\left( 1+\lambda \right) ^{3/2}} \\
&=&\frac{1}{2}\frac{\sqrt{1+\lambda }-\sqrt{1+\lambda }+1}{\left( 1+\lambda
\right) ^{3/2}} \\
&=&\frac{1}{2\left( 1+\lambda \right) ^{3/2}}>0
\end{eqnarray*}%
so that, in particular, $g\left( \lambda \right) $ is an increasing function
of $\lambda $ for $\lambda \geq 0$. It follows that%
\begin{eqnarray*}
&&\left\Vert \widehat{G}_{1}-G_{1}\right\Vert _{2} \\
&\leq &\left\Vert \left( I_{Kp}+R^{\prime }R\right)
^{-1/2}-I_{Kp}\right\Vert _{2}+\left\Vert R\right\Vert _{2}\left\Vert \left(
I_{Kp}+R^{\prime }R\right) ^{-1/2}\right\Vert _{2} \\
&=&\left\Vert I_{Kp}-\left( I_{Kp}+R^{\prime }R\right) ^{-1/2}\right\Vert
_{2}+\left\Vert R\right\Vert _{2}\left\Vert \left( I_{Kp}+R^{\prime
}R\right) ^{-1/2}\right\Vert _{2} \\
&=&\sqrt{\lambda _{\max }\left( \left[ I_{Kp}-\left( I_{Kp}+R^{\prime
}R\right) ^{-1/2}\right] ^{\prime }\left[ I_{Kp}-\left( I_{Kp}+R^{\prime
}R\right) ^{-1/2}\right] \right) } \\
&&+\left\Vert R\right\Vert _{2}\sqrt{\lambda _{\max }\left( \left(
I_{Kp}+R^{\prime }R\right) ^{-1/2\prime }\left( I_{Kp}+R^{\prime }R\right)
^{-1/2}\right) } \\
&=&\lambda _{\max }\left[ I_{Kp}-\left( I_{Kp}+R^{\prime }R\right) ^{-1/2}%
\right] +\left\Vert R\right\Vert _{2}\lambda _{\max }\left[ \left(
I_{Kp}+R^{\prime }R\right) ^{-1/2}\right] \\
&&\left( \text{since }I_{Kp}-\left( I_{Kp}+R^{\prime }R\right) ^{-1/2}\text{
and }\left( I_{Kp}+R^{\prime }R\right) ^{-1/2}\text{ are both symmetric and
positive semidefinite}\right) \\
&\leq &\frac{\sqrt{1+\lambda _{\max }\left( R^{\prime }R\right) }-1}{\sqrt[%
\backslash ]{1+\lambda _{\min }\left( R^{\prime }R\right) }}+\frac{%
\left\Vert R\right\Vert _{2}}{\sqrt[\backslash ]{1+\lambda _{\min }\left(
R^{\prime }R\right) }} \\
&\leq &\sqrt{1+\left\Vert R\right\Vert _{2}^{2}}-1+\left\Vert R\right\Vert
_{2}\text{ \ }\left( \text{since }\lambda _{\min }\left( R^{\prime }R\right)
\geq 0\text{ given that }R^{\prime }R\text{ is positive semi-definite}\right)
\\
&=&o_{p}\left( 1\right) \text{ as }N_{1},\text{ }N_{2}\text{, and }%
T\rightarrow \infty \text{ }\left( \text{since }\left\Vert R\right\Vert
_{2}=o_{p}\left( 1\right) \right) \text{. }
\end{eqnarray*}%
This shows result (b).

To show part (c), note that, by the result given in part (a) above, the
columns of $\widehat{G}_{1}=\left( G_{1}+G_{2}R\right) \left(
I_{r}+R^{\prime }R\right) ^{-1/2}$ form an orthonormal basis for a subspace
that is invariant for $A+E$. It then follows immediately from Lemma D-13
that there exists a unique matrix $L$ such that%
\begin{eqnarray*}
\left( A+E\right) \widehat{G}_{1} &=&\left( A+E\right) \left(
G_{1}+G_{2}R\right) \left( I_{r}+R^{\prime }R\right) ^{-1/2} \\
&=&\left( G_{1}+G_{2}R\right) \left( I_{r}+R^{\prime }R\right) ^{-1/2}L \\
&=&\widehat{G}_{1}L\text{. }
\end{eqnarray*}%
Note further that%
\begin{eqnarray*}
\widehat{G}_{1}^{\prime }\widehat{G}_{1} &=&\left( I_{Kp}+R^{\prime
}R\right) ^{-1/2}\left( G_{1}^{\prime }+R^{\prime }G_{2}^{\prime }\right)
\left( G_{1}+G_{2}R\right) \left( I_{Kp}+R^{\prime }R\right) ^{-1/2} \\
&=&\left( I_{Kp}+R^{\prime }R\right) ^{-1/2}\left( G_{1}^{\prime
}G_{1}+R^{\prime }G_{2}^{\prime }G_{1}+G_{1}^{\prime }G_{2}R+R^{\prime
}G_{2}^{\prime }G_{2}R\right) \left( I_{Kp}+R^{\prime }R\right) ^{-1/2} \\
&=&\left( I_{Kp}+R^{\prime }R\right) ^{-1/2}\left( I_{Kp}+R^{\prime
}R\right) \left( I_{Kp}+R^{\prime }R\right) ^{-1/2} \\
&&\left( \text{since by assumption }G=\left[ 
\begin{array}{cc}
G_{1} & G_{2}%
\end{array}%
\right] \text{ is an orthogonal matrix}\right) \\
&=&I_{Kp}
\end{eqnarray*}%
which, in turn, implies that%
\begin{eqnarray*}
\widehat{G}_{1}^{\prime }\left( A+E\right) \widehat{G}_{1} &=&\widehat{G}%
_{1}^{\prime }\left( \frac{\Gamma M_{FF}\Gamma ^{\prime }}{N_{1}}+E\right) 
\widehat{G}_{1}=\widehat{G}_{1}^{\prime }\widehat{G}_{1}L \\
&=&L
\end{eqnarray*}%
so that $L$ must be symmetric since, in our situation here,%
\begin{equation*}
A+E=\frac{\Gamma M_{FF}\Gamma ^{\prime }}{N_{1}}+\widehat{\Sigma }\left( 
\widehat{H^{c}}\right) -\frac{\Gamma M_{FF}\Gamma ^{\prime }}{N_{1}}=%
\widehat{\Sigma }\left( \widehat{H^{c}}\right) =\frac{Z\left( \widehat{H^{c}}%
\right) ^{\prime }Z\left( \widehat{H^{c}}\right) }{N_{1}T_{0}}
\end{equation*}%
is a symmetric matrix. Now, let $\widehat{\Lambda }=diag\left( \widehat{%
\lambda }_{1},...,\widehat{\lambda }_{Kp}\right) $ and%
\begin{equation*}
\widehat{V}=\left( 
\begin{array}{cccc}
\widehat{v}_{1} & \widehat{v}_{2} & \cdots & \widehat{v}_{Kp}%
\end{array}%
\right)
\end{equation*}%
be as defined in expressions (\ref{LambdaL}) and (\ref{V eigenvector matrix}%
). The fact that $L$ is symmetric implies that $\widehat{V}$ is an
orthogonal matrix. In addition, further application of Lemma C-13 shows that 
$\left( \widehat{G}_{1}\widehat{v}_{g},\widehat{\lambda }_{g}\right) $ is an
eigenpair for the matrix $A+E$ for $g=1,...,Kp$.

Finally, to show part (d), let $G=\left( 
\begin{array}{cc}
G_{1} & G_{2}%
\end{array}%
\right) $, and note that, by assumption, 
\begin{equation*}
G^{\prime }AG=\left( 
\begin{array}{cc}
G_{1}^{\prime }AG_{1} & G_{1}^{\prime }AG_{2} \\ 
G_{2}^{\prime }AG_{1} & G_{2}^{\prime }AG_{2}%
\end{array}%
\right) =\left( 
\begin{array}{cc}
\Lambda _{1} & 0 \\ 
0 & 0%
\end{array}%
\right) =\Lambda
\end{equation*}%
where $\Lambda _{1}=diag\left( \lambda _{1,1.},...,\lambda _{1,Kp}\right) $
contains the $Kp$ largest eigenvalues of $A$. Without loss of generality, we
can further assume that $\lambda _{1,1.},...,\lambda _{1,Kp}$ are ordered,
so that $\lambda _{1,j}=\lambda _{\left( j\right) }\left( A\right) $, i.e., $%
\lambda _{1,j}$ is the $j^{th}$ largest eigenvalue of $A$.\footnote{%
If this is not the case; then, we can always define a permutation matrix $%
\mathcal{P}$ such that 
\begin{equation*}
\Lambda ^{\ast }=\mathcal{P}^{\prime }\Lambda \mathcal{P}
\end{equation*}%
results in a diagonal matrix whose diagonal elements are repermutated in
such a way, so that the required ordering of the eigenvalues is satisfied.
Moreover, since $\mathcal{P}$ is an orthogonal matrix, it further follows
that%
\begin{equation*}
A=G\mathcal{PP}^{\prime }\Lambda \mathcal{PP}^{\prime }G^{\prime }=G\mathcal{%
P}\Lambda ^{\ast }\mathcal{P}^{\prime }G^{\prime }\text{.}
\end{equation*}%
Now, define $\widetilde{G}=G\mathcal{P}$, and note that $\widetilde{G}$ is
an orthogonal matrix whose columns are just the columns of $G$ repermutated.
Hence, we can simply proceed with our analysis using $\widetilde{G}$ in lieu
of $G$, and the associated eigenvalues will be in the order which we have
assumed.} Given that, $G^{\prime }G=GG^{\prime }=I_{N}$, we have 
\begin{equation*}
\left( 
\begin{array}{cc}
AG_{1} & AG_{2}%
\end{array}%
\right) =AG=G\Lambda =\left( 
\begin{array}{cc}
G_{1}\Lambda _{1} & 0%
\end{array}%
\right)
\end{equation*}%
from which it follows that%
\begin{equation}
AG_{1}G_{1}^{\prime }\widehat{G}_{1}\widehat{v}_{\ell }=G_{1}\Lambda
_{1}G_{1}^{\prime }\widehat{G}_{1}\widehat{v}_{\ell }\text{, for }\ell \in
\left\{ 1,...,Kp\right\} \text{.}  \label{AG1G1'G1hatv1hat}
\end{equation}%
Now, the result of part (c) above shows $\left( \widehat{G}_{1}\widehat{v}%
_{\ell },\widehat{\lambda }_{\ell }\right) $ to be an eigenpair of the
matrix $A+E$ for $\ell \in \left\{ 1,...,Kp\right\} $, so that 
\begin{equation}
\left( A+E\right) \widehat{G}_{1}\widehat{v}_{\ell }=\widehat{\lambda }%
_{\ell }\widehat{G}_{1}\widehat{v}_{\ell }\text{ for }\ell \in \left\{
1,...,Kp\right\}  \label{A+E eigen eqn}
\end{equation}%
where $\widehat{G}_{1}=\left( G_{1}+G_{2}R\right) \left( I_{Kp}+R^{\prime
}R\right) ^{-1/2}$ as given in the result for part (a). Multiplying both
sides of expression (\ref{A+E eigen eqn}) by $\widehat{v}_{\ell }^{\prime }%
\widehat{G}_{1}^{\prime }G_{1}G_{1}^{\prime }$, we get 
\begin{eqnarray}
\widehat{\lambda }_{\ell }\widehat{v}_{\ell }^{\prime }\widehat{G}%
_{1}^{\prime }G_{1}G_{1}^{\prime }\widehat{G}_{1}\widehat{v}_{\ell } &=&%
\widehat{v}_{\ell }^{\prime }\widehat{G}_{1}^{\prime }G_{1}G_{1}^{\prime
}\left( A+E\right) \widehat{G}_{1}\widehat{v}_{\ell }  \notag \\
&=&\widehat{v}_{\ell }^{\prime }\widehat{G}_{1}^{\prime }G_{1}G_{1}^{\prime
}A\widehat{G}_{1}\widehat{v}_{\ell }+\widehat{v}_{\ell }^{\prime }\widehat{G}%
_{1}^{\prime }G_{1}G_{1}^{\prime }E\widehat{G}_{1}\widehat{v}_{\ell }
\label{lambdavGhatG1G1'Ghatv}
\end{eqnarray}%
Since $A=\Gamma M_{FF}\Gamma ^{\prime }/N_{1}$ is symmetric, it further
follows by expression (\ref{AG1G1'G1hatv1hat}) that%
\begin{equation}
\widehat{v}_{\ell }^{\prime }\widehat{G}_{1}^{\prime }G_{1}G_{1}^{\prime }A=%
\widehat{v}_{\ell }^{\prime }\widehat{G}_{1}^{\prime }G_{1}G_{1}^{\prime
}A^{\prime }=\widehat{v}_{\ell }^{\prime }\widehat{G}_{1}^{\prime
}G_{1}\Lambda _{1}G_{1}^{\prime }  \label{vGhatG1G1'A}
\end{equation}%
Moreover, note that%
\begin{eqnarray*}
0 &\leq &\left( \widehat{v}_{\ell }^{\prime }\widehat{G}_{1}^{\prime
}G_{1}G_{1}^{\prime }E\widehat{G}_{1}\widehat{v}_{\ell }\right) ^{2} \\
&\leq &\left( \widehat{v}_{\ell }^{\prime }\widehat{G}_{1}^{\prime
}G_{1}G_{1}^{\prime }G_{1}G_{1}^{\prime }\widehat{G}_{1}\widehat{v}_{\ell
}\right) \left( \widehat{v}_{\ell }^{\prime }\widehat{G}_{1}^{\prime
}E^{\prime }E\widehat{G}_{1}\widehat{v}_{\ell }\right) \text{ }\left( \text{%
by CS inequality}\right) \\
&=&\left( \widehat{v}_{\ell }^{\prime }\widehat{G}_{1}^{\prime
}G_{1}G_{1}^{\prime }\widehat{G}_{1}\widehat{v}_{\ell }\right) \left( 
\widehat{v}_{\ell }^{\prime }\widehat{G}_{1}^{\prime }E^{\prime }E\widehat{G}%
_{1}\widehat{v}_{\ell }\right) \text{ }\left( \text{since }G_{1}^{\prime
}G_{1}=I_{Kp}\right) \\
&=&\left[ \widehat{v}_{\ell }^{\prime }\left( I_{Kp}+R^{\prime }R\right)
^{-1/2}\left( G_{1}^{\prime }+R^{\prime }G_{2}^{\prime }\right)
G_{1}G_{1}^{\prime }\left( G_{1}+G_{2}R\right) \left( I_{Kp}+R^{\prime
}R\right) ^{-1/2}\widehat{v}_{\ell }\right] \left( \widehat{v}_{\ell
}^{\prime }\widehat{G}_{1}^{\prime }E^{\prime }E\widehat{G}_{1}\widehat{v}%
_{\ell }\right) \\
&=&\left[ \widehat{v}_{\ell }^{\prime }\left( I_{Kp}+R^{\prime }R\right)
^{-1}\widehat{v}_{\ell }\right] \left( \widehat{v}_{\ell }^{\prime }\widehat{%
G}_{1}^{\prime }E^{\prime }E\widehat{G}_{1}\widehat{v}_{\ell }\right) \\
&\leq &\left[ \widehat{v}_{\ell }^{\prime }\left( I_{Kp}+R^{\prime }R\right)
^{-1}\widehat{v}_{\ell }\right] \lambda _{\max }\left( E^{\prime }E\right)
\end{eqnarray*}%
from which it follows that%
\begin{eqnarray}
-\sqrt{\widehat{v}_{\ell }^{\prime }\left( I_{Kp}+R^{\prime }R\right) ^{-1}%
\widehat{v}_{\ell }}\left\Vert E\right\Vert _{2} &=&-\sqrt{\widehat{v}_{\ell
}^{\prime }\left( I_{Kp}+R^{\prime }R\right) ^{-1}\widehat{v}_{\ell }}\sqrt{%
\lambda _{\max }\left( E^{\prime }E\right) }  \notag \\
&\leq &-\sqrt{\left( \widehat{v}_{\ell }^{\prime }\widehat{G}_{1}^{\prime
}G_{1}G_{1}^{\prime }E\widehat{G}_{1}\widehat{v}_{\ell }\right) ^{2}}  \notag
\\
&\leq &-\left\vert \widehat{v}_{\ell }^{\prime }\widehat{G}_{1}^{\prime
}G_{1}G_{1}^{\prime }E\widehat{G}_{1}\widehat{v}_{\ell }\right\vert  \notag
\\
&\leq &\widehat{v}_{\ell }^{\prime }\widehat{G}_{1}^{\prime
}G_{1}G_{1}^{\prime }E\widehat{G}_{1}\widehat{v}_{\ell }
\label{vG1hatG1G1'EG1hatv}
\end{eqnarray}%
where the last inequality follows from the fact that%
\begin{equation*}
\widehat{v}_{\ell }^{\prime }\widehat{G}_{1}^{\prime }G_{1}G_{1}^{\prime }E%
\widehat{G}_{1}\widehat{v}_{\ell }>-\left\vert \widehat{v}_{\ell }^{\prime }%
\widehat{G}_{1}^{\prime }G_{1}G_{1}^{\prime }E\widehat{G}_{1}\widehat{v}%
_{\ell }\right\vert \text{ if }\widehat{v}_{\ell }^{\prime }\widehat{G}%
_{1}^{\prime }G_{1}G_{1}^{\prime }E\widehat{G}_{1}\widehat{v}_{\ell }>0
\end{equation*}%
whereas%
\begin{equation*}
\widehat{v}_{\ell }^{\prime }\widehat{G}_{1}^{\prime }G_{1}G_{1}^{\prime }E%
\widehat{G}_{1}\widehat{v}_{\ell }=-\left\vert \widehat{v}_{\ell }^{\prime }%
\widehat{G}_{1}^{\prime }G_{1}G_{1}^{\prime }E\widehat{G}_{1}\widehat{v}%
_{\ell }\right\vert \text{ if }\widehat{v}_{\ell }^{\prime }\widehat{G}%
_{1}^{\prime }G_{1}G_{1}^{\prime }E\widehat{G}_{1}\widehat{v}_{\ell }\leq 0
\end{equation*}%
Combining expressions (\ref{lambdavGhatG1G1'Ghatv}), (\ref{vGhatG1G1'A}),
and (\ref{vG1hatG1G1'EG1hatv}), we see that%
\begin{eqnarray}
\widehat{\lambda }_{\ell }\widehat{v}_{\ell }^{\prime }\widehat{G}%
_{1}^{\prime }G_{1}G_{1}^{\prime }\widehat{G}_{1}\widehat{v}_{\ell } &=&%
\widehat{v}_{\ell }^{\prime }\widehat{G}_{1}^{\prime }G_{1}G_{1}^{\prime }A%
\widehat{G}_{1}\widehat{v}_{\ell }+\widehat{v}_{\ell }^{\prime }\widehat{G}%
_{1}^{\prime }G_{1}G_{1}^{\prime }E\widehat{G}_{1}\widehat{v}_{\ell }  \notag
\\
&\geq &\widehat{v}_{\ell }^{\prime }\widehat{G}_{1}^{\prime }G_{1}\Lambda
_{1}G_{1}^{\prime }\widehat{G}_{1}\widehat{v}_{\ell }-\sqrt{\widehat{v}%
_{\ell }^{\prime }\left( I_{Kp}+R^{\prime }R\right) ^{-1}\widehat{v}_{\ell }}%
\left\Vert E\right\Vert _{2}  \label{lambdavGhatG1G1'Ghatv bd}
\end{eqnarray}%
for $\ell \in \left\{ 1,...,Kp\right\} $. In addition, note that%
\begin{eqnarray*}
\widehat{v}_{\ell }^{\prime }\widehat{G}_{1}^{\prime }G_{1}G_{1}^{\prime }%
\widehat{G}_{1}\widehat{v}_{\ell } &=&\widehat{v}_{\ell }^{\prime }\widehat{G%
}_{1}^{\prime }G_{1}G_{1}^{\prime }\left( G_{1}+G_{2}R\right) \left(
I_{Kp}+R^{\prime }R\right) ^{-1/2}\widehat{v}_{\ell } \\
&=&\widehat{v}_{\ell }^{\prime }\widehat{G}_{1}^{\prime }G_{1}\left(
I_{Kp}+R^{\prime }R\right) ^{-1/2}\widehat{v}_{\ell } \\
&=&\widehat{v}_{\ell }^{\prime }\left( I_{Kp}+R^{\prime }R\right)
^{-1/2}\left( G_{1}^{\prime }+R^{\prime }G_{2}^{\prime }\right) G_{1}\left(
I_{Kp}+R^{\prime }R\right) ^{-1/2}\widehat{v}_{\ell } \\
&=&\widehat{v}_{\ell }^{\prime }\left( I_{Kp}+R^{\prime }R\right) ^{-1}%
\widehat{v}_{\ell } \\
&>&0
\end{eqnarray*}%
Hence, dividing both sides of expression (\ref{lambdavGhatG1G1'Ghatv bd}) by 
$\widehat{v}_{\ell }^{\prime }\widehat{G}_{1}^{\prime }G_{1}G_{1}^{\prime }%
\widehat{G}_{1}\widehat{v}_{\ell }$, we obtain%
\begin{eqnarray*}
\widehat{\lambda }_{\ell } &\geq &\frac{\widehat{v}_{\ell }^{\prime }%
\widehat{G}_{1}^{\prime }G_{1}\Lambda _{1}G_{1}^{\prime }\widehat{G}_{1}%
\widehat{v}_{\ell }}{\widehat{v}_{\ell }^{\prime }\widehat{G}_{1}^{\prime
}G_{1}G_{1}^{\prime }\widehat{G}_{1}\widehat{v}_{\ell }}-\frac{\sqrt{%
\widehat{v}_{\ell }^{\prime }\left( I_{Kp}+R^{\prime }R\right) ^{-1}\widehat{%
v}_{\ell }}\left\Vert E\right\Vert _{2}}{\widehat{v}_{\ell }^{\prime }%
\widehat{G}_{1}^{\prime }G_{1}G_{1}^{\prime }\widehat{G}_{1}\widehat{v}%
_{\ell }}\text{ } \\
&=&\widetilde{v}_{\ell }^{\prime }\Lambda _{1}\widetilde{v}_{\ell }-\frac{%
\sqrt{\widehat{v}_{\ell }^{\prime }\left( I_{Kp}+R^{\prime }R\right) ^{-1}%
\widehat{v}_{\ell }}\left\Vert E\right\Vert _{2}}{\widehat{v}_{\ell
}^{\prime }\left( I_{Kp}+R^{\prime }R\right) ^{-1}\widehat{v}_{\ell }} \\
&=&\widetilde{v}_{\ell }^{\prime }\Lambda _{1}\widetilde{v}_{\ell }-\frac{%
\left\Vert E\right\Vert _{2}}{\sqrt{\widehat{v}_{\ell }^{\prime }\left(
I_{Kp}+R^{\prime }R\right) ^{-1}\widehat{v}_{\ell }}} \\
&=&\dsum\limits_{j=1}^{Kp}\widetilde{v}_{\ell ,j}^{2}\lambda _{1,j}-\frac{%
\left\Vert E\right\Vert _{2}}{\sqrt{\widehat{v}_{\ell }^{\prime }\left(
I_{Kp}+R^{\prime }R\right) ^{-1}\widehat{v}_{\ell }}}
\end{eqnarray*}%
where%
\begin{equation*}
\widetilde{v}_{\ell }=\frac{G_{1}^{\prime }\widehat{G}_{1}\widehat{v}_{\ell }%
}{\widehat{v}_{\ell }^{\prime }\widehat{G}_{1}^{\prime }G_{1}G_{1}^{\prime }%
\widehat{G}_{1}\widehat{v}_{\ell }}\text{ so that }\left\Vert \widetilde{v}%
_{\ell }\right\Vert _{2}^{2}=\dsum\limits_{\ell =1}^{Kp}\widetilde{v}_{\ell
,j}^{2}=1\text{.}
\end{equation*}%
Note also that%
\begin{eqnarray*}
\widehat{v}_{\ell }^{\prime }\left( I_{Kp}+R^{\prime }R\right) ^{-1}\widehat{%
v}_{\ell } &\geq &\lambda _{\min }\left\{ \left( I_{Kp}+R^{\prime }R\right)
^{-1}\right\} \widehat{v}_{\ell }^{\prime }\widehat{v}_{\ell } \\
&=&\lambda _{\min }\left\{ \left( I_{Kp}+R^{\prime }R\right) ^{-1}\right\} 
\text{ \ }\left( \text{since }\left\Vert \widehat{v}_{\ell }\right\Vert
_{2}^{2}=1\right) \\
&=&\frac{1}{\lambda _{\max }\left( I_{Kp}+R^{\prime }R\right) } \\
&\geq &\frac{1}{1+\lambda _{\max }\left( R^{\prime }R\right) } \\
&=&\frac{1}{1+\left\Vert R\right\Vert _{2}^{2}} \\
&\geq &\left[ 1+\frac{16\left\Vert E_{21}\right\Vert _{2}^{2}}{\left( \text{%
sep}\left( \Lambda _{1},\Lambda _{2}\right) \right) ^{2}}\right] ^{-1}\text{
\ }\left( \text{by Lemma D-12}\right) \\
&\geq &\left[ 1+\frac{16\left\Vert E\right\Vert _{2}^{2}}{\left( \text{sep}%
\left( \Lambda _{1},\Lambda _{2}\right) \right) ^{2}}\right] ^{-1}\text{ \ }%
\left( \text{by Lemma D-3 }\right) \\
&\geq &\left[ 1+\frac{16\left( \text{sep}\left( \Lambda _{1},\Lambda
_{2}\right) \right) ^{2}/25}{\left( \text{sep}\left( \Lambda _{1},\Lambda
_{2}\right) \right) ^{2}}\right] ^{-1}\text{ \ }\left( \text{by Lemma D-12}%
\right) \\
&=&\frac{25}{41}
\end{eqnarray*}%
Making use of this lower bound, we obtain 
\begin{eqnarray*}
\widehat{\lambda }_{\ell } &\geq &\dsum\limits_{j=1}^{Kp}\widetilde{v}_{\ell
,j}^{2}\lambda _{1,j}-\frac{\left\Vert E\right\Vert _{2}}{\sqrt{\widehat{v}%
_{\ell }^{\prime }\left( I_{Kp}+R^{\prime }R\right) ^{-1}\widehat{v}_{\ell }}%
} \\
&=&\dsum\limits_{j=1}^{Kp}\widetilde{v}_{\ell ,j}^{2}\lambda _{1,j}-\frac{25%
}{41}\left\Vert E\right\Vert _{2}\text{.}
\end{eqnarray*}%
Next, recall the notations we have introduced previously on the ordering of
the eigenvalues of the matrices $A+E$ and $A$, i.e.,%
\begin{eqnarray*}
\lambda _{\left( 1\right) }\left( A+E\right) &\geq &\cdot \cdot \cdot \geq
\lambda _{\left( Kp\right) }\left( A+E\right) \geq \lambda _{\left(
Kp+1\right) }\left( A+E\right) \geq \cdot \cdot \cdot \geq \lambda _{\left(
N\right) }\left( A+E\right) \text{,} \\
\lambda _{\left( 1\right) }\left( A\right) &\geq &\cdot \cdot \cdot \geq
\lambda _{\left( Kp\right) }\left( A\right) \geq \lambda _{\left(
Kp+1\right) }\left( A\right) \geq \cdot \cdot \cdot \geq \lambda _{\left(
N\right) }\left( A\right)
\end{eqnarray*}%
Since $A=\Gamma M_{FF}\Gamma ^{\prime }\NEG{/}N_{1}$ and since part (a) of
Lemma D-11 shows that $Rank\left( A\right) =Kp$ for all $N_{1}$, $N_{2}$,
and $T$ sufficiently large; it follows that%
\begin{equation}
\lambda _{\left( Kp+1\right) }\left( A\right) =\cdot \cdot \cdot =\lambda
_{\left( N\right) }\left( A\right) =0\text{.}  \label{zero eigenvalues}
\end{equation}

\noindent In addition, by Corollary 8.1.6 of Golub and van Loan (1996), we
have the inequality.%
\begin{equation}
\lambda _{\left( Kp+1\right) }\left( A+E\right) \leq \lambda _{\left(
Kp+1\right) }\left( A\right) +\left\Vert E\right\Vert _{2}\text{.}
\label{Corollary 8.1.6}
\end{equation}%
Making use of expressions (\ref{zero eigenvalues}) and (\ref{Corollary 8.1.6}%
); we see that, for any $\ell \in \left\{ 1,...,Kp\right\} $,%
\begin{eqnarray*}
\widehat{\lambda }_{\ell }-\lambda _{\left( Kp+1\right) }\left( A+E\right)
&\geq &\dsum\limits_{j=1}^{Kp}\widetilde{v}_{\ell ,j}^{2}\lambda _{1,j}-%
\frac{25}{41}\left\Vert E\right\Vert _{2}-\left\{ \lambda _{\left(
Kp+1\right) }\left( A\right) +\left\Vert E\right\Vert _{2}\right\} \\
&=&\dsum\limits_{j=1}^{Kp}\widetilde{v}_{\ell ,j}^{2}\lambda _{1,j}-\frac{66%
}{41}\left\Vert E\right\Vert _{2}\text{ \ }\left( \text{since }\lambda
_{\left( Kp+1\right) }\left( A\right) =0\text{ here}\right) \\
&=&\dsum\limits_{j=1}^{Kp}\widetilde{v}_{\ell ,j}^{2}\lambda _{\left(
j\right) }\left( A\right) -\frac{66}{41}\left\Vert E\right\Vert _{2}\text{ }
\\
&&\left( \text{since }\lambda _{1,j}=\lambda _{\left( j\right) }\left(
A\right) \text{ as discussed previously}\right) \\
&\geq &\dsum\limits_{j=1}^{Kp}\widetilde{v}_{\ell ,j}^{2}\lambda _{\left(
j\right) }\left( A\right) -\frac{66}{41}\frac{\text{sep}\left( \Lambda
_{1},\Lambda _{2}\right) }{5}\text{ \ \ \ \ }\left( \text{by Lemma D-12}%
\right) \\
&=&\dsum\limits_{j=1}^{Kp}\widetilde{v}_{\ell ,j}^{2}\lambda _{\left(
j\right) }\left( A\right) -\frac{66}{205}\text{sep}\left( \Lambda
_{1},0\right) \text{ }\left( \text{since }\Lambda _{2}=0\text{ here}\right)
\\
&\geq &\lambda _{\min }\left( \Lambda _{1}\right) --\frac{66}{205}\text{sep}%
\left( \Lambda _{1},0\right) \text{ }\left( \text{since }\Lambda
_{1}=diag\left( \lambda _{\left( 1\right) }\left( A\right) ,...,\lambda
_{\left( Kp\right) }\left( A\right) \right) \right) \\
&=&\frac{139}{205}\text{sep}\left( \Lambda _{1},0\right) \text{ } \\
&&\left( \text{since sep}\left( \Lambda _{1},0\right) =\lambda _{\min
}\left( \Lambda _{1}\right) \text{ by Theorem 3.1 of Stewart and Sun (1990)}%
\right) \\
&\geq &\frac{139}{205}\underline{c}>0\text{ \ }\left( \text{by part (c) of
Lemma D-11}\right) \text{. }
\end{eqnarray*}%
This shows that the set $\left\{ \widehat{\lambda }_{1},...,\widehat{\lambda 
}_{Kp}\right\} $ contains the $Kp$ largest eigenvalues of the matrix $A+E$.
It further follows from the result given in part (c) that the columns of the
matrix%
\begin{equation*}
\widehat{G}_{1}\widehat{V}=\widehat{G}_{1}\left( 
\begin{array}{cccc}
\widehat{v}_{1} & \widehat{v}_{2} & \cdots & \widehat{v}_{Kp}%
\end{array}%
\right) =\left( 
\begin{array}{cccc}
\widehat{G}_{1}\widehat{v}_{1} & \widehat{G}_{1}\widehat{v}_{2} & \cdots & 
\widehat{G}_{1}\widehat{v}_{Kp}%
\end{array}%
\right)
\end{equation*}%
are the eigenvectors associated with the $Kp$ largest eigenvalues of the
matrix $A+E$. $\square $

\medskip

\noindent \textbf{Lemma D-15: }Suppose that Assumptions 3-1, 3-2, 3-3, 3-4,
3-5, 3-6, 3-7, 3-8, 3-9, 3-10, and 3-11* hold. Then, the following
statements are true.

\begin{enumerate}
\item[(a)] 
\begin{equation*}
\frac{\widehat{N}_{1}-N_{1}}{N_{1}}\overset{p}{\rightarrow }0
\end{equation*}

\item[(b)] 
\begin{equation*}
\left\Vert \frac{\Gamma \left( \widehat{H^{c}}\right) -\Gamma }{\sqrt{N_{1}}}%
\right\Vert _{2}\overset{p}{\rightarrow }0
\end{equation*}

\item[(c)] Let 
\begin{equation*}
\widehat{G}_{1}=\left( G_{1}+G_{2}R\right) \left( I_{Kp}+R^{\prime }R\right)
^{-1/2}
\end{equation*}%
where $G_{1}$, $G_{2}$, and $R$ are as defined in Lemma D-14 above. Also,
let $\widehat{V}$ be the $Kp\times Kp$ orthogonal matrix given in expression
(\ref{V eigenvector matrix}) of Lemma D-14. Then, there exists some positive
constant $\overline{C}$ such that 
\begin{equation*}
\left\Vert \frac{\widehat{V}^{\prime }\widehat{G}_{1}^{\prime }\Gamma }{%
\sqrt{N_{1}}}\right\Vert _{2}\leq \sqrt{\lambda _{\max }\left( \frac{\Gamma
^{\prime }\Gamma }{N_{1}}\right) }\leq \overline{C}<\infty
\end{equation*}%
for $N_{1},N_{2}$, and $T$ sufficiently large. In addition,%
\begin{equation*}
\left\Vert \frac{\widehat{V}^{\prime }\widehat{G}_{1}^{\prime }\Gamma }{%
\sqrt{N_{1}}}-Q^{\prime }\right\Vert _{2}\overset{p}{\rightarrow }0
\end{equation*}%
where%
\begin{equation*}
Q=\left( \frac{\Gamma ^{\prime }\Gamma }{N_{1}}\right) ^{\frac{{\Large 1}}{%
{\Large 2}}}\Xi \widehat{V}\text{,}
\end{equation*}%
with $\Xi $ being the $Kp\times Kp$ orthogonal matrix whose columns are the
eigenvectors of the matrix 
\begin{equation*}
M_{FF}^{\ast }=\left( \frac{\Gamma ^{\prime }\Gamma }{N_{1}}\right)
^{1/2}M_{FF}\left( \frac{\Gamma ^{\prime }\Gamma }{N_{1}}\right)
^{1/2}=\left( \frac{\Gamma ^{\prime }\Gamma }{N_{1}}\right) ^{1/2}\frac{1}{%
T-p+1}\dsum\limits_{t=p}^{T}E\left[ \underline{F}_{t}\underline{F}%
_{t}^{\prime }\right] \left( \frac{\Gamma ^{\prime }\Gamma }{N_{1}}\right)
^{1/2}\text{.}
\end{equation*}

\item[(d)] For all fixed index $t$%
\begin{equation*}
\left\Vert \frac{G_{1}^{\prime }U_{t,N}\left( \widehat{H^{c}}\right) }{\sqrt{%
N_{1}}}\right\Vert _{2}=o_{p}\left( 1\right) \text{.}
\end{equation*}

\item[(e)] For all fixed index $t$%
\begin{equation*}
\left\Vert \frac{U_{t,N}\left( \widehat{H^{c}}\right) }{\sqrt{N_{1}}}%
\right\Vert _{2}^{2}=O_{p}\left( 1\right) \text{.}
\end{equation*}

\item[(f)] For all fixed index $t$,%
\begin{equation*}
\left\Vert \frac{G_{2}^{\prime }U_{t,N}\left( \widehat{H^{c}}\right) }{\sqrt{%
N_{1}}}\right\Vert _{2}=O_{p}\left( 1\right) \text{.}
\end{equation*}

\item[(g)] Let 
\begin{equation*}
\widehat{G}_{1}=\left( G_{1}+G_{2}R\right) \left( I_{Kp}+R^{\prime }R\right)
^{-1/2}
\end{equation*}%
where $G_{1}$, $G_{2}$, and $R$ are as defined in Lemma D-14 above. Also,
let $\widehat{V}$ be the $Kp\times Kp$ orthogonal matrix given in expression
(\ref{V eigenvector matrix}) of Lemma D-14. Then, for all fixed index $t$,%
\begin{equation*}
\left\Vert \frac{\widehat{V}^{\prime }\widehat{G}_{1}^{\prime }U_{t,N}\left( 
\widehat{H^{c}}\right) }{\sqrt{\widehat{N}_{1}}}\right\Vert _{2}\overset{p}{%
\rightarrow }0\text{ as }N_{1}\text{, }N_{2}\text{, and }T\rightarrow \infty 
\text{. }
\end{equation*}

\item[(h)] 
\begin{equation*}
\left\Vert \frac{\Gamma \left( \widehat{H^{c}}\right) ^{{\Large \prime }}%
\widehat{G}_{1}\widehat{V}}{\sqrt{\widehat{N}_{1}}}-Q\right\Vert
_{2}=\left\Vert \frac{\widehat{V}^{\prime }\widehat{G}_{1}^{{\Large \prime }%
}\Gamma \left( \widehat{H^{c}}\right) }{\sqrt{\widehat{N}_{1}}}-Q^{\prime
}\right\Vert _{2}=o_{p}\left( 1\right) \text{ as }N_{1},N_{2},T\rightarrow
\infty \text{.}
\end{equation*}%
where $Q$ is as defined in part (c) above.

\item[(i)] 
\begin{equation*}
\left\Vert \underline{F}_{t}\right\Vert _{2}=O_{p}\left( 1\right) \text{ for
all }t\text{.}
\end{equation*}

\item[(j)] 
\begin{equation*}
\left\Vert \widehat{\underline{F}}_{T}-Q^{\prime }\underline{F}%
_{T}\right\Vert _{2}=o_{p}\left( 1\right) \text{ as }N_{1}\text{, }N_{2}%
\text{, and }T\rightarrow \infty
\end{equation*}%
where $\widehat{\underline{F}}_{T}$ denotes the principal component
estimator of the factor vector $\underline{F}_{T}$ obtained after the
variables have been pre-screened based on the decision rule 
\begin{equation*}
i\in \left\{ 
\begin{array}{cc}
\widehat{H}^{c} & \text{ if }\mathbb{S}_{i,T}^{+}>\Phi ^{-1}\left( 1-\frac{%
\varphi }{2N}\right) \\ 
\widehat{H} & \text{if }\mathbb{S}_{i,T}^{+}\leq \Phi ^{-1}\left( 1-\frac{%
\varphi }{2N}\right)%
\end{array}%
\right. \text{,}
\end{equation*}%
as described in section 3. Here, $\mathbb{S}_{i,T}^{+}$ may be either the
statistic $\dsum\nolimits_{\ell =1}^{d}\varpi _{\ell }\left\vert S_{i,\ell
,T}\right\vert $or the statistic $\max_{1\leq \ell \leq d}\left\vert
S_{i,\ell ,T}\right\vert $.
\end{enumerate}

\medskip

\noindent \textbf{Proof of Lemma D-15:}

To show part (a), note first that, for any $\epsilon >0$,%
\begin{eqnarray*}
\left\{ \left\vert \frac{\widehat{N}_{1}-N_{1}}{N_{1}}\right\vert \geq
\epsilon \right\} &=&\left\{ \left\vert \frac{1}{N_{1}}\dsum\limits_{i=1}^{N}%
\mathbb{I}\left\{ i\in \widehat{H^{c}}\right\} -1\right\vert \geq \epsilon
\right\} \\
&=&\left\{ \left\vert \frac{1}{N_{1}}\dsum\limits_{i\in H^{{\large c}}}%
\mathbb{I}\left\{ i\in \widehat{H^{c}}\right\} +\frac{1}{N_{1}}%
\dsum\limits_{i\in H}\mathbb{I}\left\{ i\in \widehat{H^{c}}\right\}
-1\right\vert \geq \epsilon \right\} \\
&\subseteq &\left\{ \left\vert \frac{1}{N_{1}}\dsum\limits_{i\in H^{{\large c%
}}}\mathbb{I}\left\{ i\in \widehat{H^{c}}\right\} -1\right\vert +\left\vert 
\frac{1}{N_{1}}\dsum\limits_{i\in H}\mathbb{I}\left\{ i\in \widehat{H^{c}}%
\right\} \right\vert \geq \epsilon \right\} \\
&\subseteq &\left\{ \left\vert \frac{1}{N_{1}}\dsum\limits_{i\in H^{{\large c%
}}}\mathbb{I}\left\{ i\in \widehat{H^{c}}\right\} -1\right\vert \geq \frac{%
\epsilon }{2}\right\} \cup \left\{ \left\vert \frac{1}{N_{1}}%
\dsum\limits_{i\in H}\mathbb{I}\left\{ i\in \widehat{H^{c}}\right\}
\right\vert \geq \frac{\epsilon }{2}\right\} \\
&=&\left\{ \left\vert \frac{1}{N_{1}}\dsum\limits_{i\in H^{{\large c}%
}}\left( \mathbb{I}\left\{ i\in \widehat{H^{c}}\right\} -1\right)
\right\vert \geq \frac{\epsilon }{2}\right\} \cup \left\{ \left\vert \frac{1%
}{N_{1}}\dsum\limits_{i\in H}\mathbb{I}\left\{ i\in \widehat{H^{c}}\right\}
\right\vert \geq \frac{\epsilon }{2}\right\}
\end{eqnarray*}%
By Markov's inequality, we have%
\begin{eqnarray*}
&&\Pr \left( \left\vert \frac{1}{N_{1}}\dsum\limits_{i\in H^{{\large c}%
}}\left( \mathbb{I}\left\{ i\in \widehat{H^{c}}\right\} -1\right)
\right\vert \geq \frac{\epsilon }{2}\right) \\
&=&\Pr \left( \left\vert \frac{1}{N_{1}}\dsum\limits_{i\in H^{{\large c}%
}}\left( \mathbb{I}\left\{ i\in \widehat{H^{c}}\right\} -1\right)
\right\vert ^{2}\geq \frac{\epsilon ^{2}}{4}\right) \\
&\leq &\frac{4}{\epsilon ^{2}}E\left\{ \left\vert \frac{1}{N_{1}}%
\dsum\limits_{i\in H^{{\large c}}}\left( \mathbb{I}\left\{ i\in \widehat{%
H^{c}}\right\} -1\right) \right\vert ^{2}\right\} \\
&=&\frac{4}{\epsilon ^{2}}\frac{1}{N_{1}^{2}}\dsum\limits_{i\in H^{{\large c}%
}}\dsum\limits_{k\in H^{{\large c}}}E\left[ \left( \mathbb{I}\left\{ i\in 
\widehat{H^{c}}\right\} -1\right) \left( \mathbb{I}\left\{ k\in \widehat{%
H^{c}}\right\} -1\right) \right] \\
&=&\frac{4}{\epsilon ^{2}}\frac{1}{N_{1}^{2}}\dsum\limits_{i\in H^{{\large c}%
}}\dsum\limits_{k\in H^{{\large c}}}\left( E\left[ \mathbb{I}\left\{ i\in 
\widehat{H^{c}}\right\} \mathbb{I}\left\{ k\in \widehat{H^{c}}\right\} %
\right] -E\left[ \mathbb{I}\left\{ k\in \widehat{H^{c}}\right\} \right] -E%
\left[ \mathbb{I}\left\{ i\in \widehat{H^{c}}\right\} \right] +1\right) \\
&=&\frac{4}{\epsilon ^{2}}\frac{1}{N_{1}^{2}}\dsum\limits_{i\in H^{{\large c}%
}}\dsum\limits_{k\in H^{{\large c}}}\left\{ \Pr \left( \left\{ i\in \widehat{%
H^{c}}\right\} \cap \left\{ k\in \widehat{H^{c}}\right\} \right) -\Pr \left(
k\in \widehat{H^{c}}\right) \right\} \\
&&+\frac{4}{\epsilon ^{2}}\frac{1}{N_{1}^{2}}\dsum\limits_{i\in H^{{\large c}%
}}\dsum\limits_{k\in H^{{\large c}}}\left\{ 1-\Pr \left( i\in \widehat{H^{c}}%
\right) \right\} \\
&\leq &\frac{4}{\epsilon ^{2}}\frac{1}{N_{1}^{2}}\dsum\limits_{i\in H^{%
{\large c}}}\dsum\limits_{k\in H^{{\large c}}}\left\{ \Pr \left( k\in 
\widehat{H^{c}}\right) -\Pr \left( k\in \widehat{H^{c}}\right) \right\} +%
\frac{4}{\epsilon ^{2}}\frac{1}{N_{1}}\dsum\limits_{i\in H^{{\large c}%
}}\left\{ 1-\Pr \left( i\in \widehat{H^{c}}\right) \right\} \\
&\leq &\frac{4}{\epsilon ^{2}}\frac{1}{N_{1}}\dsum\limits_{i\in H^{{\large c}%
}}\left\{ 1-\min_{i\in H^{{\large c}}}\Pr \left( i\in \widehat{H^{c}}\right)
\right\} \rightarrow 0\text{ as }N_{1},N_{2}\text{, and }T\rightarrow \infty 
\text{.}
\end{eqnarray*}%
where the last line above follows from the fact that, for $i\in H^{c}$ and
for either the case where $\mathbb{S}_{i,T}^{+}=\dsum\nolimits_{\ell
=1}^{d}\varpi _{\ell }\left\vert S_{i,\ell ,T}\right\vert $ or the case
where $\mathbb{S}_{i,T}^{+}=\max_{1\leq \ell \leq d}\left\vert S_{i,\ell
,T}\right\vert $, we can apply the results of Theorem 2 in Chao and Swanson
(2022a) to obtain%
\begin{eqnarray*}
\min_{i\in H^{{\large c}}}\Pr \left( i\in \widehat{H^{c}}\right) &\geq &\Pr
\left( \dbigcap\limits_{i\in H^{{\large c}}}\left\{ \mathbb{S}%
_{i,T}^{+}>\Phi ^{-1}\left( 1-\frac{\varphi }{2N}\right) \right\} \right) \\
&=&P\left( \min_{{\large i\in }H^{{\large c}}}\mathbb{S}_{i,T}^{+}>\Phi
^{-1}\left( 1-\frac{\varphi }{2N}\right) \right) \\
&\rightarrow &1
\end{eqnarray*}%
Also, making use of Markov's inequality, we obtain, for either the case
where $\mathbb{S}_{i,T}^{+}=\dsum\nolimits_{\ell =1}^{d}\varpi _{\ell
}\left\vert S_{i,\ell ,T}\right\vert $ or the case where $\mathbb{S}%
_{i,T}^{+}=\max_{1\leq \ell \leq d}\left\vert S_{i,\ell ,T}\right\vert $,%
\begin{eqnarray*}
&&\Pr \left( \left\vert \frac{1}{N_{1}}\dsum\limits_{i\in H}\mathbb{I}%
\left\{ i\in \widehat{H^{c}}\right\} \right\vert \geq \frac{\epsilon }{2}%
\right) \\
&=&\Pr \left( \frac{1}{N_{1}}\dsum\limits_{i\in H}\mathbb{I}\left\{ i\in 
\widehat{H^{c}}\right\} \geq \frac{\epsilon }{2}\right) \\
&\leq &\frac{2}{\epsilon }E\left[ \frac{1}{N_{1}}\dsum\limits_{i\in H}%
\mathbb{I}\left\{ i\in \widehat{H^{c}}\right\} \right] \\
&=&\frac{2}{\epsilon }\frac{1}{N_{1}}\dsum\limits_{i\in H}\Pr \left( i\in 
\widehat{H^{c}}\right) \\
&=&\frac{2}{\epsilon }\frac{1}{N_{1}}\dsum\limits_{i\in H}\Pr \left( \mathbb{%
S}_{i,T}^{+}>\Phi ^{-1}\left( 1-\frac{\varphi }{2N}\right) \right) \\
&\leq &\frac{2}{\epsilon }\frac{dN_{2}\varphi }{NN_{1}}\left[ 1+o\left(
1\right) \right] \text{ \ } \\
&&\left( \text{following an argument similar to that given in the proof of
Theorem 1 in Chao and Swanson (2022a)}\right) \\
&\rightarrow &0\text{ }\left( \text{since }\frac{\varphi }{N_{1}}\rightarrow
0\text{ and }\frac{N_{2}}{N}=O\left( 1\right) \right) \text{.}
\end{eqnarray*}%
Combining these results, we have that%
\begin{eqnarray*}
&&\Pr \left( \left\vert \frac{\widehat{N}_{1}-N_{1}}{N_{1}}\right\vert \geq
\epsilon \right) \\
&\leq &\Pr \left( \left\{ \left\vert \frac{1}{N_{1}}\dsum\limits_{i\in H^{%
{\large c}}}\left( \mathbb{I}\left\{ i\in \widehat{H^{c}}\right\} -1\right)
\right\vert \geq \frac{\epsilon }{2}\right\} \cup \left\{ \left\vert \frac{1%
}{N_{1}}\dsum\limits_{i\in H}\mathbb{I}\left\{ i\in \widehat{H^{c}}\right\}
\right\vert \geq \frac{\epsilon }{2}\right\} \right) \\
&\leq &\Pr \left( \left\vert \frac{1}{N_{1}}\dsum\limits_{i\in H^{{\large c}%
}}\left( \mathbb{I}\left\{ i\in \widehat{H^{c}}\right\} -1\right)
\right\vert \geq \frac{\epsilon }{2}\right) +\Pr \left( \left\vert \frac{1}{%
N_{1}}\dsum\limits_{i\in H}\mathbb{I}\left\{ i\in \widehat{H^{c}}\right\}
\right\vert \geq \frac{\epsilon }{2}\right) \\
&&\left( \text{by the union bound}\right) \\
&\rightarrow &0
\end{eqnarray*}

For part (b), note that%
\begin{eqnarray*}
\left\Vert \frac{\Gamma \left( \widehat{H^{c}}\right) -\Gamma }{\sqrt{N_{1}}}%
\right\Vert _{F}^{2} &=&\frac{1}{N_{1}}tr\left\{ \left( \Gamma \left( 
\widehat{H^{c}}\right) -\Gamma \right) ^{\prime }\left( \Gamma \left( 
\widehat{H^{c}}\right) -\Gamma \right) \right\} \\
&=&\frac{1}{N_{1}}\dsum\limits_{i=1}^{N}tr\left\{ \left( \mathbb{I}\left\{
i\in \widehat{H^{c}}\right\} \gamma _{i}-\gamma _{i}\right) \left( \mathbb{I}%
\left\{ i\in \widehat{H^{c}}\right\} \gamma _{i}-\gamma _{i}\right) ^{\prime
}\right\} \\
&=&\frac{1}{N_{1}}\dsum\limits_{i=1}^{N}\left( \mathbb{I}\left\{ i\in 
\widehat{H^{c}}\right\} \gamma _{i}-\gamma _{i}\right) ^{\prime }\left( 
\mathbb{I}\left\{ i\in \widehat{H^{c}}\right\} \gamma _{i}-\gamma _{i}\right)
\\
&=&\frac{1}{N_{1}}\dsum\limits_{i=1}^{N}\gamma _{i}^{\prime }\gamma _{i}%
\left[ 1-\mathbb{I}\left\{ i\in \widehat{H^{c}}\right\} \right] \\
&=&\frac{1}{N_{1}}\dsum\limits_{i\in H^{{\large c}}}\gamma _{i}^{\prime
}\gamma _{i}\left[ 1-\mathbb{I}\left\{ i\in \widehat{H^{c}}\right\} \right] 
\text{ }\left( \text{since }\gamma _{i}=0\text{ for }i\in H\right)
\end{eqnarray*}%
Applying Markov's inequality, we have, for any $\epsilon >0$,%
\begin{eqnarray*}
\Pr \left( \left\Vert \frac{\Gamma \left( \widehat{H^{c}}\right) -\Gamma }{%
\sqrt{N_{1}}}\right\Vert _{F}^{2}\geq \epsilon \right) &\leq &\frac{1}{%
\epsilon }E\left\{ \frac{1}{N_{1}}\dsum\limits_{i\in H^{{\large c}}}\gamma
_{i}^{\prime }\gamma _{i}\left[ 1-\mathbb{I}\left\{ i\in \widehat{H^{c}}%
\right\} \right] \right\} \\
&=&\frac{1}{\epsilon }\frac{1}{N_{1}}\dsum\limits_{i\in H^{{\large c}%
}}\gamma _{i}^{\prime }\gamma _{i}\left[ 1-\Pr \left( i\in \widehat{H^{c}}%
\right) \right] \\
&\leq &\frac{1}{\epsilon }\left[ 1-\min_{i\in H^{{\large c}}}\Pr \left( i\in 
\widehat{H^{c}}\right) \right] \frac{1}{N_{1}}\dsum\limits_{i\in H^{{\large c%
}}}\gamma _{i}^{\prime }\gamma _{i} \\
&\leq &\frac{1}{\epsilon }\left[ 1-\min_{i\in H^{{\large c}}}\Pr \left( i\in 
\widehat{H^{c}}\right) \right] \left( \sup_{i\in H^{{\large c}}}\left\Vert
\gamma _{i}\right\Vert _{2}\right) ^{2} \\
&\leq &\frac{1}{\epsilon }\left[ 1-\min_{i\in H^{{\large c}}}\Pr \left( i\in 
\widehat{H^{c}}\right) \right] \overline{C}^{2}\text{ \ }\left( \text{by
Assumption 3-5}\right) \\
&\rightarrow &0\text{ }\left( \text{since }\min_{i\in H^{{\large c}}}\Pr
\left( i\in \widehat{H^{c}}\right) \rightarrow 1\text{ for }i\in H^{c}\text{
by Theorem 2 in Chao and Swanson (2022a)}\right)
\end{eqnarray*}%
from which we further deduce that%
\begin{equation*}
\left\Vert \frac{\Gamma \left( \widehat{H^{c}}\right) -\Gamma }{\sqrt{N_{1}}}%
\right\Vert _{2}\leq \left\Vert \frac{\Gamma \left( \widehat{H^{c}}\right)
-\Gamma }{\sqrt{N_{1}}}\right\Vert _{F}\overset{p}{\rightarrow }0\text{. }
\end{equation*}

Turning our attention to part (c), note that since, by definition, 
\begin{equation*}
\widehat{G}_{1}=\left( G_{1}+G_{2}R\right) \left( I_{Kp}+R^{\prime }R\right)
^{-1/2}
\end{equation*}%
where $G_{1}^{\prime }G_{1}=I_{Kp}$, $G_{2}^{\prime }G_{2}=I_{N-Kp}$, and $%
G_{1}^{\prime }G_{2}=0$; it follows that%
\begin{eqnarray*}
\widehat{G}_{1}^{\prime }\widehat{G}_{1} &=&\left( I_{Kp}+R^{\prime
}R\right) ^{-1/2}\left( G_{1}^{\prime }+R^{\prime }G_{2}^{\prime }\right)
\left( G_{1}+G_{2}R\right) \left( I_{Kp}+R^{\prime }R\right) ^{-1/2} \\
&=&\left( I_{Kp}+R^{\prime }R\right) ^{-1/2}\left( I_{Kp}+R^{\prime
}R\right) \left( I_{Kp}+R^{\prime }R\right) ^{-1/2} \\
&=&I_{Kp}
\end{eqnarray*}%
Hence, by Assumption 3-6, 
\begin{eqnarray*}
\left\Vert \frac{\widehat{V}^{\prime }\widehat{G}_{1}^{{\Large \prime }%
}\Gamma }{\sqrt{N_{1}}}\right\Vert _{2} &\leq &\left\Vert \widehat{V}%
^{\prime }\widehat{G}_{1}^{{\Large \prime }}\right\Vert _{2}\left\Vert \frac{%
\Gamma }{\sqrt{N_{1}}}\right\Vert _{2} \\
&=&\sqrt{\lambda _{\max }\left( \widehat{G}_{1}\widehat{V}\widehat{V}%
^{\prime }\widehat{G}_{1}^{{\Large \prime }}\right) }\sqrt{\lambda _{\max
}\left( \frac{\Gamma ^{\prime }\Gamma }{N_{1}}\right) } \\
&=&\sqrt{\lambda _{\max }\left( \widehat{V}^{\prime }\widehat{G}_{1}^{%
{\Large \prime }}\widehat{G}_{1}\widehat{V}\right) }\sqrt{\lambda _{\max
}\left( \frac{\Gamma ^{\prime }\Gamma }{N_{1}}\right) } \\
&=&\sqrt{\lambda _{\max }\left( I_{Kp}\right) }\sqrt{\lambda _{\max }\left( 
\frac{\Gamma ^{\prime }\Gamma }{N_{1}}\right) }\text{ \ }\left( \text{since }%
\widehat{V}\text{ is an orthogonal matrix}\right) \\
&=&\sqrt{\lambda _{\max }\left( \frac{\Gamma ^{\prime }\Gamma }{N_{1}}%
\right) }\leq \overline{C}<\infty \text{ for }N_{1},N_{2}\text{ sufficiently
large\ }
\end{eqnarray*}%
Now, to show the second result in part (c), note that, since%
\begin{equation*}
Q=\left( \frac{\Gamma ^{\prime }\Gamma }{N_{1}}\right) ^{\frac{{\Large 1}}{%
{\Large 2}}}\Xi \widehat{V}\text{ and }G_{1}=\frac{\Gamma }{\sqrt{N_{1}}}%
\left( \frac{\Gamma ^{\prime }\Gamma }{N_{1}}\right) ^{-1/2}\Xi =\Gamma
\left( \Gamma ^{\prime }\Gamma \right) ^{-1/2}\Xi \text{ ,}
\end{equation*}%
we can write%
\begin{eqnarray*}
\frac{\widehat{V}^{\prime }\widehat{G}_{1}^{\prime }\Gamma }{\sqrt{N_{1}}}%
-Q^{\prime } &=&\frac{\widehat{V}^{\prime }\widehat{G}_{1}^{\prime }\Gamma }{%
\sqrt{N_{1}}}-\widehat{V}^{\prime }\Xi ^{\prime }\left( \frac{\Gamma
^{\prime }\Gamma }{N_{1}}\right) ^{\frac{{\Large 1}}{{\Large 2}}} \\
&=&\frac{\widehat{V}^{\prime }\widehat{G}_{1}^{\prime }\Gamma }{\sqrt{N_{1}}}%
-\widehat{V}^{\prime }\Xi ^{\prime }\left( \frac{\Gamma ^{\prime }\Gamma }{%
N_{1}}\right) ^{-1/2}\frac{\Gamma ^{\prime }\Gamma }{N_{1}} \\
&=&\frac{\widehat{V}^{\prime }\widehat{G}_{1}^{\prime }\Gamma }{\sqrt{N_{1}}}%
-\frac{\widehat{V}^{\prime }G_{1}^{\prime }\Gamma }{\sqrt{N_{1}}} \\
&=&\widehat{V}^{\prime }\left( \widehat{G}_{1}-G_{1}\right) ^{\prime }\frac{%
\Gamma }{\sqrt{N_{1}}}
\end{eqnarray*}%
from which it follows that%
\begin{eqnarray*}
\left\Vert \frac{\widehat{V}^{\prime }\widehat{G}_{1}^{\prime }\Gamma }{%
\sqrt{N_{1}}}-Q^{\prime }\right\Vert _{2} &\leq &\left\Vert \widehat{V}%
^{\prime }\right\Vert _{2}\left\Vert \left( \widehat{G}_{1}-G_{1}\right)
^{\prime }\right\Vert _{2}\left\Vert \frac{\Gamma }{\sqrt{N_{1}}}\right\Vert
_{2} \\
&=&\sqrt{\lambda _{\max }\left( \widehat{V}\widehat{V}^{\prime }\right) }%
\sqrt{\lambda _{\max }\left\{ \left( \widehat{G}_{1}-G_{1}\right) \left( 
\widehat{G}_{1}-G_{1}\right) ^{\prime }\right\} }\sqrt{\lambda _{\max
}\left( \frac{\Gamma ^{\prime }\Gamma }{N_{1}}\right) } \\
&=&\sqrt{\lambda _{\max }\left( I_{Kp}\right) }\sqrt{\lambda _{\max }\left\{
\left( \widehat{G}_{1}-G_{1}\right) ^{\prime }\left( \widehat{G}%
_{1}-G_{1}\right) \right\} }\sqrt{\lambda _{\max }\left( \frac{\Gamma
^{\prime }\Gamma }{N_{1}}\right) } \\
&&\left( \text{since }\widehat{V}\text{ is an orthogonal matrix and since }%
\lambda _{\max }\left( AA^{\prime }\right) =\lambda _{\max }\left( A^{\prime
}A\right) \right) \\
&\leq &\sqrt{\overline{C}}\left\Vert \widehat{G}_{1}-G_{1}\right\Vert _{2}%
\text{ \ }\left( \text{by Assumption 3-6}\right) \\
&=&o_{p}\left( 1\right) \text{ \ as }N_{1},\text{ }N_{2}\text{, and }%
T\rightarrow \infty \text{\ }\left( \text{by part (b) of Lemma D-14}\right) 
\text{.}
\end{eqnarray*}

Next, to show part (d), we first write%
\begin{eqnarray}
\left\Vert \frac{G_{1}^{\prime }U_{t,N}\left( \widehat{H^{c}}\right) }{\sqrt{%
N_{1}}}\right\Vert _{2}^{2} &=&\dsum\limits_{k=1}^{Kp}\left(
\dsum\limits_{i=1}^{N}\mathbb{I}\left\{ i\in \widehat{H^{c}}\right\} \frac{%
g_{1,ik}u_{i,t}}{\sqrt{N_{1}}}\right) ^{2}  \notag \\
&=&\dsum\limits_{k=1}^{Kp}\left( \dsum\limits_{i\in H^{{\large c}}}\mathbb{I}%
\left\{ i\in \widehat{H^{c}}\right\} \frac{g_{1,ik}u_{i,t}}{\sqrt{N_{1}}}%
+\dsum\limits_{i\in H}\mathbb{I}\left\{ i\in \widehat{H^{c}}\right\} \frac{%
g_{1,ik}u_{i,t}}{\sqrt{N_{1}}}\right) ^{2}  \notag \\
&\leq &2\dsum\limits_{k=1}^{K}\left( \dsum\limits_{i\in H^{{\large c}}}%
\mathbb{I}\left\{ i\in \widehat{H^{c}}\right\} \frac{g_{1,ik}u_{i,t}}{\sqrt{%
N_{1}}}\right) ^{2}+2\dsum\limits_{k=1}^{Kp}\left( \dsum\limits_{i\in H}%
\mathbb{I}\left\{ i\in \widehat{H^{c}}\right\} \frac{g_{1,ik}u_{i,t}}{\sqrt{%
N_{1}}}\right) ^{2}  \notag \\
&=&\frac{2}{N_{1}}\dsum\limits_{k=1}^{Kp}\dsum\limits_{i\in H^{{\large c}%
}}\dsum\limits_{j\in H^{{\large c}}}\mathbb{I}\left\{ i\in \widehat{H^{c}}%
\right\} \mathbb{I}\left\{ j\in \widehat{H^{c}}\right\}
g_{1,ik}g_{1,jk}u_{i,t}u_{j,t}  \notag \\
&&+\frac{2}{N_{1}}\dsum\limits_{k=1}^{Kp}\dsum\limits_{i\in
H}\dsum\limits_{j\in H}\mathbb{I}\left\{ i\in \widehat{H^{c}}\right\} 
\mathbb{I}\left\{ j\in \widehat{H^{c}}\right\} g_{1,ik}g_{1,jk}u_{i,t}u_{j,t}
\label{GU/N1^0.5}
\end{eqnarray}%
where $g_{1,ik}$ denotes the $\left( i,k\right) ^{th}$ element of $G_{1}$.
Now, consider the first term on the right-hand side of expression (\ref%
{GU/N1^0.5}). Write%
\begin{eqnarray*}
&&\frac{2}{N_{1}}\dsum\limits_{k=1}^{Kp}\dsum\limits_{i\in H^{{\large c}%
}}\dsum\limits_{j\in H^{{\large c}}}\mathbb{I}\left\{ i\in \widehat{H^{c}}%
\right\} \mathbb{I}\left\{ j\in \widehat{H^{c}}\right\}
g_{1,ik}g_{1,jk}u_{i,t}u_{j,t} \\
&=&\frac{2}{N_{1}}\dsum\limits_{k=1}^{Kp}\dsum\limits_{i\in H^{{\large c}%
}}\dsum\limits_{j\in H^{{\large c}}}\left( \mathbb{I}\left\{ i\in \widehat{%
H^{c}}\right\} -1+1\right) \left( \mathbb{I}\left\{ j\in \widehat{H^{c}}%
\right\} -1+1\right) g_{1,ik}g_{1,jk}u_{i,t}u_{j,t} \\
&=&\frac{2}{N_{1}}\dsum\limits_{k=1}^{Kp}\dsum\limits_{i\in H^{{\large c}%
}}\left( \mathbb{I}\left\{ i\in \widehat{H^{c}}\right\} -1\right)
g_{1,ik}u_{i,t}\dsum\limits_{j\in H^{{\large c}}}\left( \mathbb{I}\left\{
j\in \widehat{H^{c}}\right\} -1\right) g_{1,jk}u_{j,t} \\
&&+\frac{2}{N_{1}}\dsum\limits_{k=1}^{Kp}\dsum\limits_{i\in H^{{\large c}%
}}\dsum\limits_{j\in H^{{\large c}}}g_{1,ik}g_{1,jk}u_{i,t}u_{j,t} \\
&&+\frac{2}{N_{1}}\dsum\limits_{k=1}^{Kp}\dsum\limits_{i\in H^{{\large c}%
}}\left( \mathbb{I}\left\{ i\in \widehat{H^{c}}\right\} -1\right)
g_{1,ik}u_{i,t}\dsum\limits_{j\in H^{{\large c}}}g_{1,jk}u_{j,t} \\
&&+\frac{2}{N_{1}}\dsum\limits_{k=1}^{Kp}\dsum\limits_{i\in H^{{\large c}%
}}g_{1,ik}u_{i,t}\dsum\limits_{j\in H^{{\large c}}}\left( \mathbb{I}\left\{
j\in \widehat{H^{c}}\right\} -1\right) g_{1,jk}u_{j,t} \\
&=&\mathcal{E}_{1,1,t}+\mathcal{E}_{1,2,t}+\mathcal{E}_{1,3,t}+\mathcal{E}%
_{1,4,t}
\end{eqnarray*}%
Focusing first on the term $\mathcal{E}_{1,1,t}$, we have%
\begin{eqnarray*}
&&\frac{2}{N_{1}}\dsum\limits_{k=1}^{Kp}\dsum\limits_{i\in H^{{\large c}%
}}\left( \mathbb{I}\left\{ i\in \widehat{H^{c}}\right\} -1\right)
g_{1,ik}u_{i,t}\dsum\limits_{j\in H^{{\large c}}}\left( \mathbb{I}\left\{
j\in \widehat{H^{c}}\right\} -1\right) g_{1,jk}u_{j,t} \\
&=&\frac{2}{N_{1}}\dsum\limits_{k=1}^{Kp}\left( \dsum\limits_{i\in H^{%
{\large c}}}\left( \mathbb{I}\left\{ i\in \widehat{H^{c}}\right\} -1\right)
g_{1,ik}u_{i,t}\right) ^{2} \\
&\leq &\frac{2}{N_{1}}\dsum\limits_{k=1}^{Kp}\left( \left\vert
\dsum\limits_{i\in H^{{\large c}}}\left( \mathbb{I}\left\{ i\in \widehat{%
H^{c}}\right\} -1\right) g_{1,ik}u_{i,t}\right\vert \right) ^{2} \\
&\leq &2\dsum\limits_{k=1}^{Kp}\left( \frac{1}{N_{1}}\dsum\limits_{i\in H^{%
{\large c}}}\left( \mathbb{I}\left\{ i\in \widehat{H^{c}}\right\} -1\right)
^{2}\right) \left( \dsum\limits_{i\in H^{{\large c}}}g_{1,ik}^{2}u_{i,t}^{2}%
\right) \\
&=&2\dsum\limits_{k=1}^{Kp}\left[ \frac{1}{N_{1}}\dsum\limits_{i\in H^{%
{\large c}}}\left( \mathbb{I}\left\{ i\in \widehat{H^{c}}\right\} -2\mathbb{I%
}\left\{ i\in \widehat{H^{c}}\right\} +1\right) \right] \left(
\dsum\limits_{i\in H^{{\large c}}}g_{1,ik}^{2}u_{i,t}^{2}\right) \\
&=&2\dsum\limits_{k=1}^{Kp}\left[ \frac{1}{N_{1}}\dsum\limits_{i\in H^{%
{\large c}}}\left( 1-\mathbb{I}\left\{ i\in \widehat{H^{c}}\right\} \right) %
\right] \left( \dsum\limits_{i\in H^{{\large c}}}g_{1,ik}^{2}u_{i,t}^{2}%
\right)
\end{eqnarray*}%
Now, for either the case where $\mathbb{S}_{i,T}^{+}=\dsum\nolimits_{\ell
=1}^{d}\varpi _{\ell }\left\vert S_{i,\ell ,T}\right\vert $ or the case
where $\mathbb{S}_{i,T}^{+}=\max_{1\leq \ell \leq d}\left\vert S_{i,\ell
,T}\right\vert $, we have%
\begin{eqnarray*}
0 &\leq &E\left[ \frac{1}{N_{1}}\dsum\limits_{i\in H^{{\large c}}}\left( 1-%
\mathbb{I}\left\{ i\in \widehat{H^{c}}\right\} \right) \right] \\
&=&\frac{1}{N_{1}}\dsum\limits_{i\in H^{{\large c}}}\left[ 1-\Pr \left( i\in 
\widehat{H^{c}}\right) \right] \\
&=&\frac{1}{N_{1}}\dsum\limits_{i\in H^{{\large c}}}\left[ 1-P\left( \mathbb{%
S}_{i,T}^{+}\geq \Phi ^{-1}\left( 1-\frac{\varphi }{2N}\right) \right) %
\right] \\
&\leq &1-P\left( \min_{i\in H^{{\large c}}}\mathbb{S}_{i,T}^{+}\geq \Phi
^{-1}\left( 1-\frac{\varphi }{2N}\right) \right) \\
&&\left( \text{given that }N_{1}=\#\left\{ H^{c}\right\} \text{, where }%
\#\left\{ H^{c}\right\} \text{ denotes the cardinality of the set }%
H^{c}\right) \\
&\rightarrow &0\text{ ,}
\end{eqnarray*}%
since, by Theorem 2 in Chao and Swanson (2022a), $P\left( \min_{i\in H^{%
{\large c}}}\mathbb{S}_{i,T}^{+}\geq \Phi ^{-1}\left( 1-\frac{\varphi }{2N}%
\right) \right) \rightarrow 1$. Moreover, by part (b) of Assumption 3-3, we
have%
\begin{equation*}
E\left[ \dsum\limits_{i\in H^{{\large c}}}g_{1,ik}^{2}u_{i,t}^{2}\right]
=\dsum\limits_{i\in H^{{\large c}}}g_{1,ik}^{2}E\left[ u_{i,t}^{2}\right]
\leq C\dsum\limits_{i=1}^{N}g_{1,ik}^{2}\leq C
\end{equation*}%
It follows by Markov's inequality that%
\begin{equation*}
\frac{1}{N_{1}}\dsum\limits_{i\in H^{{\large c}}}\left( 1-\mathbb{I}\left\{
i\in \widehat{H^{c}}\right\} \right) =o_{p}\left( 1\right) \text{ and }%
\dsum\limits_{i\in H^{{\large c}}}g_{1,ik}^{2}u_{i,t}^{2}=O_{p}\left(
1\right)
\end{equation*}%
from which we deduce that%
\begin{eqnarray*}
\mathcal{E}_{1,1,t} &=&\frac{2}{N_{1}}\dsum\limits_{k=1}^{Kp}\left(
\dsum\limits_{i\in H^{{\large c}}}\left( \mathbb{I}\left\{ i\in \widehat{%
H^{c}}\right\} -1\right) g_{1,ik}u_{i,t}\right) ^{2} \\
&\leq &2\dsum\limits_{k=1}^{Kp}\left[ \frac{1}{N_{1}}\dsum\limits_{i\in H^{%
{\large c}}}\left( 1-\mathbb{I}\left\{ i\in \widehat{H^{c}}\right\} \right) %
\right] \left( \dsum\limits_{i\in H^{{\large c}}}g_{1,ik}^{2}u_{i,t}^{2}%
\right) \\
&=&o_{p}\left( 1\right)
\end{eqnarray*}

Consider next the term $\mathcal{E}_{1,2,t}$. To proceed, let $U_{t,N}\left(
H^{c}\right) $ denote an $N\times 1$ vector whose $i^{th}$ component $%
U_{i,t,N}\left( H^{c}\right) $ is given by%
\begin{equation*}
U_{i,t,N}\left( H^{c}\right) =\left\{ 
\begin{array}{cc}
u_{i,t} & \text{if }i\in H^{c} \\ 
0 & \text{if }i\in H%
\end{array}%
\right. .
\end{equation*}%
and we can write%
\begin{eqnarray*}
\mathcal{E}_{1,2,t} &=&\frac{2}{N_{1}}\dsum\limits_{k=1}^{Kp}\dsum\limits_{i%
\in H^{{\large c}}}\dsum\limits_{j\in H^{{\large c}%
}}g_{1,ik}g_{1,jk}u_{i,t}u_{j,t} \\
&=&2\left\Vert \frac{G_{1}^{\prime }U_{t,N}\left( H^{c}\right) }{\sqrt{N_{1}}%
}\right\Vert _{2}^{2} \\
&\leq &2tr\left\{ \frac{G_{1}^{\prime }U_{t,N}\left( H^{c}\right)
U_{t,N}\left( H^{c}\right) ^{\prime }G_{1}}{N_{1}}\right\} \\
&=&2tr\left\{ \Xi ^{\prime }\left( \frac{\Gamma ^{\prime }\Gamma }{N_{1}}%
\right) ^{-1/2}\frac{\Gamma ^{\prime }}{\sqrt{N_{1}}}\frac{U_{t,N}\left(
H^{c}\right) U_{t,N}\left( H^{c}\right) ^{\prime }}{N_{1}}\frac{\Gamma }{%
\sqrt{N_{1}}}\left( \frac{\Gamma ^{\prime }\Gamma }{N_{1}}\right) ^{-1/2}\Xi
\right\} \\
&=&2tr\left\{ \left( \frac{\Gamma ^{\prime }\Gamma }{N_{1}}\right) ^{-1/2}%
\frac{\Gamma ^{\prime }}{\sqrt{N_{1}}}\frac{U_{t,N}\left( H^{c}\right)
U_{t,N}\left( H^{c}\right) ^{\prime }}{N_{1}}\frac{\Gamma }{\sqrt{N_{1}}}%
\left( \frac{\Gamma ^{\prime }\Gamma }{N_{1}}\right) ^{-1/2}\right\} \\
&=&2tr\left\{ \frac{\Gamma _{\ast }^{\prime }U_{t,N}\left( H^{c}\right)
U_{t,N}\left( H^{c}\right) ^{\prime }\Gamma _{\ast }}{N_{1}^{2}}\right\} 
\text{ }\left( \text{where }\Gamma _{\ast }=\Gamma \left( \frac{\Gamma
^{\prime }\Gamma }{N_{1}}\right) ^{-1/2}\right) \\
&=&\frac{2}{N_{1}^{2}}U_{t,N}\left( H^{c}\right) ^{\prime }\Gamma _{\ast
}\Gamma _{\ast }^{\prime }U_{t,N}\left( H^{c}\right) \\
&=&\frac{2}{N_{1}^{2}}\dsum\limits_{i\in H^{{\large c}}}\dsum\limits_{j\in
H^{{\large c}}}\gamma _{\ast ,i}^{\prime }\gamma _{\ast ,j}u_{i,t}u_{j,t}
\end{eqnarray*}%
where $\gamma _{\ast ,i}^{\prime }$ denotes the $i^{th}$ row of $\Gamma
_{\ast }=\Gamma \left( \Gamma ^{\prime }\Gamma /N_{1}\right) ^{-1/2}$. Hence,%
\begin{eqnarray*}
0 &\leq &E\left[ \mathcal{E}_{1,2,t}\right] \\
&=&\frac{2}{N_{1}}\dsum\limits_{k=1}^{Kp}\dsum\limits_{i\in H^{{\large c}%
}}\dsum\limits_{j\in H^{{\large c}}}g_{1,ik}g_{1,jk}E\left[ u_{i,t}u_{j,t}%
\right] \\
&=&\frac{2}{N_{1}^{2}}\dsum\limits_{i\in H^{{\large c}}}\dsum\limits_{j\in
H^{{\large c}}}\gamma _{\ast ,i}^{\prime }\gamma _{\ast ,j}E\left[
u_{i,t}u_{j,t}\right] \\
&=&\frac{2}{N_{1}^{2}}\dsum\limits_{i\in H^{{\large c}}}\dsum\limits_{j\in
H^{{\large c}}}\gamma _{i}^{\prime }\left( \frac{\Gamma ^{\prime }\Gamma }{%
N_{1}}\right) ^{-1/2}\left( \frac{\Gamma ^{\prime }\Gamma }{N_{1}}\right)
^{-1/2}\gamma _{j}E\left[ u_{i,t}u_{j,t}\right] \\
&\leq &\frac{2}{N_{1}^{2}}\dsum\limits_{i\in H^{{\large c}%
}}\dsum\limits_{j\in H^{{\large c}}}\left\vert \gamma _{i}^{\prime }\left( 
\frac{\Gamma ^{\prime }\Gamma }{N_{1}}\right) ^{-1/2}\left( \frac{\Gamma
^{\prime }\Gamma }{N_{1}}\right) ^{-1/2}\gamma _{j}\right\vert \left\vert E%
\left[ u_{i,t}u_{j,t}\right] \right\vert \\
&\leq &\frac{2}{N_{1}^{2}}\dsum\limits_{i\in H^{{\large c}%
}}\dsum\limits_{j\in H^{{\large c}}}\sqrt{\gamma _{i}^{\prime }\left( \frac{%
\Gamma ^{\prime }\Gamma }{N_{1}}\right) ^{-1}\gamma _{i}}\sqrt{\gamma
_{i}^{\prime }\left( \frac{\Gamma ^{\prime }\Gamma }{N_{1}}\right)
^{-1}\gamma _{i}}\left\vert E\left[ u_{i,t}u_{j,t}\right] \right\vert \\
&\leq &\frac{2\overline{c}}{\underline{C}}\frac{1}{N_{1}^{2}}%
\dsum\limits_{i\in H^{{\large c}}}\dsum\limits_{j\in H^{{\large c}%
}}\left\vert E\left[ u_{i,t}u_{j,t}\right] \right\vert \\
&&\left( \text{since, under Assumptions 3-5 and 3-6, there exist positive
constants }\overline{c}\text{ and }\underline{C}\text{ such that }\right. \\
&&\left. \sup_{i\in H^{{\large c}}}\left\Vert \gamma _{i}\right\Vert
_{2}\leq \overline{c}<\infty \text{ and }\lambda _{\min }\left( \frac{\Gamma
^{\prime }\Gamma }{N_{1}}\right) \geq \underline{C}>0\right) \\
&\leq &\frac{2\overline{c}}{\underline{C}}\frac{\overline{C}}{N_{1}}\text{ }%
\rightarrow 0\text{ as }N_{1}\rightarrow \infty \text{.}\left( \text{since,
under Assumption 3-3(d), there exists a positive constant }\overline{C}%
\right. \text{ } \\
&&\text{ \ \ \ \ \ \ \ \ \ \ \ \ \ \ \ \ \ \ \ \ \ \ \ \ \ \ \ \ \ \ \ \ \ \
\ \ \ \ \ \ \ }\left. \text{such that }\sup_{t}\frac{1}{N_{1}}%
\dsum\limits_{i\in H^{{\large c}}}\dsum\limits_{j\in H^{{\large c}%
}}\left\vert E\left[ u_{i,t}u_{j,t}\right] \right\vert \leq \overline{C}%
<\infty \right)
\end{eqnarray*}%
It follows by Markov's inequality that%
\begin{equation*}
\mathcal{E}_{1,s,t}=o_{p}\left( 1\right) \text{. }
\end{equation*}

Now, for $\mathcal{E}_{1,3,t}$, write

\begin{eqnarray*}
&&\left\vert \mathcal{E}_{1,3,t}\right\vert \\
&=&\left\vert \frac{2}{N_{1}}\dsum\limits_{k=1}^{Kp}\dsum\limits_{i\in H^{%
{\large c}}}\left( \mathbb{I}\left\{ i\in \widehat{H^{c}}\right\} -1\right)
g_{1,ik}u_{i,t}\dsum\limits_{j\in H^{{\large c}}}g_{1,jk}u_{j,t}\right\vert
\\
&=&\left\vert \frac{2}{N_{1}}\dsum\limits_{k=1}^{Kp}\dsum\limits_{j\in H^{%
{\large c}}}g_{1,jk}u_{j,t}\dsum\limits_{i\in H^{{\large c}}}\left( \mathbb{I%
}\left\{ i\in \widehat{H^{c}}\right\} -1\right) g_{1,ik}u_{i,t}\right\vert \\
&\leq &\frac{2}{N_{1}}\dsum\limits_{k=1}^{Kp}\left\vert \dsum\limits_{j\in
H^{{\large c}}}g_{1,jk}u_{j,t}\right\vert \left\vert \dsum\limits_{i\in H^{%
{\large c}}}\left( \mathbb{I}\left\{ i\in \widehat{H^{c}}\right\} -1\right)
g_{1,ik}u_{i,t}\right\vert \\
&\leq &\frac{2}{N_{1}}\dsum\limits_{k=1}^{Kp}\sqrt{\dsum\limits_{i\in H^{%
{\large c}}}\left( \mathbb{I}\left\{ i\in \widehat{H^{c}}\right\} -1\right)
^{2}}\sqrt{\dsum\limits_{i\in H^{{\large c}}}g_{1,ik}^{2}u_{i,t}^{2}}%
\left\vert \dsum\limits_{j\in H^{{\large c}}}g_{1,jk}u_{j,t}\right\vert \\
&\leq &\frac{1}{N_{1}}\dsum\limits_{k=1}^{Kp}\sqrt{\dsum\limits_{i\in H^{%
{\large c}}}\left( \mathbb{I}\left\{ i\in \widehat{H^{c}}\right\} -1\right)
^{2}}\dsum\limits_{i\in H^{{\large c}}}g_{1,ik}^{2}u_{i,t}^{2} \\
&&+\frac{1}{N_{1}}\dsum\limits_{k=1}^{Kp}\sqrt{\dsum\limits_{i\in H^{{\large %
c}}}\left( \mathbb{I}\left\{ i\in \widehat{H^{c}}\right\} -1\right) ^{2}}%
\left( \dsum\limits_{j\in H^{{\large c}}}g_{1,jk}u_{j,t}\right) ^{2} \\
&&\left( \text{by the inequality }\left\vert XY\right\vert \leq \frac{1}{2}%
X^{2}+\frac{1}{2}Y^{2}\right) \\
&=&\frac{1}{\sqrt{N_{1}}}\dsum\limits_{k=1}^{Kp}\sqrt{\frac{1}{N_{1}}%
\dsum\limits_{i\in H^{{\large c}}}\left( 1-\mathbb{I}\left\{ i\in \widehat{%
H^{c}}\right\} \right) }\dsum\limits_{i\in H^{{\large c}%
}}g_{1,ik}^{2}u_{i,t}^{2} \\
&&+\sqrt{\frac{1}{N_{1}}\dsum\limits_{i\in H^{{\large c}}}\left( 1-\mathbb{I}%
\left\{ i\in \widehat{H^{c}}\right\} \right) }\frac{1}{\sqrt{N_{1}}}%
\dsum\limits_{k=1}^{Kp}\left( \dsum\limits_{j\in H^{{\large c}%
}}g_{1,jk}u_{j,t}\right) ^{2}
\end{eqnarray*}%
Observe that%
\begin{eqnarray*}
&&E\left[ \frac{1}{\sqrt{N_{1}}}\dsum\limits_{k=1}^{Kp}\left(
\dsum\limits_{j\in H^{{\large c}}}g_{1,jk}u_{j,t}\right) ^{2}\right] \\
&=&\frac{1}{\sqrt{N_{1}}}\dsum\limits_{k=1}^{Kp}\dsum\limits_{j\in H^{%
{\large c}}}\dsum\limits_{\ell \in H^{{\large c}}}g_{1,jk}g_{1,\ell k}E\left[
u_{j,t}u_{\ell ,t}\right] \\
&=&\frac{1}{\sqrt{N_{1}}}\dsum\limits_{j\in H^{{\large c}}}\dsum\limits_{%
\ell \in H^{{\large c}}}\dsum\limits_{k=1}^{Kp}g_{1,jk}g_{1,\ell k}E\left[
u_{j,t}u_{\ell ,t}\right] \\
&=&\frac{1}{\sqrt{N_{1}}}\dsum\limits_{j\in H^{{\large c}}}\dsum\limits_{%
\ell \in H^{{\large c}}}\frac{e_{j,N}^{\prime }\Gamma }{\sqrt{N_{1}}}\left( 
\frac{\Gamma ^{\prime }\Gamma }{N_{1}}\right)
^{-1/2}\dsum\limits_{k=1}^{Kp}\Xi e_{k,Kp}e_{k,Kp}^{\prime }\Xi ^{\prime
}\left( \frac{\Gamma ^{\prime }\Gamma }{N_{1}}\right) ^{-1/2}\frac{\Gamma
^{\prime }e_{\ell ,N}}{\sqrt{N_{1}}}E\left[ u_{j,t}u_{\ell ,t}\right] \\
&&\left( \text{since }G_{1}=\frac{\Gamma }{\sqrt{N_{1}}}\left( \frac{\Gamma
^{\prime }\Gamma }{N_{1}}\right) ^{-1/2}\Xi \right) \\
&=&\frac{1}{N_{1}^{\frac{{\Large 3}}{{\Large 2}}}}\dsum\limits_{j\in H^{%
{\large c}}}\dsum\limits_{\ell \in H^{{\large c}}}e_{j,N}^{\prime }\Gamma
_{\ast }\Xi \Xi ^{\prime }\Gamma _{\ast }^{\prime }e_{\ell ,N}E\left[
u_{j,t}u_{\ell ,t}\right] \text{ }\left( \text{where }\Gamma _{\ast }=\Gamma
\left( \frac{\Gamma ^{\prime }\Gamma }{N_{1}}\right) ^{-1/2}\right) \\
&=&\frac{1}{N_{1}^{\frac{{\Large 3}}{{\Large 2}}}}\dsum\limits_{j\in H^{%
{\large c}}}\dsum\limits_{\ell \in H^{{\large c}}}e_{j,N}^{\prime }\Gamma
_{\ast }\Gamma _{\ast }^{\prime }e_{\ell ,N}E\left[ u_{j,t}u_{\ell ,t}\right]
\\
&&\left( \text{since }\Xi \text{ is an orthogonal matrix}\right) \\
&=&\frac{1}{N_{1}^{\frac{{\Large 3}}{{\Large 2}}}}\dsum\limits_{j\in H^{%
{\large c}}}\dsum\limits_{\ell \in H^{{\large c}}}\gamma _{\ast ,j}^{\prime
}\gamma _{\ast ,\ell }E\left[ u_{j,t}u_{\ell ,t}\right]
\end{eqnarray*}%
where we take 
\begin{equation*}
\gamma _{\ast ,j}=\left( \frac{\Gamma ^{\prime }\Gamma }{N_{1}}\right)
^{-1/2}\gamma _{j}\text{,}
\end{equation*}%
Applying the triangle and Cauchy-Schwarz inequalities, we further obtain%
\begin{eqnarray*}
&&E\left[ \frac{1}{\sqrt{N_{1}}}\dsum\limits_{k=1}^{Kp}\left(
\dsum\limits_{j\in H^{{\large c}}}g_{1,jk}u_{j,t}\right) ^{2}\right] \\
&=&\frac{1}{N_{1}^{\frac{{\Large 3}}{{\Large 2}}}}\dsum\limits_{j\in H^{%
{\large c}}}\dsum\limits_{\ell \in H^{{\large c}}}\gamma _{\ast ,j}^{\prime
}\gamma _{\ast ,\ell }E\left[ u_{j,t}u_{\ell ,t}\right] \\
&=&\frac{1}{N_{1}^{\frac{{\Large 3}}{{\Large 2}}}}\dsum\limits_{j\in H^{%
{\large c}}}\dsum\limits_{\ell \in H^{{\large c}}}\gamma _{j}^{\prime
}\left( \frac{\Gamma ^{\prime }\Gamma }{N_{1}}\right) ^{-1/2}\left( \frac{%
\Gamma ^{\prime }\Gamma }{N_{1}}\right) ^{-1/2}\gamma _{\ell }E\left[
u_{j,t}u_{\ell ,t}\right] \\
&\leq &\frac{1}{N_{1}^{\frac{{\Large 3}}{{\Large 2}}}}\dsum\limits_{j\in H^{%
{\large c}}}\dsum\limits_{\ell \in H^{{\large c}}}\left\vert \gamma
_{j}^{\prime }\left( \frac{\Gamma ^{\prime }\Gamma }{N_{1}}\right)
^{-1/2}\left( \frac{\Gamma ^{\prime }\Gamma }{N_{1}}\right) ^{-1/2}\gamma
_{\ell }\right\vert \left\vert E\left[ u_{j,t}u_{\ell ,t}\right] \right\vert
\\
&\leq &\frac{1}{N_{1}^{\frac{{\Large 3}}{{\Large 2}}}}\dsum\limits_{j\in H^{%
{\large c}}}\dsum\limits_{\ell \in H^{{\large c}}}\sqrt{\gamma _{j}^{\prime
}\left( \frac{\Gamma ^{\prime }\Gamma }{N_{1}}\right) ^{-1}\gamma _{j}}\sqrt{%
\gamma _{\ell }^{\prime }\left( \frac{\Gamma ^{\prime }\Gamma }{N_{1}}%
\right) ^{-1}\gamma _{\ell }}\left\vert E\left[ u_{j,t}u_{\ell ,t}\right]
\right\vert \\
&\leq &\frac{\overline{c}}{\underline{C}}\frac{1}{\sqrt{N_{1}}}\frac{1}{N_{1}%
}\dsum\limits_{j\in H^{{\large c}}}\dsum\limits_{\ell \in H^{{\large c}%
}}\left\vert E\left[ u_{j,t}u_{\ell ,t}\right] \right\vert \\
&&\left( \text{since, under Assumptions 3-5 and 3-6, there exist positive
constants }\overline{c}\text{ and }\underline{C}\text{ such that }\right. \\
&&\left. \sup_{i\in H^{{\large c}}}\left\Vert \gamma _{i}\right\Vert
_{2}\leq \overline{c}<\infty \text{ and }\lambda _{\min }\left( \frac{\Gamma
^{\prime }\Gamma }{N_{1}}\right) \geq \underline{C}>0\right) \\
&\leq &\frac{\overline{c}}{\underline{C}}\frac{\overline{C}}{\sqrt{N}_{1}}%
\text{ }\rightarrow 0\text{ as }N_{1}\rightarrow \infty \text{.}\left( \text{%
since, under Assumption 3-3(d) that there exists a }\right. \text{ } \\
&&\text{ \ \ \ \ \ \ \ \ \ \ \ \ \ \ \ \ \ \ \ \ \ }\left. \text{positive
constant }\overline{C}\text{ such that }\sup_{t}\frac{1}{N_{1}}%
\dsum\limits_{j\in H}\dsum\limits_{\ell \in H^{{\large c}}}\left\vert E\left[
u_{j,t}u_{\ell ,t}\right] \right\vert \leq \overline{C}<\infty \right)
\end{eqnarray*}%
from which we further deduce, upon applying Markov's inequality, that%
\begin{equation*}
\frac{1}{\sqrt{N_{1}}}\dsum\limits_{k=1}^{Kp}\left( \dsum\limits_{j\in H^{%
{\large c}}}g_{1,jk}u_{j,t}\right) ^{2}=o_{p}\left( 1\right) \text{.}
\end{equation*}%
Moreover, since we have previously shown that%
\begin{equation*}
\frac{1}{N_{1}}\dsum\limits_{i\in H^{{\large c}}}\left( 1-\mathbb{I}\left\{
i\in \widehat{H^{c}}\right\} \right) =o_{p}\left( 1\right) \text{ and }%
\dsum\limits_{i\in H^{{\large c}}}g_{1,ik}^{2}u_{i,t}^{2}=O_{p}\left(
1\right) \text{,}
\end{equation*}%
it follows from these calculations that%
\begin{eqnarray*}
\left\vert \mathcal{E}_{1,3,t}\right\vert &\leq &\frac{1}{\sqrt{N_{1}}}%
\dsum\limits_{k=1}^{Kp}\sqrt{\frac{1}{N_{1}}\dsum\limits_{i\in H^{{\large c}%
}}\left( 1-\mathbb{I}\left\{ i\in \widehat{H^{c}}\right\} \right) }%
\dsum\limits_{i\in H^{{\large c}}}g_{1,ik}^{2}u_{i,t}^{2} \\
&&+\sqrt{\frac{1}{N_{1}}\dsum\limits_{i\in H^{{\large c}}}\left( 1-\mathbb{I}%
\left\{ i\in \widehat{H^{c}}\right\} \right) }\frac{1}{\sqrt{N_{1}}}%
\dsum\limits_{k=1}^{Kp}\left( \dsum\limits_{j\in H^{{\large c}%
}}g_{1,jk}u_{j,t}\right) ^{2} \\
&=&o_{p}\left( 1\right) \text{.}
\end{eqnarray*}

\noindent In a similar way, we can also show that%
\begin{equation*}
\left\vert \mathcal{E}_{1,4,t}\right\vert =o_{p}\left( 1\right) \text{.}
\end{equation*}

\noindent Finally, application of the Slutsky's theorem then allows us to
deduce that%
\begin{eqnarray*}
\frac{2}{N_{1}}\dsum\limits_{k=1}^{Kp}\dsum\limits_{i\in H^{{\large c}%
}}\dsum\limits_{j\in H^{{\large c}}}\mathbb{I}\left\{ i\in \widehat{H^{c}}%
\right\} \mathbb{I}\left\{ j\in \widehat{H^{c}}\right\}
g_{1,ik}g_{1,jk}u_{i,t}u_{j,t} &=&\mathcal{E}_{1,1,t}+\mathcal{E}_{1,2,t}+%
\mathcal{E}_{1,3,t}+\mathcal{E}_{1,4,t} \\
&=&o_{p}\left( 1\right) +o_{p}\left( 1\right) +o_{p}\left( 1\right)
+o_{p}\left( 1\right) \\
&=&o_{p}\left( 1\right) \text{.}
\end{eqnarray*}

Next, consider the second term on the right-hand side of expression (\ref%
{GU/N1^0.5}). In this case, write 
\begin{eqnarray*}
&&\frac{2}{N_{1}}\dsum\limits_{k=1}^{Kp}\dsum\limits_{i\in
H}\dsum\limits_{j\in H}\mathbb{I}\left\{ i\in \widehat{H^{c}}\right\} 
\mathbb{I}\left\{ j\in \widehat{H^{c}}\right\} g_{1,ik}g_{1,jk}u_{i,t}u_{j,t}
\\
&=&\frac{2}{N_{1}}\dsum\limits_{k=1}^{Kp}\left( \dsum\limits_{i\in H}\mathbb{%
I}\left\{ i\in \widehat{H^{c}}\right\} g_{1,ik}u_{i,t}\right) ^{2} \\
&=&\frac{2}{N_{1}}\dsum\limits_{k=1}^{Kp}\left( \left\vert
\dsum\limits_{i\in H}\mathbb{I}\left\{ i\in \widehat{H^{c}}\right\}
g_{1,ik}u_{i,t}\right\vert \right) ^{2} \\
&\leq &2\dsum\limits_{k=1}^{Kp}\left[ \frac{1}{N_{1}}\dsum\limits_{i\in H}%
\mathbb{I}\left\{ i\in \widehat{H^{c}}\right\} \right] \left[
\dsum\limits_{i\in H}g_{1,ik}^{2}u_{i,t}^{2}\right]
\end{eqnarray*}%
Note that, for either the case where $\mathbb{S}_{i,T}^{+}=\dsum\nolimits_{%
\ell =1}^{d}\varpi _{\ell }\left\vert S_{i,\ell ,T}\right\vert $ or the case
where $\mathbb{S}_{i,T}^{+}=\max_{1\leq \ell \leq d}\left\vert S_{i,\ell
,T}\right\vert $, we have, by applying an argument similar to that given in
the proof of Theorem 1 in Chao and Swanson (2022a),%
\begin{eqnarray*}
0 &\leq &E\left[ \frac{1}{N_{1}}\dsum\limits_{i\in H}\mathbb{I}\left\{ i\in 
\widehat{H^{c}}\right\} \right] \\
&=&\frac{1}{N_{1}}\dsum\limits_{i\in H}\Pr \left( i\in \widehat{H^{c}}\right)
\\
&=&\frac{1}{N_{1}}\dsum\limits_{i\in H}P\left( \mathbb{S}_{i,T}^{+}\geq \Phi
^{-1}\left( 1-\frac{\varphi }{2N}\right) \right) \\
&\leq &\frac{N_{2}\varphi }{NN_{1}}\left\{ 1+2^{2}AT_{0}^{-\left( 1-\alpha _{%
{\large 1}}\right) \frac{{\large 1}}{{\large 2}}}+2^{2}A\Phi ^{-1}\left( 1-%
\frac{\varphi }{2N}\right) ^{3}T_{0}^{-\left( 1-\alpha _{{\large 1}}\right) 
\frac{{\large 1}}{{\large 2}}}\right\} \\
&=&\frac{N_{2}\varphi }{N_{1}\left( N_{1}+N_{2}\right) }\left[ 1+o\left(
1\right) \right] \\
&\rightarrow &0\text{ as }N_{1},N_{2},T\rightarrow \infty
\end{eqnarray*}%
Moreover, making use of part (b) of Assumption 3-3, we have%
\begin{equation*}
E\left[ \dsum\limits_{i\in H}g_{1,ik}^{2}u_{i,t}^{2}\right]
=\dsum\limits_{i\in H}g_{1,ik}^{2}E\left[ u_{i,t}^{2}\right] \leq
C\dsum\limits_{i=1}^{N}g_{1,ik}^{2}\leq C.
\end{equation*}%
It follows by Markov's inequality that%
\begin{equation*}
\frac{1}{N_{1}}\dsum\limits_{i\in H}\mathbb{I}\left\{ i\in \widehat{H^{c}}%
\right\} =o_{p}\left( 1\right) \text{ and }\dsum\limits_{i\in
H}g_{1,ik}^{2}u_{i,t}^{2}=O_{p}\left( 1\right)
\end{equation*}%
from which we deduce that%
\begin{eqnarray*}
&&\frac{2}{N_{1}}\dsum\limits_{k=1}^{Kp}\dsum\limits_{i\in
H}\dsum\limits_{j\in H}\mathbb{I}\left\{ i\in \widehat{H^{c}}\right\} 
\mathbb{I}\left\{ j\in \widehat{H^{c}}\right\} g_{1,ik}g_{1,jk}u_{i,t}u_{j,t}
\\
&\leq &2\dsum\limits_{k=1}^{Kp}\left[ \frac{1}{N_{1}}\dsum\limits_{i\in H}%
\mathbb{I}\left\{ i\in \widehat{H^{c}}\right\} \right] \left[
\dsum\limits_{i\in H}g_{1,ik}^{2}u_{i,t}^{2}\right] \\
&=&o_{p}\left( 1\right) \text{. }
\end{eqnarray*}

Combining these results and using the inequality $\sqrt{a_{1}+a_{2}}\leq 
\sqrt{a_{1}}+\sqrt{a_{2}}$, we further obtain, for all $t$,%
\begin{eqnarray*}
\left\Vert \frac{G_{1}^{\prime }U_{t,N}\left( \widehat{H^{c}}\right) }{\sqrt{%
N_{1}}}\right\Vert _{2} &\leq &\sqrt{\frac{2}{N_{1}}\dsum\limits_{k=1}^{K}%
\dsum\limits_{i\in H^{{\large c}}}\dsum\limits_{j\in H^{{\large c}}}\mathbb{I%
}\left\{ i\in \widehat{H^{c}}\right\} \mathbb{I}\left\{ j\in \widehat{H^{c}}%
\right\} g_{1,ik}g_{1,jk}u_{i,t}u_{j,t}} \\
&&+\sqrt{\frac{2}{N_{1}}\dsum\limits_{k=1}^{K}\dsum\limits_{i\in
H}\dsum\limits_{j\in H}\mathbb{I}\left\{ i\in \widehat{H^{c}}\right\} 
\mathbb{I}\left\{ j\in \widehat{H^{c}}\right\} g_{1,ik}g_{1,jk}u_{i,t}u_{j,t}%
} \\
&=&o_{p}\left( 1\right) +o_{p}\left( 1\right) \\
&=&o_{p}\left( 1\right) \text{.}
\end{eqnarray*}

For part (e), write 
\begin{eqnarray*}
\left\Vert \frac{U_{t,N}\left( \widehat{H^{c}}\right) }{\sqrt{N_{1}}}%
\right\Vert _{2}^{2} &=&\frac{U_{t,N}\left( \widehat{H^{c}}\right) ^{\prime
}U_{t,N}\left( \widehat{H^{c}}\right) }{N_{1}} \\
&=&\frac{1}{N_{1}}\dsum\limits_{i=1}^{N}\mathbb{I}\left\{ i\in \widehat{H^{c}%
}\right\} u_{i,t}^{2} \\
&=&\frac{1}{N_{1}}\dsum\limits_{i\in H^{{\large c}}}\mathbb{I}\left\{ i\in 
\widehat{H^{c}}\right\} u_{i,t}^{2}+\frac{1}{N_{1}}\dsum\limits_{i\in H}%
\mathbb{I}\left\{ i\in \widehat{H^{c}}\right\} u_{i,t}^{2} \\
&\leq &\frac{1}{N_{1}}\dsum\limits_{i\in H^{{\large c}}}u_{i,t}^{2}+\frac{1}{%
N_{1}}\dsum\limits_{i\in H}\mathbb{I}\left\{ i\in \widehat{H^{c}}\right\}
u_{i,t}^{2}
\end{eqnarray*}%
Note that, by Assumption 3-3(b),%
\begin{equation*}
E\left[ \frac{1}{N_{1}}\dsum\limits_{i\in H^{{\large c}}}u_{i,t}^{2}\right] =%
\frac{1}{N_{1}}\dsum\limits_{i\in H^{{\large c}}}E\left[ u_{i,t}^{2}\right]
\leq C\text{ }\left( \text{since }N_{1}=\#\left\{ H^{c}\right\} \right)
\end{equation*}%
so that, by applying Markov's inequality, we obtain%
\begin{equation*}
\frac{1}{N_{1}}\dsum\limits_{i\in H^{{\large c}}}u_{i,t}^{2}=O_{p}\left(
1\right) \text{.}
\end{equation*}%
Moreover, note that, for any $\epsilon >0$,%
\begin{equation*}
\dbigcap\limits_{i\in H}\left\{ i\notin \widehat{H^{c}}\right\} \subseteq
\left\{ \frac{1}{N_{1}}\dsum\limits_{i\in H}\mathbb{I}\left\{ i\in \widehat{%
H^{c}}\right\} u_{i,t}^{2}<\epsilon \right\}
\end{equation*}%
so that by DeMorgan's law%
\begin{equation*}
\left\{ \frac{1}{N_{1}}\dsum\limits_{i\in H}\mathbb{I}\left\{ i\in \widehat{%
H^{c}}\right\} u_{i,t}^{2}\geq \epsilon \right\} \subseteq \left\{
\dbigcap\limits_{i\in H}\left\{ i\notin \widehat{H^{c}}\right\} \right\}
^{c}=\dbigcup\limits_{i\in H}\left\{ i\in \widehat{H^{c}}\right\} \text{ }
\end{equation*}%
Hence, for either the case where $\mathbb{S}_{i,T}^{+}=\dsum\nolimits_{\ell
=1}^{d}\varpi _{\ell }\left\vert S_{i,\ell ,T}\right\vert $ or the case
where $\mathbb{S}_{i,T}^{+}=\max_{1\leq \ell \leq d}\left\vert S_{i,\ell
,T}\right\vert $, we have, by applying an argument similar to that given in
the proof of Theorem 1 in Chao and Swanson (2022a),%
\begin{eqnarray*}
&&\Pr \left\{ \frac{1}{N_{1}}\dsum\limits_{i\in H}\mathbb{I}\left\{ i\in 
\widehat{H^{c}}\right\} u_{i,t}^{2}\geq \epsilon \right\} \\
&\leq &\Pr \left\{ \dbigcup\limits_{i\in H}\left\{ i\in \widehat{H^{c}}%
\right\} \right\} \\
&\leq &\dsum\limits_{i\in H}\Pr \left\{ i\in \widehat{H^{c}}\right\} \\
&=&\dsum\limits_{i\in H}P\left( \mathbb{S}_{i,T}^{+}\geq \Phi ^{-1}\left( 1-%
\frac{\varphi }{2N}\right) \right) \\
&\leq &\frac{N_{2}\varphi }{N}\left\{ 1+2^{2}AT_{0}^{-\left( 1-\alpha _{%
{\large 1}}\right) \frac{{\large 1}}{{\large 2}}}+2^{2}A\Phi ^{-1}\left( 1-%
\frac{\varphi }{2N}\right) ^{3}T_{0}^{-\left( 1-\alpha _{{\large 1}}\right) 
\frac{{\large 1}}{{\large 2}}}\right\} \\
&=&\frac{N_{2}\varphi }{N_{1}+N_{2}}\left[ 1+o\left( 1\right) \right] \\
&\rightarrow &0\text{ as }N_{1},N_{2},T\rightarrow \infty
\end{eqnarray*}%
Hence,%
\begin{equation*}
\frac{1}{N_{1}}\dsum\limits_{i\in H}\mathbb{I}\left\{ i\in \widehat{H^{c}}%
\right\} u_{i,t}^{2}=o_{p}\left( 1\right)
\end{equation*}%
from which it further follows that%
\begin{eqnarray*}
\left\Vert \frac{U_{t,N}\left( \widehat{H^{c}}\right) }{\sqrt{N_{1}}}%
\right\Vert _{2}^{2} &\leq &\frac{1}{N_{1}}\dsum\limits_{i\in H^{{\large c}%
}}u_{i,t}^{2}+\frac{1}{N_{1}}\dsum\limits_{i\in H}\mathbb{I}\left\{ i\in 
\widehat{H^{c}}\right\} u_{i,t}^{2} \\
&=&O_{p}\left( 1\right) +o_{p}\left( 1\right) \\
&=&O_{p}\left( 1\right) \text{.}
\end{eqnarray*}

Turning our attention to part (f), note first that since $G=\left[ 
\begin{array}{cc}
G_{1} & G_{2}%
\end{array}%
\right] $ is an orthogonal matrix, we have $I_{N}=GG^{\prime
}=G_{1}G_{1}^{\prime }+G_{2}G_{2}^{\prime }$ or $G_{2}G_{2}^{\prime
}=I_{N}-G_{1}G_{1}^{\prime }$. Hence, we can write%
\begin{eqnarray*}
\left\Vert \frac{G_{2}^{\prime }U_{t,N}\left( \widehat{H^{c}}\right) }{\sqrt{%
N_{1}}}\right\Vert _{2}^{2} &=&\frac{U_{t,N}\left( \widehat{H^{c}}\right)
^{\prime }U_{t,N}\left( \widehat{H^{c}}\right) }{N_{1}}-\frac{U_{t,N}\left( 
\widehat{H^{c}}\right) ^{\prime }G_{1}G_{1}^{\prime }U_{t,N}\left( \widehat{%
H^{c}}\right) }{N_{1}} \\
&\leq &\frac{U_{t,N}\left( \widehat{H^{c}}\right) ^{\prime }U_{t,N}\left( 
\widehat{H^{c}}\right) }{N_{1}}+\frac{U_{t,N}\left( \widehat{H^{c}}\right)
^{\prime }G_{1}G_{1}^{\prime }U_{t,N}\left( \widehat{H^{c}}\right) }{N_{1}}
\\
&=&\left\Vert \frac{U_{t,N}\left( \widehat{H^{c}}\right) }{\sqrt{N_{1}}}%
\right\Vert _{2}^{2}+\left\Vert \frac{G_{1}^{\prime }U_{t,N}\left( \widehat{%
H^{c}}\right) }{\sqrt{N_{1}}}\right\Vert _{2}^{2}
\end{eqnarray*}%
Applying the results from parts (d) and (e) above, we then obtain%
\begin{eqnarray*}
\left\Vert \frac{G_{2}^{\prime }U_{t,N}\left( \widehat{H^{c}}\right) }{\sqrt{%
N_{1}}}\right\Vert _{2}^{2} &\leq &\left\Vert \frac{U_{t,N}\left( \widehat{%
H^{c}}\right) }{\sqrt{N_{1}}}\right\Vert _{2}^{2}+\left\Vert \frac{%
G_{1}^{\prime }U_{t,N}\left( \widehat{H^{c}}\right) }{\sqrt{N_{1}}}%
\right\Vert _{2}^{2} \\
&=&O_{p}\left( 1\right) +o_{p}\left( 1\right) \\
&=&O_{p}\left( 1\right) \text{.}
\end{eqnarray*}%
so that%
\begin{equation*}
\left\Vert \frac{G_{2}^{\prime }U_{t,N}\left( \widehat{H^{c}}\right) }{\sqrt{%
N_{1}}}\right\Vert _{2}=O_{p}\left( 1\right) \text{.}
\end{equation*}

Now, to show part (g), first write 
\begin{eqnarray*}
\frac{\widehat{V}^{\prime }\widehat{G}_{1}^{{\Large \prime }}U_{t,N}\left( 
\widehat{H^{c}}\right) }{\sqrt{\widehat{N}_{1}}} &=&\frac{\widehat{V}%
^{\prime }\widehat{G}_{1}^{{\Large \prime }}U_{t,N}\left( \widehat{H^{c}}%
\right) }{\sqrt{N_{1}}\sqrt{\left( \widehat{N}_{1}-N_{1}+N_{1}\right) /N_{1}}%
} \\
&=&\left( 1+\frac{\widehat{N}_{1}-N_{1}}{N_{1}}\right) ^{-\frac{{\large 1}}{%
{\large 2}}}\frac{\widehat{V}^{\prime }\widehat{G}_{1}^{{\Large \prime }%
}U_{t,N}\left( \widehat{H^{c}}\right) }{\sqrt{N_{1}}}
\end{eqnarray*}%
Note that%
\begin{eqnarray*}
&&\left\Vert \frac{\widehat{V}^{\prime }\widehat{G}_{1}^{{\Large \prime }%
}U_{t,N}\left( \widehat{H^{c}}\right) }{\sqrt{N_{1}}}\right\Vert _{2} \\
&=&\left\Vert \frac{\widehat{V}^{\prime }\left( I_{Kp}+R^{\prime }R\right)
^{-1/2}\left[ G_{1}^{\prime }U_{t,N}\left( \widehat{H^{c}}\right) +R^{\prime
}G_{2}^{\prime }U_{t,N}\left( \widehat{H^{c}}\right) \right] }{\sqrt{N_{1}}}%
\right\Vert _{2} \\
&\leq &\left\Vert \widehat{V}\right\Vert _{2}\left\Vert \left(
I_{Kp}+R^{\prime }R\right) ^{-1/2}\right\Vert _{2}\left\Vert \frac{%
G_{1}^{\prime }U_{t,N}\left( \widehat{H^{c}}\right) }{\sqrt{N_{1}}}%
\right\Vert _{2} \\
&&+\left\Vert \widehat{V}\right\Vert _{2}\left\Vert \left( I_{Kp}+R^{\prime
}R\right) ^{-1/2}\right\Vert _{2}\left\Vert R\right\Vert _{2}\left\Vert 
\frac{G_{2}^{\prime }U_{t,N}\left( \widehat{H^{c}}\right) }{\sqrt{N_{1}}}%
\right\Vert _{2} \\
&=&\left\Vert \left( I_{Kp}+R^{\prime }R\right) ^{-1/2}\right\Vert
_{2}\left\Vert \frac{G_{1}^{\prime }U_{t,N}\left( \widehat{H^{c}}\right) }{%
\sqrt{N_{1}}}\right\Vert _{2}+\left\Vert \left( I_{Kp}+R^{\prime }R\right)
^{-1/2}\right\Vert _{2}\left\Vert R\right\Vert _{2}\left\Vert \frac{%
G_{2}^{\prime }U_{t,N}\left( \widehat{H^{c}}\right) }{\sqrt{N_{1}}}%
\right\Vert _{2} \\
&&\left( \text{since }\widehat{V}^{\prime }\widehat{V}=I_{Kp}\text{ so that }%
\left\Vert \widehat{V}\right\Vert _{2}=1\right)
\end{eqnarray*}%
It follows that

\begin{eqnarray*}
&&\left\Vert \frac{\widehat{V}^{\prime }\widehat{G}_{1}^{{\Large \prime }%
}U_{t,N}\left( \widehat{H^{c}}\right) }{\sqrt{\widehat{N}_{1}}}\right\Vert
_{2} \\
&=&\left\vert \left( 1+\frac{\widehat{N}_{1}-N_{1}}{N_{1}}\right) ^{-\frac{%
{\large 1}}{{\large 2}}}\right\vert \left\Vert \frac{\widehat{V}^{\prime }%
\widehat{G}_{1}^{{\Large \prime }}U_{t,N}\left( \widehat{H^{c}}\right) }{%
\sqrt{N_{1}}}\right\Vert _{2} \\
&\leq &\left\vert \left( 1+\frac{\widehat{N}_{1}-N_{1}}{N_{1}}\right) ^{-%
\frac{{\large 1}}{{\large 2}}}\right\vert \left\{ \left\Vert \left(
I_{Kp}+R^{\prime }R\right) ^{-1/2}\right\Vert _{2}\left\Vert \frac{%
G_{1}^{\prime }U_{t,N}\left( \widehat{H^{c}}\right) }{\sqrt{N_{1}}}%
\right\Vert _{2}\right. \\
&&\left. +\left\Vert \left( I_{Kp}+R^{\prime }R\right) ^{-1/2}\right\Vert
_{2}\left\Vert R\right\Vert _{2}\left\Vert \frac{G_{2}^{\prime
}U_{t,N}\left( \widehat{H^{c}}\right) }{\sqrt{N_{1}}}\right\Vert _{2}\right\}
\\
&\leq &\left\vert \left( 1+\frac{\widehat{N}_{1}-N_{1}}{N_{1}}\right) ^{-%
\frac{{\large 1}}{{\large 2}}}\right\vert \left\{ \frac{1}{\sqrt[\backslash ]%
{1+\lambda _{\min }\left( R^{\prime }R\right) }}\left\Vert \frac{%
G_{1}^{\prime }U_{t,N}\left( \widehat{H^{c}}\right) }{\sqrt{N_{1}}}%
\right\Vert _{2}\right. \\
&&\left. +\frac{\left\Vert R\right\Vert _{2}}{\sqrt[\backslash ]{1+\lambda
_{\min }\left( R^{\prime }R\right) }}\left\Vert \frac{G_{2}^{\prime
}U_{t,N}\left( \widehat{H^{c}}\right) }{\sqrt{N_{1}}}\right\Vert _{2}\right\}
\\
&\leq &\left\vert \left( 1+\frac{\widehat{N}_{1}-N_{1}}{N_{1}}\right) ^{-%
\frac{{\large 1}}{{\large 2}}}\right\vert \left\{ \left\Vert \frac{%
G_{1}^{\prime }U_{t,N}\left( \widehat{H^{c}}\right) }{\sqrt{N_{1}}}%
\right\Vert _{2}+\left\Vert R\right\Vert _{2}\left\Vert \frac{G_{2}^{\prime
}U_{t,N}\left( \widehat{H^{c}}\right) }{\sqrt{N_{1}}}\right\Vert _{2}\right\}
\\
&=&o_{p}\left( 1\right)
\end{eqnarray*}%
where the last line follows from the fact that%
\begin{equation*}
\left\Vert R\right\Vert _{2}\overset{p}{\rightarrow }0\text{, }\left\vert
\left( 1+\frac{\widehat{N}_{1}-N_{1}}{N_{1}}\right) ^{-\frac{{\large 1}}{%
{\large 2}}}\right\vert \overset{p}{\rightarrow }1\text{, }\left\Vert \frac{%
G_{1}^{\prime }U_{t,N}\left( \widehat{H^{c}}\right) }{\sqrt{N_{1}}}%
\right\Vert _{2}\overset{p}{\rightarrow }0\text{, and }\left\Vert \frac{%
G_{2}^{\prime }U_{t,N}\left( \widehat{H^{c}}\right) }{\sqrt{N_{1}}}%
\right\Vert _{2}=O_{p}\left( 1\right)
\end{equation*}%
as shown in part (a) in Lemma D-14 and in parts (a), (d), and (f) of this
lemma.

Turning our attention to part (h), we write%
\begin{eqnarray*}
&&\frac{\widehat{V}^{\prime }\widehat{G}_{1}^{{\Large \prime }}\Gamma \left( 
\widehat{H^{c}}\right) }{\sqrt{\widehat{N}_{1}}} \\
&=&Q^{\prime }+\left( \frac{\widehat{V}^{\prime }\widehat{G}_{1}^{{\Large %
\prime }}\Gamma }{\sqrt{\widehat{N}_{1}}}-Q^{\prime }\right) +\widehat{V}%
^{\prime }\widehat{G}_{1}^{{\Large \prime }}\left( \frac{\Gamma \left( 
\widehat{H^{c}}\right) -\Gamma }{\sqrt{\widehat{N}_{1}}}\right) \\
&=&Q^{\prime }+\left( \left[ \left( 1+\frac{\widehat{N}_{1}-N_{1}}{N_{1}}%
\right) ^{-\frac{{\large 1}}{{\large 2}}}-1+1\right] \frac{\widehat{V}%
^{\prime }\widehat{G}_{1}^{{\Large \prime }}\Gamma }{\sqrt{N_{1}}}-Q^{\prime
}\right) \\
&&+\left[ \left( 1+\frac{\widehat{N}_{1}-N_{1}}{N_{1}}\right) ^{-\frac{%
{\large 1}}{{\large 2}}}-1+1\right] \widehat{V}^{\prime }\widehat{G}_{1}^{%
{\Large \prime }}\left( \frac{\Gamma \left( \widehat{H^{c}}\right) -\Gamma }{%
\sqrt{N}_{1}}\right) \\
&=&Q^{\prime }+\left( \frac{\widehat{V}^{\prime }\widehat{G}_{1}^{{\Large %
\prime }}\Gamma }{\sqrt{N_{1}}}-Q^{\prime }\right) +\left[ \left( 1+\frac{%
\widehat{N}_{1}-N_{1}}{N_{1}}\right) ^{-\frac{{\large 1}}{{\large 2}}}-1%
\right] \frac{\widehat{V}^{\prime }\widehat{G}_{1}^{{\Large \prime }}\Gamma 
}{\sqrt{N_{1}}} \\
&&+\left[ \left( 1+\frac{\widehat{N}_{1}-N_{1}}{N_{1}}\right) ^{-\frac{%
{\large 1}}{{\large 2}}}-1\right] \widehat{V}^{\prime }\widehat{G}_{1}^{%
{\Large \prime }}\left( \frac{\Gamma \left( \widehat{H^{c}}\right) -\Gamma }{%
\sqrt{N}_{1}}\right) +\widehat{V}^{\prime }\widehat{G}_{1}^{{\Large \prime }%
}\left( \frac{\Gamma \left( \widehat{H^{c}}\right) -\Gamma }{\sqrt{N}_{1}}%
\right)
\end{eqnarray*}%
so that, by the triangle inequality%
\begin{eqnarray*}
&&\left\Vert \frac{\Gamma \left( \widehat{H^{c}}\right) ^{{\Large \prime }}%
\widehat{G}_{1}\widehat{V}}{\sqrt{\widehat{N}_{1}}}-Q\right\Vert _{2} \\
&=&\left\Vert \frac{\widehat{V}^{\prime }\widehat{G}_{1}^{{\Large \prime }%
}\Gamma \left( \widehat{H^{c}}\right) }{\sqrt{\widehat{N}_{1}}}-Q^{\prime
}\right\Vert _{2} \\
&\leq &\left\Vert \frac{\widehat{V}^{\prime }\widehat{G}_{1}^{{\Large \prime 
}}\Gamma }{\sqrt{N_{1}}}-Q^{\prime }\right\Vert _{2}+\left\Vert \left[
\left( 1+\frac{\widehat{N}_{1}-N_{1}}{N_{1}}\right) ^{-\frac{{\large 1}}{%
{\large 2}}}-1\right] \frac{\widehat{V}^{\prime }\widehat{G}_{1}^{{\Large %
\prime }}\Gamma }{\sqrt{N_{1}}}\right\Vert _{2} \\
&&+\left\Vert \left[ \left( 1+\frac{\widehat{N}_{1}-N_{1}}{N_{1}}\right) ^{-%
\frac{{\large 1}}{{\large 2}}}-1\right] \widehat{V}^{\prime }\widehat{G}%
_{1}^{{\Large \prime }}\left( \frac{\Gamma \left( \widehat{H^{c}}\right)
-\Gamma }{\sqrt{N}_{1}}\right) \right\Vert _{2}+\left\Vert \widehat{V}%
^{\prime }\widehat{G}_{1}^{{\Large \prime }}\left( \frac{\Gamma \left( 
\widehat{H^{c}}\right) -\Gamma }{\sqrt{N}_{1}}\right) \right\Vert _{2} \\
&\leq &\left\Vert \frac{\widehat{V}^{\prime }\widehat{G}_{1}^{{\Large \prime 
}}\Gamma }{\sqrt{N_{1}}}-Q^{\prime }\right\Vert _{2}+\left\vert \left( 1+%
\frac{\widehat{N}_{1}-N_{1}}{N_{1}}\right) ^{-\frac{{\large 1}}{{\large 2}}%
}-1\right\vert \left\Vert \frac{\widehat{V}^{\prime }\widehat{G}_{1}^{%
{\Large \prime }}\Gamma }{\sqrt{N_{1}}}\right\Vert _{2} \\
&&+\left\vert \left( 1+\frac{\widehat{N}_{1}-N_{1}}{N_{1}}\right) ^{-\frac{%
{\large 1}}{{\large 2}}}-1\right\vert \left\Vert \widehat{V}^{\prime }%
\widehat{G}_{1}^{{\Large \prime }}\right\Vert _{2}\left\Vert \frac{\Gamma
\left( \widehat{H^{c}}\right) -\Gamma }{\sqrt{N}_{1}}\right\Vert
_{2}+\left\Vert \widehat{V}^{\prime }\widehat{G}_{1}^{{\Large \prime }%
}\right\Vert _{2}\left\Vert \frac{\Gamma \left( \widehat{H^{c}}\right)
-\Gamma }{\sqrt{N}_{1}}\right\Vert _{2} \\
&=&\left\Vert \frac{\widehat{V}^{\prime }\widehat{G}_{1}^{{\Large \prime }%
}\Gamma }{\sqrt{N_{1}}}-Q^{\prime }\right\Vert _{2}+\left\vert \left( 1+%
\frac{\widehat{N}_{1}-N_{1}}{N_{1}}\right) ^{-\frac{{\large 1}}{{\large 2}}%
}-1\right\vert \left\Vert \frac{\widehat{V}^{\prime }\widehat{G}_{1}^{%
{\Large \prime }}\Gamma }{\sqrt{N_{1}}}\right\Vert _{2} \\
&&+\left\vert \left( 1+\frac{\widehat{N}_{1}-N_{1}}{N_{1}}\right) ^{-\frac{%
{\large 1}}{{\large 2}}}-1\right\vert \left\Vert \frac{\Gamma \left( 
\widehat{H^{c}}\right) -\Gamma }{\sqrt{N}_{1}}\right\Vert _{2}+\left\Vert 
\frac{\Gamma \left( \widehat{H^{c}}\right) -\Gamma }{\sqrt{N}_{1}}%
\right\Vert _{2}
\end{eqnarray*}%
where the last equality follows from the fact that%
\begin{equation*}
\left\Vert \widehat{V}^{\prime }\widehat{G}_{1}^{{\Large \prime }%
}\right\Vert _{2}=\left\Vert \widehat{G}_{1}\widehat{V}\right\Vert _{2}=%
\sqrt{\lambda _{\max }\left( \widehat{V}^{\prime }\widehat{G}_{1}^{{\Large %
\prime }}\widehat{G}_{1}\widehat{V}\right) }=\sqrt{\lambda _{\max }\left(
I_{Kp}\right) }=1\text{.}
\end{equation*}%
Now, by parts (a), (b), and (c) of this lemma, we have that%
\begin{equation*}
\left( 1+\frac{\widehat{N}_{1}-N_{1}}{N_{1}}\right) ^{-\frac{{\large 1}}{%
{\large 2}}}-1\overset{p}{\rightarrow }0\text{, }\left\Vert \frac{\Gamma
\left( \widehat{H^{c}}\right) -\Gamma }{\sqrt{N_{1}}}\right\Vert _{2}\overset%
{p}{\rightarrow }0\text{,}\left\Vert \frac{\widehat{V}^{\prime }\widehat{G}%
_{1}^{{\Large \prime }}\Gamma }{\sqrt{N_{1}}}-Q^{\prime }\right\Vert _{2}%
\overset{p}{\rightarrow }0\text{, and }
\end{equation*}%
and%
\begin{equation*}
\left\Vert \frac{\widehat{V}^{\prime }\widehat{G}_{1}^{\prime }\Gamma }{%
\sqrt{N_{1}}}\right\Vert _{2}\leq \sqrt{\lambda _{\max }\left( \frac{\Gamma
^{\prime }\Gamma }{N_{1}}\right) }\leq \overline{C}<\infty \text{ for all }%
N_{1},N_{2}\text{ sufficiently large.}
\end{equation*}%
It follows that%
\begin{eqnarray*}
\left\Vert \frac{\Gamma \left( \widehat{H^{c}}\right) ^{{\Large \prime }}%
\widehat{G}_{1}\widehat{V}}{\sqrt{\widehat{N}_{1}}}-Q\right\Vert _{2}
&=&\left\Vert \frac{\widehat{V}^{\prime }\widehat{G}_{1}^{{\Large \prime }%
}\Gamma \left( \widehat{H^{c}}\right) }{\sqrt{\widehat{N}_{1}}}-Q^{\prime
}\right\Vert _{2} \\
&\leq &\left\Vert \frac{\widehat{V}^{\prime }\widehat{G}_{1}^{{\Large \prime 
}}\Gamma }{\sqrt{N_{1}}}-Q^{\prime }\right\Vert _{2}+\left\vert \left( 1+%
\frac{\widehat{N}_{1}-N_{1}}{N_{1}}\right) ^{-\frac{{\large 1}}{{\large 2}}%
}-1\right\vert \left\Vert \frac{\widehat{V}^{\prime }\widehat{G}_{1}^{%
{\Large \prime }}\Gamma }{\sqrt{N_{1}}}\right\Vert _{2} \\
&&+\left\vert \left( 1+\frac{\widehat{N}_{1}-N_{1}}{N_{1}}\right) ^{-\frac{%
{\large 1}}{{\large 2}}}-1\right\vert \left\Vert \frac{\Gamma \left( 
\widehat{H^{c}}\right) -\Gamma }{\sqrt{N}_{1}}\right\Vert _{2}+\left\Vert 
\frac{\Gamma \left( \widehat{H^{c}}\right) -\Gamma }{\sqrt{N}_{1}}%
\right\Vert _{2} \\
&=&o_{p}\left( 1\right) \text{. }
\end{eqnarray*}

To show part (i), let $\overline{C}$ be the positive constant given in Lemma
C-5 such that%
\begin{equation*}
E\left\Vert \underline{F}_{t}\right\Vert _{2}^{6}\leq \overline{C}<\infty 
\text{ for all }t\text{;}
\end{equation*}%
and, for any $\epsilon >0$, we let $C_{\epsilon }=\overline{C}^{\frac{%
{\Large 1}}{{\Large 6}}}/\sqrt{\epsilon }$. Applying Markov's inequality, we
see that%
\begin{eqnarray*}
\Pr \left( \left\Vert \underline{F}_{t}\right\Vert _{2}\geq C_{\epsilon
}\right) &\leq &\Pr \left( \left\Vert \underline{F}_{t}\right\Vert
_{2}^{2}\geq C_{\epsilon }^{2}\right) \\
&\leq &\frac{1}{C_{\epsilon }^{2}}E\left\Vert \underline{F}_{t}\right\Vert
_{2}^{2} \\
&\leq &\frac{1}{C_{\epsilon }^{2}}\left( E\left\Vert \underline{F}%
_{t}\right\Vert _{2}^{6}\right) ^{\frac{{\Large 1}}{{\Large 3}}} \\
&&\left( \text{by Liapunov's inequality}\right) \\
&\leq &\frac{\epsilon }{\overline{C}^{\frac{{\Large 1}}{{\Large 3}}}}%
\overline{C}^{\frac{{\Large 1}}{{\Large 3}}} \\
&\leq &\epsilon \text{ }
\end{eqnarray*}%
from which it follows that $\left\Vert \underline{F}_{t}\right\Vert
_{2}=O_{p}\left( 1\right) $ for all $t$.

Lastly, to show part (j), note that, similar to the derivation given in the
proof of Theorem 4.1, except that we replace the fixed index $t$ with the
sample size $T$, we can write%
\begin{eqnarray*}
\widehat{\underline{F}}_{T}-Q^{\prime }\underline{F}_{T} &=&\left( \frac{%
\widehat{V}^{\prime }\widehat{G}_{1}^{{\Large \prime }}\Gamma \left( 
\widehat{H^{c}}\right) }{\sqrt{\widehat{N}_{1}}}-Q^{\prime }\right) 
\underline{F}_{T}+\frac{\widehat{V}^{\prime }\widehat{G}_{1}^{{\Large \prime 
}}U_{T,N}\left( \widehat{H^{c}}\right) }{\sqrt{\widehat{N}_{1}}} \\
&=&\left( \frac{\widehat{V}^{\prime }\widehat{G}_{1}^{{\Large \prime }%
}\Gamma }{\sqrt{N_{1}}}-Q^{\prime }\right) \underline{F}_{T}+\left[ \left( 1+%
\frac{\widehat{N}_{1}-N_{1}}{N_{1}}\right) ^{-\frac{{\large 1}}{{\large 2}}%
}-1\right] \frac{\widehat{V}^{\prime }\widehat{G}_{1}^{{\Large \prime }%
}\Gamma }{\sqrt{N_{1}}}\underline{F}_{T} \\
&&+\left[ \left( 1+\frac{\widehat{N}_{1}-N_{1}}{N_{1}}\right) ^{-\frac{%
{\large 1}}{{\large 2}}}\right] \widehat{V}^{\prime }\widehat{G}_{1}^{%
{\Large \prime }}\left( \frac{\Gamma \left( \widehat{H^{c}}\right) -\Gamma }{%
\sqrt{N}_{1}}\right) \underline{F}_{T}+\frac{\widehat{V}^{\prime }\widehat{G}%
_{1}^{{\Large \prime }}U_{T,N}\left( \widehat{H^{c}}\right) }{\sqrt{\widehat{%
N}_{1}}}
\end{eqnarray*}%
Next, note that, by following the same derivation as that given for the
proof of part (g), we can show that%
\begin{eqnarray*}
&&\left\Vert \frac{\widehat{V}^{\prime }\widehat{G}_{1}^{{\Large \prime }%
}U_{T,N}\left( \widehat{H^{c}}\right) }{\sqrt{\widehat{N}_{1}}}\right\Vert
_{2} \\
&\leq &\left\vert \left( 1+\frac{\widehat{N}_{1}-N_{1}}{N_{1}}\right) ^{-%
\frac{{\large 1}}{{\large 2}}}\right\vert \left\{ \left\Vert \frac{%
G_{1}^{\prime }U_{T,N}\left( \widehat{H^{c}}\right) }{\sqrt{N_{1}}}%
\right\Vert _{2}+\left\Vert R\right\Vert _{2}\left\Vert \frac{G_{2}^{\prime
}U_{T,N}\left( \widehat{H^{c}}\right) }{\sqrt{N_{1}}}\right\Vert _{2}\right\}
\end{eqnarray*}%
Moreover, by argument similar to that given for parts (d) and (f) of this
lemma, we can show that, as $N_{1}$, $N_{2}$, and $T\rightarrow \infty $;%
\begin{equation}
\left\Vert \frac{G_{1}^{\prime }U_{T,N}\left( \widehat{H^{c}}\right) }{\sqrt{%
N_{1}}}\right\Vert _{2}\overset{p}{\rightarrow }0\text{ }  \label{G1UTN}
\end{equation}%
and%
\begin{equation}
\left\Vert \frac{G_{2}^{\prime }U_{T,N}\left( \widehat{H^{c}}\right) }{\sqrt{%
N_{1}}}\right\Vert _{2}=O_{p}\left( 1\right) \text{.}  \label{G2UTN}
\end{equation}%
It follows from applying expressions (\ref{G1UTN}) and (\ref{G2UTN}), part
(a) of this lemma, and part (a) of Lemma D-14 that%
\begin{equation}
\left\Vert \frac{\widehat{V}^{\prime }\widehat{G}_{1}^{{\Large \prime }%
}U_{T,N}\left( \widehat{H^{c}}\right) }{\sqrt{\widehat{N}_{1}}}\right\Vert
_{2}\overset{p}{\rightarrow }0\text{ as }N_{1},N_{2},\text{ and }%
T\rightarrow \infty \text{. }  \label{VG1UTN}
\end{equation}%
In addition, note that by applying Lemma C-5 and the Markov's inequality in
a way similar to the argument given for the proof of part (i) above, we can
show that%
\begin{equation}
\left\Vert \underline{F}_{T}\right\Vert _{2}=O_{p}\left( 1\right) \text{. \ }
\label{order of FT}
\end{equation}%
Making use of the results given in expressions (\ref{VG1UTN}) and (\ref%
{order of FT}) and applying the triangle inequality as well as parts (a)-(c)
of this lemma, expression (\ref{order of FT}), and the Slutsky's theorem; we
then obtain, as $N_{1},N_{2},$ and $T\rightarrow \infty $;%
\begin{eqnarray*}
&&\left\Vert \widehat{\underline{F}}_{T}-Q^{\prime }\underline{F}%
_{T}\right\Vert _{2} \\
&\leq &\left\Vert \frac{\widehat{V}^{\prime }\widehat{G}_{1}^{{\Large \prime 
}}\Gamma }{\sqrt{N_{1}}}-Q^{\prime }\right\Vert _{2}\left\Vert \underline{F}%
_{T}\right\Vert _{2}+\left\vert \left( 1+\frac{\widehat{N}_{1}-N_{1}}{N_{1}}%
\right) ^{-\frac{{\large 1}}{{\large 2}}}-1\right\vert \left\Vert \frac{%
\widehat{V}^{\prime }\widehat{G}_{1}^{{\Large \prime }}\Gamma }{\sqrt{N_{1}}}%
\right\Vert _{2}\left\Vert \underline{F}_{T}\right\Vert _{2} \\
&&+\left\vert \left( 1+\frac{\widehat{N}_{1}-N_{1}}{N_{1}}\right) ^{-\frac{%
{\large 1}}{{\large 2}}}\right\vert \left\Vert \widehat{V}^{\prime }\widehat{%
G}_{1}^{{\Large \prime }}\right\Vert _{2}\left\Vert \frac{\Gamma \left( 
\widehat{H^{c}}\right) -\Gamma }{\sqrt{N}_{1}}\right\Vert _{2}\left\Vert 
\underline{F}_{T}\right\Vert _{2}+\left\Vert \frac{\widehat{V}^{\prime }%
\widehat{G}_{1}^{{\Large \prime }}U_{T,N}\left( \widehat{H^{c}}\right) }{%
\sqrt{\widehat{N}_{1}}}\right\Vert _{2} \\
&=&\left\Vert \frac{\widehat{V}^{\prime }\widehat{G}_{1}^{{\Large \prime }%
}\Gamma }{\sqrt{N_{1}}}-Q^{\prime }\right\Vert _{2}\left\Vert \underline{F}%
_{T}\right\Vert _{2}+\left\vert \left( 1+\frac{\widehat{N}_{1}-N_{1}}{N_{1}}%
\right) ^{-\frac{{\large 1}}{{\large 2}}}-1\right\vert \left\Vert \frac{%
\widehat{V}^{\prime }\widehat{G}_{1}^{{\Large \prime }}\Gamma }{\sqrt{N_{1}}}%
\right\Vert _{2}\left\Vert \underline{F}_{T}\right\Vert _{2} \\
&&+\left\vert \left( 1+\frac{\widehat{N}_{1}-N_{1}}{N_{1}}\right) ^{-\frac{%
{\large 1}}{{\large 2}}}\right\vert \left\Vert \frac{\Gamma \left( \widehat{%
H^{c}}\right) -\Gamma }{\sqrt{N}_{1}}\right\Vert _{2}\left\Vert \underline{F}%
_{T}\right\Vert _{2}+\left\Vert \frac{\widehat{V}^{\prime }\widehat{G}_{1}^{%
{\Large \prime }}U_{T,N}\left( \widehat{H^{c}}\right) }{\sqrt{\widehat{N}_{1}%
}}\right\Vert _{2} \\
&&\left( \text{again since }\left\Vert \widehat{V}^{\prime }\widehat{G}_{1}^{%
{\Large \prime }}\right\Vert _{2}=\lambda _{\max }\left( \widehat{G}_{1}%
\widehat{V}\widehat{V}^{\prime }\widehat{G}_{1}^{{\Large \prime }}\right)
=\lambda _{\max }\left( \widehat{V}^{\prime }\widehat{G}_{1}^{{\Large \prime 
}}\widehat{G}_{1}\widehat{V}\right) =\lambda _{\max }\left( I_{Kp}\right)
=1\right) \\
&=&o_{p}\left( 1\right) O_{p}\left( 1\right) +o_{p}\left( 1\right)
O_{p}\left( 1\right) O_{p}\left( 1\right) +O_{p}\left( 1\right) o_{p}\left(
1\right) O_{p}\left( 1\right) +o_{p}\left( 1\right) \\
&=&o_{p}\left( 1\right) \text{. }\square
\end{eqnarray*}

\medskip

\bigskip

\noindent \textbf{Lemma D-16: }Suppose that Assumptions 3-1, 3-2, 3-3, 3-4,
3-5, 3-6, 3-7, 3-8, 3-9, 3-10, and 3-11* hold. Then, the following
statements are true as $N_{1},N_{2},T\rightarrow \infty $.

\begin{enumerate}
\item[(a)] 
\begin{equation*}
\frac{1}{T_{h}}\dsum\limits_{t=p}^{T-h}\left\Vert \frac{G_{1}^{\prime
}U_{t,N}\left( \widehat{H^{c}}\right) }{\sqrt{N_{1}}}\right\Vert
_{2}=o_{p}\left( 1\right) \text{, where }T_{h}=T-h-p+1\text{.}
\end{equation*}

\item[(b)] 
\begin{equation*}
\frac{1}{T_{h}}\dsum\limits_{t=p}^{T-h}\left\Vert \frac{U_{t,N}\left( 
\widehat{H^{c}}\right) }{\sqrt{N_{1}}}\right\Vert _{2}^{2}=O_{p}\left(
1\right) .
\end{equation*}

\item[(c)] 
\begin{equation*}
\text{, }\frac{1}{T_{h}}\dsum\limits_{t=p}^{T-h}\left\Vert \frac{%
G_{2}^{\prime }U_{t,N}\left( \widehat{H^{c}}\right) }{\sqrt{N_{1}}}%
\right\Vert _{2}=O_{p}\left( 1\right)
\end{equation*}

\item[(d)] 
\begin{equation*}
\frac{1}{T_{h}}\dsum\limits_{t=p}^{T-h}\left\Vert \underline{F}%
_{t}\right\Vert _{2}^{2}=O_{p}\left( 1\right) \text{ and}\left\Vert \frac{1}{%
T_{h}}\dsum\limits_{t=p}^{T-h}\underline{F}_{t}\underline{F}_{t}^{\prime
}\right\Vert _{2}=O_{p}\left( 1\right)
\end{equation*}

\item[(e)] 
\begin{equation*}
\left\Vert \frac{\widehat{V}^{\prime }\widehat{G}_{1}^{{\Large \prime }%
}U\left( \widehat{H^{c}}\right) ^{{\Large \prime }}U\left( \widehat{H^{c}}%
\right) \widehat{G}_{1}\widehat{V}}{T_{h}\widehat{N}_{1}}\right\Vert
_{2}=o_{p}\left( 1\right)
\end{equation*}

\item[(f)] 
\begin{equation*}
\left\Vert \frac{\underline{F}^{\prime }U\left( \widehat{H^{c}}\right) 
\widehat{G}_{1}\widehat{V}}{T_{h}\sqrt{\widehat{N}_{1}}}\right\Vert
_{2}=o_{p}\left( 1\right)
\end{equation*}

\item[(g)] 
\begin{equation*}
\left\Vert \frac{1}{T_{h}}\dsum\limits_{t=p}^{T-h}\left( \widehat{F}%
_{t}-Q^{\prime }\underline{F}_{t}\right) \left( \widehat{F}_{t}-Q^{\prime }%
\underline{F}_{t}\right) ^{\prime }\right\Vert _{2}=o_{p}\left( 1\right) 
\text{.}
\end{equation*}
\end{enumerate}

\bigskip

\noindent \textbf{Proof of Lemma D-16:}

For part (a), first write%
\begin{eqnarray}
&&\frac{1}{T_{h}}\dsum\limits_{t=p}^{T-h}\left\Vert \frac{G_{1}^{\prime
}U_{t,N}\left( \widehat{H^{c}}\right) }{\sqrt{N_{1}}}\right\Vert _{2}^{2} 
\notag \\
&=&\frac{1}{T_{h}}\dsum\limits_{t=p}^{T-h}\dsum\limits_{k=1}^{Kp}\left(
\dsum\limits_{i=1}^{N}\mathbb{I}\left\{ i\in \widehat{H^{c}}\right\} \frac{%
g_{1,ik}u_{i,t}}{\sqrt{N_{1}}}\right) ^{2}  \notag \\
&=&\frac{1}{T_{h}}\dsum\limits_{t=p}^{T-h}\dsum\limits_{k=1}^{Kp}\left(
\dsum\limits_{i\in H^{{\large c}}}\mathbb{I}\left\{ i\in \widehat{H^{c}}%
\right\} \frac{g_{1,ik}u_{i,t}}{\sqrt{N_{1}}}+\dsum\limits_{i\in H}\mathbb{I}%
\left\{ i\in \widehat{H^{c}}\right\} \frac{g_{1,ik}u_{i,t}}{\sqrt{N_{1}}}%
\right) ^{2}  \notag \\
&\leq &\frac{2}{T_{h}}\dsum\limits_{t=p}^{T-h}\dsum\limits_{k=1}^{Kp}\left(
\dsum\limits_{i\in H^{{\large c}}}\mathbb{I}\left\{ i\in \widehat{H^{c}}%
\right\} \frac{g_{1,ik}u_{i,t}}{\sqrt{N_{1}}}\right) ^{2}+\frac{2}{T_{h}}%
\dsum\limits_{t=p}^{T-h}\dsum\limits_{k=1}^{Kp}\left( \dsum\limits_{i\in H}%
\mathbb{I}\left\{ i\in \widehat{H^{c}}\right\} \frac{g_{1,ik}u_{i,t}}{\sqrt{%
N_{1}}}\right) ^{2}  \notag \\
&=&\frac{2}{N_{1}T_{h}}\dsum\limits_{t=p}^{T-h}\dsum\limits_{k=1}^{Kp}\dsum%
\limits_{i\in H^{{\large c}}}\dsum\limits_{j\in H^{{\large c}}}\mathbb{I}%
\left\{ i\in \widehat{H^{c}}\right\} \mathbb{I}\left\{ j\in \widehat{H^{c}}%
\right\} g_{1,ik}g_{1,jk}u_{i,t}u_{j,t}  \notag \\
&&+\frac{2}{N_{1}T_{h}}\dsum\limits_{t=p}^{T-h}\dsum\limits_{k=1}^{Kp}\dsum%
\limits_{i\in H}\dsum\limits_{j\in H}\mathbb{I}\left\{ i\in \widehat{H^{c}}%
\right\} \mathbb{I}\left\{ j\in \widehat{H^{c}}\right\}
g_{1,ik}g_{1,jk}u_{i,t}u_{j,t}  \label{G1U term}
\end{eqnarray}%
where $g_{1,ik}$ denotes the $\left( i,k\right) ^{th}$ element of 
\begin{equation*}
G_{1}=\frac{\Gamma _{\ast }\Xi }{\sqrt{N_{1}}}=\frac{\Gamma }{\sqrt{N_{1}}}%
\left( \frac{\Gamma ^{\prime }\Gamma }{N_{1}}\right) ^{-1/2}\Xi \text{ }
\end{equation*}%
Now, where%
\begin{eqnarray*}
&&\frac{2}{N_{1}T_{h}}\dsum\limits_{t=p}^{T-h}\dsum\limits_{k=1}^{Kp}\dsum%
\limits_{i\in H^{{\large c}}}\dsum\limits_{j\in H^{{\large c}}}\mathbb{I}%
\left\{ i\in \widehat{H^{c}}\right\} \mathbb{I}\left\{ j\in \widehat{H^{c}}%
\right\} g_{1,ik}g_{1,jk}u_{i,t}u_{j,t} \\
&=&\frac{2}{N_{1}T_{h}}\dsum\limits_{t=p}^{T-h}\dsum\limits_{k=1}^{Kp}\dsum%
\limits_{i\in H^{{\large c}}}\dsum\limits_{j\in H^{{\large c}}}\left( 
\mathbb{I}\left\{ i\in \widehat{H^{c}}\right\} -1+1\right) \left( \mathbb{I}%
\left\{ j\in \widehat{H^{c}}\right\} -1+1\right)
g_{1,ik}g_{1,jk}u_{i,t}u_{j,t} \\
&=&\frac{2}{N_{1}T_{h}}\dsum\limits_{t=p}^{T-h}\dsum\limits_{k=1}^{Kp}\dsum%
\limits_{i\in H^{{\large c}}}\left( \mathbb{I}\left\{ i\in \widehat{H^{c}}%
\right\} -1\right) g_{1,ik}u_{i,t}\dsum\limits_{j\in H^{{\large c}}}\left( 
\mathbb{I}\left\{ j\in \widehat{H^{c}}\right\} -1\right) g_{1,jk}u_{j,t} \\
&&+\frac{2}{N_{1}T_{h}}\dsum\limits_{t=p}^{T-h}\dsum\limits_{k=1}^{Kp}\dsum%
\limits_{i\in H^{{\large c}}}\left( \mathbb{I}\left\{ i\in \widehat{H^{c}}%
\right\} -1\right) g_{1,ik}u_{i,t}\dsum\limits_{j\in H^{{\large c}%
}}g_{1,jk}u_{j,t} \\
&&+\frac{2}{N_{1}T_{h}}\dsum\limits_{t=p}^{T-h}\dsum\limits_{k=1}^{Kp}\dsum%
\limits_{i\in H^{{\large c}}}g_{1,ik}u_{i,t}\dsum\limits_{j\in H^{{\large c}%
}}\left( \mathbb{I}\left\{ j\in \widehat{H^{c}}\right\} -1\right)
g_{1,jk}u_{j,t} \\
&&+\frac{2}{N_{1}T_{h}}\dsum\limits_{t=p}^{T-h}\dsum\limits_{k=1}^{Kp}\dsum%
\limits_{i\in H^{{\large c}}}\dsum\limits_{j\in H^{{\large c}%
}}g_{1,ik}g_{1,jk}u_{i,t}u_{j,t} \\
&=&\underline{\mathcal{E}}_{1,1}+\underline{\mathcal{E}}_{1,2}+\underline{%
\mathcal{E}}_{1,3}+\underline{\mathcal{E}}_{1,4}
\end{eqnarray*}%
Focusing first on the term $\underline{\mathcal{E}}_{1,1}$, we have%
\begin{eqnarray*}
&&\frac{2}{N_{1}T_{h}}\dsum\limits_{t=p}^{T-h}\dsum\limits_{k=1}^{Kp}\dsum%
\limits_{i\in H^{{\large c}}}\left( \mathbb{I}\left\{ i\in \widehat{H^{c}}%
\right\} -1\right) g_{1,ik}u_{i,t}\dsum\limits_{j\in H^{{\large c}}}\left( 
\mathbb{I}\left\{ j\in \widehat{H^{c}}\right\} -1\right) g_{1,jk}u_{j,t} \\
&=&\frac{2}{N_{1}T_{h}}\dsum\limits_{t=p}^{T-h}\dsum\limits_{k=1}^{Kp}\left(
\dsum\limits_{i\in H^{{\large c}}}\left( \mathbb{I}\left\{ i\in \widehat{%
H^{c}}\right\} -1\right) g_{1,ik}u_{i,t}\right) ^{2} \\
&\leq &\frac{2}{N_{1}T_{h}}\dsum\limits_{t=p}^{T-h}\dsum\limits_{k=1}^{Kp}%
\left( \left\vert \dsum\limits_{i\in H^{{\large c}}}\left( \mathbb{I}\left\{
i\in \widehat{H^{c}}\right\} -1\right) g_{1,ik}u_{i,t}\right\vert \right)
^{2} \\
&\leq &2\dsum\limits_{k=1}^{Kp}\left( \frac{1}{N_{1}}\dsum\limits_{i\in H^{%
{\large c}}}\left( \mathbb{I}\left\{ i\in \widehat{H^{c}}\right\} -1\right)
^{2}\right) \frac{1}{T_{h}}\dsum\limits_{t=p}^{T-h}\left( \dsum\limits_{i\in
H^{{\large c}}}g_{1,ik}^{2}u_{i,t}^{2}\right) \\
&=&2\dsum\limits_{k=1}^{Kp}\left[ \frac{1}{N_{1}}\dsum\limits_{i\in H^{%
{\large c}}}\left( \mathbb{I}\left\{ i\in \widehat{H^{c}}\right\} -2\mathbb{I%
}\left\{ i\in \widehat{H^{c}}\right\} +1\right) \right] \frac{1}{T_{h}}%
\dsum\limits_{t=p}^{T-h}\left( \dsum\limits_{i\in H^{{\large c}%
}}g_{1,ik}^{2}u_{i,t}^{2}\right) \\
&=&2\dsum\limits_{k=1}^{Kp}\left[ \frac{1}{N_{1}}\dsum\limits_{i\in H^{%
{\large c}}}\left( 1-\mathbb{I}\left\{ i\in \widehat{H^{c}}\right\} \right) %
\right] \frac{1}{T_{h}}\dsum\limits_{t=p}^{T-h}\left( \dsum\limits_{i\in H^{%
{\large c}}}g_{1,ik}^{2}u_{i,t}^{2}\right)
\end{eqnarray*}%
Now, for either the case where $\mathbb{S}_{i,T}^{+}=\dsum\nolimits_{\ell
=1}^{d}\varpi _{\ell }\left\vert S_{i,\ell ,T}\right\vert $ or the case
where $\mathbb{S}_{i,T}^{+}=\max_{1\leq \ell \leq d}\left\vert S_{i,\ell
,T}\right\vert $, we have, by applying Theorem 2 in Chao and Swanson (2022a),%
\begin{eqnarray*}
0 &\leq &E\left[ \frac{1}{N_{1}}\dsum\limits_{i\in H^{{\large c}}}\left( 1-%
\mathbb{I}\left\{ i\in \widehat{H^{c}}\right\} \right) \right] \\
&=&\frac{1}{N_{1}}\dsum\limits_{i\in H^{{\large c}}}\left[ 1-\Pr \left( i\in 
\widehat{H^{c}}\right) \right] \\
&=&\frac{1}{N_{1}}\dsum\limits_{i\in H^{{\large c}}}\left[ 1-P\left( \mathbb{%
S}_{i,T}^{+}\geq \Phi ^{-1}\left( 1-\frac{\varphi }{2N}\right) \right) %
\right] \\
&\leq &1-P\left( \min_{i\in H^{{\large c}}}\mathbb{S}_{i,T}^{+}\geq \Phi
^{-1}\left( 1-\frac{\varphi }{2N}\right) \right) \\
&&\left( \text{given that }N_{1}=\#\left\{ H^{c}\right\} \text{, where }%
\#\left\{ H^{c}\right\} \text{ denotes the cardinality of the set }%
H^{c}\right) \\
&\rightarrow &0\text{ }\left( \text{since }P\left( \min_{i\in H^{{\large c}}}%
\mathbb{S}_{i,T}^{+}\geq \Phi ^{-1}\left( 1-\frac{\varphi }{2N}\right)
\right) \rightarrow 1\right) \text{.}
\end{eqnarray*}%
Moreover, making use of part (b) of Assumption 3-3, we have 
\begin{eqnarray*}
E\left[ \frac{1}{T_{h}}\dsum\limits_{t=p}^{T-h}\dsum\limits_{i\in H^{{\large %
c}}}g_{1,ik}^{2}u_{i,t}^{2}\right] &=&\frac{1}{T_{h}}\dsum%
\limits_{t=p}^{T-h}\dsum\limits_{i\in H^{{\large c}}}g_{1,ik}^{2}E\left[
u_{i,t}^{2}\right] \\
&\leq &C\frac{T-h-p+1}{T_{h}}\dsum\limits_{i=1}^{N}g_{1,ik}^{2} \\
&\leq &C\text{ }\left( \text{since }\dsum\limits_{i=1}^{N}g_{1,ik}^{2}=1%
\text{ and }T_{h}=T-h-p+1\right)
\end{eqnarray*}%
It follows by Markov's inequality that%
\begin{equation*}
\frac{1}{N_{1}}\dsum\limits_{i\in H^{{\large c}}}\left( 1-\mathbb{I}\left\{
i\in \widehat{H^{c}}\right\} \right) =o_{p}\left( 1\right) \text{ and }\frac{%
1}{T_{h}}\dsum\limits_{t=p}^{T-h}\dsum\limits_{i\in H^{{\large c}%
}}g_{1,ik}^{2}u_{i,t}^{2}=O_{p}\left( 1\right)
\end{equation*}%
from which we deduce that%
\begin{eqnarray*}
\underline{\mathcal{E}}_{1,1} &=&\frac{2}{N_{1}T_{h}}\dsum%
\limits_{t=p}^{T-h}\dsum\limits_{k=1}^{Kp}\left( \dsum\limits_{i\in H^{%
{\large c}}}\left( \mathbb{I}\left\{ i\in \widehat{H^{c}}\right\} -1\right)
g_{1,ik}u_{i,t}\right) ^{2} \\
&\leq &2\dsum\limits_{k=1}^{Kp}\left[ \frac{1}{N_{1}}\dsum\limits_{i\in H^{%
{\large c}}}\left( 1-\mathbb{I}\left\{ i\in \widehat{H^{c}}\right\} \right) %
\right] \frac{1}{T_{h}}\dsum\limits_{t=p}^{T-h}\left( \dsum\limits_{i\in H^{%
{\large c}}}g_{1,ik}^{2}u_{i,t}^{2}\right) \\
&=&o_{p}\left( 1\right) \text{.}
\end{eqnarray*}

Next, consider the term $\underline{\mathcal{E}}_{1,2}$. To proceed, write%
\begin{eqnarray*}
&&\left\vert \underline{\mathcal{E}}_{1,2}\right\vert \\
&=&\left\vert \frac{2}{N_{1}T_{h}}\dsum\limits_{t=p}^{T-h}\dsum%
\limits_{k=1}^{Kp}\dsum\limits_{i\in H^{{\large c}}}\left( \mathbb{I}\left\{
i\in \widehat{H^{c}}\right\} -1\right) g_{1,ik}u_{i,t}\dsum\limits_{j\in H^{%
{\large c}}}g_{1,jk}u_{j,t}\right\vert \\
&=&\left\vert \frac{2}{N_{1}T_{h}}\dsum\limits_{t=p}^{T-h}\dsum%
\limits_{k=1}^{Kp}\dsum\limits_{j\in H^{{\large c}}}g_{1,jk}u_{j,t}\dsum%
\limits_{i\in H^{{\large c}}}\left( \mathbb{I}\left\{ i\in \widehat{H^{c}}%
\right\} -1\right) g_{1,ik}u_{i,t}\right\vert \\
&\leq &\frac{2}{N_{1}T_{h}}\dsum\limits_{t=p}^{T-h}\dsum\limits_{k=1}^{Kp}%
\left\vert \dsum\limits_{j\in H^{{\large c}}}g_{1,jk}u_{j,t}\right\vert
\left\vert \dsum\limits_{i\in H^{{\large c}}}\left( \mathbb{I}\left\{ i\in 
\widehat{H^{c}}\right\} -1\right) g_{1,ik}u_{i,t}\right\vert \\
&\leq &\frac{2}{N_{1}}\dsum\limits_{k=1}^{Kp}\sqrt{\dsum\limits_{i\in H^{%
{\large c}}}\left( \mathbb{I}\left\{ i\in \widehat{H^{c}}\right\} -1\right)
^{2}}\frac{1}{T_{h}}\dsum\limits_{t=p}^{T-h}\sqrt{\dsum\limits_{i\in H^{%
{\large c}}}g_{1,ik}^{2}u_{i,t}^{2}}\left\vert \dsum\limits_{j\in H^{{\large %
c}}}g_{1,jk}u_{j,t}\right\vert \\
&\leq &\frac{1}{N_{1}}\dsum\limits_{k=1}^{Kp}\sqrt{\dsum\limits_{i\in H^{%
{\large c}}}\left( \mathbb{I}\left\{ i\in \widehat{H^{c}}\right\} -1\right)
^{2}}\frac{1}{T_{h}}\dsum\limits_{t=p}^{T-h}\dsum\limits_{i\in H^{{\large c}%
}}g_{1,ik}^{2}u_{i,t}^{2} \\
&&+\frac{1}{N_{1}}\dsum\limits_{k=1}^{Kp}\sqrt{\dsum\limits_{i\in H^{{\large %
c}}}\left( \mathbb{I}\left\{ i\in \widehat{H^{c}}\right\} -1\right) ^{2}}%
\frac{1}{T_{h}}\dsum\limits_{t=p}^{T-h}\left( \dsum\limits_{j\in H^{{\large c%
}}}g_{1,jk}u_{j,t}\right) ^{2} \\
&&\left( \text{by the inequality }\left\vert XY\right\vert \leq \frac{1}{2}%
X^{2}+\frac{1}{2}Y^{2}\right) \\
&=&\frac{1}{\sqrt{N_{1}}}\dsum\limits_{k=1}^{Kp}\sqrt{\frac{1}{N_{1}}%
\dsum\limits_{i\in H^{{\large c}}}\left( 1-\mathbb{I}\left\{ i\in \widehat{%
H^{c}}\right\} \right) }\frac{1}{T_{h}}\dsum\limits_{t=p}^{T-h}\dsum%
\limits_{i\in H^{{\large c}}}g_{1,ik}^{2}u_{i,t}^{2} \\
&&+\sqrt{\frac{1}{N_{1}}\dsum\limits_{i\in H^{{\large c}}}\left( 1-\mathbb{I}%
\left\{ i\in \widehat{H^{c}}\right\} \right) }\frac{1}{T_{h}}%
\dsum\limits_{t=p}^{T-h}\frac{1}{\sqrt{N_{1}}}\dsum\limits_{k=1}^{Kp}\left(
\dsum\limits_{j\in H^{{\large c}}}g_{1,jk}u_{j,t}\right) ^{2}
\end{eqnarray*}%
Observe that%
\begin{eqnarray*}
&&E\left[ \frac{1}{T_{h}}\dsum\limits_{t=p}^{T-h}\frac{1}{\sqrt{N_{1}}}%
\dsum\limits_{k=1}^{Kp}\left( \dsum\limits_{j\in H^{{\large c}%
}}g_{1,jk}u_{j,t}\right) ^{2}\right] \\
&=&\frac{1}{T_{h}}\dsum\limits_{t=p}^{T-h}\frac{1}{\sqrt{N_{1}}}%
\dsum\limits_{k=1}^{Kp}\dsum\limits_{j\in H^{{\large c}}}\dsum\limits_{\ell
\in H^{{\large c}}}g_{1,jk}g_{1,\ell k}E\left[ u_{j,t}u_{\ell ,t}\right] \\
&=&\frac{1}{T_{h}}\dsum\limits_{t=p}^{T-h}\frac{1}{\sqrt{N_{1}}}%
\dsum\limits_{j\in H^{{\large c}}}\dsum\limits_{\ell \in H^{{\large c}%
}}\dsum\limits_{k=1}^{Kp}g_{1,jk}g_{1,\ell k}E\left[ u_{j,t}u_{\ell ,t}%
\right] \\
&=&\frac{1}{T_{h}}\dsum\limits_{t=p}^{T-h}\frac{1}{\sqrt{N_{1}}}%
\dsum\limits_{j\in H^{{\large c}}}\dsum\limits_{\ell \in H^{{\large c}}}%
\frac{e_{j,N}^{\prime }\Gamma _{\ast }}{\sqrt{N_{1}}}\dsum\limits_{k=1}^{Kp}%
\Xi e_{k,Kp}e_{k,Kp}^{\prime }\Xi ^{\prime }\frac{\Gamma _{\ast }^{\prime
}e_{\ell ,N}}{\sqrt{N_{1}}}E\left[ u_{j,t}u_{\ell ,t}\right] \\
&&\left( \text{since }G_{1}=\frac{\Gamma _{\ast }\Xi }{\sqrt{N_{1}}}\text{
with }\Gamma _{\ast }=\Gamma \left( \frac{\Gamma ^{\prime }\Gamma }{N_{1}}%
\right) ^{-1/2}\right) \\
&=&\frac{1}{T_{h}}\dsum\limits_{t=p}^{T-h}\frac{1}{N_{1}^{\frac{{\Large 3}}{%
{\Large 2}}}}\dsum\limits_{j\in H^{{\large c}}}\dsum\limits_{\ell \in H^{%
{\large c}}}e_{j,N}^{\prime }\Gamma _{\ast }\Xi \Xi ^{\prime }\Gamma _{\ast
}^{\prime }e_{\ell ,N}E\left[ u_{j,t}u_{\ell ,t}\right] \\
&=&\frac{1}{T_{h}}\dsum\limits_{t=p}^{T-h}\frac{1}{N_{1}^{\frac{{\Large 3}}{%
{\Large 2}}}}\dsum\limits_{j\in H^{{\large c}}}\dsum\limits_{\ell \in H^{%
{\large c}}}e_{j,N}^{\prime }\Gamma _{\ast }\Gamma _{\ast }^{\prime }e_{\ell
,N}E\left[ u_{j,t}u_{\ell ,t}\right] \\
&&\left( \text{since }\Xi \text{ is an orthogonal matrix}\right) \\
&=&\frac{1}{T_{h}}\dsum\limits_{t=p}^{T-h}\frac{1}{N_{1}^{\frac{{\Large 3}}{%
{\Large 2}}}}\dsum\limits_{j\in H^{{\large c}}}\dsum\limits_{\ell \in H^{%
{\large c}}}\gamma _{\ast ,j}^{\prime }\gamma _{\ast ,\ell }E\left[
u_{j,t}u_{\ell ,t}\right]
\end{eqnarray*}%
where $\gamma _{\ast ,j}=\left( \Gamma ^{\prime }\Gamma /N_{1}\right)
^{-1/2}\gamma _{j}$. Applying the triangle and Cauchy-Schwarz inequalities,
we further obtain%
\begin{eqnarray*}
&&E\left[ \frac{1}{T_{h}}\dsum\limits_{t=p}^{T-h}\frac{1}{\sqrt{N_{1}}}%
\dsum\limits_{k=1}^{Kp}\left( \dsum\limits_{j\in H^{{\large c}%
}}g_{1,jk}u_{j,t}\right) ^{2}\right] \\
&=&\frac{1}{T_{h}}\dsum\limits_{t=p}^{T-h}\frac{1}{N_{1}^{\frac{{\Large 3}}{%
{\Large 2}}}}\dsum\limits_{j\in H^{{\large c}}}\dsum\limits_{\ell \in H^{%
{\large c}}}\gamma _{\ast ,j}^{\prime }\gamma _{\ast ,\ell }E\left[
u_{j,t}u_{\ell ,t}\right] \\
&=&\frac{1}{T_{h}}\dsum\limits_{t=p}^{T-h}\frac{1}{N_{1}^{\frac{{\Large 3}}{%
{\Large 2}}}}\dsum\limits_{j\in H^{{\large c}}}\dsum\limits_{\ell \in H^{%
{\large c}}}\gamma _{j}^{\prime }\left( \frac{\Gamma ^{\prime }\Gamma }{N_{1}%
}\right) ^{-1/2}\left( \frac{\Gamma ^{\prime }\Gamma }{N_{1}}\right)
^{-1/2}\gamma _{\ell }E\left[ u_{j,t}u_{\ell ,t}\right] \\
&\leq &\frac{1}{T_{h}}\dsum\limits_{t=p}^{T-h}\frac{1}{N_{1}^{\frac{{\Large 3%
}}{{\Large 2}}}}\dsum\limits_{j\in H^{{\large c}}}\dsum\limits_{\ell \in H^{%
{\large c}}}\left\vert \gamma _{j}^{\prime }\left( \frac{\Gamma ^{\prime
}\Gamma }{N_{1}}\right) ^{-1/2}\left( \frac{\Gamma ^{\prime }\Gamma }{N_{1}}%
\right) ^{-1/2}\gamma _{\ell }\right\vert \left\vert E\left[ u_{j,t}u_{\ell
,t}\right] \right\vert \\
&\leq &\frac{1}{T_{h}}\dsum\limits_{t=p}^{T-h}\frac{1}{N_{1}^{\frac{{\Large 3%
}}{{\Large 2}}}}\dsum\limits_{j\in H^{{\large c}}}\dsum\limits_{\ell \in H^{%
{\large c}}}\sqrt{\gamma _{j}^{\prime }\left( \frac{\Gamma ^{\prime }\Gamma 
}{N_{1}}\right) ^{-1}\gamma _{j}}\sqrt{\gamma _{\ell }^{\prime }\left( \frac{%
\Gamma ^{\prime }\Gamma }{N_{1}}\right) ^{-1}\gamma _{\ell }}\left\vert E%
\left[ u_{j,t}u_{\rightarrow ,t}\right] \right\vert \\
&\leq &\frac{\overline{c}}{\underline{C}}\frac{1}{\sqrt{N_{1}}T_{h}}%
\dsum\limits_{t=p}^{T-h}\frac{1}{N_{1}}\dsum\limits_{j\in H^{{\large c}%
}}\dsum\limits_{\ell \in H^{{\large c}}}\left\vert E\left[ u_{j,t}u_{\ell ,t}%
\right] \right\vert \\
&&\left( \text{since, under Assumptions 3-5 and 3-6, there exist positive
constants }\overline{c}\text{ and }\underline{C}\text{ such that }\right. \\
&&\left. \sup_{i\in H^{{\large c}}}\left\Vert \gamma _{i}\right\Vert
_{2}\leq \overline{c}<\infty \text{ and }\lambda _{\min }\left( \frac{\Gamma
^{\prime }\Gamma }{N_{1}}\right) \geq \underline{C}>0\right) \\
&\leq &\frac{\overline{c}}{\underline{C}}\frac{\overline{C}}{\sqrt{N}_{1}}%
\text{ }\rightarrow 0\text{ as }N_{1}\rightarrow \infty \text{.}\left( \text{%
since, under Assumption 3-3(d), there exists a }\right. \text{ } \\
&&\text{ \ \ \ \ \ \ \ \ \ \ \ }\left. \text{positive constant }\overline{C}%
\text{ such that }\sup_{t}\frac{1}{N_{1}}\dsum\limits_{j\in H^{{\large c}%
}}\dsum\limits_{\ell \in H^{{\large c}}}\left\vert E\left[ u_{j,t}u_{\ell ,t}%
\right] \right\vert \leq \overline{C}<\infty \right)
\end{eqnarray*}%
from which we further deduce, upon applying Markov's inequality, that%
\begin{equation*}
\frac{1}{T_{h}}\dsum\limits_{t=p}^{T-h}\frac{1}{\sqrt{N_{1}}}%
\dsum\limits_{k=1}^{Kp}\left( \dsum\limits_{j\in H^{{\large c}%
}}g_{1,jk}u_{j,t}\right) ^{2}=o_{p}\left( 1\right)
\end{equation*}%
Moreover, since we have previously shown that%
\begin{equation*}
\frac{1}{N_{1}}\dsum\limits_{i\in H^{{\large c}}}\left( 1-\mathbb{I}\left\{
i\in \widehat{H^{c}}\right\} \right) =o_{p}\left( 1\right) \text{ and }\frac{%
1}{T_{h}}\dsum\limits_{t=p}^{T-h}\dsum\limits_{i\in H^{{\large c}%
}}g_{1,ik}^{2}u_{i,t}^{2}=O_{p}\left( 1\right) \text{,}
\end{equation*}%
it follows from these calculations that%
\begin{eqnarray*}
\left\vert \underline{\mathcal{E}}_{1,2}\right\vert &\leq &\frac{1}{\sqrt{%
N_{1}}}\dsum\limits_{k=1}^{Kp}\sqrt{\frac{1}{N_{1}}\dsum\limits_{i\in H^{%
{\large c}}}\left( 1-\mathbb{I}\left\{ i\in \widehat{H^{c}}\right\} \right) }%
\frac{1}{T_{h}}\dsum\limits_{t=p}^{T-h}\dsum\limits_{i\in H^{{\large c}%
}}g_{1,ik}^{2}u_{i,t}^{2} \\
&&+\sqrt{\frac{1}{N_{1}}\dsum\limits_{i\in H^{{\large c}}}\left( 1-\mathbb{I}%
\left\{ i\in \widehat{H^{c}}\right\} \right) }\frac{1}{T_{h}}%
\dsum\limits_{t=p}^{T-h}\frac{1}{\sqrt{N_{1}}}\dsum\limits_{k=1}^{Kp}\left(
\dsum\limits_{j\in H^{{\large c}}}g_{1,jk}u_{j,t}\right) ^{2} \\
&=&o_{p}\left( 1\right) \text{.}
\end{eqnarray*}

In a similar way, we can also show that%
\begin{equation*}
\left\vert \underline{\mathcal{E}}_{1,3}\right\vert =o_{p}\left( 1\right) 
\text{.}
\end{equation*}

Finally, let $U_{t,N}\left( H^{c}\right) $ denote an $N\times 1$ vector
whose $i^{th}$ component $U_{i,t,N}\left( H^{c}\right) $ is given by%
\begin{equation*}
U_{i,t,N}\left( H^{c}\right) =\left\{ 
\begin{array}{cc}
u_{i,t} & \text{if }i\in H^{c} \\ 
0 & \text{if }i\in H%
\end{array}%
\right. .
\end{equation*}%
and we can write%
\begin{eqnarray*}
\underline{\mathcal{E}}_{1,4} &=&\frac{2}{N_{1}T_{h}}\dsum%
\limits_{t=p}^{T-h}\dsum\limits_{k=1}^{Kp}\dsum\limits_{i\in H^{{\large c}%
}}\dsum\limits_{j\in H^{{\large c}}}g_{1,ik}g_{1,jk}u_{i,t}u_{j,t} \\
&=&\frac{2}{T_{h}}\dsum\limits_{t=p}^{T-h}\left\Vert \frac{G_{1}^{\prime
}U_{t,N}\left( H^{c}\right) }{\sqrt{N_{1}}}\right\Vert _{2}^{2} \\
&=&\frac{2}{T_{h}}\dsum\limits_{t=p}^{T-h}tr\left\{ \frac{G_{1}^{\prime
}U_{t,N}\left( H^{c}\right) U_{t,N}\left( H^{c}\right) ^{\prime }G_{1}}{N_{1}%
}\right\} \\
&=&\frac{2}{T_{h}}\dsum\limits_{t=p}^{T-h}tr\left\{ \Xi ^{\prime }\left( 
\frac{\Gamma ^{\prime }\Gamma }{N_{1}}\right) ^{-1/2}\frac{\Gamma ^{\prime }%
}{\sqrt{N_{1}}}\frac{U_{t,N}\left( H^{c}\right) U_{t,N}\left( H^{c}\right)
^{\prime }}{N_{1}}\frac{\Gamma }{\sqrt{N_{1}}}\left( \frac{\Gamma ^{\prime
}\Gamma }{N_{1}}\right) ^{-1/2}\Xi \right\} \\
&=&\frac{2}{T_{h}}\dsum\limits_{t=p}^{T-h}tr\left\{ \left( \frac{\Gamma
^{\prime }\Gamma }{N_{1}}\right) ^{-1/2}\frac{\Gamma ^{\prime }}{\sqrt{N_{1}}%
}\frac{U_{t,N}\left( H^{c}\right) U_{t,N}\left( H^{c}\right) ^{\prime }}{%
N_{1}}\frac{\Gamma }{\sqrt{N_{1}}}\left( \frac{\Gamma ^{\prime }\Gamma }{%
N_{1}}\right) ^{-1/2}\right\} \\
&=&\frac{2}{T_{h}}\dsum\limits_{t=p}^{T-h}tr\left\{ \frac{\Gamma _{\ast
}^{\prime }U_{t,N}\left( H^{c}\right) U_{t,N}\left( H^{c}\right) ^{\prime
}\Gamma _{\ast }}{N_{1}^{2}}\right\} \\
&=&\frac{2}{T_{h}N_{1}^{2}}\dsum\limits_{t=p}^{T-h}U_{t,N}\left(
H^{c}\right) ^{\prime }\Gamma _{\ast }\Gamma _{\ast }^{\prime }U_{t,N}\left(
H^{c}\right) \\
&=&\frac{2}{T_{h}N_{1}^{2}}\dsum\limits_{t=p}^{T-h}\dsum\limits_{i\in H^{%
{\large c}}}\dsum\limits_{j\in H^{{\large c}}}\gamma _{\ast ,i}^{\prime
}\gamma _{\ast ,j}u_{i,t}u_{j,t}
\end{eqnarray*}%
where $\gamma _{\ast ,i}^{\prime }$ denotes the $i^{th}$ row of $\Gamma
_{\ast }=\Gamma \left( \Gamma ^{\prime }\Gamma /N_{1}\right) ^{-1/2}$.
Taking expectation, we then obtain%
\begin{eqnarray*}
0 &\leq &E\left[ \underline{\mathcal{E}}_{1,4}\right] \\
&=&\frac{2}{N_{1}T_{h}}\dsum\limits_{t=p}^{T-h}\dsum\limits_{k=1}^{Kp}\dsum%
\limits_{i\in H^{{\large c}}}\dsum\limits_{j\in H^{{\large c}%
}}g_{1,ik}g_{1,jk}E\left[ u_{i,t}u_{j,t}\right] \\
&=&\frac{2}{T_{h}N_{1}^{2}}\dsum\limits_{t=p}^{T-h}\dsum\limits_{i\in H^{%
{\large c}}}\dsum\limits_{j\in H^{{\large c}}}\gamma _{\ast ,i}^{\prime
}\gamma _{\ast ,j}E\left[ u_{i,t}u_{j,t}\right] \\
&=&\frac{2}{T_{h}N_{1}^{2}}\dsum\limits_{t=p}^{T-h}\dsum\limits_{i\in H^{%
{\large c}}}\dsum\limits_{j\in H^{{\large c}}}\gamma _{i}^{\prime }\left( 
\frac{\Gamma ^{\prime }\Gamma }{N_{1}}\right) ^{-1/2}\left( \frac{\Gamma
^{\prime }\Gamma }{N_{1}}\right) ^{-1/2}\gamma _{j}E\left[ u_{i,t}u_{j,t}%
\right] \\
&\leq &\frac{2}{T_{h}N_{1}^{2}}\dsum\limits_{t=p}^{T-h}\dsum\limits_{i\in H^{%
{\large c}}}\dsum\limits_{j\in H^{{\large c}}}\left\vert \gamma _{i}^{\prime
}\left( \frac{\Gamma ^{\prime }\Gamma }{N_{1}}\right) ^{-1/2}\left( \frac{%
\Gamma ^{\prime }\Gamma }{N_{1}}\right) ^{-1/2}\gamma _{j}\right\vert
\left\vert E\left[ u_{i,t}u_{j,t}\right] \right\vert \\
&\leq &\frac{2}{T_{h}N_{1}^{2}}\dsum\limits_{t=p}^{T-h}\dsum\limits_{i\in H^{%
{\large c}}}\dsum\limits_{j\in H^{{\large c}}}\sqrt{\gamma _{i}^{\prime
}\left( \frac{\Gamma ^{\prime }\Gamma }{N_{1}}\right) ^{-1}\gamma _{i}}\sqrt{%
\gamma _{i}^{\prime }\left( \frac{\Gamma ^{\prime }\Gamma }{N_{1}}\right)
^{-1}\gamma _{i}}\left\vert E\left[ u_{i,t}u_{j,t}\right] \right\vert \\
&\leq &\frac{2\overline{c}}{\underline{C}}\frac{1}{T_{h}N_{1}^{2}}%
\dsum\limits_{t=p}^{T-h}\dsum\limits_{i\in H^{{\large c}}}\dsum\limits_{j\in
H^{{\large c}}}\left\vert E\left[ u_{i,t}u_{j,t}\right] \right\vert \\
&&\left( \text{since, under Assumptions 3-5 and 3-6, there exist positive
constants }\overline{c}\text{ and }\underline{C}\text{ such that }\right. \\
&&\left. \sup_{i\in H^{{\large c}}}\left\Vert \gamma _{i}\right\Vert
_{2}\leq \overline{c}<\infty \text{ and }\lambda _{\min }\left( \frac{\Gamma
^{\prime }\Gamma }{N_{1}}\right) \geq \underline{C}>0\right) \\
&\leq &\frac{2\overline{c}}{\underline{C}}\frac{\overline{C}}{N_{1}}\frac{%
T-h-p+1}{T_{h}}\text{ }=\frac{2\overline{c}}{\underline{C}}\frac{\overline{C}%
}{N_{1}}\rightarrow 0\text{ as }N_{1},T\rightarrow \infty \text{.} \\
&&\left( \text{since, under Assumption 3-3(d), there exist a positive
constant }\overline{C}\right. \text{ } \\
&&\text{ \ \ }\left. \text{such that }\sup_{t}\frac{1}{N_{1}}%
\dsum\limits_{i\in H^{{\large c}}}\dsum\limits_{j\in H^{{\large c}%
}}\left\vert E\left[ u_{i,t}u_{j,t}\right] \right\vert \leq \overline{C}%
<\infty \right)
\end{eqnarray*}%
It follows by Markov's inequality that%
\begin{equation*}
\underline{\mathcal{E}}_{1,4}=o_{p}\left( 1\right) \text{. }
\end{equation*}%
Application of the Slutsky's theorem then allows us to deduce that%
\begin{eqnarray*}
\frac{2}{T_{h}N_{1}}\dsum\limits_{t=p}^{T-h}\dsum\limits_{k=1}^{Kp}\dsum%
\limits_{i\in H^{{\large c}}}\dsum\limits_{j\in H^{{\large c}}}\mathbb{I}%
\left\{ i\in \widehat{H^{c}}\right\} \mathbb{I}\left\{ j\in \widehat{H^{c}}%
\right\} g_{1,ik}g_{1,jk}u_{i,t}u_{j,t} &=&\underline{\mathcal{E}}_{1,1}+%
\underline{\mathcal{E}}_{1,2}+\underline{\mathcal{E}}_{1,3}+\underline{%
\mathcal{E}}_{1,4} \\
&=&o_{p}\left( 1\right) +o_{p}\left( 1\right) +o_{p}\left( 1\right)
+o_{p}\left( 1\right) \\
&=&o_{p}\left( 1\right) \text{.}
\end{eqnarray*}

Consider now the second term on the extreme right-hand side of expression (%
\ref{G1U term}) 
\begin{eqnarray*}
&&\frac{2}{T_{h}N_{1}}\dsum\limits_{t=p}^{T-h}\dsum\limits_{k=1}^{Kp}\dsum%
\limits_{i\in H}\dsum\limits_{j\in H}\mathbb{I}\left\{ i\in \widehat{H^{c}}%
\right\} \mathbb{I}\left\{ j\in \widehat{H^{c}}\right\}
g_{1,ik}g_{1,jk}u_{i,t}u_{j,t} \\
&=&\frac{2}{T_{h}N_{1}}\dsum\limits_{t=p}^{T-h}\dsum\limits_{k=1}^{Kp}\left(
\dsum\limits_{i\in H}\mathbb{I}\left\{ i\in \widehat{H^{c}}\right\}
g_{1,ik}u_{i,t}\right) ^{2} \\
&=&\frac{2}{T_{h}N_{1}}\dsum\limits_{t=p}^{T-h}\dsum\limits_{k=1}^{Kp}\left(
\left\vert \dsum\limits_{i\in H}\mathbb{I}\left\{ i\in \widehat{H^{c}}%
\right\} g_{1,ik}u_{i,t}\right\vert \right) ^{2} \\
&\leq &2\dsum\limits_{k=1}^{Kp}\left[ \frac{1}{N_{1}}\dsum\limits_{i\in H}%
\mathbb{I}\left\{ i\in \widehat{H^{c}}\right\} \right] \left[ \frac{1}{T_{h}}%
\dsum\limits_{t=p}^{T-h}\dsum\limits_{i\in H}g_{1,ik}^{2}u_{i,t}^{2}\right]
\end{eqnarray*}%
Note that, for either the case where $\mathbb{S}_{i,T}^{+}=\dsum\nolimits_{%
\ell =1}^{d}\varpi _{\ell }\left\vert S_{i,\ell ,T}\right\vert $ or the case
where $\mathbb{S}_{i,T}^{+}=\max_{1\leq \ell \leq d}\left\vert S_{i,\ell
,T}\right\vert $, we have 
\begin{eqnarray*}
0 &\leq &E\left[ \frac{1}{N_{1}}\dsum\limits_{i\in H}\mathbb{I}\left\{ i\in 
\widehat{H^{c}}\right\} \right] \\
&=&\frac{1}{N_{1}}\dsum\limits_{i\in H}\Pr \left( i\in \widehat{H^{c}}\right)
\\
&=&\frac{1}{N_{1}}\dsum\limits_{i\in H}P\left( \mathbb{S}_{i,T}^{+}\geq \Phi
^{-1}\left( 1-\frac{\varphi }{2N}\right) \right) \\
&\leq &\frac{N_{2}\varphi }{N_{1}N}\left\{ 1+2^{1+\delta }AT_{0}^{-\left(
1-\alpha _{{\large 1}}\right) \frac{{\large \delta }}{{\large 2}}%
}+2^{1+\delta }A\Phi ^{-1}\left( 1-\frac{\varphi }{2N}\right) ^{2+\delta
}T_{0}^{-\left( 1-\alpha _{{\large 1}}\right) \frac{{\large \delta }}{%
{\large 2}}}\right\} \\
&=&\frac{N_{2}\varphi }{N_{1}N}\left[ 1+o\left( 1\right) \right] \\
&&\left( \text{following an argument similar to that given in the proof of
Theorem 1 in Chao and Swanson (2022a)}\right) \\
&\rightarrow &0\text{ as }N_{1},N_{2},T\rightarrow \infty
\end{eqnarray*}%
Moreover, making use of part (b) of Assumption 3-3, we have%
\begin{eqnarray*}
E\left[ \frac{1}{T_{h}}\dsum\limits_{t=p}^{T-h}\dsum\limits_{i\in
H}g_{1,ik}^{2}u_{i,t}^{2}\right] &=&\frac{1}{T_{h}}\dsum\limits_{t=p}^{T-h}%
\dsum\limits_{i\in H}g_{1,ik}^{2}E\left[ u_{i,t}^{2}\right] \\
&\leq &C\frac{1}{T_{h}}\dsum\limits_{t=p}^{T-h}\dsum%
\limits_{i=1}^{N}g_{1,ik}^{2} \\
&\leq &C\frac{T-h-p+1}{T_{h}}\text{ \ } \\
&&\left( \text{given that }\dsum\limits_{i=1}^{N}g_{1,ik}^{2}=1\text{ and }%
T_{h}=T-h-p+1\right) \\
&\leq &C<\infty
\end{eqnarray*}%
It follows by Markov's inequality that%
\begin{equation*}
\frac{1}{N_{1}}\dsum\limits_{i\in H}\mathbb{I}\left\{ i\in \widehat{H^{c}}%
\right\} =o_{p}\left( 1\right) \text{ and }\frac{1}{T_{h}}%
\dsum\limits_{t=p}^{T-h}\dsum\limits_{i\in
H}g_{1,ik}^{2}u_{i,t}^{2}=O_{p}\left( 1\right)
\end{equation*}%
from which we deduce that%
\begin{eqnarray*}
&&\frac{2}{T_{h}N_{1}}\dsum\limits_{t=p}^{T-h}\dsum\limits_{k=1}^{Kp}\dsum%
\limits_{i\in H}\dsum\limits_{i\in H}\mathbb{I}\left\{ i\in \widehat{H^{c}}%
\right\} \mathbb{I}\left\{ j\in \widehat{H^{c}}\right\}
g_{1,ik}g_{1,jk}u_{i,t}u_{j,t} \\
&\leq &2\dsum\limits_{k=1}^{Kp}\left[ \frac{1}{N_{1}}\dsum\limits_{i\in H}%
\mathbb{I}\left\{ i\in \widehat{H^{c}}\right\} \right] \left[ \frac{1}{T_{h}}%
\dsum\limits_{t=p}^{T-h}\dsum\limits_{i\in H}g_{1,ik}^{2}u_{i,t}^{2}\right]
\\
&=&o_{p}\left( 1\right)
\end{eqnarray*}

Combining these results, we further obtain%
\begin{eqnarray*}
\frac{1}{T_{h}}\dsum\limits_{t=p}^{T-h}\left\Vert \frac{G_{1}^{\prime
}U_{t,N}\left( \widehat{H^{c}}\right) }{\sqrt{N_{1}}}\right\Vert _{2}^{2}
&\leq &\frac{2}{T_{h}N_{1}}\dsum\limits_{t=p}^{T-h}\dsum\limits_{k=1}^{Kp}%
\dsum\limits_{i\in H^{{\large c}}}\dsum\limits_{j\in H^{{\large c}}}\mathbb{I%
}\left\{ i\in \widehat{H^{c}}\right\} \mathbb{I}\left\{ j\in \widehat{H^{c}}%
\right\} g_{1,ik}g_{1,jk}u_{i,t}u_{j,t} \\
&&+\frac{2}{T_{h}N_{1}}\dsum\limits_{t=p}^{T-h}\dsum\limits_{k=1}^{Kp}\dsum%
\limits_{i\in H}\dsum\limits_{j\in H}\mathbb{I}\left\{ i\in \widehat{H^{c}}%
\right\} \mathbb{I}\left\{ j\in \widehat{H^{c}}\right\}
g_{1,ik}g_{1,jk}u_{i,t}u_{j,t} \\
&=&o_{p}\left( 1\right) +o_{p}\left( 1\right) \\
&=&o_{p}\left( 1\right) \text{. }
\end{eqnarray*}

To show part (b), write 
\begin{eqnarray*}
\frac{1}{T_{h}}\dsum\limits_{t=p}^{T-h}\left\Vert \frac{U_{t,N}\left( 
\widehat{H^{c}}\right) }{\sqrt{N_{1}}}\right\Vert _{2}^{2} &=&\frac{1}{T_{h}}%
\dsum\limits_{t=p}^{T-h}\frac{U_{t,N}\left( \widehat{H^{c}}\right) ^{\prime
}U_{t,N}\left( \widehat{H^{c}}\right) }{N_{1}} \\
&=&\frac{1}{T_{h}N_{1}}\dsum\limits_{t=p}^{T-h}\dsum\limits_{i=1}^{N}\mathbb{%
I}\left\{ i\in \widehat{H^{c}}\right\} u_{i,t}^{2} \\
&=&\frac{1}{T_{h}N_{1}}\dsum\limits_{t=p}^{T-h}\dsum\limits_{i\in H^{{\large %
c}}}\mathbb{I}\left\{ i\in \widehat{H^{c}}\right\} u_{i,t}^{2}+\frac{1}{%
T_{h}N_{1}}\dsum\limits_{t=p}^{T-h}\dsum\limits_{i\in H}\mathbb{I}\left\{
i\in \widehat{H^{c}}\right\} u_{i,t}^{2} \\
&\leq &\frac{1}{T_{h}N_{1}}\dsum\limits_{t=p}^{T-h}\dsum\limits_{i\in H^{%
{\large c}}}u_{i,t}^{2}+\frac{1}{N_{1}}\dsum\limits_{i\in H}\mathbb{I}%
\left\{ i\in \widehat{H^{c}}\right\} \frac{1}{T_{h}}\dsum%
\limits_{t=p}^{T-h}u_{i,t}^{2}
\end{eqnarray*}%
Next, note that, by making use of part (b) of Assumption 3-3, we have%
\begin{eqnarray*}
E\left[ \frac{1}{T_{h}N_{1}}\dsum\limits_{t=p}^{T-h}\dsum\limits_{i\in H^{%
{\large c}}}u_{i,t}^{2}\right] &=&\frac{1}{T_{h}N_{1}}\dsum%
\limits_{t=p}^{T-h}\dsum\limits_{i\in H^{{\large c}}}E\left[ u_{i,t}^{2}%
\right] \\
&\leq &C\frac{T-h-p+1}{T_{h}}\text{ }\left( \text{since }N_{1}=\#\left\{
H\right\} \text{,}\right. \text{ } \\
&&\left. \text{where }\#\left\{ H\right\} \text{ denotes the cardinality of
the set }H\right) \text{ } \\
&\leq &C\text{ }\left( \text{since }T_{h}=T-h-p+1\right)
\end{eqnarray*}%
so that, by Markov's inequality,%
\begin{equation*}
\frac{1}{T_{h}N_{1}}\dsum\limits_{t=p}^{T-h}\dsum\limits_{i\in H^{{\large c}%
}}u_{i,t}^{2}=O_{p}\left( 1\right) \text{.}
\end{equation*}%
Moreover, note that, for any $\epsilon >0$,%
\begin{equation*}
\dbigcap\limits_{i\in H}\left\{ i\notin \widehat{H^{c}}\right\} \subseteq
\left\{ \frac{1}{N_{1}}\dsum\limits_{i\in H}\mathbb{I}\left\{ i\in \widehat{%
H^{c}}\right\} \frac{1}{T_{h}}\dsum\limits_{t=p}^{T-h}u_{i,t}^{2}<\epsilon
\right\}
\end{equation*}%
so that, applying DeMorgan's law, we obtain 
\begin{equation*}
\left\{ \frac{1}{N_{1}}\dsum\limits_{i\in H}\mathbb{I}\left\{ i\in \widehat{%
H^{c}}\right\} \frac{1}{T_{h}}\dsum\limits_{t=p}^{T-h}u_{i,t}^{2}\geq
\epsilon \right\} \subseteq \left\{ \dbigcap\limits_{i\in H}\left\{ i\notin 
\widehat{H^{c}}\right\} \right\} ^{c}=\dbigcup\limits_{i\in H}\left\{ i\in 
\widehat{H^{c}}\right\} \text{ }
\end{equation*}%
It follows that, for any $\epsilon >0$ and for either the case where $%
\mathbb{S}_{i,T}^{+}=\dsum\nolimits_{\ell =1}^{d}\varpi _{\ell }\left\vert
S_{i,\ell ,T}\right\vert $ or the case where $\mathbb{S}_{i,T}^{+}=\max_{1%
\leq \ell \leq d}\left\vert S_{i,\ell ,T}\right\vert $, we have%
\begin{eqnarray*}
&&\Pr \left\{ \frac{1}{N_{1}}\dsum\limits_{i\in H}\mathbb{I}\left\{ i\in 
\widehat{H^{c}}\right\} \frac{1}{T_{h}}\dsum\limits_{t=p}^{T-h}u_{i,t}^{2}%
\geq \epsilon \right\} \\
&\leq &\Pr \left\{ \dbigcup\limits_{i\in H}\left\{ i\in \widehat{H^{c}}%
\right\} \right\} \\
&\leq &\dsum\limits_{i\in H}\Pr \left\{ i\in \widehat{H^{c}}\right\} \\
&=&\dsum\limits_{i\in H}P\left( \mathbb{S}_{i,T}^{+}\geq \Phi ^{-1}\left( 1-%
\frac{\varphi }{2N}\right) \right) \\
&\leq &\frac{N_{2}\varphi }{N}\left\{ 1+2^{1+\delta }AT_{0}^{-\left(
1-\alpha _{{\large 1}}\right) \frac{{\large \delta }}{{\large 2}}%
}+2^{1+\delta }A\Phi ^{-1}\left( 1-\frac{\varphi }{2N}\right) ^{2+\delta
}T_{0}^{-\left( 1-\alpha _{{\large 1}}\right) \frac{{\large \delta }}{%
{\large 2}}}\right\} \\
&=&\frac{N_{2}\varphi }{N}\left[ 1+o\left( 1\right) \right] \\
&&\left( \text{following an argument similar to that given in the proof of
Theorem 1 in Chao and Swanson (2022a)}\right) \\
&\rightarrow &0\text{ as }N_{1},N_{2},T\rightarrow \infty
\end{eqnarray*}%
Hence,%
\begin{equation*}
\frac{1}{N_{1}}\dsum\limits_{i\in H}\mathbb{I}\left\{ i\in \widehat{H^{c}}%
\right\} \frac{1}{T_{h}}\dsum\limits_{t=p}^{T-h}u_{i,t}^{2}=o_{p}\left(
1\right)
\end{equation*}%
from which it we further deduce that%
\begin{eqnarray*}
\frac{1}{T_{h}}\dsum\limits_{t=p}^{T-h}\left\Vert \frac{U_{t,N}\left( 
\widehat{H^{c}}\right) }{\sqrt{N_{1}}}\right\Vert _{2}^{2} &\leq &\frac{1}{%
T_{h}N_{1}}\dsum\limits_{t=p}^{T-h}\dsum\limits_{i\in H^{{\large c}%
}}u_{i,t}^{2}+\frac{1}{N_{1}}\dsum\limits_{i\in H}\mathbb{I}\left\{ i\in 
\widehat{H^{c}}\right\} \frac{1}{T_{h}}\dsum\limits_{t=p}^{T-h}u_{i,t}^{2} \\
&=&O_{p}\left( 1\right) +o_{p}\left( 1\right) \\
&=&O_{p}\left( 1\right) \text{.}
\end{eqnarray*}

Now, for part (c), note first that since $G=\left[ 
\begin{array}{cc}
G_{1} & G_{2}%
\end{array}%
\right] $ is an orthogonal matrix, we have $I_{N}=GG^{\prime
}=G_{1}G_{1}^{\prime }+G_{2}G_{2}^{\prime }$ or $G_{2}G_{2}^{\prime
}=I_{N}-G_{1}G_{1}^{\prime }$. Hence, we can write%
\begin{eqnarray*}
\frac{1}{T_{h}}\dsum\limits_{t=p}^{T-h}\left\Vert \frac{G_{2}^{\prime
}U_{t,N}\left( \widehat{H^{c}}\right) }{\sqrt{N_{1}}}\right\Vert _{2}^{2} &=&%
\frac{1}{T_{h}}\dsum\limits_{t=p}^{T-h}\frac{U_{t,N}\left( \widehat{H^{c}}%
\right) ^{\prime }U_{t,N}\left( \widehat{H^{c}}\right) }{N_{1}}-\frac{1}{%
T_{h}}\dsum\limits_{t=p}^{T-h}\frac{U_{t,N}\left( \widehat{H^{c}}\right)
^{\prime }G_{1}G_{1}^{\prime }U_{t,N}\left( \widehat{H^{c}}\right) }{N_{1}}
\\
&\leq &\frac{1}{T_{h}}\dsum\limits_{t=p}^{T-h}\frac{U_{t,N}\left( \widehat{%
H^{c}}\right) ^{\prime }U_{t,N}\left( \widehat{H^{c}}\right) }{N_{1}}+\frac{1%
}{T_{h}}\dsum\limits_{t=p}^{T-h}\frac{U_{t,N}\left( \widehat{H^{c}}\right)
^{\prime }G_{1}G_{1}^{\prime }U_{t,N}\left( \widehat{H^{c}}\right) }{N_{1}}
\\
&=&\frac{1}{T_{h}}\dsum\limits_{t=p}^{T-h}\left\Vert \frac{U_{t,N}\left( 
\widehat{H^{c}}\right) }{\sqrt{N_{1}}}\right\Vert _{2}^{2}+\frac{1}{T_{h}}%
\dsum\limits_{t=p}^{T-h}\left\Vert \frac{G_{1}^{\prime }U_{t,N}\left( 
\widehat{H^{c}}\right) }{\sqrt{N_{1}}}\right\Vert _{2}^{2}
\end{eqnarray*}%
Applying the results from parts (a) and (b) of this lemma, we then obtain%
\begin{eqnarray*}
\frac{1}{T_{h}}\dsum\limits_{t=p}^{T-h}\left\Vert \frac{G_{2}^{\prime
}U_{t,N}\left( \widehat{H^{c}}\right) }{\sqrt{N_{1}}}\right\Vert _{2}^{2}
&\leq &\frac{1}{T_{h}}\dsum\limits_{t=p}^{T-h}\left\Vert \frac{U_{t,N}\left( 
\widehat{H^{c}}\right) }{\sqrt{N_{1}}}\right\Vert _{2}^{2}+\frac{1}{T_{h}}%
\dsum\limits_{t=p}^{T-h}\left\Vert \frac{G_{1}^{\prime }U_{t,N}\left( 
\widehat{H^{c}}\right) }{\sqrt{N_{1}}}\right\Vert _{2}^{2} \\
&=&O_{p}\left( 1\right) +o_{p}\left( 1\right) \\
&=&O_{p}\left( 1\right) \text{.}
\end{eqnarray*}

Next, to show part (d), let $\overline{C}$ be the constant given in Lemma
C-5 such that%
\begin{equation*}
E\left\Vert \underline{F}_{t}\right\Vert _{2}^{6}\leq \overline{C}<\infty 
\text{ for all }t\text{.}
\end{equation*}%
Now, for any $\epsilon >0$, let $C_{\epsilon }^{\ast }=\overline{C}^{\frac{%
{\Large 1}}{{\Large 3}}}/\epsilon $; then, upon application of Markov's
inequality, we have%
\begin{eqnarray*}
\Pr \left( \frac{1}{T_{h}}\dsum\limits_{t=p}^{T-h}\left\Vert \underline{F}%
_{t}\right\Vert _{2}^{2}\geq C_{\epsilon }^{\ast }\right) &\leq &\frac{1}{%
C_{\epsilon }^{\ast }}\frac{1}{T_{h}}\dsum\limits_{t=p}^{T-h}E\left\Vert 
\underline{F}_{t}\right\Vert _{2}^{2} \\
&\leq &\frac{1}{C_{\epsilon }^{\ast }}\frac{1}{T_{h}}\dsum%
\limits_{t=p}^{T-h}\left( E\left\Vert \underline{F}_{t}\right\Vert
_{2}^{6}\right) ^{\frac{{\Large 1}}{{\Large 3}}}\left( \text{by Liapunov's
inequality}\right) \\
&=&\frac{\epsilon }{\overline{C}^{\frac{{\Large 1}}{{\Large 3}}}}\frac{1}{%
T_{h}}\dsum\limits_{t=p}^{T-h}\overline{C}^{\frac{{\Large 1}}{{\Large 3}}} \\
&=&\epsilon \frac{T-h-p+1}{T_{h}} \\
&\leq &\epsilon \text{ }\left( \text{since }T_{h}=T-h-p+1\right)
\end{eqnarray*}%
so that%
\begin{equation*}
\frac{1}{T_{h}}\dsum\limits_{t=p}^{T-h}\left\Vert \underline{F}%
_{t}\right\Vert _{2}^{2}=O_{p}\left( 1\right)
\end{equation*}%
In addition, note that%
\begin{eqnarray*}
\left\Vert \frac{1}{T_{h}}\dsum\limits_{t=p}^{T-h}\underline{F}_{t}%
\underline{F}_{t}^{\prime }\right\Vert _{2} &\leq &\frac{1}{T_{h}}%
\dsum\limits_{t=p}^{T-h}\left\Vert \underline{F}_{t}\underline{F}%
_{t}^{\prime }\right\Vert _{2} \\
&=&\frac{1}{T_{h}}\dsum\limits_{t=p}^{T-h}\sqrt{\lambda _{\max }\left( 
\underline{F}_{t}\underline{F}_{t}^{\prime }\underline{F}_{t}\underline{F}%
_{t}^{\prime }\right) } \\
&=&\frac{1}{T_{h}}\dsum\limits_{t=p}^{T-h}\sqrt{\left\Vert \underline{F}%
_{t}\right\Vert _{2}^{2}\lambda _{\max }\left( \underline{F}_{t}\underline{F}%
_{t}^{\prime }\right) } \\
&=&\frac{1}{T_{h}}\dsum\limits_{t=p}^{T-h}\sqrt{\left\Vert \underline{F}%
_{t}\right\Vert _{2}^{2}\lambda _{\max }\left( \underline{F}_{t}^{\prime }%
\underline{F}_{t}\right) } \\
&=&\frac{1}{T_{h}}\dsum\limits_{t=p}^{T-h}\sqrt{\left\Vert \underline{F}%
_{t}\right\Vert _{2}^{4}} \\
&=&\frac{1}{T_{h}}\dsum\limits_{t=p}^{T-h}\left\Vert \underline{F}%
_{t}\right\Vert _{2}^{2} \\
&=&O_{p}\left( 1\right)
\end{eqnarray*}

Turning our attention to part (e), write%
\begin{eqnarray*}
&&\frac{\widehat{V}^{\prime }\widehat{G}_{1}^{{\Large \prime }}U\left( 
\widehat{H^{c}}\right) ^{{\Large \prime }}U\left( \widehat{H^{c}}\right) 
\widehat{G}_{1}^{{\Large \prime }}\widehat{V}}{T_{h}\widehat{N}_{1}} \\
&=&\frac{1}{T_{h}}\dsum\limits_{t=p}^{T-h}\frac{\widehat{V}^{\prime }%
\widehat{G}_{1}^{{\Large \prime }}U\left( \widehat{H^{c}}\right) ^{{\Large %
\prime }}\mathbf{e}_{t,T}}{\sqrt{\widehat{N}_{1}}}\frac{\mathbf{e}%
_{t,T}^{\prime }U\left( \widehat{H^{c}}\right) \widehat{G}_{1}\widehat{V}}{%
\sqrt{\widehat{N}_{1}}} \\
&=&\frac{1}{T_{h}}\dsum\limits_{t=p}^{T-h}\frac{\widehat{V}^{\prime }%
\widehat{G}_{1}^{{\Large \prime }}U_{t,N}\left( \widehat{H^{c}}\right) }{%
\sqrt{N_{1}}\sqrt{\left( \widehat{N}_{1}-N_{1}+N_{1}\right) /N_{1}}}\frac{%
U_{t,N}^{\prime }\left( \widehat{H^{c}}\right) \widehat{G}_{1}\widehat{V}}{%
\sqrt{N_{1}}\sqrt{\left( \widehat{N}_{1}-N_{1}+N_{1}\right) /N_{1}}} \\
&=&\left( 1+\frac{\widehat{N}_{1}-N_{1}}{N_{1}}\right) ^{-1}\frac{1}{%
N_{1}T_{h}}\dsum\limits_{t=p}^{T-h}\widehat{V}^{\prime }\widehat{G}_{1}^{%
{\Large \prime }}U_{t,N}\left( \widehat{H^{c}}\right) U_{t,N}\left( \widehat{%
H^{c}}\right) ^{\prime }\widehat{G}_{1}\widehat{V} \\
&=&\left( 1+\frac{\widehat{N}_{1}-N_{1}}{N_{1}}\right) ^{-1}\frac{1}{%
N_{1}T_{h}}\dsum\limits_{t=p}^{T-h}\left\{ \widehat{V}^{\prime }\left(
I_{Kp}+R^{\prime }R\right) ^{-1/2}\left[ G_{1}^{\prime }+R^{\prime
}G_{2}^{\prime }\right] U_{t,N}\left( \widehat{H^{c}}\right) \right. \\
&&\text{ \ \ \ \ \ \ \ \ \ \ \ \ \ \ \ \ \ \ \ \ \ \ \ \ \ \ \ \ \ \ \ \ \ \
\ \ \ \ \ \ \ \ }\left. \times U_{t,N}\left( \widehat{H^{c}}\right) ^{\prime
}\left[ G_{1}+G_{2}R\right] \left( I_{Kp}+R^{\prime }R\right) ^{-1/2}%
\widehat{V}\right\} \\
&=&\left( 1+\frac{\widehat{N}_{1}-N_{1}}{N_{1}}\right) ^{-1}\left\{ \widehat{%
V}^{\prime }\left( I_{Kp}+R^{\prime }R\right) ^{-1/2}G_{1}^{\prime }\right.
\\
&&\left. \times \frac{1}{N_{1}T_{h}}\dsum\limits_{t=p}^{T-h}U_{t,N}\left( 
\widehat{H^{c}}\right) U_{t,N}\left( \widehat{H^{c}}\right) ^{\prime
}G_{1}\left( I_{Kp}+R^{\prime }R\right) ^{-1/2}\widehat{V}\right\} \\
&&+\left( 1+\frac{\widehat{N}_{1}-N_{1}}{N_{1}}\right) ^{-1}\left\{ \widehat{%
V}^{\prime }\left( I_{Kp}+R^{\prime }R\right) ^{-1/2}G_{1}^{\prime }\right.
\\
&&\text{ \ \ \ }\left. \times \frac{1}{N_{1}T_{h}}\dsum%
\limits_{t=p}^{T-h}U_{t,N}\left( \widehat{H^{c}}\right) U_{t,N}\left( 
\widehat{H^{c}}\right) ^{\prime }G_{2}R\left( I_{Kp}+R^{\prime }R\right)
^{-1/2}\widehat{V}\right\} \\
&&+\left( 1+\frac{\widehat{N}_{1}-N_{1}}{N_{1}}\right) ^{-1}\left\{ \widehat{%
V}^{\prime }\left( I_{Kp}+R^{\prime }R\right) ^{-1/2}R^{\prime
}G_{2}^{\prime }\right. \\
&&\text{ \ \ \ \ }\left. \times \frac{1}{N_{1}T_{h}}\dsum%
\limits_{t=p}^{T-h}U_{t,N}\left( \widehat{H^{c}}\right) U_{t,N}\left( 
\widehat{H^{c}}\right) ^{\prime }G_{1}\left( I_{Kp}+R^{\prime }R\right)
^{-1/2}\widehat{V}\right\} \\
&&+\left( 1+\frac{\widehat{N}_{1}-N_{1}}{N_{1}}\right) ^{-1}\left\{ \widehat{%
V}^{\prime }\left( I_{Kp}+R^{\prime }R\right) ^{-1/2}R^{\prime
}G_{2}^{\prime }\right. \\
&&\text{ \ \ \ \ \ }\left. \times \frac{1}{N_{1}T_{h}}\dsum%
\limits_{t=p}^{T-h}U_{t,N}\left( \widehat{H^{c}}\right) U_{t,N}\left( 
\widehat{H^{c}}\right) ^{\prime }G_{2}R\left( I_{Kp}+R^{\prime }R\right)
^{-1/2}\widehat{V}\right\}
\end{eqnarray*}%
To analyze the four terms on the right-hand side of the expression above,
note first that, by the homogeneity of matrix norm and the triangle
inequality,%
\begin{eqnarray*}
\left\Vert G_{1}^{\prime }\frac{1}{N_{1}T_{h}}\dsum%
\limits_{t=p}^{T-h}U_{t,N}\left( \widehat{H^{c}}\right) U_{t,N}\left( 
\widehat{H^{c}}\right) ^{\prime }G_{1}\right\Vert _{2} &\leq &\frac{1}{T_{h}}%
\dsum\limits_{t=p}^{T-h}\left\Vert \frac{G_{1}^{\prime }U_{t,N}\left( 
\widehat{H^{c}}\right) U_{t,N}\left( \widehat{H^{c}}\right) ^{\prime }G_{1}}{%
N_{1}}\right\Vert _{2} \\
&=&\frac{1}{T_{h}}\dsum\limits_{t=p}^{T-h}\sqrt{\lambda _{\max }\left\{
\left( \frac{G_{1}^{\prime }U_{t,N}\left( \widehat{H^{c}}\right)
U_{t,N}\left( \widehat{H^{c}}\right) ^{\prime }G_{1}}{N_{1}}\right)
^{2}\right\} } \\
&=&\frac{1}{T_{h}}\dsum\limits_{t=p}^{T-h}\sqrt{\lambda _{\max }^{2}\left( 
\frac{G_{1}^{\prime }U_{t,N}\left( \widehat{H^{c}}\right) U_{t,N}\left( 
\widehat{H^{c}}\right) ^{\prime }G_{1}}{N_{1}}\right) } \\
&=&\frac{1}{T_{h}}\dsum\limits_{t=p}^{T-h}\lambda _{\max }\left( \frac{%
G_{1}^{\prime }U_{t,N}\left( \widehat{H^{c}}\right) U_{t,N}\left( \widehat{%
H^{c}}\right) ^{\prime }G_{1}}{N_{1}}\right) \\
&=&\frac{1}{T_{h}}\dsum\limits_{t=p}^{T-h}\left\Vert \frac{G_{1}^{\prime
}U_{t,N}\left( \widehat{H^{c}}\right) }{\sqrt{N_{1}}}\right\Vert _{2}^{2}
\end{eqnarray*}%
Similarly, we obtain%
\begin{eqnarray*}
&&\left\Vert G_{1}^{\prime }\frac{1}{N_{1}T_{h}}\dsum%
\limits_{t=p}^{T-h}U_{t,N}\left( \widehat{H^{c}}\right) U_{t,N}\left( 
\widehat{H^{c}}\right) ^{\prime }G_{2}\right\Vert _{2} \\
&\leq &\frac{1}{T_{h}}\dsum\limits_{t=p}^{T-h}\left\Vert \frac{G_{1}^{\prime
}U_{t,N}\left( \widehat{H^{c}}\right) U_{t,N}\left( \widehat{H^{c}}\right)
^{\prime }G_{2}}{N_{1}}\right\Vert _{2} \\
&=&\frac{1}{T_{h}}\dsum\limits_{t=p}^{T-h}\sqrt{\lambda _{\max }\left\{
\left( \frac{G_{2}^{\prime }U_{t,N}\left( \widehat{H^{c}}\right)
U_{t,N}\left( \widehat{H^{c}}\right) ^{\prime }G_{1}G_{1}^{\prime
}U_{t,N}\left( \widehat{H^{c}}\right) U_{t,N}\left( \widehat{H^{c}}\right)
^{\prime }G_{2}}{N_{1}^{2}}\right) \right\} } \\
&=&\frac{1}{T_{h}}\dsum\limits_{t=p}^{T-h}\sqrt{\left\Vert \frac{%
G_{1}^{\prime }U_{t,N}\left( \widehat{H^{c}}\right) }{\sqrt{N_{1}}}%
\right\Vert _{2}^{2}\lambda _{\max }\left\{ \left( \frac{G_{2}^{\prime
}U_{t,N}\left( \widehat{H^{c}}\right) U_{t,N}\left( \widehat{H^{c}}\right)
^{\prime }G_{2}}{N_{1}}\right) \right\} } \\
&=&\frac{1}{T_{h}}\dsum\limits_{t=p}^{T-h}\left\Vert \frac{G_{1}^{\prime
}U_{t,N}\left( \widehat{H^{c}}\right) }{\sqrt{N_{1}}}\right\Vert _{2}\sqrt{%
\lambda _{\max }\left( \frac{G_{2}^{\prime }U_{t,N}\left( \widehat{H^{c}}%
\right) U_{t,N}\left( \widehat{H^{c}}\right) ^{\prime }G_{2}}{N_{1}}\right) }
\\
&\leq &\frac{1}{T_{h}}\dsum\limits_{t=p}^{T-h}\left\Vert \frac{G_{1}^{\prime
}U_{t,N}\left( \widehat{H^{c}}\right) }{\sqrt{N_{1}}}\right\Vert
_{2}\left\Vert \frac{G_{2}^{\prime }U_{t,N}\left( \widehat{H^{c}}\right) }{%
\sqrt{N_{1}}}\right\Vert _{2}
\end{eqnarray*}%
and%
\begin{eqnarray*}
\left\Vert G_{2}^{\prime }\frac{1}{N_{1}T_{h}}\dsum%
\limits_{t=p}^{T-h}U_{t,N}\left( \widehat{H^{c}}\right) U_{t,N}\left( 
\widehat{H^{c}}\right) ^{\prime }G_{2}\right\Vert _{2} &\leq &\frac{1}{T_{h}}%
\dsum\limits_{t=p}^{T-h}\left\Vert \frac{G_{2}^{\prime }U_{t,N}\left( 
\widehat{H^{c}}\right) U_{t,N}\left( \widehat{H^{c}}\right) ^{\prime }G_{2}}{%
N_{1}}\right\Vert _{2} \\
&=&\frac{1}{T_{h}}\dsum\limits_{t=p}^{T-h}\left\Vert \frac{G_{2}^{\prime
}U_{t,N}\left( \widehat{H^{c}}\right) }{\sqrt{N_{1}}}\right\Vert _{2}^{2}%
\text{.}
\end{eqnarray*}%
Hence,%
\begin{eqnarray*}
&&\left\Vert \frac{\widehat{V}^{\prime }\widehat{G}_{1}^{{\Large \prime }%
}U\left( \widehat{H^{c}}\right) ^{{\Large \prime }}U\left( \widehat{H^{c}}%
\right) \widehat{G}_{1}^{{\Large \prime }}\widehat{V}}{T_{h}\widehat{N}_{1}}%
\right\Vert _{2} \\
&\leq &\left\vert 1+\frac{\widehat{N}_{1}-N_{1}}{N_{1}}\right\vert
^{-1}\left\Vert \widehat{V}\right\Vert _{2}^{2}\left\Vert \left(
I_{Kp}+R^{\prime }R\right) ^{-1/2}\right\Vert _{2}^{2}\frac{1}{T_{h}}%
\dsum\limits_{t=p}^{T-h}\left\Vert \frac{G_{1}^{\prime }U_{t,N}\left( 
\widehat{H^{c}}\right) }{\sqrt{N_{1}}}\right\Vert _{2}^{2} \\
&&+2\left\vert 1+\frac{\widehat{N}_{1}-N_{1}}{N_{1}}\right\vert
^{-1}\left\Vert \widehat{V}\right\Vert _{2}^{2}\left\Vert \left(
I_{Kp}+R^{\prime }R\right) ^{-1/2}\right\Vert _{2}^{2}\left\Vert
R\right\Vert _{2}\frac{1}{T_{h}}\dsum\limits_{t=p}^{T-h}\left\Vert \frac{%
G_{1}^{\prime }U_{t,N}\left( \widehat{H^{c}}\right) }{\sqrt{N_{1}}}%
\right\Vert _{2}\left\Vert \frac{G_{2}^{\prime }U_{t,N}\left( \widehat{H^{c}}%
\right) }{\sqrt{N_{1}}}\right\Vert _{2} \\
&&+\left\vert 1+\frac{\widehat{N}_{1}-N_{1}}{N_{1}}\right\vert
^{-1}\left\Vert \widehat{V}\right\Vert _{2}^{2}\left\Vert \left(
I_{Kp}+R^{\prime }R\right) ^{-1/2}\right\Vert _{2}^{2}\left\Vert
R\right\Vert _{2}^{2}\frac{1}{T_{h}}\dsum\limits_{t=p}^{T-h}\left\Vert \frac{%
G_{2}^{\prime }U_{t,N}\left( \widehat{H^{c}}\right) }{\sqrt{N_{1}}}%
\right\Vert _{2}^{2} \\
&=&\left\vert 1+\frac{\widehat{N}_{1}-N_{1}}{N_{1}}\right\vert
^{-1}\left\Vert \left( I_{Kp}+R^{\prime }R\right) ^{-1/2}\right\Vert _{2}^{2}%
\frac{1}{T_{h}}\dsum\limits_{t=p}^{T-h}\left\Vert \frac{G_{1}^{\prime
}U_{t,N}\left( \widehat{H^{c}}\right) }{\sqrt{N_{1}}}\right\Vert _{2}^{2} \\
&&+2\left\vert 1+\frac{\widehat{N}_{1}-N_{1}}{N_{1}}\right\vert
^{-1}\left\Vert \left( I_{Kp}+R^{\prime }R\right) ^{-1/2}\right\Vert _{2}^{2}%
\frac{1}{T_{h}}\dsum\limits_{t=p}^{T-h}\left\Vert \frac{G_{1}^{\prime
}U_{t,N}\left( \widehat{H^{c}}\right) }{\sqrt{N_{1}}}\right\Vert
_{2}\left\Vert R\right\Vert _{2}\left\Vert \frac{G_{2}^{\prime
}U_{t,N}\left( \widehat{H^{c}}\right) }{\sqrt{N_{1}}}\right\Vert _{2} \\
&&+\left\vert 1+\frac{\widehat{N}_{1}-N_{1}}{N_{1}}\right\vert
^{-1}\left\Vert \left( I_{Kp}+R^{\prime }R\right) ^{-1/2}\right\Vert
_{2}^{2}\left\Vert R\right\Vert _{2}^{2}\frac{1}{T_{h}}\dsum%
\limits_{t=p}^{T-h}\left\Vert \frac{G_{2}^{\prime }U_{t,N}\left( \widehat{H}%
\right) }{\sqrt{N_{1}}}\right\Vert _{2}^{2} \\
&&\left( \text{since }\widehat{V}^{\prime }\widehat{V}=I_{Kp}\text{ so that }%
\left\Vert \widehat{V}\right\Vert _{2}=1\right) \\
&\leq &2\left\vert 1+\frac{\widehat{N}_{1}-N_{1}}{N_{1}}\right\vert
^{-1}\left\Vert \left( I_{Kp}+R^{\prime }R\right) ^{-1/2}\right\Vert _{2}^{2}%
\frac{1}{T_{h}}\dsum\limits_{t=p}^{T-h}\left\Vert \frac{G_{1}^{\prime
}U_{t,N}\left( \widehat{H^{c}}\right) }{\sqrt{N_{1}}}\right\Vert _{2}^{2} \\
&&+2\left\vert 1+\frac{\widehat{N}_{1}-N_{1}}{N_{1}}\right\vert
^{-1}\left\Vert \left( I_{Kp}+R^{\prime }R\right) ^{-1/2}\right\Vert
_{2}^{2}\left\Vert R\right\Vert _{2}^{2}\frac{1}{T_{h}}\dsum%
\limits_{t=p}^{T-h}\left\Vert \frac{G_{2}^{\prime }U_{t,N}\left( \widehat{%
H^{c}}\right) }{\sqrt{N_{1}}}\right\Vert _{2}^{2}
\end{eqnarray*}%
It follows that

\begin{eqnarray*}
&&\left\Vert \frac{\widehat{V}^{\prime }\widehat{G}_{1}^{{\Large \prime }%
}U\left( \widehat{H^{c}}\right) ^{{\Large \prime }}U\left( \widehat{H^{c}}%
\right) \widehat{G}_{1}\widehat{V}}{T_{h}\widehat{N}_{1}}\right\Vert _{2} \\
&\leq &2\left\vert 1+\frac{\widehat{N}_{1}-N_{1}}{N_{1}}\right\vert
^{-1}\left\Vert \left( I_{Kp}+R^{\prime }R\right) ^{-1/2}\right\Vert _{2}^{2}%
\frac{1}{T_{h}}\dsum\limits_{t=p}^{T-h}\left\Vert \frac{G_{1}^{\prime
}U_{t,N}\left( \widehat{H^{c}}\right) }{\sqrt{N_{1}}}\right\Vert _{2}^{2} \\
&&+2\left\vert 1+\frac{\widehat{N}_{1}-N_{1}}{N_{1}}\right\vert
^{-1}\left\Vert \left( I_{Kp}+R^{\prime }R\right) ^{-1/2}\right\Vert
_{2}^{2}\left\Vert R\right\Vert _{2}^{2}\frac{1}{T_{h}}\dsum%
\limits_{t=p}^{T-h}\left\Vert \frac{G_{2}^{\prime }U_{t,N}\left( \widehat{%
H^{c}}\right) }{\sqrt{N_{1}}}\right\Vert _{2}^{2} \\
&\leq &2\left\vert 1+\frac{\widehat{N}_{1}-N_{1}}{N_{1}}\right\vert
^{-1}\left( \frac{1}{\sqrt[\backslash ]{1+\lambda _{\min }\left( R^{\prime
}R\right) }}\right) ^{2}\frac{1}{T_{h}}\dsum\limits_{t=p}^{T-h}\left\Vert 
\frac{G_{1}^{\prime }U_{t,N}\left( \widehat{H^{c}}\right) }{\sqrt{N_{1}}}%
\right\Vert _{2}^{2} \\
&&+2\left\vert 1+\frac{\widehat{N}_{1}-N_{1}}{N_{1}}\right\vert ^{-1}\left( 
\frac{\left\Vert R\right\Vert _{2}}{\sqrt[\backslash ]{1+\lambda _{\min
}\left( R^{\prime }R\right) }}\right) ^{2}\frac{1}{T_{h}}\dsum%
\limits_{t=p}^{T-h}\left\Vert \frac{G_{2}^{\prime }U_{t,N}\left( \widehat{%
H^{c}}\right) }{\sqrt{N_{1}}}\right\Vert _{2}^{2} \\
&=&2\left\vert 1+\frac{\widehat{N}_{1}-N_{1}}{N_{1}}\right\vert ^{-1}\frac{1%
}{1+\lambda _{\min }\left( R^{\prime }R\right) }\frac{1}{T_{h}}%
\dsum\limits_{t=p}^{T-h}\left\Vert \frac{G_{1}^{\prime }U_{t,N}\left( 
\widehat{H^{c}}\right) }{\sqrt{N_{1}}}\right\Vert _{2}^{2} \\
&&+2\left\vert 1+\frac{\widehat{N}_{1}-N_{1}}{N_{1}}\right\vert ^{-1}\frac{%
\left\Vert R\right\Vert _{2}^{2}}{1+\lambda _{\min }\left( R^{\prime
}R\right) }\frac{1}{T_{h}}\dsum\limits_{t=p}^{T-h}\left\Vert \frac{%
G_{2}^{\prime }U_{t,N}\left( \widehat{H^{c}}\right) }{\sqrt{N_{1}}}%
\right\Vert _{2}^{2} \\
&\leq &2\left\vert 1+\frac{\widehat{N}_{1}-N_{1}}{N_{1}}\right\vert ^{-1}%
\left[ \frac{1}{T_{h}}\dsum\limits_{t=p}^{T-h}\left\Vert \frac{G_{1}^{\prime
}U_{t,N}\left( \widehat{H^{c}}\right) }{\sqrt{N_{1}}}\right\Vert
_{2}^{2}+\left\Vert R\right\Vert _{2}^{2}\frac{1}{T_{h}}\dsum%
\limits_{t=p}^{T-h}\left\Vert \frac{G_{2}^{\prime }U_{t,N}\left( \widehat{%
H^{c}}\right) }{\sqrt{N_{1}}}\right\Vert _{2}^{2}\right] \\
&=&o_{p}\left( 1\right) \text{ \ }\left( \text{applying part (a) of Lemma
D-14, part (a) of Lemma D-15,}\right. \text{ } \\
&&\text{ \ \ \ \ \ \ \ \ \ \ \ \ }\left. \text{ parts (a) and (c) of this
lemma, and Slutsky's theorem}\right)
\end{eqnarray*}

To show part (f), first write%
\begin{eqnarray*}
\left\Vert \frac{\underline{F}^{\prime }U\left( \widehat{H^{c}}\right) 
\widehat{G}_{1}\widehat{V}}{T_{h}\sqrt{\widehat{N}_{1}}}\right\Vert _{2}
&=&\left\Vert \frac{1}{T_{h}}\dsum\limits_{t=p}^{T-h}\frac{\underline{F}%
_{t}U_{t,N}^{\prime }\left( \widehat{H^{c}}\right) \widehat{G}_{1}\widehat{V}%
}{\sqrt{\widehat{N}_{1}}}\right\Vert _{2} \\
&\leq &\frac{1}{T_{h}}\dsum\limits_{t=p}^{T-h}\left\Vert \frac{\underline{F}%
_{t}U_{t,N}^{\prime }\left( \widehat{H^{c}}\right) \widehat{G}_{1}\widehat{V}%
}{\sqrt{\widehat{N}_{1}}}\right\Vert _{2} \\
&=&\frac{1}{T_{h}}\dsum\limits_{t=p}^{T-h}\sqrt{\lambda _{\max }\left( \frac{%
\widehat{V}^{\prime }\widehat{G}_{1}^{{\Large \prime }}U_{t,N}\left( 
\widehat{H^{c}}\right) \underline{F}_{t}^{\prime }\underline{F}_{t}U_{t,N}^{%
{\Large \prime }}\left( \widehat{H^{c}}\right) \widehat{G}_{1}\widehat{V}}{%
\widehat{N}_{1}}\right) } \\
&=&\frac{1}{T_{h}}\dsum\limits_{t=p}^{T-h}\left\Vert \underline{F}%
_{t}\right\Vert _{2}\sqrt{\lambda _{\max }\left( \frac{\widehat{V}^{\prime }%
\widehat{G}_{1}^{{\Large \prime }}U_{t,N}\left( \widehat{H^{c}}\right)
U_{t,N}^{{\Large \prime }}\left( \widehat{H^{c}}\right) \widehat{G}_{1}%
\widehat{V}}{\widehat{N}_{1}}\right) } \\
&=&\frac{1}{T_{h}}\dsum\limits_{t=p}^{T-h}\left\Vert \underline{F}%
_{t}\right\Vert _{2}\sqrt{\frac{U_{t,N}^{{\Large \prime }}\left( \widehat{%
H^{c}}\right) \widehat{G}_{1}\widehat{V}\widehat{V}^{\prime }\widehat{G}%
_{1}^{{\Large \prime }}U_{t,N}\left( \widehat{H^{c}}\right) }{\widehat{N}_{1}%
}} \\
&=&\frac{1}{T_{h}}\dsum\limits_{t=p}^{T-h}\left\Vert \underline{F}%
_{t}\right\Vert _{2}\left\Vert \frac{\widehat{V}^{\prime }\widehat{G}_{1}^{%
{\Large \prime }}U_{t,N}\left( \widehat{H^{c}}\right) }{\sqrt{\widehat{N}_{1}%
}}\right\Vert _{2} \\
&\leq &\sqrt{\frac{1}{T_{h}}\dsum\limits_{t=p}^{T-h}\left\Vert \underline{F}%
_{t}\right\Vert _{2}^{2}}\sqrt{\frac{1}{T_{h}}\dsum\limits_{t=p}^{T-h}\left%
\Vert \frac{\widehat{V}^{\prime }\widehat{G}_{1}^{{\Large \prime }%
}U_{t,N}\left( \widehat{H^{c}}\right) }{\sqrt{\widehat{N}_{1}}}\right\Vert
_{2}^{2}}
\end{eqnarray*}%
Next, note that%
\begin{eqnarray*}
&&\left\Vert \frac{\widehat{V}^{\prime }\widehat{G}_{1}^{{\Large \prime }%
}U_{t,N}\left( \widehat{H^{c}}\right) }{\sqrt{N_{1}}}\right\Vert _{2}^{2} \\
&=&\left( \left\Vert \frac{\widehat{V}^{\prime }\left( I_{Kp}+R^{\prime
}R\right) ^{-1/2}\left[ G_{1}^{\prime }+R^{\prime }G_{2}^{\prime }\right]
U_{t,N}\left( \widehat{H^{c}}\right) }{\sqrt{N_{1}}}\right\Vert _{2}\right)
^{2} \\
&\leq &\left( \left\Vert \widehat{V}\right\Vert _{2}\left\Vert \left(
I_{Kp}+R^{\prime }R\right) ^{-1/2}\right\Vert _{2}\left\Vert \frac{%
G_{1}^{\prime }U_{t,N}\left( \widehat{H^{c}}\right) }{\sqrt{N_{1}}}%
\right\Vert _{2}\right. \\
&&\left. +\left\Vert \widehat{V}\right\Vert _{2}\left\Vert \left(
I_{Kp}+R^{\prime }R\right) ^{-1/2}\right\Vert _{2}\left\Vert R\right\Vert
_{2}\left\Vert \frac{G_{2}^{\prime }U_{t,N}\left( \widehat{H^{c}}\right) }{%
\sqrt{N_{1}}}\right\Vert _{2}\right) ^{2} \\
&=&\left( \left\Vert \left( I_{Kp}+R^{\prime }R\right) ^{-1/2}\right\Vert
_{2}\left\Vert \frac{G_{1}^{\prime }U_{t,N}\left( \widehat{H^{c}}\right) }{%
\sqrt{N_{1}}}\right\Vert _{2}+\left\Vert \left( I_{Kp}+R^{\prime }R\right)
^{-1/2}\right\Vert _{2}\left\Vert R\right\Vert _{2}\left\Vert \frac{%
G_{2}^{\prime }U_{t,N}\left( \widehat{H^{c}}\right) }{\sqrt{N_{1}}}%
\right\Vert _{2}\right) ^{2} \\
&&\left( \text{since }\widehat{V}^{\prime }\widehat{V}=I_{Kp}\text{ so that }%
\left\Vert \widehat{V}\right\Vert _{2}=1\right) \\
&\leq &2\left\Vert \left( I_{Kp}+R^{\prime }R\right) ^{-1/2}\right\Vert
_{2}^{2}\left\Vert \frac{G_{1}^{\prime }U_{t,N}\left( \widehat{H^{c}}\right) 
}{\sqrt{N_{1}}}\right\Vert _{2}^{2}+2\left\Vert \left( I_{Kp}+R^{\prime
}R\right) ^{-1/2}\right\Vert _{2}^{2}\left\Vert R\right\Vert
_{2}^{2}\left\Vert \frac{G_{2}^{\prime }U_{t,N}\left( \widehat{H^{c}}\right) 
}{\sqrt{N_{1}}}\right\Vert _{2}^{2}
\end{eqnarray*}%
from which we obtain%
\begin{eqnarray*}
&&\frac{1}{T_{h}}\dsum\limits_{t=p}^{T-h}\left\Vert \frac{\widehat{V}%
^{\prime }\widehat{G}_{1}^{{\Large \prime }}U_{t,N}\left( \widehat{H^{c}}%
\right) }{\sqrt{\widehat{N}_{1}}}\right\Vert _{2}^{2} \\
&=&\frac{1}{T_{h}}\dsum\limits_{t=p}^{T-h}\left\Vert \frac{\widehat{V}%
^{\prime }\widehat{G}_{1}^{{\Large \prime }}U_{t,N}\left( \widehat{H^{c}}%
\right) }{\sqrt{N_{1}}\sqrt{\left( N_{1}+\widehat{N}_{1}-N_{1}\right) /N_{1}}%
}\right\Vert _{2}^{2} \\
&=&\left\vert 1+\frac{\widehat{N}_{1}-N_{1}}{N_{1}}\right\vert ^{-1}\frac{1}{%
T_{h}}\dsum\limits_{t=p}^{T-h}\left\Vert \frac{\widehat{V}^{\prime }\widehat{%
G}_{1}^{{\Large \prime }}U_{t,N}\left( \widehat{H^{c}}\right) }{\sqrt{N_{1}}}%
\right\Vert _{2}^{2} \\
&\leq &\left\vert 1+\frac{\widehat{N}_{1}-N_{1}}{N_{1}}\right\vert
^{-1}\left\{ 2\left\Vert \left( I_{K}+R^{\prime }R\right) ^{-1/2}\right\Vert
_{2}^{2}\frac{1}{T_{h}}\dsum\limits_{t=p}^{T-h}\left\Vert \frac{%
G_{1}^{\prime }U_{t,N}\left( \widehat{H^{c}}\right) }{\sqrt{N_{1}}}%
\right\Vert _{2}^{2}\right. \\
&&\left. +2\left\Vert \left( I_{K}+R^{\prime }R\right) ^{-1/2}\right\Vert
_{2}^{2}\left\Vert R\right\Vert _{2}^{2}\frac{1}{T_{h}}\dsum%
\limits_{t=p}^{T-h}\left\Vert \frac{G_{2}^{\prime }U_{t,N}\left( \widehat{%
H^{c}}\right) }{\sqrt{N_{1}}}\right\Vert _{2}^{2}\right\} \\
&\leq &\left\vert 1+\frac{\widehat{N}_{1}-N_{1}}{N_{1}}\right\vert
^{-1}\left\{ \frac{2}{1+\lambda _{\min }\left( R^{\prime }R\right) }\frac{1}{%
T_{h}}\dsum\limits_{t=p}^{T-h}\left\Vert \frac{G_{1}^{\prime }U_{t,N}\left( 
\widehat{H^{c}}\right) }{\sqrt{N_{1}}}\right\Vert _{2}^{2}\right. \\
&&\left. +\frac{2\left\Vert R\right\Vert _{2}^{2}}{\sqrt[\backslash ]{%
1+\lambda _{\min }\left( R^{\prime }R\right) }}\frac{1}{T_{h}}%
\dsum\limits_{t=p}^{T-h}\left\Vert \frac{G_{2}^{\prime }U_{t,N}\left( 
\widehat{H^{c}}\right) }{\sqrt{N_{1}}}\right\Vert _{2}^{2}\right\} \\
&\leq &\left\vert 1+\frac{\widehat{N}_{1}-N_{1}}{N_{1}}\right\vert
^{-1}\left\{ \frac{2}{T_{h}}\dsum\limits_{t=p}^{T-h}\left\Vert \frac{%
G_{1}^{\prime }U_{t,N}\left( \widehat{H^{c}}\right) }{\sqrt{N_{1}}}%
\right\Vert _{2}^{2}+2\left\Vert R\right\Vert _{2}^{2}\frac{1}{T_{h}}%
\dsum\limits_{t=p}^{T-h}\left\Vert \frac{G_{2}^{\prime }U_{t,N}\left( 
\widehat{H^{c}}\right) }{\sqrt{N_{1}}}\right\Vert _{2}^{2}\right\} \\
&=&o_{p}\left( 1\right) \text{ \ }\left( \text{applying part (a) of Lemma
D-14, part (a) of Lemma D-15, }\right. \\
&&\text{ \ \ \ \ \ \ \ \ \ \ \ }\left. \text{parts (a) and (c) of this
lemma, and Slutsky's theorem}\right)
\end{eqnarray*}%
It then follows from part (d) of this lemma and the Slutsky's theorem that%
\begin{eqnarray*}
\left\Vert \frac{\underline{F}^{\prime }U\left( \widehat{H^{c}}\right) 
\widehat{G}_{1}\widehat{V}}{T_{h}\sqrt{\widehat{N}_{1}}}\right\Vert _{2}
&\leq &\sqrt{\frac{1}{T_{h}}\dsum\limits_{t=p}^{T-h}\left\Vert \underline{F}%
_{t}\right\Vert _{2}^{2}}\sqrt{\frac{1}{T_{h}}\dsum\limits_{t=p}^{T-h}\left%
\Vert \frac{\widehat{V}^{\prime }\widehat{G}_{1}^{{\Large \prime }%
}U_{t,N}\left( \widehat{H^{c}}\right) }{\sqrt{\widehat{N}_{1}}}\right\Vert
_{2}^{2}} \\
&=&O_{p}\left( 1\right) o_{p}\left( 1\right) \\
&=&o_{p}\left( 1\right)
\end{eqnarray*}

Lastly, to show part (g), first write

\begin{eqnarray*}
&&\frac{1}{T_{h}}\dsum\limits_{t=p}^{T-h}\left( \widehat{\underline{F}}%
_{t}-Q^{\prime }\underline{F}_{t}\right) \left( \widehat{\underline{F}}%
_{t}-Q^{\prime }\underline{F}_{t}\right) ^{\prime } \\
&=&\frac{1}{T_{h}}\dsum\limits_{t=p}^{T-h}\left\{ \left( \frac{\widehat{V}%
^{\prime }\widehat{G}_{1}^{{\Large \prime }}\Gamma \left( \widehat{H^{c}}%
\right) \underline{F}_{t}}{\sqrt{\widehat{N}_{1}}}-Q^{\prime }\underline{F}%
_{t}+\frac{\widehat{V}^{\prime }\widehat{G}_{1}^{{\Large \prime }%
}U_{t,N}\left( \widehat{H^{c}}\right) }{\sqrt{\widehat{N}_{1}}}\right)
\right. \\
&&\text{ \ \ \ \ \ \ \ \ }\left. \times \left( \frac{\widehat{V}^{\prime }%
\widehat{G}_{1}^{{\Large \prime }}\Gamma \left( \widehat{H^{c}}\right) 
\underline{F}_{t}}{\sqrt{\widehat{N}_{1}}}-Q^{\prime }\underline{F}_{t}+%
\frac{\widehat{V}^{\prime }\widehat{G}_{1}^{{\Large \prime }}U_{t,N}\left( 
\widehat{H^{c}}\right) }{\sqrt{\widehat{N}_{1}}}\right) \right\} ^{\prime }
\\
&=&\frac{1}{T_{h}}\dsum\limits_{t=p}^{T-h}\left( \frac{\widehat{V}^{\prime }%
\widehat{G}_{1}^{{\Large \prime }}\Gamma \left( \widehat{H^{c}}\right) 
\underline{F}_{t}}{\sqrt{\widehat{N}_{1}}}-Q^{\prime }\underline{F}%
_{t}\right) \left( \frac{\widehat{V}^{\prime }\widehat{G}_{1}^{{\Large %
\prime }}\Gamma \left( \widehat{H^{c}}\right) \underline{F}_{t}}{\sqrt{%
\widehat{N}_{1}}}-Q^{\prime }\underline{F}_{t}\right) ^{\prime } \\
&&+\frac{1}{T_{h}}\dsum\limits_{t=p}^{T-h}\left( \frac{\widehat{V}^{\prime }%
\widehat{G}_{1}^{{\Large \prime }}\Gamma \left( \widehat{H^{c}}\right) 
\underline{F}_{t}}{\sqrt{\widehat{N}_{1}}}-Q^{\prime }\underline{F}%
_{t}\right) \frac{U_{t,N}\left( \widehat{H^{c}}\right) ^{\prime }\widehat{G}%
_{1}\widehat{V}}{\sqrt{\widehat{N}_{1}}} \\
&&+\frac{1}{T_{h}}\dsum\limits_{t=p}^{T-h}\frac{\widehat{V}^{\prime }%
\widehat{G}_{1}^{{\Large \prime }}U_{t,N}\left( \widehat{H^{c}}\right) }{%
\sqrt{\widehat{N}_{1}}}\left( \frac{\widehat{V}^{\prime }\widehat{G}_{1}^{%
{\Large \prime }}\Gamma \left( \widehat{H^{c}}\right) \underline{F}_{t}}{%
\sqrt{\widehat{N}_{1}}}-Q^{\prime }\underline{F}_{t}\right) ^{\prime } \\
&&+\frac{1}{T_{h}}\dsum\limits_{t=p}^{T-h}\frac{\widehat{V}^{\prime }%
\widehat{G}_{1}^{{\Large \prime }}U_{t,N}\left( \widehat{H^{c}}\right) }{%
\sqrt{\widehat{N}_{1}}}\frac{U_{t,N}\left( \widehat{H^{c}}\right) ^{\prime }%
\widehat{G}_{1}\widehat{V}}{\sqrt{\widehat{N}_{1}}} \\
&=&\left( \frac{\widehat{V}^{\prime }\widehat{G}_{1}^{{\Large \prime }%
}\Gamma \left( \widehat{H^{c}}\right) }{\sqrt{\widehat{N}_{1}}}-Q^{\prime
}\right) \frac{1}{T_{h}}\dsum\limits_{t=p}^{T-h}\underline{F}_{t}\underline{F%
}_{t}^{\prime }\left( \frac{\Gamma \left( \widehat{H^{c}}\right) ^{{\Large %
\prime }}\widehat{G}_{1}\widehat{V}}{\sqrt{\widehat{N}_{1}}}-Q\right) \\
&&+\left( \frac{\widehat{V}^{\prime }\widehat{G}_{1}^{{\Large \prime }%
}\Gamma \left( \widehat{H^{c}}\right) }{\sqrt{\widehat{N}_{1}}}-Q^{\prime
}\right) \frac{\underline{F}^{\prime }U\left( \widehat{H^{c}}\right) 
\widehat{G}_{1}\widehat{V}}{T_{h}\sqrt{\widehat{N}_{1}}} \\
&&+\frac{\widehat{V}^{\prime }\widehat{G}_{1}^{{\Large \prime }}U\left( 
\widehat{H^{c}}\right) ^{{\Large \prime }}\underline{F}}{T_{h}\sqrt{\widehat{%
N}_{1}}}\left( \frac{\Gamma \left( \widehat{H^{c}}\right) ^{{\Large \prime }}%
\widehat{G}_{1}\widehat{V}}{\sqrt{\widehat{N}_{1}}}-Q\right) +\frac{\widehat{%
V}^{\prime }\widehat{G}_{1}^{{\Large \prime }}U\left( \widehat{H^{c}}\right)
^{{\Large \prime }}U\left( \widehat{H^{c}}\right) \widehat{G}_{1}\widehat{V}%
}{T_{h}\widehat{N}_{1}}
\end{eqnarray*}%
where $U_{t,N}\left( \widehat{H^{c}}\right) =U\left( \widehat{H^{c}}\right)
^{{\Large \prime }}\mathbf{e}_{t,T}=\left( 
\begin{array}{cccc}
\mathbb{I}\left\{ 1\in \widehat{H^{c}}\right\} u_{1,t} & \mathbb{I}\left\{
2\in \widehat{H^{c}}\right\} u_{2,t} & \cdots & \mathbb{I}\left\{ N\in 
\widehat{H^{c}}\right\} u_{N,t}%
\end{array}%
\right) ^{\prime }$. Applying part (h) of Lemma D-15 and parts (d), (e), and
(f) of this lemma and the Slutsky's theorem, we obtain%
\begin{eqnarray*}
&&\left\Vert \frac{1}{T_{h}}\dsum\limits_{t=p}^{T-h}\left( \widehat{F}%
_{t}-Q^{\prime }\underline{F}_{t}\right) \left( \widehat{F}_{t}-Q^{\prime }%
\underline{F}_{t}\right) ^{\prime }\right\Vert _{2} \\
&\leq &\left\Vert \left( \frac{\widehat{V}^{\prime }\widehat{G}_{1}^{{\Large %
\prime }}\Gamma \left( \widehat{H^{c}}\right) }{\sqrt{\widehat{N}_{1}}}%
-Q^{\prime }\right) \frac{1}{T_{h}}\dsum\limits_{t=p}^{T-h}\underline{F}_{t}%
\underline{F}_{t}^{\prime }\left( \frac{\Gamma \left( \widehat{H^{c}}\right)
^{{\Large \prime }}\widehat{G}_{1}\widehat{V}}{\sqrt{\widehat{N}_{1}}}%
-Q\right) \right\Vert _{2} \\
&&+\left\Vert \left( \frac{\widehat{V}^{\prime }\widehat{G}_{1}^{{\Large %
\prime }}\Gamma \left( \widehat{H^{c}}\right) }{\sqrt{\widehat{N}_{1}}}%
-Q^{\prime }\right) \frac{\underline{F}^{\prime }U\left( \widehat{H^{c}}%
\right) \widehat{G}_{1}\widehat{V}}{T_{h}\sqrt{\widehat{N}_{1}}}\right\Vert
_{2} \\
&&+\left\Vert \frac{\widehat{V}^{\prime }\widehat{G}_{1}^{{\Large \prime }%
}U\left( \widehat{H^{c}}\right) ^{{\Large \prime }}\underline{F}}{T_{h}\sqrt{%
\widehat{N}_{1}}}\left( \frac{\Gamma \left( \widehat{H^{c}}\right) ^{{\Large %
\prime }}\widehat{G}_{1}\widehat{V}}{\sqrt{\widehat{N}_{1}}}-Q\right)
\right\Vert _{2}+\left\Vert \frac{\widehat{V}^{\prime }\widehat{G}_{1}^{%
{\Large \prime }}U\left( \widehat{H^{c}}\right) ^{{\Large \prime }}U\left( 
\widehat{H^{c}}\right) \widehat{G}_{1}\widehat{V}}{T_{h}\widehat{N}_{1}}%
\right\Vert _{2} \\
&=&\left\Vert \left( \frac{\widehat{V}^{\prime }\widehat{G}_{1}^{{\Large %
\prime }}\Gamma \left( \widehat{H^{c}}\right) }{\sqrt{\widehat{N}_{1}}}%
-Q^{\prime }\right) \frac{1}{T_{h}}\dsum\limits_{t=p}^{T-h}\underline{F}_{t}%
\underline{F}_{t}^{\prime }\left( \frac{\Gamma \left( \widehat{H^{c}}\right)
^{{\Large \prime }}\widehat{G}_{1}\widehat{V}}{\sqrt{\widehat{N}_{1}}}%
-Q\right) \right\Vert _{2} \\
&&+2\left\Vert \left( \frac{\widehat{V}^{\prime }\widehat{G}_{1}^{{\Large %
\prime }}\Gamma \left( \widehat{H^{c}}\right) }{\sqrt{\widehat{N}_{1}}}%
-Q^{\prime }\right) \frac{\underline{F}^{\prime }U\left( \widehat{H^{c}}%
\right) \widehat{G}_{1}\widehat{V}}{T_{h}\sqrt{\widehat{N}_{1}}}\right\Vert
_{2}+\left\Vert \frac{\widehat{V}^{\prime }\widehat{G}_{1}^{{\Large \prime }%
}U\left( \widehat{H^{c}}\right) ^{{\Large \prime }}U\left( \widehat{H^{c}}%
\right) \widehat{G}_{1}\widehat{V}}{T_{h}\widehat{N}_{1}}\right\Vert _{2} \\
&\leq &\left\Vert \frac{\widehat{V}^{\prime }\widehat{G}_{1}^{{\Large \prime 
}}\Gamma \left( \widehat{H^{c}}\right) }{\sqrt{\widehat{N}_{1}}}-Q^{\prime
}\right\Vert _{2}^{2}\left\Vert \frac{1}{T_{h}}\dsum\limits_{t=p}^{T}%
\underline{F}_{t}\underline{F}_{t}^{\prime }\right\Vert _{2}+2\left\Vert 
\frac{\widehat{V}^{\prime }\widehat{G}_{1}^{{\Large \prime }}\Gamma \left( 
\widehat{H^{c}}\right) }{\sqrt{\widehat{N}_{1}}}-Q^{\prime }\right\Vert
_{2}\left\Vert \frac{\underline{F}^{\prime }U\left( \widehat{H^{c}}\right) 
\widehat{G}_{1}\widehat{V}}{T_{h}\sqrt{\widehat{N}_{1}}}\right\Vert _{2} \\
&&+\left\Vert \frac{\widehat{V}^{\prime }\widehat{G}_{1}^{{\Large \prime }%
}U\left( \widehat{H^{c}}\right) ^{{\Large \prime }}U\left( \widehat{H^{c}}%
\right) \widehat{G}_{1}\widehat{V}}{T_{h}\widehat{N}_{1}}\right\Vert _{2} \\
&=&o_{p}\left( 1\right) O_{p}\left( 1\right) +o_{p}\left( 1\right)
o_{p}\left( 1\right) +o_{p}\left( 1\right) \\
&=&o_{p}\left( 1\right) \text{. }\square
\end{eqnarray*}

\bigskip

\noindent \textbf{Lemma D-17: }Suppose that Assumptions 3-1, 3-2, 3-3, 3-4,
3-5, 3-6, 3-7, 3-8, 3-9, 3-10, and 3-11* hold. Then, the following
statements are true.

\begin{enumerate}
\item[(a)] 
\begin{equation*}
\frac{\widehat{\underline{F}}^{\prime }\widehat{\underline{F}}}{T_{h}}-\frac{%
1}{T_{h}}\dsum\limits_{t=p}^{T-h}Q^{\prime }E\left[ \underline{F}_{t}%
\underline{F}_{t}^{\prime }\right] Q=o_{p}\left( 1\right) \text{, where }%
T_{h}=T-h-p+1\text{.}
\end{equation*}

\item[(b)] 
\begin{equation*}
\frac{\widehat{\underline{F}}^{\prime }\underline{Y}}{T_{h}}-\frac{1}{T_{h}}%
\dsum\limits_{t=p}^{T-h}Q^{\prime }E\left[ \underline{F}_{t}\underline{Y}%
_{t}^{\prime }\right] =o_{p}\left( 1\right)
\end{equation*}

\item[(c)] 
\begin{equation*}
\frac{\widehat{\underline{F}}^{\prime }\iota _{T_{h}}}{T_{h}}-\frac{1}{T_{h}}%
\dsum\limits_{t=p}^{T-h}Q^{\prime }E\left[ \underline{F}_{t}\right]
=o_{p}\left( 1\right) ,
\end{equation*}%
where $\iota _{T_{h}}=\left( 1,1,...,1\right) ^{\prime }$ is a $T_{h}\times
1 $ vector.

\item[(d)] 
\begin{equation*}
\frac{\underline{\widehat{F}}^{\prime }\left( \underline{\widehat{F}}-%
\underline{F}Q\right) Q^{-1}B_{2}}{T_{h}}=o_{p}\left( 1\right)
\end{equation*}

\item[(e)] 
\begin{equation*}
\frac{\underline{Y}^{\prime }\left( \underline{\widehat{F}}-\underline{F}%
Q\right) Q^{-1}B_{2}}{T_{h}}=o_{p}\left( 1\right)
\end{equation*}

\item[(f)] 
\begin{equation*}
\frac{\iota _{T_{h}}^{\prime }\left( \underline{\widehat{F}}-\underline{F}%
Q\right) Q^{-1}B_{2}}{T_{h}}=o_{p}\left( 1\right)
\end{equation*}

\item[(g)] 
\begin{equation*}
\frac{\underline{\widehat{F}}^{\prime }\mathfrak{H}}{T_{h}}=\frac{1}{T_{h}}%
\dsum\limits_{t=p}^{T-h}\underline{\widehat{F}}_{t}\eta _{t{\LARGE +}%
h}^{\prime }=o_{p}\left( 1\right)
\end{equation*}
\end{enumerate}

\bigskip

\noindent \textbf{Proof of Lemma D-17:}

To show part (a), first write%
\begin{eqnarray*}
&&\frac{\widehat{\underline{F}}^{\prime }\widehat{\underline{F}}}{T_{h}}-%
\frac{1}{T_{h}}\dsum\limits_{t=p}^{T-h}Q^{\prime }E\left[ \underline{F}_{t}%
\underline{F}_{t}^{\prime }\right] Q \\
&=&\frac{1}{T_{h}}\dsum\limits_{t=p}^{T-h}\widehat{\underline{F}}_{t}%
\widehat{\underline{F}}_{t}^{\prime }-\frac{1}{T_{h}}\dsum%
\limits_{t=p}^{T-h}Q^{\prime }E\left[ \underline{F}_{t}\underline{F}%
_{t}^{\prime }\right] Q \\
&=&\frac{1}{T_{h}}\dsum\limits_{t=p}^{T-h}\left( \widehat{\underline{F}}%
_{t}-Q^{\prime }\underline{F}_{t}+Q^{\prime }\underline{F}_{t}\right) \left( 
\widehat{\underline{F}}_{t}-Q^{\prime }\underline{F}_{t}+Q^{\prime }%
\underline{F}_{t}\right) ^{\prime }-\frac{1}{T_{h}}\dsum%
\limits_{t=p}^{T-h}Q^{\prime }E\left[ \underline{F}_{t}\underline{F}%
_{t}^{\prime }\right] Q \\
&=&\frac{1}{T_{h}}\dsum\limits_{t=p}^{T-h}\left( \widehat{\underline{F}}%
_{t}-Q^{\prime }\underline{F}_{t}\right) \left( \widehat{\underline{F}}%
_{t}-Q^{\prime }\underline{F}_{t}\right) ^{\prime }+Q^{\prime }\frac{1}{T_{h}%
}\dsum\limits_{t=p}^{T-h}\underline{F}_{t}\left( \widehat{\underline{F}}%
_{t}-Q^{\prime }\underline{F}_{t}\right) ^{\prime } \\
&&+\frac{1}{T_{h}}\dsum\limits_{t=p}^{T-h}\left( \widehat{\underline{F}}%
_{t}-Q^{\prime }\underline{F}_{t}\right) \underline{F}_{t}^{\prime
}Q+Q^{\prime }\left( \frac{1}{T_{h}}\dsum\limits_{t=p}^{T-h}\underline{F}_{t}%
\underline{F}_{t}^{\prime }-\frac{1}{T_{h}}\dsum\limits_{t=p}^{T-h}E\left[ 
\underline{F}_{t}\underline{F}_{t}^{\prime }\right] \right) Q
\end{eqnarray*}%
Now, by part (g) of Lemma D-16, we have that%
\begin{equation*}
\frac{1}{T_{h}}\dsum\limits_{t=p}^{T-h}\left( \widehat{\underline{F}}%
_{t}-Q^{\prime }\underline{F}_{t}\right) \left( \widehat{\underline{F}}%
_{t}-Q^{\prime }\underline{F}_{t}\right) ^{\prime }\overset{p}{\rightarrow }0
\end{equation*}%
Moreover, for any $a,b\in \mathbb{R}^{Kp}$ such that $\left\Vert
a\right\Vert _{2}=\left\Vert b\right\Vert _{2}=1$ 
\begin{eqnarray*}
&&\left\vert a^{\prime }Q^{\prime }\frac{1}{T_{h}}\dsum\limits_{t=p}^{T-h}%
\underline{F}_{t}\left( \widehat{\underline{F}}_{t}-Q^{\prime }\underline{F}%
_{t}\right) ^{\prime }b\right\vert \\
&\leq &\sqrt{a^{\prime }Q^{\prime }\left( \frac{1}{T_{h}}\dsum%
\limits_{t=p}^{T-h}\underline{F}_{t}\underline{F}_{t}^{\prime }\right) Qa}%
\sqrt{\frac{1}{T_{h}}\dsum\limits_{t=p}^{T-h}b^{\prime }\left( \widehat{%
\underline{F}}_{t}-Q^{\prime }\underline{F}_{t}\right) \left( \widehat{%
\underline{F}}_{t}-Q^{\prime }\underline{F}_{t}\right) ^{\prime }b} \\
&\leq &\sqrt{a^{\prime }Q^{\prime }Qa\lambda _{\max }\left( \frac{1}{T_{h}}%
\dsum\limits_{t=p}^{T-h}\underline{F}_{t}\underline{F}_{t}^{\prime }\right) }%
\sqrt{\frac{1}{T_{h}}\dsum\limits_{t=p}^{T-h}b^{\prime }\left( \widehat{%
\underline{F}}_{t}-Q^{\prime }\underline{F}_{t}\right) \left( \widehat{%
\underline{F}}_{t}-Q^{\prime }\underline{F}_{t}\right) ^{\prime }b} \\
&=&\sqrt{a^{\prime }Q^{\prime }Qa\left\Vert \frac{1}{T_{h}}%
\dsum\limits_{t=p}^{T-h}\underline{F}_{t}\underline{F}_{t}^{\prime
}\right\Vert _{2}}\sqrt{\frac{1}{T_{h}}\dsum\limits_{t=p}^{T-h}b^{\prime
}\left( \widehat{\underline{F}}_{t}-Q^{\prime }\underline{F}_{t}\right)
\left( \widehat{\underline{F}}_{t}-Q^{\prime }\underline{F}_{t}\right)
^{\prime }b} \\
&&\left( \text{since, for a symmetric psd matrix }A\text{, }\right. \\
&&\left. \left\Vert A\right\Vert _{2}=\sqrt{\lambda _{\max }\left( A^{\prime
}A\right) }=\sqrt{\lambda _{\max }\left( A^{2}\right) }=\sqrt{\left[ \lambda
_{\max }\left( A\right) \right] ^{2}}=\lambda _{\max }\left( A\right) \right)
\end{eqnarray*}%
Now, by Assumption 3-6, there exists a positive constant $C$ such that%
\begin{eqnarray}
a^{\prime }Q^{\prime }Qa &=&a^{\prime }\widehat{V}^{\prime }\Xi ^{\prime
}\left( \frac{\Gamma ^{\prime }\Gamma }{N_{1}}\right) ^{1/2}\left( \frac{%
\Gamma ^{\prime }\Gamma }{N_{1}}\right) ^{1/2}\Xi \widehat{V}a  \notag \\
&=&a^{\prime }\widehat{V}^{\prime }\Xi ^{\prime }\left( \frac{\Gamma
^{\prime }\Gamma }{N_{1}}\right) \Xi \widehat{V}a  \notag \\
&\leq &\lambda _{\max }\left( \frac{\Gamma ^{\prime }\Gamma }{N_{1}}\right)
a^{\prime }\widehat{V}^{\prime }\Xi ^{\prime }\Xi \widehat{V}a  \notag \\
&=&\lambda _{\max }\left( \frac{\Gamma ^{\prime }\Gamma }{N_{1}}\right) 
\text{ }\left( \text{since }\Xi ^{\prime }\Xi =I_{Kp}\text{, }\widehat{V}%
^{\prime }\widehat{V}=I_{Kp}\text{, and }a^{\prime }a=1\right)  \notag \\
&\leq &C\text{ for all }N_{1},N_{2}\text{ sufficiently large.}
\label{bd aQQa}
\end{eqnarray}%
while, applying the triangle inequality and part (d) of Lemma D-16 allow us
to show that%
\begin{eqnarray*}
\left\Vert \frac{1}{T_{h}}\dsum\limits_{t=p}^{T-h}\underline{F}_{t}%
\underline{F}_{t}^{\prime }\right\Vert _{2} &\leq &\frac{1}{T_{h}}%
\dsum\limits_{t=p}^{T-h}\left\Vert \underline{F}_{t}\underline{F}%
_{t}^{\prime }\right\Vert _{2} \\
&=&\frac{1}{T_{h}}\dsum\limits_{t=p}^{T-h}\sqrt{\lambda _{\max }\left( 
\underline{F}_{t}\underline{F}_{t}^{\prime }\underline{F}_{t}\underline{F}%
_{t}^{\prime }\right) } \\
&=&\frac{1}{T_{h}}\dsum\limits_{t=p}^{T-h}\sqrt{\left[ \lambda _{\max
}\left( \underline{F}_{t}\underline{F}_{t}^{\prime }\right) \right] ^{2}} \\
&=&\frac{1}{T_{h}}\dsum\limits_{t=p}^{T-h}\sqrt{\left\Vert \underline{F}%
_{t}\right\Vert _{2}^{4}} \\
&=&\frac{1}{T_{h}}\dsum\limits_{t=p}^{T-h}\left\Vert \underline{F}%
_{t}\right\Vert _{2}^{2} \\
&=&O_{p}\left( 1\right)
\end{eqnarray*}%
Combining this result with part (g) of Lemma D-16 and the Slutsky's Theorem,
we deduce that%
\begin{eqnarray*}
&&\left\vert a^{\prime }Q^{\prime }\frac{1}{T_{h}}\dsum\limits_{t=p}^{T-h}%
\underline{F}_{t}\left( \widehat{\underline{F}}_{t}-Q^{\prime }\underline{F}%
_{t}\right) ^{\prime }b\right\vert \\
&\leq &\sqrt{a^{\prime }Q^{\prime }Qa\left\Vert \frac{1}{T_{h}}%
\dsum\limits_{t=p}^{T-h}\underline{F}_{t}\underline{F}_{t}^{\prime
}\right\Vert _{2}}\sqrt{\frac{1}{T_{h}}\dsum\limits_{t=p}^{T-h}b^{\prime
}\left( \widehat{\underline{F}}_{t}-Q^{\prime }\underline{F}_{t}\right)
\left( \widehat{\underline{F}}_{t}-Q^{\prime }\underline{F}_{t}\right)
^{\prime }b} \\
&=&o_{p}\left( 1\right)
\end{eqnarray*}%
Since this argument holds for all $a,b\in \mathbb{R}^{Kp}$ such that $%
\left\Vert a\right\Vert _{2}=\left\Vert b\right\Vert _{2}=1$, we further
obtain 
\begin{equation*}
Q^{\prime }\frac{1}{T_{h}}\dsum\limits_{t=p}^{T-h}\underline{F}_{t}\left( 
\widehat{\underline{F}}_{t}-Q^{\prime }\underline{F}_{t}\right) ^{\prime
}=o_{p}\left( 1\right)
\end{equation*}%
Now, given that%
\begin{equation*}
\frac{1}{T_{h}}\dsum\limits_{t=p}^{T-h}\left( \widehat{\underline{F}}%
_{t}-Q^{\prime }\underline{F}_{t}\right) \underline{F}_{t}^{\prime }Q=\left[
Q^{\prime }\frac{1}{T_{h}}\dsum\limits_{t=p}^{T-h}\underline{F}_{t}\left( 
\widehat{\underline{F}}_{t}-Q^{\prime }\underline{F}_{t}\right) ^{\prime }%
\right] ^{\prime }\text{,}
\end{equation*}%
a similar argument also shows that%
\begin{equation*}
\frac{1}{T_{h}}\dsum\limits_{t=p}^{T-h}\left( \widehat{\underline{F}}%
_{t}-Q^{\prime }\underline{F}_{t}\right) \underline{F}_{t}^{\prime
}Q=o_{p}\left( 1\right) \text{. }
\end{equation*}

\noindent Making use of part (b) of Lemma D-2 and the Slutsky's theorem, we
also see that%
\begin{equation*}
Q^{\prime }\left( \frac{1}{T_{h}}\dsum\limits_{t=p}^{T-h}\underline{F}_{t}%
\underline{F}_{t}^{\prime }-\frac{1}{T_{h}}\dsum\limits_{t=p}^{T-h}E\left[ 
\underline{F}_{t}\underline{F}_{t}^{\prime }\right] \right) Q\overset{p}{%
\rightarrow }0
\end{equation*}%
Putting these results together and apply Slutsky's theorem, we then obtain%
\begin{eqnarray*}
&&\frac{\widehat{\underline{F}}^{\prime }\widehat{\underline{F}}}{T_{h}}-%
\frac{1}{T_{h}}\dsum\limits_{t=p}^{T-h}Q^{\prime }E\left[ \underline{F}_{t}%
\underline{F}_{t}^{\prime }\right] Q \\
&=&\frac{1}{T_{h}}\dsum\limits_{t=p}^{T-h}\left( \widehat{\underline{F}}%
_{t}-Q^{\prime }\underline{F}_{t}\right) \left( \widehat{\underline{F}}%
_{t}-Q^{\prime }\underline{F}_{t}\right) ^{\prime }+Q^{\prime }\frac{1}{T_{h}%
}\dsum\limits_{t=p}^{T-h}\underline{F}_{t}\left( \widehat{\underline{F}}%
_{t}-Q^{\prime }\underline{F}_{t}\right) ^{\prime } \\
&&+\frac{1}{T_{h}}\dsum\limits_{t=p}^{T-h}\left( \widehat{\underline{F}}%
_{t}-Q^{\prime }\underline{F}_{t}\right) \underline{F}_{t}^{\prime
}Q+Q^{\prime }\left( \frac{1}{T_{h}}\dsum\limits_{t=p}^{T-h}\underline{F}_{t}%
\underline{F}_{t}^{\prime }-\frac{1}{T_{h}}\dsum\limits_{t=p}^{T-h}E\left[ 
\underline{F}_{t}\underline{F}_{t}^{\prime }\right] \right) Q \\
&=&o_{p}\left( 1\right)
\end{eqnarray*}

To show part (b), first write, for any $a\in \mathbb{R}^{Kp}$ and $b\in 
\mathbb{R}^{dp}$ such that $\left\Vert a\right\Vert _{2}=1$ and $\left\Vert
b\right\Vert _{2}=1$,%
\begin{eqnarray*}
&&\frac{a^{\prime }\widehat{\underline{F}}^{\prime }\underline{Y}b}{T_{h}}-%
\frac{1}{T_{h}}\dsum\limits_{t=p}^{T-h}a^{\prime }Q^{\prime }E\left[ 
\underline{F}_{t}\underline{Y}_{t}^{\prime }\right] b \\
&=&\frac{1}{T_{h}}\dsum\limits_{t=p}^{T-h}a^{\prime }\widehat{\underline{F}}%
_{t}\underline{Y}_{t}^{\prime }b-\frac{1}{T_{h}}\dsum\limits_{t=p}^{T-h}a^{%
\prime }Q^{\prime }E\left[ \underline{F}_{t}\underline{Y}_{t}^{\prime }%
\right] b \\
&=&\frac{1}{T_{h}}\dsum\limits_{t=p}^{T-h}a^{\prime }\left( \widehat{%
\underline{F}}_{t}-Q^{\prime }\underline{F}_{t}+Q^{\prime }\underline{F}%
_{t}\right) \underline{Y}_{t}^{\prime }b-\frac{1}{T_{h}}\dsum%
\limits_{t=p}^{T-h}a^{\prime }Q^{\prime }E\left[ \underline{F}_{t}\underline{%
Y}_{t}^{\prime }\right] b \\
&=&\frac{1}{T_{h}}\dsum\limits_{t=p}^{T-h}a^{\prime }\left( \widehat{%
\underline{F}}_{t}-Q^{\prime }\underline{F}_{t}\right) \underline{Y}%
_{t}^{\prime }b+a^{\prime }Q^{\prime }\left( \frac{1}{T_{h}}%
\dsum\limits_{t=p}^{T-h}\underline{F}_{t}\underline{Y}_{t}^{\prime }-\frac{1%
}{T_{h}}\dsum\limits_{t=p}^{T-h}E\left[ \underline{F}_{t}\underline{Y}%
_{t}^{\prime }\right] \right) b
\end{eqnarray*}%
Focusing first on the first term on last line above, we note that, 
\begin{eqnarray*}
&&\left\vert \frac{1}{T_{h}}\dsum\limits_{t=p}^{T-h}a^{\prime }\left( 
\widehat{\underline{F}}_{t}-Q^{\prime }\underline{F}_{t}\right) \underline{Y}%
_{t}^{\prime }b\right\vert \\
&\leq &\sqrt{a^{\prime }\frac{1}{T_{h}}\dsum\limits_{t=p}^{T-h}\left( 
\widehat{\underline{F}}_{t}-Q^{\prime }\underline{F}_{t}\right) \left( 
\widehat{\underline{F}}_{t}-Q^{\prime }\underline{F}_{t}\right) ^{\prime }a}%
\sqrt{\frac{1}{T_{h}}\dsum\limits_{t=p}^{T-h}b^{\prime }\underline{Y}_{t}%
\underline{Y}_{t}^{\prime }b} \\
&=&\sqrt{a^{\prime }\frac{1}{T_{h}}\dsum\limits_{t=p}^{T-h}\left( \widehat{%
\underline{F}}_{t}-Q^{\prime }\underline{F}_{t}\right) \left( \widehat{%
\underline{F}}_{t}-Q^{\prime }\underline{F}_{t}\right) ^{\prime }a} \\
&&\sqrt{b^{\prime }\left( \frac{1}{T_{h}}\dsum\limits_{t=p}^{T-h}\underline{Y%
}_{t}\underline{Y}_{t}^{\prime }-\frac{1}{T_{h}}\dsum\limits_{t=p}^{T-h}E%
\left[ \underline{Y}_{t}\underline{Y}_{t}^{\prime }\right] \right) b+\frac{1%
}{T_{h}}\dsum\limits_{t=p}^{T-h}b^{\prime }E\left[ \underline{Y}_{t}%
\underline{Y}_{t}^{\prime }\right] b} \\
&\leq &\sqrt{a^{\prime }\frac{1}{T_{h}}\dsum\limits_{t=p}^{T-h}\left( 
\widehat{\underline{F}}_{t}-Q^{\prime }\underline{F}_{t}\right) \left( 
\widehat{\underline{F}}_{t}-Q^{\prime }\underline{F}_{t}\right) ^{\prime }a}%
\sqrt[\backslash ]{b^{\prime }\left( \frac{1}{T_{h}}\dsum\limits_{t=p}^{T-h}%
\underline{Y}_{t}\underline{Y}_{t}^{\prime }-\frac{1}{T_{h}}%
\dsum\limits_{t=p}^{T-h}E\left[ \underline{Y}_{t}\underline{Y}_{t}^{\prime }%
\right] \right) b} \\
&&+\sqrt{a^{\prime }\frac{1}{T_{h}}\dsum\limits_{t=p}^{T-h}\left( \widehat{%
\underline{F}}_{t}-Q^{\prime }\underline{F}_{t}\right) \left( \widehat{%
\underline{F}}_{t}-Q^{\prime }\underline{F}_{t}\right) ^{\prime }a}\sqrt{%
\frac{1}{T_{h}}\dsum\limits_{t=p}^{T-h}b^{\prime }E\left[ \underline{Y}_{t}%
\underline{Y}_{t}^{\prime }\right] b}\text{ } \\
&&\left( \text{since }\sqrt{a_{1}+a_{2}}\leq \sqrt{a_{1}}+\sqrt{a_{2}}\right)
\\
&\leq &\sqrt{a^{\prime }\frac{1}{T_{h}}\dsum\limits_{t=p}^{T-h}\left( 
\widehat{\underline{F}}_{t}-Q^{\prime }\underline{F}_{t}\right) \left( 
\widehat{\underline{F}}_{t}-Q^{\prime }\underline{F}_{t}\right) ^{\prime }a}%
\sqrt[\backslash ]{b^{\prime }\left( \frac{1}{T_{h}}\dsum\limits_{t=p}^{T-h}%
\underline{Y}_{t}\underline{Y}_{t}^{\prime }-\frac{1}{T_{h}}%
\dsum\limits_{t=p}^{T-h}E\left[ \underline{Y}_{t}\underline{Y}_{t}^{\prime }%
\right] \right) b} \\
&&+\sqrt{a^{\prime }\frac{1}{T_{h}}\dsum\limits_{t=p}^{T-h}\left( \widehat{%
\underline{F}}_{t}-Q^{\prime }\underline{F}_{t}\right) \left( \widehat{%
\underline{F}}_{t}-Q^{\prime }\underline{F}_{t}\right) ^{\prime }a}\sqrt{%
\frac{1}{T_{h}}\dsum\limits_{t=r+1}^{T-h}E\left\Vert \underline{Y}%
_{t}\right\Vert _{2}^{2}} \\
&&\left( \text{since }b^{\prime }E\left[ \underline{Y}_{t}\underline{Y}%
_{t}^{\prime }\right] b=E\left[ \left( b^{\prime }\underline{Y}_{t}\right)
^{2}\right] \leq E\left[ b^{\prime }b\underline{Y}_{t}^{\prime }\underline{Y}%
_{t}\right] =E\left[ \left\Vert \underline{Y}_{t}\right\Vert _{2}^{2}\right]
\right) \\
&=&o_{p}\left( 1\right)
\end{eqnarray*}%
by part (b) of Lemma D-2 and parts (d) and (g) of Lemma D-16. In addition,
note that, by making use of part (b) of Lemma D-2, Assumption 3-6, and
Slutsky's theorem; we obtain 
\begin{eqnarray*}
&&\left\vert a^{\prime }Q^{\prime }\left( \frac{1}{T_{h}}\dsum%
\limits_{t=p}^{T-h}\underline{F}_{t}\underline{Y}_{t}^{\prime }-\frac{1}{%
T_{h}}\dsum\limits_{t=p}^{T-h}E\left[ \underline{F}_{t}\underline{Y}%
_{t}^{\prime }\right] \right) b\right\vert \\
&\leq &\sqrt{a^{\prime }Q^{\prime }Qa}\sqrt{b^{\prime }\left( \frac{1}{T_{h}}%
\dsum\limits_{t=p}^{T-h}\underline{F}_{t}\underline{Y}_{t}^{\prime }-\frac{1%
}{T_{h}}\dsum\limits_{t=p}^{T-h}E\left[ \underline{F}_{t}\underline{Y}%
_{t}^{\prime }\right] \right) ^{\prime }\left( \frac{1}{T_{h}}%
\dsum\limits_{t=p}^{T-h}\underline{F}_{t}\underline{Y}_{t}^{\prime }-\frac{1%
}{T_{h}}\dsum\limits_{t=p}^{T-h}E\left[ \underline{F}_{t}\underline{Y}%
_{t}^{\prime }\right] \right) b} \\
&\leq &\sqrt{\lambda _{\max }\left( \frac{\Gamma ^{\prime }\Gamma }{N_{1}}%
\right) }\sqrt{b^{\prime }\left( \frac{1}{T_{h}}\dsum\limits_{t=p}^{T-h}%
\underline{F}_{t}\underline{Y}_{t}^{\prime }-\frac{1}{T_{h}}%
\dsum\limits_{t=p}^{T-h}E\left[ \underline{F}_{t}\underline{Y}_{t}^{\prime }%
\right] \right) ^{\prime }\left( \frac{1}{T_{h}}\dsum\limits_{t=p}^{T-h}%
\underline{F}_{t}\underline{Y}_{t}^{\prime }-\frac{1}{T_{h}}%
\dsum\limits_{t=p}^{T-h}E\left[ \underline{F}_{t}\underline{Y}_{t}^{\prime }%
\right] \right) b} \\
&=&o_{p}\left( 1\right) \text{.}
\end{eqnarray*}%
Combining these results, we then get%
\begin{eqnarray*}
&&\left\vert \frac{a^{\prime }\widehat{\underline{F}}^{\prime }\underline{Y}b%
}{T_{h}}-\frac{1}{T_{h}}\dsum\limits_{t=p}^{T-h}a^{\prime }Q^{\prime }E\left[
\underline{F}_{t}\underline{Y}_{t}^{\prime }\right] b\right\vert \\
&\leq &\left\vert \frac{1}{T_{h}}\dsum\limits_{t=p}^{T-h}a^{\prime }\left( 
\widehat{\underline{F}}_{t}-Q^{\prime }\underline{F}_{t}\right) \underline{Y}%
_{t}^{\prime }b\right\vert +\left\vert a^{\prime }Q^{\prime }\left( \frac{1}{%
T_{h}}\dsum\limits_{t=p}^{T-h}\underline{F}_{t}\underline{Y}_{t}^{\prime }-%
\frac{1}{T_{h}}\dsum\limits_{t=p}^{T-h}E\left[ \underline{F}_{t}\underline{Y}%
_{t}^{\prime }\right] \right) b\right\vert \\
&=&o_{p}\left( 1\right)
\end{eqnarray*}

\noindent Since the above argument holds for all $a\in \mathbb{R}^{Kp}$ and $%
b\in \mathbb{R}^{dp}$ such that $\left\Vert a\right\Vert _{2}=1$ and $%
\left\Vert b\right\Vert _{2}=1$; we further deduce that%
\begin{equation*}
\frac{\widehat{\underline{F}}^{\prime }\underline{Y}}{T_{h}}-\frac{1}{T_{h}}%
\dsum\limits_{t=p}^{T-h}Q^{\prime }E\left[ \underline{F}_{t}\underline{Y}%
_{t}^{\prime }\right] =o_{p}\left( 1\right) \text{.}
\end{equation*}

To show part (c), first write, for any $a\in \mathbb{R}^{Kp}$ such that $%
\left\Vert a\right\Vert _{2}=1$,%
\begin{eqnarray*}
&&\frac{a^{\prime }\widehat{\underline{F}}^{\prime }\iota _{T_{h}}}{T_{h}}%
-a^{\prime }Q^{\prime }S_{K}^{\prime }\mathcal{P}_{\left( d{\LARGE +}%
K\right) p}\left( I_{\left( d{\LARGE +}K\right) p}-A\right) ^{-1}J_{d{\LARGE %
+}K}^{\prime }\mu \\
&=&\frac{1}{T_{h}}\dsum\limits_{t=p}^{T-h}a^{\prime }\widehat{\underline{F}}%
_{t}-a^{\prime }Q^{\prime }S_{K}^{\prime }\mathcal{P}_{\left( d{\LARGE +}%
K\right) p}\left( I_{\left( d{\LARGE +}K\right) p}-A\right) ^{-1}J_{d{\LARGE %
+}K}^{\prime }\mu \\
&=&\frac{1}{T_{h}}\dsum\limits_{t=p}^{T-h}a^{\prime }\left( \widehat{%
\underline{F}}_{t}-Q^{\prime }\underline{F}_{t}+Q^{\prime }\underline{F}%
_{t}\right) -a^{\prime }Q^{\prime }S_{K}^{\prime }\mathcal{P}_{\left( d%
{\LARGE +}K\right) p}\left( I_{\left( d{\LARGE +}K\right) p}-A\right)
^{-1}J_{d{\LARGE +}K}^{\prime }\mu \\
&=&\frac{1}{T_{h}}\dsum\limits_{t=p}^{T-h}a^{\prime }\left( \widehat{%
\underline{F}}_{t}-Q^{\prime }\underline{F}_{t}\right) +a^{\prime }Q^{\prime
}\left( \frac{1}{T_{h}}\dsum\limits_{t=p}^{T-h}\underline{F}%
_{t}-S_{K}^{\prime }\mathcal{P}_{\left( d{\LARGE +}K\right) p}\left(
I_{\left( d{\LARGE +}K\right) p}-A\right) ^{-1}J_{d{\LARGE +}K}^{\prime }\mu
\right)
\end{eqnarray*}%
Focusing first on the first term on last line above, we note that, 
\begin{eqnarray*}
\left\vert \frac{1}{T_{h}}\dsum\limits_{t=p}^{T-h}a^{\prime }\left( \widehat{%
\underline{F}}_{t}-Q^{\prime }\underline{F}_{t}\right) \right\vert &\leq &%
\frac{1}{T_{h}}\dsum\limits_{t=p}^{T-h}\left\vert a^{\prime }\left( \widehat{%
\underline{F}}_{t}-Q^{\prime }\underline{F}_{t}\right) \right\vert \text{ \ }%
\left( \text{by triangle inequality}\right) \\
&\leq &\sqrt{a^{\prime }\frac{1}{T_{h}}\dsum\limits_{t=p}^{T-h}\left( 
\widehat{\underline{F}}_{t}-Q^{\prime }\underline{F}_{t}\right) \left( 
\widehat{\underline{F}}_{t}-Q^{\prime }\underline{F}_{t}\right) ^{\prime }a}%
\text{ \ }\left( \text{by Liapunov's inequality}\right) \\
&\leq &\sqrt{\left\Vert \frac{1}{T_{h}}\dsum\limits_{t=p}^{T-h}\left( 
\widehat{\underline{F}}_{t}-Q^{\prime }\underline{F}_{t}\right) \left( 
\widehat{\underline{F}}_{t}-Q^{\prime }\underline{F}_{t}\right) ^{\prime
}\right\Vert _{2}} \\
&=&o_{p}\left( 1\right)
\end{eqnarray*}%
by part (g) of Lemma D-16 and Slutsky's theorem. In addition, note that, by
making use of part (d) of Lemma D-2, Assumption 3-6, and Slutsky's theorem;
we obtain 
\begin{eqnarray*}
&&\left\vert a^{\prime }Q^{\prime }\left( \frac{1}{T_{h}}\dsum%
\limits_{t=p}^{T-h}\underline{F}_{t}-S_{K}^{\prime }\mathcal{P}_{\left( d%
{\LARGE +}K\right) p}\left( I_{\left( d{\LARGE +}K\right) p}-A\right)
^{-1}J_{d{\LARGE +}K}^{\prime }\mu \right) \right\vert \\
&\leq &\sqrt{a^{\prime }Q^{\prime }Qa}\left[ \left( \frac{1}{T_{h}}%
\dsum\limits_{t=p}^{T-h}\underline{F}_{t}-S_{K}^{\prime }\mathcal{P}_{\left(
d{\LARGE +}K\right) p}\left( I_{\left( d{\LARGE +}K\right) p}-A\right)
^{-1}J_{d{\LARGE +}K}^{\prime }\mu \right) ^{\prime }\right. \\
&&\text{ \ \ \ \ \ \ \ \ \ \ \ \ \ \ }\left. \times \left( \frac{1}{T_{h}}%
\dsum\limits_{t=p}^{T-h}\underline{F}_{t}-S_{K}^{\prime }\mathcal{P}_{\left(
d{\LARGE +}K\right) p}\left( I_{\left( d{\LARGE +}K\right) p}-A\right)
^{-1}J_{d{\LARGE +}K}^{\prime }\mu \right) \right] ^{1/2} \\
&\leq &\sqrt{\lambda _{\max }\left( \frac{\Gamma ^{\prime }\Gamma }{N_{1}}%
\right) }\left[ \left( \frac{1}{T_{h}}\dsum\limits_{t=p}^{T-h}\underline{F}%
_{t}-S_{K}^{\prime }\mathcal{P}_{\left( d{\LARGE +}K\right) p}\left(
I_{\left( d{\LARGE +}K\right) p}-A\right) ^{-1}J_{d{\LARGE +}K}^{\prime }\mu
\right) ^{\prime }\right. \\
&&\text{ \ \ \ \ \ \ \ \ \ \ \ \ \ \ \ \ \ \ \ \ \ \ }\left. \times \left( 
\frac{1}{T_{h}}\dsum\limits_{t=p}^{T-h}\underline{F}_{t}-S_{K}^{\prime }%
\mathcal{P}_{\left( d{\LARGE +}K\right) p}\left( I_{\left( d{\LARGE +}%
K\right) p}-A\right) ^{-1}J_{d{\LARGE +}K}^{\prime }\mu \right) \right]
^{1/2} \\
&=&o_{p}\left( 1\right) \text{.}
\end{eqnarray*}%
Combining these results and applying Slutsky's theorem, we then get%
\begin{eqnarray*}
&&\left\vert \frac{a^{\prime }\widehat{\underline{F}}^{\prime }\iota _{T_{h}}%
}{T_{h}}-a^{\prime }Q^{\prime }S_{K}^{\prime }\mathcal{P}_{\left( d{\LARGE +}%
K\right) p}\left( I_{\left( d{\LARGE +}K\right) p}-A\right) ^{-1}J_{d{\LARGE %
+}K}^{\prime }\mu \right\vert \\
&\leq &\left\vert \frac{1}{T_{h}}\dsum\limits_{t=p}^{T-h}a^{\prime }\left( 
\widehat{\underline{F}}_{t}-Q^{\prime }\underline{F}_{t}\right) \right\vert
+\left\vert a^{\prime }Q^{\prime }\left( \frac{1}{T_{h}}\dsum%
\limits_{t=p}^{T-h}\underline{F}_{t}-S_{K}^{\prime }\mathcal{P}_{\left( d%
{\LARGE +}K\right) p}\left( I_{\left( d{\LARGE +}K\right) p}-A\right)
^{-1}J_{d{\LARGE +}K}^{\prime }\mu \right) \right\vert \\
&=&o_{p}\left( 1\right)
\end{eqnarray*}

\noindent Since the above argument holds for all $a\in \mathbb{R}^{Kp}$ such
that $\left\Vert a\right\Vert _{2}=1$; we further deduce that%
\begin{equation*}
\frac{\widehat{\underline{F}}^{\prime }\iota _{T_{h}}}{T_{h}}-Q^{\prime
}S_{K}^{\prime }\mathcal{P}_{\left( d{\LARGE +}K\right) p}\left( I_{\left( d%
{\LARGE +}K\right) p}-A\right) ^{-1}J_{d{\LARGE +}K}^{\prime }\mu
=o_{p}\left( 1\right) \text{.}
\end{equation*}

Turning our attention to part (d), note that for any $a\in \mathbb{R}^{Kp}$
and $b\in \mathbb{R}^{d}$ such that $\left\Vert a\right\Vert _{2}=1$ and $%
\left\Vert b\right\Vert _{2}=1$, we can write

\begin{eqnarray*}
&&\left\vert \text{ }\frac{a^{\prime }\underline{\widehat{F}}^{\prime
}\left( \underline{\widehat{F}}-\underline{F}Q\right) Q^{-1}B_{2}b}{T_{h}}%
\right\vert \\
&=&\left\vert \frac{1}{T_{h}}\dsum\limits_{t=p}^{T-h}a^{\prime }\widehat{%
\underline{F}}_{t}\left( \widehat{\underline{F}}_{t}^{\prime }-\underline{F}%
_{t}^{\prime }Q\right) Q^{-1}B_{2}b\right\vert \\
&\leq &\sqrt{a^{\prime }\left( \frac{1}{T_{h}}\dsum\limits_{t=p}^{T-h}%
\widehat{F}_{t}\widehat{F}_{t}^{\prime }\right) a}\sqrt{b^{\prime
}B_{2}^{\prime }Q^{\prime -1}\frac{1}{T_{h}}\dsum\limits_{t=p}^{T-h}\left( 
\widehat{\underline{F}}_{t}^{\prime }-\underline{F}_{t}^{\prime }Q\right)
^{\prime }\left( \widehat{\underline{F}}_{t}^{\prime }-\underline{F}%
_{t}^{\prime }Q\right) Q^{-1}B_{2}b} \\
&\leq &\sqrt{\left\vert a^{\prime }\left( \frac{1}{T_{h}}\dsum%
\limits_{t=p}^{T-h}\widehat{F}_{t}\widehat{F}_{t}^{\prime }-\frac{1}{T_{h}}%
\dsum\limits_{t=p}^{T-h}Q^{\prime }E\left[ \underline{F}_{t}\underline{F}%
_{t}^{\prime }\right] Q\right) a\right\vert +\frac{1}{T_{h}}%
\dsum\limits_{t=p}^{T-h}a^{\prime }Q^{\prime }E\left[ \underline{F}_{t}%
\underline{F}_{t}^{\prime }\right] Qa} \\
&&\times \sqrt{b^{\prime }B_{2}^{\prime }Q^{\prime -1}\frac{1}{T_{h}}%
\dsum\limits_{t=p}^{T-h}\left( \widehat{\underline{F}}_{t}^{\prime }-%
\underline{F}_{t}^{\prime }Q\right) ^{\prime }\left( \widehat{\underline{F}}%
_{t}^{\prime }-\underline{F}_{t}^{\prime }Q\right) Q^{-1}B_{2}b} \\
&\leq &\left\{ \sqrt{\left\vert a^{\prime }\left( \frac{1}{T_{h}}%
\dsum\limits_{t=p}^{T-h}\widehat{F}_{t}\widehat{F}_{t}^{\prime }-\frac{1}{%
T_{h}}\dsum\limits_{t=p}^{T-h}Q^{\prime }E\left[ \underline{F}_{t}\underline{%
F}_{t}^{\prime }\right] Q\right) a\right\vert }\right. \\
&&\left. \times \sqrt{b^{\prime }B_{2}^{\prime }Q^{\prime -1}\frac{1}{T_{h}}%
\dsum\limits_{t=p}^{T-h}\left( \widehat{\underline{F}}_{t}^{\prime }-%
\underline{F}_{t}^{\prime }Q\right) ^{\prime }\left( \widehat{\underline{F}}%
_{t}^{\prime }-\underline{F}_{t}^{\prime }Q\right) Q^{-1}B_{2}b}\right\} \\
&&+\sqrt{\frac{1}{T_{h}}\dsum\limits_{t=p}^{T-h}a^{\prime }Q^{\prime }E\left[
\underline{F}_{t}\underline{F}_{t}^{\prime }\right] Qa}\sqrt{b^{\prime
}B_{2}^{\prime }Q^{\prime -1}\frac{1}{T_{h}}\dsum\limits_{t=p}^{T-h}\left( 
\widehat{\underline{F}}_{t}^{\prime }-\underline{F}_{t}^{\prime }Q\right)
^{\prime }\left( \widehat{\underline{F}}_{t}^{\prime }-\underline{F}%
_{t}^{\prime }Q\right) Q^{-1}B_{2}b} \\
&&\left( \text{using the inequality }\sqrt{a_{1}+a_{2}}\leq \sqrt{a_{1}}+%
\sqrt{a_{2}}\text{ for }a_{1}\geq 0\text{ and }a_{2}\geq 0\right) \\
&\leq &\left\{ \sqrt{\left\vert a^{\prime }\left( \frac{1}{T_{h}}%
\dsum\limits_{t=p}^{T-h}\widehat{F}_{t}\widehat{F}_{t}^{\prime }-\frac{1}{%
T_{h}}\dsum\limits_{t=p}^{T-h}Q^{\prime }E\left[ \underline{F}_{t}\underline{%
F}_{t}^{\prime }\right] Q\right) a\right\vert }\right. \\
&&\left. \times \sqrt{\left\Vert \frac{1}{T_{h}}\dsum\limits_{t=p}^{T-h}%
\left( \widehat{\underline{F}}_{t}^{\prime }-\underline{F}_{t}^{\prime
}Q\right) ^{\prime }\left( \widehat{\underline{F}}_{t}^{\prime }-\underline{F%
}_{t}^{\prime }Q\right) \right\Vert _{2}b^{\prime }B_{2}^{\prime }Q^{\prime
-1}Q^{-1}B_{2}b}\right\} \\
&&+\sqrt{a^{\prime }Q^{\prime }Qa\frac{1}{T_{h}}\dsum\limits_{t=p}^{T-h}E%
\left[ \left\Vert \underline{F}_{t}\right\Vert _{2}^{2}\right] }\sqrt{%
\left\Vert \frac{1}{T_{h}}\dsum\limits_{t=p}^{T-h}\left( \widehat{\underline{%
F}}_{t}^{\prime }-\underline{F}_{t}^{\prime }Q\right) ^{\prime }\left( 
\widehat{\underline{F}}_{t}^{\prime }-\underline{F}_{t}^{\prime }Q\right)
\right\Vert _{2}b^{\prime }B_{2}^{\prime }Q^{\prime -1}Q^{-1}B_{2}b} \\
&&\left( \text{since for a symmetric psd matrix }A\text{, }\left\Vert
A\right\Vert _{2}=\sqrt{\lambda _{\max }\left( A^{\prime }A\right) }=\sqrt{%
\lambda _{\max }\left( A^{2}\right) }=\sqrt{\left[ \lambda _{\max }\left(
A\right) \right] ^{2}}\right. \\
&&\left. =\lambda _{\max }\left( A\right) \text{ and since }a^{\prime
}Q^{\prime }E\left[ \underline{F}_{t}\underline{F}_{t}^{\prime }\right] Qa=E%
\left[ \left( a^{\prime }Q^{\prime }\underline{F}_{t}\right) ^{2}\right]
\leq E\left[ a^{\prime }Q^{\prime }Qa\underline{F}_{t}^{\prime }\underline{F}%
_{t}\right] =a^{\prime }Q^{\prime }QaE\left[ \left\Vert \underline{F}%
_{t}\right\Vert _{2}^{2}\right] \right)
\end{eqnarray*}%
\begin{eqnarray*}
&=&\left\{ \sqrt{\left\vert a^{\prime }\left( \frac{1}{T_{h}}%
\dsum\limits_{t=p}^{T-h}\widehat{F}_{t}\widehat{F}_{t}^{\prime }-\frac{1}{%
T_{h}}\dsum\limits_{t=p}^{T-h}Q^{\prime }E\left[ \underline{F}_{t}\underline{%
F}_{t}^{\prime }\right] Q\right) a\right\vert }\right. \\
&&\left. \times \sqrt{\left\Vert \frac{1}{T_{h}}\dsum\limits_{t=p}^{T-h}%
\left( \widehat{\underline{F}}_{t}-Q^{\prime }\underline{F}_{t}\right)
\left( \widehat{\underline{F}}_{t}-Q^{\prime }\underline{F}_{t}\right)
^{\prime }\right\Vert _{2}b^{\prime }B_{2}^{\prime }Q^{\prime -1}Q^{-1}B_{2}b%
}\right\} \\
&&+\sqrt{a^{\prime }Q^{\prime }Qa\frac{1}{T_{h}}\dsum\limits_{t=p}^{T-h}E%
\left[ \left\Vert \underline{F}_{t}\right\Vert _{2}^{2}\right] }\sqrt{%
\left\Vert \frac{1}{T_{h}}\dsum\limits_{t=p}^{T-h}\left( \widehat{\underline{%
F}}_{t}-Q^{\prime }\underline{F}_{t}\right) \left( \widehat{\underline{F}}%
_{t}-Q^{\prime }\underline{F}_{t}\right) ^{\prime }\right\Vert _{2}b^{\prime
}B_{2}^{\prime }Q^{\prime -1}Q^{-1}B_{2}b}
\end{eqnarray*}%
Now, by part (a) of this lemma and Slutsky's theorem, we have%
\begin{equation}
\left\vert a^{\prime }\left( \frac{1}{T_{h}}\dsum\limits_{t=p}^{T-h}\widehat{%
F}_{t}\widehat{F}_{t}^{\prime }-\frac{1}{T_{h}}\dsum\limits_{t=p}^{T-h}Q^{%
\prime }E\left[ \underline{F}_{t}\underline{F}_{t}^{\prime }\right] Q\right)
a\right\vert =o_{p}\left( 1\right)  \label{convergence avg FF}
\end{equation}%
Note also that%
\begin{eqnarray}
\frac{1}{T_{h}}\dsum\limits_{t=p}^{T-h}E\left[ \left\Vert \underline{F}%
_{t}\right\Vert _{2}^{2}\right] &\leq &\frac{1}{T_{h}}\dsum%
\limits_{t=p}^{T-h}\left( E\left[ \left\Vert \underline{F}_{t}\right\Vert
_{2}^{6}\right] \right) ^{\frac{{\Large 1}}{{\Large 3}}}\text{ }\left( \text{%
by Liapunov's inequality}\right)  \notag \\
&\leq &\frac{1}{T_{h}}\dsum\limits_{t=p}^{T-h}\left( \overline{C}\right) ^{%
\frac{{\Large 1}}{{\Large 3}}}\text{ \ }\left( \text{by Lemma C-5}\right) 
\notag \\
&=&\left( \overline{C}\right) ^{\frac{{\Large 1}}{{\Large 3}}}\text{.}
\label{bd EF^2}
\end{eqnarray}%
In addition, note that, by Assumption 3-7, there exists a positive constant $%
C$ such that%
\begin{eqnarray*}
&&\lambda _{\max }\left( B_{2}^{\prime }B_{2}\right) \\
&=&\lambda _{\max }\left( J_{d}A^{h}\mathcal{P}_{\left( d{\LARGE +}K\right)
p}^{\prime }S_{K}S_{K}^{\prime }\mathcal{P}_{\left( d{\LARGE +}K\right)
p}\left( A^{h}\right) ^{\prime }J_{d}^{\prime }\right) \\
&\leq &\lambda _{\max }\left( S_{K}S_{K}^{\prime }\right) \lambda _{\max
}\left( \mathcal{P}_{\left( d{\LARGE +}K\right) p}^{\prime }\mathcal{P}%
_{\left( d{\LARGE +}K\right) p}\right) \lambda _{\max }\left\{ A^{h}\left(
A^{h}\right) ^{\prime }\right\} \lambda _{\max }\left( J_{d}J_{d}^{\prime
}\right) \\
&=&\lambda _{\max }\left( S_{K}S_{K}^{\prime }\right) \lambda _{\max
}\left\{ A^{h}\left( A^{h}\right) ^{\prime }\right\} \text{ }\left( \text{%
since }\mathcal{P}_{\left( d{\LARGE +}K\right) p}^{\prime }\mathcal{P}%
_{\left( d{\LARGE +}K\right) p}=I_{\left( d{\LARGE +}K\right) p}\text{ and }%
J_{d}J_{d}^{\prime }=I_{d}\right. \text{ } \\
&&\text{ \ \ \ \ \ \ \ \ \ \ \ \ \ \ \ \ \ \ \ \ \ \ \ \ \ \ \ \ \ \ \ \ \ \
\ \ \ \ \ \ \ \ \ \ \ \ \ }\left. \text{so }\lambda _{\max }\left( \mathcal{P%
}_{\left( d{\LARGE +}K\right) p}^{\prime }\mathcal{P}_{\left( d{\LARGE +}%
K\right) p}\right) =\lambda _{\max }\left( J_{d}J_{d}^{\prime }\right)
=1\right) \\
&=&\lambda _{\max }\left( S_{K}^{\prime }S_{K}\right) \lambda _{\max
}\left\{ \left( A^{h}\right) ^{\prime }A^{h}\right\} \\
&=&\lambda _{\max }\left\{ \left( A^{h}\right) ^{\prime }A^{h}\right\} \text{
}\left( \text{since }S_{K}^{\prime }S_{K}=I_{Kp}\text{ so }\lambda _{\max
}\left( S_{K}^{\prime }S_{K}\right) =1\right) \\
&=&\sigma _{\max }^{2}\left( A^{h}\right) \\
&\leq &C\max \left\{ \left\vert \lambda _{\max }\left( A^{h}\right)
\right\vert ^{2},\left\vert \lambda _{\min }\left( A^{h}\right) \right\vert
^{2}\right\} \text{ }\left( \text{by Assumption 3-7}\right) \\
&=&C\max \left\{ \left\vert \lambda _{\max }\left( A\right) \right\vert
^{2h},\left\vert \lambda _{\min }\left( A\right) \right\vert ^{2h}\right\} \\
&=&C\phi _{\max }^{2h} \\
&<&C\text{ for integer }h\geq 1\text{,}
\end{eqnarray*}%
where $\phi _{\max }=\max \left\{ \left\vert \lambda _{\max }\left( A\right)
\right\vert ,\left\vert \lambda _{\min }\left( A\right) \right\vert \right\} 
$ and where the last equality follows from the fact that $0<\phi _{\max }<1$
given that Assumption 3-1 implies that all eigenvalues of $A$ have modulus
less than $1$. The boundedness of $\lambda _{\max }\left( B_{2}^{\prime
}B_{2}\right) $ allows us to further deduce that 
\begin{eqnarray}
&&b^{\prime }B_{2}^{\prime }Q^{\prime -1}Q^{-1}B_{2}b  \notag \\
&=&b^{\prime }B_{2}^{\prime }\left( \frac{\Gamma ^{\prime }\Gamma }{N_{1}}%
\right) ^{-1/2}\Xi \widehat{V}\widehat{V}^{\prime }\Xi ^{\prime }\left( 
\frac{\Gamma ^{\prime }\Gamma }{N_{1}}\right) ^{-1/2}B_{2}b  \notag \\
&=&b^{\prime }B_{2}^{\prime }\left( \frac{\Gamma ^{\prime }\Gamma }{N_{1}}%
\right) ^{-1}B_{2}b  \notag \\
&\leq &\left[ \lambda _{\min }\left( \frac{\Gamma ^{\prime }\Gamma }{N_{1}}%
\right) \right] ^{-1}b^{\prime }B_{2}^{\prime }B_{2}b  \notag \\
&\leq &\left[ \lambda _{\min }\left( \frac{\Gamma ^{\prime }\Gamma }{N_{1}}%
\right) \right] ^{-1}\lambda _{\max }\left( B_{2}^{\prime }B_{2}\right)
b^{\prime }b  \notag \\
&=&\left[ \lambda _{\min }\left( \frac{\Gamma ^{\prime }\Gamma }{N_{1}}%
\right) \right] ^{-1}\lambda _{\max }\left( B_{2}^{\prime }B_{2}\right) 
\notag \\
&\leq &C^{\ast }<\infty \text{ }  \label{quad form HinvB2}
\end{eqnarray}%
for some positive constant $C^{\ast }$ in light of Assumption 3-6. It
follows by applying expression (\ref{bd aQQa}) in the proof for part (a),
expressions (\ref{convergence avg FF})-(\ref{quad form HinvB2}) here, as
well as the result given in part (g) of Lemma D-16 and the Slutsky' theorem
that%
\begin{eqnarray*}
&&\left\vert \text{ }\frac{a^{\prime }\underline{\widehat{F}}^{\prime
}\left( \underline{\widehat{F}}-\underline{F}Q\right) Q^{-1}B_{2}b}{T_{h}}%
\right\vert \\
&\leq &\left\{ \sqrt{\left\vert a^{\prime }\left( \frac{1}{T_{h}}%
\dsum\limits_{t=p}^{T-h}\widehat{F}_{t}\widehat{F}_{t}^{\prime }-\frac{1}{%
T_{h}}\dsum\limits_{t=p}^{T-h}Q^{\prime }E\left[ \underline{F}_{t}\underline{%
F}_{t}^{\prime }\right] Q\right) a\right\vert }\right. \\
&&\left. \times \sqrt{\left\Vert \frac{1}{T_{h}}\dsum\limits_{t=p}^{T-h}%
\left( \widehat{\underline{F}}_{t}-Q^{\prime }\underline{F}_{t}\right)
\left( \widehat{\underline{F}}_{t}-Q^{\prime }\underline{F}_{t}\right)
^{\prime }\right\Vert _{2}b^{\prime }B_{2}^{\prime }Q^{\prime -1}Q^{-1}B_{2}b%
}\right\} \\
&&+\sqrt{a^{\prime }Q^{\prime }Qa\frac{1}{T_{h}}\dsum\limits_{t=p}^{T-h}E%
\left[ \left\Vert \underline{F}_{t}\right\Vert _{2}^{2}\right] }\sqrt{%
\left\Vert \frac{1}{T_{h}}\dsum\limits_{t=p}^{T-h}\left( \widehat{\underline{%
F}}_{t}-Q^{\prime }\underline{F}_{t}\right) \left( \widehat{\underline{F}}%
_{t}-Q^{\prime }\underline{F}_{t}\right) ^{\prime }\right\Vert _{2}b^{\prime
}B_{2}^{\prime }Q^{\prime -1}Q^{-1}B_{2}b} \\
&=&o_{p}\left( 1\right) \text{. }
\end{eqnarray*}%
Since the above argument holds for all $a\in \mathbb{R}^{Kp}$ and $b\in 
\mathbb{R}^{d}$ such that $\left\Vert a\right\Vert _{2}=1$ and $\left\Vert
b\right\Vert _{2}=1$, we further deduce that%
\begin{equation*}
\frac{\underline{\widehat{F}}^{\prime }\left( \underline{\widehat{F}}-%
\underline{F}Q\right) Q^{-1}B_{2}}{T_{h}}=o_{p}\left( 1\right) \text{.}
\end{equation*}

To show part (e), note that for any $a\in \mathbb{R}^{dp}$ and $b\in \mathbb{%
R}^{d}$ such that $\left\Vert a\right\Vert _{2}=1$ and $\left\Vert
b\right\Vert _{2}=1$, we can write

\begin{eqnarray*}
&&\left\vert \text{ }\frac{a^{\prime }\underline{Y}^{\prime }\left( 
\underline{\widehat{F}}-\underline{F}Q\right) Q^{-1}B_{2}b}{T_{h}}\right\vert
\\
&=&\left\vert \frac{1}{T_{h}}\dsum\limits_{t=p}^{T-h}a^{\prime }\underline{Y}%
_{t}\left( \widehat{\underline{F}}_{t}^{\prime }-\underline{F}_{t}^{\prime
}Q\right) Q^{-1}B_{2}b\right\vert \\
&\leq &\sqrt{a^{\prime }\left( \frac{1}{T_{h}}\dsum\limits_{t=p}^{T-h}%
\underline{Y}_{t}\underline{Y}_{t}^{\prime }\right) a}\sqrt{b^{\prime
}B_{2}^{\prime }Q^{\prime -1}\frac{1}{T_{h}}\dsum\limits_{t=p}^{T-h}\left( 
\widehat{\underline{F}}_{t}^{\prime }-\underline{F}_{t}^{\prime }Q\right)
^{\prime }\left( \widehat{\underline{F}}_{t}^{\prime }-\underline{F}%
_{t}^{\prime }Q\right) Q^{-1}B_{2}b} \\
&\leq &\left\{ \sqrt{\left\vert a^{\prime }\left( \frac{1}{T_{h}}%
\dsum\limits_{t=p}^{T-h}\underline{Y}_{t}\underline{Y}_{t}^{\prime }-\frac{1%
}{T_{h}}\dsum\limits_{t=p}^{T-h}E\left[ \underline{Y}_{t}\underline{Y}%
_{t}^{\prime }\right] \right) a\right\vert +\left\vert \frac{1}{T_{h}}%
\dsum\limits_{t=p}^{T-h}a^{\prime }E\left[ \underline{Y}_{t}\underline{Y}%
_{t}^{\prime }\right] a\right\vert }\right. \\
&&\left. \times \sqrt{b^{\prime }B_{2}^{\prime }Q^{\prime -1}\frac{1}{T_{h}}%
\dsum\limits_{t=p}^{T-h}\left( \widehat{\underline{F}}_{t}^{\prime }-%
\underline{F}_{t}^{\prime }Q\right) ^{\prime }\left( \widehat{\underline{F}}%
_{t}^{\prime }-\underline{F}_{t}^{\prime }Q\right) Q^{-1}B_{2}b}\right\} \\
&\leq &\left\{ \sqrt{\left\vert a^{\prime }\left( \frac{1}{T_{h}}%
\dsum\limits_{t=p}^{T-h}\underline{Y}_{t}\underline{Y}_{t}^{\prime }-\frac{1%
}{T_{h}}\dsum\limits_{t=p}^{T-h}E\left[ \underline{Y}_{t}\underline{Y}%
_{t}^{\prime }\right] \right) a\right\vert }\right. \\
&&\left. \times \sqrt{b^{\prime }B_{2}^{\prime }Q^{\prime -1}\frac{1}{T_{h}}%
\dsum\limits_{t=p}^{T-h}\left( \widehat{\underline{F}}_{t}^{\prime }-%
\underline{F}_{t}^{\prime }Q\right) ^{\prime }\left( \widehat{\underline{F}}%
_{t}^{\prime }-\underline{F}_{t}^{\prime }Q\right) Q^{-1}B_{2}b}\right\} \\
&&+\sqrt{\frac{1}{T_{h}}\dsum\limits_{t=p}^{T-h}a^{\prime }E\left[ 
\underline{Y}_{t}\underline{Y}_{t}^{\prime }\right] a}\sqrt{b^{\prime
}B_{2}^{\prime }Q^{\prime -1}\frac{1}{T_{h}}\dsum\limits_{t=p}^{T-h}\left( 
\widehat{\underline{F}}_{t}^{\prime }-\underline{F}_{t}^{\prime }Q\right)
^{\prime }\left( \widehat{\underline{F}}_{t}^{\prime }-\underline{F}%
_{t}^{\prime }Q\right) Q^{-1}B_{2}b} \\
&&\left( \text{using the inequality }\sqrt{a_{1}+a_{2}}\leq \sqrt{a_{1}}+%
\sqrt{a_{2}}\text{ for }a_{1}\geq 0\text{ and }a_{2}\geq 0\right)
\end{eqnarray*}%
\begin{eqnarray*}
&\leq &\left\{ \sqrt{\left\vert a^{\prime }\left( \frac{1}{T_{h}}%
\dsum\limits_{t=p}^{T-h}\underline{Y}_{t}\underline{Y}_{t}^{\prime }-\frac{1%
}{T_{h}}\dsum\limits_{t=p}^{T-h}E\left[ \underline{Y}_{t}\underline{Y}%
_{t}^{\prime }\right] \right) a\right\vert }\right. \\
&&\left. \times \sqrt{\left\Vert \frac{1}{T_{h}}\dsum\limits_{t=p}^{T-h}%
\left( \widehat{\underline{F}}_{t}^{\prime }-\underline{F}_{t}^{\prime
}Q\right) ^{\prime }\left( \widehat{\underline{F}}_{t}^{\prime }-\underline{F%
}_{t}^{\prime }Q\right) \right\Vert _{2}b^{\prime }B_{2}^{\prime }Q^{\prime
-1}Q^{-1}B_{2}b}\right\} \\
&&+\sqrt{a^{\prime }a\frac{1}{T_{h}}\dsum\limits_{t=p}^{T-h}E\left[
\left\Vert \underline{Y}_{t}\right\Vert _{2}^{2}\right] }\sqrt{\left\Vert 
\frac{1}{T_{h}}\dsum\limits_{t=p}^{T-h}\left( \widehat{\underline{F}}%
_{t}^{\prime }-\underline{F}_{t}^{\prime }Q\right) ^{\prime }\left( \widehat{%
\underline{F}}_{t}^{\prime }-\underline{F}_{t}^{\prime }Q\right) \right\Vert
_{2}b^{\prime }B_{2}^{\prime }Q^{\prime -1}Q^{-1}B_{2}b} \\
&&\left( \text{since for a symmetric psd matrix }A\text{, }\left\Vert
A\right\Vert _{2}=\sqrt{\lambda _{\max }\left( A^{\prime }A\right) }=\sqrt{%
\lambda _{\max }\left( A^{2}\right) }=\sqrt{\left[ \lambda _{\max }\left(
A\right) \right] ^{2}}\right. \\
&&\left. =\lambda _{\max }\left( A\right) \text{ and since }a^{\prime }E%
\left[ \underline{Y}_{t}\underline{Y}_{t}^{\prime }\right] a=E\left[ \left(
a^{\prime }\underline{Y}_{t}\right) ^{2}\right] \leq E\left[ a^{\prime }a%
\underline{Y}_{t}^{\prime }\underline{Y}_{t}\right] =E\left[ \left\Vert 
\underline{Y}_{t}\right\Vert _{2}^{2}\right] \right) \\
&=&\left\{ \sqrt{\left\vert a^{\prime }\left( \frac{1}{T_{h}}%
\dsum\limits_{t=p}^{T-h}\underline{Y}_{t}\underline{Y}_{t}^{\prime }-\frac{1%
}{T_{h}}\dsum\limits_{t=p}^{T-h}E\left[ \underline{Y}_{t}\underline{Y}%
_{t}^{\prime }\right] \right) a\right\vert }\right. \\
&&\left. \times \sqrt{\left\Vert \frac{1}{T_{h}}\dsum\limits_{t=p}^{T-h}%
\left( \widehat{\underline{F}}_{t}-Q^{\prime }\underline{F}_{t}\right)
\left( \widehat{\underline{F}}_{t}-Q^{\prime }\underline{F}_{t}\right)
^{\prime }\right\Vert _{2}b^{\prime }B_{2}^{\prime }Q^{\prime -1}Q^{-1}B_{2}b%
}\right\} \\
&&+\sqrt{\frac{1}{T_{h}}\dsum\limits_{t=p}^{T-h}E\left[ \left\Vert 
\underline{Y}_{t}\right\Vert _{2}^{2}\right] }\sqrt{\left\Vert \frac{1}{T_{h}%
}\dsum\limits_{t=p}^{T-h}\left( \widehat{\underline{F}}_{t}-Q^{\prime }%
\underline{F}_{t}\right) \left( \widehat{\underline{F}}_{t}-Q^{\prime }%
\underline{F}_{t}\right) ^{\prime }\right\Vert _{2}b^{\prime }B_{2}^{\prime
}Q^{\prime -1}Q^{-1}B_{2}b} \\
&&\left( \text{since }\frac{1}{T_{h}}\dsum\limits_{t=p}^{T-h}\left( \widehat{%
\underline{F}}_{t}^{\prime }-\underline{F}_{t}^{\prime }Q\right) ^{\prime
}\left( \widehat{\underline{F}}_{t}^{\prime }-\underline{F}_{t}^{\prime
}Q\right) =\frac{1}{T_{h}}\dsum\limits_{t=p}^{T-h}\left( \widehat{\underline{%
F}}_{t}-Q^{\prime }\underline{F}_{t}\right) \left( \widehat{\underline{F}}%
_{t}-Q^{\prime }\underline{F}_{t}\right) ^{\prime }\right)
\end{eqnarray*}%
Now, by part (b) of Lemma D-2 and Slutsky's theorem, we have%
\begin{equation}
\left\vert a^{\prime }\left( \frac{1}{T_{h}}\dsum\limits_{t=p}^{T-h}%
\underline{Y}_{t}\underline{Y}_{t}^{\prime }-\frac{1}{T_{h}}%
\dsum\limits_{t=p}^{T-h}E\left[ \underline{Y}_{t}\underline{Y}_{t}^{\prime }%
\right] \right) a\right\vert =O_{p}\left( \frac{1}{\sqrt{T}}\right)
=o_{p}\left( 1\right)  \label{convergence avg YY}
\end{equation}%
Note also that%
\begin{eqnarray}
\frac{1}{T_{h}}\dsum\limits_{t=p}^{T-h}E\left[ \left\Vert \underline{Y}%
_{t}\right\Vert _{2}^{2}\right] &\leq &\frac{1}{T_{h}}\dsum%
\limits_{t=p}^{T-h}\left( E\left[ \left\Vert \underline{Y}_{t}\right\Vert
_{2}^{6}\right] \right) ^{\frac{{\Large 1}}{{\Large 3}}}\text{ }\left( \text{%
by Liapunov's inequality}\right)  \notag \\
&\leq &\frac{1}{T_{h}}\dsum\limits_{t=p}^{T-h}\left( \overline{C}\right) ^{%
\frac{{\Large 1}}{{\Large 3}}}\text{ \ }\left( \text{by Lemma C-5}\right) 
\notag \\
&=&\left( \overline{C}\right) ^{\frac{{\Large 1}}{{\Large 3}}}\text{.}
\label{bd EY^2}
\end{eqnarray}%
It follows by applying expressions (\ref{quad form HinvB2}), (\ref%
{convergence avg YY}), and (\ref{bd EY^2}) as well as the result given in
part (g) of Lemma D-16 and the Slutsky' theorem that%
\begin{eqnarray*}
&&\left\vert \text{ }\frac{a^{\prime }\underline{Y}^{\prime }\left( 
\underline{\widehat{F}}-\underline{F}Q\right) Q^{-1}B_{2}b}{T_{h}}\right\vert
\\
&\leq &\left\{ \sqrt{\left\vert a^{\prime }\left( \frac{1}{T_{h}}%
\dsum\limits_{t=p}^{T-h}\underline{Y}_{t}\underline{Y}_{t}^{\prime }-\frac{1%
}{T_{h}}\dsum\limits_{t=p}^{T-h}E\left[ \underline{Y}_{t}\underline{Y}%
_{t}^{\prime }\right] \right) a\right\vert }\right. \\
&&\left. \times \sqrt{\left\Vert \frac{1}{T_{h}}\dsum\limits_{t=p}^{T-h}%
\left( \widehat{\underline{F}}_{t}-Q^{\prime }\underline{F}_{t}\right)
\left( \widehat{\underline{F}}_{t}-Q^{\prime }\underline{F}_{t}\right)
^{\prime }\right\Vert _{2}b^{\prime }B_{2}^{\prime }Q^{\prime -1}Q^{-1}B_{2}b%
}\right\} \\
&&+\sqrt{\frac{1}{T_{h}}\dsum\limits_{t=p}^{T-h}E\left[ \left\Vert 
\underline{Y}_{t}\right\Vert _{2}^{2}\right] }\sqrt{\left\Vert \frac{1}{T_{h}%
}\dsum\limits_{t=p}^{T-h}\left( \widehat{\underline{F}}_{t}-Q^{\prime }%
\underline{F}_{t}\right) \left( \widehat{\underline{F}}_{t}-Q^{\prime }%
\underline{F}_{t}\right) ^{\prime }\right\Vert _{2}b^{\prime }B_{2}^{\prime
}Q^{\prime -1}Q^{-1}B_{2}b} \\
&=&o_{p}\left( 1\right) \text{. }
\end{eqnarray*}%
Since the above argument holds for all $a\in \mathbb{R}^{dp}$ and $b\in 
\mathbb{R}^{d}$ such that $\left\Vert a\right\Vert _{2}=1$ and $\left\Vert
b\right\Vert _{2}=1$, we further deduce that%
\begin{equation*}
\frac{\underline{Y}^{\prime }\left( \underline{\widehat{F}}-\underline{F}%
Q\right) Q^{-1}B_{2}}{T_{h}}=o_{p}\left( 1\right) \text{.}
\end{equation*}

To show part (f), note that for any $b\in \mathbb{R}^{d}$ such that $%
\left\Vert b\right\Vert _{2}=1$, we can write

\begin{eqnarray*}
&&\left\vert \text{ }\frac{\iota _{T_{h}}^{\prime }\left( \underline{%
\widehat{F}}-\underline{F}Q\right) Q^{-1}B_{2}b}{T_{h}}\right\vert \\
&=&\left\vert \frac{1}{T_{h}}\dsum\limits_{t=p}^{T-h}\left( \widehat{%
\underline{F}}_{t}^{\prime }-\underline{F}_{t}^{\prime }Q\right)
Q^{-1}B_{2}b\right\vert \\
&\leq &\frac{1}{T_{h}}\dsum\limits_{t=p}^{T-h}\left\vert \left( \widehat{%
\underline{F}}_{t}^{\prime }-\underline{F}_{t}^{\prime }Q\right)
Q^{-1}B_{2}b\right\vert \text{ \ }\left( \text{by triangle inequality}\right)
\\
&\leq &\sqrt{b^{\prime }B_{2}^{\prime }Q^{\prime -1}\frac{1}{T_{h}}%
\dsum\limits_{t=p}^{T-h}\left( \widehat{\underline{F}}_{t}^{\prime }-%
\underline{F}_{t}^{\prime }Q\right) ^{\prime }\left( \widehat{\underline{F}}%
_{t}^{\prime }-\underline{F}_{t}^{\prime }Q\right) Q^{-1}B_{2}b}\text{ \ \ }%
\left( \text{by Liapunov's inequality}\right) \\
&\leq &\sqrt{\left\Vert \frac{1}{T_{h}}\dsum\limits_{t=p}^{T-h}\left( 
\widehat{\underline{F}}_{t}^{\prime }-\underline{F}_{t}^{\prime }Q\right)
^{\prime }\left( \widehat{\underline{F}}_{t}^{\prime }-\underline{F}%
_{t}^{\prime }Q\right) \right\Vert _{2}b^{\prime }B_{2}^{\prime }Q^{\prime
-1}Q^{-1}B_{2}b}\text{ }\left( \text{since for a symmetric }\right. \text{ }
\\
&&\text{ }\left. \text{psd matrix }A\text{, }\left\Vert A\right\Vert _{2}=%
\sqrt{\lambda _{\max }\left( A^{\prime }A\right) }=\sqrt{\lambda _{\max
}\left( A^{2}\right) }=\sqrt{\left[ \lambda _{\max }\left( A\right) \right]
^{2}}=\lambda _{\max }\left( A\right) \right) \\
&=&\sqrt{\left\Vert \frac{1}{T_{h}}\dsum\limits_{t=p}^{T-h}\left( \widehat{%
\underline{F}}_{t}-Q^{\prime }\underline{F}_{t}\right) \left( \widehat{%
\underline{F}}_{t}-Q^{\prime }\underline{F}_{t}\right) ^{\prime }\right\Vert
_{2}b^{\prime }B_{2}^{\prime }Q^{\prime -1}Q^{-1}B_{2}b} \\
&&\left( \text{since }\frac{1}{T_{h}}\dsum\limits_{t=p}^{T-h}\left( \widehat{%
\underline{F}}_{t}^{\prime }-\underline{F}_{t}^{\prime }Q\right) ^{\prime
}\left( \widehat{\underline{F}}_{t}^{\prime }-\underline{F}_{t}^{\prime
}Q\right) =\frac{1}{T_{h}}\dsum\limits_{t=p}^{T-h}\left( \widehat{\underline{%
F}}_{t}-Q^{\prime }\underline{F}_{t}\right) \left( \widehat{\underline{F}}%
_{t}-Q^{\prime }\underline{F}_{t}\right) ^{\prime }\right)
\end{eqnarray*}%
It follows by applying expression (\ref{quad form HinvB2}), the result given
in part (g) of Lemma D-16, and the Slutsky' theorem that%
\begin{eqnarray*}
\left\vert \text{ }\frac{\iota _{T_{h}}^{\prime }\left( \underline{\widehat{F%
}}-\underline{F}Q\right) Q^{-1}B_{2}b}{T_{h}}\right\vert &\leq &\sqrt{%
\left\Vert \frac{1}{T_{h}}\dsum\limits_{t=p}^{T-h}\left( \widehat{\underline{%
F}}_{t}-Q^{\prime }\underline{F}_{t}\right) \left( \widehat{\underline{F}}%
_{t}-Q^{\prime }\underline{F}_{t}\right) ^{\prime }\right\Vert _{2}b^{\prime
}B_{2}^{\prime }Q^{\prime -1}Q^{-1}B_{2}b} \\
&=&o_{p}\left( 1\right) \text{. }
\end{eqnarray*}%
Since the above argument holds for all $b\in \mathbb{R}^{d}$ such that $%
\left\Vert b\right\Vert _{2}=1$, we further deduce that%
\begin{equation*}
\frac{\iota _{T_{h}}^{\prime }\left( \underline{\widehat{F}}-\underline{F}%
Q\right) Q^{-1}B_{2}}{T_{h}}=o_{p}\left( 1\right) \text{.}
\end{equation*}

For part (g), note that, for any $a\in \mathbb{R}^{Kp}$ and $b\in \mathbb{R}%
^{d}$ such that $\left\Vert a\right\Vert _{2}=1$ and $\left\Vert
b\right\Vert _{2}=1$, we have, by direct calculation, 
\begin{eqnarray*}
&&\frac{a^{\prime }\underline{\widehat{F}}^{\prime }\mathfrak{H}b}{T_{h}} \\
&=&\frac{1}{T_{h}}\dsum\limits_{t=p}^{T-h}a^{\prime }\underline{\widehat{F}}%
_{t}\eta _{t{\LARGE +}h}^{\prime }b \\
&=&\frac{1}{T_{h}}\dsum\limits_{t=p}^{T-h}a^{\prime }\left( \widehat{%
\underline{F}}_{t}-Q^{\prime }\underline{F}_{t}+Q^{\prime }\underline{F}%
_{t}\right) \eta _{t{\LARGE +}h}^{\prime }b \\
&=&\frac{1}{T_{h}}\dsum\limits_{t=p}^{T-h}a^{\prime }\left( \widehat{%
\underline{F}}_{t}-Q^{\prime }\underline{F}_{t}\right) \eta _{t{\LARGE +}%
h}^{\prime }b+\frac{1}{T_{h}}\dsum\limits_{t=p}^{T-h}a^{\prime }Q^{\prime }%
\underline{F}_{t}\eta _{t{\LARGE +}h}^{\prime }b
\end{eqnarray*}%
Focusing first on the first term on last line above, we note that 
\begin{eqnarray*}
&&\left\vert \frac{1}{T_{h}}\dsum\limits_{t=p}^{T-h}a^{\prime }\left( 
\widehat{\underline{F}}_{t}-Q^{\prime }\underline{F}_{t}\right) \eta _{t%
{\LARGE +}h}^{\prime }b\right\vert \\
&\leq &\sqrt{\frac{1}{T_{h}}\dsum\limits_{t=p}^{T-h}a^{\prime }\left( 
\widehat{\underline{F}}_{t}-Q^{\prime }\underline{F}_{t}\right) \left( 
\widehat{\underline{F}}_{t}-Q^{\prime }\underline{F}_{t}\right) ^{\prime }a}%
\sqrt{b^{\prime }\frac{1}{T_{h}}\dsum\limits_{t=p}^{T-h}\eta _{t{\LARGE +}%
h}\eta _{t{\LARGE +}h}^{\prime }b} \\
&\leq &\sqrt{\frac{1}{T_{h}}\dsum\limits_{t=p}^{T-h}a^{\prime }\left( 
\widehat{\underline{F}}_{t}-Q^{\prime }\underline{F}_{t}\right) \left( 
\widehat{\underline{F}}_{t}-Q^{\prime }\underline{F}_{t}\right) ^{\prime }a}
\\
&&\times \sqrt{\left\vert b^{\prime }\left( \frac{1}{T_{h}}%
\dsum\limits_{t=p}^{T-h}\eta _{t{\LARGE +}h}\eta _{t{\LARGE +}h}^{\prime }-%
\frac{1}{T_{h}}\dsum\limits_{t=p}^{T-h}E\left[ \eta _{t{\LARGE +}h}\eta _{t%
{\LARGE +}h}^{\prime }\right] \right) b\right\vert +\left\vert \frac{1}{T_{h}%
}\dsum\limits_{t=p}^{T-h}b^{\prime }E\left[ \eta _{t{\LARGE +}h}\eta _{t%
{\LARGE +}h}^{\prime }\right] b\right\vert } \\
&\leq &\sqrt{\frac{1}{T_{h}}\dsum\limits_{t=p}^{T-h}a^{\prime }\left( 
\widehat{\underline{F}}_{t}-Q^{\prime }\underline{F}_{t}\right) \left( 
\widehat{\underline{F}}_{t}-Q^{\prime }\underline{F}_{t}\right) ^{\prime }a}%
\sqrt{\left\vert b^{\prime }\left( \frac{1}{T_{h}}\dsum\limits_{t=p}^{T-h}%
\eta _{t{\LARGE +}h}\eta _{t{\LARGE +}h}^{\prime }-\frac{1}{T_{h}}%
\dsum\limits_{t=p}^{T-h}E\left[ \eta _{t{\LARGE +}h}\eta _{t{\LARGE +}%
h}^{\prime }\right] \right) b\right\vert } \\
&&+\sqrt{\frac{1}{T_{h}}\dsum\limits_{t=p}^{T-h}a^{\prime }\left( \widehat{%
\underline{F}}_{t}-Q^{\prime }\underline{F}_{t}\right) \left( \widehat{%
\underline{F}}_{t}-Q^{\prime }\underline{F}_{t}\right) ^{\prime }a}\sqrt{%
\frac{1}{T_{h}}\dsum\limits_{t=p}^{T-h}b^{\prime }E\left[ \eta _{t{\LARGE +}%
h}\eta _{t{\LARGE +}h}^{\prime }\right] b} \\
&&\left( \text{by the inequality }\sqrt{a_{1}+a_{2}}\leq \sqrt{a_{1}}+\sqrt{%
a_{2}}\text{ for }a_{1}\geq 0\text{ and }a_{2}\geq 0\right)
\end{eqnarray*}%
Note that, by part (g) of Lemma D-16, we have%
\begin{equation*}
\frac{1}{T_{h}}\dsum\limits_{t=p}^{T-h}\left( \widehat{\underline{F}}%
_{t}-Q^{\prime }\underline{F}_{t}\right) \left( \widehat{\underline{F}}%
_{t}-Q^{\prime }\underline{F}_{t}\right) ^{\prime }=o_{p}\left( 1\right) .
\end{equation*}%
and, by part (h) of Lemma D-2,%
\begin{equation*}
\frac{1}{T_{h}}\dsum\limits_{t=p}^{T-h}\eta _{t{\LARGE +}h}\eta _{t{\LARGE +}%
h}^{\prime }-\frac{1}{T_{h}}\dsum\limits_{t=p}^{T-h}E\left[ \eta _{t{\LARGE +%
}h}\eta _{t{\LARGE +}h}^{\prime }\right] =O_{p}\left( \frac{1}{\sqrt{T}}%
\right) \text{.}
\end{equation*}%
Moreover, note that%
\begin{equation*}
\eta _{t{\LARGE +}h}=\dsum\limits_{j=0}^{h-1}J_{d}A^{j}J_{d{\LARGE +}%
K}^{\prime }\varepsilon _{t{\LARGE +}h-j}
\end{equation*}%
and, using expression (\ref{Eeta^4}) given in the proof of part (e) of Lemma
D-2 and Assumption 3-2(b), we see that there exists a positive constant $%
C^{\ast }$ such that 
\begin{eqnarray*}
&&\frac{1}{T_{h}}\dsum\limits_{t=p}^{T-h}b^{\prime }E\left[ \eta _{t{\LARGE +%
}h}\eta _{t{\LARGE +}h}^{\prime }\right] b \\
&=&\frac{1}{T_{h}}\dsum\limits_{t=p}^{T-h}E\left[ \left( b^{\prime }\eta _{t%
{\LARGE +}h}\right) ^{2}\right] \\
&\leq &\frac{1}{T_{h}}\dsum\limits_{t=p}^{T-h}\left( E\left[ \left(
b^{\prime }\eta _{t{\LARGE +}h}\right) ^{4}\right] \right) ^{\frac{{\Large 1}%
}{{\Large 2}}} \\
&\leq &\frac{1}{T_{h}}\dsum\limits_{t=p}^{T-h}\left( \text{ }C^{\ast
}\right) ^{\frac{{\Large 1}}{{\Large 2}}}\text{ } \\
&&\left( \text{for some positive constant }C^{\ast }\text{ as shown in
expression (\ref{Eeta^4})}\right) \\
&\leq &\left( \text{ }C^{\ast }\right) ^{\frac{{\Large 1}}{{\Large 2}}%
}<\infty
\end{eqnarray*}%
Making use of these calculations and applying Slutsky's theorem, we deduce
that 
\begin{eqnarray*}
&&\left\vert \frac{1}{T_{h}}\dsum\limits_{t=p}^{T-h}a^{\prime }\left( 
\widehat{\underline{F}}_{t}-Q^{\prime }\underline{F}_{t}\right) \eta _{t%
{\LARGE +}h}^{\prime }b\right\vert \\
&\leq &\sqrt{\frac{1}{T_{h}}\dsum\limits_{t=p}^{T-h}a^{\prime }\left( 
\widehat{\underline{F}}_{t}-Q^{\prime }\underline{F}_{t}\right) \left( 
\widehat{\underline{F}}_{t}-Q^{\prime }\underline{F}_{t}\right) ^{\prime }a}%
\sqrt{\left\vert b^{\prime }\left( \frac{1}{T_{h}}\dsum\limits_{t=p}^{T-h}%
\eta _{t{\LARGE +}h}\eta _{t{\LARGE +}h}^{\prime }-\frac{1}{T_{h}}%
\dsum\limits_{t=p}^{T-h}E\left[ \eta _{t{\LARGE +}h}\eta _{t{\LARGE +}%
h}^{\prime }\right] \right) b\right\vert } \\
&&+\sqrt{\frac{1}{T_{h}}\dsum\limits_{t=p}^{T-h}a^{\prime }\left( \widehat{%
\underline{F}}_{t}-Q^{\prime }\underline{F}_{t}\right) \left( \widehat{%
\underline{F}}_{t}-Q^{\prime }\underline{F}_{t}\right) ^{\prime }a}\sqrt{%
\frac{1}{T_{h}}\dsum\limits_{t=p}^{T-h}b^{\prime }E\left[ \eta _{t{\LARGE +}%
h}\eta _{t{\LARGE +}h}^{\prime }\right] b} \\
&=&o_{p}\left( 1\right) \text{.}
\end{eqnarray*}

\noindent Next, note that, by part (f) of Lemma D-2 and Slutsky's theorem,
we see that%
\begin{eqnarray*}
\frac{1}{T_{h}}\dsum\limits_{t=p}^{T-h}a^{\prime }Q^{\prime }\underline{F}%
_{t}\eta _{t{\LARGE +}h}^{\prime }b &=&a^{\prime }Q^{\prime }\left( \frac{1}{%
T_{h}}\dsum\limits_{t=p}^{T-h}\underline{F}_{t}\eta _{t{\LARGE +}h}^{\prime
}\right) b \\
&=&O_{p}\left( \frac{1}{\sqrt{T}}\right) =o_{p}\left( 1\right)
\end{eqnarray*}%
Putting everything together and applying Slutsky's theorem once more, we
then obtain 
\begin{eqnarray*}
\frac{a^{\prime }\underline{\widehat{F}}^{\prime }\mathfrak{H}b}{T_{h}} &=&%
\frac{1}{T_{h}}\dsum\limits_{t=p}^{T-h}a^{\prime }\underline{\widehat{F}}%
_{t}\eta _{t{\LARGE +}h}^{\prime }b \\
&=&\frac{1}{T_{h}}\dsum\limits_{t=p}^{T-h}a^{\prime }\left( \widehat{%
\underline{F}}_{t}-Q^{\prime }\underline{F}_{t}\right) \eta _{t{\LARGE +}%
h}^{\prime }b+\frac{1}{T_{h}}\dsum\limits_{t=p}^{T-h}a^{\prime }Q^{\prime }%
\underline{F}_{t}\eta _{t{\LARGE +}h}^{\prime }b \\
&=&o_{p}\left( 1\right) \text{.}
\end{eqnarray*}%
Since the above argument holds for all $a\in \mathbb{R}^{Kp}$ and $b\in 
\mathbb{R}^{d}$ such that $\left\Vert a\right\Vert _{2}=1$ and $\left\Vert
b\right\Vert _{2}=1$; we further deduce that%
\begin{equation*}
\frac{\underline{\widehat{F}}^{\prime }\mathfrak{H}}{T_{h}}=\frac{1}{T_{h}}%
\dsum\limits_{t=p}^{T-h}\underline{\widehat{F}}_{t}\eta _{t{\LARGE +}%
h}^{\prime }=o_{p}\left( 1\right) \text{. }\square
\end{equation*}

\bigskip

\noindent \textbf{Lemma D-18: }Suppose that Assumptions 3-1, 3-2, 3-3, 3-4,
3-5, 3-6, 3-7, 3-8, 3-9, 3-10, and 3-11* hold. Then,%
\begin{equation*}
\left( 
\begin{array}{c}
\widehat{\beta }_{0}^{\prime }-\beta _{0}^{\prime } \\ 
\widehat{B}_{1}-B_{1} \\ 
\widehat{B}_{2}-Q^{-1}B_{2}%
\end{array}%
\right) =o_{p}\left( 1\right) .
\end{equation*}%
Here, $\widehat{\beta }_{0}$, $\widehat{B}_{1}$, and $\widehat{B}_{2}$
denote the OLS estimators of the coefficient parameters in the (feasible) $h$%
-step ahead forecast equation 
\begin{eqnarray*}
Y_{t{\LARGE +}h} &=&\beta _{0}+\dsum\limits_{g=1}^{p}B_{1,g}^{\prime }Y_{t-g%
{\LARGE +}1}+\dsum\limits_{g=1}^{p}B_{2,g}^{\prime }\widehat{F}_{t-g{\LARGE +%
}1}+\widehat{\eta }_{t{\LARGE +}h} \\
&=&\beta _{0}+B_{1}^{\prime }\underline{Y}_{t}+B_{2}^{\prime }\underline{%
\widehat{F}}_{t}+\widehat{\eta }_{t{\LARGE +}h}\text{, }
\end{eqnarray*}%
for $t=p,....,T-h$, where the unobserved factor vector $\underline{F}_{t}$
is replaced by the estimate $\underline{\widehat{F}}_{t}$ and where $%
\widehat{\eta }_{t{\LARGE +}h}=\eta _{t{\LARGE +}h}-B_{2}^{\prime }\left( 
\underline{\widehat{F}}_{t}-\underline{F}_{t}\right) $ with $\eta _{t{\LARGE %
+}h}=\dsum\nolimits_{j=0}^{h-1}J_{d}A^{j}J_{d{\LARGE +}K}^{\prime
}\varepsilon _{t{\LARGE +}h-j}$ as previously defined.

\medskip

\noindent \textbf{Proof of Lemma D-18: }To proceed, we first stack the
observations to obtain the representation%
\begin{equation}
\underset{T_{h}\times d}{Y\left( h\right) }=\underset{T_{h}\times 1}{\iota
_{T_{h}}}\underset{1\times d}{\beta _{0}^{\prime }}+\underset{T_{h}\times dp}%
{\underline{Y}}\underset{dp\times d}{B_{1}}+\underset{T_{h}\times Kp}{%
\underline{\widehat{F}}}\underset{Kp\times d}{B_{2}}+\underset{T_{h}\times d}%
{\widehat{\mathfrak{H}}}  \label{forecast eqn stacked}
\end{equation}%
where $T_{h}=T-h-p+1$ and where%
\begin{equation*}
\underset{T_{h}\times d}{Y\left( h\right) }=\left( 
\begin{array}{c}
Y_{h{\LARGE +}p}^{\prime } \\ 
\vdots \\ 
Y_{T}^{\prime }%
\end{array}%
\right) \text{, }\underset{T_{h}\times dp}{\underline{Y}}=\left( 
\begin{array}{c}
\underline{Y}_{p}^{\prime } \\ 
\vdots \\ 
\underline{Y}_{T-h}^{\prime }%
\end{array}%
\right) \text{, }\underset{T_{h}\times Kp}{\underline{\widehat{F}}}=\left( 
\begin{array}{c}
\underline{\widehat{F}}_{p}^{\prime } \\ 
\vdots \\ 
\underline{\widehat{F}}_{T-h}^{\prime }%
\end{array}%
\right) \text{, and }\underset{T_{h}\times d}{\widehat{\mathfrak{H}}}=\left( 
\begin{array}{c}
\widehat{\eta }_{h{\LARGE +}p}^{\prime } \\ 
\vdots \\ 
\widehat{\eta }_{T}^{\prime }%
\end{array}%
\right) \text{.}
\end{equation*}%
It is easily seen from expression (\ref{forecast eqn stacked}) that the OLS
estimators of the coefficients $\beta _{0}$, $B_{1}$, and $B_{2}$ are given
by%
\begin{equation*}
\left( 
\begin{array}{c}
\widehat{\beta }_{0}^{\prime } \\ 
\widehat{B}_{1} \\ 
\widehat{B}_{2}%
\end{array}%
\right) =\left( 
\begin{array}{ccc}
T_{h} & \iota _{T_{h}}^{\prime }\underline{Y} & \iota _{T_{h}}^{\prime }%
\underline{\widehat{F}} \\ 
\underline{Y}^{\prime }\iota _{T_{h}} & \underline{Y}^{\prime }\underline{Y}
& \underline{Y}^{\prime }\underline{\widehat{F}} \\ 
\underline{\widehat{F}}^{\prime }\iota _{T_{h}} & \underline{\widehat{F}}%
^{\prime }\underline{Y} & \underline{\widehat{F}}^{\prime }\underline{%
\widehat{F}}%
\end{array}%
\right) ^{-1}\left[ 
\begin{array}{c}
\iota _{T_{h}}^{\prime }Y\left( h\right) \\ 
\underline{Y}^{\prime }Y\left( h\right) \\ 
\underline{\widehat{F}}^{\prime }Y\left( h\right)%
\end{array}%
\right] \text{.}
\end{equation*}%
Now, rewrite expression (\ref{forecast eqn stacked}) as

\begin{eqnarray*}
Y\left( h\right) &=&\iota _{T_{h}}\beta _{0}^{\prime }+\underline{Y}B_{1}+%
\underline{\widehat{F}}B_{2}+\widehat{\mathfrak{H}} \\
&=&\iota _{T_{h}}\beta _{0}^{\prime }+\underline{Y}B_{1}+\underline{\widehat{%
F}}B_{2}+\mathfrak{H}-\left( \widehat{\underline{F}}-\underline{F}\right)
B_{2} \\
&=&\iota _{T_{h}}\beta _{0}^{\prime }+\underline{Y}B_{1}+\underline{F}B_{2}+%
\mathfrak{H} \\
&=&\iota _{T_{h}}\beta _{0}^{\prime }+\underline{Y}B_{1}+\underline{F}%
QQ^{-1}B_{2}+\mathfrak{H} \\
&=&\iota _{T_{h}}\beta _{0}^{\prime }+\underline{Y}B_{1}+\left( \underline{%
\widehat{F}}+\underline{F}Q-\underline{\widehat{F}}\right) Q^{-1}B_{2}+%
\mathfrak{H} \\
&=&\iota _{T_{h}}\beta _{0}^{\prime }+\underline{Y}B_{1}+\underline{\widehat{%
F}}Q^{-1}B_{2}-\left( \underline{\widehat{F}}-\underline{F}Q\right)
Q^{-1}B_{2}+\mathfrak{H} \\
&=&\left[ 
\begin{array}{ccc}
\iota _{T_{h}} & \underline{Y} & \underline{\widehat{F}}%
\end{array}%
\right] \left( 
\begin{array}{c}
\beta _{0}^{\prime } \\ 
B_{1} \\ 
Q^{-1}B_{2}%
\end{array}%
\right) -\left( \underline{\widehat{F}}-\underline{F}Q\right) Q^{-1}B_{2}+%
\mathfrak{H}\text{,}
\end{eqnarray*}%
and it follows that%
\begin{eqnarray*}
&&\left( 
\begin{array}{c}
\widehat{\beta }_{0}^{\prime }-\beta _{0}^{\prime } \\ 
\widehat{B}_{1}-B_{1} \\ 
\widehat{B}_{2}-Q^{-1}B_{2}%
\end{array}%
\right) \\
&=&\left( 
\begin{array}{ccc}
T_{h} & \iota _{T_{h}}^{\prime }\underline{Y} & \iota _{T_{h}}^{\prime }%
\underline{\widehat{F}} \\ 
\underline{Y}^{\prime }\iota _{T_{h}} & \underline{Y}^{\prime }\underline{Y}
& \underline{Y}^{\prime }\underline{\widehat{F}} \\ 
\underline{\widehat{F}}^{\prime }\iota _{T_{h}} & \underline{\widehat{F}}%
^{\prime }\underline{Y} & \underline{\widehat{F}}^{\prime }\underline{%
\widehat{F}}%
\end{array}%
\right) ^{-1}\left[ 
\begin{array}{c}
\iota _{T_{h}}^{\prime } \\ 
\underline{Y}^{\prime } \\ 
\underline{\widehat{F}}^{\prime }%
\end{array}%
\right] Y\left( h\right) -\left( 
\begin{array}{c}
\beta _{0}^{\prime } \\ 
B_{1} \\ 
Q^{-1}B_{2}%
\end{array}%
\right) \\
&=&\left( 
\begin{array}{ccc}
T_{h} & \iota _{T_{h}}^{\prime }\underline{Y} & \iota _{T_{h}}^{\prime }%
\underline{\widehat{F}} \\ 
\underline{Y}^{\prime }\iota _{T_{h}} & \underline{Y}^{\prime }\underline{Y}
& \underline{Y}^{\prime }\underline{\widehat{F}} \\ 
\underline{\widehat{F}}^{\prime }\iota _{T_{h}} & \underline{\widehat{F}}%
^{\prime }\underline{Y} & \underline{\widehat{F}}^{\prime }\underline{%
\widehat{F}}%
\end{array}%
\right) ^{-1}\left[ 
\begin{array}{c}
\iota _{T_{h}}^{\prime } \\ 
\underline{Y}^{\prime } \\ 
\underline{\widehat{F}}^{\prime }%
\end{array}%
\right] \left[ 
\begin{array}{ccc}
\iota _{T_{h}} & \underline{Y} & \underline{\widehat{F}}%
\end{array}%
\right] \left( 
\begin{array}{c}
\beta _{0}^{\prime } \\ 
B_{1} \\ 
Q^{-1}B_{2}%
\end{array}%
\right) -\left( 
\begin{array}{c}
\beta _{0}^{\prime } \\ 
B_{1} \\ 
Q^{-1}B_{2}%
\end{array}%
\right) \\
&&-\left( 
\begin{array}{ccc}
T_{h} & \iota _{T_{h}}^{\prime }\underline{Y} & \iota _{T_{h}}^{\prime }%
\underline{\widehat{F}} \\ 
\underline{Y}^{\prime }\iota _{T_{h}} & \underline{Y}^{\prime }\underline{Y}
& \underline{Y}^{\prime }\underline{\widehat{F}} \\ 
\underline{\widehat{F}}^{\prime }\iota _{T_{h}} & \underline{\widehat{F}}%
^{\prime }\underline{Y} & \underline{\widehat{F}}^{\prime }\underline{%
\widehat{F}}%
\end{array}%
\right) ^{-1}\left[ 
\begin{array}{c}
\iota _{T_{h}}^{\prime } \\ 
\underline{Y}^{\prime } \\ 
\underline{\widehat{F}}^{\prime }%
\end{array}%
\right] \left( \underline{\widehat{F}}-\underline{F}Q\right) Q^{-1}B_{2} \\
&&+\left( 
\begin{array}{ccc}
T_{h} & \iota _{T_{h}}^{\prime }\underline{Y} & \iota _{T_{h}}^{\prime }%
\underline{\widehat{F}} \\ 
\underline{Y}^{\prime }\iota _{T_{h}} & \underline{Y}^{\prime }\underline{Y}
& \underline{Y}^{\prime }\underline{\widehat{F}} \\ 
\underline{\widehat{F}}^{\prime }\iota _{T_{h}} & \underline{\widehat{F}}%
^{\prime }\underline{Y} & \underline{\widehat{F}}^{\prime }\underline{%
\widehat{F}}%
\end{array}%
\right) ^{-1}\left[ 
\begin{array}{c}
\iota _{T_{h}}^{\prime } \\ 
\underline{Y}^{\prime } \\ 
\underline{\widehat{F}}^{\prime }%
\end{array}%
\right] \mathfrak{H}\text{.} \\
&=&-\left( 
\begin{array}{ccc}
T_{h} & \iota _{T_{h}}^{\prime }\underline{Y} & \iota _{T_{h}}^{\prime }%
\underline{\widehat{F}} \\ 
\underline{Y}^{\prime }\iota _{T_{h}} & \underline{Y}^{\prime }\underline{Y}
& \underline{Y}^{\prime }\underline{\widehat{F}} \\ 
\underline{\widehat{F}}^{\prime }\iota _{T_{h}} & \underline{\widehat{F}}%
^{\prime }\underline{Y} & \underline{\widehat{F}}^{\prime }\underline{%
\widehat{F}}%
\end{array}%
\right) ^{-1}\left[ 
\begin{array}{c}
\iota _{T_{h}}^{\prime }\left( \underline{\widehat{F}}-\underline{F}Q\right)
Q^{-1}B_{2} \\ 
\underline{Y}^{\prime }\left( \underline{\widehat{F}}-\underline{F}Q\right)
Q^{-1}B_{2} \\ 
\underline{\widehat{F}}^{\prime }\left( \underline{\widehat{F}}-\underline{F}%
Q\right) Q^{-1}B_{2}%
\end{array}%
\right] \\
&&+\left( 
\begin{array}{ccc}
T_{h} & \iota _{T_{h}}^{\prime }\underline{Y} & \iota _{T_{h}}^{\prime }%
\underline{\widehat{F}} \\ 
\underline{Y}^{\prime }\iota _{T_{h}} & \underline{Y}^{\prime }\underline{Y}
& \underline{Y}^{\prime }\underline{\widehat{F}} \\ 
\underline{\widehat{F}}^{\prime }\iota _{T_{h}} & \underline{\widehat{F}}%
^{\prime }\underline{Y} & \underline{\widehat{F}}^{\prime }\underline{%
\widehat{F}}%
\end{array}%
\right) ^{-1}\left[ 
\begin{array}{c}
\iota _{T_{h}}^{\prime }\mathfrak{H} \\ 
\underline{Y}^{\prime }\mathfrak{H} \\ 
\underline{\widehat{F}}^{\prime }\mathfrak{H}%
\end{array}%
\right]
\end{eqnarray*}%
Next, applying parts (b) and (d) of Lemma D-2 and parts (a), (b), (c), and
(d) of Lemma D-17, we obtain%
\begin{eqnarray*}
&&\left( 
\begin{array}{ccc}
1 & \iota _{T_{h}}^{\prime }\underline{Y}/T_{h} & \iota _{T_{h}}^{\prime }%
\underline{\widehat{F}}/T_{h} \\ 
\underline{Y}^{\prime }\iota _{T_{h}}/T_{h} & \underline{Y}^{\prime }%
\underline{Y}/T_{h} & \underline{Y}^{\prime }\underline{\widehat{F}}/T_{h}
\\ 
\underline{\widehat{F}}^{\prime }\iota _{T_{h}}/T_{h} & \underline{\widehat{F%
}}^{\prime }\underline{Y}/T_{h} & \underline{\widehat{F}}^{\prime }%
\underline{\widehat{F}}/T_{h}%
\end{array}%
\right) \\
&&-\left( 
\begin{array}{ccc}
1 & T_{h}^{-1}\dsum\nolimits_{t=p}^{T-h}E\left[ \underline{Y}_{t}^{\prime }%
\right] & T_{h}^{-1}\dsum\nolimits_{t=p}^{T-h}E\left[ \underline{F}%
_{t}^{\prime }\right] Q \\ 
T_{h}^{-1}\dsum\nolimits_{t=p}^{T-h}E\left[ \underline{Y}_{t}\right] & 
T_{h}^{-1}\dsum\nolimits_{t=p}^{T-h}E\left[ \underline{Y}_{t}\underline{Y}%
_{t}^{\prime }\right] & T_{h}^{-1}\dsum\nolimits_{t=p}^{T-h}E\left[ 
\underline{Y}_{t}\underline{F}_{t}^{\prime }\right] Q \\ 
T_{h}^{-1}\dsum\nolimits_{t=p}^{T-h}Q^{\prime }E\left[ \underline{F}_{t}%
\right] & T_{h}^{-1}\dsum\nolimits_{t=p}^{T-h}Q^{\prime }E\left[ \underline{F%
}_{t}\underline{Y}_{t}^{\prime }\right] & T_{h}^{-1}\dsum%
\nolimits_{t=p}^{T-h}Q^{\prime }E\left[ \underline{F}_{t}\underline{F}%
_{t}^{\prime }\right] Q%
\end{array}%
\right) \\
&=&o_{p}\left( 1\right) \text{.}
\end{eqnarray*}%
Moreover, note that%
\begin{eqnarray*}
&&\left( 
\begin{array}{ccc}
1 & T_{h}^{-1}\dsum\nolimits_{t=p}^{T-h}E\left[ \underline{Y}_{t}^{\prime }%
\right] & T_{h}^{-1}\dsum\nolimits_{t=p}^{T-h}E\left[ \underline{F}%
_{t}^{\prime }\right] Q \\ 
T_{h}^{-1}\dsum\nolimits_{t=p}^{T-h}E\left[ \underline{Y}_{t}\right] & 
T_{h}^{-1}\dsum\nolimits_{t=p}^{T-h}E\left[ \underline{Y}_{t}\underline{Y}%
_{t}^{\prime }\right] & T_{h}^{-1}\dsum\nolimits_{t=p}^{T-h}E\left[ 
\underline{Y}_{t}\underline{F}_{t}^{\prime }\right] Q \\ 
T_{h}^{-1}\dsum\nolimits_{t=p}^{T-h}Q^{\prime }E\left[ \underline{F}_{t}%
\right] & T_{h}^{-1}\dsum\nolimits_{t=p}^{T-h}Q^{\prime }E\left[ \underline{F%
}_{t}\underline{Y}_{t}^{\prime }\right] & T_{h}^{-1}\dsum%
\nolimits_{t=p}^{T-h}Q^{\prime }E\left[ \underline{F}_{t}\underline{F}%
_{t}^{\prime }\right] Q%
\end{array}%
\right) \\
&=&\frac{1}{T_{h}}\dsum\limits_{t=p}^{T-h}\left( 
\begin{array}{ccc}
1 & E\left[ \underline{Y}_{t}^{\prime }\right] & E\left[ \underline{F}%
_{t}^{\prime }\right] Q \\ 
E\left[ \underline{Y}_{t}\right] & E\left[ \underline{Y}_{t}\underline{Y}%
_{t}^{\prime }\right] & E\left[ \underline{Y}_{t}\underline{F}_{t}^{\prime }%
\right] Q \\ 
Q^{\prime }E\left[ \underline{F}_{t}\right] & Q^{\prime }E\left[ \underline{F%
}_{t}\underline{Y}_{t}^{\prime }\right] & Q^{\prime }E\left[ \underline{F}%
_{t}\underline{F}_{t}^{\prime }\right] Q%
\end{array}%
\right) \\
&=&\left( 
\begin{array}{ccc}
1 & 0 & 0 \\ 
0 & I_{dp} & 0 \\ 
0 & 0 & Q^{\prime }%
\end{array}%
\right) \frac{1}{T_{h}}\dsum\limits_{t=p}^{T-h}\left( 
\begin{array}{ccc}
1 & E\left[ \underline{Y}_{t}^{\prime }\right] & E\left[ \underline{F}%
_{t}^{\prime }\right] \\ 
E\left[ \underline{Y}_{t}\right] & E\left[ \underline{Y}_{t}\underline{Y}%
_{t}^{\prime }\right] & E\left[ \underline{Y}_{t}\underline{F}_{t}^{\prime }%
\right] \\ 
E\left[ \underline{F}_{t}\right] & E\left[ \underline{F}_{t}\underline{Y}%
_{t}^{\prime }\right] & E\left[ \underline{F}_{t}\underline{F}_{t}^{\prime }%
\right]%
\end{array}%
\right) \left( 
\begin{array}{ccc}
1 & 0 & 0 \\ 
0 & I_{dp} & 0 \\ 
0 & 0 & Q%
\end{array}%
\right)
\end{eqnarray*}%
which is non-singular and, therefore, also positive definite for all $T$
sufficiently large in light of the result given in part (b) of Lemma D-1.

In addition, applying parts (f) and (g) of Lemma D-2 and parts (d), (e),
(f), and (g) of Lemma D-17, we have%
\begin{eqnarray*}
\frac{\iota _{T_{h}}^{\prime }\mathfrak{H}}{T_{h}} &=&\frac{1}{T_{h}}%
\dsum\limits_{t=p}^{T-h}\eta _{t{\LARGE +}h}^{\prime }=O_{p}\left( \frac{1}{%
\sqrt{T}}\right) , \\
\frac{\underline{Y}^{\prime }\mathfrak{H}}{T_{h}} &=&\frac{1}{T_{h}}%
\dsum\limits_{t=p}^{T-h}\underline{Y}_{t}\eta _{t{\LARGE +}h}^{\prime
}=O_{p}\left( \frac{1}{\sqrt{T}}\right)
\end{eqnarray*}%
and%
\begin{eqnarray*}
\frac{\iota _{T_{h}}^{\prime }\left( \underline{\widehat{F}}-\underline{F}%
Q\right) Q^{-1}B_{2}}{T_{h}} &=&\frac{1}{T_{h}}\dsum\limits_{t=p}^{T-h}%
\left( \underline{\widehat{F}}_{t}^{\prime }-\underline{F}_{t}^{\prime
}Q\right) Q^{-1}B_{2}=o_{p}(1), \\
\frac{\underline{Y}^{\prime }\left( \underline{\widehat{F}}-\underline{F}%
Q\right) Q^{-1}B_{2}}{T_{h}} &=&\frac{1}{T_{h}}\dsum\limits_{t=p}^{T-h}%
\underline{Y}_{t}\left( \underline{\widehat{F}}_{t}^{\prime }-\underline{F}%
_{t}^{\prime }Q\right) Q^{-1}B_{2}=o_{p}(1), \\
\frac{\underline{\widehat{F}}^{\prime }\left( \underline{\widehat{F}}-%
\underline{F}Q\right) Q^{-1}B_{2}}{T_{h}} &=&\frac{1}{T_{h}}%
\dsum\limits_{t=p}^{T-h}\underline{\widehat{F}}_{t}\left( \underline{%
\widehat{F}}_{t}^{\prime }-\underline{F}_{t}^{\prime }Q\right)
Q^{-1}B_{2}=o_{p}(1), \\
\frac{\underline{\widehat{F}}^{\prime }\mathfrak{H}}{T_{h}} &=&\frac{1}{T_{h}%
}\dsum\limits_{t=p}^{T-h}\underline{\widehat{F}}_{t}\eta _{t{\LARGE +}%
h}^{\prime }=o_{p}(1)
\end{eqnarray*}%
Putting everything together and applying the Slutsky's theorem%
\begin{eqnarray*}
&&\left( 
\begin{array}{c}
\widehat{\beta }_{0}^{\prime }-\beta _{0}^{\prime } \\ 
\widehat{B}_{1}-B_{1} \\ 
\widehat{B}_{2}-Q^{-1}B_{2}%
\end{array}%
\right) \\
&=&-\left( 
\begin{array}{ccc}
T_{h} & \iota _{T_{h}}^{\prime }\underline{Y} & \iota _{T_{h}}^{\prime }%
\underline{\widehat{F}} \\ 
\underline{Y}^{\prime }\iota _{T_{h}} & \underline{Y}^{\prime }\underline{Y}
& \underline{Y}^{\prime }\underline{\widehat{F}} \\ 
\underline{\widehat{F}}^{\prime }\iota _{T_{h}} & \underline{\widehat{F}}%
^{\prime }\underline{Y} & \underline{\widehat{F}}^{\prime }\underline{%
\widehat{F}}%
\end{array}%
\right) ^{-1}\left[ 
\begin{array}{c}
\iota _{T_{h}}^{\prime }\left( \underline{\widehat{F}}-\underline{F}Q\right)
Q^{-1}B_{2} \\ 
\underline{Y}^{\prime }\left( \underline{\widehat{F}}-\underline{F}Q\right)
Q^{-1}B_{2} \\ 
\underline{\widehat{F}}^{\prime }\left( \underline{\widehat{F}}-\underline{F}%
Q\right) Q^{-1}B_{2}%
\end{array}%
\right] \\
&&+\left( 
\begin{array}{ccc}
T_{h} & \iota _{T_{h}}^{\prime }\underline{Y} & \iota _{T_{h}}^{\prime }%
\underline{\widehat{F}} \\ 
\underline{Y}^{\prime }\iota _{T_{h}} & \underline{Y}^{\prime }\underline{Y}
& \underline{Y}^{\prime }\underline{\widehat{F}} \\ 
\underline{\widehat{F}}^{\prime }\iota _{T_{h}} & \underline{\widehat{F}}%
^{\prime }\underline{Y} & \underline{\widehat{F}}^{\prime }\underline{%
\widehat{F}}%
\end{array}%
\right) ^{-1}\left[ 
\begin{array}{c}
\iota _{T_{h}}^{\prime }\mathfrak{H} \\ 
\underline{Y}^{\prime }\mathfrak{H} \\ 
\underline{\widehat{F}}^{\prime }\mathfrak{H}%
\end{array}%
\right] \\
&=&-\left( 
\begin{array}{ccc}
1 & T_{h}^{-1}\iota _{T_{h}}^{\prime }\underline{Y} & T_{h}^{-1}\iota
_{T_{h}}^{\prime }\underline{\widehat{F}} \\ 
T_{h}^{-1}\underline{Y}^{\prime }\iota _{T_{h}} & T_{h}^{-1}\underline{Y}%
^{\prime }\underline{Y} & T_{h}^{-1}\underline{Y}^{\prime }\underline{%
\widehat{F}} \\ 
T_{h}^{-1}\underline{\widehat{F}}^{\prime }\iota _{T_{h}} & T_{h}^{-1}%
\underline{\widehat{F}}^{\prime }\underline{Y} & T_{h}^{-1}\underline{%
\widehat{F}}^{\prime }\underline{\widehat{F}}%
\end{array}%
\right) ^{-1}\left[ 
\begin{array}{c}
T_{h}^{-1}\iota _{T_{h}}^{\prime }\left( \underline{\widehat{F}}-\underline{F%
}Q\right) Q^{-1}B_{2} \\ 
T_{h}^{-1}\underline{Y}^{\prime }\left( \underline{\widehat{F}}-\underline{F}%
Q\right) Q^{-1}B_{2} \\ 
T_{h}^{-1}\underline{\widehat{F}}^{\prime }\left( \underline{\widehat{F}}-%
\underline{F}Q\right) Q^{-1}B_{2}%
\end{array}%
\right] \\
&&+\left( 
\begin{array}{ccc}
1 & T_{h}^{-1}\iota _{T_{h}}^{\prime }\underline{Y} & T_{h}^{-1}\iota
_{T_{h}}^{\prime }\underline{\widehat{F}} \\ 
T_{h}^{-1}\underline{Y}^{\prime }\iota _{T_{h}} & T_{h}^{-1}\underline{Y}%
^{\prime }\underline{Y} & T_{h}^{-1}\underline{Y}^{\prime }\underline{%
\widehat{F}} \\ 
T_{h}^{-1}\underline{\widehat{F}}^{\prime }\iota _{T_{h}} & T_{h}^{-1}%
\underline{\widehat{F}}^{\prime }\underline{Y} & T_{h}^{-1}\underline{%
\widehat{F}}^{\prime }\underline{\widehat{F}}%
\end{array}%
\right) ^{-1}\left[ 
\begin{array}{c}
T_{h}^{-1}\iota _{T_{h}}^{\prime }\mathfrak{H} \\ 
T_{h}^{-1}\underline{Y}^{\prime }\mathfrak{H} \\ 
T_{h}^{-1}\underline{\widehat{F}}^{\prime }\mathfrak{H}%
\end{array}%
\right] \\
&=&o_{p}\left( 1\right) \text{. }\square
\end{eqnarray*}

\bigskip

\noindent

\pagebreak

\noindent

\begin{thebibliography}{99}
\bibitem{} Anatolyev, S. and A. Mikusheva (2021): \textquotedblleft Factor
Models with Many Assets: Strong Factors, Weak Factors, and the Two-Pass
Procedure,\textquotedblright\ \textit{Journal of Econometrics}, forthcoming.

\bibitem{} Andrews, D.W.K. (1984): \textquotedblleft Non-strong Mixing
Autoregressive Processes,\textquotedblright\ \textit{Journal of Applied
Probability}, 21, 930-934.

\bibitem{} Bai, J. and S. Ng (2002): \textquotedblleft Determining the
Number of Factors in Approximate Factor Models,\textquotedblright\ \textit{%
Econometrica}, 70, 191-221.

\bibitem{} Bai, J. (2003): \textquotedblleft Inferential Theory for Factor
Models of Large Dimensions,\textquotedblright\ \textit{Econometrica}, 71,
135-171.

\bibitem{} Bai, J. and S. Ng (2008): \textquotedblleft Forecasting Economic
Time Series Using Targeted Predictors,\textquotedblright\ \textit{Journal of
Econometrics}, 146, 304-317.

\bibitem{} Bai, J. and S. Ng (2021): \textquotedblleft Approximate Factor
Models with Weaker Loading,\textquotedblright\ Working Paper, Columbia
University.

\bibitem{} Bai, Z. D. and Y. Q. Yin (1993): \textquotedblleft Limit of the
Smallest Eigenvalue of a Large Dimensional Sample Covariance Matrix," 
\textit{Annals of Probability}, 21, 1275-1294.

\bibitem{} Bair, E., T. Hastie, D. Paul, and R. Tibshirani (2006):
\textquotedblleft Prediction by Supervised Principal
Components,\textquotedblright\ \textit{Journal of the American Statistical
Association}, 101, 119-137.

\bibitem{} Belloni, A., D. Chen, V. Chernozhukov, and C. Hansen (2012):
\textquotedblleft Sparse Models and Methods for Optimal Instruments with an
Application to Eminent Domain,\textquotedblright\ \textit{Econometrica}, 80,
2369-2429.

\bibitem{} Billingsley, P. (1995): \textit{Probability and Measure}. New
York: John Wiley \& Sons.

\bibitem{} Borovkova, S., R. Burton, and H. Dehling (2001):
\textquotedblleft Limit Theorems for Functionals of Mixing Processes to
U-Statistics and Dimension Estimation,\textquotedblright\ \textit{%
Transactions of the American Mathematical Society}, 353, 4261-4318.

\bibitem{} Bryzgalova, S. (2016): \textquotedblleft Spurious Factors in
Linear Asset Pricing Models,\textquotedblright\ Working Paper, Stanford
Graduate School of Business.

\bibitem{} Burnside, C. (2016): \textquotedblleft Identification and
Inference in Linear Stochastic Discount Factor Models with Excess
Returns,\textquotedblright\ \textit{Journal of Financial Econometrics}, 14,
295-330.

\bibitem{} Chao, J. C. and N. R. Swanson (2022a): \textquotedblleft
Consistent Estimation, Variable Selection, and Forecasting in
Factor-Augmented VAR Models,\textquotedblright\ Working Paper, Rutgers
University and University of Maryland.

\bibitem{} Chao, J. C. and N. R. Swanson (2022b): Technical Appendix to
\textquotedblleft Consistent Estimation, Variable Selection, and Forecasting
in Factor-Augmented VAR Models,\textquotedblright\ Working Paper, Rutgers
University and University of Maryland.

\bibitem{} Chen, X., Q. Shao, W. B. Wu, and L. Xu (2016): \textquotedblleft
Self-normalized Cram\'{e}r-type Moderate Deviations under
Dependence,\textquotedblright\ \textit{Annals of Statistics}, 44, 1593-1617.

\bibitem{} Davidson. J. (1994): \textit{Stochastic Limit Theory: An
Introduction for Econometricians}. New York: Oxford University Press.

\bibitem{} Davidson, K. R. and S. J. Szarek (2001): \textquotedblleft Local
Operator Theory, Random Matrices and Banach Spaces.\textquotedblright\ In 
\textit{Handbook of the Geometry of Banach Spaces}, 1, 317-366. Amsterdam:
North-Holland.

\bibitem{} Fan, J., Y. Liao, and M. Mincheva (2011): \textquotedblleft
High-dimensional Covariance Matrix Estimation in Approximate Factor
Models,\textquotedblright\ \textit{Annals of Statistics}, 39, 3320-3356.

\bibitem{} Fan, J., Y. Liao, and M. Mincheva (2013): \textquotedblleft Large
Covariance Estimation by Thresholding Principal Orthogonal Complements," 
\textit{Journal of the Royal Statistical Society, Series B}, 75, 603-680.

\bibitem{} Freyaldenhoven, S. (2021a): \textquotedblleft Factor Models with
Local Factors - Determining the Number of Relevant
Factors,\textquotedblright\ \textit{Journal of Econometrics}, forthcoming.

\bibitem{} Freyaldenhoven, S. (2021b): \textquotedblleft Identification
through Sparsity in Factor Models: The $\ell _{1}$-Rotation
Criterion,\textquotedblright\ Working Paper, Federal Reserve Bank of
Philadelphia.

\bibitem{} Giglio, S., D. Xiu, and D. Zhang (2021): \textquotedblleft Test
Assets and Weak Factors,\textquotedblright\ Working Paper, Yale School of
Management and the Booth School of Business, University of Chicago.

\bibitem{} Golub, G. H. and C. F. van Loan (1996): \textit{Matrix
Computations}, 3rd Edition. Baltimore: The Johns Hopkins University Press.

\bibitem{} Goroketskii, V. V. (1977): \textquotedblleft On the Strong Mixing
Property for Linear Sequences,\textquotedblright\ \textit{Theory of
Probability and Applications}, 22, 411-413.

\bibitem{} Gospodinov, N., R. Kan, and C. Robotti (2017): \textquotedblleft
Spurious Inference in Reduced-Rank Asset Pricing Models,\textquotedblright\ 
\textit{Econometrica}, 85, 1613-1628.

\bibitem{} Harding, M. C. (2008): \textquotedblleft Explaining the Single
Factor Bias of Arbitrage Pricing Models in Finite
Samples,\textquotedblright\ \textit{Economics Letters}, 99, 85-88.

\bibitem{} Horn, R. and C. Johnson (1985): \textit{Matrix Analysis}.
Cambridge University Press.

\bibitem{} Jagannathan, R. and Z. Wang (1998): \textquotedblleft An
Asymptotic Theory for Estimating Beta-Pricing Models Using Cross-Sectional
Regression,\textquotedblright\ \textit{Journal of Finance}, 53, 1285-1309.

\bibitem{} Johnstone, I. M. and A. Lu (2009): \textquotedblleft On
Consistency and Sparsity for Principal Components Analysis in High
Dimensions,\textquotedblright\ \textit{Journal of the American Statistical
Association}, 104, 682-697.

\bibitem{} Johnstone, I. M. and D. Paul (2018): \textquotedblleft PCA in
High Dimensions: An Orientation,\textquotedblright\ \textit{Proceedings of
the IEEE}, 106, 1277-1292.

\bibitem{} Kan, R. and C. Zhang (1999): \textquotedblleft Two-Pass Tests of
Asset Pricing Models with Useless Factors,\textquotedblright\ \textit{%
Journal of Finance}, 54, 203-235.

\bibitem{} Kleibergen, F. (2009): \textquotedblleft Tests of Risk Premia in
Linear Factor Models,\textquotedblright\ \textit{Journal of Econometrics},
149, 149-173.

\bibitem{} L\"{u}tkepohl, H. (2005): \textit{New Introduction to Multiple
Time Series Analysis}. New York: Springer.

\bibitem{} Nadler, B. (2008): \textquotedblleft Finite Sample Approximation
Results for Principal Component Analysis: A Matrix Perturbation
Approach,\textquotedblright\ \textit{Annals of Statistics}, 36, 2791-2817.

\bibitem{} Onatski, A. (2012): \textquotedblleft Asymptotics of the
Principal Components Estimator of Large Factor Models with Weakly
Influential Factors,\textquotedblright\ \textit{Journal of Econometrics},
168, 244-258.

\bibitem{} Paul, D. (2007): \textquotedblleft Asymptotics of Sample
Eigenstructure for a Large Dimensional Spiked Covariance
Model,\textquotedblright\ \textit{Statistica Sinica}, 17, 1617-1642.

\bibitem{} Pham, T. D. and L. T. Tran (1985): \textquotedblleft Some Mixing
Properties of Time Series Models,\textquotedblright\ \textit{Stochastic
Processes and Their Applications}, 19, 297-303.

\bibitem{} Ruhe, A. (1975): \textquotedblleft On the Closeness of
Eigenvalues and Singular Values for Almost Normal
Matrices,\textquotedblright\ \textit{Linear Algebra and Its Applications},
11, 87-94.

\bibitem{} Shen, D., H. Shen, H. Zhu, J.S. Marron (2016): \textquotedblleft
The Statistics and Mathematics of High Dimension Low Sample Size
Asymptotics,\textquotedblright\ \textit{Statistica Sinica}, 26, 1747-1770.

\bibitem{} Stewart, G.W. (1973): \textquotedblleft Error and Perturbation
Bounds for Subspaces Associated with Certain Eigenvalue
Problems,\textquotedblright\ \textit{SIAM Review}, 15, 727-764.

\bibitem{} Stewart, G.W. and J. Sun (1990): \textit{Matrix Perturbation
Theory}. Boston: Academic Press.

\bibitem{} Stock, J. H. and M. W. Watson (2002a): \textquotedblleft
Forecasting Using Principal Components from a Large Number of
Predictors,\textquotedblright\ \textit{Journal of the American Statistical
Association}, 97, 1167-1179.

\bibitem{} Stock, J. H. and M. W. Watson (2002b): \textquotedblleft
Macroeconomic Forecasting Using Diffusion Indexes,\textquotedblright\ 
\textit{Journal of Business and Economic Statistics}, 20, 147-162.

\bibitem{} Vershynin, R. (2012): \textquotedblleft Introduction to the
Non-asymptotic Analysis of Random Matrices,\textquotedblright\ In \textit{%
Compressed Sensing}, \textit{Theory and Applications, }210-268. Cambridge
University Press.

\bibitem{} Wang, W. and J. Fan (2017): \textquotedblleft Asymptotics of
Empirical Eigenstructure for High Dimensional Spiked
Covariance,\textquotedblright\ \textit{Annals of Statistics}, 45, 1342-1374.
\end{thebibliography}

\end{document}
