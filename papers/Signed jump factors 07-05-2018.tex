
\documentclass[a4paper,11pt]{article}%
\usepackage[top=1.25in, bottom=1.25in, left=1.25in, right=1.25in]{geometry}
\usepackage[utf8]{inputenc}
% \DeclareUnicodeCharacter{2010}{-}% support older LaTeX versions
\usepackage[T1]{fontenc}
\usepackage{lmodern}
\usepackage[english]{babel}
\usepackage{color}
\usepackage{csvsimple}
\usepackage{tabularx}
\usepackage{subcaption}
\usepackage{tabulary}
\usepackage{array}
\newcolumntype{L}[1]{>{\raggedright\let\newline\\\arraybackslash\hspace{0pt}}m{#1}}
\newcolumntype{C}[1]{>{\centering\let\newline\\\arraybackslash\hspace{0pt}}m{#1}}
\newcolumntype{R}[1]{>{\raggedleft\let\newline\\\arraybackslash\hspace{0pt}}m{#1}}
\usepackage{booktabs}
\usepackage[flushleft]{threeparttable}
\usepackage{multirow}
\usepackage{siunitx}
\usepackage{graphicx}
\usepackage[singlelinecheck=false ]{caption}
\usepackage[font={small,it}]{caption}
%\usepackage[justification=centering]{caption}
\usepackage{mdframed}
\usepackage{times}
\usepackage{pdflscape}
\usepackage{adjustbox}
\usepackage{ragged2e}
%\usepackage{float}
%\usepackage{epstopdf}
%\floatstyle{boxed} 
%\restylefloat{figure}
\usepackage{hanging}
\usepackage[bottom]{footmisc}
\usepackage{geometry}
%\geometry{
%	a4paper,
%	total={170mm,257mm},
%	left=20mm,
%	top=20mm,
\geometry{
	a4paper,
	total={6in,7in},
	left=30mm,
	top=30mm,
	bottom=40mm
}
%\textwidth 16cm 
\newenvironment{sciabstract}{
\begin{quote} \bf}
{\end{quote}}

\renewcommand\refname{References and Notes}
\newcommand{\head}[1]{\textnormal{\textbf{#1}}}
\newcounter{lastnote}
\newenvironment{scilastnote}{
\setcounter{lastnote}{\value{enumiv}}
\addtocounter{lastnote}{+1}
\begin{list}
{\arabic{lastnote}.}
{\setlength{\leftmargin}{.22in}}
{\setlength{\labelsep}{.5em}}}
{\end{list}}

\begin{document}
	
\begin{center}

{\Huge{Disentangling the Effects of News, Small Jumps, and Large Jumps on Stock Return Predictability}}

\bigskip
\bigskip

{\Large

Bruce Mizrach, Norman R. Swanson, and Bo Yu

\bigskip 
\bigskip

Rutgers University

\bigskip

July 2018}

\end{center}	

\thispagestyle{empty}

\bigskip
\bigskip
	
\begin{abstract}
The relationship between stock and market news, the difference between good and bad news, and the connection between news and so-called jumps has been the subject of numerous papers in recent years (see e.g., Bollerslev, Li, and Zhao (2017), and the references cited therein). In order to further investigate the connection between news and jumps, we decompose jump semivariances into large and small jump components, using high frequency intraday stock return data. When predicting cross-sectional stock price returns, we find that downside large and small jump variations have stronger return predictability than their upside counterparts. Additionally, signed small jump variations, which capture asymmetric information, are the most important for predication, after controlling for risk factors and firm characteristics, when compared with a number of other risk measures. In particular, an investment strategy that buys stocks with the lowest signed (small) jump variations and sells stocks with the highest signed (small) jump variations generates up to 36 basis points in risk-adjusted weekly returns. Upon exploring the source of this alpha, we find that investors are willing to accept low returns for stocks with a large probability of having extremely large positive jumps. On the other hand, industries with lower (negative) signed jump variations are preferred, which means that investors like safer industries with smaller left-tail risk. Finally, we confirm the finding in Bollerslev, Li, and Zhao (2017) that the negative association between realized skewness and one-week ahead returns is reversed when controlling for signed jump variations. However, this reversal disappears when stocks are first sorted by signed large jump variations, indicating that the return predictability of large jumps is fully dominated by skewness. 
\end{abstract}

\bigskip 
\bigskip
\bigskip

\vspace{.1in}

\noindent {\it Keywords}: Forecasting, Integrated Volatility, High-Frequency Data, Jumps, Realized Skewness, Cross-Sectional Stock Returns, Signed Jump Variation.
	
\noindent {\it JEL Classification}: C22, C52, C53, C58.	

\renewcommand{\baselinestretch}{1.1}%
%TCIMACRO{\TeXButton{nomalsize}{\normalsize{}}}%
%BeginExpansion
\normalsize{}%
%EndExpansion
----------------------------------------

\noindent {\footnotesize Bruce Mizrach, Department of Economics, Rutgers University, 75 Hamilton Street, New Brunswick, NJ 08901, USA; mizrach@econ.rutgers.edu. Norman R. Swanson, Department of Economics, Rutgers University, 75 Hamilton Street, New Brunswick, NJ 08901, USA; nswanson@economics.rutgers.edu. Bo Yu, Department of Economics, Rutgers University, 75 Hamilton Street, New Brunswick, NJ 08901, USA; byu@econ.rutgers.edu. The authors are grateful to Mingmian Cheng, Yuan Liao, Xiye Yang and seminar participants at the 2018 Rutgers meeting on financial econometrics for comments and discussions that have been utilized in the preparation of this paper.}



\setlength{\baselineskip}{1.5\baselineskip}
\hyphenpenalty=5000 \tolerance=1000


\setcounter {page} {0}\newpage

\newpage

\section*{1  { } Introduction}

Theoretical models of the risk-return relationship anticipate that volatility should be priced, and that investors should demand higher expected returns for more volatile assets.  However, ex-ante risk measures are not directly observable, and must be estimated (see e.g., Rossi and Timmermann (2015)). Given the necessity of estimating volatility, various different risk estimators have been utilized in the empirical literature studying the strength and sign of the risk-return relationship. Unfortunately, the evidence from the literature is mixed, in the sense that researchers have found both negative and positive relationships between return and volatility. One possible reason for these surprisingly contradictory findings is that  
the risk-return relationship is nonlinear. Examples of papers pursuing this hypothesis include Campbell and Vuolteenaho (2004), who incorporate different factor betas based on good and bad news about cash flows and discount rates; and Woodward and Anderson (2009) who find that bull and bear market betas differ substantially across most industries. This research has helped to spawn the "smart-beta" approach to factor investing.\footnote{In 2017, Morningstar reported that this approach to investing has attracted over one trillion dollars in assets (see e.g., Jennifer Thompson, Financial Times, December 27, 2017).} In related research, Feunou, Jahan-Parvar, and T\'edongap (2012) model the effects of volatility in positive and negative return states separately. They define so-called disappointment aversion preferences, and show that investors should demand a higher return for downside variability. These authors find empirical support for their model in the U.S. and several foreign markets using a bi-normal GARCH process to estimate volatility.

In this paper, we focus on the importance of jumps in volatility for understanding the risk-return relationship. Much of the empirical research that explores the importance of jumps in this context focuses on estimation of continuous and jump variation components using nonparameteric realized measures constructed with high frequency financial data. A key paper in this area is Bollerslev, Li and Zhao (2017), who examine the relationship between signed jumps and future stock returns in the cross-section. They document that signed jump variations, which capture the asymmetric impact of upside and downside jump risks, are better predictors of returns for small and illiquid stocks.\footnote{ In a related earlier paper, Duong and Swanson (2015) construct both large and small jump measures based on some fixed truncation levels. They exploit the risk predictabilities of different jump measures using both index data and Dow 30 stocks and find that small jumps have more volatility predictability than large jump variations.} In the current paper, we add to this literature by decomposing jump variations into signed large and small components and evaluating the importance of these elements in a cross section of stock returns. Our analysis sheds new light on not only the importance of signed small jumps in the cross-section of equity returns, but also on the predictive content of realized skewness. Additionally, we isolate the forecast information contained in small and large jumps. This is done by using intraday data to compute weekly realized moments for equity returns, as well as to estimate various jump variation measures; and by analyzing quintile stock portfolios, controlling for various factors, such as the Fama-French and Carhart factors.

The motivation for our paper can be traced back to Yan (2011) and Jiang and Yao (2013), who show that large, infrequent jumps are priced in the cross-section of returns. Feunou, Jahan-Parvar, and Okou (2017) take the decomposition used by these authors one step further, and model jumps in the realized semi-variances of market returns. They construct a new measure of the variance risk premium, and find a strong positive premium for downside risk. Fang, Jian and Luo (2017) find a similar result for Chinese market returns. In a related line of research, various authors study 
the information content in the upside and downside variations. For example, Guo, Wang, and Zhou (2015) document that at the market level, a negative jump component in realized volatility predicts an increasing future equity premium. Bollerslev, Todorov, and Xu (2015) identify both left and right jump tail risks under the risk-neutral measure. They find that the left jump tail risk is an appropriate proxy for market fear. They also find that including a variance risk premium together with jump tail risk measures as predictors significantly improves market return forecasts. They also show that jump risk helps explain the high-low book-to-market and winners versus losers portfolio returns. 

Building on this literature, we decompose the jump variations into four distinct components depending on both the direction (semi-variances) and magnitude (large and small) of the jumps.\footnote{The methods that we implement to separate jump variations rely on recent advances in financial econometrics due to Andersen, Bollerslev, Diebold and Labys (2003), Andersen, Bollerslev, and Diebold (2007), Jacod (2008), Mancini (2009), Barndorff, Kinnebrock, and Shephard (2010), Todorov and Tauchen (2010), A\"{\i}t-Sahalia and Jacod (2012), and Patton and Shephard (2015).} More specifically, we decompose individual stock jump semi-variances into large and small components. High frequency intra-day data is used to construct various realized jump variations, including large upside/downside, small upside/downside, and the difference between upside large (small) and downside large (small) jump variations. We then investigate the relationship between these various jump measures and future returns, using stock portfolios sorted across quintiles.

The reason that we decompose jump semi-variances into large and small components is that this decomposition allows us to explore the possibility that they contain different information relevant to investing and return predictability. As Maheu and McCurdy (2004) note, large jumps may reflect important individual stock and market news announcements.  Smaller jumps (or continuous variations) may result from slower news dissemination or liquidity and strategic trading. In our analysis, we utilize the cross-section of individual stocks because aggregate index returns may mask small jump effects on return predictability. Indeed, many studies document that aggregation  may diversify away idiosyncratic small jumps in the cross-section (see e.g., A\"{\i}t-Sahalia and Jacod (2012) and Duong and Swanson (2015)). Our findings can be summarized as follows. 

First, we find broad evidence that disentangling large and small jump variations leads to a deeper understanding of the predictability of excess returns associated with jumps. For example, differences in signed small jump risk measures (e.g., the differences between small positive and negative jump variations) are important for explaining cross-sectional variation in future returns. Moreover, by separately examining small and large jumps, we are able to assess the extent to which the information contained in realized skewness is also contained in jumps, and vice-versa. 

Second, analysis of high minus low portfolio returns indicates that both large and small upside (downside) jump variations negatively (positively) predict subsequent weekly excess returns. Also, the difference between small positive and negative jumps (signed small jump variations) is negatively associated with subsequent weekly returns. This negative relationship remains, even after controlling for different return moments, the Fama-French and Carhart factors, and firm characteristics. By contrast, signed large jump variations do not have significant predictive content, in the cross-section. Moreover, our analysis of various double portfolio sorts indicates that the predictive information in signed large jump variations are also contained in realized skewness. Additionally, signed small jump variations are the main driver of signed jump variations, indicating that this component of variation has distinct information about future returns.

Third, an investment strategy that buys stocks with the lowest signed (small) jump variations and sells stocks with the highest signed (small) jump variations generates up to 36 basis points in risk-adjusted weekly returns. In order to further investigate the sources of this alpha, we double sort portfolios based on industry and individual stock signed jump variations. We find that investors are willing to accept low returns for stocks with a large probability of extremely large positive jumps. On the other hand, industries with lower (negative) signed jump variations are preferred, which means that investors like safer industries with smaller left-tail risk.  

Fourth, our findings are closely related to those of Amaya, Christoffersen, Jacobs, and Vasquez (2015) and Bollerslev, Li, and Zhao (2017). In particular, we confirm the result in Amaya, Christoffersen, Jacobs, and Vasquez (2015) that realized skewness is negatively associated with future returns in the cross-section. We also confirm the finding in Bollerslev, Li, and Zhao (2017) that the negative association between realized skewness and one-week ahead returns is reversed when controlling for signed jump variations. However, this reversal disappears when stocks are first sorted by signed large jump variations, indicating that the return predictability of large jumps is fully dominated by skewness. This reversion is still significant when controlling for signed small jump variations, especially for value-weighted portfolios. On the other hand, when we sort stocks by signed large/small jump variations after controlling for skewness, the negative relationship remains only between small jump variations and future returns. These results again suggest that signed small jump variations have distinct information about future returns. 

Finally, our findings depend strongly on the threshold used for separating jump variations. When we use a four or five standard deviation truncation level to filter out large rare jumps, signed small jump measures have higher predictive power, in the cross-section; and even stronger predictability than that of signed total jump variations. This finding is more pronounced for large capitalization stocks (which are more liquid and have faster price discovery). Thus, information in large infrequent jump variations is short-lived and not very useful for weekly return predictability. On the other hand, small jump variations are more persistent, and are more informative in the cross-section - a finding that is consistent with the out-of-sample volatility forecasting evidence discussed in Duong and Swanson (2015).

The rest of this paper is organized as follows. In Section 2 we discuss the model setup and define the jump risk measures that we utilize. Section 3 contains a discussion of the data used in our empirical analysis, and highlights key summary statistics taken from our dataset; we construct different variances and control variables, together with the summary statistics of each realized measure. Section 4 includes a discussion of our main empirical findings, including a discussion of results based on single portfolio sorts, double portfolio sorts, and cross-sectional Fama-MacBeth regressions. Section 5 contains concluding remarks.   	  

\section*{2  { } Model Setup and Estimation Methodology}

Following A\"{\i}t-Sahalia and Jacod (2012), assume that the log price, $X_t$, of a security follows an It\^o semimartingale, formally defined as: 
$$X_t=X_0+\int_0^tb_sds+\int_0^t\sigma_sdW_s+\int_0^t\int_{\{|x|\le\epsilon\}}x(\mu-\nu)(ds,dx)+\int_0^t\int_{\{|x|\ge\epsilon\}}x\mu(ds,dx),$$
where $b$ and $\sigma$ denote the drift and diffusive volatility processes, respectively; $W$ is a standard Brownian motion; $\mu$ is a Poisson random measure with its compensator $\nu$; and $\epsilon$ is the (arbitrary) fixed cutoff level (threshold) used to distinguish between large and small jumps. As pointed out in A\"{\i}t-Sahalia and Jacod (2012), the continuous part of this model (i.e., the $\int_0^t\sigma_sdW_s$ term) captures normal hedgeable risk of the asset. The ``big jumps'' part of the model (i.e., the $\int_0^t\int_{\{|x|\ge\epsilon\}}x\mu(ds,dx)$ term) may capture big news-related events such as default risk, and the ``small jumps'' part of the model (i.e., the $\int_0^t\int_{\{|x|\le\epsilon\}}x(\mu-\nu)(ds,dx)$ term) may capture large price movements on the time scale of a few seconds. If jumps are summable (e.g., when jumps have finite activity, so that $\sum_{s\le t}\Delta X_s < \inf$, for all $t$\footnote{A jump process has finite activity when it makes a finite number of jumps, almost surely, in each finite time interval, otherwise it is said to have infinite activity.}), then   
the size of a jump at time $s$ is defined as $\Delta X_s = X_s-X_{s-}.$ 
In this context, the ``true'' price of risk is often defined by the quadratic variation, $QV_t$, of the process $X_t$. Namely,  
$$QV_t=\int_0^t\sigma_s^2ds+\sum_{s\le t}\Delta X^2_s,$$
where the variation of the continuous component (i.e., the integrated volatility) is given by $IV_t=\int_0^t\sigma_s^2ds$, and the variation of the price jump component is given by  $QJ_t=\sum_{s\le t}\Delta X^2_s $. 

In the sequel, intraday stock returns are assumed to be observed over equally spaced time intervals in a given day, where the sampling interval is denoted by $\Delta_n$, and the number of intraday observations is $n$. Thus the intraday log-return over the $i$th interval is defined as  $$r_{i,t}=X_{i\Delta_n,t}-X_{(i-1)\Delta_n,t}.$$
It is well known that when the sampling interval goes to zero, the realized volatility, $RV_t$, which is calculated by summing up all successive intraday squared returns, converges to $QV_t$, as $n \rightarrow \inf$, where 
$$RV_{t}=\sum_{i=1}^{n}r_{i,n}^{2}\rightarrow_{u} QV_t=IV_t+QJ_t,$$
where $\rightarrow_{u}$ denotes uniform convergence in probability.

To separate jump variation from integrated volatility, Andersen, Bollerslev and Diebold (2007) show that the jump and continuous components of realized variance can be constructed as: 
$$RVJ_t=max(RV_t-\widehat{IV_t},0)$$
and
$$RVC_t=RV_t-RVJ_t,$$
respectively, where $\widehat{IV_t}$ is an estimator of $\int_0^t\sigma_s^2ds$. Following Barndorff-Nielsen and Shephard (2004), and Barndorff-Nielsen, Graverson, Jacod, Podolskij, and Shephard (2006), we use tripower variation to estimate the integrated volatility. In particular, define  $$\widehat{IV_t}=V_{\frac{2}{3},\frac{2}{3},\frac{2}{3}}\mu_{\frac{2}{3}}^{-3},$$
where $\mu_q=E(|Z|^q)$ is the $q$th absolute moment of the standard normal distribution, and $$V_{m_1,m_2,...m_k}=\sum_{i=k}^n |r_{i,t}|^{m_1}|r_{i-1,t}|^{m_2}...|r_{i-k+1,t}|^{m_k},$$ where $m_1,$ $m_2$ ...$m_k$ are positive, such that $\sum_1^km_i=q$. Based on the above decomposition approach, Duong and Swanson (2011, 2015) separate jump variation into large and small variations, using various truncation levels, $\gamma$. In particular, they define realized large and small jump variations as: 
$$RVLJ_{\gamma,t}=min(RVJ_t, \sum_{i=1}^n r_{i,t}^2I_{|r_{i,t}|\ge\gamma})$$
and
$$RVSJ_{\gamma,t}=RVJ_t-RVLJ_{\gamma,t},$$
respectively, where $I(\cdot )$ denotes the indicator function, which equals one if the absolute return is larger than the truncation level, and is otherwise equal to zero. 

We are also interested in upside and downside variations associated with positive and negative returns. Thus, following, Barndorff-Nielsen, Kinnebrock, and Shephard (2010) we construct realized semi-variances, defined as: 
$RS_t^+=\sum_{i=1}^n r_{i,t}^2I_{\{r_{i,t}>0\}},$  
$RS_t^-=\sum_{i=1}^n r_{i,t}^2I_{\{r_{i,t}<0\}},$
and $RV_t=RS_t^++RS_t^-.$ They show that the upside and downside semi-variances ($RS_t^+$ and $RS_t^-$, respectively) each converge to the sum of one-half of the integrated volatility and the corresponding jump variation. Namely, 
$$RS_t^+\rightarrow_u \frac{1}{2}\int_0^t\sigma_s^2ds+\sum_{s \le t}\Delta X_s^2I_{\{\Delta X_s>0\}}$$
and 
$$RS_t^-\rightarrow_u \frac{1}{2}\int_0^t\sigma_s^2ds+\sum_{s \le t}\Delta X_s^2I_{\{\Delta X_s<0\}}.$$ 
Given our use of tripower variation
to measure the integrated volatility, we can thus construct upside and downside jump variations as follows: 
\begin{equation} 
RVJP_t=max(RS_t^+-\frac{1}{2}\widehat{IV_t},0) \label{eq1}
\end{equation}
and
\begin{equation}
RVJN_t=max(RS_t^--\frac{1}{2}\widehat{IV_t},0). \label{eq2}
\end{equation}
In addition, signed jump variations can be calculated as the difference between these upside and downside jump measures,
\begin{equation}
SRVJ_t=RVJP_t-RVJN_t. \label{eq3}
\end{equation} This measure captures asymmetry in upside and downside jump variations. 

In our analysis to follow, we further decompose upside and downside jump variations into large and small components using the thresholding method discussed in Mancini (2009). In particular, upside large jump variation based on truncation level $\alpha_n$ is defined as:
\begin{equation} 
RVLJP_t=min(RVJP_t,\sum_{i=1}^n{r_{i,t}^2}I_{\{r_{i,t}>\alpha_n\}}). \label{eq4}
\end{equation}
The threshold, $\alpha_n$, is constructed iteratively by estimating $\alpha_n^{(i)}=\alpha\sqrt{\frac{1}{t}\widehat{IV}_t^{(i)}}\Delta_n^{0.49}$, and is thus data-driven, accounting for the time-varying diffusive spot volatility of different stocks in the cross section.\footnote{For each stock, bi-power variation is used as the initial value for the integrated volatility $\widehat{IV}_t^{(0)}$, and $\widehat{IV}_t^{(i)}$ is estimated using truncated bi-power variation with threshold $\alpha_n^{(i-1)}$. This iterative process stops when $|\widehat{IV}_t^{(i)}-\widehat{IV}_t^{(i-1)}|$ is smaller than $5\%\times\widehat{IV}_t^{(i-1)}$. When tri-power variation is used instead of bi-power variation in our calculations, the results reported in Section 4 are qualitatively the same as those discussed in the sequel.} Similarly, downside large jump variation is defined as:
\begin{equation}   RVLJN_t=min(RVJN_t,\sum_{i=1}^n{r_{i,t}^2}I_{\{r_{i,t}<-\alpha_n\}}), \label{eq5}
\end{equation}
and signed large jump variation (i.e., large jump asymmetry) is measured using
\begin{equation}
SRVLJ_t=RVLJP_t-RVLJN_t. \label{eq6}
\end{equation} Corresponding small jump variations are defined as the differences between jump measures and large jump variations. Namely,
\begin{equation} 
RVSJP_t=RVJP_t-RVLJP_t \label{eq7}
\end{equation}
and
\begin{equation}
RVSJN_t=RVJN_t-RVLJN_t. \label{eq8}
\end{equation}
 Signed small jump variation is defined as
\begin{equation}
SRVSJ_t=RVSJP_t-RVSJN_t. \label{eq9}
\end{equation} 
 In order to analyze the predictability of various jump measures in the cross section, we normalize each of the jump variations discussed above by total realized variation. 

In order to measure skewness and kurtosis, we also construct higher order realized return moments. Following Amaya, Christoffersen, Jacobs, and Vasquez (2015),
\begin{equation}
RSK_t=\frac{\sqrt{n}\sum_{i=1}^n r_{i,t}^3}{RV_t^{\frac{3}{2}}}, \label{eq10}
\end{equation} standardized daily skewness is defined as: 
and normalized daily realized kurtosis is defined as:
\begin{equation}
RKT_t=\frac{n\sum_{i=1}^n r_{i,t}^4}{RV_t^{2}}. \label{eq11}
\end{equation} 

Finally, it should be noted that we follow Amaya, Christoffersen, Jacobs, and Vasquez (2015) and Bollerslev, Li, and Zhao (2017), and conduct our cross-sectional analysis at the weekly frequency. In particular, on each Tuesday, we compute the corresponding weekly measures as follows: 
$RV_t^W=(\frac{252}{5}\sum_{i=0}^4RV_{t-i})^{1/2}$ and
$RM_t^W=\frac{1}{5}(\sum_{i=0}^4RM_{t-i}),$
where $RM_t^W$ denotes the relevant daily realized measure defined above. Hereafter, we shall drop the superscript ``W'' for the sake of notational brevity. All of the descriptors used to denote the various realized measures constructed in our empirical analysis are summarized in Table 1.

\section*{3  { } Data}
We utilize high frequency trading data obtained from the consolidated Trade and Quote (TAQ) database. In particular, we analyze all stocks in the TAQ database that are listed on the NYSE, Amex, and NASDAQ stock exchanges. There are 15,585 unique stocks during the 1,246 weeks analyzed in this paper.\footnote {In some cases, multiple TAQ symbols are matched with a unique Center for Research in Security Prices (CRSP) PERMNO. Over each quarter, the TAQ symbol which has the most observations is kept and the other overlapping observations are dropped.} The sample period is from January 4, 1993 to December 31, 2016. Intraday prices are sampled at five minute intervals from 9:30 a.m. to 4:00 p.m. from Monday to Friday. Overnight returns are not considered in this paper, and days with less than 80 transactions at a 5 minute frequency are eliminated. All high frequency data used in this paper are cleaned to remove trades outside of exchange hours, negative or zero prices or volumes, trade corrections and non-standard sale conditions, using the methodology described in Appendix A.1 in Bollerslev, Li, and Zhao (2017).

We constructed two variants of our dataset. The first is cleaned as discussed above. The second classifies five minute intraday returns greater than 15\% as abnormal and replaces them with zeros. In the sequel, results based on analysis of the second dataset are reported. However, results based on utilization of the first dataset are qualitatively the same; and indeed key return results reported in this paper generally change by 1 basis point or less when the former dataset is used in our analysis. Complete results are available upon request from the authors. 
 
Daily and monthly returns, and adjusted numbers of shares for individual securities are collected from the CRSP database. Delisting returns in CRSP are used as returns after the last trading day. Daily Fama-French and Carhart four factor (FFC) portfolio returns are obtained from Kenneth R. French's website. 

Following Amaya, Christoffersen, Jacobs, and Vasquez (2015) and Bollerslev, Li, and Zhao (2017), we also construct various lower frequency firm level variables that might be related to future returns, such as the market beta (BETA), the firm size, the book-to-market ratio (BEME), momentum (MOM), short-term reversals (REV), idiosyncratic volatility (IVOL), coskewness (CSK), cokurtosis (CKT), maximum (MAX) and minimum (MIN) daily return in the previous week, and the Amihud (2002) illiquidity measure (ILLIQ). For a list of these variables, refer to Table 1. For a detailed description of these variables, including the methodology used to construct them, see Appendix A.2 in Bollerslev, Li, and Zhao (2017). 

Finally, following the Fama-French classification, we group stocks into 49 industries based on their SIC codes, which are obtained from CRSP. 
 
\subsection*{3.1  { }  Unconditional distributions of realized measures}

Figure 1 displays kernel density estimates of the unconditional distributions of each of our realized measures, across all firms and weeks. The top two panels in the figure show the distributions of signed jump variation and realized skewness. Both of these distributions are approximately symmetric and peaked around zero. The skewness distribution is more fat-tailed, however.\footnote{The kurtosis of signed jump variation is 4.36. For realized skewness, the analogous statistic is 12.04.} The middle two panels of Figure 1 display the distributions of signed large and small jump variations. Similar to signed jump variation, both signed large and small jumps are approximately symmetric around zero, but signed small jump variations are less fat-tailed.\footnote{The unconditional kurtosis is 6.43 and 3.51, for signed large and small jump variations, based on truncation level $\alpha_n^1 $; and 8.87 and 3.09 based on truncation level $\alpha_n^2$, respectively.} Consistent with the results in Amaya, Christoffersen, Jacobs, and Vasquez (2015) and Bollerslev, Li, and Zhao (2017), realized volatility and realized kurtosis are both right skewed and very fat-tailed, as shown in the bottom two panels of the figure.\footnote{The kurtosis is 15.85 and 27.24, for the unconditional distribution of realized volatility and realized kurtosis, respectively.} 

Figure 2 shows the time variation in the cross-sectional distribution of each realized measure using 10-week moving averages. In particular, 10th, 50th, and 90th percentiles for each realized measure in the cross section are plotted. Thus, dispersion at any given time in these plots reflects information about the cross-sectional distribution of the realized measure. Inspection of Panels A and B in the figure reveal that signed jump variation and realized skewness have stable dispersion, for all three cross-sectional percentiles, over time, while the cross-sectional dispersion in realized volatility and kurtosis are rather time-dependent (see Panels C and D). Additionally, similar to the cross-sectional distribution of signed jump variation, the percentiles for signed large and small jump variations are quite steady over time, as indicated in Panels E-H. 

\color{black}

\subsection*{3.2  { } Summary statistics and portfolio characteristics}

Table 2 contains various summary statistics for all of the realized measures summarized in Table 1. In Panel A, the cross-sectional means and standard errors for each of the realized measures is given. This is done for two different truncation levels, denoted as $\alpha_n^{1}=4\sqrt{\frac{1}{t}\widehat{IV}_t^{(i)}}\Delta_n^{0.49}$ and 
$\alpha_n^{2}=5\sqrt{\frac{1}{t}\widehat{IV}_t^{(i)}}\Delta_n^{0.49}$. As might be expected, jump variation is quite sensitive to the choice of $\alpha_n$. For example, the (normalized) mean of RVSJP (positive (upside) small jump variation) increases from 0.1179 to 0.1712 when the threshold is increased from $\alpha_n^{1}$ to $\alpha_n^{2} $. Needless to say, various measures remains the same, as they are independent of $\alpha_n$. 

Panel B of Table 2 contains cross-sectional correlations for all of the realized measures. In accord with the findings reported by Amaya, Christoffersen, Jacobs, and Vasquez (2015) and Bollerslev, Li, and Zhao (2017), signed jump variation (SRVJ) and realized skewness (RSK) are highly correlated with each other and have significantly positive correlations with the short term reversal variable (REV); as well as with maximium (MAX) and minimum (MIN) daily returns in the previous week. 

Interestingly, we also find that signed large jump variation (SRVLJ) is highly correlated with SRVJ and with RSK. However, signed small jump variation (SRVSJ) has lower correlation with SRVJ and much smaller positive correlation with RSK. This finding is consistent with our finding discussed below that realized skewness captures information that is primarily contained in large jumps; and serves as an important distinction between the findings in this paper and those reported in the papers discussed above.

Table 3 complements Table 2 by sorting stocks into quintile portfolios based on different realized measures. On each Tuesday, stocks are ranked by the realized variation measures, and we calculate the equal-weighted averages of each firm characteristic in the same week. Panels A, D, G and K report summary statistics for portfolios sorted by SRVJ, SRVLJ, SRVSJ, and RSK, respectively. Consistent with the correlations contained in Table 2, firms with larger signed large and small jump variations tend to have higher signed jump variation, realized skewness, REV, MAX and MIN. Firms with high realized volatility and realized kurtosis (see Panels J and L) tend to be illiquid and small, while firms with high large and small upside/downside jump variations have no substantial relations with firm size and liquidity (see Panels D through I). 

\section*{4  { } Empirical Results }

In this section, results based on stocks that are sorted into quintile portfolios based on a single different realized measure are first reported. These single (univariate) portfolio sort results are collected in Tables 4 to 7. Results based on double sorts are reported (in Tables 8 to 15). We assume a weekly holding period, and return calculations reported in the tables are carried out as follows. At the end of each Tuesday, stocks are sorted into quintile portfolios based on different realized variation measures (see Panel A of Table 1). We then calculate equal-weighted and value-weighted portfolio returns over the subsequent week. We report the time series average of these weekly returns for each portfolio (these returns are called ``Mean Return'' in the tables) In addition, we regress excess return of each portfolio on the Fama-French and Carhart (FFC4) factors to control for systematic risks, using regression of the form
\begin{equation}
r_{i,t}-r_{f,t}=\alpha_i+\beta_{i}^{MKT}(MKT_t-r_{f,t})+\beta_{i}^{SMB}SMB_t +\beta_{i}^{HML} HML_t +\beta_{i}^{UMD} UMD_t+\epsilon_{i,t} \label{eq12}
\end{equation}
where $r_{i,t}$ denotes the weekly return for firm i, $ r_{f,t}$ is the risk-free rate; and $MKT_t$, $SMB_t$, $HML_t$, and $UMD_t$ denote FFC4 market, size, value and momentum factors, respectively. The intercepts from these regressions (called ``Alpha'' in our tabulated results), measure risk-adjusted excess returns, and are also reported in Tables 4 to 15. Needless to say, our objective in these tables is to assess whether predictability exists, after controlling for various systematic risk factors. Finally, in Tables 16 and 17, we report the results of cross-sectional (firm level) Fama-MacBeth regressions used to investigate return predictability when simultaneously controlling for multiple realized measures and firm characteristics. 

\subsection*{4.1  { }  Single (univariate) portfolio sorts based on realized measures}

Consider Table 4. Recall that the ``Mean-Return'' in this table is an average taken over our entire time series of equal-weighted and value-weighted portfolio returns, for single sorted portfolios based on positive jump variation (RVJP), negative jump variation (RVJN) and signed jump variation (SRVJ). Values in parentheses are Newey-West $t$-statistics (see Bollerslev, Todorov, and Xu (2015) and Petersen (2009) for further discussion). Panel A provides results for portfolios sorted by RVJP. Inspection of the entries in this panel indicate that mean returns and alphas of high minus low portfolios are all negative, indicating a negative association between RVJP and subsequent stock returns. Interestingly, the alpha of -5.96 basis points (bps) is insignificant for the high minus low spread for the equal weighted portfolio, while the mean return of -6.16 bps is insignificant for the high minus low spread for the value weighted portfolio. 

The lack of statistical significance for some of the mean return values reported in Panel A does not characterize our findings when negative and signed jump variations are utilized for sorting. Moreover, the magnitudes of the mean returns and alphas are usually three or more times larger when sorting on negative and signed jump variation (to see this, turn to Panels B and C of Table 4). In Panel B, the high minus low spread of mean returns equals 32.34 bps, with a $t$-statistic of 5.85 for the equal-weighted portfolio, and 15.25 bps with a $t$-statistic of 3.77 for the value-weighted portfolio. Moreover, both equal-weighted and value-weighted portfolios generate significant positive abnormal future returns measured by the alphas. These results clearly point to a statistically significant positive association between negative jump variation and the following week's returns. 

Panel C in Table 4 contains results for portfolios sorted by signed jump variation. The negative high minus low spreads indicate a statistically significant negative association between signed jump variation and future returns. In particular, a strategy buying stocks in the lowest signed jump variation quintile and selling stocks in the highest signed jump variation quintile earns a mean return of 36.33 bps with a $t$-statistic of 8.56 each week for the equal-weighted portfolio and 24.68 bps with a $t$-statistic of 5.70 for the value-weighted portfolio. These results are consistent with the results reported in Bollerslev, Li, and Zhao (2017). 

A key question that we provide evidence on in this paper is whether the results summarized in Table 4 carry over to the case where large and small jump variation is separately sorted on. First, consider large jumps. Table 5 reports the results for portfolios sorted by positive, negative and signed large jump variation, respectively. Similar to positive jump variation, positive large jumps negatively predict subsequent returns, but the predictability is not significant, regardless of the truncation level ($\alpha_n$) used to separate large and small jumps, and regardless of portfolio weighting used. This is evidenced by the fact that the $t$-statistics for mean returns and alphas of high minus low portfolios all indicate insignificance, at a 5\% testing level, regardless of truncation level. Thus, there is no ambiguity, as in Panel A of Table 4. Positive jump variation is not a significant predictor, under our large jump scenario. On the other hand, we shall see that sorting on large and small negative variations yields significant excess returns, as does sorting on small positive jump variations, under equal weighting. 

As just noted, equal-weighted high minus low portfolios sorted on large negative jump variations generate significant positive returns and alphas (see Panel B of Table 5). However, analogous returns and alphas under value weighting are not significant. Signed large jump variation is sorted on in Panel C of Table 5. Signed large jump variation is useful for undertaking a long-short trading strategy based on the difference between large upside and downside jump variations. Inspection of the results in this panel of the table reveals that the high minus low spread for the equal weighted portfolio generates an average risk-adjusted weekly return of -25.31 bps with a $t$-statistics of -8.01 and -10.26 bps with a $t$-statistics of -3.13 for the value-weighted portfolio, for truncation level equal to $\alpha_n^1$. Results based on $\alpha_n^2$ (i.e., our larger truncation level) are also significant, although magnitudes are lesser and only for our equal weighted portfolio. In particular, observe that when large jump variations are constructed using $\alpha_n^2$, the high minus low spreads for value-weighted portfolios sorted by downside or signed large jump variations are insignificant, suggesting that small firms have stronger relationships (than larger firms) between signed (or negative) large jump variations and subsequent returns. This may be due to the fact that smaller firms are in some ways more susceptible to changing market conditions than larger firms.

Table 6 summarizes results analogous to those reported in Table 5, but for positive, negative and signed small jump variations. Similar to large jump measures, positive and signed small jump variations negatively predict future returns, and negative small jump variations positively predict returns in the following week. By contrast, the differences in average (risk-adjusted) returns between equal-weighted and value-weighted long-short portfolios based on RVSJP and RVSJN are smaller than those for portfolios based on large jumps (compare the entries for the high minus low quintiles under the two weighting schemes in Panels A and B of Table 6 with like entries in Panels A and B of Table 5). These results indicate that big firms have a stronger relationship between small jump variations and future returns than that between large jumps and subsequent weekly returns. Since stocks for big firms are more liquid and price discovery more rapid, the predictabilities of large jumps are much weaker or insignificant for big firms. This finding is in line with Bollerslev et al. (2017), who document that the predictability of signed jump variation is stronger for small and illiquid firms and is driven by investor overreaction. In addition, when using our larger truncation level, $\alpha_n^2$, the value-weighted high minus low spreads based on signed small jump variations are larger than those based on signed total jump variations and signed large jump variations.\footnote{ We carried out our analysis using a third truncation level, $\alpha_n^3=6\sqrt{\frac{1}{t}\widehat{IV}_t}\triangle_n^{0.49}$, which is higher than $\alpha_n^1$ and $\alpha_n^2$. Under this scenario, the equal-weighted high-low spreads based on the new signed small jump variations are smaller than those for portfolios sorted by SRVJ, and the value-weighted spreads are smaller than those for portfolios based on SRVSJ constructed by using $\alpha_n^2$.} This result implies that a long-short strategy associated with signed small jump variations can generate the highest value-weighted risk-adjusted returns, given the use of an appropriate truncation level to separate small and large jumps. 

Table 7 reports results for portfolios sorted by realized volatility, realized skewness, realized kurtosis, and continuous variance. Consistent with the results in Amaya, Christoffersen, Jacobs, and Vasquez (2015) and Bollerslev, Li, and Zhao (2017), there is a significant negative relationship between realized skewness and future returns, while the association is not significant between either realized volatility or realized kurtosis and returns in the following week, regardless of portfolio weighting scheme. In addition, continuous variance significantly and negatively predicts one-week ahead returns for equal-weighted portfolios, but this negative association is not significant for value-weighted portfolios.  

\subsection*{4.2  { }  Cumulative returns and Sharpe ratios}

Not surprisingly, our findings based on univariate portfolio sorts suggest that strategies that utilize different realized measures deliver different risk-adjusted average returns.  In order to investigate this result further, we calculate cumulative returns and Sharpe ratios for short-long portfolios, sorted on various risk measures that are described in Table 1, including SRVJ, RSJ, SRVLJ, SRVSJ, and RSK. In addition, for comparison purposes, we also carry out our analysis using the relative signed jump variation measure (called RSJ) that is examined by Bollerslev, Li, and Zhao (2017). Our experiments are carried out as follows. Beginning in January 1993, various short-long portfolios are constructed, with an initial investment of \$1. These portfolios are re-balanced and accumulated at a weekly frequency, until the end of 2016.\footnote{Our cumulative returns calculations do not include the risk-free rate. For a definition of cumulative returns both with and without the weekly risk-free rate, see Bollerslev, Li, and Zhao (2017).}  
Figure 3 plots portfolio values over time. Consistent with our results based on single portfolio sorts, inspection of the plots in this figure indicates that for equal-weighted portfolios sorting on signed jump variations (SRVJ) yields the largest portfolio accumulations; and for value-weighted portfolios, sorting on signed small jump variations (SRVSJ) yields the largest portfolio accumulations.
\footnote{Note that RSJ, which measures the same signed jump variation as SRVJ, although using different estimation methodology, generates the highest cumulative return for equal-weighted portfolios, but is dominated by SRVLJ for value-weighted portfolios.}

Now, consider the Sharpe ratios reported below, which are reported for various jump measures, and are constructed based on truncation level $\alpha_n^{2}=5\sqrt{\frac{1}{t}\widehat{IV}_t}\triangle_n^{0.49}$.

\begin{table}[htbp!]
	\centering
	Sharpe Ratios
	
	\resizebox{0.6\textwidth}{!}{
		\begin{tabular}{lrccccc}
			\toprule
			\multicolumn{2}{r}{} & SRVJ  & RSJ & SRVLJ & SRVSJ & RSK    \\
			\midrule
			\multicolumn{2}{l}{Equal-Weighted} & 1.8498 &2.0560 & 1.4760  & 1.7561  & 1.8986   \\
			\multicolumn{2}{l}{Value-Weighted } & 1.1278 & 1.1454 & 0.1983  & 1.3014  & 0.8425    \\
			\bottomrule\\
	\end{tabular}}\\	 

\end{table}

The entries in this table are Sharpe ratios for equal and value-weighted short-long portfolios constructed using SRVJ, RSJ, SRVLJ, SRVSJ, and RSK. Recall that RSK is realized skewness (see Table 1 for definitions of these measures). The sample of stocks used for Sharpe ratio calculations includes all NYSE, NASDAQ and AMEX listed stocks for the period January 1993 to December 2016. At the end of each Tuesday, all of the stocks in the sample are sorted into quintile portfolios based on ascending values of various realized risk measures. A low-high spread portfolio is then formed as the difference between portfolio 1 and portfolio 5, and held for one week, where 1 and 5 refer to quintiles, as in Tables 1 to 7. The Sharpe ratio is calculated with the one-week ahead returns. 

Interestingly, for equal-weighted portfolios, the RSK-based short-long strategy yields the second highest Sharpe ratio (i.e., 1.8986), although the ratio of 1.8498 for SRVJ is approximately the same, and the ratio of 2.0560 for RSJ is higher. Still, the success of the RSK measure is likely due to its relatively stable performance, compared with other jump-based strategies. This finding is similar to the findings discussed in Xiong, Idzorek, and Ibbotson (2016), who show that tail-risks can be substantially reduced by forecasting skewness. Note also that the signed small jump variation (SRVSJ) based portfolio has the highest Sharpe ratio, among all value-weighted portfolios. However, it is clear that all equal-weighted portfolios outperform their corresponding value-weighted counterparts. This result is consistent with the finding discussed above that small and illiquid firms tend to react more strongly to realized risk measures. 

\subsection*{4.3  { }  Double portfolio sorts based on realized measures}

Bollerslev, Li, and Zhao (2017) document that the negative association between realized skewness and one-week ahead returns is reversed when controlling for the signed jump variation. To further investigate the relationship between skewness and different jump variations, we use double portfolio sorts to control for different effects that are associated with cross-sectional variation in future returns. 

Panel A of Table 8 reports average returns and corresponding $t$-statistics for 25 portfolios sorted by SRVJ (signed jump variation), controlling for realized skewness (RSK). At the end of each Tuesday, stocks are first sorted into quintiles based on realized skewness, and then within each quintile portfolio, we further sort stocks into quintiles based on signed jump variations. We also report the equal- and value-weighted returns in the following week and Fama-French and Carhart four-factor alphas for the long-short portfolios and the averaged portfolios across quintiles. Inspection of the results in this table indicates that the negative association between SRVJ and future returns still exists, after controlling for RSK, indicating that there is unique predictive information contained in signed jump variation. Panel B in this table reports results for portfolios sorted first by SRVJ and then by RSK. The high minus low spreads of the averaged portfolios are positive after controlling for SRVJ, confirming the results reported in Bollerslev, Li, and Zhao (2017). 

Table 9 contains results for portfolios sorted by SRVLJ (signed large jump variations) after controlling for RSK. As noted above, the negative association between SRVLJ and future returns is reversed after controlling for skewness. By contrast, this issue doesn't exist for portfolios sorted by SRVSJ (signed small jump variations) when controlling for skewness, as shown in Table 10, indicating that signed small jump variations have unique information about future return premia. However, first accounting for skewness negates the usefulness that signed large jump variations have for predicting future returns. This finding serves as an important distinction between the predictive content of large and small jumps. Whereas the former can be forecast by realized skewness, the latter cannot.

Finally, Tables 11 and 12 contain results for portfolios sorted on RSK, after controlling for SRVLJ and SRVSJ, respectively. Inspection of the entries in these tables indicates that the high minus low spreads are negative, except in select value weighted portfolio cases, when controlling for SRVSJ. This is not surprising since skewness captures information from both SRVLJ and SRVSJ, while the negative association between realized skewness and subsequent returns remains, when controlling for either SRVLJ or SRVSJ, in most cases. Of note is that this negative association disappears for some value-weighted portfolios, when controlling for SRVSJ, suggesting that signed small jump variations (especially for big firms) are the main driver of the signed total jump variations. These findings are consistent with the findings documented by Bollerslev and Todorov (2011) that S\&P 500 market portfolios tend to have symmetric jump tails (large jumps). 
  
\subsection*{4.4  { }  Using double portfolio sorts to examine stock-level versus industry-level predictability}

Similar to low risk investing, low signed jump variation investing (buying stocks with low signed jump variations and shorting stocks with high signed jump variations) can deliver significant risk-adjusted returns. To examine whether this investment strategy relies on industry betting or stock selection within industries (or both), we form double sorted portfolios based on industry-level and stock-level signed jump risk variations. In particular, each Tuesday we group stocks into 49 industries based on SIC codes. Industry-level signed jump risk is calculated as the value-weighted average of signed (large/small) jump variations for stocks within each industry. Thus, stocks in the same industry have the same industry signed (large/small) jump variation during a given week. Stock-level signed jump risk is calculated as outlined in the above. Double sorts are then used to investigate the selection effects at industry- and stock-level. Namely, stocks are sorted into 25 portfolios based on industry- and stock-level signed (large/small) jump variation quintiles. With this particular variety of sorting, results are independent of the order in which stocks are sorted.  

Figure 4 depicts the percentage of stocks in each portfolio (see Panel A), and the market capitalization in these portfolios (see Panel B). If industry-level selection and stock-level selection lead to different quintile portfolios (i.e. off-diagonal portfolios in the figures have non-zero membership), it is possible to separate these two effects using double sorts. Namely, there are different industry- and stock-level effects. Both panels indicate this to be the case. 

Tables 13 to 15 report our empirical findings based on our double portfolio sort experiments. In particular, Table 13 reports results for sorting done on signed jump variation (SRVJ), while Tables 14 and 15 report results for sorting done on signed large jump variations (SRVLJ) and signed small jump variations (SRVSJ), respectively.  Entries in the tables are mean returns and alphas, as in previous tables. However, in these tables we also report industry-level effects and stock-level effects. These are reported in the last two rows of entries in each panel of the tables. The first of these two rows, called ``Industry-Level Effect'' reports average high minus low returns and alphas by averaging across quintiles in the high minus low and alpha columns of the table (these are industry-level results).  The second of these two rows, called ``Stock-Level Effect'' reports average high minus low returns and alphas by averaging across quintiles in the high minus low and alpha rows of the table (these are stock-level results).
Summarizing, rows in these tables display portfolios formed by stocks in the same stock-level SRVJ, SRVLJ, or SRVSJ quintiles, while columns report results for portfolios formed by stocks in the same industry-level SRVJ, SRVLJ, or SRVSJ quintiles.

Turning to Table 13, notice, for example, that a strategy of buying stocks in the highest industry SRVJ quintile and selling stocks in the lowest industry SRVJ quintile generates an equal-weighted average return of 29.84 bps with a $t$-statistic of 5.74, and the corresponding value-weighed average return is 13.43 bps with a $t$-statistics of 2.62 (see Table 13). This finding is interesting, as it suggests that the negative association between SRVJ and future returns is reversed in the industry level. The equal-weighted average of the high minus low row (i.e., the average stock-level effect) is -40.61 bps with a $t$-statistic of -9.87 and the alpha is -40.50 bps with a $t$-statistic of -10.19, indicating that the stock-level effect is economically significant. At the stock-level, investors prefer stocks with high SRVJ, requiring lower returns under higher SRVJ, given that there is a large probability of extremely large positive jumps. By contrast, when sorting at the industry-level, investors are more interested in industry exposure with lower SRVJ, or in return distributions concentrated to the right. Lottery-like payoff exposure comes from individual stocks, not from industry bets. These results are mirrored in Tables 14 and 15, where SRVLJ and SRVSJ are the sorting measures. However, average stock- and industry-level returns and alphas are much higher under SRVSJ sorting than under SRVLJ sorting. For example, buying stocks in the highest industry SRVSJ quintile and selling stocks in the lowest industry SRVSJ quintile generates an equal-weighted average return of 8.59 bps with a $t$-statistic of 2.08 under SRVLJ sorting (see Table 14), versus an equal-weighted average return of 28.91 bps with a $t$-statistic of 5.32 under SRVSJ sorting (see Table 15). 

\subsection*{4.5  { }  Firm-level Fama-MacBeth regressions}

Tables 16 and 17 gather results based on firm-level Fama-Macbeth regressions, which we run in order to investigate the return predictability associated with variation measures, when controlling for multiple firm specific characteristics. Regressions are carried out as follows. At the end of each Tuesday, we run the cross-sectional regression, 
\begin{equation}
r_{i,t+1} =\gamma_{0,t} +\sum_{j=1}^{K_1}\gamma_{j,t}X_{i,j,t}+\sum_{s=1}^{K_2}\phi_{s,t}Z_{i,s,t}+\epsilon_{i,t+1},      \ \ \ t=1,...,T, \label{eq13} 
\end{equation}
where $r_{i,t+1}$ denotes the stock return for firm i in week t+1, $K_1$ is the number of potential variation measures, and $X_{i,j,t}$ denotes a relevant realized measure at the end of week $t$. In addition, there are $K_2$ variables measuring firm characteristics, which are denoted by $Z_{i,j,t}$. After estimating the cross-sectional regression coefficients on a weekly basis, we form the time series average of the resulting $T$ weekly $\widehat{\gamma}_{j,t}$ and $\widehat{\phi}_{s,t}$ values, in order to estimate the average risk premium associated with each risk measure. Namely, we construct
$$\widehat{\gamma}_{j}=\frac{1}{T}\sum_{t=1}^T \widehat{\gamma}_{j,t}, \ \ \ {\textrm{and}} \ \ \    \widehat{\phi}_{s}=\frac{1}{T}\sum_{t=1}^T \widehat{\phi}_{s,t}, \ \ \  {\textrm{for}} \ \ \ j=1,...,K_1, {\  }s=1,...,K_2.$$ 

Table 16 reports results for regressions on various realized variation measures for two different truncation levels, without controlling for firm specific characteristics. Consistent with our results based univariate sorting, signed jump variation (SRVJ) significantly negatively predicts cross-sectional variation, in these weekly returns regressions. Additionally, both signed large and small jump variations negatively predict future weekly returns. Finally, both large and small upside (downside) jump variations negatively (positively) predict subsequent weekly returns. However, when including measures that contain information from both large and small jump variations, as well as realized skewness, the negative association between skewness and future returns is reversed (see the results for the regressions labeled IX, XII, XV, XVI). In particular, skewness drives out signed large jump variation in regression XIII by reverting the negative association between the latter and future returns. If only small jumps are considered as control variables, skewness still negatively predicts future returns. This again indicates that signed small jump variation has unique and significant information about future returns.

Table 17 reports regression results for the same set of regressions in Table 16, but controlling for various firm specific characteristics, ranging from BETA to ILLIQ (see Table 1 for details). In these regressions, signed (small) jump variations are always significant. Additionally, skewness significantly negatively predicts future returns in regressions that only include small jump variations. This provides yet further evidence that signed small jump variation has unique and significant information about future returns, while large jumps have information in common with realized skewness.

 
\section*{5  { } Concluding Remarks }

In this paper, we add to the literature that explores the relationship between equity return and volatility. In particular, we focus on the strand of this literature that explores the data for evidence of asymmetry (non-linearity) in the return volatility trade-off. The idea in papers such as Ang, Chen, and Xing (2006) is that more accurate return predictions can be obtained in the cross-section by separately pricing upside and downside risk. We build on their paper, and on other key papers which show that sorting portfolios based on jump measures produces significant differences in cross-sectional returns.  Jiang and Yao (2013) find that large jumps, which are associated with significant market and stock  specific news, are priced; Unfortunately, Duong and Swanson (2015) find that large jumps are largely unpredictable, and this information may not be useful for portfolio selection. 

To explore the importance of jumps for predicting future returns, we decompose, following Bollerslev, Li and Zhao (2017), realized variation into upside and downside semi-variances (good and bad volatilities).  We take an additional step, and partition the semi-variances into large and small components. We find that both large and small upside (downside) jump variations negatively (positively) predict subsequent weekly returns. Additionally, signed (small) jump variations significantly negatively predict future weekly returns, based on results from single and double portfolio sorts and cross-sectional Fama-MacBeth regressions, with multiple control variables. We also add to the findings of Amaya, Christoffersen, Jacobs, and Vasquez (2015), by showing that jump variation is not just a proxy for skewness.  We start by confirming their result that realized skewness is negatively associated with future returns. Then, we show that by decomposing jump variations into large and small components, we see that large jumps are highly correlated with realized skewness and display similar dispersion, and this impacts the sign and significance of skewness in the cross-section. Moreover, we show that signed small jump variation is the main driver of signed jump risk, and it contains unique information about future returns. We also find that stocks with small signed jump variations generate relatively high returns. Finally, we decompose the sources of low signed (small) jump variation alphas into stock- and industry-level effects, using double portfolio sorts.

Our findings also indicate that investors prefer stocks with high probability of large positive jump variation, and tend to overweight safer industries. An equal-weighted long-short portfolio based on signed jump variation (SRVJ) generates a risk-adjusted weekly return of up to 36 basis points.  By contrast,  a value-weighted long-short portfolio using signed small jump variations, earns a risk-adjusted return of 27 basis points. A comparison of Sharpe ratios also indicates that equal-weighted portfolios outperform value weighted portfolios, with SRVJ and relative signed jump variation measures yielding portfolios that perform as well or better than portfolios constructed using realized skewness. \color{black}

In summary, our findings add to the growing body of evidence suggesting that pricing ``good'' and ``bad'' volatility leads to improved cross-sectional return predictions. Moreover, further decomposition of risk measures into not only positive and negative, but also large and small components leads to further improvements in predictability. Finally, we present new evidence suggesting that the marginal predictive content of jumps derives largely from small jumps, while large jumps appear to be a proxy for realized skewness.

\newpage
\section*{References}

\begin{hangparas}{1em}{1} 

A\"{\i}t-Sahalia, Y., Jacod, J., 2012. Analyzing the spectrum of asset returns: Jump and volatility components in high frequency data. \textit{Journal of Economic Literature} 50, 1007-1050.

%A\"{\i}t-Sahalia, Y., Xiu, D., 2016. Increased correlation among asset classes: Are volatility or jumps to blame, or both? \textit{Journal of Econometrics} 194, 205-219.

Amaya, D., Christoffersen, P., Jacobs, K., Vasquez, A., 2015. Does realized skewness predict the cross-section of equity returns? \textit{Journal of Financial Economics} 118, 135-167.

Amihud, Y., 2002. Illiquidity and stock returns: Cross-section and time-series effects. \textit{Journal of Financial Markets} 5, 31-56.

Andersen, T. G., Bollerslev, T., Diebold, F. X., 2007. Roughing it up: Including jump components in the measurement, modeling, and forecasting of return volatility. \textit{Review of Economics and Statistics} 89, 701-720.

Andersen, T. G., Bollerlsev, T., Diebold, F. X., Labys, P., 2003. Modeling and forecasting realized volatility. \textit{Econometrica} 71, 579-625.

%Andersen, T. G., Dobrev, D., Schaumburg, E., 2012. Jump-robust volatility estimation using nearest neighbor truncation. \textit{Journal of Econometrics} 169, 75-93.

Ang, A., Chen, J. and Xing, Y., 2006. Downside risk. \textit{Review of Financial Studies} 19 4, 1191-1239.

%Baker, M., Bradley, B., Taliaferro, R., 2014. The low-risk anomaly: A decomposition into micro and macro effects. \textit{Financial Analysts Journal} 70, 43-58.

Barndorff-Nielsen, O. E., Graversen, S. E., Jacod, J., Podolskij, M., Shephard,N., 2006. A central limit theorem for realised power and bipower variations of continuous semimartingales. In \textit{Stochastic Analysis to Mathematical Finance, The Shiryaev Festschrift} (Edited by Kabanov, Y., Liptser, R., and Stoyanov, J.), Springer Verlag, 33-68.

Barndorff-Nielsen, O. E., Kinnebrock, S., Shephard, N., 2010. Measuring downside risk: realised semivariance. In \textit{Volatility and Time Series Econometrics: Essays in Honor of Robert F. Engle} (Edited by T. Bollerslev, J. Russell and M. Watson), Oxford University Press, 117-136.

Barndorff-Nielsen, O. E., Shephard, N., 2004. Power and bipower variation with stochastic volatility and jumps. \textit{Journal of Financial Econometrics} 2, 1-37.

%Barndorff-Nielsen, O. E., Shephard, N., 2006. Econometrics of testing for jumps in financial economics using bipower variation. \textit{Journal of Financial Econometrics} 4, 1-30.

%Bollerslev, T., Li, S. Z., Todorov, V., 2016. Roughing up beta: Continuous vs. discontinuous betas, and the cross section of expected stock returns. \textit{Journal of Financial Economics} 120, 464-490.

Bollerslev, T., Li, S. Z., Zhao, B., 2017. Good volatility, bad volatility, and the cross-section of stock returns. Working Paper, Duke University. 

%Bollerslev, T., Patton, A.J., Quaedvlieg, R., 2017. Realized semicovariances: Looking for signs of direction inside the covariance matrix. Working Paper, Duke University. 
%
%Bollerslev, T., Tauchen, G., Zhou, H., 2009. Expected stock returns and variance risk premia. \textit{Review of Financial Studies} 22, 4463-4492.

Bollerslev, T., Todorov, V., 2011. Estimation of jump tails. \textit{Econometrica} 79, 1727-1783.

%Bollerslev, T., Todorov, V., 2011b. Tails, fears, and risk premia. \textit{Journal of Finance} 66, 2165-2211.

Bollerslev, T., Todorov, V., Xu, L., 2015. Tail risk premia and return predictability. \textit{Journal of Financial Economics} 118, 113-134.

Campbell, J.Y. and Vuolteenaho, T., 2004, Bad beta, good beta.
\textit{American Economic Review} 94, 1249-1275.

%Chang, B. Y., Christoffersen, P., Jacobs, K., 2013. Market skewness risk and the cross section of stock returns. \textit{Journal of Financial Economics} 107, 46-68.

%Corsi, F., 2009. A simple approximate long-memory model of realized volatility. \textit{Journal of Financial Econometrics} 7, 174-196.

%Drechsler, I., Yaron, A., 2011.  What’s vol got to do with it. \textit{Review of Financial Studies} 24, 1–45.

Duong, D., Swanson, N. R., 2011, Volatility in discrete and continuous-time models: A survey with new evidence on large and small jumps. \textit{Missing Data Methods: Advances in Econometrics} 27, 179-233.

Duong, D., Swanson, N. R., 2015. Empirical evidence on the importance of aggregation, asymmetry, and jumps for volatility prediction. \textit{Journal of Econometrics} 187, 606-621.

%Fama, E. F., French, K. R., 1993. Common risk factors in the returns on stocks and bonds. \textit{Journal of Financial Economics} 33, 3-56.

%Fan, J., Furger, A., Xiu, D., 2016. Incorporating global industrial classification standard into portfolio allocation: A simple factor-based large covariance matrix estimator with high-frequency data. \textit{Journal of Business and Economic Statistics} 34, 489-503.

Fang, N., Wen, J., and Luo, R., 2017, Realized semivariances and the variation of signed jumps in China’s stock market, \textit{Emerging Markets Finance and Trade} 53, 563–586.

Feunou, B., Jahan-Pravar, M. R., Okou, C., 2017. Downside variance risk premium. \textit{Journal of Financial Econometrics}, forthcoming.

Feunou, B., Jahan-Pravar, M. R., T\'edongap, R., 2012. Modeling market downside volatility. \textit{Review of Finance} 17, 443-481.

Guo, H.,  Wang, K., Zhou H., 2015.  Good jumps, bad jumps, and conditional equity premium.  Working Paper, University of Cincinnati.

%Han, B., Zhou, Y., 2011. Variance risk premium and cross-section of stock returns. Working Paper, University of Texas, Austin.

%Hansen, P. R., Lunde, A., 2006. Realized variance and market microstructure noise. \textit{Journal of Business and Economic Statistics} 24, 127-161.

Jacod J., 2008. Asymptotic properties of realized power variations and related functionals of semimartingales. \textit{Stochastic Processes and Their Applications} 118, 517-559. 

%Jacod, J., Todorov, V., 2009. Testing for common arrivals of jumps for discretely observed multidimensional processes. \textit{Annals of Statistics} 37, 1792-1838.

Jiang, G., Yao, T., 2013. Stock price jumps and cross-sectional return predictability. \textit{Journal of Financial and Quantitative Analysis} 48, 1519-1544.

%Li, J., Todorov, V., Tauchen, G., 2017. Jump regressions. \textit{Econometrica} 85, 173-195.
%
%Li, J., Todorov, V., Tauchen, G., Chen, R., 2017. Mixed-scale jump regressions with bootstrap inference. \textit{Journal of Econometrics} 201, 417-432.

%Lo, A. W., MacKinlay, A. C., 1990. When are contrarian profits due to stock market overreaction? \textit{Review of Financial Studies} 3, 175-205.

Mancini, C., 2009. Non-parametric threshold estimation for models with stochastic diffusion coefficient and jumps. \textit{Scandinavian Journal of Statistics} 36, 270-296.

Maheu, J.M. McCurdy, T.H. (2004), News arrival, jump dynamics, and volatility components for individual stock returns. 
\textit{Journal of Finance} 59, 755-93.

%Nagel, S., 2012. Evaporating liquidity. \textit{Review of Financial Studies} 25, 2005-2039.

%Neuberger, A., 2012. Realized skewness. R\textit{eview of Financial Studies} 25, 3423-3455.

%Newey, W. K., West, K. D., 1987. A simple, positive semi-definite, heteroskedasticity and autocorrelation consistent covariance matrix. \textit{Econometrica} 55, 703-708.

Patton, A. J., Sheppard, K., 2015. Good volatility, bad volatility: Signed jumps and the persistence of volatility. \textit{Review of Economics and Statistics} 97, 683-697.

Petersen, M.A., 2009. Estimating standard errors in financial panel data sets: Comparing approaches. \textit{Review of Financial Studies} 22, 435-480.

Rossi, A., Timmermann, A., 2015, Modeling covariance risk in Merton's ICAPM. \textit{Review of Financial Studies} 28, 1428-61.

%Tauchen, G., Zhou, H., 2011. Realized jumps on financial markets and predicting credit spreads. \textit{Journal of Econometrics} 160, 102–118.

Todorov, V., Tauchen, G., 2010. Activity signature functions with application for high-frequency data analysis. \textit{Journal of Econometrics} 154, 125-138. 

%Todorov, V., Bollerslev, T., 2010. Jumps and betas: A new framework for disentangling and estimating
%systematic risks. \textit{Journal of Econometrics} 157, 220-235.

Woodward, G. and Anderson, H., 2009. Does beta react to market conditions? Estimates of ‘bull’ and ‘bear’ betas using a nonlinear market model with an endogenous threshold parameter. \textit{Quantitative Finance} 9, 913-924.

Xiong, J.X., Idzorek, T.M. and Ibbotson, R.G., 2016. The economic value of forecasting left-tail risk. \textit{The Journal of Portfolio Management} 42, 114-123.

Yan, S., 2011. Jump risk, stock returns, and slope of implied volatility smile. \textit{Journal of Financial Economics} 99, 216-233.
\end{hangparas}
%%%%%%%%%%%%%%%%%%%%%%%%%%%%%%%%%%%%%%%%%%%%%%%%%%%%%%%%%%%%%%%%%%%%%%%%%%%%%%%%%%%%%%%%%%%%%%%%%%%%%%%%%%%%%%%%%%%%%%%%%%%%%%%%%%%%%%%%%

\clearpage
%\newpage
%\section*{  { } Appendix }
%Controlling variables: 
%
%Beta: a firm's beta at the end of day t is computed using a regression based on daily returns over the past 252  days. 
%
%Size: market value of equity (in millions). Measured by the product of closing price and the number of outstanding shares. The market cap is updated daily. 
%
%BEME: following Fama and French (1993),it's the ratio of book common equity in fiscal year t-1 over market value of equity in December of year t-1. 
%
%Mom: a firm's momenturm at the end of day t is the compound gross return from day t-252 to day t-21.
%
%REV: a firm's short term reversal variable is measured by the weekly return in the past week from Tuesday to Monday.
%
%IVOL: idiosyncratic volatility is measured as the standard deviation of the error terms from the three-factor Fama French regresson using daily returns over the past 21 trading days. 
%$$r_{i,t}-r_{f,t}=\alpha_i+\beta_i(MKT_t-r_{f,t})+\eta_iSMB_t +\kappa_i HML_t +\epsilon_{i,t}$$ 
%where $r_{i,t}$ is the daily return of firm i at day t, therefore $IVOL_{i,t}=\sqrt{var(\epsilon_{i,t})}$. 
%
%CSK: coskewness of firm i at the end of day t is computed with daily returns over the past 252 trading days as 
%$$\widehat{CSK}_{i,t}=\frac{\frac{1}{N}\sum (r_{i,t}-\bar{r}_i)(r_{m,t}-\bar{r}_m)^2}{\sqrt{\frac{1}{N}\sum (r_{i,t}-\bar{r}_i)^2}(\frac{1}{N}\sum (r_{m,t}-\bar{r}_m)^2)}$$ 
%
%where $r_{i,t}$ is the daily return of firm i at day t, and $r_{m,t}$ is the daily return of market portfolio at day t. $\bar{r}_i$ and $\bar{r}_m$ denote the average daily return of stock i and the market portfolio, respectively. 
%
%CKT: cokurtosis of firm i at the end of day t is computed with daily returns over the past 252 trading days as 
%$$\widehat{CKT}_{i,t}=\frac{\frac{1}{N}\sum (r_{i,t}-\bar{r}_i)(r_{m,t}-\bar{r}_m)^3}{\sqrt{\frac{1}{N}\sum (r_{i,t}-\bar{r}_i)^2}(\frac{1}{N}\sum (r_{m,t}-\bar{r}_m)^2)^{\frac{3}{2}}}$$
%
%MAX: maximum return is measured as the maximum daily return over the past week. 
%
%MIN: minimum return is defined as the mimimum daily return over the past week. 
%
%ILLIQ: illiquidity of firm i at day t is measured as the average of the ratio of absolute daily return over the dollar trading volume over the past 252 trading days: 
%$$ILLIQ=\frac{1}{N}\sum_{d=0}^{N}(\frac{|r_{i,t-d}|}{volume_{i,t-d}*price_{i,t-d}})$$
%
%where $volume_{i,t-d}$ is the daily trading volume of stock i at day t-d. 

\newpage

\centering
\noindent Table 1: Realized Measures and Firm Characteristics$^*\>$
% Table generated by Excel2LaTeX from sheet 'notation'
\begin{table}[htbp!]
	\centering
	%\caption*{Panel A: Realized Measures Used in Portfolio Sorts and Fama-MacBeth Regressions }
	\resizebox{1.04\textwidth}{!}{
		\begin{threeparttable}
			\begin{tabular}{llrrrrrr}
				\multicolumn{8}{l}{Panel A: Realized Measures Used in Portfolio Sorts and Fama-MacBeth Regressions} \\
				&  &       &       &       &       &       &  \\
				\midrule
				RVJP  & Positive (upside) jump variation, see (\ref{eq1}).  &       &       &       &       &       &  \\
				RVJN  & Negative (downside) jump variation, see (\ref{eq2}).&       &       &       &       &       &  \\
				SRVJ  & Signed jump variation, $RVJP-RVJN$, see (\ref{eq3}).&     &       &       &       &       &    \\
				RVLJP & Positive (upside) large jump variation, see (\ref{eq4}). &       &       &       &       &       &  \\
				RVLJN & Negative (downside) large jump variation, see (\ref{eq5}). &       &       &       &       &       &  \\
				SRVLJ & Signed large jump variation, $RVLJP-RVLJN$, see (\ref{eq6}). &       &       &       &       &       &  \\
				RVSJP & Positive (upside) small jump variation, see (\ref{eq7}). &       &       &       &       &       &  \\
				RVSJN & Negative (downside) small jump variation, see (\ref{eq8}). &       &       &       &       &       &  \\
				SRVSJ & Signed small jump variation, $RVSJP-RVSJN$, see (\ref{eq9}). &       &       &       &       &       &  \\
				RVOL  & Realized volatility &       &       &       &       &       &  \\
				RSK   & Realized skewness, see (\ref{eq10}). &       &       &       &       &       &  \\
				RKT   & Realized kurtosis, see (\ref{eq11}). &       &       &       &       &       &  \\
				\midrule
				&       &       &       &       &       &       &  \\
				
				\multicolumn{8}{l}{Panel B: Explanatory Variables and Firm Characteristics Used in Fama-MacBeth Regressions} \\
				&  &       &       &       &       &       &  \\
				\midrule
				BETA  & Market beta &       &       &       &       &       &  \\
				log(Size)  & Natural logarithm of firm size &       &       &       &       &       &  \\
				BEME  & Book-to-market ratio &       &       &       &       &       &  \\
				MOM   & MOM Momentum &       &       &       &       &       &  \\
				REV   & Short-term reversal &       &       &       &       &       &  \\
				IVOL  & Idiosyncratic volatility &       &       &       &       &       &  \\
				CSK   & Coskewness &       &       &       &       &       &  \\
				CKT   & Cokurtosis &       &       &       &       &       &  \\
				MAX   & Maximum daily return &       &       &       &       &       &  \\
				MIN   & Minimum daily return &       &       &       &       &       &  \\
				ILLIQ  & Illiquidity &       &       &       &       &       &  \\
				\midrule
			\end{tabular}
			\begin{tablenotes}
				\item {}
				\item {$^*\>$ Notes: The realized measures listed in Panel A of this table are defined and discussed in Section 2. For detailed descriptions of the explanatory variables and firm characteristics listed in Panel B of this table, refer to Bollerslev, Li, and Zhao (2017), and the references cited therein.}
			\end{tablenotes}
	\end{threeparttable}}
\end{table}
\newpage
\noindent Table 2: Summary  Statistics for Various Realized Measures and Firm Characteristics Based on Two Jump Truncation Levels$^*\>$
%\footnote{We measure the cross-sectional bid-ask spreads using one month quote data. The resutls show that the bid-ask spreads for S\&P 500 stocks are much smaller than those for non S\&P stocks, indicating that stocks for large firms are more liquid. } 
\large
% Table generated by Excel2LaTeX from sheet 'mean_t4_15t80'
\begin{table}[htbp!]
	\centering
	%\caption*{\normalfont Panel A: Cross-Sectional Summary Statistics}
	\resizebox{1.04\textwidth}{!}{
		\begin{tabular}{lccccccccccccccccccccccc}
			\multicolumn{24}{l}{Panel A: Cross-Sectional Summary Statistics}\\
			&      &      &      &      &      &  & & & & & & & & & &
			&       &     &      &      &      &      & 
			\\
			\midrule
			& SRVJ  & RVJP  & RVJN  & SRVLJ & RVLJP & RVLJN & SRVSJ & RVSJP & RVSJN & RVOL  & RSK   & RKT   & BETA  & log(Size) & BEME  & MOM   & REV   & IVOL  & CSK   & CKT   & MAX   & MIN   & ILLIQ \\
			&       &       &       &       &       &       &       &       &       &       &       &       &       &       &       &       &       &       &       &       &       &       &  \\
			\multicolumn{24}{l}{I: Jump Truncation Level=   $\alpha_n^{1}$}\\
			Mean  & 0.0049  & 0.2685  & 0.2636  & 0.0034  & 0.1506  & 0.1472  & 0.0014  & 0.1179  & 0.1164  & 0.7276  & 0.0224  & 8.1802  & 1.0796  & 6.5294  & 0.5967  & 0.2026  & 0.0071  & 0.0292  & -0.0263  & 1.1441  & 0.0412  & -0.0347  & -5.2845  \\
			Std   & 0.1543  & 0.1342  & 0.1351  & 0.1429  & 0.1547  & 0.1530  & 0.0639  & 0.0777  & 0.0778  & 0.6406  & 0.8106  & 4.4528  & 0.5566  & 1.8350  & 0.7222  & 0.7465  & 0.0926  & 0.0249  & 0.3283  & 0.8473  & 0.0571  & 0.0359  & 2.4032  \\
			&       &       &       &       &       &       &       &       &       &       &       &       &       &       &       &       &       &       &       &       &       &       &  \\
			\multicolumn{24}{l}{II: Jump Truncation Level=   $\alpha_n^{2}$}\\
			Mean  & 0.0049  & 0.2685  & 0.2636  & 0.0019  & 0.0972  & 0.0953  & 0.0029  & 0.1712  & 0.1683  & 0.7276  & 0.0224  & 8.1802  & 1.0796  & 6.5294  & 0.5967  & 0.2026  & 0.0071  & 0.0292  & -0.0263  & 1.1441  & 0.0412  & -0.0347  & -5.2845  \\
			Std   & 0.1543  & 0.1342  & 0.1351  & 0.1306  & 0.1392  & 0.1375  & 0.0864  & 0.0905  & 0.0905  & 0.6406  & 0.8106  & 4.4528  & 0.5566  & 1.8350  & 0.7222  & 0.7465  & 0.0926  & 0.0249  & 0.3283  & 0.8473  & 0.0571  & 0.0359  & 2.4032  \\
			\bottomrule
	\end{tabular}}
\end{table}
% Table generated by Excel2LaTeX from sheet ''

%\begin{table}[!htbp]
%	\centering
%	%\caption{Add caption}
%	\begin{tabular}{lrrr}
%		\toprule
%		& \multicolumn{1}{l}{S\&P 500} &       & \multicolumn{1}{l}{Non\_S\&P 500} \\
%		Mean  & 0.000486  &       & 0.008282  \\
%		Std   & 0.000589  &       & 0.013490  \\
%		Median & 0.000333  &       & 0.002943  \\
%		\bottomrule
%	\end{tabular}
%\end{table}

% Table generated by Excel2LaTeX from sheet 'mean_t4_15t80'
\begin{table}[htbp!]
	
	\centering
	%\caption*{\normalfont Panel B: Cross-Sectional Correlations}
	\resizebox{1.04\textwidth}{!}{
		\begin{threeparttable}
			\begin{tabular}{lccccccccccccccccccccccc}
				\multicolumn{24}{l}{Panel B: Cross-Sectional Correlations}\\
				&      &      &      &      &      &  & & & & & & & & & &
				&       &     &      &      &      &      & 
				\\
				\midrule
				& SRVJ  & RVJP  & RVJN  & SRVLJ & RVLJP & RVLJN & SRVSJ & RVSJP & RVSJN & RVOL  & RSK   & RKT   & BETA  & log(Size) & BEME  & MOM   & REV   & IVOL  & CSK   & CKT   & MAX   & MIN   & ILLIQ \\
				&       &       &       &       &       &       &       &       &       &       &       &       &       &       &       &       &       &       &       &       &       &       &  \\
				\multicolumn{24}{l}{I: Jump Truncation Level=   $\alpha_n^{1}$}\\
				SRVJ  & 1     & 0.57  & -0.58  & 0.91  & 0.42  & -0.41  & 0.38  & 0.14  & -0.18  & -0.06  & 0.94  & 0.02  & -0.02  & 0.02  & 0.01  & 0.01  & 0.29  & -0.05  & 0.08  & 0.00  & 0.15  & 0.23  & -0.01  \\
				RVJP  &       & 1     & 0.32  & 0.52  & 0.85  & 0.37  & 0.21  & 0.04  & -0.14  & 0.25  & 0.53  & 0.44  & -0.26  & -0.50  & 0.14  & -0.10  & 0.15  & 0.11  & 0.04  & -0.24  & 0.14  & 0.06  & 0.56  \\
				RVJN  &       &       & 1     & -0.52  & 0.35  & 0.84  & -0.22  & -0.12  & 0.07  & 0.31  & -0.55  & 0.42  & -0.23  & -0.51  & 0.13  & -0.11  & -0.19  & 0.16  & -0.06  & -0.24  & -0.04  & -0.21  & 0.56  \\
				SRVLJ &       &       &       & 1     & 0.47  & -0.45  & -0.04  & -0.04  & -0.01  & -0.05  & 0.92  & 0.02  & -0.01  & 0.02  & 0.01  & 0.01  & 0.19  & -0.04  & 0.05  & 0.01  & 0.10  & 0.17  & -0.01  \\
				RVLJP &       &       &       &       & 1     & 0.57  & -0.02  & -0.46  & -0.44  & 0.26  & 0.43  & 0.61  & -0.25  & -0.47  & 0.12  & -0.06  & 0.08  & 0.13  & 0.02  & -0.24  & 0.11  & -0.02  & 0.54  \\
				RVLJN &       &       &       &       &       & 1     & 0.01  & -0.44  & -0.45  & 0.31  & -0.42  & 0.59  & -0.24  & -0.48  & 0.11  & -0.06  & -0.10  & 0.16  & -0.03  & -0.24  & 0.02  & -0.17  & 0.55  \\
				SRVSJ &       &       &       &       &       &       & 1     & 0.42  & -0.43  & -0.03  & 0.19  & 0.00  & -0.03  & 0.01  & 0.01  & 0.01  & 0.26  & -0.03  & 0.08  & 0.00  & 0.14  & 0.19  & 0.00  \\
				RVSJP &       &       &       &       &       &       &       & 1     & 0.62  & -0.06  & 0.06  & -0.40  & 0.06  & 0.02  & 0.01  & -0.06  & 0.10  & -0.02  & 0.04  & 0.03  & 0.03  & 0.11  & -0.05  \\
				RVSJN &       &       &       &       &       &       &       &       & 1.00  & -0.04  & -0.10  & -0.40  & 0.08  & 0.01  & 0.01  & -0.06  & -0.12  & 0.00  & -0.04  & 0.03  & -0.08  & -0.05  & -0.05  \\
				RVOL  &       &       &       &       &       &       &       &       &       & 1.00  & -0.05  & 0.20  & -0.05  & -0.67  & 0.10  & -0.14  & 0.06  & 0.65  & -0.01  & -0.33  & 0.49  & -0.53  & 0.68  \\
				RSK   &       &       &       &       &       &       &       &       &       &       & 1     & 0.02  & -0.01  & 0.02  & 0.00  & 0.01  & 0.21  & -0.03  & 0.06  & 0.00  & 0.11  & 0.18  & -0.01  \\
				RKT   &       &       &       &       &       &       &       &       &       &       &       & 1     & -0.21  & -0.34  & 0.09  & -0.02  & -0.01  & 0.09  & -0.01  & -0.18  & 0.06  & -0.08  & 0.39  \\
				BETA  &       &       &       &       &       &       &       &       &       &       &       &       & 1     & 0.10  & -0.09  & 0.00  & -0.04  & 0.06  & 0.01  & 0.30  & 0.03  & -0.09  & -0.16  \\
				ME    &       &       &       &       &       &       &       &       &       &       &       &       &       & 1     & -0.19  & 0.11  & -0.05  & -0.52  & 0.01  & 0.40  & -0.32  & 0.35  & -0.93  \\
				BEME  &       &       &       &       &       &       &       &       &       &       &       &       &       &       & 1     & 0.03  & 0.02  & 0.05  & 0.00  & -0.06  & 0.04  & -0.03  & 0.18  \\
				MOM   &       &       &       &       &       &       &       &       &       &       &       &       &       &       &       & 1     & 0.00  & -0.08  & -0.07  & 0.06  & -0.05  & 0.05  & -0.15  \\
				REV   &       &       &       &       &       &       &       &       &       &       &       &       &       &       &       &       & 1     & 0.12  & 0.16  & -0.04  & 0.49  & 0.29  & 0.05  \\
				IVOL  &       &       &       &       &       &       &       &       &       &       &       &       &       &       &       &       &       & 1     & 0.02  & -0.35  & 0.50  & -0.47  & 0.47  \\
				CSK   &       &       &       &       &       &       &       &       &       &       &       &       &       &       &       &       &       &       & 1     & 0.01  & 0.07  & 0.07  & 0.00  \\
				CKT   &       &       &       &       &       &       &       &       &       &       &       &       &       &       &       &       &       &       &       & 1     & -0.16  & 0.15  & -0.37  \\
				MAX   &       &       &       &       &       &       &       &       &       &       &       &       &       &       &       &       &       &       &       &       & 1     & -0.28  & 0.34  \\
				MIN   &       &       &       &       &       &       &       &       &       &       &       &       &       &       &       &       &       &       &       &       &       & 1     & -0.35  \\
				ILLIQ &       &       &       &       &       &       &       &       &       &       &       &       &       &       &       &       &       &       &       &       &       &       & 1 \\
				\midrule
				%				&       &       &       &       &       &       &       &       &       &       &       &       &       &       &       &       &       &       &       &       &       &       &  \\
				%				II:  $\alpha_n^{(2)}$ &       &       &       &       &       &       &       &       &       &       &       &       &       &       &       &       &       &       &       &       &       &       &  \\
				&       &       &       &       &       &       &       &       &       &       &       &       &       &       &       &       &       &       &       &       &       &       &  \\
				&       &       &       &       &       &       &       &       &       &       &       &       &       &       &       &       &       &       &       &       &       &       &  \\
				\midrule
				& SRVJ  & RVJP  & RVJN  & SRVLJ & RVLJP & RVLJN & SRVSJ & RVSJP & RVSJN & RVOL  & RSK   & RKT   & BETA  & log(Size) & BEME  & MOM   & REV   & IVOL  & CSK   & CKT   & MAX   & MIN   & ILLIQ \\
				&       &       &       &       &       &       &       &       &       &       &       &       &       &       &       &       &       &       &       &       &       &       &  \\
				\multicolumn{24}{l}{II: Jump Truncation Level=   $\alpha_n^{2}$}\\
				SRVJ  & 1     & 0.57  & -0.58  & 0.82  & 0.39  & -0.38  & 0.53  & 0.23  & -0.27  & -0.06  & 0.94  & 0.02  & -0.02  & 0.02  & 0.01  & 0.01  & 0.29  & -0.05  & 0.08  & 0.00  & 0.15  & 0.23  & -0.01  \\
				RVJP  &       & 1     & 0.32  & 0.47  & 0.77  & 0.33  & 0.30  & 0.30  & 0.01  & 0.25  & 0.53  & 0.44  & -0.26  & -0.50  & 0.14  & -0.10  & 0.15  & 0.11  & 0.04  & -0.24  & 0.14  & 0.06  & 0.56  \\
				RVJN  &       &       & 1     & -0.47  & 0.31  & 0.76  & -0.31  & 0.02  & 0.32  & 0.31  & -0.55  & 0.42  & -0.23  & -0.51  & 0.13  & -0.11  & -0.19  & 0.16  & -0.06  & -0.24  & -0.04  & -0.21  & 0.56  \\
				SRVLJ &       &       &       & 1     & 0.48  & -0.45  & -0.04  & -0.04  & 0.00  & -0.04  & 0.89  & 0.02  & -0.01  & 0.01  & 0.00  & 0.01  & 0.15  & -0.03  & 0.04  & 0.01  & 0.07  & 0.13  & -0.01  \\
				RVLJP &       &       &       &       & 1     & 0.56  & -0.02  & -0.36  & -0.34  & 0.23  & 0.42  & 0.64  & -0.23  & -0.40  & 0.11  & -0.05  & 0.06  & 0.11  & 0.01  & -0.21  & 0.09  & -0.02  & 0.46  \\
				RVLJN &       &       &       &       &       & 1     & 0.01  & -0.34  & -0.35  & 0.27  & -0.41  & 0.63  & -0.23  & -0.41  & 0.10  & -0.06  & -0.08  & 0.13  & -0.03  & -0.22  & 0.02  & -0.14  & 0.47  \\
				SRVSJ &       &       &       &       &       &       & 1     & 0.48  & -0.48  & -0.04  & 0.33  & 0.00  & -0.03  & 0.01  & 0.01  & 0.01  & 0.29  & -0.04  & 0.09  & 0.00  & 0.15  & 0.21  & 0.00  \\
				RVSJP &       &       &       &       &       &       &       & 1     & 0.54  & 0.03  & 0.14  & -0.30  & -0.02  & -0.15  & 0.04  & -0.07  & 0.13  & 0.03  & 0.04  & -0.05  & 0.08  & 0.10  & 0.14  \\
				RVSJN &       &       &       &       &       &       &       &       & 1.00  & 0.07  & -0.17  & -0.30  & 0.01  & -0.16  & 0.04  & -0.08  & -0.15  & 0.06  & -0.04  & -0.05  & -0.07  & -0.11  & 0.14  \\
				RVOL  &       &       &       &       &       &       &       &       &       & 1.00  & -0.05  & 0.20  & -0.05  & -0.67  & 0.10  & -0.14  & 0.06  & 0.65  & -0.01  & -0.33  & 0.49  & -0.53  & 0.68  \\
				RSK   &       &       &       &       &       &       &       &       &       &       & 1     & 0.02  & -0.01  & 0.02  & 0.00  & 0.01  & 0.21  & -0.03  & 0.06  & 0.00  & 0.11  & 0.18  & -0.01  \\
				RKT   &       &       &       &       &       &       &       &       &       &       &       & 1     & -0.21  & -0.34  & 0.09  & -0.02  & -0.01  & 0.09  & -0.01  & -0.18  & 0.06  & -0.08  & 0.39  \\
				BETA  &       &       &       &       &       &       &       &       &       &       &       &       & 1     & 0.10  & -0.09  & 0.00  & -0.04  & 0.06  & 0.01  & 0.30  & 0.03  & -0.09  & -0.16  \\
				ME    &       &       &       &       &       &       &       &       &       &       &       &       &       & 1     & -0.19  & 0.11  & -0.05  & -0.52  & 0.01  & 0.40  & -0.32  & 0.35  & -0.93  \\
				BEME  &       &       &       &       &       &       &       &       &       &       &       &       &       &       & 1     & 0.03  & 0.02  & 0.05  & 0.00  & -0.06  & 0.04  & -0.03  & 0.18  \\
				MOM   &       &       &       &       &       &       &       &       &       &       &       &       &       &       &       & 1     & 0.00  & -0.08  & -0.07  & 0.06  & -0.05  & 0.05  & -0.15  \\
				REV   &       &       &       &       &       &       &       &       &       &       &       &       &       &       &       &       & 1     & 0.12  & 0.16  & -0.04  & 0.49  & 0.29  & 0.05  \\
				IVOL  &       &       &       &       &       &       &       &       &       &       &       &       &       &       &       &       &       & 1     & 0.02  & -0.35  & 0.50  & -0.47  & 0.47  \\
				CSK   &       &       &       &       &       &       &       &       &       &       &       &       &       &       &       &       &       &       & 1     & 0.01  & 0.07  & 0.07  & 0.00  \\
				CKT   &       &       &       &       &       &       &       &       &       &       &       &       &       &       &       &       &       &       &       & 1     & -0.16  & 0.15  & -0.37  \\
				MAX   &       &       &       &       &       &       &       &       &       &       &       &       &       &       &       &       &       &       &       &       & 1     & -0.28  & 0.34  \\
				MIN   &       &       &       &       &       &       &       &       &       &       &       &       &       &       &       &       &       &       &       &       &       & 1     & -0.35  \\
				ILLIQ &       &       &       &       &       &       &       &       &       &       &       &       &       &       &       &       &       &       &       &       &       &       & 1 \\
				\bottomrule
			\end{tabular}
			
			\begin{tablenotes}
				\item {}
				\item {\large $^*\>$ Notes: See notes to Table 1. This table presents cross-sectional summary statistics and correlations for all realized measures and control variables based on two truncation levels:   $\>\alpha_n^{1}=4\sqrt{\frac{1}{t}\widehat{IV}_t}\triangle_n^{0.49} \>$ and $\>\alpha_n^{2}=5\sqrt{\frac{1}{t}\widehat{IV}_t}\triangle_n^{0.49}\>$. The entries in the table for realized measures (see columns 2-13) are constructed using 5-min intraday high frequency data. Entries for firm characteristics (see columns 14-24) are constructed using daily data, with the exception of BEME, which is constructed using monthly data. For complete details, see Sections 3 and 4.}
			\end{tablenotes}
	\end{threeparttable}}
	
\end{table}

\newpage
\large
\centering
\noindent Table 3: Realized Measures and Firm Characteristics of Portfolios Sorted by Various Realized Measures
% Table generated by Excel2LaTeX from sheet '4srvj'
\begin{table}[htbp!]
	\centering
	%\caption*{\normalfont Panel A: Stocks Sorted by SRVJ}
	\large
	\resizebox{1.04\textwidth}{!}{
		\begin{tabular}{lccccccccccccccccccccccc}
			\multicolumn{24}{l}{Panel A: Stocks Sorted by SRVJ}\\
			&      &      &      &      &      &  & & & & & & & & & &
			&       &     &      &      &      &      & 
			\\
			\toprule
			Quintile & RVJP  & RVJN  & RVLJP & RVLJN & RVSJP & RVSJN & SRVLJ & SRVSJ & SRVJ  & RVOL  & RSK   & RKT   & BETA  & log(Size) & BEME  & MOM   & REV   & IVOL  & CSK   & CKT   & MAX   & MIN   & ILLIQ \\
			%\midrule
			1     & 0.2020  & 0.3979  & 0.1163  & 0.2744  & 0.0858  & 0.1235  & -0.1582  & -0.0378  & -0.1963  & 0.8284  & -0.9365  & 9.8271  & 1.0338  & 6.1047  & 0.6265  & 0.1952  & -0.0347  & 0.0323  & -0.0703  & 1.0422  & 0.0314  & -0.0497  & -4.6498  \\
			2     & 0.2190  & 0.2774  & 0.1008  & 0.1390  & 0.1182  & 0.1384  & -0.0382  & -0.0203  & -0.0586  & 0.7181  & -0.2541  & 7.1284  & 1.1348  & 6.7014  & 0.5715  & 0.2053  & -0.0107  & 0.0293  & -0.0443  & 1.1976  & 0.0349  & -0.0383  & -5.5798  \\
			3     & 0.2417  & 0.2388  & 0.1114  & 0.1096  & 0.1303  & 0.1293  & 0.0019  & 0.0010  & 0.0029  & 0.6936  & 0.0161  & 6.8518  & 1.1311  & 6.8197  & 0.5693  & 0.2045  & 0.0079  & 0.0282  & -0.0243  & 1.2230  & 0.0397  & -0.0327  & -5.7350  \\
			4     & 0.2784  & 0.2127  & 0.1426  & 0.0996  & 0.1357  & 0.1131  & 0.0430  & 0.0226  & 0.0658  & 0.6687  & 0.2914  & 7.1837  & 1.1106  & 6.7935  & 0.5767  & 0.2110  & 0.0263  & 0.0276  & -0.0066  & 1.2070  & 0.0449  & -0.0281  & -5.6778  \\
			5     & 0.4014  & 0.1909  & 0.2819  & 0.1131  & 0.1195  & 0.0779  & 0.1688  & 0.0417  & 0.2109  & 0.7294  & 0.9969  & 9.9099  & 0.9875  & 6.2279  & 0.6398  & 0.1969  & 0.0466  & 0.0288  & 0.0144  & 1.0506  & 0.0550  & -0.0244  & -4.7801  \\
			\bottomrule
	\end{tabular}}
\end{table}
\begin{table}[htbp!]
	\centering
	%\caption*{\normalfont Panel B: Stocks Sorted by RVJP}
	\large
	\resizebox{1.04\textwidth}{!}{
		\begin{tabular}{lccccccccccccccccccccccc}
			\multicolumn{24}{l}{Panel B: Stocks Sorted by RVJP}\\
			&      &      &      &      &      &  & & & & & & & & & &
			&       &     &      &      &      &      & 
			\\
			\toprule
			Quintile & RVJP  & RVJN  & RVLJP & RVLJN & RVSJP & RVSJN & SRVLJ & SRVSJ & SRVJ  & RVOL  & RSK   & RKT   & BETA  & log(Size) & BEME  & MOM   & REV   & IVOL  & CSK   & CKT   & MAX   & MIN   & ILLIQ \\
			%\midrule
			1     & 0.1089  & 0.2018  & 0.0270  & 0.0899  & 0.0821  & 0.1119  & -0.0629  & -0.0298  & -0.0927  & 0.5785  & -0.4313  & 6.4732  & 1.2384  & 7.7514  & 0.4791  & 0.3212  & -0.0145  & 0.0260  & -0.0485  & 1.3722  & 0.0317  & -0.0392  & -7.0116  \\
			2     & 0.1866  & 0.2193  & 0.0644  & 0.0954  & 0.1222  & 0.1239  & -0.0310  & -0.0017  & -0.0327  & 0.6118  & -0.1520  & 6.7458  & 1.1624  & 7.2099  & 0.5291  & 0.2421  & 0.0017  & 0.0267  & -0.0313  & 1.2991  & 0.0369  & -0.0346  & -6.3152  \\
			3     & 0.2484  & 0.2517  & 0.1091  & 0.1216  & 0.1392  & 0.1301  & -0.0124  & 0.0091  & -0.0034  & 0.6858  & -0.0165  & 7.5051  & 1.1102  & 6.6367  & 0.5782  & 0.1969  & 0.0089  & 0.0285  & -0.0232  & 1.1935  & 0.0407  & -0.0338  & -5.5376  \\
			4     & 0.3248  & 0.3070  & 0.1802  & 0.1761  & 0.1445  & 0.1310  & 0.0041  & 0.0136  & 0.0177  & 0.8104  & 0.0730  & 8.6740  & 1.0367  & 5.9165  & 0.6476  & 0.1417  & 0.0137  & 0.0314  & -0.0189  & 1.0448  & 0.0447  & -0.0340  & -4.4548  \\
			5     & 0.4740  & 0.3382  & 0.3724  & 0.2529  & 0.1015  & 0.0853  & 0.1194  & 0.0162  & 0.1356  & 0.9520  & 0.6401  & 11.5066  & 0.8499  & 5.1305  & 0.7524  & 0.1106  & 0.0257  & 0.0337  & -0.0093  & 0.8104  & 0.0518  & -0.0317  & -3.1002  \\
			\bottomrule
	\end{tabular}}
\end{table}
\begin{table}[htbp!]
	\centering
	%\caption*{\normalfont Panel C: Stocks Sorted by RVJN}
	\large
	\resizebox{1.04\textwidth}{!}{
		\begin{tabular}{lccccccccccccccccccccccc}
			\multicolumn{24}{l}{Panel C: Stocks Sorted by RVJN}\\
			&      &      &      &      &      &  & & & & & & & & & &
			&       &     &      &      &      &      & 
			\\
			\toprule
			Quintile & RVJP  & RVJN  & RVLJP & RVLJN & RVSJP & RVSJN & SRVLJ & SRVSJ & SRVJ  & RVOL  & RSK   & RKT   & BETA  & log(Size) & BEME  & MOM   & REV   & IVOL  & CSK   & CKT   & MAX   & MIN   & ILLIQ \\
			%\midrule
			1     & 0.2123  & 0.1031  & 0.1006  & 0.0262  & 0.1117  & 0.0770  & 0.0743  & 0.0346  & 0.1090  & 0.5485  & 0.5044  & 6.6112  & 1.2009  & 7.7638  & 0.4899  & 0.3035  & 0.0318  & 0.0248  & 0.0041  & 1.3617  & 0.0453  & -0.0254  & -6.9914  \\
			2     & 0.2232  & 0.1804  & 0.1007  & 0.0629  & 0.1225  & 0.1176  & 0.0379  & 0.0049  & 0.0427  & 0.5919  & 0.2005  & 6.7837  & 1.1582  & 7.2312  & 0.5319  & 0.2495  & 0.0156  & 0.0260  & -0.0168  & 1.2976  & 0.0412  & -0.0303  & -6.3449  \\
			3     & 0.2535  & 0.2431  & 0.1237  & 0.1064  & 0.1298  & 0.1367  & 0.0173  & -0.0069  & 0.0105  & 0.6706  & 0.0533  & 7.5047  & 1.1178  & 6.6503  & 0.5773  & 0.2116  & 0.0069  & 0.0282  & -0.0302  & 1.2022  & 0.0406  & -0.0340  & -5.5631  \\
			4     & 0.3087  & 0.3215  & 0.1756  & 0.1761  & 0.1330  & 0.1454  & -0.0005  & -0.0123  & -0.0128  & 0.8071  & -0.0443  & 8.6199  & 1.0492  & 5.9098  & 0.6462  & 0.1563  & -0.0007  & 0.0315  & -0.0360  & 1.0483  & 0.0409  & -0.0380  & -4.4565  \\
			5     & 0.3447  & 0.4702  & 0.2523  & 0.3644  & 0.0924  & 0.1056  & -0.1121  & -0.0131  & -0.1252  & 1.0205  & -0.6027  & 11.3848  & 0.8714  & 5.0902  & 0.7419  & 0.0914  & -0.0183  & 0.0357  & -0.0524  & 0.8103  & 0.0378  & -0.0456  & -3.0638  \\
			\bottomrule
	\end{tabular}}
\end{table}
\begin{table}[htbp!]
	\centering
	%\caption*{\normalfont Panel D: Stocks Sorted by SRVLJ}
	\large
	\resizebox{1.04\textwidth}{!}{
		\begin{tabular}{lccccccccccccccccccccccc}
			\multicolumn{24}{l}{Panel D: Stocks Sorted by SRVLJ}\\
			&      &      &      &      &      &  & & & & & & & & & &
			&       &     &      &      &      &      & 
			\\
			\toprule
			Quintile & RVJP  & RVJN  & RVLJP & RVLJN & RVSJP & RVSJN & SRVLJ & SRVSJ & SRVJ  & RVOL  & RSK   & RKT   & BETA  & log(Size) & BEME  & MOM   & REV   & IVOL  & CSK   & CKT   & MAX   & MIN   & ILLIQ \\
			\multicolumn{24}{l}{I: Jump Truncation Level=   $\alpha_n^{1}$}\\
			1     & 0.2256  & 0.3980  & 0.1254  & 0.3039  & 0.1002  & 0.0941  & -0.1789  & 0.0061  & -0.1724  & 0.8336  & -0.9116  & 10.2561  & 1.0137  & 6.0164  & 0.6358  & 0.1919  & -0.0205  & 0.0322  & -0.0544  & 1.0308  & 0.0363  & -0.0461  & -4.5159  \\
			2     & 0.2213  & 0.2602  & 0.0869  & 0.1298  & 0.1344  & 0.1305  & -0.0429  & 0.0039  & -0.0389  & 0.7079  & -0.2037  & 6.9011  & 1.1313  & 6.7589  & 0.5685  & 0.2103  & -0.0006  & 0.0289  & -0.0330  & 1.2036  & 0.0379  & -0.0359  & -5.6590  \\
			3     & 0.2270  & 0.2230  & 0.0949  & 0.0922  & 0.1321  & 0.1309  & 0.0028  & 0.0013  & 0.0040  & 0.6780  & 0.0196  & 6.4725  & 1.1313  & 6.9054  & 0.5636  & 0.2016  & 0.0072  & 0.0281  & -0.0248  & 1.2391  & 0.0393  & -0.0327  & -5.8572  \\
			4     & 0.2712  & 0.2222  & 0.1479  & 0.0973  & 0.1233  & 0.1249  & 0.0507  & -0.0016  & 0.0490  & 0.6535  & 0.2564  & 7.1576  & 1.1249  & 6.8087  & 0.5756  & 0.2130  & 0.0165  & 0.0278  & -0.0165  & 1.2105  & 0.0415  & -0.0303  & -5.7088  \\
			5     & 0.4020  & 0.2151  & 0.3123  & 0.1226  & 0.0897  & 0.0925  & 0.1900  & -0.0028  & 0.1869  & 0.7452  & 0.9723  & 10.3489  & 0.9897  & 6.1252  & 0.6442  & 0.1927  & 0.0334  & 0.0293  & -0.0024  & 1.0448  & 0.0510  & -0.0280  & -4.6428  \\
			&       &       &       &       &       &       &       &       &       &       &       &       &       &       &       &       &       &       &       &       &       &       &  \\
			\multicolumn{24}{l}{II: Jump Truncation Level=   $\alpha_n^{2}$}\\
			1     & 0.2413  & 0.3918  & 0.0845  & 0.2423  & 0.1568  & 0.1494  & -0.1581  & 0.0074  & -0.1505  & 0.8151  & -0.8505  & 10.5187  & 1.0105  & 6.0415  & 0.6386  & 0.1888  & -0.0139  & 0.0317  & -0.0474  & 1.0365  & 0.0378  & -0.0436  & -4.5373  \\
			2     & 0.2183  & 0.2316  & 0.0341  & 0.0525  & 0.1841  & 0.1791  & -0.0184  & 0.0050  & -0.0133  & 0.6943  & -0.0801  & 6.3990  & 1.1358  & 6.8674  & 0.5632  & 0.2125  & 0.0049  & 0.0284  & -0.0280  & 1.2283  & 0.0394  & -0.0342  & -5.8166  \\
			3     & 0.2350  & 0.2334  & 0.0743  & 0.0710  & 0.1607  & 0.1624  & 0.0032  & -0.0017  & 0.0016  & 0.6996  & 0.0143  & 7.4798  & 1.1001  & 6.6105  & 0.6193  & 0.1818  & 0.0060  & 0.0311  & -0.0039  & 1.2411  & 0.0444  & -0.0363  & -5.4110  \\
			4     & 0.2973  & 0.2602  & 0.1223  & 0.0847  & 0.1750  & 0.1755  & 0.0376  & -0.0004  & 0.0371  & 0.6963  & 0.2115  & 7.9773  & 1.0913  & 6.4673  & 0.5989  & 0.1962  & 0.0124  & 0.0291  & -0.0208  & 1.1479  & 0.0414  & -0.0322  & -5.2333  \\
			5     & 0.3962  & 0.2318  & 0.2497  & 0.0829  & 0.1464  & 0.1489  & 0.1672  & -0.0025  & 0.1644  & 0.7376  & 0.9148  & 10.6485  & 0.9963  & 6.1367  & 0.6432  & 0.1939  & 0.0265  & 0.0293  & -0.0086  & 1.0494  & 0.0482  & -0.0296  & -4.6565  \\
			\bottomrule
	\end{tabular}}
\end{table}
\begin{table}[htbp!]
	\centering
	%\caption*{\normalfont Panel E: Stocks Sorted by RVLJP}
	\large
	\resizebox{1.04\textwidth}{!}{
		\begin{tabular}{lccccccccccccccccccccccc}
			\multicolumn{24}{l}{Panel E: Stocks Sorted by RVLJP}\\
			&      &      &      &      &      &  & & & & & & & & & &
			&       &     &      &      &      &      & 
			\\
			\toprule
			Quintile & RVJP  & RVJN  & RVLJP & RVLJN & RVSJP & RVSJN & SRVLJ & SRVSJ & SRVJ  & RVOL  & RSK   & RKT   & BETA  & log(Size) & BEME  & MOM   & REV   & IVOL  & CSK   & CKT   & MAX   & MIN   & ILLIQ \\
			\multicolumn{24}{l}{I: Jump Truncation Level=   $\alpha_n^{1}$}\\
			1     & 0.1561  & 0.2049  & 0.0044  & 0.0584  & 0.1517  & 0.1465  & -0.0540  & 0.0052  & -0.0488  & 0.6294  & -0.2682  & 5.5503  & 1.1796  & 7.4264  & 0.5240  & 0.2304  & -0.0012  & 0.0264  & -0.0322  & 1.3138  & 0.0357  & -0.0344  & -6.5641  \\
			2     & 0.1906  & 0.2192  & 0.0513  & 0.0807  & 0.1393  & 0.1385  & -0.0294  & 0.0008  & -0.0286  & 0.5625  & -0.1543  & 6.3876  & 1.1804  & 7.2863  & 0.5339  & 0.2379  & 0.0021  & 0.0263  & -0.0290  & 1.3073  & 0.0358  & -0.0333  & -6.3933  \\
			3     & 0.2418  & 0.2465  & 0.1084  & 0.1137  & 0.1333  & 0.1328  & -0.0053  & 0.0005  & -0.0047  & 0.6451  & -0.0198  & 7.5225  & 1.1326  & 6.7814  & 0.5682  & 0.2235  & 0.0071  & 0.0280  & -0.0260  & 1.2190  & 0.0399  & -0.0341  & -5.7308  \\
			4     & 0.3140  & 0.3017  & 0.2010  & 0.1890  & 0.1130  & 0.1128  & 0.0120  & 0.0003  & 0.0123  & 0.7871  & 0.0773  & 9.3867  & 1.0488  & 6.0305  & 0.6272  & 0.1856  & 0.0102  & 0.0310  & -0.0235  & 1.0709  & 0.0442  & -0.0358  & -4.6228  \\
			5     & 0.4603  & 0.3600  & 0.4152  & 0.3144  & 0.0451  & 0.0456  & 0.1008  & -0.0005  & 0.1003  & 1.0098  & 0.5132  & 12.6083  & 0.8364  & 5.0080  & 0.7507  & 0.1312  & 0.0175  & 0.0348  & -0.0187  & 0.7826  & 0.0505  & -0.0356  & -2.9054  \\
			&       &       &       &       &       &       &       &       &       &       &       &       &       &       &       &       &       &       &       &       &       &       &  \\
			\multicolumn{24}{l}{II: Jump Truncation Level=   $\alpha_n^{2}$}\\
			1     & 0.1875  & 0.2154  & 0.0000  & 0.0340  & 0.1875  & 0.1813  & -0.0340  & 0.0061  & -0.0278  & 0.6515  & -0.1669  & 5.9245  & 1.1608  & 7.1559  & 0.5418  & 0.2265  & 0.0035  & 0.0273  & -0.0286  & 1.2715  & 0.0381  & -0.0341  & -6.2113  \\
			2     & 0.2080  & 0.2907  & 0.0276  & 0.1166  & 0.1804  & 0.1741  & -0.0890  & 0.0063  & -0.0827  & 0.7210  & -0.4946  & 8.7020  & 1.0624  & 6.6157  & 0.5816  & 0.2695  & -0.0018  & 0.0335  & 0.0074  & 1.1596  & 0.0439  & -0.0422  & -5.4300  \\
			3     & 0.2330  & 0.2533  & 0.0538  & 0.0740  & 0.1793  & 0.1793  & -0.0203  & 0.0000  & -0.0202  & 0.6138  & -0.1150  & 7.5934  & 1.1356  & 6.7739  & 0.5668  & 0.2168  & 0.0032  & 0.0281  & -0.0282  & 1.2408  & 0.0382  & -0.0346  & -5.7059  \\
			4     & 0.2956  & 0.2827  & 0.1155  & 0.1034  & 0.1801  & 0.1793  & 0.0121  & 0.0009  & 0.0129  & 0.7089  & 0.0861  & 8.9273  & 1.0810  & 6.3708  & 0.6028  & 0.1985  & 0.0095  & 0.0295  & -0.0240  & 1.1351  & 0.0422  & -0.0345  & -5.0969  \\
			5     & 0.4414  & 0.3560  & 0.3275  & 0.2420  & 0.1141  & 0.1141  & 0.0853  & 0.0001  & 0.0854  & 0.9660  & 0.4677  & 12.9248  & 0.8589  & 5.1830  & 0.7356  & 0.1385  & 0.0145  & 0.0340  & -0.0214  & 0.8238  & 0.0488  & -0.0360  & -3.1729  \\
			\bottomrule
	\end{tabular}}
\end{table}
\newpage
\small
\noindent Table 3 (Continued)
\begin{table}[htbp!]
	
	\centering
	%\caption*{\normalfont Panel F: Stocks Sorted by RVLJN}
	\large
	\resizebox{1.04\textwidth}{!}{
		\begin{tabular}{lccccccccccccccccccccccc}
			\multicolumn{24}{l}{Panel F: Stocks Sorted by RVLJN}\\
			&      &      &      &      &      &  & & & & & & & & & &
			&       &     &      &      &      &      & 
			\\
			\toprule
			Quintile & RVJP  & RVJN  & RVLJP & RVLJN & RVSJP & RVSJN & SRVLJ & SRVSJ & SRVJ  & RVOL  & RSK   & RKT   & BETA  & log(Size) & BEME  & MOM   & REV   & IVOL  & CSK   & CKT   & MAX   & MIN   & ILLIQ \\
			\multicolumn{24}{l}{I: Jump Truncation Level=   $\alpha_n^{1}$}\\
			1     & 0.2106  & 0.1512  & 0.0625  & 0.0038  & 0.1481  & 0.1474  & 0.0587  & 0.0007  & 0.0594  & 0.6070  & 0.3149  & 5.6103  & 1.1692  & 7.4627  & 0.5280  & 0.2225  & 0.0159  & 0.0256  & -0.0130  & 1.3159  & 0.0400  & -0.0286  & -6.5962  \\
			2     & 0.2223  & 0.1865  & 0.0847  & 0.0489  & 0.1376  & 0.1376  & 0.0357  & 0.0000  & 0.0358  & 0.5497  & 0.1974  & 6.4516  & 1.1770  & 7.3037  & 0.5356  & 0.2353  & 0.0125  & 0.0258  & -0.0180  & 1.3110  & 0.0385  & -0.0298  & -6.4144  \\
			3     & 0.2495  & 0.2372  & 0.1165  & 0.1055  & 0.1330  & 0.1317  & 0.0110  & 0.0013  & 0.0123  & 0.6375  & 0.0610  & 7.5505  & 1.1342  & 6.7812  & 0.5681  & 0.2278  & 0.0089  & 0.0279  & -0.0261  & 1.2210  & 0.0408  & -0.0336  & -5.7352  \\
			4     & 0.3070  & 0.3105  & 0.1923  & 0.1982  & 0.1147  & 0.1123  & -0.0059  & 0.0024  & -0.0035  & 0.7880  & -0.0343  & 9.3634  & 1.0493  & 6.0002  & 0.6273  & 0.1970  & 0.0049  & 0.0312  & -0.0305  & 1.0668  & 0.0436  & -0.0376  & -4.5845  \\
			5     & 0.3682  & 0.4556  & 0.3198  & 0.4097  & 0.0484  & 0.0459  & -0.0899  & 0.0025  & -0.0874  & 1.0553  & -0.4672  & 12.5575  & 0.8474  & 4.9668  & 0.7464  & 0.1248  & -0.0080  & 0.0361  & -0.0428  & 0.7779  & 0.0427  & -0.0442  & -2.8616  \\
			&       &       &       &       &       &       &       &       &       &       &       &       &       &       &       &       &       &       &       &       &       &       &  \\
			\multicolumn{24}{l}{II: Jump Truncation Level=   $\alpha_n^{2}$}\\
			1     & 0.2215  & 0.1823  & 0.0367  & 0.0001  & 0.1847  & 0.1823  & 0.0367  & 0.0025  & 0.0392  & 0.6381  & 0.2172  & 5.9825  & 1.1571  & 7.1731  & 0.5436  & 0.2230  & 0.0122  & 0.0268  & -0.0192  & 1.2727  & 0.0403  & -0.0310  & -6.2291  \\
			2     & 0.2891  & 0.2071  & 0.1168  & 0.0291  & 0.1723  & 0.1781  & 0.0877  & -0.0058  & 0.0820  & 0.6707  & 0.5030  & 8.7975  & 1.0401  & 6.6879  & 0.5712  & 0.2060  & 0.0173  & 0.0321  & 0.0089  & 1.1399  & 0.0470  & -0.0354  & -5.4901  \\
			3     & 0.2481  & 0.2273  & 0.0765  & 0.0559  & 0.1717  & 0.1714  & 0.0206  & 0.0003  & 0.0208  & 0.5979  & 0.1312  & 7.7918  & 1.1288  & 6.7620  & 0.5707  & 0.2134  & 0.0087  & 0.0278  & -0.0209  & 1.2579  & 0.0399  & -0.0328  & -5.6968  \\
			4     & 0.2905  & 0.2979  & 0.1088  & 0.1187  & 0.1818  & 0.1792  & -0.0099  & 0.0025  & -0.0074  & 0.7197  & -0.0554  & 9.0835  & 1.0751  & 6.2988  & 0.6085  & 0.2035  & 0.0050  & 0.0298  & -0.0295  & 1.1224  & 0.0417  & -0.0363  & -4.9916  \\
			5     & 0.3674  & 0.4427  & 0.2520  & 0.3319  & 0.1154  & 0.1108  & -0.0799  & 0.0046  & -0.0753  & 1.0121  & -0.4329  & 13.0268  & 0.8546  & 5.0944  & 0.7411  & 0.1334  & -0.0051  & 0.0353  & -0.0402  & 0.8030  & 0.0434  & -0.0429  & -3.0406  \\
			\bottomrule
	\end{tabular}}
\end{table}
\begin{table}[htbp!]
	
	\centering
	%\caption*{\normalfont Panel G: Stocks Sorted by SRVSJ}
	\large
	\resizebox{1.04\textwidth}{!}{
		\begin{tabular}{lccccccccccccccccccccccc}
			\multicolumn{24}{l}{Panel G: Stocks Sorted by SRVSJ}\\
			&      &      &      &      &      &  & & & & & & & & & &
			&       &     &      &      &      &      & 
			\\
			\toprule
			Quintile & RVJP  & RVJN  & RVLJP & RVLJN & RVSJP & RVSJN & SRVLJ & SRVSJ & SRVJ  & RVOL  & RSK   & RKT   & BETA  & log(Size) & BEME  & MOM   & REV   & IVOL  & CSK   & CKT   & MAX   & MIN   & ILLIQ \\
			\multicolumn{24}{l}{I: Jump Truncation Level=   $\alpha_n^{1}$}\\
			1     & 0.2054  & 0.2787  & 0.1077  & 0.0967  & 0.0976  & 0.1820  & 0.0110  & -0.0844  & -0.0734  & 0.6816  & -0.1876  & 7.1422  & 1.1509  & 6.7320  & 0.5665  & 0.2173  & -0.0267  & 0.0294  & -0.0638  & 1.2000  & 0.0308  & -0.0441  & -5.6266  \\
			2     & 0.2698  & 0.2847  & 0.1749  & 0.1653  & 0.0949  & 0.1193  & 0.0096  & -0.0244  & -0.0148  & 0.7699  & -0.0236  & 8.6145  & 1.0710  & 6.4162  & 0.6065  & 0.1945  & -0.0046  & 0.0301  & -0.0417  & 1.1241  & 0.0379  & -0.0374  & -5.1194  \\
			3     & 0.3071  & 0.3021  & 0.2191  & 0.2155  & 0.0880  & 0.0866  & 0.0037  & 0.0014  & 0.0050  & 0.8341  & 0.0211  & 9.7366  & 1.0017  & 6.1085  & 0.6392  & 0.1752  & 0.0064  & 0.0308  & -0.0277  & 1.0414  & 0.0420  & -0.0355  & -4.6328  \\
			4     & 0.2726  & 0.2484  & 0.1398  & 0.1430  & 0.1328  & 0.1054  & -0.0032  & 0.0274  & 0.0242  & 0.6927  & 0.0666  & 7.9999  & 1.0916  & 6.6644  & 0.5845  & 0.2107  & 0.0189  & 0.0283  & -0.0101  & 1.1756  & 0.0433  & -0.0304  & -5.4895  \\
			5     & 0.2825  & 0.1980  & 0.1001  & 0.1051  & 0.1824  & 0.0929  & -0.0050  & 0.0895  & 0.0845  & 0.6361  & 0.2377  & 7.1716  & 1.0948  & 6.8060  & 0.5789  & 0.2216  & 0.0426  & 0.0272  & 0.0135  & 1.1957  & 0.0518  & -0.0251  & -5.6793  \\
			&       &       &       &       &       &       &       &       &       &       &       &       &       &       &       &       &       &       &       &       &       &       &  \\
			\multicolumn{24}{l}{II: Jump Truncation Level=   $\alpha_n^{2}$}\\
			1     & 0.2048  & 0.3091  & 0.0739  & 0.0649  & 0.1309  & 0.2442  & 0.0090  & -0.1133  & -0.1043  & 0.7298  & -0.3254  & 7.5972  & 1.1196  & 6.5308  & 0.5811  & 0.2160  & -0.0314  & 0.0304  & -0.0682  & 1.1525  & 0.0305  & -0.0466  & -5.3311  \\
			2     & 0.2540  & 0.2837  & 0.1084  & 0.1018  & 0.1456  & 0.1819  & 0.0065  & -0.0363  & -0.0297  & 0.7490  & -0.0832  & 8.3109  & 1.0933  & 6.5185  & 0.5944  & 0.1972  & -0.0078  & 0.0298  & -0.0427  & 1.1502  & 0.0362  & -0.0378  & -5.2795  \\
			3     & 0.2894  & 0.2857  & 0.1423  & 0.1401  & 0.1471  & 0.1455  & 0.0022  & 0.0015  & 0.0037  & 0.7897  & 0.0180  & 9.2404  & 1.0364  & 6.3338  & 0.6211  & 0.1805  & 0.0063  & 0.0300  & -0.0276  & 1.0924  & 0.0410  & -0.0345  & -4.9677  \\
			4     & 0.2806  & 0.2438  & 0.0941  & 0.0977  & 0.1865  & 0.1462  & -0.0035  & 0.0403  & 0.0368  & 0.7015  & 0.1187  & 8.1030  & 1.0854  & 6.6376  & 0.5909  & 0.2056  & 0.0217  & 0.0283  & -0.0087  & 1.1702  & 0.0443  & -0.0298  & -5.4392  \\
			5     & 0.3133  & 0.1957  & 0.0667  & 0.0712  & 0.2466  & 0.1245  & -0.0045  & 0.1221  & 0.1176  & 0.6685  & 0.3832  & 7.6348  & 1.0630  & 6.6259  & 0.5957  & 0.2157  & 0.0464  & 0.0278  & 0.0157  & 1.1531  & 0.0539  & -0.0246  & -5.4061  \\
			\bottomrule
	\end{tabular}}
\end{table}
\begin{table}[htbp!]
	
	\centering
	%\caption*{\normalfont Panel H: Stocks Sorted by RVSJP}
	\large
	\resizebox{1.04\textwidth}{!}{
		\begin{tabular}{lccccccccccccccccccccccc}
			\multicolumn{24}{l}{Panel H: Stocks Sorted by RVSJP}\\
			&      &      &      &      &      &  & & & & & & & & & &
			&       &     &      &      &      &      & 
			\\
			\toprule
			Quintile & RVJP  & RVJN  & RVLJP & RVLJN & RVSJP & RVSJN & SRVLJ & SRVSJ & SRVJ  & RVOL  & RSK   & RKT   & BETA  & log(Size) & BEME  & MOM   & REV   & IVOL  & CSK   & CKT   & MAX   & MIN   & ILLIQ \\
			\multicolumn{24}{l}{I: Jump Truncation Level=   $\alpha_n^{1}$}\\
			1     & 0.3326  & 0.3433  & 0.3183  & 0.3062  & 0.0143  & 0.0371  & 0.0121  & -0.0229  & -0.0107  & 0.9646  & -0.0210  & 12.1510  & 0.9092  & 5.6747  & 0.6795  & 0.1947  & -0.0031  & 0.0335  & -0.0429  & 0.9071  & 0.0434  & -0.0419  & -3.9201  \\
			2     & 0.2168  & 0.2373  & 0.1439  & 0.1381  & 0.0730  & 0.0992  & 0.0057  & -0.0262  & -0.0205  & 0.6518  & -0.0480  & 8.1207  & 1.1614  & 7.0313  & 0.5331  & 0.2864  & -0.0025  & 0.0279  & -0.0382  & 1.2524  & 0.0374  & -0.0370  & -6.0468  \\
			3     & 0.2303  & 0.2322  & 0.1159  & 0.1131  & 0.1144  & 0.1191  & 0.0028  & -0.0047  & -0.0019  & 0.6351  & 0.0080  & 7.4056  & 1.1492  & 6.9647  & 0.5503  & 0.2403  & 0.0063  & 0.0275  & -0.0257  & 1.2504  & 0.0390  & -0.0337  & -5.9351  \\
			4     & 0.2562  & 0.2405  & 0.1003  & 0.1001  & 0.1559  & 0.1404  & 0.0002  & 0.0155  & 0.0157  & 0.6471  & 0.0565  & 6.9562  & 1.1165  & 6.7331  & 0.5790  & 0.1876  & 0.0131  & 0.0277  & -0.0164  & 1.2127  & 0.0409  & -0.0313  & -5.5937  \\
			5     & 0.3120  & 0.2694  & 0.0803  & 0.0840  & 0.2317  & 0.1855  & -0.0036  & 0.0462  & 0.0426  & 0.7547  & 0.1196  & 6.3696  & 1.0557  & 6.1610  & 0.6489  & 0.1041  & 0.0217  & 0.0300  & -0.0078  & 1.0797  & 0.0457  & -0.0297  & -4.8101  \\
			&       &       &       &       &       &       &       &       &       &       &       &       &       &       &       &       &       &       &       &       &       &       &  \\
			\multicolumn{24}{l}{II: Jump Truncation Level=   $\alpha_n^{2}$}\\
			1     & 0.2544  & 0.2932  & 0.2039  & 0.1948  & 0.0505  & 0.0984  & 0.0091  & -0.0479  & -0.0389  & 0.8165  & -0.1267  & 11.0091  & 1.0279  & 6.4658  & 0.6066  & 0.2490  & -0.0102  & 0.0307  & -0.0479  & 1.0735  & 0.0383  & -0.0416  & -5.0815  \\
			2     & 0.2073  & 0.2298  & 0.0841  & 0.0807  & 0.1232  & 0.1491  & 0.0034  & -0.0259  & -0.0225  & 0.6363  & -0.0638  & 7.8185  & 1.1555  & 7.1345  & 0.5321  & 0.2663  & -0.0011  & 0.0273  & -0.0343  & 1.2693  & 0.0369  & -0.0359  & -6.1619  \\
			3     & 0.2426  & 0.2403  & 0.0742  & 0.0727  & 0.1684  & 0.1676  & 0.0015  & 0.0008  & 0.0023  & 0.6552  & 0.0247  & 7.5577  & 1.1228  & 6.8250  & 0.5648  & 0.2191  & 0.0077  & 0.0278  & -0.0247  & 1.2246  & 0.0396  & -0.0334  & -5.7402  \\
			4     & 0.2857  & 0.2628  & 0.0695  & 0.0702  & 0.2162  & 0.1926  & -0.0007  & 0.0236  & 0.0229  & 0.7052  & 0.0892  & 7.4364  & 1.0825  & 6.3955  & 0.6085  & 0.1713  & 0.0146  & 0.0290  & -0.0160  & 1.1439  & 0.0426  & -0.0320  & -5.1365  \\
			5     & 0.3595  & 0.2970  & 0.0559  & 0.0597  & 0.3036  & 0.2372  & -0.0038  & 0.0663  & 0.0625  & 0.8403  & 0.1955  & 7.1094  & 1.0041  & 5.7328  & 0.6809  & 0.1047  & 0.0248  & 0.0318  & -0.0072  & 0.9907  & 0.0489  & -0.0305  & -4.1697  \\
			\bottomrule
	\end{tabular}}
\end{table}
\begin{table}[htbp!]
	
	\centering
	%\caption*{\normalfont Panel I: Stocks Sorted by RVSJN}
	\large
	\resizebox{1.04\textwidth}{!}{
		\begin{tabular}{lccccccccccccccccccccccc}
			\multicolumn{24}{l}{Panel I: Stocks Sorted by RVSJN}\\
			&      &      &      &      &      &  & & & & & & & & & &
			&       &     &      &      &      &      & 
			\\
			\toprule
			Quintile & RVJP  & RVJN  & RVLJP & RVLJN & RVSJP & RVSJN & SRVLJ & SRVSJ & SRVJ  & RVOL  & RSK   & RKT   & BETA  & log(Size) & BEME  & MOM   & REV   & IVOL  & CSK   & CKT   & MAX   & MIN   & ILLIQ \\
			\multicolumn{24}{l}{I: Jump Truncation Level=   $\alpha_n^{1}$}\\
			1     & 0.3551  & 0.3214  & 0.3160  & 0.3080  & 0.0391  & 0.0134  & 0.0080  & 0.0257  & 0.0337  & 0.9590  & 0.1082  & 12.1457  & 0.8927  & 5.6753  & 0.6838  & 0.1872  & 0.0165  & 0.0332  & -0.0184  & 0.8993  & 0.0497  & -0.0365  & -3.9071  \\
			2     & 0.2429  & 0.2113  & 0.1418  & 0.1403  & 0.1011  & 0.0710  & 0.0015  & 0.0301  & 0.0317  & 0.6411  & 0.1010  & 8.1136  & 1.1447  & 7.0506  & 0.5359  & 0.2833  & 0.0200  & 0.0274  & -0.0130  & 1.2507  & 0.0437  & -0.0310  & -6.0577  \\
			3     & 0.2348  & 0.2254  & 0.1145  & 0.1128  & 0.1202  & 0.1126  & 0.0017  & 0.0076  & 0.0093  & 0.6294  & 0.0314  & 7.3957  & 1.1460  & 6.9879  & 0.5487  & 0.2464  & 0.0097  & 0.0272  & -0.0226  & 1.2554  & 0.0400  & -0.0327  & -5.9627  \\
			4     & 0.2425  & 0.2538  & 0.1015  & 0.0993  & 0.1411  & 0.1546  & 0.0022  & -0.0135  & -0.0113  & 0.6483  & -0.0296  & 6.9749  & 1.1274  & 6.7251  & 0.5789  & 0.1947  & 0.0004  & 0.0277  & -0.0321  & 1.2164  & 0.0373  & -0.0346  & -5.5925  \\
			5     & 0.2722  & 0.3121  & 0.0850  & 0.0810  & 0.1872  & 0.2311  & 0.0040  & -0.0439  & -0.0399  & 0.7773  & -0.1013  & 6.3684  & 1.0811  & 6.1132  & 0.6449  & 0.0988  & -0.0116  & 0.0312  & -0.0453  & 1.0789  & 0.0356  & -0.0389  & -4.7703  \\
			&       &       &       &       &       &       &       &       &       &       &       &       &       &       &       &       &       &       &       &       &       &       &  \\
			\multicolumn{24}{l}{II: Jump Truncation Level=   $\alpha_n^{2}$}\\
			1     & 0.3037  & 0.2449  & 0.2005  & 0.1970  & 0.1033  & 0.0479  & 0.0035  & 0.0553  & 0.0588  & 0.8010  & 0.2085  & 11.0050  & 0.9999  & 6.4752  & 0.6135  & 0.2361  & 0.0241  & 0.0300  & -0.0083  & 1.0656  & 0.0488  & -0.0320  & -5.0686  \\
			2     & 0.2343  & 0.2007  & 0.0821  & 0.0808  & 0.1522  & 0.1199  & 0.0012  & 0.0323  & 0.0336  & 0.6234  & 0.1168  & 7.7982  & 1.1433  & 7.1629  & 0.5346  & 0.2670  & 0.0180  & 0.0267  & -0.0142  & 1.2704  & 0.0425  & -0.0307  & -6.1901  \\
			3     & 0.2433  & 0.2383  & 0.0737  & 0.0730  & 0.1696  & 0.1654  & 0.0007  & 0.0043  & 0.0050  & 0.6491  & 0.0129  & 7.5670  & 1.1250  & 6.8323  & 0.5664  & 0.2274  & 0.0077  & 0.0275  & -0.0256  & 1.2267  & 0.0399  & -0.0333  & -5.7531  \\
			4     & 0.2654  & 0.2842  & 0.0713  & 0.0702  & 0.1941  & 0.2140  & 0.0011  & -0.0199  & -0.0188  & 0.7077  & -0.0646  & 7.4684  & 1.0945  & 6.3760  & 0.6068  & 0.1811  & -0.0010  & 0.0292  & -0.0347  & 1.1478  & 0.0384  & -0.0362  & -5.1156  \\
			5     & 0.3009  & 0.3580  & 0.0599  & 0.0565  & 0.2410  & 0.3015  & 0.0034  & -0.0605  & -0.0571  & 0.8751  & -0.1711  & 7.0781  & 1.0306  & 5.6903  & 0.6731  & 0.0959  & -0.0143  & 0.0334  & -0.0491  & 0.9894  & 0.0366  & -0.0417  & -4.1414  \\
			\bottomrule
	\end{tabular}}
\end{table}
\newpage
\noindent Table 3 (Continued)$^*\>$

\begin{table}[htbp!]
	\centering
	%\caption*{\normalfont Panel J: Stocks Sorted by RVOL}
	\large
	\resizebox{1.04\textwidth}{!}{
		\begin{tabular}{lccccccccccccccccccccccc}
			\multicolumn{24}{l}{Panel J: Stocks Sorted by RVOL}\\
			&      &      &      &      &      &  & & & & & & & & & &
			&       &     &      &      &      &      & 
			\\
			\toprule
			Quintile & RVJP  & RVJN  & RVLJP & RVLJN & RVSJP & RVSJN & SRVLJ & SRVSJ & SRVJ  & RVOL  & RSK   & RKT   & BETA  & log(Size) & BEME  & MOM   & REV   & IVOL  & CSK   & CKT   & MAX   & MIN   & ILLIQ \\
			%\midrule
			1     & 0.2266  & 0.2148  & 0.0595  & 0.0550  & 0.1671  & 0.1598  & 0.0045  & 0.0073  & 0.0118  & 0.2262  & 0.0502  & 6.8833  & 0.8315  & 8.3951  & 0.5374  & 0.1713  & 0.0042  & 0.0137  & -0.0197  & 1.4289  & 0.0176  & -0.0152  & -7.5683  \\
			2     & 0.2356  & 0.2261  & 0.0680  & 0.0637  & 0.1676  & 0.1624  & 0.0044  & 0.0052  & 0.0095  & 0.3601  & 0.0413  & 7.2723  & 1.0488  & 7.5277  & 0.5468  & 0.1944  & 0.0046  & 0.0187  & -0.0216  & 1.3822  & 0.0258  & -0.0227  & -6.5465  \\
			3     & 0.2538  & 0.2464  & 0.0855  & 0.0814  & 0.1683  & 0.1651  & 0.0041  & 0.0032  & 0.0073  & 0.5418  & 0.0344  & 7.8782  & 1.2299  & 6.6426  & 0.5532  & 0.2713  & 0.0052  & 0.0255  & -0.0255  & 1.2453  & 0.0350  & -0.0307  & -5.5936  \\
			4     & 0.2848  & 0.2807  & 0.1103  & 0.1076  & 0.1745  & 0.1732  & 0.0028  & 0.0013  & 0.0040  & 0.8249  & 0.0209  & 8.6859  & 1.2707  & 5.7644  & 0.5969  & 0.3071  & 0.0060  & 0.0340  & -0.0316  & 1.0513  & 0.0467  & -0.0407  & -4.4426  \\
			5     & 0.3379  & 0.3460  & 0.1617  & 0.1679  & 0.1762  & 0.1781  & -0.0062  & -0.0020  & -0.0081  & 1.7389  & -0.0329  & 10.1258  & 1.0064  & 4.3884  & 0.7408  & 0.0826  & 0.0163  & 0.0554  & -0.0330  & 0.6119  & 0.0827  & -0.0652  & -2.4052  \\
			\bottomrule
	\end{tabular}}
\end{table}
\begin{table}[htbp!]
	\centering
	%\caption*{\normalfont Panel K: Stocks Sorted by RSK}
	\large
	\resizebox{1.04\textwidth}{!}{
		\begin{tabular}{lccccccccccccccccccccccc}
			\multicolumn{24}{l}{Panel K: Stocks Sorted by RSK}\\
			&      &      &      &      &      &  & & & & & & & & & &
			&       &     &      &      &      &      & 
			\\
			\toprule
			Quintile & RVJP  & RVJN  & RVLJP & RVLJN & RVSJP & RVSJN & SRVLJ & SRVSJ & SRVJ  & RVOL  & RSK   & RKT   & BETA  & log(Size) & BEME  & MOM   & REV   & IVOL  & CSK   & CKT   & MAX   & MIN   & ILLIQ \\
			%\midrule
			1     & 0.2076  & 0.3955  & 0.0763  & 0.2166  & 0.1313  & 0.1790  & -0.1403  & -0.0477  & -0.1880  & 0.8227  & -1.0001  & 10.3914  & 1.0244  & 6.1284  & 0.6248  & 0.2070  & -0.0260  & 0.0318  & -0.0607  & 1.0479  & 0.0339  & -0.0474  & -4.6775  \\
			2     & 0.2191  & 0.2769  & 0.0529  & 0.0814  & 0.1662  & 0.1955  & -0.0285  & -0.0293  & -0.0578  & 0.7186  & -0.2700  & 6.8546  & 1.1294  & 6.6893  & 0.5736  & 0.2051  & -0.0076  & 0.0294  & -0.0416  & 1.1945  & 0.0361  & -0.0379  & -5.5563  \\
			3     & 0.2410  & 0.2386  & 0.0574  & 0.0569  & 0.1836  & 0.1816  & 0.0005  & 0.0019  & 0.0024  & 0.6988  & 0.0129  & 6.4478  & 1.1286  & 6.8001  & 0.5746  & 0.1939  & 0.0076  & 0.0284  & -0.0241  & 1.2137  & 0.0394  & -0.0327  & -5.7048  \\
			4     & 0.2772  & 0.2132  & 0.0822  & 0.0520  & 0.1949  & 0.1613  & 0.0303  & 0.0337  & 0.0639  & 0.6715  & 0.3009  & 6.8878  & 1.1150  & 6.7711  & 0.5784  & 0.2067  & 0.0229  & 0.0279  & -0.0100  & 1.2025  & 0.0442  & -0.0290  & -5.6479  \\
			5     & 0.3982  & 0.1971  & 0.2191  & 0.0735  & 0.1791  & 0.1237  & 0.1457  & 0.0554  & 0.2010  & 0.7315  & 1.0529  & 10.4423  & 0.9964  & 6.2355  & 0.6342  & 0.2011  & 0.0379  & 0.0290  & 0.0045  & 1.0556  & 0.0523  & -0.0266  & -4.8018  \\
			\bottomrule
	\end{tabular}}
\end{table}
\begin{table}[htbp!]
	\centering
	%\caption*{\normalfont Panel L: Stocks Sorted by RKT}
	\large
	\resizebox{1.04\textwidth}{!}{
		\begin{threeparttable}	
			\begin{tabular}{lccccccccccccccccccccccc}
				\multicolumn{24}{l}{Panel L: Stocks Sorted by RKT}\\
				&      &      &      &      &      &  & & & & & & & & & &
				&       &     &      &      &      &      & 
				\\
				\toprule
				Quintile & RVJP  & RVJN  & RVLJP & RVLJN & RVSJP & RVSJN & SRVLJ & SRVSJ & SRVJ  & RVOL  & RSK   & RKT   & BETA  & log(Size) & BEME  & MOM   & REV   & IVOL  & CSK   & CKT   & MAX   & MIN   & ILLIQ \\
				%\midrule
				1     & 0.1807  & 0.1789  & 0.0055  & 0.0055  & 0.1751  & 0.1735  & 0.0001  & 0.0017  & 0.0017  & 0.5742  & 0.0103  & 4.4564  & 1.1919  & 7.5935  & 0.5326  & 0.1860  & 0.0061  & 0.0251  & -0.0222  & 1.3497  & 0.0350  & -0.0294  & -6.7372  \\
				2     & 0.2248  & 0.2216  & 0.0286  & 0.0281  & 0.1963  & 0.1935  & 0.0005  & 0.0028  & 0.0032  & 0.6580  & 0.0153  & 5.7883  & 1.1572  & 6.9202  & 0.5546  & 0.2185  & 0.0082  & 0.0279  & -0.0250  & 1.2461  & 0.0397  & -0.0329  & -5.9273  \\
				3     & 0.2639  & 0.2597  & 0.0692  & 0.0683  & 0.1947  & 0.1914  & 0.0009  & 0.0033  & 0.0043  & 0.7256  & 0.0194  & 7.1066  & 1.1084  & 6.4820  & 0.5838  & 0.2165  & 0.0084  & 0.0298  & -0.0265  & 1.1574  & 0.0422  & -0.0351  & -5.3165  \\
				4     & 0.3077  & 0.3021  & 0.1341  & 0.1321  & 0.1736  & 0.1699  & 0.0020  & 0.0037  & 0.0057  & 0.7919  & 0.0252  & 9.0206  & 1.0407  & 6.0670  & 0.6199  & 0.2075  & 0.0072  & 0.0313  & -0.0280  & 1.0591  & 0.0438  & -0.0371  & -4.6637  \\
				5     & 0.3731  & 0.3634  & 0.2591  & 0.2527  & 0.1140  & 0.1107  & 0.0064  & 0.0033  & 0.0097  & 0.9025  & 0.0433  & 14.9583  & 0.8864  & 5.4986  & 0.7007  & 0.1840  & 0.0054  & 0.0325  & -0.0292  & 0.8875  & 0.0455  & -0.0390  & -3.6495  \\
				\bottomrule
			\end{tabular}
			\begin{tablenotes}
				\large
				\item {}
				\item {\large $^*\>$ Notes: See notes to Table 2. Entries in this table are time series averages of equal-weighted realized measures and firm characteristics of stocks sorted by various realized measures. The sample includes all NYSE, NASDAQ and AMEX listed stocks for the period January 1993 to December 2016. At the end of each Tuesday, all of the stocks in the sample are sorted into quintile portfolios, based on ascending values of various realized measures. The equal-weighted realized measures and firm characteristics of each quintile portfolio are calculated over the same week. Additionally, $\>\alpha_n^{1}=4\sqrt{\frac{1}{t}\widehat{IV}_t}\triangle_n^{0.49} \>$ and $\>\alpha_n^{2}=5\sqrt{\frac{1}{t}\widehat{IV}_t}\triangle_n^{0.49}\>$ are jump truncation levels. See Sections 2 and 4 for further details.}
			\end{tablenotes}
	\end{threeparttable}}
\end{table}
\newpage
\centering 
Table 4: Univariate Portfolio Sorts Based on Positive, Negative, and Signed Total Jump Variation$^*\>$

\small
% Table generated by Excel2LaTeX from sheet 'r_rvjp'
\begin{table}[htbp!]
	\centering
	\resizebox{1.04\textwidth}{!}{
		\begin{threeparttable}
			%	\begin{tabular}{llllllllllllll}
			\begin{tabular}{lrrrrrrrrrrrrr}	
				%				Panel A: RVJP &      &      &      &      &      & 
				%				&       &     &      &      &      &      & 
				%				\\
				\multicolumn{14}{l}{Panel A: Stocks Sorted by RVJP}\\
				&      &      &      &      &      & 
				&       &     &      &      &      &      & 
				\\
				&    & \multicolumn{4}{c}{Equal-Weighted Returns and Alphas }     &       &       &  & \multicolumn{4}{c}{Value-Weighted Returns and Alphas}         \\
				\midrule
				Quintile & 1(Low) & 2     & 3     & 4     & 5(High) & High-Low &       & 1(Low) & 2     & 3     & 4     & 5(High) & High-Low \\
				\multicolumn{1}{l}{Mean Return} & 33.59 & 29.98 & 31.91 & 26.99 & 22.58 & -11.01** &       & 23.50 & 19.57 & 17.99 & 20.95 & 17.34 & -6.16 \\
				& (3.52) & (3.22) & (3.39) & (2.76) & (2.44) & (-2.42) &       & (3.53) & (3.29) & (2.93) & (3.37) & (2.71) & (-1.62) \\
				\multicolumn{1}{l}{Alpha} & 10.59 & 6.89  & 9.99  & 6.39  & 4.64  & -5.96 &       & 2.89  & -0.68 & -2.34 & 0.60  & -3.09 & -5.98** \\
				& (4.13) & (3.35) & (4.18) & (2.04) & (1.22) & (-1.34) &       & (2.33) & (-0.49) & (-1.26) & (0.28) & (-1.33) & (-1.99) \\
				\midrule
				&       &       &       &       &       &       &       &       &       &       &       &       &  \\
				\multicolumn{14}{l}{Panel B: Stocks Sorted by RVJN}\\
				&      &      &      &      &      & 
				&       &     &      &      &      &      & 
				\\
				&    & \multicolumn{4}{c}{Equal-Weighted Returns and Alphas }     &       &       &  & \multicolumn{4}{c}{Value-Weighted Returns and Alphas}         \\
				\midrule
				Quintile & 1(Low) & 2     & 3     & 4     & 5(High) & High-Low &       & 1(Low) & 2     & 3     & 4     & 5(High) & High-Low \\
				\multicolumn{1}{l}{Mean Return} & 14.45 & 23.61 & 27.12 & 33.12 & 46.78 & 32.34*** &       & 16.24 & 25.76 & 26.38 & 26.27 & 31.49 & 15.25*** \\
				& (1.66) & (2.66) & (2.89) & (3.32) & (4.41) & (5.85) &       & (2.55) & (4.15) & (4.07) & (3.89) & (4.32) & (3.77) \\
				\multicolumn{1}{l}{Alpha} & -8.08 & 0.54  & 4.50  & 12.20 & 29.38 & 37.46*** &       & -3.58 & 5.32  & 5.11  & 5.30  & 10.62 & 14.20*** \\
				& (-3.68) & (0.32) & (2.03) & (3.86) & (6.07) & (7.00) &       & (-3.13) & (3.30) & (2.55) & (2.30) & (4.29) & (4.69) \\
				\midrule
				&       &       &       &       &       &       &       &       &       &       &       &       &  \\
				\multicolumn{14}{l}{Panel C: Stocks Sorted by SRVJ}\\
				&      &      &      &      &      & 
				&       &     &      &      &      &      & 
				\\
				&    & \multicolumn{4}{c}{Equal-Weighted Returns and Alphas }     &       &       &  & \multicolumn{4}{c}{Value-Weighted Returns and Alphas}         \\
				\midrule
				Quintile & 1(Low) & 2     & 3     & 4     & 5(High) & High-Low &       & 1(Low) & 2     & 3     & 4     & 5(High) & High-Low \\
				\multicolumn{1}{l}{Mean Return} & 50.20 & 37.96 & 25.45 & 17.55 & 13.87 & -36.33*** &       & 34.30 & 27.74 & 19.78 & 13.62 & 9.61  & -24.68*** \\
				& (4.87) & (3.80) & (2.64) & (1.98) & (1.67) & (-8.56) &       & (4.81) & (4.16) & (3.09) & (2.16) & (1.57) & (-5.70) \\
				\multicolumn{1}{l}{Alpha} & 29.21 & 16.62 & 3.68  & -4.19 & -6.85 & -36.06*** &       & 13.01 & 7.19  & -0.60 & -6.48 & -10.34 & -23.35*** \\
				& (7.75) & (5.76) & (1.47) & (-2.07) & (-2.82) & (-9.00) &       & (4.90) & (4.05) & (-0.45) & (-4.47) & (-4.49) & (-5.47) \\
				\bottomrule
			\end{tabular}
			\begin{tablenotes}
				
				\item {}
				\item { $^*\>$ Notes: Entries in this table are average returns and risk-adjusted alphas for single-sorted portfolios based on RVJP, RVJN and SRVJ, which are described in Table 2. The sample includes all NYSE, NASDAQ and AMEX listed stocks for the period January 1993 to December 2016. At the end of each Tuesday, all the stocks in the sample are sorted into quintile portfolios based on ascending values of the various jump variation measures listed in the titel of each panel. Each portfolio is held for one week. The row labeled "Mean Return" reports the time series average values of one-week ahead equal-weighted and value-weighted returns for quintile portfolios. The row labeled "Alpha" reports Fama-French-Carhart four-factor alphas, based on the model (\ref{eq12}), for each of the quintile portfolios, as well as for the difference between portfolio 5 and portfolio 1. Newey-West $t$-statistics are given in parentheses; and
					*, **, and *** denote means and alphas that are significant at the 10\%, 5\%, and 1\% levels, respectively. }
			\end{tablenotes}
	\end{threeparttable}}
\end{table}

\newpage
\noindent Table 5: Univariate Portfolio Sorts Based on Positive, Negative, and Signed Large Jump Variation$^*\>$ 
\small
% Table generated by Excel2LaTeX from sheet 'r_rvljp'
\begin{table}[htbp!]
	\centering
	\resizebox{1.04\textwidth}{!}{
		\begin{threeparttable}
			%\begin{tabular}{llllllllllllll}
			\begin{tabular}{lrrrrrrrrrrrrr}
				\multicolumn{14}{l}{Panel A: Stocks Sorted by RVLJP}\\
				&      &      &      &      &      & 
				&       &     &      &      &      &      & 
				\\
				&    & \multicolumn{4}{c}{Equal-Weighted Returns and Alphas }     &       &       &  & \multicolumn{4}{c}{Value-Weighted Returns and Alphas}         \\
				\midrule
				Quintile & 1(Low) & 2     & 3     & 4     & 5(High) & High-Low &       & 1(Low) & 2     & 3     & 4     & 5(High) & High-Low \\
				\multicolumn{5}{l}{I: Jump Truncation Level=$\alpha_n^{1}$} &       &       &       &       &       &       &       &         \\
				\multicolumn{1}{l}{Mean Return} & 30.66 & 29.05 & 28.26 & 30.05 & 26.09 & -4.57 &       & 21.66 & 20.51 & 20.51 & 18.51 & 18.35 & -3.31 \\
				& (3.23) & (3.08) & (3.02) & (3.16) & (2.76) & (-1.25) &       & (3.39) & (3.18) & (3.30) & (2.94) & (2.76) & (-0.90) \\
				\multicolumn{1}{l}{Alpha} & 9.19  & 7.12  & 5.55  & 8.02  & 8.30  & -0.89 &       & 1.80  & 0.79  & -0.63 & -2.41 & -2.03 & -3.83 \\
				& (3.86) & (3.22) & (2.56) & (3.00) & (2.17) & (-0.25) &       & (1.77) & (0.59) & (-0.40) & (-1.32) & (-0.84) & (-1.34) \\
				&       &       &       &       &       &       &       &       &       &       &       &       &  \\
				\multicolumn{5}{l}{II: Jump Truncation Level=$\alpha_n^{2}$} &       &       &       &       &       &       &       &         \\
				\multicolumn{1}{l}{Mean Return} & 28.50 & 38.62 & 28.85 & 28.57 & 27.71 & -0.79 &       & 20.27 & 31.60 & 19.78 & 22.41 & 20.45 & 0.18 \\
				& (3.06) & (2.00) & (2.90) & (3.02) & (2.97) & (-0.25) &       & (3.22) & (1.77) & (2.93) & (3.62) & (3.14) & (0.05) \\
				\multicolumn{1}{l}{Alpha} & 6.72  & 30.26 & 6.58  & 6.42  & 9.57  & 2.85  &       & 0.28  & 20.92 & -0.39 & 1.55  & -0.38 & -0.66 \\
				& (3.36) & (3.82) & (2.55) & (2.45) & (2.58) & (0.90) &       & (0.44) & (1.62) & (-0.19) & (0.82) & (-0.16) & (-0.25) \\
				\midrule
				&       &       &       &       &       &       &       &       &       &       &       &       &  \\
				\multicolumn{14}{l}{Panel B: Stocks Sorted by RVLJN}\\
				&      &      &      &      &      & 
				&       &     &      &      &      &      & 
				\\
				&    & \multicolumn{4}{c}{Equal-Weighted Returns and Alphas }     &       &       &  & \multicolumn{4}{c}{Value-Weighted Returns and Alphas}         \\
				\midrule
				Quintile & 1(Low) & 2     & 3     & 4     & 5(High) & High-Low &       & 1(Low) & 2     & 3     & 4     & 5(High) & High-Low \\
				\multicolumn{5}{l}{I: Jump Truncation Level=$\alpha_n^{1}$} &       &       &       &       &       &       &       &         \\
				\multicolumn{1}{l}{Raw Returns} & 23.90 & 22.69 & 27.46 & 29.22 & 40.66 & 16.76*** &       & 19.51 & 19.76 & 25.43 & 22.62 & 25.57 & 6.06 \\
				& (2.67) & (2.50) & (2.97) & (2.98) & (4.01) & (4.18) &       & (3.08) & (3.12) & (3.95) & (3.35) & (3.52) & (1.53) \\
				\multicolumn{1}{l}{Alpha} & 2.31  & 0.39  & 4.69  & 7.39  & 22.96 & 20.66*** &       & -0.38 & -1.04 & 4.37  & 1.38  & 4.74  & 5.13* \\
				& (1.14) & (0.20) & (2.31) & (2.60) & (5.23) & (5.27) &       & (-0.36) & (-0.75) & (2.38) & (0.63) & (1.97) & (1.79) \\
				&       &       &       &       &       &       &       &       &       &       &       &       &  \\
				\multicolumn{5}{l}{II: Jump Truncation Level=$\alpha_n^{2}$} &       &       &       &       &       &       &       &         \\
				\multicolumn{1}{l}{Raw Returns} & 24.40 & 7.76  & 27.44 & 28.96 & 38.77 & 14.36*** &       & 19.68 & 12.05 & 21.05 & 23.34 & 22.64 & 2.96 \\
				& (2.68) & (0.42) & (2.73) & (3.03) & (3.87) & (4.07) &       & (3.13) & (0.88) & (3.00) & (3.62) & (3.22) & (0.84) \\
				\multicolumn{1}{l}{Alpha} & 2.55  & 6.99  & 6.26  & 6.74  & 20.92 & 18.37*** &       & -0.33 & 3.58  & 0.98  & 2.46  & 2.05  & 2.38 \\
				& (1.44) & (1.14) & (2.50) & (2.73) & (4.79) & (5.14) &       & (-0.51) & (0.54) & (0.45) & (1.15) & (0.83) & (0.86) \\
				\midrule
				&       &       &       &       &       &       &       &       &       &       &       &       &  \\
				\multicolumn{14}{l}{Panel C: Stocks Sorted by SRVLJ}\\
				&      &      &      &      &      & 
				&       &     &      &      &      &      & 
				\\
				&    & \multicolumn{4}{c}{Equal-Weighted Returns and Alphas }     &       &       &  & \multicolumn{4}{c}{Value-Weighted Returns and Alphas}         \\
				\midrule
				Quintile & 1(Low) & 2     & 3     & 4     & 5(High) & High-Low &       & 1(Low) & 2     & 3     & 4     & 5(High) & High-Low \\
				\multicolumn{5}{l}{I: Jump Truncation Level=$\alpha_n^{1}$} &       &       &       &       &       &       &       &         \\
				\multicolumn{1}{l}{Mean Return} & 43.77 & 31.40 & 28.80 & 22.13 & 18.95 & -24.82*** &       & 26.80 & 22.74 & 21.38 & 17.94 & 15.64 & -11.15*** \\
				& (4.36) & (3.23) & (2.98) & (2.40) & (2.23) & (-7.49) &       & (4.00) & (3.55) & (3.16) & (2.83) & (2.60) & (-3.39) \\
				\multicolumn{1}{l}{Alpha} & 23.26 & 9.74  & 7.05  & 0.09  & -2.04 & -25.31*** &       & 5.44  & 2.19  & 1.15  & -2.53 & -4.82 & -10.26*** \\
				& (6.63) & (3.87) & (2.78) & (0.04) & (-0.88) & (-8.01) &       & (2.43) & (1.45) & (0.74) & (-1.76) & (-2.46) & (-3.13) \\
				&       &       &       &       &       &       &       &       &       &       &       &       &  \\
				\multicolumn{5}{l}{II: Jump Truncation Level=$\alpha_n^{2}$} &       &       &       &       &       &       &       &         \\
				\multicolumn{1}{l}{Mean Return} & 39.80 & 27.71 & 3.61  & 24.22 & 21.49 & -18.31*** &       & 23.36 & 20.68 & 5.71  & 18.96 & 20.35 & -3.01 \\
				& (4.03) & (2.87) & (0.19) & (2.57) & (2.48) & (-6.56) &       & (3.52) & (3.23) & (0.41) & (3.04) & (3.30) & (-0.98) \\
				\multicolumn{1}{l}{Alpha} & 19.24 & 7.13  & 1.22  & 2.09  & 0.39  & -18.85*** &       & 2.50  & 0.74  & -2.13 & -2.06 & -0.15 & -2.65 \\
				& (5.67) & (3.10) & (0.18) & (0.82) & (0.17) & (-6.88) &       & (1.18) & (0.67) & (-0.38) & (-0.99) & (-0.07) & (-0.87) \\
				\bottomrule
			\end{tabular}
			\begin{tablenotes}
				
				\item{}
				\item { $^*\>$ Notes: See notes to Table 4. Entries are average returns and risk-adjusted alphas for single-sorted portfolios based on RVLJP, RVLJN and SRVLJ. Jump truncation levels are  $\>\alpha_n^{1}=4\sqrt{\frac{1}{t}\widehat{IV}_t}\triangle_n^{0.49} \>$ and $\>\alpha_n^{2}=5\sqrt{\frac{1}{t}\widehat{IV}_t}\triangle_n^{0.49}\>$. }
			\end{tablenotes}
	\end{threeparttable}}
\end{table}
\newpage
\noindent Table 6: Univariate Portfolio Sorts Based on Positive, Negative, and Signed Small Jump Variation$^*\>$ 

\small
% Table generated by Excel2LaTeX from sheet 'r_rvsjp'
% Table generated by Excel2LaTeX from sheet 'r_rvsjp'
\begin{table}[htbp!]
	\centering
	\resizebox{1.04\textwidth}{!}{
		\begin{threeparttable} 
			%\begin{tabular}{llllllllllllll}
			\begin{tabular}{lrrrrrrrrrrrrr}
				\multicolumn{14}{l}{Panel A: Stocks Sorted by RVSJP}\\
				&      &      &      &      &      & 
				&       &     &      &      &      &      & 
				\\
				&    & \multicolumn{4}{c}{Equal-Weighted Returns and Alphas }     &       &       &  & \multicolumn{4}{c}{Value-Weighted Returns and Alphas}         \\
				\midrule
				Quintile & 1(Low) & 2     & 3     & 4     & 5(High) & High-Low &       & 1(Low) & 2     & 3     & 4     & 5(High) & High-Low \\
				\multicolumn{5}{l}{I: Jump Truncation Level=$\alpha_n^{1}$} &       &       &       &       &       &       &       &         \\
				\multicolumn{1}{l}{Mean Return} & 33.45 & 32.60 & 28.54 & 25.46 & 24.79 & -8.66*** &       & 28.07 & 26.13 & 18.48 & 16.51 & 14.70 & -13.37*** \\
				& (3.58) & (3.44) & (3.04) & (2.74) & (2.57) & (-2.89) &       & (3.67) & (3.85) & (2.92) & (2.75) & (2.34) & (-2.73) \\
				\multicolumn{1}{l}{Alpha} & 14.46 & 9.56  & 5.83  & 3.40  & 5.09  & -9.37*** &       & 7.48  & 5.15  & -1.69 & -3.00 & -5.45 & -12.92*** \\
				& (4.13) & (4.31) & (2.80) & (1.63) & (1.58) & (-3.44) &       & (2.50) & (3.28) & (-1.34) & (-1.59) & (-2.52) & (-3.06) \\
				&       &       &       &       &       &       &       &       &       &       &       &       &  \\
				\multicolumn{5}{l}{II: Jump Truncation Level=$\alpha_n^{2}$} &       &       &       &       &       &       &       &         \\
				\multicolumn{1}{l}{Raw Returns} & 35.68 & 31.17 & 27.86 & 27.70 & 22.59 & -13.10*** &       & 29.12 & 19.23 & 17.69 & 14.63 & 17.96 & -11.16** \\
				& (3.84) & (3.36) & (2.98) & (2.94) & (2.34) & (-4.33) &       & (4.07) & (3.03) & (2.93) & (2.36) & (2.77) & (-2.44) \\
				\multicolumn{1}{l}{Alpha} & 15.57 & 8.48  & 5.20  & 6.04  & 3.16  & -12.41*** &       & 8.21  & -1.38 & -2.25 & -5.47 & -2.15 & -10.36*** \\
				& (5.21) & (4.10) & (2.46) & (2.51) & (0.96) & (-4.53) &       & (3.82) & (-1.10) & (-1.41) & (-2.75) & (-0.85) & (-2.62) \\
				\midrule
				&       &       &       &       &       &       &       &       &       &       &       &       &  \\
				\multicolumn{14}{l}{Panel B: Stocks Sorted by RVSJN}\\
				&      &      &      &      &      & 
				&       &     &      &      &      &      & 
				\\
				&    & \multicolumn{4}{c}{Equal-Weighted Returns and Alphas }     &       &       &  & \multicolumn{4}{c}{Value-Weighted Returns and Alphas}         \\
				\midrule
				Quintile & 1(Low) & 2     & 3     & 4     & 5(High) & High-Low &       & 1(Low) & 2     & 3     & 4     & 5(High) & High-Low \\
				\multicolumn{5}{l}{I: Jump Truncation Level=$\alpha_n^{1}$} &       &       &       &       &       &       &       &         \\
				\multicolumn{1}{l}{Mean Return} & 25.39 & 21.47 & 27.03 & 30.55 & 40.60 & 15.22*** &       & 7.51  & 15.66 & 22.33 & 27.71 & 31.31 & 23.80*** \\
				& (2.81) & (2.37) & (2.96) & (3.25) & (3.85) & (4.30) &       & (1.08) & (2.38) & (3.52) & (4.33) & (4.71) & (4.83) \\
				\multicolumn{1}{l}{Alpha} & 7.03  & -1.32 & 4.19  & 8.33  & 20.23 & 13.20*** &       & -12.73 & -4.64 & 2.17  & 7.37  & 10.91 & 23.64*** \\
				& (2.06) & (-0.66) & (2.20) & (3.68) & (5.53) & (4.44) &       & (-4.22) & (-3.09) & (1.62) & (3.72) & (4.18) & (5.05) \\
				&       &       &       &       &       &       &       &       &       &       &       &       &  \\
				\multicolumn{5}{l}{II: Jump Truncation Level=$\alpha_n^{2}$} &       &       &       &       &       &       &       &         \\
				\multicolumn{1}{l}{Mean Return} & 21.90 & 22.19 & 26.78 & 31.57 & 42.65 & 20.75*** &       & 14.12 & 18.74 & 26.42 & 29.31 & 32.36 & 18.24*** \\
				& (2.49) & (2.52) & (2.95) & (3.27) & (3.96) & (5.27) &       & (2.11) & (2.99) & (4.24) & (4.47) & (4.36) & (3.69) \\
				\multicolumn{1}{l}{Alpha} & 2.16  & -0.30 & 4.35  & 9.46  & 22.87 & 20.71*** &       & -5.60 & -1.41 & 6.51  & 8.03  & 11.21 & 16.80*** \\
				& (0.77) & (-0.17) & (2.25) & (3.84) & (5.74) & (6.32) &       & (-2.77) & (-1.18) & (3.93) & (3.80) & (3.90) & (4.13) \\
				\midrule
				&       &       &       &       &       &       &       &       &       &       &       &       &  \\
				\multicolumn{14}{l}{Panel C: Stocks Sorted by SRVSJ}\\
				&      &      &      &      &      & 
				&       &     &      &      &      &      & 
				\\
				&    & \multicolumn{4}{c}{Equal-Weighted Returns and Alphas }     &       &       &  & \multicolumn{4}{c}{Value-Weighted Returns and Alphas}         \\
				\midrule
				Quintile & 1(Low) & 2     & 3     & 4     & 5(High) & High-Low &       & 1(Low) & 2     & 3     & 4     & 5(High) & High-Low \\
				\multicolumn{5}{l}{I: Jump Truncation Level=$\alpha_n^{1}$} &       &       &       &       &       &       &       &         \\
				\multicolumn{1}{l}{Mean Return} & 45.69 & 40.65 & 24.63 & 19.11 & 13.38 & -32.30*** &       & 34.89 & 24.28 & 17.52 & 17.95 & 10.45 & -24.44*** \\
				& (4.49) & (4.06) & (2.66) & (2.08) & (1.58) & (-8.53) &       & (5.05) & (3.61) & (2.72) & (2.86) & (1.72) & (-6.67) \\
				\multicolumn{1}{l}{Alpha} & 22.86 & 19.86 & 4.62  & -2.14 & -8.56 & -31.42*** &       & 14.02 & 3.44  & -3.04 & -2.16 & -9.27 & -23.28*** \\
				& (7.47) & (6.27) & (1.48) & (-0.92) & (-4.38) & (-9.00) &       & (6.30) & (2.04) & (-1.44) & (-1.36) & (-4.99) & (-6.59) \\
				&       &       &       &       &       &       &       &       &       &       &       &       &  \\
				\multicolumn{5}{l}{II: Jump Truncation Level=$\alpha_n^{2}$} &       &       &       &       &       &       &       &         \\
				\multicolumn{1}{l}{Mean Return} & 46.86 & 41.73 & 27.46 & 17.42 & 11.43 & -35.42*** &       & 37.42 & 25.19 & 18.30 & 14.93 & 9.27  & -28.15*** \\
				& (4.60) & (4.13) & (2.91) & (1.93) & (1.37) & (-8.77) &       & (5.41) & (3.78) & (2.84) & (2.37) & (1.50) & (-7.09) \\
				\multicolumn{1}{l}{Alpha} & 24.41 & 20.79 & 7.13  & -3.68 & -10.39 & -34.80*** &       & 16.57 & 4.46  & -1.82 & -5.32 & -10.55 & -27.12*** \\
				& (7.53) & (6.89) & (2.31) & (-1.57) & (-5.23) & (-9.27) &       & (6.82) & (2.56) & (-1.16) & (-3.30) & (-5.10) & (-6.86) \\
				\bottomrule
			\end{tabular}
			\begin{tablenotes}
				
				\item{}
				\item { $^*\>$ Notes: See notes to Table 5.}
			\end{tablenotes}
	\end{threeparttable}}
\end{table}
\newpage
\small
\noindent Table 7: Univariate Portfolio Sorts Based on Realized Volatility, Skewness, Kurtosis and Continuous Variance$^*\>$ 

\small
% Table generated by Excel2LaTeX from sheet 'rvol'
% Table generated by Excel2LaTeX from sheet 'rvol'
\begin{table}[htbp!]
	\centering
	\resizebox{1.04\textwidth}{!}{
		\begin{threeparttable}
			\begin{tabular}{lrrrrrrrrrrrrr}
				\multicolumn{14}{l}{Panel A: Stocks Sorted by RVOL}\\
				&      &      &      &      &      & 
				&       &     &      &      &      &      & 
				\\
				&    & \multicolumn{4}{c}{Equal-Weighted Returns and Alphas }     &       &       &  & \multicolumn{4}{c}{Value-Weighted Returns and Alphas}         \\
				\midrule
				Quintile & 1(Low) & 2     & 3     & 4     & 5(High) & High-Low &       & 1(Low) & 2     & 3     & 4     & 5(High) & High-Low \\
				\multicolumn{1}{l}{Mean Return} & 24.41 & 27.99 & 29.22 & 31.58 & 32.24 & 7.83  &       & 20.75 & 21.20 & 20.39 & 26.26 & 28.01 & 7.26 \\
				& (4.64) & (3.92) & (3.01) & (2.58) & (2.14) & (0.61) &       & (4.09) & (2.81) & (1.91) & (1.95) & (1.85) & (0.55) \\
				\multicolumn{1}{l}{Alpha} & 5.69  & 4.98  & 5.48  & 8.45  & 14.35 & 8.66  &       & 1.93  & -1.68 & -3.24 & 1.99  & 3.69  & 1.76 \\
				& (2.54) & (2.93) & (2.71) & (2.49) & (1.94) & (1.05) &       & (1.32) & (-0.84) & (-0.83) & (0.36) & (0.48) & (0.21) \\
				\midrule
				&       &       &       &       &       &       &       &       &       &       &       &       &  \\
				\multicolumn{14}{l}{Panel B: Stocks Sorted by RSK}\\
				&      &      &      &      &      & 
				&       &     &      &      &      &      & 
				\\
				&    & \multicolumn{4}{c}{Equal-Weighted Returns and Alphas }     &       &       &  & \multicolumn{4}{c}{Value-Weighted Returns and Alphas}         \\
				\midrule
				Quintile & 1(Low) & 2     & 3     & 4     & 5(High) & High-Low &       & 1(Low) & 2     & 3     & 4     & 5(High) & High-Low \\
				\multicolumn{1}{l}{Mean Return} & 46.41 & 36.99 & 26.90 & 19.08 & 15.66 & -30.75*** &       & 29.13 & 27.58 & 19.44 & 14.29 & 13.34 & -15.78*** \\
				& (4.67) & (3.72) & (2.78) & (2.11) & (1.86) & (-8.61) &       & (4.23) & (4.22) & (3.02) & (2.25) & (2.22) & (-4.18) \\
				\multicolumn{1}{l}{Alpha} & 25.36 & 15.59 & 5.58  & -2.73 & -5.31 & -30.67*** &       & 7.41  & 7.01  & -0.66 & -5.63 & -6.72 & -14.13*** \\
				& (7.40) & (5.43) & (2.19) & (-1.36) & (-2.32) & (-8.88) &       & (3.12) & (4.41) & (-0.47) & (-3.91) & (-3.25) & (-3.73) \\
				\midrule
				&       &       &       &       &       &       &       &       &       &       &       &       &  \\
				\multicolumn{14}{l}{Panel C: Stocks Sorted by RKT}\\
				&      &      &      &      &      & 
				&       &     &      &      &      &      & 
				\\
				&    & \multicolumn{4}{c}{Equal-Weighted Returns and Alphas }     &       &       &  & \multicolumn{4}{c}{Value-Weighted Returns and Alphas}         \\
				\midrule
				Quintile & 1(Low) & 2     & 3     & 4     & 5(High) & High-Low &       & 1(Low) & 2     & 3     & 4     & 5(High) & High-Low \\
				\multicolumn{1}{l}{Mean Return} & 28.28 & 27.88 & 28.84 & 29.61 & 30.46 & 2.18  &       & 19.90 & 21.67 & 20.51 & 22.77 & 19.51 & -0.39 \\
				& (3.00) & (2.92) & (3.03) & (3.11) & (3.39) & (0.67) &       & (3.12) & (3.43) & (3.28) & (3.48) & (2.96) & (-0.12) \\
				\multicolumn{1}{l}{Alpha} & 7.75  & 5.77  & 6.30  & 7.95  & 10.75 & 3.00  &       & 0.26  & 0.73  & -0.79 & 1.62  & -1.90 & -2.16 \\
				& (2.99) & (2.55) & (2.72) & (3.05) & (3.49) & (1.03) &       & (0.25) & (0.52) & (-0.45) & (0.85) & (-0.89) & (-0.82) \\
				\midrule
				&       &       &       &       &       &       &       &       &       &       &       &       &  \\
				\multicolumn{14}{l}{Panel D: Stocks Sorted by RVC}\\
				&      &      &      &      &      & 
				&       &     &      &      &      &      & 
				\\
				&    & \multicolumn{4}{c}{Equal-Weighted Returns and Alphas }     &       &       &  & \multicolumn{4}{c}{Value-Weighted Returns and Alphas}         \\
				\midrule
				Quintile & 1(Low) & 2     & 3     & 4     & 5(High) & High-Low &       & 1(Low) & 2     & 3     & 4     & 5(High) & High-Low \\
				\multicolumn{1}{l}{Mean Return} & 37.18 & 30.05 & 28.14 & 27.48 & 22.21 & -14.96*** &       & 24.18 & 24.44 & 23.64 & 23.38 & 20.34 & -3.83 \\
				& (3.62) & (3.06) & (3.00) & (3.07) & (2.42) & (-2.75) &       & (3.52) & (3.59) & (3.69) & (3.87) & (3.19) & (-0.99) \\
				\multicolumn{1}{l}{Alpha} & 20.99 & 8.57  & 5.28  & 4.63  & -0.97 & -21.96*** &       & 4.59  & 2.90  & 3.06  & 3.15  & -0.01 & -4.59 \\
				& (4.24) & (2.89) & (2.50) & (2.49) & (-0.42) & (-4.08) &       & (1.90) & (1.42) & (1.65) & (2.04) & (-0.01) & (-1.62) \\
				\bottomrule
			\end{tabular}
			\begin{tablenotes}
				
				\item{}
				\item { $^*\>$ Notes: See notes to Tables 5. }
			\end{tablenotes}
	\end{threeparttable}}
\end{table}

%\newpage
%\centering
%\noindent Table 8: Sharpe Ratio \footnote{The reported signed large and small jump variations are contructed based on truncation level $\alpha_n^{(2)}=5\sqrt{\frac{1}{t}\widehat{IV}_t}\triangle_n^{0.49}$.}\\
%
%\begin{table}[htbp!]
%	\centering
%	\resizebox{0.6\textwidth}{!}{
%		\begin{tabular}{lrcccc}
%			\toprule
%			\multicolumn{2}{r}{} & SRVJ  & SRVLJ & SRVSJ & RSK    \\
%			\midrule
%			\multicolumn{2}{l}{Equal-Weighted} & 1.8498  & 1.4760  & 1.7561  & 1.8986   \\
%			\multicolumn{2}{l}{Value-Weighted } & 1.1278  & 0.1983  & 1.3014  & 0.8425    \\
%			\bottomrule\\
%	\end{tabular}}\\	 
%	\justify
%	Notes: This table presents the annual Sharpe ratio of equal-weighted and value-weighted short-long portfolios based on different risk measures. The sample includes all the NYSE, NASDAQ and AMEX listed stocks from January 1993 to December 2016. At the end of each Tuesday, all the stocks in the sample are sorted into quintile portfolios based on ascending value of various realized variations. A low-high spread portfolio is formed as the difference between portfolio 1 and portfolio 5 and hold for one week. The Sharpe ratio is calculated with the one-week ahead returns. See notes to Table 1.\\
%	
%\end{table}
\newpage
\small
\noindent Table 8: Double-Sorted Portfolios: Portfolios Sorted by SRVJ Controlling for RSK, and Vice Versa$^*\>$

\small
% Table generated by Excel2LaTeX from sheet '4RSK_SRV'
\begin{table}[htbp!]
	\centering
	%\caption*{\normalfont Panel A: Sorted by SRVJ Controlling for RSK}
	\resizebox{1.04\textwidth}{!}{
		\begin{tabular}{lrrrrrrrrrrrrr}
			\multicolumn{14}{l}{Panel A: Stocks Sorted by SRVJ Controlling for RSK}\\
			&      &      &      &      &      & 
			&       &     &      &      &      &      & 
			\\
			&    & \multicolumn{4}{c}{Equal-Weighted Returns and Alphas }     &       &       &  & \multicolumn{4}{c}{Value-Weighted Returns and Alphas}         \\
			\midrule
			&    & \multicolumn{4}{c}{RSK Quintile }     &       &       &  & \multicolumn{4}{c}{RSK Quintile}         \\
			SRVJ Quintile & 1(Low) & 2     & 3     & 4     & 5(High) & High-Low &       & 1(Low) & 2     & 3     & 4     & 5(High) & High-Low \\
			%			&       &       & \multicolumn{1}{l}{Equal } & \multicolumn{1}{l}{Weighted} &       &       &       &       &       & \multicolumn{1}{l}{Value} & \multicolumn{1}{l}{Weighted} &       &  \\
			%			\midrule
			%			&       &       & \multicolumn{1}{l}{RSK} & \multicolumn{1}{l}{Quintiles} &       &       &       &       &       & \multicolumn{1}{l}{RSK} & \multicolumn{1}{l}{Quintiles} &       &  \\
			%			& \multicolumn{1}{c}{1(Low)} & \multicolumn{1}{c}{2} & \multicolumn{1}{c}{3} & \multicolumn{1}{c}{4} & \multicolumn{1}{c}{5(High)} & \multicolumn{1}{c}{Average} &       & \multicolumn{1}{c}{1(Low)} & \multicolumn{1}{c}{2} & \multicolumn{1}{c}{3} & \multicolumn{1}{c}{4} & \multicolumn{1}{c}{5(High)} & \multicolumn{1}{c}{Average} \\
			%			I: Point Estimates &       &       &       &       &       &       &       &       &       &       &       &       &  \\
			\multicolumn{14}{l}{I: Mean Return and Alpha   }                         \\
			1(Low) & 57.27  & 52.02  & 40.92  & 29.56  & 21.37  & 40.23  &       & 43.54  & 38.54  & 32.10  & 24.56  & 16.21  & 30.99  \\
			2     & 53.66  & 42.56  & 34.20  & 22.34  & 17.01  & 33.95  &       & 42.31  & 36.16  & 22.57  & 15.80  & 14.29  & 26.23  \\
			3     & 50.01  & 40.42  & 22.39  & 18.62  & 12.78  & 28.84  &       & 35.49  & 26.22  & 18.75  & 14.76  & 11.91  & 21.42  \\
			4     & 40.11  & 28.01  & 21.02  & 13.05  & 14.65  & 23.37  &       & 28.23  & 21.55  & 15.70  & 9.88  & 12.15  & 17.50  \\
			5(High) & 30.50  & 21.29  & 15.66  & 11.62  & 12.26  & 18.27  &       & 17.97  & 21.01  & 12.61  & 7.63  & 13.02  & 14.45  \\
			High-Low & -26.76  & -30.73  & -25.27  & -17.94  & -9.11  & -21.96  &       & -25.57  & -17.53  & -19.48  & -16.93  & -3.19  & -16.54  \\
			Alpha & -30.82  & -31.34  & -25.06  & -16.59  & -4.94  & -21.75  &       & -26.35  & -17.93  & -19.92  & -15.88  & -2.91  & -16.60  \\
			&       &       &       &       &       &       &       &       &       &       &       &       &  \\
			\multicolumn{5}{l}{II: t-Statistics} &       &       &       &       &       &       &       &         \\
			1(Low) & 5.60  & 4.64  & 3.85  & 2.91  & 2.33  & 4.05  &       & 5.86  & 4.70  & 4.24  & 3.25  & 2.36  & 4.59  \\
			2     & 5.02  & 4.03  & 3.24  & 2.31  & 1.91  & 3.47  &       & 5.26  & 4.72  & 3.14  & 2.23  & 2.11  & 3.96  \\
			3     & 4.65  & 3.85  & 2.24  & 2.07  & 1.49  & 3.05  &       & 4.59  & 3.69  & 2.63  & 2.20  & 1.85  & 3.38  \\
			4     & 3.87  & 2.87  & 2.17  & 1.45  & 1.75  & 2.56  &       & 3.69  & 2.97  & 2.26  & 1.43  & 1.84  & 2.74  \\
			5(High) & 3.23  & 2.34  & 1.78  & 1.35  & 1.42  & 2.14  &       & 2.51  & 3.17  & 1.82  & 1.11  & 1.94  & 2.35  \\
			High-Low & -4.91  & -6.21  & -5.21  & -4.10  & -1.90  & -6.66  &       & -5.03  & -3.27  & -3.60  & -3.08  & -0.57  & -5.28  \\
			Alpha & -5.56  & -6.63  & -5.38  & -4.04  & -1.02  & -7.02  &       & -5.25  & -3.44  & -3.69  & -2.87  & -0.53  & -5.50  \\		
			\bottomrule
	\end{tabular}}
\end{table}
\begin{table}[htbp!]
	\centering
	%\caption*{Panel B: Sorted by RSK Controlling for SRVJ}
	\resizebox{1.04\textwidth}{!}{
		\begin{threeparttable}
			\begin{tabular}{lrrrrrrrrrrrrr}		
				\multicolumn{14}{l}{Panel B: Stocks Sorted by RSK Controlling for SRVJ}\\
				&      &      &      &      &      & 
				&       &     &      &      &      &      & 
				\\
				&    & \multicolumn{4}{c}{Equal-Weighted Returns and Alphas }     &       &       &  & \multicolumn{4}{c}{Value-Weighted Returns and Alphas}         \\
				\midrule
				&    & \multicolumn{4}{c}{SRVJ Quintile }     &       &       &  & \multicolumn{4}{c}{SRVJ Quintile}         \\
				RSK Quintile & 1(Low) & 2     & 3     & 4     & 5(High) & High-Low &       & 1(Low) & 2     & 3     & 4     & 5(High) & High-Low \\
				\multicolumn{14}{l}{I: Mean Return and Alpha   }                         \\
				1(Low) & 50.58  & 28.87  & 22.47  & 16.37  & 12.05  & 26.07  &       & 37.04  & 18.91  & 21.03  & 12.37  & 3.24  & 18.52  \\
				2     & 47.66  & 38.03  & 23.14  & 14.97  & 13.62  & 27.49  &       & 36.23  & 33.70  & 15.02  & 8.29  & 10.54  & 20.75  \\
				3     & 49.84  & 35.51  & 26.73  & 17.64  & 13.40  & 28.63  &       & 31.20  & 23.04  & 20.48  & 14.76  & 15.54  & 21.01  \\
				4     & 52.65  & 42.37  & 27.90  & 18.42  & 16.16  & 31.50  &       & 35.04  & 29.47  & 20.58  & 16.91  & 10.39  & 22.48  \\
				5(High) & 49.83  & 44.70  & 26.61  & 20.27  & 13.92  & 31.06  &       & 34.59  & 33.07  & 24.21  & 19.35  & 15.58  & 25.36  \\
				High-Low & -0.75  & 15.83  & 4.14  & 3.89  & 1.88  & 5.00  &       & -2.45  & 14.16  & 3.18  & 6.98  & 12.34  & 6.84  \\
				Alpha & -4.31  & 16.10  & 3.60  & 3.17  & 4.29  & 4.57  &       & -2.88  & 15.45  & 3.60  & 6.26  & 11.07  & 6.70  \\
				&       &       &       &       &       &       &       &       &       &       &       &       &  \\
				\multicolumn{5}{l}{II: t-Statistics} &       &       &       &       &       &       &       &         \\
				1(Low) & 5.15  & 3.08  & 2.43  & 1.89  & 1.41  & 2.98  &       & 5.01  & 2.67  & 3.16  & 1.77  & 0.47  & 2.97  \\
				2     & 4.42  & 3.74  & 2.37  & 1.66  & 1.55  & 2.93  &       & 4.71  & 4.55  & 2.08  & 1.22  & 1.60  & 3.22  \\
				3     & 4.79  & 3.44  & 2.64  & 1.91  & 1.54  & 3.03  &       & 3.95  & 3.23  & 2.97  & 2.22  & 2.39  & 3.30  \\
				4     & 4.81  & 3.96  & 2.76  & 1.99  & 1.93  & 3.30  &       & 4.51  & 4.01  & 2.84  & 2.43  & 1.59  & 3.47  \\
				5(High) & 4.49  & 4.17  & 2.65  & 2.18  & 1.62  & 3.26  &       & 4.19  & 4.17  & 3.27  & 2.77  & 2.32  & 3.82  \\
				High-Low & -0.16  & 3.90  & 1.12  & 1.10  & 0.45  & 2.37  &       & -0.45  & 2.67  & 0.64  & 1.44  & 2.66  & 2.67  \\
				Alpha & -0.93  & 4.17  & 0.97  & 0.93  & 0.98  & 2.29  &       & -0.55  & 2.93  & 0.75  & 1.31  & 2.46  & 2.68  \\
				\bottomrule
			\end{tabular}
			\begin{tablenotes}
				
				\item{}
				\item { $^*\>$ Notes: See notes to Table 5. This table presents average returns (called ``Mean Return'') and risk-adjusted alphas (called ``Alpha'') for portfolios sorted by SRVJ controlling for RSK, and vice versa. The sample includes NYSE, NASDAQ and AMEX listed stocks for the period January 1993 to December 2016. At the end of each Tuesday, all the stocks in the sample are sorted into quintile portfolios based on ascending value of RSK (SRVJ), and then within each quintile portfolio, stocks are further sorted by the value of SRVJ (RSK), resulting in 25 portfolios in total. Each portfolio is held for one week. The row labeled ``High-Low'' reports the average values of one-week ahead returns in Part I of Panel A (corresponding Newey-West $t$-statistics are given in Part II of the panel). The row labeled ``Alpha'' reports Fama-French-Carhart four-factor alphas in Part I of Part A (corresponding Newey-West $t$-statistics are again given in Part II of the panel) for each of the quintile portfolios, as well as for the difference between portfolio 5 and portfolio 1.}
			\end{tablenotes}
	\end{threeparttable}}
\end{table}
\newpage
\noindent Table 9: Double-Sorted Portfolios: Portfolios Sorted by SRVLJ Controlling for RSK$^*\>$ 

% Table generated by Excel2LaTeX from sheet '4RSK_L'
\begin{table}[htbp!]
	\centering
	%\caption*{Add caption}
	\resizebox{1.04\textwidth}{!}{
		\begin{threeparttable}
			\begin{tabular}{lrrrrrrrrrrrrr}
				\multicolumn{14}{l}{Panel A: Stocks Sorted by SRVLJ Controlling for RSK Based on $\alpha_n^{1}$}\\
				&      &      &      &      &      & 
				&       &     &      &      &      &      & 
				\\
				&    & \multicolumn{4}{c}{Equal-Weighted Returns and Alphas }     &       &       &  & \multicolumn{4}{c}{Value-Weighted Returns and Alphas}         \\
				\midrule
				&    & \multicolumn{4}{c}{RSK Quintile }     &       &       &  & \multicolumn{4}{c}{RSK Quintile}         \\
				SRVLJ Quintile & 1(Low) & 2     & 3     & 4     & 5(High) & High-Low &       & 1(Low) & 2     & 3     & 4     & 5(High) & High-Low \\
				
				\multicolumn{14}{l}{I: Mean Return and Alpha   }                         \\
				1(Low) & 50.39  & 32.00  & 22.73  & 14.14  & 10.29  & 25.91  &       & 28.70  & 23.41  & 14.12  & 14.21  & 12.17  & 18.52  \\
				2     & 43.36  & 35.12  & 22.04  & 18.04  & 18.49  & 27.90  &       & 29.60  & 24.25  & 18.83  & 16.59  & 19.70  & 22.25  \\
				3     & 45.39  & 36.61  & 25.63  & 16.57  & 14.57  & 28.71  &       & 28.06  & 24.52  & 22.19  & 15.92  & 7.45  & 20.43  \\
				4     & 47.05  & 35.21  & 26.18  & 22.82  & 20.19  & 30.01  &       & 27.19  & 27.39  & 12.34  & 19.32  & 17.46  & 20.60  \\
				5(High) & 45.29  & 42.28  & 31.31  & 23.14  & 14.45  & 31.29  &       & 34.45  & 29.40  & 19.14  & 22.59  & 18.65  & 24.85  \\
				High-Low & -5.10  & 10.28  & 8.59  & 9.00  & 4.17  & 5.39  &       & 5.75  & 6.00  & 5.02  & 8.38  & 6.48  & 6.32  \\
				Alpha & -10.62  & 8.25  & 7.59  & 9.39  & 7.09  & 4.34  &       & 4.00  & 6.19  & 4.09  & 7.77  & 5.63  & 5.54  \\
				&       &       &       &       &       &       &       &       &       &       &       &       &  \\
				\multicolumn{5}{l}{II: t-Statistics} &       &       &       &       &       &       &       &         \\
				1(Low) & 4.97  & 3.07  & 2.36  & 1.59  & 1.20  & 2.83  &       & 3.70  & 3.17  & 2.06  & 2.10  & 1.88  & 2.92  \\
				2     & 4.17  & 3.42  & 2.16  & 1.87  & 2.12  & 2.93  &       & 3.92  & 3.45  & 2.53  & 2.25  & 3.00  & 3.48  \\
				3     & 4.32  & 3.55  & 2.32  & 1.72  & 1.64  & 3.03  &       & 3.63  & 3.56  & 2.65  & 2.20  & 1.14  & 3.17  \\
				4     & 4.50  & 3.26  & 2.46  & 2.48  & 2.34  & 3.19  &       & 3.80  & 3.44  & 1.63  & 2.70  & 2.61  & 3.17  \\
				5(High) & 4.60  & 4.16  & 3.22  & 2.51  & 1.65  & 3.42  &       & 4.48  & 3.97  & 2.67  & 3.28  & 2.72  & 3.86  \\
				High-Low & -1.01  & 2.72  & 2.76  & 2.68  & 0.94  & 3.03  &       & 1.09  & 1.20  & 1.10  & 1.91  & 1.33  & 2.95  \\
				Alpha & -2.05  & 2.13  & 2.43  & 2.78  & 1.51  & 2.43  &       & 0.76  & 1.20  & 0.88  & 1.81  & 1.17  & 2.55  \\
				\hline
				&       &       &       &       &       &       &       &       &       &       &       &       &  \\
				&       &       &       &       &       &       &       &       &       &       &       &       &  \\
				\multicolumn{14}{l}{Panel B: Stocks Sorted by SRVLJ Controlling for RSK Based on $\alpha_n^{2}$}\\
				&      &      &      &      &      & 
				&       &     &      &      &      &      & 
				\\
				&    & \multicolumn{4}{c}{Equal-Weighted Returns and Alphas }     &       &       &  & \multicolumn{4}{c}{Value-Weighted Returns and Alphas}         \\
				\midrule
				&    & \multicolumn{4}{c}{RSK Quintile }     &       &       &  & \multicolumn{4}{c}{RSK Quintile}         \\
				SRVLJ Quintile & 1(Low) & 2     & 3     & 4     & 5(High) & High-Low &       & 1(Low) & 2     & 3     & 4     & 5(High) & High-Low \\
				
				\multicolumn{14}{l}{I: Mean Return and Alpha   }                         \\
				1(Low) & 47.98  & 31.96  & 19.10  & 13.20  & 9.70  & 24.39  &       & 25.87  & 20.51  & 18.34  & 9.78  & 10.14  & 16.93  \\
				2     & 44.65  & 33.30  & 15.95  & 16.96  & 10.55  & 28.70  &       & 26.73  & 26.20  & 11.31  & 1.56  & 14.06  & 21.08  \\
				3     & 45.50  & 35.22  & 18.81  & 18.27  & 17.35  & 28.56  &       & 26.92  & 24.20  & 32.75  & 16.67  & 18.32  & 22.08  \\
				4     & 46.04  & 11.93  & 29.94  & 18.03  & 23.67  & 28.13  &       & 28.18  & 12.76  & 21.73  & 16.59  & 16.81  & 20.63  \\
				5(High) & 41.78  & 40.49  & 32.71  & 26.75  & 16.00  & 31.59  &       & 35.53  & 33.78  & 21.24  & 23.85  & 20.48  & 27.00  \\
				High-Low & -5.98  & 8.53  & 13.61  & 13.55  & 6.30  & 7.20  &       & 9.87  & 13.27  & 2.91  & 14.07  & 10.34  & 10.09  \\
				Alpha & -11.83  & 7.58  & 13.81  & 14.15  & 9.61  & 6.66  &       & 7.82  & 13.20  & 2.06  & 13.48  & 9.80  & 9.27  \\
				&       &       &       &       &       &       &       &       &       &       &       &       &  \\
				\multicolumn{5}{l}{II: t-Statistics} &       &       &       &       &       &       &       &         \\
				1(Low) & 4.84  & 3.17  & 2.00  & 1.50  & 1.12  & 2.69  &       & 3.40  & 2.78  & 2.65  & 1.51  & 1.54  & 2.68  \\
				2     & 4.21  & 3.28  & 1.28  & 0.60  & 1.18  & 2.99  &       & 3.59  & 3.76  & 1.29  & 0.07  & 2.08  & 3.28  \\
				3     & 4.35  & 2.90  & 0.66  & 1.62  & 1.99  & 3.02  &       & 3.72  & 2.86  & 1.25  & 1.94  & 2.83  & 3.42  \\
				4     & 4.27  & 0.47  & 2.50  & 1.90  & 2.71  & 2.98  &       & 3.71  & 0.69  & 2.29  & 2.35  & 2.53  & 3.17  \\
				5(High) & 4.01  & 3.80  & 3.26  & 2.78  & 1.83  & 3.37  &       & 4.04  & 4.01  & 2.92  & 3.28  & 2.98  & 4.01  \\
				High-Low & -1.03  & 2.04  & 4.23  & 4.14  & 1.46  & 3.71  &       & 1.56  & 2.21  & 0.64  & 3.19  & 2.07  & 4.10  \\
				Alpha & -1.98  & 1.77  & 4.17  & 4.28  & 2.16  & 3.30  &       & 1.24  & 2.04  & 0.43  & 3.07  & 1.98  & 3.58  \\
				\hline
			\end{tabular}
			\begin{tablenotes}
				
				\item{}
				\item { $^*\>$ Notes: See notes to Table 8. Portfolios sorted by SRVLJ, controlling for RSK, and using two truncation levels, $\alpha_n^{1}$ and $\alpha_n^{2}$, as discussed in the footnote to Table 2.}
			\end{tablenotes}
	\end{threeparttable}}
	
\end{table}
\newpage
\noindent Table 10: Double-Sorted Portfolios: Portfolios Sorted by SRVSJ Controlling for RSK$^*\>$

% Table generated by Excel2LaTeX from sheet '4RSK_S'
\begin{table}[htbp!]
	\centering
	%\caption{Add caption}
	\resizebox{1.04\textwidth}{!}{
		\begin{threeparttable}
			\begin{tabular}{lrrrrrrrrrrrrr}
				\multicolumn{14}{l}{Panel A: Stocks Sorted by SRVSJ Controlling for RSK Based on $\alpha_n^{1}$}\\
				&      &      &      &      &      & 
				&       &     &      &      &      &      & 
				\\
				&    & \multicolumn{4}{c}{Equal-Weighted Returns and Alphas }     &       &       &  & \multicolumn{4}{c}{Value-Weighted Returns and Alphas}         \\
				\midrule
				&    & \multicolumn{4}{c}{RSK Quintile }     &       &       &  & \multicolumn{4}{c}{RSK Quintile}         \\
				SRVSJ Quintile & 1(Low) & 2     & 3     & 4     & 5(High) & High-Low &       & 1(Low) & 2     & 3     & 4     & 5(High) & High-Low \\
				
				\multicolumn{14}{l}{I: Mean Return and Alpha   }                         \\
				1(Low) & 59.75  & 51.04  & 39.73  & 29.67  & 25.10  & 41.06  &       & 43.04  & 38.99  & 26.51  & 30.45  & 19.77  & 31.75  \\
				2     & 50.43  & 48.41  & 36.47  & 29.11  & 21.68  & 37.89  &       & 27.77  & 32.21  & 25.89  & 15.84  & 21.93  & 24.85  \\
				3     & 48.64  & 37.14  & 26.87  & 12.01  & 9.97  & 26.64  &       & 25.28  & 24.11  & 17.37  & 16.81  & 15.44  & 19.62  \\
				4     & 44.87  & 24.89  & 16.07  & 15.31  & 12.08  & 22.29  &       & 30.33  & 20.93  & 15.52  & 15.32  & 13.86  & 19.42  \\
				5(High) & 29.80  & 22.42  & 14.47  & 8.85  & 9.13  & 16.93  &       & 18.31  & 19.40  & 10.19  & 6.18  & 8.82  & 12.58  \\
				High-Low & -29.95  & -28.62  & -25.27  & -20.82  & -15.96  & -24.13  &       & -24.74  & -19.59  & -16.32  & -24.28  & -10.95  & -19.17  \\
				Alpha & -28.91  & -28.24  & -24.53  & -20.14  & -15.10  & -23.39  &       & -23.19  & -19.67  & -15.95  & -23.30  & -9.36  & -18.29  \\
				&       &       &       &       &       &       &       &       &       &       &       &       &  \\
				\multicolumn{5}{l}{II: t-Statistics} &       &       &       &       &       &       &       &         \\
				1(Low) & 5.45  & 4.73  & 3.74  & 3.03  & 2.75  & 4.12  &       & 5.36  & 5.11  & 3.59  & 4.16  & 2.85  & 4.67  \\
				2     & 4.69  & 4.53  & 3.52  & 2.95  & 2.36  & 3.86  &       & 3.74  & 4.10  & 3.61  & 2.14  & 3.03  & 3.67  \\
				3     & 4.84  & 3.60  & 2.75  & 1.28  & 1.13  & 2.87  &       & 3.31  & 3.41  & 2.48  & 2.38  & 2.11  & 3.07  \\
				4     & 4.22  & 2.49  & 1.66  & 1.73  & 1.41  & 2.43  &       & 3.76  & 3.15  & 2.19  & 2.19  & 2.15  & 3.08  \\
				5(High) & 3.16  & 2.39  & 1.59  & 1.03  & 1.11  & 1.96  &       & 2.60  & 2.77  & 1.46  & 0.92  & 1.35  & 2.02  \\
				High-Low & -6.63  & -6.01  & -5.86  & -5.03  & -4.09  & -7.62  &       & -4.65  & -3.70  & -3.23  & -4.90  & -2.28  & -6.17  \\
				Alpha & -6.48  & -5.90  & -5.80  & -5.10  & -4.00  & -7.81  &       & -4.18  & -3.74  & -3.14  & -4.83  & -1.97  & -5.95  \\
				\hline
				&       &       &       &       &       &       &       &       &       &       &       &       &  \\
				&       &       &       &       &       &       &       &       &       &       &       &       &  \\
				\multicolumn{14}{l}{Panel B: Stocks Sorted by SRVSJ Controlling for RSK Based on $\alpha_n^{2}$}\\
				&      &      &      &      &      & 
				&       &     &      &      &      &      & 
				\\
				&    & \multicolumn{4}{c}{Equal-Weighted Returns and Alphas }     &       &       &  & \multicolumn{4}{c}{Value-Weighted Returns and Alphas}         \\
				\midrule
				&    & \multicolumn{4}{c}{RSK Quintile }     &       &       &  & \multicolumn{4}{c}{RSK Quintile}         \\
				SRVSJ Quintile & 1(Low) & 2     & 3     & 4     & 5(High) & High-Low &       & 1(Low) & 2     & 3     & 4     & 5(High) & High-Low \\
				
				\multicolumn{14}{l}{I: Mean Return and Alpha   }                         \\
				1(Low) & 58.26  & 47.14  & 41.93  & 31.99  & 28.21  & 41.50  &       & 44.52  & 40.24  & 31.33  & 31.42  & 21.82  & 33.87  \\
				2     & 53.36  & 47.25  & 36.59  & 23.90  & 16.43  & 35.51  &       & 32.06  & 34.60  & 24.48  & 18.84  & 23.93  & 26.78  \\
				3     & 52.90  & 37.07  & 24.46  & 18.12  & 13.08  & 29.13  &       & 26.72  & 22.89  & 15.74  & 11.77  & 16.54  & 18.73  \\
				4     & 36.60  & 32.56  & 16.57  & 12.67  & 12.70  & 22.21  &       & 17.24  & 21.99  & 15.18  & 12.71  & 10.73  & 15.56  \\
				5(High) & 28.56  & 20.21  & 14.55  & 8.35  & 6.35  & 15.60  &       & 17.51  & 19.10  & 14.04  & 8.01  & 6.52  & 13.04  \\
				High-Low & -29.70  & -26.93  & -27.37  & -23.64  & -21.86  & -25.90  &       & -27.01  & -21.14  & -17.29  & -23.42  & -15.30  & -20.83  \\
				Alpha & -27.69  & -26.57  & -27.26  & -23.33  & -22.66  & -25.50  &       & -25.15  & -22.35  & -17.39  & -21.97  & -14.76  & -20.32  \\
				&       &       &       &       &       &       &       &       &       &       &       &       &  \\
				\multicolumn{5}{l}{II: t-Statistics} &       &       &       &       &       &       &       &         \\
				1(Low) & 5.46  & 4.40  & 3.95  & 3.22  & 3.00  & 4.18  &       & 5.31  & 5.12  & 4.25  & 4.36  & 3.20  & 4.95  \\
				2     & 5.00  & 4.37  & 3.46  & 2.44  & 1.89  & 3.63  &       & 4.21  & 4.77  & 3.31  & 2.61  & 3.43  & 4.06  \\
				3     & 4.98  & 3.62  & 2.49  & 1.95  & 1.50  & 3.08  &       & 3.57  & 3.17  & 2.22  & 1.64  & 2.46  & 2.89  \\
				4     & 3.69  & 3.24  & 1.73  & 1.42  & 1.51  & 2.46  &       & 2.41  & 3.17  & 2.24  & 1.89  & 1.62  & 2.52  \\
				5(High) & 3.03  & 2.18  & 1.61  & 0.98  & 0.77  & 1.81  &       & 2.39  & 2.81  & 2.03  & 1.19  & 0.98  & 2.10  \\
				High-Low & -6.40  & -5.91  & -6.25  & -5.60  & -5.03  & -8.37  &       & -4.58  & -3.94  & -3.49  & -4.68  & -3.01  & -6.36  \\
				Alpha & -6.10  & -6.00  & -6.43  & -5.74  & -5.27  & -8.80  &       & -4.25  & -4.07  & -3.42  & -4.31  & -2.90  & -6.14  \\
				\hline
			\end{tabular}
			\begin{tablenotes}
				
				\item{}
				\item { $^*\>$ Notes: See notes to Tables 8. Portfolios sorted by SRVSJ, controlling for RSK, and using two truncation levels, $\alpha_n^{1}$ and $\alpha_n^{2}$, as discussed in the footnote to Table 2. }
			\end{tablenotes}
	\end{threeparttable}} 
\end{table}
\newpage
\noindent Table 11: Double-Sorted Portfolios: Portfolios Sorted by RSK Controlling for SRVLJ$^*\>$

% Table generated by Excel2LaTeX from sheet '4L_RSK'
\begin{table}[htbp!]
	\centering
	%\caption{Add caption}
	\resizebox{1.04\textwidth}{!}{
		\begin{threeparttable}
			\begin{tabular}{lrrrrrrrrrrrrr}
				\multicolumn{14}{l}{Panel A: Stocks Sorted by RSK Controlling for SRVLJ Based on $\alpha_n^{1}$}\\
				&      &      &      &      &      & 
				&       &     &      &      &      &      & 
				\\
				&    & \multicolumn{4}{c}{Equal-Weighted Returns and Alphas }     &       &       &  & \multicolumn{4}{c}{Value-Weighted Returns and Alphas}         \\
				\midrule
				&    & \multicolumn{4}{c}{SRVLJ Quintile }     &       &       &  & \multicolumn{4}{c}{SRVLJ Quintile}         \\
				RSK Quintile & 1(Low) & 2     & 3     & 4     & 5(High) & High-Low &       & 1(Low) & 2     & 3     & 4     & 5(High) & High-Low \\
				
				\multicolumn{14}{l}{I: Mean Return and Alpha   }                         \\
				1(Low) & 50.12  & 44.15  & 43.09  & 33.01  & 25.77  & 39.01  &       & 36.54  & 30.35  & 31.78  & 23.31  & 20.48  & 28.39  \\
				2     & 48.30  & 40.78  & 34.78  & 27.78  & 18.01  & 33.91  &       & 33.02  & 26.05  & 27.38  & 17.08  & 15.35  & 23.81  \\
				3     & 44.13  & 30.72  & 27.74  & 21.97  & 18.71  & 28.55  &       & 22.19  & 20.44  & 19.85  & 20.00  & 17.86  & 20.00  \\
				4     & 43.77  & 25.68  & 22.05  & 15.96  & 17.49  & 24.92  &       & 25.66  & 19.96  & 19.40  & 13.22  & 10.92  & 17.78  \\
				5(High) & 31.95  & 15.30  & 15.07  & 11.65  & 14.44  & 17.55  &       & 25.07  & 14.49  & 10.70  & 16.66  & 15.25  & 16.56  \\
				High-Low & -18.17  & -28.85  & -27.62  & -21.36  & -11.33  & -21.45  &       & -11.47  & -15.86  & -20.68  & -6.64  & -5.23  & -11.82  \\
				Alpha & -20.67  & -27.06  & -27.43  & -20.20  & -8.19  & -20.69  &       & -12.82  & -13.04  & -19.71  & -6.84  & -5.29  & -11.37  \\
				&       &       &       &       &       &       &       &       &       &       &       &       &  \\
				\multicolumn{5}{l}{II: t-Statistics} &       &       &       &       &       &       &       &         \\
				1(Low) & 5.12  & 4.40  & 4.19  & 3.24  & 2.82  & 4.12  &       & 4.98  & 4.05  & 3.93  & 3.15  & 3.00  & 4.24  \\
				2     & 4.50  & 3.90  & 3.21  & 2.91  & 2.00  & 3.48  &       & 4.29  & 3.57  & 3.57  & 2.44  & 2.29  & 3.62  \\
				3     & 4.30  & 3.09  & 2.74  & 2.29  & 2.11  & 3.03  &       & 2.90  & 3.11  & 2.69  & 2.88  & 2.75  & 3.18  \\
				4     & 4.13  & 2.58  & 2.20  & 1.72  & 2.07  & 2.68  &       & 3.52  & 2.91  & 2.50  & 1.83  & 1.65  & 2.78  \\
				5(High) & 3.11  & 1.62  & 1.63  & 1.34  & 1.68  & 1.99  &       & 3.48  & 2.10  & 1.53  & 2.65  & 2.27  & 2.72  \\
				High-Low & -4.07  & -7.75  & -5.43  & -5.17  & -2.55  & -8.70  &       & -2.13  & -3.41  & -3.64  & -1.32  & -1.03  & -4.07  \\
				Alpha & -4.61  & -7.19  & -5.51  & -5.25  & -1.80  & -8.47  &       & -2.33  & -2.82  & -3.46  & -1.37  & -1.06  & -4.02  \\
				\hline
				&       &       &       &       &       &       &       &       &       &       &       &       &  \\
				&       &       &       &       &       &       &       &       &       &       &       &       &  \\
				\multicolumn{14}{l}{Panel B: Stocks Sorted by RSK Controlling for SRVLJ Based on $\alpha_n^{2}$}\\
				&      &      &      &      &      & 
				&       &     &      &      &      &      & 
				\\
				&    & \multicolumn{4}{c}{Equal-Weighted Returns and Alphas }     &       &       &  & \multicolumn{4}{c}{Value-Weighted Returns and Alphas}         \\
				\midrule
				&    & \multicolumn{4}{c}{SRVLJ Quintile }     &       &       &  & \multicolumn{4}{c}{SRVLJ Quintile}         \\
				RSK Quintile & 1(Low) & 2     & 3     & 4     & 5(High) & High-Low &       & 1(Low) & 2     & 3     & 4     & 5(High) & High-Low \\
				
				\multicolumn{14}{l}{I: Mean Return and Alpha   }                         \\
				1(Low) & 50.46  & 45.20  & 18.12  & 40.17  & 33.89  & 41.63  &       & 35.68  & 33.83  & 10.33  & 32.68  & 32.08  & 32.67  \\
				2     & 47.58  & 36.38  & 15.46  & 32.46  & 21.29  & 34.31  &       & 29.74  & 24.35  & 10.92  & 17.04  & 18.06  & 22.37  \\
				3     & 41.05  & 27.92  & -3.18  & 19.38  & 17.93  & 25.36  &       & 24.89  & 20.35  & -12.74  & 18.64  & 17.44  & 18.55  \\
				4     & 37.64  & 20.44  & 14.57  & 11.83  & 18.64  & 22.56  &       & 19.95  & 18.54  & 9.44  & 14.42  & 15.56  & 17.24  \\
				5(High) & 21.61  & 8.27  & -20.42  & 13.08  & 15.39  & 13.26  &       & 14.99  & 8.83  & -16.92  & 14.41  & 16.61  & 12.17  \\
				High-Low & -28.85  & -36.94  & -46.62  & -25.97  & -18.50  & -28.76  &       & -20.69  & -25.00  & -35.38  & -17.14  & -15.47  & -20.91  \\
				Alpha & -31.44  & -36.48  & -47.17  & -26.35  & -15.40  & -28.41  &       & -20.70  & -23.15  & -31.93  & -15.11  & -16.20  & -19.73  \\
				&       &       &       &       &       &       &       &       &       &       &       &       &  \\
				\multicolumn{5}{l}{II: t-Statistics} &       &       &       &       &       &       &       &         \\
				1(Low) & 5.13  & 4.38  & 0.80  & 3.86  & 3.48  & 4.29  &       & 4.83  & 4.56  & 0.57  & 4.13  & 4.55  & 4.79  \\
				2     & 4.47  & 3.57  & 0.71  & 3.15  & 2.30  & 3.48  &       & 3.91  & 3.49  & 0.59  & 2.28  & 2.63  & 3.39  \\
				3     & 3.92  & 2.77  & -0.15  & 1.92  & 2.03  & 2.67  &       & 3.17  & 3.07  & -0.72  & 2.56  & 2.60  & 2.91  \\
				4     & 3.61  & 2.14  & 0.72  & 1.22  & 2.18  & 2.43  &       & 2.77  & 2.82  & 0.58  & 2.01  & 2.39  & 2.75  \\
				5(High) & 2.30  & 0.91  & -1.06  & 1.49  & 1.80  & 1.55  &       & 2.14  & 1.28  & -1.00  & 2.14  & 2.47  & 1.96  \\
				High-Low & -6.82  & -8.96  & -2.43  & -4.97  & -3.81  & -8.35  &       & -3.96  & -5.60  & -1.82  & -2.85  & -3.04  & -5.91  \\
				Alpha & -7.11  & -9.12  & -2.68  & -5.26  & -3.19  & -8.69  &       & -3.89  & -5.25  & -1.81  & -2.46  & -3.30  & -5.67  \\
				\hline
			\end{tabular}
			\begin{tablenotes}
				
				\item{}
				\item { $^*\>$ Notes: See notes to Table 8. Portfolios sorted by RSK, controlling for SRVLJ, and using two truncation levels, $\alpha_n^{1}$ and $\alpha_n^{2}$, as discussed in the footnote to Table 2. }
			\end{tablenotes}
	\end{threeparttable}}
\end{table}
\newpage
\noindent Table 12: Double-Sorted Portfolios: Portfolios Sorted by RSK Controlling for SRVSJ$^*\>$

% Table generated by Excel2LaTeX from sheet '4S_RSK'
\begin{table}[htbp!]
	\centering
	%\caption{Add caption}
	\resizebox{1.04\textwidth}{!}{
		\begin{threeparttable}
			\begin{tabular}{lrrrrrrrrrrrrrr}
				\multicolumn{14}{l}{Panel A: Stocks Sorted by RSK Controlling for SRVSJ Based on $\alpha_n^{1}$}\\
				&      &      &      &      &      & 
				&       &     &      &      &      &      & 
				\\
				&    & \multicolumn{4}{c}{Equal-Weighted Returns and Alphas }     &       &       &  & & \multicolumn{4}{c}{Value-Weighted Returns and Alphas}         \\
				\midrule
				&    & \multicolumn{4}{c}{SRVSJ Quintile }     &       &     &  &  & \multicolumn{4}{c}{SRVSJ Quintile}         \\
				RSK Quintile & 1(Low) & 2     & 3     & 4     & 5(High) & High-Low &    &   & 1(Low) & 2     & 3     & 4     & 5(High) & High-Low \\
				
				\multicolumn{14}{l}{I: Mean Return and Alpha   }                         \\
				1(Low) & 59.07  & 56.68  & 37.38  & 31.20  & 21.87  & 41.24  &       &       & 43.22  & 33.95  & 17.37  & 18.93  & 21.30  & 26.95  \\
				2     & 52.62  & 43.64  & 31.16  & 21.45  & 14.58  & 32.69  &       &       & 36.64  & 25.02  & 24.60  & 15.72  & 10.21  & 22.44  \\
				3     & 48.55  & 38.14  & 24.53  & 17.57  & 12.09  & 28.18  &       &       & 37.59  & 26.22  & 11.11  & 17.89  & 8.72  & 20.31  \\
				4     & 42.20  & 34.54  & 18.68  & 15.88  & 6.90  & 23.64  &       &       & 30.33  & 21.77  & 19.64  & 19.53  & 6.95  & 19.64  \\
				5(High) & 25.66  & 29.69  & 10.59  & 9.20  & 11.37  & 17.30  &       &       & 27.76  & 22.10  & 20.20  & 14.48  & 12.12  & 19.33  \\
				High-Low & -33.41  & -27.00  & -26.79  & -21.99  & -10.50  & -23.94  &       &       & -15.47  & -11.85  & 2.83  & -4.45  & -9.18  & -7.62  \\
				Alpha & -33.36  & -27.16  & -28.09  & -21.48  & -10.30  & -24.08  &       &       & -13.89  & -11.06  & 4.28  & -3.69  & -8.44  & -6.56  \\
				&       &       &       &       &       &       &       &       &       &       &       &       &       &  \\
				\multicolumn{5}{l}{II: t-Statistics} &       &       &       &       &       &       &       &         \\
				1(Low) & 5.54  & 5.20  & 3.80  & 3.09  & 2.42  & 4.24  &       &       & 5.41  & 4.65  & 2.26  & 2.44  & 3.22  & 4.00  \\
				2     & 4.83  & 4.11  & 3.08  & 2.18  & 1.64  & 3.36  &       &       & 4.73  & 3.36  & 3.49  & 2.20  & 1.58  & 3.46  \\
				3     & 4.50  & 3.67  & 2.49  & 1.82  & 1.39  & 2.95  &       &       & 4.87  & 3.72  & 1.56  & 2.55  & 1.27  & 3.11  \\
				4     & 4.04  & 3.41  & 1.94  & 1.73  & 0.79  & 2.54  &       &       & 4.07  & 2.90  & 2.79  & 2.81  & 1.04  & 3.05  \\
				5(High) & 2.72  & 3.17  & 1.24  & 1.07  & 1.40  & 2.05  &       &       & 4.04  & 3.12  & 2.81  & 2.17  & 1.87  & 3.15  \\
				High-Low & -7.52  & -5.47  & -5.83  & -4.99  & -2.85  & -7.65  &       &       & -3.04  & -2.47  & 0.50  & -0.79  & -2.04  & -2.58  \\
				Alpha & -7.40  & -5.45  & -5.98  & -5.11  & -2.81  & -7.89  &       &       & -2.66  & -2.20  & 0.74  & -0.65  & -1.84  & -2.21  \\
				\hline
				&       &       &       &       &       &       &       &       &       &       &       &       &       &  \\
				&       &       &       &       &       &       &       &       &       &       &       &       &       &  \\
				\multicolumn{14}{l}{Panel B: Stocks Sorted by RSK Controlling for SRVLJ Based on $\alpha_n^{2}$}\\
				&      &      &      &      &      & 
				&       &     &      &      &      &      & 
				\\
				&    & \multicolumn{4}{c}{Equal-Weighted Returns and Alphas }     &       &       &  & & \multicolumn{4}{c}{Value-Weighted Returns and Alphas}         \\
				\midrule
				&    & \multicolumn{4}{c}{SRVSJ Quintile }     &       &     &  &  & \multicolumn{4}{c}{SRVSJ Quintile}         \\
				RSK Quintile & 1(Low) & 2     & 3     & 4     & 5(High) & High-Low &    &   & 1(Low) & 2     & 3     & 4     & 5(High) & High-Low \\
				
				\multicolumn{14}{l}{I: Mean Return and Alpha   }                         \\
				1(Low) & 57.52  & 53.46  & 37.88  & 24.74  & 17.42  & 38.20  &       &       & 44.22  & 25.05  & 19.15  & 15.91  & 12.67  & 23.40  \\
				2     & 52.38  & 40.35  & 31.08  & 15.93  & 11.31  & 30.21  &       &       & 37.21  & 25.75  & 15.93  & 21.47  & 8.88  & 21.85  \\
				3     & 49.95  & 41.34  & 24.82  & 18.26  & 10.34  & 28.94  &       &       & 34.56  & 24.25  & 19.76  & 8.99  & 7.57  & 19.03  \\
				4     & 42.26  & 40.33  & 26.64  & 16.46  & 9.28  & 26.99  &       &       & 39.75  & 24.96  & 18.28  & 16.23  & 10.77  & 22.00  \\
				5(High) & 31.84  & 32.74  & 16.24  & 11.48  & 8.65  & 20.19  &       &       & 32.87  & 30.53  & 19.63  & 16.15  & 9.24  & 21.68  \\
				High-Low & -25.68  & -20.73  & -21.64  & -13.26  & -8.76  & -18.02  &       &       & -11.35  & 5.49  & 0.48  & 0.24  & -3.43  & -1.72  \\
				Alpha & -25.05  & -20.78  & -22.39  & -13.87  & -8.64  & -18.15  &       &       & -10.08  & 7.20  & 0.79  & -0.58  & -2.91  & -1.12  \\
				&       &       &       &       &       &       &       &       &       &       &       &       &       &  \\
				\multicolumn{5}{l}{II: t-Statistics} &       &       &       &       &       &       &       &         \\
				1(Low) & 5.47  & 5.14  & 3.71  & 2.60  & 1.95  & 4.01  &       &       & 5.77  & 3.30  & 2.56  & 2.11  & 1.79  & 3.51  \\
				2     & 4.77  & 3.85  & 3.06  & 1.68  & 1.29  & 3.14  &       &       & 4.70  & 3.40  & 2.18  & 3.31  & 1.33  & 3.36  \\
				3     & 4.64  & 3.76  & 2.56  & 1.95  & 1.19  & 3.02  &       &       & 4.52  & 3.31  & 2.89  & 1.29  & 1.13  & 2.95  \\
				4     & 3.96  & 3.95  & 2.72  & 1.78  & 1.08  & 2.87  &       &       & 5.34  & 3.49  & 2.54  & 2.30  & 1.56  & 3.39  \\
				5(High) & 3.38  & 3.33  & 1.86  & 1.32  & 1.07  & 2.35  &       &       & 4.31  & 4.30  & 2.65  & 2.45  & 1.43  & 3.46  \\
				High-Low & -6.19  & -4.47  & -4.77  & -3.59  & -2.41  & -6.93  &       &       & -2.12  & 1.07  & 0.09  & 0.04  & -0.70  & -0.60  \\
				Alpha & -5.95  & -4.25  & -5.01  & -3.83  & -2.46  & -7.05  &       &       & -1.93  & 1.38  & 0.15  & -0.11  & -0.58  & -0.40  \\
				\hline
			\end{tabular}
			\begin{tablenotes}
				
				\item{}
				\item { $^*\>$ Notes: See notes to Table 8. Portfolios sorted by RSK controlling for SRVSJ, and using two truncation levels, $\alpha_n^{1}$ and $\alpha_n^{2}$, as discussed in the footnote to Table 2. }
			\end{tablenotes}
	\end{threeparttable}}
\end{table}
\newpage
\noindent Table 13: Double-Sorted Portfolios and Jumps: Portfolios Sorted by Stock- and Industry-Level SRVJ Independently$^*\>$

% Table generated by Excel2LaTeX from sheet '4srvj'
\begin{table}[htbp!]
	\centering
	%\caption{Add caption}
	\resizebox{1.04\textwidth}{!}{
		\begin{threeparttable}
			\begin{tabular}{llllllrrrrrrrrrrr}
				%\begin{tabular}{lrrrrrrrrrrrrrrrr}
				&    & \multicolumn{4}{c}{Equal-Weighted Returns and Alphas }     &       &       &  & & & \multicolumn{4}{c}{Value-Weighted Returns and Alphas}   &     \\
				\midrule
				&    & \multicolumn{4}{c}{Industry-Level Quintile }     &       &   &  &  &  & \multicolumn{4}{c}{Industry-Level Quintile}   &      \\
				
				Stock-Level Quintile & \multicolumn{1}{c}{1(Low)} & \multicolumn{1}{c}{2} & \multicolumn{1}{c}{3} & \multicolumn{1}{c}{4} & \multicolumn{1}{c}{5(High)} & \multicolumn{1}{c}{High-Low} & \multicolumn{1}{c}{Alpha} &       &       & \multicolumn{1}{c}{1(Low)} & \multicolumn{1}{c}{2} & \multicolumn{1}{c}{3} & \multicolumn{1}{c}{4} & \multicolumn{1}{c}{5(High)} & \multicolumn{1}{c}{High-Low} & \multicolumn{1}{c}{Alpha} \\
				\multicolumn{17}{l}{A: Mean Return and Alpha   }                         \\
				1(Low) & \multicolumn{1}{r}{37.76 } & \multicolumn{1}{r}{47.60 } & \multicolumn{1}{r}{46.85 } & \multicolumn{1}{r}{58.29 } & \multicolumn{1}{r}{73.53 } & 35.77  & 37.77  &       &       & 35.33  & 35.11  & 32.24  & 35.57  & 49.34  & 14.02  & 13.64  \\
				2     & \multicolumn{1}{r}{28.58 } & \multicolumn{1}{r}{32.06 } & \multicolumn{1}{r}{37.81 } & \multicolumn{1}{r}{46.77 } & \multicolumn{1}{r}{56.48 } & 27.90  & 29.31  &       &       & 27.97  & 26.10  & 23.72  & 34.36  & 37.23  & 9.27  & 8.74  \\
				3     & \multicolumn{1}{r}{13.10 } & \multicolumn{1}{r}{21.00 } & \multicolumn{1}{r}{23.23 } & \multicolumn{1}{r}{30.71 } & \multicolumn{1}{r}{41.95 } & 28.85  & 29.93  &       &       & 16.13  & 24.76  & 17.18  & 20.60  & 29.80  & 13.67  & 15.27  \\
				4     & \multicolumn{1}{r}{6.01 } & \multicolumn{1}{r}{13.21 } & \multicolumn{1}{r}{17.62 } & \multicolumn{1}{r}{16.73 } & \multicolumn{1}{r}{35.23 } & 29.22  & 30.13  &       &       & 11.73  & 11.59  & 11.13  & 12.97  & 23.89  & 12.16  & 12.56  \\
				5(High) & \multicolumn{1}{r}{-0.05 } & \multicolumn{1}{r}{4.43 } & \multicolumn{1}{r}{10.15 } & \multicolumn{1}{r}{19.03 } & \multicolumn{1}{r}{27.41 } & 27.45  & 26.96  &       &       & -3.24  & 11.02  & 3.32  & 14.02  & 14.78  & 18.02  & 20.43  \\
				High-Low & \multicolumn{1}{r}{-37.81 } & \multicolumn{1}{r}{-43.18 } & \multicolumn{1}{r}{-36.70 } & \multicolumn{1}{r}{-39.26 } & \multicolumn{1}{r}{-46.12 } &       &       &       &       & -38.57  & -24.10  & -28.92  & -21.55  & -34.56  &       &  \\
				Alpha & \multicolumn{1}{r}{-36.03 } & \multicolumn{1}{r}{-43.31 } & \multicolumn{1}{r}{-35.59 } & \multicolumn{1}{r}{-40.71 } & \multicolumn{1}{r}{-46.84 } &       &       &       &       & -39.49  & -23.26  & -27.97  & -21.02  & -32.70  &       &  \\
				\multicolumn{6}{l}{Industry-Level Effect (average of High-Low column; Alpha column)} & 29.84  & 30.82  &       &       &       &       &       &       &       & 13.43  & 14.13  \\
				\multicolumn{6}{l}{Stock-Level Effect (average of High-Low row; Alpha row)} & -40.61  & -40.50  &       &       &       &       &       &       &       & -29.54  & -28.89  \\
				&       &       &       &       &       &       &       &       &       &       &       &       &       &       &       &  \\
				&       &       &       &       &       &       &       &       &       &       &       &       &       &       &       &  \\
				\multicolumn{17}{l}{B: t-Statistics}         \\
				1(Low) & \multicolumn{1}{r}{3.65 } & \multicolumn{1}{r}{4.38 } & \multicolumn{1}{r}{4.27 } & \multicolumn{1}{r}{5.36 } & \multicolumn{1}{r}{6.72 } & 5.36  & 5.48  &       &       & 4.44  & 4.21  & 3.82  & 4.43  & 5.58  & 1.87  & 1.77  \\
				2     & \multicolumn{1}{r}{2.90 } & \multicolumn{1}{r}{3.09 } & \multicolumn{1}{r}{3.38 } & \multicolumn{1}{r}{4.41 } & \multicolumn{1}{r}{5.37 } & 4.27  & 4.21  &       &       & 3.76  & 3.43  & 2.72  & 4.31  & 4.66  & 1.44  & 1.30  \\
				3     & \multicolumn{1}{r}{1.29 } & \multicolumn{1}{r}{2.10 } & \multicolumn{1}{r}{2.19 } & \multicolumn{1}{r}{3.10 } & \multicolumn{1}{r}{4.33 } & 4.59  & 4.57  &       &       & 2.08  & 3.22  & 2.18  & 2.78  & 4.20  & 1.99  & 2.20  \\
				4     & \multicolumn{1}{r}{0.62 } & \multicolumn{1}{r}{1.39 } & \multicolumn{1}{r}{1.82 } & \multicolumn{1}{r}{1.85 } & \multicolumn{1}{r}{4.11 } & 4.78  & 5.00  &       &       & 1.53  & 1.58  & 1.45  & 1.67  & 3.56  & 1.87  & 1.87  \\
				5(High) & \multicolumn{1}{r}{0.00 } & \multicolumn{1}{r}{0.48 } & \multicolumn{1}{r}{1.09 } & \multicolumn{1}{r}{2.21 } & \multicolumn{1}{r}{3.35 } & 4.39  & 4.25  &       &       & -0.39  & 1.44  & 0.42  & 2.05  & 2.26  & 2.65  & 2.96  \\
				High-Low & \multicolumn{1}{r}{-6.69 } & \multicolumn{1}{r}{-7.65 } & \multicolumn{1}{r}{-7.03 } & \multicolumn{1}{r}{-6.72 } & \multicolumn{1}{r}{-7.73 } &       &       &       &       & -6.87  & -4.01  & -5.48  & -3.88  & -5.77  &       &  \\
				Alpha & \multicolumn{1}{r}{-6.18 } & \multicolumn{1}{r}{-7.77 } & \multicolumn{1}{r}{-6.87 } & \multicolumn{1}{r}{-7.24 } & \multicolumn{1}{r}{-8.29 } &       &       &       &       & -7.10  & -3.83  & -5.42  & -3.85  & -6.02  &       &  \\
				\multicolumn{6}{l}{Industry-Level Effect (average of High-Low column; Alpha column)} & 5.74  & 5.66  &       &       &       &       &       &       &       & 2.62  & 2.57  \\
				\multicolumn{6}{l}{Stock-Level Effect (average of High-Low row; Alpha row)} & -9.87  & -10.19  &       &       &       &       &       &       &       & -9.02  & -9.20  \\
				\bottomrule
			\end{tabular}
			\begin{tablenotes}	
				\item{}
				\item { $^*\>$ Notes: See notes to Table 8. This table presents average returns and risk-adjusted alphas for portfolios sorted by stock-level and industry-level SRVJ. The sample includes all NYSE, NASDAQ and AMEX listed stocks for the period January 1993 to December 2016. A stock's industry signed jump variation (SRVJ) is the capitalization-weighted average of the SRVJ of all stocks within the industry. At the end of each Tuesday, all stocks in the sample are sorted into quintile portfolios based on stock-level and industry-level SRVJ, independently, resulting in 25 portfolios. Each portfolio is held for one week. The row labeled ``Industry-Level Effect'' reports average values of one-week ahead returns (and Fama-French-Carhart four-factor alphas in the High-Low (Alpha) column) in Panel A (corresponding Newey-West $t$-statistics are given in Panel B). The row labeled ``Stock-Level Effect'' reports the average values of one-week 
				ahead returns (and alphas) in panel A (corresponding Newey-West $t$-statistics are again given in Panel B).  }
			\end{tablenotes}
	\end{threeparttable}}
\end{table}
\newpage
\noindent Table 14: Double-Sorted Portfolios and Large Jumps: Portfolios Sorted by Stock- and Industry-Level SRVLJ Independently$^*\>$

% Table generated by Excel2LaTeX from sheet '4srvlj'
\begin{table}[htbp!]
	\centering
	%\caption*{\normalfont Panel A: Portfolios Sorted Based on Truncation Level $\alpha_n^{(1)}$}
	\resizebox{1.04\textwidth}{!}{
		\begin{tabular}{llllllrrrrrrrrrrr}
			\multicolumn{17}{l}{Panel A: Portfolios Sorted Based on Truncation Level $\alpha_n^{1}$}\\
			&      &      &      &      &      &  & & &
			&       &     &      &      &      &      & 
			\\
			&    & \multicolumn{4}{c}{Equal-Weighted Returns and Alphas }     &       &       &  & & & \multicolumn{4}{c}{Value-Weighted Returns and Alphas}   &     \\
			\midrule
			&    & \multicolumn{4}{c}{Industry-Level Quintile }     &       &   &  &  &  & \multicolumn{4}{c}{Industry-Level Quintile}   &      \\
			
			Stock-Level Quintile & \multicolumn{1}{c}{1(Low)} & \multicolumn{1}{c}{2} & \multicolumn{1}{c}{3} & \multicolumn{1}{c}{4} & \multicolumn{1}{c}{5(High)} & \multicolumn{1}{c}{High-Low} & \multicolumn{1}{c}{Alpha} &       &       & \multicolumn{1}{c}{1(Low)} & \multicolumn{1}{c}{2} & \multicolumn{1}{c}{3} & \multicolumn{1}{c}{4} & \multicolumn{1}{c}{5(High)} & \multicolumn{1}{c}{High-Low} & \multicolumn{1}{c}{Alpha} \\
			\multicolumn{17}{l}{I: Mean Return and Alpha   }                         \\
			1(Low) & \multicolumn{1}{r}{39.87 } & \multicolumn{1}{r}{32.74 } & \multicolumn{1}{r}{44.02 } & \multicolumn{1}{r}{55.68 } & \multicolumn{1}{r}{49.96 } & 10.09  & 10.02  &       &       & 23.94  & 25.73  & 30.09  & 30.05  & 30.88  & 6.94  & 4.47  \\
			2     & \multicolumn{1}{r}{28.60 } & \multicolumn{1}{r}{22.34 } & \multicolumn{1}{r}{33.27 } & \multicolumn{1}{r}{39.05 } & \multicolumn{1}{r}{32.64 } & 4.05  & 3.50  &       &       & 21.20  & 17.86  & 31.66  & 29.16  & 24.58  & 3.37  & 1.26  \\
			3     & \multicolumn{1}{r}{28.73 } & \multicolumn{1}{r}{18.53 } & \multicolumn{1}{r}{29.08 } & \multicolumn{1}{r}{32.87 } & \multicolumn{1}{r}{29.32 } & 3.37  & 2.56  &       &       & 20.54  & 9.03  & 24.60  & 27.47  & 28.85  & 11.99  & 11.18  \\
			4     & \multicolumn{1}{r}{21.39 } & \multicolumn{1}{r}{11.76 } & \multicolumn{1}{r}{21.71 } & \multicolumn{1}{r}{25.80 } & \multicolumn{1}{r}{30.50 } & 9.11  & 7.65  &       &       & 16.04  & 19.00  & 13.25  & 16.13  & 20.90  & 4.87  & 4.22  \\
			5(High) & \multicolumn{1}{r}{11.98 } & \multicolumn{1}{r}{13.19 } & \multicolumn{1}{r}{17.73 } & \multicolumn{1}{r}{18.83 } & \multicolumn{1}{r}{27.33 } & 15.34  & 14.82  &       &       & 16.77  & 12.29  & 17.80  & 18.52  & 17.96  & 1.19  & 2.57  \\
			High-Low & \multicolumn{1}{r}{-27.88 } & \multicolumn{1}{r}{-19.55 } & \multicolumn{1}{r}{-26.30 } & \multicolumn{1}{r}{-36.84 } & \multicolumn{1}{r}{-22.64 } &       &       &       &       & -7.17  & -13.43  & -12.29  & -11.54  & -12.92  &       &  \\
			Alpha & \multicolumn{1}{r}{-27.87 } & \multicolumn{1}{r}{-19.54 } & \multicolumn{1}{r}{-27.63 } & \multicolumn{1}{r}{-37.77 } & \multicolumn{1}{r}{-23.08 } &       &       &       &       & -7.77  & -13.59  & -11.30  & -11.43  & -9.67  &       &  \\
			\multicolumn{6}{l}{Industry-Level Effect (average of High-Low column; Alpha column)} & 8.59  & 7.90  &       &       &       &       &       &       &       & 5.77  & 4.82  \\
			\multicolumn{6}{l}{Stock-Level Effect (average of High-Low row; Alpha row)} & -26.64  & -27.18  &       &       &       &       &       &       &       & -11.47  & -10.75  \\
			&       &       &       &       &       &       &       &       &       &       &       &       &       &       &       &  \\
			&       &       &       &       &       &       &       &       &       &       &       &       &       &       &       &  \\
			\multicolumn{17}{l}{II: t-Statistics}         \\
			1(Low) & \multicolumn{1}{r}{4.09 } & \multicolumn{1}{r}{3.00 } & \multicolumn{1}{r}{3.93 } & \multicolumn{1}{r}{5.27 } & \multicolumn{1}{r}{4.86 } & 1.95  & 1.89  &       &       & 3.29  & 3.19  & 3.84  & 3.65  & 3.71  & 1.13  & 0.71  \\
			2     & \multicolumn{1}{r}{3.08 } & \multicolumn{1}{r}{2.08 } & \multicolumn{1}{r}{3.07 } & \multicolumn{1}{r}{3.82 } & \multicolumn{1}{r}{3.35 } & 0.79  & 0.68  &       &       & 3.12  & 2.31  & 3.88  & 3.82  & 3.21  & 0.59  & 0.22  \\
			3     & \multicolumn{1}{r}{2.83 } & \multicolumn{1}{r}{1.70 } & \multicolumn{1}{r}{2.69 } & \multicolumn{1}{r}{3.11 } & \multicolumn{1}{r}{3.03 } & 0.48  & 0.36  &       &       & 2.66  & 1.08  & 2.94  & 3.37  & 3.59  & 1.60  & 1.48  \\
			4     & \multicolumn{1}{r}{2.20 } & \multicolumn{1}{r}{1.14 } & \multicolumn{1}{r}{2.17 } & \multicolumn{1}{r}{2.63 } & \multicolumn{1}{r}{3.47 } & 1.77  & 1.54  &       &       & 2.19  & 2.32  & 1.71  & 2.07  & 3.03  & 0.88  & 0.78  \\
			5(High) & \multicolumn{1}{r}{1.33 } & \multicolumn{1}{r}{1.38 } & \multicolumn{1}{r}{1.90 } & \multicolumn{1}{r}{2.09 } & \multicolumn{1}{r}{3.15 } & 2.88  & 2.73  &       &       & 2.25  & 1.57  & 2.40  & 2.47  & 2.79  & 0.21  & 0.46  \\
			High-Low & \multicolumn{1}{r}{-6.71 } & \multicolumn{1}{r}{-4.00 } & \multicolumn{1}{r}{-5.35 } & \multicolumn{1}{r}{-7.88 } & \multicolumn{1}{r}{-4.36 } &       &       &       &       & -1.42  & -2.42  & -2.38  & -2.04  & -2.40  &       &  \\
			Alpha & \multicolumn{1}{r}{-6.77 } & \multicolumn{1}{r}{-3.98 } & \multicolumn{1}{r}{-5.67 } & \multicolumn{1}{r}{-8.16 } & \multicolumn{1}{r}{-4.49 } &       &       &       &       & -1.57  & -2.46  & -2.17  & -2.04  & -1.90  &       &  \\
			\multicolumn{6}{l}{Industry-Level Effect (average of High-Low column; Alpha column)} & 2.08  & 1.89  &       &       &       &       &       &       &       & 1.41  & 1.15  \\
			\multicolumn{6}{l}{Stock-Level Effect (average of High-Low row; Alpha row)} & -8.16  & -8.62  &       &       &       &       &       &       &       & -4.43  & -4.30  \\
			\bottomrule
	\end{tabular}}
\end{table}
\begin{table}[htbp!]
	\centering
	%\caption*{\normalfont Panel B: Portfolios Sorted Based on Truncation Level $\alpha_n^{(2)}$}
	\resizebox{1.04\textwidth}{!}{
		\begin{threeparttable}
			\begin{tabular}{llllllrrrrrrrrrrr}
				\multicolumn{17}{l}{Panel B: Portfolios Sorted Based on Truncation Level $\alpha_n^{2}$}\\
				&      &      &      &      &      &  & & &
				&       &     &      &      &      &      & 
				\\
				&    & \multicolumn{4}{c}{Equal-Weighted Returns and Alphas }     &       &       &  & & & \multicolumn{4}{c}{Value-Weighted Returns and Alphas}   &     \\
				\midrule
				&    & \multicolumn{4}{c}{Industry-Level Quintile }     &       &   &  &  &  & \multicolumn{4}{c}{Industry-Level Quintile}   &      \\
				
				Stock-Level Quintile & \multicolumn{1}{c}{1(Low)} & \multicolumn{1}{c}{2} & \multicolumn{1}{c}{3} & \multicolumn{1}{c}{4} & \multicolumn{1}{c}{5(High)} & \multicolumn{1}{c}{High-Low} & \multicolumn{1}{c}{Alpha} &       &       & \multicolumn{1}{c}{1(Low)} & \multicolumn{1}{c}{2} & \multicolumn{1}{c}{3} & \multicolumn{1}{c}{4} & \multicolumn{1}{c}{5(High)} & \multicolumn{1}{c}{High-Low} & \multicolumn{1}{c}{Alpha} \\
				\multicolumn{17}{l}{I: Mean Return and Alpha   }                         \\
				1(Low) & \multicolumn{1}{r}{35.69 } & \multicolumn{1}{r}{33.44 } & \multicolumn{1}{r}{30.98 } & \multicolumn{1}{r}{50.42 } & \multicolumn{1}{r}{50.02 } & 14.33  & 14.59  &       &       & 21.46  & 24.80  & 22.08  & 27.67  & 27.24  & 5.78  & 3.37  \\
				2     & \multicolumn{1}{r}{26.89 } & \multicolumn{1}{r}{23.40 } & \multicolumn{1}{r}{19.33 } & \multicolumn{1}{r}{34.38 } & \multicolumn{1}{r}{35.22 } & 8.33  & 6.98  &       &       & 18.63  & 23.16  & 18.94  & 25.43  & 27.20  & 8.57  & 7.37  \\
				3     & \multicolumn{1}{r}{9.46 } & \multicolumn{1}{r}{-7.23 } & \multicolumn{1}{r}{-7.19 } & \multicolumn{1}{r}{11.52 } & \multicolumn{1}{r}{39.79 } & 22.26  & 16.90  &       &       & 3.96  & 12.49  & -5.02  & -11.83  & 37.15  & 25.13  & 19.15  \\
				4     & \multicolumn{1}{r}{18.20 } & \multicolumn{1}{r}{18.68 } & \multicolumn{1}{r}{17.08 } & \multicolumn{1}{r}{27.97 } & \multicolumn{1}{r}{36.04 } & 18.37  & 17.80  &       &       & 14.04  & 23.22  & 9.17  & 14.37  & 27.31  & 13.87  & 14.76  \\
				5(High) & \multicolumn{1}{r}{16.83 } & \multicolumn{1}{r}{17.04 } & \multicolumn{1}{r}{14.55 } & \multicolumn{1}{r}{25.94 } & \multicolumn{1}{r}{26.30 } & 9.48  & 9.26  &       &       & 15.97  & 21.80  & 15.74  & 18.61  & 22.86  & 6.89  & 9.45  \\
				High-Low & \multicolumn{1}{r}{-18.86 } & \multicolumn{1}{r}{-16.41 } & \multicolumn{1}{r}{-16.43 } & \multicolumn{1}{r}{-24.48 } & \multicolumn{1}{r}{-23.72 } &       &       &       &       & -5.49  & -3.00  & -6.33  & -9.06  & -4.38  &       &  \\
				Alpha & \multicolumn{1}{r}{-19.14 } & \multicolumn{1}{r}{-16.81 } & \multicolumn{1}{r}{-16.90 } & \multicolumn{1}{r}{-25.22 } & \multicolumn{1}{r}{-24.47 } &       &       &       &       & -7.08  & -3.71  & -6.69  & -8.48  & -0.99  &       &  \\
				\multicolumn{6}{l}{Industry-Level Effect (average of High-Low column; Alpha column)} & 13.03  & 12.27  &       &       &       &       &       &       &       & 9.75  & 9.29  \\
				\multicolumn{6}{l}{Stock-Level Effect (average of High-Low row; Alpha row)} & -19.98  & -20.51  &       &       &       &       &       &       &       & -5.65  & -5.39  \\
				&       &       &       &       &       &       &       &       &       &       &       &       &       &       &       &  \\
				&       &       &       &       &       &       &       &       &       &       &       &       &       &       &       &  \\
				\multicolumn{17}{l}{II: t-Statistics}         \\
				1(Low) & \multicolumn{1}{r}{3.67 } & \multicolumn{1}{r}{3.17 } & \multicolumn{1}{r}{2.86 } & \multicolumn{1}{r}{4.74 } & \multicolumn{1}{r}{5.08 } & 2.87  & 2.85  &       &       & 3.15  & 3.06  & 2.68  & 3.40  & 3.44  & 0.99  & 0.56  \\
				2     & \multicolumn{1}{r}{2.75 } & \multicolumn{1}{r}{2.24 } & \multicolumn{1}{r}{1.90 } & \multicolumn{1}{r}{3.47 } & \multicolumn{1}{r}{3.68 } & 1.77  & 1.47  &       &       & 2.71  & 3.19  & 2.50  & 3.50  & 3.76  & 1.64  & 1.35  \\
				3     & \multicolumn{1}{r}{0.44 } & \multicolumn{1}{r}{-0.30 } & \multicolumn{1}{r}{-0.33 } & \multicolumn{1}{r}{0.56 } & \multicolumn{1}{r}{1.63 } & 1.19  & 0.95  &       &       & 0.24  & 0.57  & -0.29  & -0.64  & 1.74  & 1.35  & 1.08  \\
				4     & \multicolumn{1}{r}{1.81 } & \multicolumn{1}{r}{1.74 } & \multicolumn{1}{r}{1.61 } & \multicolumn{1}{r}{2.80 } & \multicolumn{1}{r}{3.96 } & 3.22  & 3.21  &       &       & 1.87  & 2.80  & 1.11  & 1.84  & 3.89  & 2.35  & 2.46  \\
				5(High) & \multicolumn{1}{r}{1.80 } & \multicolumn{1}{r}{1.70 } & \multicolumn{1}{r}{1.51 } & \multicolumn{1}{r}{2.92 } & \multicolumn{1}{r}{3.15 } & 2.13  & 2.13  &       &       & 2.07  & 2.80  & 2.07  & 2.54  & 3.58  & 1.20  & 1.77  \\
				High-Low & \multicolumn{1}{r}{-5.08 } & \multicolumn{1}{r}{-3.12 } & \multicolumn{1}{r}{-3.65 } & \multicolumn{1}{r}{-5.30 } & \multicolumn{1}{r}{-5.77 } &       &       &       &       & -1.06  & -0.60  & -1.40  & -1.84  & -0.85  &       &  \\
				Alpha & \multicolumn{1}{r}{-5.21 } & \multicolumn{1}{r}{-3.06 } & \multicolumn{1}{r}{-3.70 } & \multicolumn{1}{r}{-5.46 } & \multicolumn{1}{r}{-6.18 } &       &       &       &       & -1.43  & -0.74  & -1.45  & -1.71  & -0.21  &       &  \\
				\multicolumn{6}{l}{Industry effect (average of High-Low column; Alpha column)} & 3.30  & 3.06  &       &       &       &       &       &       &       & 2.51  & 2.25  \\
				\multicolumn{6}{l}{Within-industry effect (average of High-Low row; Alpha row)} & -7.19  & -7.43  &       &       &       &       &       &       &       & -2.35  & -2.31  \\
				\bottomrule
			\end{tabular}
			\begin{tablenotes}	
				\item{}
				\item { $^*\>$ Notes: See notes to Table 13. Jumps are decomposed using two truncation levels, as discussed in the footnote to Table 2. }
			\end{tablenotes}
	\end{threeparttable}}
\end{table}
\newpage
\noindent Table 15: Double-Sorted Portfolios and Small Jumps: Portfolios Sorted by Stock- and Industry-Level SRVSJ Independently$^*\>$

% Table generated by Excel2LaTeX from sheet '4srvsj'
\begin{table}[htbp!]
	\centering
	%\caption*{\normalfont Panel A: Portfolios Sorted Based on Truncation Level $\alpha_n^{(1)}$}
	\resizebox{1.04\textwidth}{!}{
		\begin{tabular}{llllllrrrrrrrrrrr}
			\multicolumn{17}{l}{Panel A: Portfolios Sorted Based on Truncation Level $\alpha_n^{1}$}\\
			&      &      &      &      &      &  & & &
			&       &     &      &      &      &      & 
			\\
			&    & \multicolumn{4}{c}{Equal-Weighted Returns and Alphas }     &       &       &  & & & \multicolumn{4}{c}{Value-Weighted Returns and Alphas}   &     \\
			\midrule
			&    & \multicolumn{4}{c}{Industry-Level Quintile }     &       &   &  &  &  & \multicolumn{4}{c}{Industry-Level Quintile}   &      \\
			
			Stock-Level Quintile & \multicolumn{1}{c}{1(Low)} & \multicolumn{1}{c}{2} & \multicolumn{1}{c}{3} & \multicolumn{1}{c}{4} & \multicolumn{1}{c}{5(High)} & \multicolumn{1}{c}{High-Low} & \multicolumn{1}{c}{Alpha} &       &       & \multicolumn{1}{c}{1(Low)} & \multicolumn{1}{c}{2} & \multicolumn{1}{c}{3} & \multicolumn{1}{c}{4} & \multicolumn{1}{c}{5(High)} & \multicolumn{1}{c}{High-Low} & \multicolumn{1}{c}{Alpha} \\
			\multicolumn{17}{l}{I: Mean Return and Alpha   }                         \\
			1(Low) & \multicolumn{1}{r}{36.01 } & \multicolumn{1}{r}{40.51 } & \multicolumn{1}{r}{46.50 } & \multicolumn{1}{r}{50.96 } & \multicolumn{1}{r}{71.94 } & 35.93  & 38.05  &       &       & 37.02  & 37.16  & 30.09  & 38.93  & 49.15  & 12.13  & 11.11  \\
			2     & \multicolumn{1}{r}{30.33 } & \multicolumn{1}{r}{37.39 } & \multicolumn{1}{r}{42.67 } & \multicolumn{1}{r}{42.29 } & \multicolumn{1}{r}{58.00 } & 27.67  & 28.40  &       &       & 24.86  & 20.79  & 27.32  & 25.52  & 37.11  & 12.25  & 12.07  \\
			3     & \multicolumn{1}{r}{10.63 } & \multicolumn{1}{r}{21.86 } & \multicolumn{1}{r}{25.74 } & \multicolumn{1}{r}{27.65 } & \multicolumn{1}{r}{42.61 } & 31.99  & 34.44  &       &       & 14.35  & 20.99  & 11.39  & 20.23  & 20.50  & 6.14  & 8.03  \\
			4     & \multicolumn{1}{r}{7.60 } & \multicolumn{1}{r}{8.46 } & \multicolumn{1}{r}{20.22 } & \multicolumn{1}{r}{21.78 } & \multicolumn{1}{r}{35.16 } & 27.57  & 27.80  &       &       & 15.50  & 19.53  & 25.75  & 11.07  & 21.54  & 6.05  & 7.82  \\
			5(High) & \multicolumn{1}{r}{3.90 } & \multicolumn{1}{r}{3.70 } & \multicolumn{1}{r}{12.38 } & \multicolumn{1}{r}{16.57 } & \multicolumn{1}{r}{25.31 } & 21.41  & 22.72  &       &       & 5.10  & 12.36  & 14.36  & 11.98  & 12.27  & 7.17  & 9.97  \\
			High-Low & \multicolumn{1}{r}{-32.11 } & \multicolumn{1}{r}{-36.81 } & \multicolumn{1}{r}{-34.13 } & \multicolumn{1}{r}{-34.39 } & \multicolumn{1}{r}{-46.63 } &       &       &       &       & -31.92  & -24.81  & -15.74  & -26.94  & -36.89  &       &  \\
			Alpha & \multicolumn{1}{r}{-31.11 } & \multicolumn{1}{r}{-36.08 } & \multicolumn{1}{r}{-33.19 } & \multicolumn{1}{r}{-34.09 } & \multicolumn{1}{r}{-46.44 } &       &       &       &       & -33.24  & -25.06  & -14.60  & -25.89  & -34.38  &       &  \\
			\multicolumn{6}{l}{Industry-Level Effect (average of High-Low column; Alpha column)} & 28.91  & 30.29  &       &       &       &       &       &       &       & 8.75  & 9.80  \\
			\multicolumn{6}{l}{Stock-Level Effect (average of High-Low row; Alpha row)} & -36.81  & -36.18  &       &       &       &       &       &       &       & -27.26  & -26.63  \\
			&       &       &       &       &       &       &       &       &       &       &       &       &       &       &       &  \\
			&       &       &       &       &       &       &       &       &       &       &       &       &       &       &       &  \\
			\multicolumn{17}{l}{II: t-Statistics}         \\
			1(Low) & \multicolumn{1}{r}{3.46 } & \multicolumn{1}{r}{3.81 } & \multicolumn{1}{r}{4.34 } & \multicolumn{1}{r}{4.66 } & \multicolumn{1}{r}{6.12 } & 5.00  & 5.05  &       &       & 4.87  & 4.42  & 3.71  & 4.74  & 5.80  & 1.76  & 1.58  \\
			2     & \multicolumn{1}{r}{2.89 } & \multicolumn{1}{r}{3.46 } & \multicolumn{1}{r}{4.11 } & \multicolumn{1}{r}{4.03 } & \multicolumn{1}{r}{5.52 } & 4.10  & 4.05  &       &       & 3.13  & 2.65  & 3.45  & 3.19  & 4.71  & 1.74  & 1.69  \\
			3     & \multicolumn{1}{r}{1.07 } & \multicolumn{1}{r}{2.21 } & \multicolumn{1}{r}{2.60 } & \multicolumn{1}{r}{2.86 } & \multicolumn{1}{r}{4.37 } & 5.19  & 5.41  &       &       & 1.83  & 2.74  & 1.56  & 2.59  & 2.31  & 0.78  & 0.96  \\
			4     & \multicolumn{1}{r}{0.74 } & \multicolumn{1}{r}{0.86 } & \multicolumn{1}{r}{2.06 } & \multicolumn{1}{r}{2.35 } & \multicolumn{1}{r}{3.66 } & 4.37  & 4.37  &       &       & 1.90  & 2.60  & 3.45  & 1.54  & 2.96  & 0.91  & 1.22  \\
			5(High) & \multicolumn{1}{r}{0.40 } & \multicolumn{1}{r}{0.38 } & \multicolumn{1}{r}{1.36 } & \multicolumn{1}{r}{1.90 } & \multicolumn{1}{r}{2.96 } & 3.35  & 3.59  &       &       & 0.62  & 1.59  & 1.91  & 1.60  & 1.95  & 1.08  & 1.55  \\
			High-Low & \multicolumn{1}{r}{-6.62 } & \multicolumn{1}{r}{-8.26 } & \multicolumn{1}{r}{-7.56 } & \multicolumn{1}{r}{-6.67 } & \multicolumn{1}{r}{-7.34 } &       &       &       &       & -5.96  & -5.18  & -3.36  & -5.62  & -6.31  &       &  \\
			Alpha & \multicolumn{1}{r}{-6.47 } & \multicolumn{1}{r}{-8.05 } & \multicolumn{1}{r}{-7.61 } & \multicolumn{1}{r}{-7.12 } & \multicolumn{1}{r}{-8.09 } &       &       &       &       & -6.39  & -5.22  & -3.09  & -5.41  & -6.21  &       &  \\
			\multicolumn{6}{l}{Industry-Level Effect (average of High-Low column; Alpha column)} & 5.32  & 5.34  &       &       &       &       &       &       &       & 1.69  & 1.83  \\
			\multicolumn{6}{l}{Stock-Level Effect (average of High-Low row; Alpha row)} & -10.22  & -11.02  &       &       &       &       &       &       &       & -9.14  & -9.12  \\
			\bottomrule
	\end{tabular}}
\end{table}
\begin{table}[htbp!]
	\centering
	%\caption*{\normalfont Panel B: Portfolios Sorted Based on Truncation Level $\alpha_n^{(2)}$}
	\resizebox{1.04\textwidth}{!}{
		\begin{threeparttable}
			\begin{tabular}{llllllrrrrrrrrrrr}
				\multicolumn{17}{l}{Panel B: Portfolios Sorted Based on Truncation Level $\alpha_n^{2}$}\\
				&      &      &      &      &      &  & & &
				&       &     &      &      &      &      & 
				\\
				&    & \multicolumn{4}{c}{Equal-Weighted Returns and Alphas }     &       &       &  & & & \multicolumn{4}{c}{Value-Weighted Returns and Alphas}   &     \\
				\midrule
				&    & \multicolumn{4}{c}{Industry-Level Quintile }     &       &   &  &  &  & \multicolumn{4}{c}{Industry-Level Quintile}   &      \\
				
				Stock-Level Quintile & \multicolumn{1}{c}{1(Low)} & \multicolumn{1}{c}{2} & \multicolumn{1}{c}{3} & \multicolumn{1}{c}{4} & \multicolumn{1}{c}{5(High)} & \multicolumn{1}{c}{High-Low} & \multicolumn{1}{c}{Alpha} &       &       & \multicolumn{1}{c}{1(Low)} & \multicolumn{1}{c}{2} & \multicolumn{1}{c}{3} & \multicolumn{1}{c}{4} & \multicolumn{1}{c}{5(High)} & \multicolumn{1}{c}{High-Low} & \multicolumn{1}{c}{Alpha} \\
				\multicolumn{17}{l}{I: Mean Return and Alpha   }                         \\
				1(Low) & \multicolumn{1}{r}{39.41 } & \multicolumn{1}{r}{41.49 } & \multicolumn{1}{r}{42.85 } & \multicolumn{1}{r}{53.73 } & \multicolumn{1}{r}{66.43 } & 27.03  & 28.82  &       &       & 43.07  & 39.28  & 34.00  & 37.21  & 46.32  & 3.25  & 3.29  \\
				2     & \multicolumn{1}{r}{31.72 } & \multicolumn{1}{r}{40.89 } & \multicolumn{1}{r}{42.06 } & \multicolumn{1}{r}{44.72 } & \multicolumn{1}{r}{60.99 } & 29.27  & 30.84  &       &       & 22.07  & 25.29  & 24.78  & 27.02  & 40.50  & 18.43  & 19.21  \\
				3     & \multicolumn{1}{r}{14.32 } & \multicolumn{1}{r}{25.25 } & \multicolumn{1}{r}{27.44 } & \multicolumn{1}{r}{28.17 } & \multicolumn{1}{r}{46.29 } & 31.97  & 32.89  &       &       & 18.03  & 17.24  & 23.08  & 17.13  & 25.70  & 7.68  & 7.99  \\
				4     & \multicolumn{1}{r}{6.66 } & \multicolumn{1}{r}{8.95 } & \multicolumn{1}{r}{17.65 } & \multicolumn{1}{r}{16.36 } & \multicolumn{1}{r}{32.12 } & 25.46  & 26.03  &       &       & 10.67  & 14.13  & 22.47  & 10.76  & 26.48  & 15.81  & 17.41  \\
				5(High) & \multicolumn{1}{r}{-0.40 } & \multicolumn{1}{r}{3.95 } & \multicolumn{1}{r}{7.94 } & \multicolumn{1}{r}{11.63 } & \multicolumn{1}{r}{22.53 } & 22.93  & 24.41  &       &       & 7.98  & 9.51  & 11.49  & 7.94  & 11.34  & 3.36  & 6.64  \\
				High-Low & \multicolumn{1}{r}{-39.80 } & \multicolumn{1}{r}{-37.53 } & \multicolumn{1}{r}{-34.90 } & \multicolumn{1}{r}{-42.10 } & \multicolumn{1}{r}{-43.91 } &       &       &       &       & -35.09  & -29.77  & -22.52  & -29.27  & -34.99  &       &  \\
				Alpha & \multicolumn{1}{r}{-39.32 } & \multicolumn{1}{r}{-37.57 } & \multicolumn{1}{r}{-33.54 } & \multicolumn{1}{r}{-41.84 } & \multicolumn{1}{r}{-43.73 } &       &       &       &       & -36.29  & -30.14  & -21.15  & -28.97  & -32.94  &       &  \\
				\multicolumn{6}{l}{Industry-Level Effect (average of High-Low column; Alpha column)} & 27.33  & 28.60  &       &       &       &       &       &       &       & 9.71  & 10.91  \\
				\multicolumn{6}{l}{Stock-Level Effect (average of High-Low row; Alpha row)} & -39.65  & -39.20  &       &       &       &       &       &       &       & -30.33  & -29.90  \\
				&       &       &       &       &       &       &       &       &       &       &       &       &       &       &       &  \\
				&       &       &       &       &       &       &       &       &       &       &       &       &       &       &       &  \\
				\multicolumn{17}{l}{II: t-Statistics}         \\
				1(Low) & \multicolumn{1}{r}{3.80 } & \multicolumn{1}{r}{3.82 } & \multicolumn{1}{r}{3.90 } & \multicolumn{1}{r}{4.87 } & \multicolumn{1}{r}{6.14 } & 3.91  & 3.85  &       &       & 5.43  & 4.79  & 3.93  & 4.55  & 5.68  & 0.47  & 0.45  \\
				2     & \multicolumn{1}{r}{2.94 } & \multicolumn{1}{r}{3.84 } & \multicolumn{1}{r}{3.97 } & \multicolumn{1}{r}{4.32 } & \multicolumn{1}{r}{5.69 } & 4.12  & 4.07  &       &       & 2.92  & 3.18  & 2.99  & 3.42  & 5.21  & 2.90  & 2.87  \\
				3     & \multicolumn{1}{r}{1.47 } & \multicolumn{1}{r}{2.44 } & \multicolumn{1}{r}{2.70 } & \multicolumn{1}{r}{2.93 } & \multicolumn{1}{r}{4.81 } & 5.25  & 5.27  &       &       & 2.24  & 2.12  & 2.92  & 2.27  & 3.46  & 1.03  & 1.08  \\
				4     & \multicolumn{1}{r}{0.68 } & \multicolumn{1}{r}{0.92 } & \multicolumn{1}{r}{1.80 } & \multicolumn{1}{r}{1.74 } & \multicolumn{1}{r}{3.60 } & 3.97  & 3.97  &       &       & 1.36  & 1.96  & 2.83  & 1.41  & 3.88  & 2.42  & 2.68  \\
				5(High) & \multicolumn{1}{r}{-0.04 } & \multicolumn{1}{r}{0.42 } & \multicolumn{1}{r}{0.85 } & \multicolumn{1}{r}{1.34 } & \multicolumn{1}{r}{2.76 } & 3.70  & 3.92  &       &       & 0.96  & 1.15  & 1.48  & 1.10  & 1.76  & 0.49  & 0.93  \\
				High-Low & \multicolumn{1}{r}{-7.84 } & \multicolumn{1}{r}{-7.71 } & \multicolumn{1}{r}{-6.66 } & \multicolumn{1}{r}{-7.77 } & \multicolumn{1}{r}{-7.41 } &       &       &       &       & -6.30  & -5.32  & -4.05  & -5.57  & -6.27  &       &  \\
				Alpha & \multicolumn{1}{r}{-7.85 } & \multicolumn{1}{r}{-7.90 } & \multicolumn{1}{r}{-6.51 } & \multicolumn{1}{r}{-8.00 } & \multicolumn{1}{r}{-7.86 } &       &       &       &       & -6.61  & -5.54  & -3.84  & -5.59  & -5.92  &       &  \\
				\multicolumn{6}{l}{Industry-Level Effect (average of High-Low column; Alpha column)} & 5.09  & 5.03  &       &       &       &       &       &       &       & 1.86  & 1.97  \\
				\multicolumn{6}{l}{Stock-Level Effect (average of High-Low row; Alpha row)} & -10.24  & -10.81  &       &       &       &       &       &       &       & -9.01  & -9.01  \\
				\bottomrule
			\end{tabular}
			\begin{tablenotes}	
				\item{}
				\item { $^*\>$ Notes: See notes to Table 14. }
			\end{tablenotes}
	\end{threeparttable}}
\end{table}
\newpage
\noindent Table 16: Fama-MacBeth Cross-Sectional Regressions without Control Variables$^*\>$

% Table generated by Excel2LaTeX from sheet 'param5'
\begin{table}[htbp!]
	\centering
	%\caption*{\normalfont Panel A: Regressions Based on Truncation Level $\alpha_n^{(1)}$ }
	\resizebox{1.04\textwidth}{!}{
		\begin{tabular}{lcccccccccccccccc}
			\multicolumn{17}{l}{Panel A: Regressions Based on Truncation Level $\alpha_n^{1}$}\\
			&      &      &      &      &      &  & & &
			&       &     &      &      &      &      & 
			\\
			\toprule
			& I     & II    & III   & IV    & V     & VI    & VII   & VIII  & IX    & X     & XI    & XII   & XIII  & XIV   & XV    & XVI \\
			Intercept & 18.83 & 26.02 & 27.35 & 16.34 & 30.08 & 30.33 & 30.95 & 30.70 & 21.21 & 27.73 & 19.42 & 18.04 & 26.87 & 27.92 & 28.80 & 28.45 \\
			& (1.97) & (2.87) & (3.16) & (1.75) & (3.30) & (3.32) & (3.37) & (3.35) & (2.36) & (3.40) & (2.05) & (1.95) & (3.25) & (3.36) & (3.51) & (3.47) \\
			RVJP  & -53.27 &       &       &       &       &       &       &       & -154.73 &       &       &       &       &       &       &  \\
			& (-4.86) &       &       &       &       &       &       &       & (-5.54) &       &       &       &       &       &       &  \\
			RVJN  & 93.96 &       &       &       &       &       &       &       & 211.36 &       &       &       &       &       &       &  \\
			& (7.15) &       &       &       &       &       &       &       & (8.73) &       &       &       &       &       &       &  \\
			RVLJP &       & -48.03 &       & -39.06 &       &       &       &       &       & 3.45  &       & -126.41 &       &       &       &  \\
			&       & (-5.26) &       & (-3.72) &       &       &       &       &       & (0.20) &       & (-4.20) &       &       &       &  \\
			RVLJN &       & 70.11 &       & 79.47 &       &       &       &       &       & 29.92 &       & 181.47 &       &       &       &  \\
			&       & (6.95) &       & (6.32) &       &       &       &       &       & (2.17) &       & (7.21) &       &       &       &  \\
			RVSJP &       &       & -168.59 & -111.49 &       &       &       &       &       &       & -117.68 & -167.22 &       &       &       &  \\
			&       &       & (-8.57) & (-5.24) &       &       &       &       &       &       & (-6.76) & (-6.22) &       &       &       &  \\
			RVSJN &       &       & 190.47 & 178.49 &       &       &       &       &       &       & 156.75 & 238.48 &       &       &       &  \\
			&       &       & (7.36) & (6.20) &       &       &       &       &       &       & (7.36) & (8.22) &       &       &       &  \\
			SRVLJ &       &       &       &       & -46.45 &       & -49.78 &       &       &       &       &       & 6.52  &       & -132.80 &  \\
			&       &       &       &       & (-5.86) &       & (-6.19) &       &       &       &       &       & (0.39) &       & (-4.45) &  \\
			SRVSJ &       &       &       &       &       & -171.94 & -176.11 &       &       &       &       &       &       & -128.18 & -198.11 &  \\
			&       &       &       &       &       & (-8.44) & (-8.52) &       &       &       &       &       &       & (-7.86) & (-7.61) &  \\
			SRVJ  &       &       &       &       &       &       &       & -70.09 &       &       &       &       &       &       &       & -172.75 \\
			&       &       &       &       &       &       &       & (-7.93) &       &       &       &       &       &       &       & (-6.96) \\
			RVOL  &       &       &       &       &       &       &       &       & -17.20 & -15.52 & -13.66 & -17.44 & -12.91 & -13.57 & -13.80 & -13.72 \\
			&       &       &       &       &       &       &       &       & (-2.73) & (-2.43) & (-2.24) & (-2.78) & (-2.09) & (-2.21) & (-2.25) & (-2.23) \\
			RSK   &       &       &       &       &       &       &       &       & 23.44 & -9.57 & -9.13 & 18.92 & -12.48 & -9.33 & 15.71 & 21.80 \\
			&       &       &       &       &       &       &       &       & (4.70) & (-3.01) & (-7.25) & (3.44) & (-3.32) & (-7.37) & (2.49) & (4.32) \\
			RKT   &       &       &       &       &       &       &       &       & 0.31  & 0.47  & 1.44  & 0.54  & 0.99  & 1.00  & 0.95  & 0.97 \\
			&       &       &       &       &       &       &       &       & (0.61) & (0.66) & (2.10) & (0.83) & (1.84) & (1.81) & (1.92) & (1.93) \\
			Adjusted Rsquares & 0.0065  & 0.0041  & 0.0033  & 0.0088  & 0.0008  & 0.0018  & 0.0027  & 0.0017  & 0.0220  & 0.0194  & 0.0194  & 0.0233  & 0.0172  & 0.0181  & 0.0186  & 0.0181  \\
			\bottomrule
	\end{tabular}}
\end{table}
\begin{table}[htbp!]
	\centering
	%\caption*{\normalfont Panel B: Regressions Based on Truncation Level $\alpha_n^{(2)}$}
	\resizebox{1.04\textwidth}{!}{
		\begin{threeparttable}
			\begin{tabular}{lcccccccccccccccc}
				\multicolumn{17}{l}{Panel B: Regressions Based on Truncation Level $\alpha_n^{2}$}\\
				&      &      &      &      &      &  & & &
				&       &     &      &      &      &      & 
				\\
				\toprule
				& I     & II    & III   & IV    & V     & VI    & VII   & VIII  & IX    & X     & XI    & XII   & XIII  & XIV   & XV    & XVI \\
				Intercept & 18.83 & 26.77 & 27.18 & 18.70 & 29.77 & 30.65 & 30.98 & 30.70 & 21.21 & 29.81 & 18.31 & 20.76 & 26.96 & 28.05 & 28.69 & 28.45 \\
				& (1.97) & (2.95) & (3.16) & (2.02) & (3.28) & (3.34) & (3.37) & (3.35) & (2.36) & (3.63) & (1.88) & (2.26) & (3.24) & (3.38) & (3.50) & (3.47) \\
				RVJP  & -53.27 &       &       &       &       &       &       &       & -154.73 &       &       &       &       &       &       &  \\
				& (-4.86) &       &       &       &       &       &       &       & (-5.54) &       &       &       &       &       &       &  \\
				RVJN  & 93.96 &       &       &       &       &       &       &       & 211.36 &       &       &       &       &       &       &  \\
				& (7.15) &       &       &       &       &       &       &       & (8.73) &       &       &       &       &       &       &  \\
				RVLJP &       & -36.58 &       & -31.38 &       &       &       &       &       & 64.81 &       & -118.88 &       &       &       &  \\
				&       & (-3.75) &       & (-2.88) &       &       &       &       &       & (4.20) &       & (-3.32) &       &       &       &  \\
				RVLJN &       & 62.24 &       & 69.87 &       &       &       &       &       & -30.17 &       & 174.70 &       &       &       &  \\
				&       & (6.19) &       & (5.77) &       &       &       &       &       & (-2.29) &       & (6.29) &       &       &       &  \\
				RVSJP &       &       & -135.15 & -102.14 &       &       &       &       &       &       & -91.88 & -157.56 &       &       &       &  \\
				&       &       & (-9.10) & (-6.46) &       &       &       &       &       &       & (-6.90) & (-5.87) &       &       &       &  \\
				RVSJN &       &       & 150.86 & 145.95 &       &       &       &       &       &       & 128.49 & 212.18 &       &       &       &  \\
				&       &       & (7.57) & (6.65) &       &       &       &       &       &       & (8.07) & (7.89) &       &       &       &  \\
				SRVLJ &       &       &       &       & -36.51 &       & -39.77 &       &       &       &       &       & 58.99 &       & -123.62 &  \\
				&       &       &       &       & (-4.38) &       & (-4.71) &       &       &       &       &       & (4.51) &       & (-3.66) &  \\
				SRVSJ &       &       &       &       &       & -137.98 & -139.65 &       &       &       &       &       &       & -105.12 & -174.19 &  \\
				&       &       &       &       &       & (-8.89) & (-8.89) &       &       &       &       &       &       & (-8.58) & (-6.65) &  \\
				SRVJ  &       &       &       &       &       &       &       & -70.09 &       &       &       &       &       &       &       & -172.75 \\
				&       &       &       &       &       &       &       & (-7.93) &       &       &       &       &       &       &       & (-6.96) \\
				RVOL  &       &       &       &       &       &       &       &       & -17.20 & -15.26 & -14.05 & -17.41 & -12.99 & -13.64 & -13.75 & -13.72 \\
				&       &       &       &       &       &       &       &       & (-2.73) & (-2.41) & (-2.30) & (-2.77) & (-2.11) & (-2.22) & (-2.24) & (-2.23) \\
				RSK   &       &       &       &       &       &       &       &       & 23.44 & -18.78 & -7.06 & 18.23 & -20.28 & -7.32 & 14.74 & 21.80 \\
				&       &       &       &       &       &       &       &       & (4.70) & (-7.21) & (-5.56) & (2.98) & (-7.04) & (-5.75) & (2.17) & (4.32) \\
				RKT   &       &       &       &       &       &       &       &       & 0.31  & 0.44  & 1.37  & 0.40  & 1.01  & 1.00  & 0.95  & 0.97 \\
				&       &       &       &       &       &       &       &       & (0.61) & (0.57) & (2.13) & (0.58) & (1.79) & (1.83) & (1.91) & (1.93) \\
				Adjusted Rsquares & 0.0065  & 0.0034  & 0.0036  & 0.0084  & 0.0005  & 0.0020  & 0.0026  & 0.0017  & 0.0220  & 0.0191  & 0.0197  & 0.0231  & 0.0173  & 0.0181  & 0.0185  & 0.0181  \\
				\bottomrule
			\end{tabular}
			\begin{tablenotes}	
				\item{}
				\item {\large $^*\>$ Notes: See notes to Tables 1 and 5. This table reports results for cross-sectional Fama-MacBeth regressions, based on model \ref{eq13}), of future weekly returns on various realized measures.  The two panels rely on different jump truncation levels discussed in the footnote to Table 2.  Regression are carried out at the end of each Tuesday, and time series averages of the coefficient estimates ($\frac{1}{T}\sum_{t=1}^T \widehat{\gamma}_{j,t}$) are reported, along with Newey-West $t$-statistics (in parentheses). For complete details, see Section 4. }
			\end{tablenotes}
	\end{threeparttable}}
\end{table}
\newpage
\noindent Table 17: Fama-MacBeth Cross-Sectional Regressions with Control Variables

% Table generated by Excel2LaTeX from sheet 't4_15t80'
\begin{table}[htbp!]
	\centering
	%\caption*{\normalfont Panel A: Regressions Based on Truncation Level $\alpha_n^{(1)}$}
	\resizebox{1.04\textwidth}{!}{
		\begin{tabular}{lcccccccccccccccc}
			\multicolumn{17}{l}{Panel A: Regressions Based on Truncation Level $\alpha_n^{1}$}\\
			&      &      &      &      &      &  & & &
			&       &     &      &      &      &      & 
			\\
			\toprule
			& I     & II    & III   & IV    & V     & VI    & VII   & VIII  & IX    & X     & XI    & XII   & XIII  & XIV   & XV    & XVI \\
			Intercept & 100.35 & 13.44 & 95.10 & 97.37 & 98.02 & 98.04 & 98.98 & 98.50 & 102.54 & 101.11 & 92.16 & 98.39 & 98.43 & 98.70 & 100.84 & 100.27 \\
			& (4.13) & (2.08) & (5.67) & (4.08) & (5.61) & (5.69) & (5.65) & (5.61) & (5.76) & (6.52) & (5.79) & (5.44) & (6.64) & (6.63) & (6.97) & (6.92) \\
			RVJP  & -23.89 &       &       &       &       &       &       &       & -56.43 &       &       &       &       &       &       &  \\
			& (-1.91) &       &       &       &       &       &       &       & (-3.12) & \textcolor[rgb]{ 1,  0,  0}{} & \textcolor[rgb]{ 1,  0,  0}{} & \textcolor[rgb]{ 1,  0,  0}{} & \textcolor[rgb]{ 1,  0,  0}{} & \textcolor[rgb]{ 1,  0,  0}{} & \textcolor[rgb]{ 1,  0,  0}{} & \textcolor[rgb]{ 1,  0,  0}{} \\
			RVJN  & 24.44 &       &       &       &       &       &       &       & 55.77 &       & \textcolor[rgb]{ 1,  0,  0}{} & \textcolor[rgb]{ 1,  0,  0}{} & \textcolor[rgb]{ 1,  0,  0}{} & \textcolor[rgb]{ 1,  0,  0}{} & \textcolor[rgb]{ 1,  0,  0}{} & \textcolor[rgb]{ 1,  0,  0}{} \\
			& (3.05) &       &       &       &       &       &       &       & (2.73) & \textcolor[rgb]{ 1,  0,  0}{} & \textcolor[rgb]{ 1,  0,  0}{} & \textcolor[rgb]{ 1,  0,  0}{} & \textcolor[rgb]{ 1,  0,  0}{} & \textcolor[rgb]{ 1,  0,  0}{} & \textcolor[rgb]{ 1,  0,  0}{} & \textcolor[rgb]{ 1,  0,  0}{} \\
			RVLJP &       & -23.31 &       & -19.99 &       &       &       &       &       & -18.25 &       & -56.88 & \textcolor[rgb]{ 1,  0,  0}{} & \textcolor[rgb]{ 1,  0,  0}{} & \textcolor[rgb]{ 1,  0,  0}{} & \textcolor[rgb]{ 1,  0,  0}{} \\
			& \textcolor[rgb]{ 1,  0,  0}{} & (-2.80) &       & (-1.60) &       &       &       &       &       & (-1.41) &       & (-2.88) & \textcolor[rgb]{ 1,  0,  0}{} & \textcolor[rgb]{ 1,  0,  0}{} & \textcolor[rgb]{ 1,  0,  0}{} & \textcolor[rgb]{ 1,  0,  0}{} \\
			RVLJN &       & 13.44 &       & 21.76 &       &       &       &       &       & 5.61  &       & 53.97 & \textcolor[rgb]{ 1,  0,  0}{} & \textcolor[rgb]{ 1,  0,  0}{} & \textcolor[rgb]{ 1,  0,  0}{} &  \\
			& \textcolor[rgb]{ 1,  0,  0}{} & (2.08) &       & (2.71) &       &       &       &       &       & (0.46) &       & (2.46) & \textcolor[rgb]{ 1,  0,  0}{} & \textcolor[rgb]{ 1,  0,  0}{} & \textcolor[rgb]{ 1,  0,  0}{} & \textcolor[rgb]{ 1,  0,  0}{} \\
			RVSJP &       &       & -28.52 & -31.52 &       &       &       &       &       &       & -21.88 & -52.53 & \textcolor[rgb]{ 1,  0,  0}{} & \textcolor[rgb]{ 1,  0,  0}{} & \textcolor[rgb]{ 1,  0,  0}{} & \textcolor[rgb]{ 1,  0,  0}{} \\
			& \textcolor[rgb]{ 1,  0,  0}{} & \textcolor[rgb]{ 1,  0,  0}{} & (-2.40) & (-1.91) &       &       &       &       &       &       & (-1.73) & (-2.87) & \textcolor[rgb]{ 1,  0,  0}{} & \textcolor[rgb]{ 1,  0,  0}{} & \textcolor[rgb]{ 1,  0,  0}{} & \textcolor[rgb]{ 1,  0,  0}{} \\
			RVSJN &       &       & 52.78 & 56.41 &       &       &       &       &       &       & 46.77 & 71.19 &       &       &       &  \\
			& \textcolor[rgb]{ 1,  0,  0}{} & \textcolor[rgb]{ 1,  0,  0}{} & (3.68) & (3.66) &       &       &       &       &       &       & (3.18) & (3.19) & \textcolor[rgb]{ 1,  0,  0}{} & \textcolor[rgb]{ 1,  0,  0}{} & \textcolor[rgb]{ 1,  0,  0}{} & \textcolor[rgb]{ 1,  0,  0}{} \\
			SRVLJ &       &       &       &       & -17.56 &       & -20.26 &       &       &       &       &       & -10.47 &       & -55.17 &  \\
			& \textcolor[rgb]{ 1,  0,  0}{} & \textcolor[rgb]{ 1,  0,  0}{} & \textcolor[rgb]{ 1,  0,  0}{} & \textcolor[rgb]{ 1,  0,  0}{} & (-2.87) &       & (-3.17) &       &       &       &       &       & (-0.84) &       & (-2.49) & \textcolor[rgb]{ 1,  0,  0}{} \\
			SRVSJ &       &       &       &       &       & -35.62 & -41.89 &       &       &       &       &       &       & -29.11 & -60.56 &  \\
			& \textcolor[rgb]{ 1,  0,  0}{} & \textcolor[rgb]{ 1,  0,  0}{} & \textcolor[rgb]{ 1,  0,  0}{} & \textcolor[rgb]{ 1,  0,  0}{} & \textcolor[rgb]{ 1,  0,  0}{} & (-3.27) & (-3.64) &       &       &       &       &       &       & (-2.72) & (-3.30) & \textcolor[rgb]{ 1,  0,  0}{} \\
			SRVJ  &       &       &       &       &       &       &       & -22.90 &       &       &       &       &       &       &       & -54.71 \\
			& \textcolor[rgb]{ 1,  0,  0}{} & \textcolor[rgb]{ 1,  0,  0}{} & \textcolor[rgb]{ 1,  0,  0}{} & \textcolor[rgb]{ 1,  0,  0}{} & \textcolor[rgb]{ 1,  0,  0}{} & \textcolor[rgb]{ 1,  0,  0}{} & \textcolor[rgb]{ 1,  0,  0}{} & (-3.72) &       &       &       &       &       &       &       & (-3.03) \\
			RVOL  &       &       &       &       &       &       &       &       & -1.07 & -0.64 & -0.52 & -1.07 & -0.46 & -0.51 & -0.81 & -0.84 \\
			&       &       &       &       &       &       &       &       & (-0.20) & (-0.12) & (-0.10) & (-0.20) & (-0.09) & (-0.10) & (-0.15) & (-0.16) \\
			RSK   &       &       &       &       &       &       &       &       & 7.12  & -0.92 & -2.79 & 7.26  & -1.07 & -2.91 & 7.40  & 6.93 \\
			& \textcolor[rgb]{ 1,  0,  0}{} & \textcolor[rgb]{ 1,  0,  0}{} & \textcolor[rgb]{ 1,  0,  0}{} & \textcolor[rgb]{ 1,  0,  0}{} & \textcolor[rgb]{ 1,  0,  0}{} & \textcolor[rgb]{ 1,  0,  0}{} & \textcolor[rgb]{ 1,  0,  0}{} & \textcolor[rgb]{ 1,  0,  0}{} & (1.80) & (-0.36) & (-2.67) & (1.70) & (-0.38) & (-2.78) & (1.54) & (1.76) \\
			RKT   &       &       &       &       &       &       &       &       & -0.04 & 0.25  & 0.33  & 0.23  & -0.03 & 0.00  & -0.07 & -0.05 \\
			& \textcolor[rgb]{ 1,  0,  0}{} & \textcolor[rgb]{ 1,  0,  0}{} & \textcolor[rgb]{ 1,  0,  0}{} & \textcolor[rgb]{ 1,  0,  0}{} & \textcolor[rgb]{ 1,  0,  0}{} & \textcolor[rgb]{ 1,  0,  0}{} & \textcolor[rgb]{ 1,  0,  0}{} & \textcolor[rgb]{ 1,  0,  0}{} & (-0.13) & (0.56) & (0.67) & (0.58) & (-0.07) & (-0.01) & (-0.23) & (-0.15) \\
			BETA  & -7.75 & -7.85 & -7.91 & -7.86 & -7.66 & -7.84 & -7.80 & -7.73 & -7.59 & -7.66 & -7.56 & -7.63 & -7.62 & -7.66 & -7.74 & -7.64 \\
			& (-1.36) & (-1.36) & (-1.35) & (-1.38) & (-1.30) & (-1.33) & (-1.33) & (-1.31) & (-1.35) & (-1.35) & (-1.32) & (-1.36) & (-1.33) & (-1.34) & (-1.35) & (-1.33) \\
			log(Size) & -14.96 & -14.72 & -14.54 & -14.84 & -14.67 & -14.70 & -14.63 & -14.62 & -14.95 & -14.75 & -14.76 & -14.88 & -14.75 & -14.78 & -14.70 & -14.71 \\
			& (-5.31) & (-5.31) & (-5.20) & (-5.38) & (-5.19) & (-5.15) & (-5.18) & (-5.17) & (-5.74) & (-5.76) & (-5.83) & (-5.82) & (-5.71) & (-5.72) & (-5.70) & (-5.72) \\
			BEME & -0.79 & -0.80 & -0.68 & -0.84 & -0.64 & -0.59 & -0.59 & -0.61 & -1.03 & -1.09 & -1.05 & -1.11 & -0.94 & -0.93 & -0.90 & -0.92 \\
			& (-0.39) & (-0.39) & (-0.33) & (-0.41) & (-0.31) & (-0.29) & (-0.29) & (-0.30) & (-0.51) & (-0.54) & (-0.52) & (-0.55) & (-0.46) & (-0.46) & (-0.45) & (-0.46) \\
			MOM   & 0.00  & 0.00  & 0.00  & 0.00  & 0.00  & 0.00  & 0.00  & 0.00  & 0.00  & 0.00  & 0.00  & 0.00  & 0.00  & 0.00  & 0.00  & 0.00 \\
			& (0.98) & (0.99) & (0.96) & (1.01) & (0.94) & (0.92) & (0.93) & (0.94) & (1.30) & (1.28) & (1.28) & (1.32) & (1.27) & (1.26) & (1.27) & (1.27) \\
			REV   & -0.01 & -0.01 & -0.01 & -0.01 & -0.02 & -0.01 & -0.01 & -0.01 & -0.01 & -0.01 & -0.01 & -0.01 & -0.01 & -0.01 & -0.01 & -0.01 \\
			& (-5.76) & (-5.88) & (-5.78) & (-5.64) & (-5.89) & (-5.81) & (-5.66) & (-5.76) & (-5.88) & (-6.10) & (-5.97) & (-5.83) & (-6.14) & (-5.98) & (-5.91) & (-5.96) \\
			IVOL  & -305.41 & -299.68 & -298.50 & -311.03 & -298.25 & -294.19 & -304.48 & -301.76 & -310.86 & -299.03 & -306.51 & -315.02 & -297.68 & -300.28 & -308.89 & -307.89 \\
			& (-2.21) & (-2.18) & (-2.19) & (-2.26) & (-2.18) & (-2.16) & (-2.21) & (-2.19) & (-2.21) & (-2.09) & (-2.16) & (-2.27) & (-2.06) & (-2.07) & (-2.20) & (-2.19) \\
			CSK   & -7.26 & -8.26 & -8.16 & -6.98 & -8.34 & -8.22 & -7.36 & -7.69 & -6.93 & -7.75 & -7.31 & -6.78 & -7.82 & -7.29 & -7.20 & -7.46 \\
			& (-1.69) & (-1.91) & (-1.90) & (-1.63) & (-1.93) & (-1.91) & (-1.71) & (-1.78) & (-1.66) & (-1.85) & (-1.76) & (-1.63) & (-1.87) & (-1.75) & (-1.72) & (-1.78) \\
			CKT   & 2.24  & 2.32  & 2.23  & 2.31  & 2.25  & 2.19  & 2.31  & 2.29  & 2.17  & 2.20  & 2.08  & 2.18  & 2.18  & 2.14  & 2.24  & 2.21 \\
			& (1.13) & (1.16) & (1.12) & (1.18) & (1.12) & (1.10) & (1.15) & (1.13) & (1.13) & (1.13) & (1.07) & (1.14) & (1.11) & (1.10) & (1.15) & (1.14) \\
			MAX   & -0.03 & -0.03 & -0.03 & -0.03 & -0.03 & -0.03 & -0.03 & -0.03 & -0.03 & -0.03 & -0.03 & -0.03 & -0.03 & -0.03 & -0.03 & -0.03 \\
			& (-5.49) & (-5.75) & (-5.66) & (-5.47) & (-5.75) & (-5.66) & (-5.70) & (-5.74) & (-5.69) & (-6.22) & (-5.89) & (-5.79) & (-6.10) & (-5.96) & (-5.83) & (-5.76) \\
			MIN   & -0.02 & -0.02 & -0.02 & -0.02 & -0.02 & -0.02 & -0.02 & -0.02 & -0.02 & -0.02 & -0.02 & -0.02 & -0.02 & -0.02 & -0.02 & -0.02 \\
			& (-2.63) & (-2.83) & (-2.98) & (-2.64) & (-2.86) & (-2.98) & (-2.66) & (-2.68) & (-2.63) & (-2.74) & (-2.67) & (-2.62) & (-2.75) & (-2.67) & (-2.61) & (-2.62) \\
			ILLIQ & -7.84 & -7.41 & -7.69 & -7.69 & -7.87 & -7.92 & -7.79 & -7.81 & -7.79 & -7.60 & -8.12 & -7.74 & -8.02 & -8.06 & -7.79 & -7.85 \\
			& (-5.23) & (-5.05) & (-5.05) & (-5.16) & (-5.13) & (-5.11) & (-5.09) & (-5.10) & (-4.70) & (-4.59) & (-4.89) & (-4.70) & (-4.75) & (-4.80) & (-4.60) & (-4.65) \\
			Adjusted Rsquares & 0.0600 & 0.0596 & 0.0595 & 0.0609 & 0.0588 & 0.0589 & 0.0592 & 0.0590 & 0.0636 & 0.0631 & 0.0632 & 0.0643 & 0.0625 & 0.0627 & 0.0629 & 0.0628 \\
			\bottomrule
	\end{tabular}}
\end{table}	
\newpage
\noindent Table 17 (Continued)$^*\>$	
\begin{table}[htbp!]
	\centering
	%\caption*{\normalfont Panel B: Regressions Based on Truncation Level $\alpha_n^{(2)}$}
	\resizebox{1.04\textwidth}{!}{	
		\begin{threeparttable}
			\begin{tabular}{lcccccccccccccccc}
				\multicolumn{17}{l}{Panel B: Regressions Based on Truncation Level $\alpha_n^{2}$}\\
				&      &      &      &      &      &  & & &
				&       &     &      &      &      &      & 
				\\	
				\toprule
				& I     & II    & III   & IV    & V     & VI    & VII   & VIII  & IX    & X     & XI    & XII   & XIII  & XIV   & XV    & XVI \\
				Intercept & 100.35 & 17.23 & 97.28 & 100.53 & 97.86 & 98.18 & 98.92 & 98.50 & 102.54 & 96.35 & 98.00 & 102.66 & 97.73 & 98.55 & 100.36 & 100.27 \\
				& (4.13) & (2.66) & (5.65) & (4.19) & (5.64) & (5.67) & (5.65) & (5.61) & (5.76) & (6.34) & (5.55) & (5.50) & (6.52) & (6.61) & (6.91) & (6.92) \\
				RVJP  & -23.89 &       &       &       &       &       &       &       & -56.43 &       &       &       &       &       &       &  \\
				& (-1.91) &       &       &       &       &       &       &       & (-3.12) & \textcolor[rgb]{ 1,  0,  0}{} & \textcolor[rgb]{ 1,  0,  0}{} & \textcolor[rgb]{ 1,  0,  0}{} & \textcolor[rgb]{ 1,  0,  0}{} & \textcolor[rgb]{ 1,  0,  0}{} & \textcolor[rgb]{ 1,  0,  0}{} & \textcolor[rgb]{ 1,  0,  0}{} \\
				RVJN  & 24.44 &       &       &       &       &       &       &       & 55.77 &       & \textcolor[rgb]{ 1,  0,  0}{} & \textcolor[rgb]{ 1,  0,  0}{} & \textcolor[rgb]{ 1,  0,  0}{} & \textcolor[rgb]{ 1,  0,  0}{} & \textcolor[rgb]{ 1,  0,  0}{} & \textcolor[rgb]{ 1,  0,  0}{} \\
				& (3.05) &       &       &       &       &       &       &       & (2.73) & \textcolor[rgb]{ 1,  0,  0}{} & \textcolor[rgb]{ 1,  0,  0}{} & \textcolor[rgb]{ 1,  0,  0}{} & \textcolor[rgb]{ 1,  0,  0}{} & \textcolor[rgb]{ 1,  0,  0}{} & \textcolor[rgb]{ 1,  0,  0}{} & \textcolor[rgb]{ 1,  0,  0}{} \\
				RVLJP &       & -17.17 &       & -20.22 &       &       &       &       &       & 0.41  &       & -61.23 & \textcolor[rgb]{ 1,  0,  0}{} & \textcolor[rgb]{ 1,  0,  0}{} & \textcolor[rgb]{ 1,  0,  0}{} & \textcolor[rgb]{ 1,  0,  0}{} \\
				& \textcolor[rgb]{ 1,  0,  0}{} & (-2.07) &       & (-1.58) &       &       &       &       &       & (0.03) &       & (-2.43) & \textcolor[rgb]{ 1,  0,  0}{} & \textcolor[rgb]{ 1,  0,  0}{} & \textcolor[rgb]{ 1,  0,  0}{} & \textcolor[rgb]{ 1,  0,  0}{} \\
				RVLJN &       & 17.23 &       & 20.03 &       &       &       &       &       & 0.66  &       & 61.46 & \textcolor[rgb]{ 1,  0,  0}{} & \textcolor[rgb]{ 1,  0,  0}{} & \textcolor[rgb]{ 1,  0,  0}{} &  \\
				& \textcolor[rgb]{ 1,  0,  0}{} & (2.66) &       & (2.43) &       &       &       &       &       & (0.06) &       & (2.39) & \textcolor[rgb]{ 1,  0,  0}{} & \textcolor[rgb]{ 1,  0,  0}{} & \textcolor[rgb]{ 1,  0,  0}{} & \textcolor[rgb]{ 1,  0,  0}{} \\
				RVSJP &       &       & -32.40 & -35.84 &       &       &       &       &       &       & -26.75 & -62.17 & \textcolor[rgb]{ 1,  0,  0}{} & \textcolor[rgb]{ 1,  0,  0}{} & \textcolor[rgb]{ 1,  0,  0}{} & \textcolor[rgb]{ 1,  0,  0}{} \\
				& \textcolor[rgb]{ 1,  0,  0}{} & \textcolor[rgb]{ 1,  0,  0}{} & (-3.42) & (-2.50) &       &       &       &       &       &       & (-2.58) & (-3.51) & \textcolor[rgb]{ 1,  0,  0}{} & \textcolor[rgb]{ 1,  0,  0}{} & \textcolor[rgb]{ 1,  0,  0}{} & \textcolor[rgb]{ 1,  0,  0}{} \\
				RVSJN &       &       & 34.73 & 36.47 &       &       &       &       &       &       & 25.34 & 57.65 &       &       &       &  \\
				& \textcolor[rgb]{ 1,  0,  0}{} & \textcolor[rgb]{ 1,  0,  0}{} & (3.27) & (3.10) &       &       &       &       &       &       & (2.23) & (2.60) & \textcolor[rgb]{ 1,  0,  0}{} & \textcolor[rgb]{ 1,  0,  0}{} & \textcolor[rgb]{ 1,  0,  0}{} & \textcolor[rgb]{ 1,  0,  0}{} \\
				SRVLJ &       &       &       &       & -16.05 &       & -18.83 &       &       &       &       &       & 1.45  &       & -58.58 &  \\
				& \textcolor[rgb]{ 1,  0,  0}{} & \textcolor[rgb]{ 1,  0,  0}{} & \textcolor[rgb]{ 1,  0,  0}{} & \textcolor[rgb]{ 1,  0,  0}{} & (-2.52) &       & (-2.81) &       &       &       &       &       & (0.13) &       & (-2.20) & \textcolor[rgb]{ 1,  0,  0}{} \\
				SRVSJ &       &       &       &       &       & -32.14 & -35.76 &       &       &       &       &       &       & -24.15 & -58.56 &  \\
				& \textcolor[rgb]{ 1,  0,  0}{} & \textcolor[rgb]{ 1,  0,  0}{} & \textcolor[rgb]{ 1,  0,  0}{} & \textcolor[rgb]{ 1,  0,  0}{} & \textcolor[rgb]{ 1,  0,  0}{} & (-3.70) & (-3.91) &       &       &       &       &       &       & (-2.94) & (-3.07) & \textcolor[rgb]{ 1,  0,  0}{} \\
				SRVJ  &       &       &       &       &       &       &       & -22.90 &       &       &       &       &       &       &       & -54.71 \\
				& \textcolor[rgb]{ 1,  0,  0}{} & \textcolor[rgb]{ 1,  0,  0}{} & \textcolor[rgb]{ 1,  0,  0}{} & \textcolor[rgb]{ 1,  0,  0}{} & \textcolor[rgb]{ 1,  0,  0}{} & \textcolor[rgb]{ 1,  0,  0}{} & \textcolor[rgb]{ 1,  0,  0}{} & (-3.72) &       &       &       &       &       &       &       & (-3.03) \\
				RVOL  &       &       &       &       &       &       &       &       & -1.07 & -0.31 & -0.71 & -1.11 & -0.33 & -0.50 & -0.74 & -0.84 \\
				&       &       &       &       &       &       &       &       & (-0.20) & (-0.06) & (-0.14) & (-0.21) & (-0.06) & (-0.10) & (-0.14) & (-0.16) \\
				RSK   &       &       &       &       &       &       &       &       & 7.12  & -3.06 & -2.44 & 8.21  & -3.29 & -2.54 & 7.88  & 6.93 \\
				& \textcolor[rgb]{ 1,  0,  0}{} & \textcolor[rgb]{ 1,  0,  0}{} & \textcolor[rgb]{ 1,  0,  0}{} & \textcolor[rgb]{ 1,  0,  0}{} & \textcolor[rgb]{ 1,  0,  0}{} & \textcolor[rgb]{ 1,  0,  0}{} & \textcolor[rgb]{ 1,  0,  0}{} & \textcolor[rgb]{ 1,  0,  0}{} & (1.80) & (-1.42) & (-2.32) & (1.63) & (-1.40) & (-2.43) & (1.44) & (1.76) \\
				RKT   &       &       &       &       &       &       &       &       & -0.04 & 0.08  & 0.09  & -0.01 & 0.00  & 0.01  & -0.06 & -0.05 \\
				& \textcolor[rgb]{ 1,  0,  0}{} & \textcolor[rgb]{ 1,  0,  0}{} & \textcolor[rgb]{ 1,  0,  0}{} & \textcolor[rgb]{ 1,  0,  0}{} & \textcolor[rgb]{ 1,  0,  0}{} & \textcolor[rgb]{ 1,  0,  0}{} & \textcolor[rgb]{ 1,  0,  0}{} & \textcolor[rgb]{ 1,  0,  0}{} & (-0.13) & (0.15) & (0.17) & (-0.03) & (0.01) & (0.02) & (-0.19) & (-0.15) \\
				BETA  & -7.75 & -7.46 & -7.78 & -7.76 & -7.67 & -7.88 & -7.84 & -7.73 & -7.59 & -7.26 & -7.62 & -7.51 & -7.59 & -7.69 & -7.70 & -7.64 \\
				& (-1.36) & (-1.29) & (-1.32) & (-1.36) & (-1.30) & (-1.34) & (-1.33) & (-1.31) & (-1.35) & (-1.27) & (-1.33) & (-1.34) & (-1.33) & (-1.34) & (-1.35) & (-1.33) \\
				log(Size) & -14.96 & -14.88 & -14.82 & -15.14 & -14.73 & -14.68 & -14.66 & -14.62 & -14.95 & -14.83 & -15.06 & -15.10 & -14.76 & -14.79 & -14.67 & -14.71 \\
				& (-5.31) & (-5.37) & (-5.24) & (-5.44) & (-5.21) & (-5.15) & (-5.20) & (-5.17) & (-5.74) & (-5.76) & (-5.86) & (-5.82) & (-5.70) & (-5.74) & (-5.69) & (-5.72) \\
				BEME & -0.79 & -0.72 & -0.58 & -0.76 & -0.66 & -0.58 & -0.60 & -0.61 & -1.03 & -0.96 & -0.98 & -1.01 & -0.93 & -0.93 & -0.89 & -0.92 \\
				& (-0.39) & (-0.35) & (-0.28) & (-0.37) & (-0.32) & (-0.28) & (-0.29) & (-0.30) & (-0.51) & (-0.47) & (-0.48) & (-0.50) & (-0.46) & (-0.46) & (-0.44) & (-0.46) \\
				MOM   & 0.00  & 0.00  & 0.00  & 0.00  & 0.00  & 0.00  & 0.00  & 0.00  & 0.00  & 0.00  & 0.00  & 0.00  & 0.00  & 0.00  & 0.00  & 0.00 \\
				& (0.98) & (0.96) & (0.92) & (0.97) & (0.93) & (0.93) & (0.94) & (0.94) & (1.30) & (1.27) & (1.25) & (1.28) & (1.26) & (1.26) & (1.27) & (1.27) \\
				REV   & -0.01 & -0.02 & -0.01 & -0.01 & -0.02 & -0.01 & -0.01 & -0.01 & -0.01 & -0.01 & -0.01 & -0.01 & -0.01 & -0.01 & -0.01 & -0.01 \\
				& (-5.76) & (-5.93) & (-5.75) & (-5.67) & (-5.93) & (-5.77) & (-5.67) & (-5.76) & (-5.88) & (-6.10) & (-5.98) & (-5.86) & (-6.13) & (-5.98) & (-5.93) & (-5.96) \\
				IVOL  & -305.41 & -293.05 & -298.19 & -307.15 & -295.53 & -295.56 & -303.20 & -301.76 & -310.86 & -295.34 & -305.79 & -313.21 & -296.01 & -299.71 & -308.08 & -307.89 \\
				& (-2.21) & (-2.14) & (-2.18) & (-2.22) & (-2.17) & (-2.16) & (-2.20) & (-2.19) & (-2.21) & (-2.05) & (-2.13) & (-2.25) & (-2.03) & (-2.06) & (-2.20) & (-2.19) \\
				CSK   & -7.26 & -8.49 & -8.00 & -7.09 & -8.57 & -8.04 & -7.46 & -7.69 & -6.93 & -7.72 & -7.36 & -6.84 & -7.68 & -7.34 & -7.23 & -7.46 \\
				& (-1.69) & (-1.97) & (-1.86) & (-1.66) & (-1.99) & (-1.86) & (-1.73) & (-1.78) & (-1.66) & (-1.85) & (-1.77) & (-1.65) & (-1.84) & (-1.76) & (-1.73) & (-1.78) \\
				CKT   & 2.24  & 2.25  & 2.23  & 2.30  & 2.24  & 2.24  & 2.34  & 2.29  & 2.17  & 2.15  & 2.09  & 2.17  & 2.17  & 2.18  & 2.26  & 2.21 \\
				& (1.13) & (1.12) & (1.11) & (1.17) & (1.11) & (1.12) & (1.16) & (1.13) & (1.13) & (1.10) & (1.08) & (1.13) & (1.10) & (1.12) & (1.16) & (1.14) \\
				MAX   & -0.03 & -0.03 & -0.03 & -0.03 & -0.03 & -0.03 & -0.03 & -0.03 & -0.03 & -0.03 & -0.03 & -0.03 & -0.03 & -0.03 & -0.03 & -0.03 \\
				& (-5.49) & (-5.75) & (-5.65) & (-5.50) & (-5.75) & (-5.68) & (-5.71) & (-5.74) & (-5.69) & (-6.16) & (-5.79) & (-5.76) & (-6.12) & (-5.96) & (-5.85) & (-5.76) \\
				MIN   & -0.02 & -0.02 & -0.02 & -0.02 & -0.02 & -0.02 & -0.02 & -0.02 & -0.02 & -0.02 & -0.02 & -0.02 & -0.02 & -0.02 & -0.02 & -0.02 \\
				& (-2.63) & (-2.94) & (-2.89) & (-2.61) & (-2.93) & (-2.90) & (-2.65) & (-2.68) & (-2.63) & (-2.77) & (-2.64) & (-2.62) & (-2.72) & (-2.66) & (-2.61) & (-2.62) \\
				ILLIQ & -7.84 & -8.02 & -8.01 & -7.97 & -7.94 & -7.90 & -7.83 & -7.81 & -7.79 & -8.26 & -8.29 & -7.95 & -8.10 & -8.09 & -7.80 & -7.85 \\
				& (-5.23) & (-5.36) & (-5.17) & (-5.31) & (-5.18) & (-5.09) & (-5.12) & (-5.10) & (-4.70) & (-4.88) & (-4.98) & (-4.78) & (-4.80) & (-4.81) & (-4.60) & (-4.65) \\
				Adjusted Rsquares & 0.0600 & 0.0596 & 0.0595 & 0.0608 & 0.0588 & 0.0590 & 0.0592 & 0.0590 & 0.0636 & 0.0630 & 0.0632 & 0.0642 & 0.0625 & 0.0626 & 0.0629 & 0.0628 \\
				\bottomrule
			\end{tabular}
			\begin{tablenotes}	
				\item{}
				\item {\large $^*\>$ Notes: See notes to Table 16.}
			\end{tablenotes}
	\end{threeparttable}}
\end{table}


\newpage
\noindent Figure 1: Unconditional Distributions of Realized Measures$^*\>$  

\begin{figure}[!htb]
	\minipage{0.49\textwidth}
	\caption*{\centering \normalfont Panel A: SRVJ Kernel Density Estimate }
	\includegraphics[width=\linewidth]{density_srvj.jpg}
	\endminipage\hfill
	\minipage{0.49\textwidth}
	\caption*{\centering \normalfont Panel B: RSK Kernel Density Estimate}
	\includegraphics[width=\linewidth]{density_rsk.jpg}
	\endminipage
\end{figure}
\begin{figure}[!htb]
	\minipage{0.49\textwidth}
	\caption*{\centering \normalfont Panel C: SRVLJ Kernel Density Estimate }
	\includegraphics[width=\linewidth]{density_srvlj.jpg}
	\endminipage\hfill
	\minipage{0.49\textwidth}
	\caption*{\centering \normalfont Panel D: SRVSJ Kernel Density Estimate}
	\includegraphics[width=\linewidth]{density_srvsj.jpg}
	\endminipage
\end{figure}
\begin{figure}[!htb]
	\minipage{0.49\textwidth}
	\caption*{\centering \normalfont Panel E: RKT Kernel Density Estimate }
	\includegraphics[width=\linewidth,height=0.21\textheight]{density_rkt.jpg}
	\endminipage\hfill
	\minipage{0.49\textwidth}
	\caption*{\centering \normalfont Panel F: RVOL Kernel Density Estimate}
	\includegraphics[width=\linewidth, height=0.21\textheight ]{density_rvol.jpg}
	\endminipage
\end{figure}
%\raggedright 
\justify
$^*\>$ Notes: See notes to Table 1. Panels A-F display unconditional distribution kernel density estimates of various realized measures, for the cross-section of stock returns for the period January 1993 to December 2016. Signed large and small jump variations are constructed using truncation levels  $\alpha_n^{1}=4\sqrt{\frac{1}{t}\widehat{IV}_t}\triangle_n^{0.49}$. Distributions are similar when using $\alpha_n^{2}=5\sqrt{\frac{1}{t}\widehat{IV}_t}\triangle_n^{0.49}$.   
\newpage
\centering
Figure 2: Percentiles of Realized Measures  \\

\begin{figure}[!htb]
	\minipage{0.5\textwidth}
	\caption*{\centering \normalfont Panel A: Percentiles of SRVJ }
	\includegraphics[width=\linewidth]{per_srvj.jpg}
	\endminipage\hfill
	\minipage{0.5\textwidth}
	\caption*{\centering \normalfont Panel B: Percentiles of RSK}
	\includegraphics[width=\linewidth]{per_rsk.jpg}
	\endminipage
\end{figure}
\begin{figure}[!htb]
	\minipage{0.5\textwidth}
	\caption*{\centering \normalfont Panel C: Percentiles of RKT }
	\includegraphics[width=\linewidth]{per_rkt.jpg}
	\endminipage\hfill
	\minipage{0.5\textwidth}
	\caption*{\centering \normalfont Panel D: Percentiles of RVOL}
	\includegraphics[width=\linewidth]{per_rvol.jpg}
	\endminipage
\end{figure}

\newpage 
\noindent Figure 2 (Continued)$^*\>$ \\
\begin{figure}[!htb]
	\minipage{0.5\textwidth}
	\caption*{\centering \normalfont Panel E: Percentiles of SRVLJ Based on $\alpha_n^{1}$}
	\includegraphics[width=\linewidth]{srvlj_alpha1.png}
	\endminipage\hfill
	\minipage{0.5\textwidth}
	\caption*{\centering \normalfont Panel F: Percentiles of SRVLJ Based on $\alpha_n^{2}$}
	\includegraphics[width=\linewidth]{srvlj_alpha2.png}
	\endminipage
\end{figure}
\begin{figure}[!htb]
	\minipage{0.5\textwidth}
	\caption*{\centering \normalfont Panel G: Percentiles of SRVSJ Based on $\alpha_n^{1}$}
	\includegraphics[width=\linewidth]{srvsj_alpha1.png}
	\endminipage\hfill
	\minipage{0.5\textwidth}
	\caption*{\centering \normalfont Panel H: Percentiles of SRVSJ Based on $\alpha_n^{2}$}
	\includegraphics[width=\linewidth]{srvsj_alpha2.png}
	\endminipage
\end{figure}

%\raggedright
\justify
$^*\>$ Notes: See notes to Table 1. Panels A-H display 10-week moving averages of percentiles of realized measures, for the cross-section of stocks, for the period January 1993 to December 2016. Signed large and small jump variations are contructed based on two jump truncation levels $\alpha_n^{1}=4\sqrt{\frac{1}{t}\widehat{IV}_t}\triangle_n^{0.49},\>$ and $\alpha_n^{2}=5\sqrt{\frac{1}{t}\widehat{IV}_t}\triangle_n^{0.49}$.

\newpage
\centering
\noindent Figure 3: Cumulative Gains of Short-Long Portfolios$^*\>$\\

\begin{figure}[!htb]
	\centering
	\caption*{\centering \normalfont Panel A: Equal-Weighted Mean Return}
	\includegraphics[width=0.72\linewidth]{CR_EW_RSJ.png}
	
\end{figure}
\begin{figure}[!htb]
	\centering
	\caption*{\centering \normalfont Panel B: Value-Weighted Mean Return}
	\includegraphics[width=0.72\linewidth]{CR_VW_RSJ.png}
\end{figure}

%\raggedright 
\justify
$^*\>$ Notes: Panels A-B display cumulative gains of equal-weighted and value-weighted short-long portfolios constructed using SRVJ, SRVLJ, SRVSJ, and RSK (see Table 1 and Section 2 for a discussion of these measures). RSJ is the relative signed jump variation measure defined and analyzed in Bollerslev, Li, and Zhao (2017), who include the risk-free rate in all of their calculations, while we do not (refer to Bollerslev, Li, and Zhao (2017) for complete details). In all experiments, the initial investment, made on January 1993,  is \$1. Each portfolio is re-balanced and accumulated on a weekly basis, through  2016. Signed large and small jump variations used in the experiment reported on in this figure are constructed based on truncation level $\alpha_n^{2}=5\sqrt{\frac{1}{t}\widehat{IV}_t}\triangle_n^{0.49}$. See Section 4.2 for further discussion.
\newpage
\centering
Figure 4: Distribution of Stocks in Portfolios Formed Based on Stocks' Signed Jump Variations (SRVJ) and Industry Signed Jump Variations$^*\>$ \\
\bigskip
%\raggedright  
\normalfont Panel A: Average Distribution of Stocks Across Double-Sorted Portfolios
\begin{figure}[!ht]
	\centering
	\includegraphics[width=0.70\textwidth]{frac25_t5.png}	
	
\end{figure}\\
%\raggedright
\normalfont Panel B: Average Distribution of Market Capitalization Across Double-Sorted Portfolios

\begin{figure}[!ht]
	\centering
	\includegraphics[width=0.68\textwidth]{frac25_me_t5.png}
	
\end{figure}

%\raggedright 
\justify
$^*\>$ Notes: See notes to Table 13. The vertical values in Panels A and B represent time series average proportions of stocks and market capitalizations, across double sorted portfolios.  

\end{document}




















