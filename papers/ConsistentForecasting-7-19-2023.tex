%2multibyte Version: 5.50.0.2960 CodePage: 936

\documentclass[11pt]{article}
%%%%%%%%%%%%%%%%%%%%%%%%%%%%%%%%%%%%%%%%%%%%%%%%%%%%%%%%%%%%%%%%%%%%%%%%%%%%%%%%%%%%%%%%%%%%%%%%%%%%%%%%%%%%%%%%%%%%%%%%%%%%%%%%%%%%%%%%%%%%%%%%%%%%%%%%%%%%%%%%%%%%%%%%%%%%%%%%%%%%%%%%%%%%%%%%%%%%%%%%%%%%%%%%%%%%%%%%%%%%%%%%%%%%%%%%%%%%%%%%%%%%%%%%%%%%
\usepackage{amssymb}
\usepackage{amsfonts}
\usepackage{amsmath}
\usepackage{geometry}
\usepackage[onehalfspacing]{setspace}
\usepackage{setspace}

\setcounter{MaxMatrixCols}{10}
%TCIDATA{OutputFilter=LATEX.DLL}
%TCIDATA{Version=5.50.0.2960}
%TCIDATA{Codepage=936}
%TCIDATA{<META NAME="SaveForMode" CONTENT="1">}
%TCIDATA{BibliographyScheme=Manual}
%TCIDATA{Created=Sunday, July 18, 2004 16:10:34}
%TCIDATA{LastRevised=Wednesday, July 19, 2023 11:58:58}
%TCIDATA{<META NAME="GraphicsSave" CONTENT="32">}
%TCIDATA{<META NAME="DocumentShell" CONTENT="General\Blank Document">}
%TCIDATA{Language=American English}
%TCIDATA{CSTFile=article.cst}
%TCIDATA{PageSetup=72,72,72,72,0}
%TCIDATA{AllPages=
%F=36,\PARA{038<p type="texpara" tag="Body Text" >\hfill \thepage}
%}


\newtheorem{theorem}{Theorem}
\newtheorem{acknowledgement}[theorem]{Acknowledgement}
\newtheorem{algorithm}[theorem]{Algorithm}
\newtheorem{axiom}[theorem]{Axiom}
\newtheorem{case}[theorem]{Case}
\newtheorem{claim}[theorem]{Claim}
\newtheorem{conclusion}[theorem]{Conclusion}
\newtheorem{condition}[theorem]{Condition}
\newtheorem{conjecture}[theorem]{Conjecture}
\newtheorem{corollary}[theorem]{Corollary}
\newtheorem{criterion}[theorem]{Criterion}
\newtheorem{definition}[theorem]{Definition}
\newtheorem{example}[theorem]{Example}
\newtheorem{exercise}[theorem]{Exercise}
\newtheorem{lemma}[theorem]{Lemma}
\newtheorem{notation}[theorem]{Notation}
\newtheorem{problem}[theorem]{Problem}
\newtheorem{proposition}[theorem]{Proposition}
\newtheorem{remark}[theorem]{Remark}
\newtheorem{solution}[theorem]{Solution}
\newtheorem{summary}[theorem]{Summary}
\newenvironment{proof}[1][Proof]{\noindent\textbf{#1.} }{\ \rule{0.5em}{0.5em}}
\renewcommand{\baselinestretch}{1.0} 
\textwidth=6.8in
\textheight=8.7in
\oddsidemargin=0in
\evensidemargin=0in
\topmargin=0in
\baselineskip=10pt
\linespread{1.3}
\input{tcilatex}
\geometry{left=1in,right=1in,top=1.25in,bottom=1.25in}

\begin{document}


\begin{center}
{\Large {Consistent Factor Estimation and Forecasting in Factor-Augmented
VAR Models$^{\ast }$}}

\bigskip

John C. Chao$^{1},$ Yang Liu$^{2}$ and Norman R. Swanson$^{2}$

\medskip

$^{1}$University of Maryland and $^{2}$Rutgers University

\medskip

July 19, 2023

\medskip

preliminary and incomplete

\bigskip \bigskip

Abstract
\end{center}

\begin{spacing}{1.01}
\noindent In this paper we establish that conditional mean functions associated with $h$-step ahead 
forecasting equations implied by a factor augmented vector autoregressions (FAVARs) can be consistently estimated
in the case where factor pervasiveness does not hold. In particular,  we begin by stating a common assumption of 
factor pervasiveness in which all available predictor variables (excepting a negligible subset) 
load significantly on the underlying factors. We then establish that 
even when this assumption is relaxed, consistent factor estimation can be achieved
if one pre-screens the variables and successfully prunes out the irrelevant
ones. Furthermore, use of factors estimated in this manner
when constructing $h$-step ahead forecasting equations
implied by FAVAR models enables the consistent estimation of the
conditional mean function of said equations, and conditional mean functions constructed 
used our precedure are
consistently estimable in a wide range of situations, including cases
where violation of factor pervasiveness is such that consistent estimation
is precluded in the absence of variable pre-screening. 
\end{spacing}

\bigskip \bigskip \bigskip

\noindent \textit{Keywords: }Factor analysis, factor augmented vector
autoregression, forecasting, moderate deviation, principal components,
self-normalization, variable selection.

\medskip

\noindent \textit{JEL Classification: }C32, C33, C38, C52, C53, C55.

\bigskip \bigskip

\begin{spacing}{1.01}
\noindent $^{\ast }$\textit{Corresponding Author:} John C. Chao, Department of Economics, 7343 Preinkert
Drive, University of Maryland, chao@econ.umd.edu.

\medskip

\noindent Norman R. Swanson, Department of Economics, 9500 Hamilton Street, Rutgers University,
nswanson@econ.rutgers.edu. The authors are grateful to Simon Freyaldenhoven,
Yuan Liao, Minchul Shin, Xiye Yang, and seminar participants at the Federal
Reserve Bank of Pihadelphia for useful comments received on earlier versions
of this paper. Chao thanks the University of Maryland for research support.
\end{spacing}

\newpage

\noindent \noindent \setcounter{page}{2}

\section{Introduction}

\noindent \qquad In economics, three of the key areas in machine learning
that have drawn considerable attention in recent years include variable
selection, dimension reduction and shrinkage. One reason for this is the
availability of new high frequency and high dimensional datasets that are
being analyzed in areas ranging from targeted marketing and customer
segmentation to forecasting and maroeconomic policy making. This has in turn
led to numerous theoretical advances in the areas of estimation,
implementation, and inference using techinques such as the least absolute
shrinkage operator (lasso) and principal components analysis (PCA). In this
paper, we build on pathbreaking work due to Bai and Ng (2002), Stock and
Watson (2002a,b), Bai (2003), Forni, Hallin, Lippi, and Reichlin (2005), and
Bai and Ng (2008), in which methods for constructing forecasts based on
factor-augmented regression models are developed and analyzed. In
particular, we establish that latent factors that are critical to the
estimation of factor augmented vector autoregressions (FAVARs) can be
consistently estimated in cases where factor pervasiveness does not hold,
where by factor pervasivness we mean that (almost)\ all available predictors
load significantly on a set of factors that we wish to estimate. To do so,
we draw on results of Chao and Swanson (CS: 2022a), where a completely
consistent variable selection procedure useful for specifying FAVAR models
is developed. We then establish that the conditional mean of the infeasible $%
h$-step ahead forecasting equation implied by an FAVAR can be consistently
estimated.

As discussed above, a key assumption commonly used in the factor analysis
literature to show consistent factor estimation is that of factor
pervasiveness. This assumption presupposes that all available predictor
variables in a dataset, with the possible exception of a negligible number
of them, load significantly on the underlying factors. Needless to say, this
assumption may not be satisfied by many datasets that are available for
empirical research. Indeed, a likely scenario is that there is significant
underlying heterogeneity, so that some of the available variables are
relevant in the sense that they load significantly on the underlying
factors, whereas others are irrelevant, in the sense that they do not share
any common dynamic structure with each other or with the relevant variables
in the dataset. In this paper, we begin by establishing that, under failure
of factor pervasiveness in a stylized model with one factor, consistency
cannot be achieved, and indeed \textit{\ }$\widehat{f}_{t}\overset{p}{%
\rightarrow }0,$ as $N,T\rightarrow \infty $, where $f_{t}$ is a latent
factor, $N$ is the number of variables in the dataset being modelled, and $T$
is the number of time series observations. Findings such as this are the
impetus for the work of Chao and Swanson (CS: 2022a), where a variable
selection procedure is developed for pre-selecting relevant predictor
variables for use in the consistent estimation of latent factors in an FAVAR
model. Their variable selection procedure is based on the use of easy to
construct self normalized statistics measuring the covariation between
target variables to be predicted and possible predictor variables to be used
in factor estimation. CS (2022a) show that for their procedure, the
probability of Type I and Type II errors goes to zero, asymptotically,
implying that the procedure is completely consistent. This property turns
out to be important because if one tries to simply control the probability
of a Type I error at some predetermined level, which is the typical approach
used in multiple hypothesis testing, then one will not in general be able to
estimate factors consistently, even up to an invertible matrix
transformation. A main result of the current paper is to show that factors
estimated using predictor variables selected using the procedure of CS
(2022a)\ are consistent, up to a rotation. With these results in hand, we
then show that by using variables selected via our pre-screening procedure
to estimate the underlying factors, and then inserting these factor
estimates into $h$-step ahead forecasting equations implied by a FAVAR
model, we can consistently estimate the conditional mean function of the
said equations. Importantly, we argue that this result allows the
conditional mean function of a factor-augmented forecasting equation to be
consistently estimable in a wide range of situations, and in particular in
situations where there are violations of factor pervasiveness.

Finally, in order to illustrate the methods discussed in this paper, we
analyze a large dataset. This part of the paper is to be completed.

Some of the research reported here is related to the well-known supervised
principal components method proposed by Bair, Hastie, Paul, and Tibshirani
(2006). Additionally, our research is related to some interesting recent
work by Giglio, Xiu, and Zhang (2021), who propose a method for selecting
test assets, with the objective of estimating risk premia in a Fama-MacBeth
type framework. A crucial difference between the variable selection
procedure proposed in our paper and those proposed in these papers is that
we use a score statistic that is self-nomalized, whereas the aforementioned
papers do not make use of statistics that involve self-normalization. An
important advantage of self-normalized statistics is their ability to
accommodate a much wider range of possible tail behavior in the underlying
distributions, relative to their non-self-normalized counterparts. This
makes self-normalized statistics better suited for various types of economic
and financial applications, where the data are known not to exhibit the type
of exponentially decaying tail behavior assumed in much of the statistics
literature on high-dimensional models. In addition, the type of models
studied in Bair, Hastie, Paul, and Tibshirani (2006) and Giglio, Xiu, and
Zhang (2021) differ significantly from the FAVAR model studied here. In
particular, Bair, Hastie, Paul, and Tibshirani (2006) study a one-factor
model in an $i.i.d.$ Gaussian framework so that complications introduced by
dependence and non-normality of distribution are not considered in their
paper. Giglio, Xiu, and Zhang (2021) do make certain high-level assumptions
which may potentially accommodate some dependence both cross-sectionally and
intertemporally, but they do not consider the implications of variable
selection and factor estimation for forecasting, and the model that they
consider is very different from the type of dynamic vector time series model
studied here.

Our research is also closely related to the work of Bai and Ng (2021), who
provide results which show that factors can still be estimated consistently
in certain situations where the factor loadings are weaker than that implied
by the conventional pervasiveness assumption, although in such cases the
rate of convergence of the factor estimator is slower and additional
assumptions are needed. As discussed in the next section of this paper,
their factor consistency result relies on a key condition, and the
appropriateness of this condition depends on how severely the condition of
factor pervasiveness is violated, which is ultimately an empirical issue.%
\footnote{%
Various authors have documented cases in economics-related research where
empirical results suggest that the underlying factors may be quite weak, so
that the rate condition given in Bai and Ng (2021) may not be appropriate.
See, for example, the discussions in Jagannathan and Wang (1998), Kan and
Zhang (1999), Harding (2008), Kleibergen (2009), Ontaski (2012), Bryzgalova
(2016), Burnside (2016), Gospodinov, Kan, and Robotti (2017), Anatolyev and
Mikusheva (2021), and Freyaldenhoven (2021a,b).}

The rest of the paper is organized as follows. In Section 2 , we provide our
counterexample, stated formally as Theorem 2.1, which shows that a latent
factor may be inconsistently estimated when the standard assumption of
factor pervasiveness does not hold. In Section 3, we discuss the FAVAR
model, the variable selection procedure of CS (2022a), and the assumptions
that are required in the sequel. Section 4 gathers our theoretical results
on the consistent estimation of latent factors, up to an invertible matrix
transformation, as well as results on the consistent estimation of the $h$%
-step ahead predictor, based on the FAVAR model. Section 5 presents the
results of an empirical illustration where our forecasting approach is
compared with related approaches in the literature. Finally, Section 6
offers concluding remarks. Proofs of the main theorems and supporting lemmas
are given in an appendix as well as a separate online technical supplement
(see Chao and Swanson (2022c).

Before proceeding, we first say a few words about some of the frequently
used notation in this paper. Throughout, let $\lambda _{\left( j\right)
}\left( A\right) $, $\lambda _{\max }\left( A\right) $, $\lambda _{\min
}\left( A\right) $, and $tr\left( A\right) $ denote, respectively, the $%
j^{th}$ largest eigenvalue, the maximal eigenvalue, the minimal eigenvalue,
and the trace of a square matrix $A$. Similarly, let $\sigma _{\left(
j\right) }\left( B\right) $, $\sigma _{\max }\left( B\right) $, and $\sigma
_{\min }\left( B\right) $ denote, respectively, the $j^{th}$ largest
singular value, the maximal singular value, and the minimal singular value
of a matrix $B$, which is not restricted to be a square matrix. In addition,
let $\left\Vert a\right\Vert _{2}$ denote the usual Euclidean norm when
applied to a (finite-dimensional) vector $a$. Also, for a matrix $A$, $%
\left\Vert A\right\Vert _{2}\equiv \max \left\{ \sqrt{\lambda \left(
A^{\prime }A\right) }:\lambda \left( A^{\prime }A\right) \text{ is an
eigenvalue of }A^{\prime }A\right\} $ denotes the matrix spectral norm, and $%
\left\Vert A\right\Vert _{F}\equiv \sqrt{tr\left\{ A^{\prime }A\right\} }$
denotes the Frobenius norm. For two random variables $X$ and $Y$, write $%
X\sim Y,$ if $X/Y=O_{p}\left( 1\right) $ and $Y/X=O_{p}\left( 1\right) $.
Furthermore, let $\left\vert z\right\vert $ denote the absolute value or the
modulus of the number $z$; let $\left\lfloor \cdot \right\rfloor $ denote
the floor function, so that $\left\lfloor x\right\rfloor $ gives the integer
part of the real number $x$, and let $\iota _{p}=\left( 1,1,...,1\right)
^{\prime }$ denote a $p\times 1$ vector of ones. Finally, the abbreviation
w.p.a.1 stands for \textquotedblleft with probability approaching
one\textquotedblright .

\section{Inconsistency in High-Dimensional Factor Estimation}

To provide some motivation for the problem we will be studying in this
paper, consider the following simple, stylized one-factor model:

\begin{equation}
\underset{N\times 1}{Z_{t}}=\underset{N\times 1}{\gamma }\underset{1\times 1}%
{f_{t}}+\underset{N\times 1}{u_{t}}\text{, }t=1,...,T
\label{one factor model}
\end{equation}%
for which we make the following assumption.

\noindent \textbf{Assumption 2-1: }(a)\textbf{\ }$\left\{ u_{t}\right\}
\equiv i.i.d.N\left( 0,I_{N}\right) ;$(b) $\left\{ f_{t}\right\} \equiv
i.i.d.N\left( 0,1\right) ;$and (c) $u_{s}$ and $f_{t}$ are independent for
all $t,s.$

\noindent Much of the literature on factor analysis focuses on the case
where the factors are pervasive. In the special case of the simple one
factor model given in expression (\ref{one factor model}) above,
pervasiveness means that:%
\begin{equation*}
\frac{\left\Vert \gamma \right\Vert _{2}^{2}}{N}\rightarrow c,
\end{equation*}%
for some constant $c$ such that $0<c<\infty ,$ where $\left\Vert \gamma
\right\Vert _{2}=\sqrt{\gamma ^{\prime }\gamma }$. In practice, however, one
may have a high-dimensional data vector $Z_{t}$ such that not all of the
components of $Z_{t}$ load significantly on the underlying factor, $f_{t}$.
In particular, let $\mathcal{P}$ be a permutation matrix which reorders the
components of $Z_{t}$, so that $\mathcal{P}Z_{t}$ can be partitioned as
follows:%
\begin{equation*}
\mathcal{P}Z_{t}=\left( 
\begin{array}{c}
\underset{N_{1}\times 1}{Z_{t}^{\left( 1\right) }} \\ 
\underset{N_{2}\times 1}{Z_{t}^{\left( 2\right) }}%
\end{array}%
\right) ,
\end{equation*}%
where $Z_{t}^{\left( 1\right) }=\gamma ^{\left( 1\right)
}f_{t}+u_{t}^{\left( 1\right) }$ and $Z_{t}^{\left( 2\right) }=u_{t}^{\left(
2\right) }$ and where all components of the $N_{1}\times 1$ vector $\gamma
^{\left( 1\right) }$ are different from zero, so that the components of $%
Z_{t}^{\left( 1\right) }$ all load significantly on $f_{t}$, whereas the
components of $Z_{t}^{\left( 2\right) }$ do not. Of course, an empirical
researcher will not typically have \`{a} priori knowledge as to which
components of $Z_{t}$ will load significantly on $f_{t}$ and which will not.
The following result shows that if one proceeds with factor estimation
assuming that the factor is pervasive, then the usual estimator of a factor
based on principal component methods may be inconsistent and may, in fact,
behave in a rather pathological manner in large samples. To consider this
possibility, assume the following condition, which implies a violation of
the pervasiveness assumption.

\noindent \textbf{Assumption 2-2: }As $N,T\rightarrow \infty $, let $%
\left\Vert \gamma \right\Vert _{2}\rightarrow \infty $ such that: 
\begin{equation*}
\frac{N}{T\left\Vert \gamma \right\Vert _{2}^{2\left( 1+\kappa \right) }}%
=c+o\left( \frac{1}{\left\Vert \gamma \right\Vert _{2}^{2}}\right) \text{, }
\end{equation*}%
for some constant $c,$ such that $0<c<\infty ,$ and for some constant $%
\kappa ,$ such that $0<\kappa <1$. Note that under Assumption 2-2:%
\begin{equation*}
\frac{\left\Vert \gamma \right\Vert _{2}^{2}}{N}\sim \left( TN^{\kappa
}\right) ^{-\frac{{\large 1}}{\left( {\large 1{\LARGE +}}{\Large \kappa }%
\right) }}\rightarrow 0\text{ as }N,T\rightarrow \infty ,
\end{equation*}%
so that the factor does not satisfy the pervasiveness assumption. This can,
of course, occur if a significant proportion of the components of $\gamma $
are zero or are very small. Next, let $\widehat{\pi }_{1}/\left\Vert 
\widehat{\pi }_{1}\right\Vert _{2}$ denote the (normalized) eigenvector
associated with the largest eigenvalue of the sample covariance matrix, $%
\widehat{\Sigma }_{Z}=\mathbf{Z}^{\prime }\mathbf{Z}/T$, where $\mathbf{Z}%
=\left( Z_{1},...,Z_{T}\right) ^{\prime }.$ Then, the usual principal
component estimator of $f_{t}$ is given by:%
\begin{equation*}
\widehat{f}_{t}=\frac{\left\langle \widehat{\pi }_{1},Z_{t}\right\rangle }{%
\sqrt{N}\left\Vert \widehat{\pi }_{1}\right\Vert _{2}}\text{.}
\end{equation*}%
The following theorem characterizes the asymptotic behavior of this
estimator under the assumptions given above.

\noindent \textbf{Theorem 2.1:}\textit{\ Suppose that Assumptions 2-1 and
2-2 hold. Then, for all }$t$\textit{: }$\widehat{f}_{t}\overset{p}{%
\rightarrow }0,$ as $N,T\rightarrow \infty .$\textit{\ }

\noindent It is well-known that without further identifying assumptions,
such as those given in Assumption F1 of Stock and Watson (2002a), factors
can only be estimated consistently up to an invertible matrix
transformation. However, even in cases where we are not willing to specify
enough conditions so as to fully identify the factors, estimating the
factors consistently up to an invertible matrix transformation will often
suffice for many purposes. One such case is when we are trying to forecast
using a factor-augmented vector autoregression (FAVAR). As we will show in
results given in Section 4 of this paper, point forecasts constructed using
factors which are estimated consistently up to an invertible matrix
transformation will nevertheless converge in probability to the desired
infeasible forecast (i.e., the conditional mean of the FAVAR), that in turn
depends on the true unobserved factors. On the other hand, the problem
illustrated by the result given in Theorem 1 is different and is in some
sense more problematic and pathological. The estimated factor in Theorem 1
converges to zero regardless of what happens to be the realized value of the
true latent factor. In this case, one clearly cannot consistently estimate
the conditional mean of the FAVAR.

Theorem 1 is related to results previously given in the statistics
literature showing the possible inconsistency of sample eigenvectors as
estimators of population eigenvectors in high dimensional situations. See,
for example, Paul (2007), Johnstone and Lu (2009), Shen, Shen, Zhu, and
Marron (2016), and Johnstone and Paul (2018). However, most of the results
in the statistics literature are not explicitly framed in the setting of a
factor model, but are instead derived for the related spiked covariance
model. Theorem 1 is intended to give an inconsistency result of this type,
but in a context that may be more familiar to researchers in economics.

It should also be noted that, in an interesting and thought-provoking recent
paper, Bai and Ng (2021) provide results which show that factors can still
be estimated consistently in certain situations where the factor loadings
are weaker than that implied by the conventional pervasiveness assumption,
but that in such cases the rate of convergence is slower and additional
assumptions are needed. To understand the relationship between their results
and the example given above, note that a key condition for the consistency
result given in their paper, when expressed in terms of our notation, is the
assumption that $N/\left( T\left\Vert \gamma \right\Vert _{2}^{2}\right)
\rightarrow 0$\footnote{%
See Assumption A4 of Bai and Ng (2021). Note that Bai and Ng (2021) state
this condition in the form $N/\left( TN^{\alpha }\right) \rightarrow 0,$ for
some $\alpha \in \left( 0,1\right] $, but since part (ii) of their
Assumption A2, when specialized to the one factor model studied here,
simplifies to the condition that $\lim_{N\rightarrow \infty }\left\Vert
\gamma \right\Vert _{2}^{2}/N^{\alpha }=\sigma _{\Lambda }>0$, it is easy to
see that their Assumption A4 is equivalent to the condition that $N/\left(
T\left\Vert \gamma \right\Vert _{2}^{2}\right) \rightarrow 0$.}. On the
other hand, if $N/\left( T\left\Vert \gamma \right\Vert _{2}^{2}\right)
\rightarrow c_{1},$ for some positive constant $c_{1},$ or even worse, if $%
N/\left( T\left\Vert \gamma \right\Vert _{2}^{2}\right) \rightarrow \infty $%
, which is essentially what is specified in Assumption 2-2 above, then
consistent factor estimation cannot be achieved\footnote{%
Note that Assumption 2-2 is actually stronger than required in order to show
inconsistency, but that we impose this condition to highlight the fact that,
in this case, not only is the estimator of the factor inconsistent but it
actually converges to zero.}. Hence, whether or not consistent factor
estimation can be attained depends on how nonpervasive the factors are,
which is ultimately an empirical question, and which depends on the
application and on the dataset employed. Moreover, various authors have now
documented cases where empirical results suggest that the underlying factors
may be quite weak, so that the rate condition given in Bai and Ng (2021) may
not be appropriate, at least for some of the situations for which factor
modeling is of interest. For example, see Jagannathan and Wang (1998), Kan
and Zhang (1999), Harding (2008), Kleibergen (2009), Ontaski (2012),
Bryzgalova (2016), Burnside (2016), Gospodinov, Kan, and Robotti (2017),
Anatolyev and Mikusheva (2021), and Freyaldenhoven (2021a,b). In such cases,
it is of interest to explore the possibility that the weakness in the
loadings is not uniform across all variables, but rather is due to the fact
that only a small percentage of the variables loads significantly on the
underlying factors. Furthermore, even if the empirical situation of interest
is one where, strictly speaking, the condition $N/\left( T\left\Vert \gamma
\right\Vert _{2}^{2}\right) \rightarrow 0$ does hold, it may still be
beneficial in some such instances to do variable pre-screening. This is
particularly true in situations where the condition $N/\left( T\left\Vert
\gamma \right\Vert _{2}^{2}\right) \rightarrow 0$ is \textquotedblleft
barely\textquotedblright\ satisfied, in which case one would expect to pay a
rather hefty finite sample price for not pruning out variables that do not
load significantly on the underlying factors, since these variables will add
unwanted noise to the estimation process. For all these reasons, there is a
clear need to develop methods that will enable empirical researchers to
pre-screen the components of $Z_{t},$ so that variables which are
informative and helpful to the estimation process can be properly identified.

\section{\noindent Model, Assumptions, and Variable Selection in High
Dimensions}

\noindent \qquad Following CS (2022a), we begin by considering the following 
$p^{th}$-order factor-augmented vector autoregression (FAVAR):%
\begin{equation}
W_{t+1}=\mu +A_{1}W_{t}+\cdot \cdot \cdot +A_{p}W_{t-p+1}+\varepsilon _{t+1}%
\text{,}  \label{FAVAR}
\end{equation}%
where%
\begin{eqnarray*}
\underset{\left( d+K\right) \times 1}{W_{t+1}} &=&\left( 
\begin{array}{c}
\underset{d\times 1}{Y_{t+1}} \\ 
\underset{K\times 1}{F_{t+1}}%
\end{array}%
\right) \text{, }\underset{\left( d+K\right) \times 1}{\varepsilon _{t+1}}%
=\left( 
\begin{array}{c}
\underset{d\times 1}{\varepsilon _{t+1}^{Y}} \\ 
\underset{K\times 1}{\varepsilon _{t+1}^{F}}%
\end{array}%
\right) \text{, }\underset{\left( d+K\right) \times 1}{\mu }=\left( 
\begin{array}{c}
\underset{d\times 1}{\mu _{Y}} \\ 
\underset{K\times 1}{\mu _{F}}%
\end{array}%
\right) ,\text{ and} \\
\text{ }\underset{\left( d+K\right) \times \left( d+K\right) }{A_{g}}
&=&\left( 
\begin{array}{cc}
\underset{d\times d}{A_{YY,g}} & \underset{d\times K}{A_{YF,g}} \\ 
\underset{K\times d}{A_{FY,g}} & \underset{K\times K}{A_{FF,g}}%
\end{array}%
\right) ,\text{ for }g=1,...,p.
\end{eqnarray*}%
This system of equations, where $Y_{t}$ denotes the vector of observable
economic variables, and $F_{t}$ is a vector of unobserved (latent) factors
can also be written is several alternative ways, the following two of which
are are variously used throughout this paper. Namely:%
\begin{eqnarray}
Y_{t+1} &=&\mu _{Y}+A_{YY}\underline{Y}_{t}+A_{YF}\underline{F}%
_{t}+\varepsilon _{t+1}^{Y},  \label{Y component FAVAR} \\
F_{t+1} &=&\mu _{F}+A_{FY}\underline{Y}_{t}+A_{FF}\underline{F}%
_{t}+\varepsilon _{t+1}^{F},  \label{F component FAVAR}
\end{eqnarray}%
where%
\begin{eqnarray}
\underset{d\times dp}{A_{YY}} &=&\left( 
\begin{array}{cccc}
A_{YY,1} & A_{YY,2} & \cdots & A_{YY,p}%
\end{array}%
\right) ,\text{ }\underset{d\times Kp}{A_{YF}}=\left( 
\begin{array}{cccc}
A_{YF,1} & A_{YF,2} & \cdots & A_{YF,p}%
\end{array}%
\right) ,\text{ }  \notag \\
\underset{K\times dp}{A_{FY}} &=&\left( 
\begin{array}{cccc}
A_{FY,1} & A_{FY,2} & \cdots & A_{FY,p}%
\end{array}%
\right) ,\text{ }\underset{K\times Kp}{A_{FF}}=\left( 
\begin{array}{cccc}
A_{FF,1} & A_{FF,2} & \cdots & A_{FF,p}%
\end{array}%
\right) ,  \notag \\
\underset{dp\times 1}{\underline{Y}_{t}} &=&\left( 
\begin{array}{c}
Y_{t} \\ 
Y_{t-1} \\ 
\vdots \\ 
Y_{t-p{\LARGE +}1}%
\end{array}%
\right) \text{, and}\underset{Kp\times 1}{\underline{F}_{t}}=\left( 
\begin{array}{c}
F_{t} \\ 
F_{t-1} \\ 
\vdots \\ 
F_{t-p{\LARGE +}1}%
\end{array}%
\right) \text{, }  \label{Yunderscore and Funderscore}
\end{eqnarray}%
and%
\begin{equation*}
\underset{\left( d+K\right) p\times 1}{\underline{W}_{t}}=\alpha +A%
\underline{W}_{t-1}+E_{t}\text{,}
\end{equation*}%
where $\underline{W}_{t}=\left( 
\begin{array}{ccccc}
W_{t}^{\prime } & W_{t-1}^{\prime } & \cdots & W_{t-p{\LARGE +}2}^{\prime }
& W_{t-p{\LARGE +}1}^{\prime }%
\end{array}%
\right) ^{\prime }$ and where%
\begin{equation}
\alpha =\left( 
\begin{array}{c}
\mu \\ 
0 \\ 
\vdots \\ 
0 \\ 
0%
\end{array}%
\right) \text{, }A=\left( 
\begin{array}{ccccc}
A_{1} & A_{2} & \cdots & A_{p-1} & A_{p} \\ 
I_{d+K} & 0 & \cdots & 0 & 0 \\ 
0 & I_{d+K} & \ddots & \vdots & 0 \\ 
\vdots & \ddots & \ddots & 0 & \vdots \\ 
0 & \cdots & 0 & I_{d+K} & 0%
\end{array}%
\right) \text{, and }E_{t}=\left( 
\begin{array}{c}
\varepsilon _{t} \\ 
0 \\ 
\vdots \\ 
0 \\ 
0%
\end{array}%
\right) \text{.}  \label{companion form notations}
\end{equation}%
The companion form given in equation (\ref{companion form notations}) is
convenient for establishing certain moment conditions on $\underline{Y}_{t}$
and $\underline{F}_{t},$ given a moment condition on $\varepsilon _{t},$ and
for establishing certain mixing properties of the FAVAR model, as shown in
the proofs of Lemmas C-5 and Lemma C-11 given in Chao and Swanson (2022b).
It remains to define the relationship between the $F_{t}$ and the variables
used to extract these factors. To do this, we assume that:%
\begin{equation}
\underset{N\times 1}{Z_{t}}\text{ }=\text{ }\underset{N\times Kp}{\Gamma }%
\underline{F}_{t}+u_{t}\text{, }  \label{overspecified factor model}
\end{equation}%
where the properties of $u_{t}$ are given in Assumptions 3-3 and 3-4, below.
Following Chao and Swanson (2022a), we assume that not all components of $%
Z_{t}$ provide useful information for estimating $\underline{F}_{t}$,
implying that the $N\times Kp$ parameter matrix $\Gamma $ may have some rows
whose elements are all zero. More precisely, let the $1\times Kp$ vector, $%
\gamma _{i}^{\prime },$ denote the $i^{th}$ row of $\Gamma $, and assume
that the rows of the matrix $\Gamma $ can be divided into two classes:%
\begin{eqnarray}
H &=&\left\{ k\in \left\{ 1,....,N\right\} :\gamma _{k}=0\right\} \text{ and}
\label{H} \\
H^{c} &=&\left\{ k\in \left\{ 1,....,N\right\} :\gamma _{k}\neq 0\right\} 
\text{.}  \label{Hc}
\end{eqnarray}%
Thus, there exists a permutation matrix $\mathcal{P}$ such that $\mathcal{P}%
Z_{t}=\left( 
\begin{array}{cc}
Z_{t}^{\left( 1\right) \prime } & Z_{t}^{\left( 2\right) \prime }%
\end{array}%
\right) ^{\prime },$ where%
\begin{eqnarray}
\underset{N_{1}\times 1}{Z_{t}^{\left( 1\right) }} &=&\Gamma _{1}\underline{F%
}_{t}+u_{t}^{\left( 1\right) }  \label{Z(1)} \\
\underset{N_{2}\times 1}{Z_{t}^{\left( 2\right) }} &=&u_{t}^{\left( 2\right)
}\text{.}  \label{Z(2)}
\end{eqnarray}%
In this way, the components of $Z_{t}^{\left( 1\right) }$ can be interpreted
as \textquotedblleft information\textquotedblright\ variables that are
useful for estimating $\underline{F}_{t}$. On the other hand, for the
purpose of factor estimation, the components of the subvector $Z_{t}^{\left(
2\right) }$ are pure \textquotedblleft noise\textquotedblright\ variables,
as they do not load on the underlying factors and only add noise if they are
included in the factor estimation process. Given that an empirical
researcher will often not have prior knowledge as to which variables are
elements of $Z_{t}^{\left( 1\right) }$ and which are elements of $%
Z_{t}^{\left( 2\right) }$, Theorem 2.1 suggests the need for a variable
selection procedure which will allow us to properly identify the components
of of $Z_{t}^{\left( 1\right) }$ and to use only these variables when we try
to estimate $\underline{F}_{t}$; for, if we unknowingly include too many
components of $Z_{t}^{\left( 2\right) }$ in the estimation process, then
inconsistent estimation in the sense described in the previous section can
result.\footnote{%
In the statistics literature, there is a growing literature on the potential
inconsistency of sample eigenvectors in high dimensional problems, as
discussed in Paul (2007), Johnstone and Lu (2009), Shen Shen, Zhu, and
Marron (2016), and Johnstone and Paul (2018).} As discussed in CS (2022a),
there is an important related paper by Bai and Ng (2021) that establishes
factor estimator consistency for cases where $N/(TN_{1})\rightarrow 0$. For
cases where $N/(TN_{1})\rightarrow c$, or $N/(TN_{1})\rightarrow \infty $,
where $c$ is a constant, their result does not hold. In this paper, we
establish that consistency can be achieved in our context even if $%
N/(TN_{1})\nrightarrow 0$, if one pre-screens variables using the
self-normalized statistics outlined below. This is important because the
degree of factor pervasiveness is ultimately data dependent, and one way to
estimate $N_{1}$ invloves utilizing the variable screening statistic that is
discussed in the sequel.

In the sequel, we require the following assumptions.

\noindent \textbf{Assumption 3-1: }Suppose that:%
\begin{equation}
\det \left\{ I_{\left( d+K\right) }-A_{1}z-\cdot \cdot \cdot
-A_{p}z^{p}\right\} =0,\text{ implies that }\left\vert z\right\vert >1\text{.%
}  \label{stability cond}
\end{equation}

\noindent \textbf{Assumption 3-2: }Let $\varepsilon _{t}$ satisfy the
following set of conditions: (a) $\left\{ \varepsilon _{t}\right\} $ is an
independent sequence of random vectors with $E\left[ \varepsilon _{t}\right]
=0$ $\forall t$; (b) there exists a positive constant $C$ such that $%
\sup_{t}E\left\Vert \varepsilon _{t}\right\Vert _{2}^{6}\leq C<\infty $; (c) 
$\varepsilon _{t}$ admits a density $g_{\varepsilon _{t}}$ such that, for
some positive constant $M<\infty ,\sup_{t}\dint \left\vert g_{\varepsilon
_{t}}\left( \upsilon -u\right) -g_{\varepsilon _{t}}\left( \upsilon \right)
\right\vert d\varepsilon \leq M\left\vert u\right\vert $, whenever $%
\left\vert u\right\vert \leq \overline{\kappa }$ for some constant $%
\overline{\kappa }>0$; and (d) there exists a constant $\underline{C}>0$
such that $\inf_{t}\lambda _{\min }\left\{ E\left[ \varepsilon
_{t}\varepsilon _{t}^{\prime }\right] \right\} \geq \underline{C}>0$.

\noindent \textbf{Assumption 3-3: }Let $u_{i,t}$ be the $i^{th}$ element of
the error vector $u_{t}$ in expression (\ref{overspecified factor model}),
and we assume that it satisfies the following conditions: (a) $E\left[
u_{i,t}\right] =0$ for all $i$ and $t$; (b) there exists a positive constant 
$\overline{C}$ such that $\sup_{i,t}E\left\vert u_{i,t}\right\vert ^{7}\leq 
\overline{C}<\infty $, and there exists a constant $\underline{C}>0$ such
that $\inf_{i,t}E\left[ u_{i,t}^{2}\right] \geq \underline{C}$; (c) define $%
\mathcal{F}_{i,-\infty }^{t}=\sigma \left(
....,u_{i,t-2},u_{i,t-1},u_{t}\right) $, $\mathcal{F}_{i,t+m}^{\infty
}=\sigma \left( u_{i,t+m},u_{i,t+m+1},u_{i,t+m+2},....\right) $, and 
\begin{equation*}
\beta _{i}\left( m\right) =\sup_{t}E\left[ \sup \left\{ \left\vert P\left( B|%
\mathcal{F}_{i,-\infty }^{t}\right) -P\left( B\right) \right\vert :B\in 
\mathcal{F}_{i,t+m}^{\infty }\right\} \right] \text{.}
\end{equation*}%
Assume that there exist constants $a_{1}>0$ and $a_{2}>0$ such that%
\begin{equation*}
\beta _{i}\left( m\right) \leq a_{1}\exp \left\{ -a_{2}m\right\} ,\text{ for
all }i\text{;}
\end{equation*}%
and (d) there exists a positive constant $C$ such that $\sup_{t}\left( \frac{%
1}{N_{1}}\dsum\limits_{i\in H^{{\large c}}}\dsum\limits_{k\in H^{{\large c}%
}}\left\vert E\left[ u_{i,t}u_{k,t}\right] \right\vert \right) \leq C<\infty 
$ for every positive integer $N_{1}$, where $H^{{\large c}}$ is defined in
expression (\ref{Hc}) above.

\noindent \textbf{Assumption 3-4: }$\varepsilon _{t}$ and $u_{i,s}$ are
independent, for all $i,t,$ and $s$.

\noindent \textbf{Assumption 3-5: }There exists a positive constant $%
\overline{C},$ such that $\sup_{i\in H^{{\large c}}}\left\Vert \gamma
_{i}\right\Vert _{2}\leq \overline{C}<\infty $ and $\left\Vert \mu
\right\Vert _{2}\leq \overline{C}<\infty $, where $\mu =\left( \mu
_{Y}^{\prime },\mu _{F}^{\prime }\right) ^{\prime }$.

\noindent \textbf{Assumption 3-6: }There exists a positive constant $%
\overline{C},$ such that:%
\begin{equation*}
0<\frac{1}{\overline{C}}\leq \lambda _{\min }\left( \frac{\Gamma ^{\prime
}\Gamma }{N_{1}}\right) \leq \lambda _{\max }\left( \frac{\Gamma ^{\prime
}\Gamma }{N_{1}}\right) \leq \overline{C}<\infty \text{ for all }N_{1}\text{%
, }N_{2}\text{ sufficiently large,}
\end{equation*}%
where $N_{1}$ is the number of components of the subvector $Z_{t}^{\left(
1\right) }$ and $N_{2}$ is the number of components of the subvector $%
Z_{t}^{\left( 2\right) }$, as previously defined in expressions (\ref{Z(1)})
and (\ref{Z(2)}).

\noindent \textbf{Assumption 3-7: }Let $A$ be as defined in expression (\ref%
{companion form notations}) above, and let the eigenvalues of the matrix $%
I_{\left( d+K\right) p}-A$ be sorted so that:%
\begin{equation*}
\left\vert \lambda _{\left( 1\right) }\left( I_{\left( d+K\right)
p}-A\right) \right\vert \geq \left\vert \lambda _{\left( 2\right) }\left(
I_{\left( d+K\right) p}-A\right) \right\vert \geq \cdot \cdot \cdot \geq
\left\vert \lambda _{\left( \left( d+K\right) p\right) }\left( I_{\left(
d+K\right) p}-A\right) \right\vert =\overline{\phi }_{\min }\text{.}
\end{equation*}%
Suppose that there is a constant $\underline{C}>0$ such that%
\begin{equation}
\sigma _{\min }\left( I_{\left( d+K\right) p}-A\right) \geq \underline{C}%
\overline{\phi }_{\min }  \label{lower bd I-A}
\end{equation}%
In addition, there exists a positive constant $\overline{C}<\infty $ such
that, for all positive integer $j$, 
\begin{equation}
\sigma _{\max }\left( A^{j}\right) \leq \overline{C}\max \left\{ \left\vert
\lambda _{\max }\left( A^{j}\right) \right\vert ,\left\vert \lambda _{\min
}\left( A^{j}\right) \right\vert \right\} .  \label{upper bd A}
\end{equation}

\noindent Assumption 3-1 is the stability condition that one typically
assumes for a stationary VAR process, although we allow for possible
heterogeneity in the distribution of $\varepsilon _{t}$ across time, so that
our FAVAR process is not necessarily a strictly stationary process. Under
Assumption 3-1, there exists a vector moving average representation for the
FAVAR process. Assumption 3-1 is a well known assumption that is equivalent
to the condition that $\det \left\{ I_{\left( d+K\right) }-Az\right\} =0$
implies that $\left\vert z\right\vert >1.$

Since the factor loading matrix $\Gamma $ is an $N\times Kp$ matrix, where $%
N=N_{1}+N_{2}$, the matrix $\Gamma ^{\prime }\Gamma $ will have order of
magnitude equal to $N$ if the factors are pervasive. Much of the factor
analysis literature in both econometrics and statistics has studied the case
where factors are pervasive in this sense. For example, see Bai and Ng
(2002), Stock and Watson (2002a), Bai (2003), and Fan, Liao, and Mincheva
(2011, 2013). Assumption 3-6 allows for possible violations of this
conventional pervasiveness assumption, which will occur in our setup when $%
N_{1}/N\rightarrow 0$.

\noindent Finally, Assumption 3-7 imposes a condition whereby the extreme
singular values of the matrices $A^{j}$ and $I_{\left( d+K\right) p}-A$ have
bounds that depend on the extreme eigenvalues of these matrices. For further
discussion of this Assumption, see CS (2022a).

Note that Assumptions 3-1, 3-2(a)-(c), and 3-7 are sufficient to prove Lemma
C-11 of Chao and Swanson (2022c)\footnote{%
A proof of Lemma C-11 was previously given in Chao and Swanson (2022b).},
which states that the process $\left\{ W_{t}\right\} $ generated by the
FAVAR model given in expression (\ref{FAVAR}) is a $\beta $-mixing process
with $\beta $-mixing coefficient satisfying:%
\begin{equation*}
\beta _{W}\left( m\right) \leq a_{1}\exp \left\{ -a_{2}m\right\} ,
\end{equation*}%
for some positive constants $a_{1}$ and $a_{2}$, with 
\begin{equation*}
\beta _{W}\left( m\right) =\sup_{t}E\left[ \sup \left\{ \left\vert P\left( B|%
\mathcal{A}_{-\infty }^{t}\right) -P\left( B\right) \right\vert :B\in 
\mathcal{A}_{t+m}^{\infty }\right\} \right] ,
\end{equation*}%
and with $\mathcal{A}_{-\infty }^{t}=\sigma \left(
...,W_{t-2},W_{t-1},W_{t}\right) $ and $\mathcal{A}_{t+m}^{\infty }=\sigma
\left( W_{t+m},W_{t+m+1},W_{t+m+2},....\right) $. Note that Assumption 3-2
(c) rules out situations such as that given in the famous counterexample
presented by Andrews (1984) which shows that a first-order autoregression
with errors having a discrete Bernoulli distribution is not $\alpha $%
-mixing, even if it satisfies the stability condition. Conditions similar to
Assumption 3-2(c) have also appeared in previous papers, such as Gorodetskii
(1977) and Pham and Tran (1985), which seek to provide sufficient conditions
for establishing the $\alpha $ or $\beta $ mixing properties of linear time
series processes.

Prior to presenting the main theorems of this paper, we first summarize the
variable selection procedure based on self-normalized statistics that is
outlined in CS (2022a), and draws on pathbreaking moderate deviation results
from Chen, Shao, Wu, and Xu (2016). To accommodate data dependence, consider
self-nomalized statistics that are constructed from observations which are
first split into blocks in a manner similar to the kind of construction one
would employ in implementing a block bootstrap or in proving a central limit
theorem using the blocking technique. One such statistic has the form of an $%
\ell _{\infty }$ norm and is given by: 
\begin{equation}
\max_{1\leq \ell \leq d}\left\vert S_{i,\ell ,T}\right\vert =\max_{1\leq
\ell \leq d}\left\vert \frac{\overline{S}_{i,\ell ,T}}{\sqrt{\overline{V}%
_{i,\ell ,T}}}\right\vert ,  \label{max statistic}
\end{equation}%
where 
\begin{eqnarray}
\overline{S}_{i,\ell ,T} &=&\dsum\limits_{r=1}^{q}\dsum\limits_{t=\left(
r-1\right) \tau +p}^{\left( r-1\right) \tau +\tau _{1}+p-1}Z_{it}y_{\ell ,t%
{\LARGE +}1}\text{ and}  \label{numerator max stat} \\
\overline{V}_{i,\ell ,T} &=&\dsum\limits_{r=1}^{q}\left[ \dsum\limits_{t=%
\left( r-1\right) \tau +p}^{\left( r-1\right) \tau +\tau
_{1}+p-1}Z_{it}y_{\ell ,t{\LARGE +}1}\right] ^{2}\text{.}
\label{denominator max stat}
\end{eqnarray}%
Here, $Z_{it}$ denotes the $i^{th}$ component of $Z_{t}$ , $y_{\ell ,t+1}$
denotes the $\ell ^{th}$ component of $Y_{t+1}$, $\tau _{1}=\left\lfloor
T_{0}^{\alpha _{{\large 1}}}\right\rfloor $, and $\tau _{2}=\left\lfloor
T_{0}^{\alpha _{{\large 2}}}\right\rfloor $, where $1>\alpha _{1}\geq \alpha
_{2}>0$, $\tau =\tau _{1}+\tau _{2}$, $q=\left\lfloor T_{0}/\tau
\right\rfloor $, and $T_{0}=T-p+1$. Note that the statistic given in
expression (\ref{max statistic}) can be interpreted as the maximum of the
(self-normalized) sample covariances between the $i^{th}$ component of $%
Z_{t} $ and the components of $Y_{t+1}$. A second statistic has the form of
a pseudo-$L_{1}$ norm and is given by: 
\begin{equation*}
\dsum\limits_{\ell =1}^{d}\varpi _{\ell }\left\vert S_{i,\ell ,T}\right\vert
=\dsum\limits_{\ell =1}^{d}\varpi _{\ell }\left\vert \frac{\overline{S}%
_{i,\ell ,T}}{\sqrt{\overline{V}_{i,\ell ,T}}}\right\vert ,
\end{equation*}%
where $\overline{S}_{i,\ell ,T}$ and $\overline{V}_{i,\ell ,T}$ are as
defined in expressions (\ref{numerator max stat}) and (\ref{denominator max
stat}) above and where $\left\{ \varpi _{\ell }:\ell =1,..,d\right\} $
denotes pre-specified weights, such that $\varpi _{\ell }\geq 0,$ for every $%
\ell \in \left\{ 1,...,d\right\} $ and $\dsum\nolimits_{\ell =1}^{d}\varpi
_{\ell }=1$. In order to keep the effects of dependence under control, the
construction of these statistics is based only on observations in every
other block. In order to consistently estimate the factors up to an
invertible matrix transformation, the variable selection procedure here must
be such that the the probability of a false positive and the probability of
a false negative converge to zero as $N_{1}$, $N_{2}$, $T\rightarrow \infty $%
\footnote{%
Here, a false positive refers to mis-classifying a variable, $Z_{it},$ as a
relevant variable for the purpose of factor estimation when its factor
loading $\gamma _{i}^{\prime }=0$, whereas a false negative refers to the
opposite case, where $\gamma _{i}^{\prime }\neq 0,$ but the variable $Z_{it}$
is mistakenly classified as irrelevant.}. This is different from the typical
multiple hypothesis testing approach whereby one tries to control the
familywise error rate (or, alternatively, the false discovery rate), so that
it is no greater than $0.05,$ say, but does not try to ensure that this
probability goes to zero as the sample size grows.

In order to implement this procedure, it reamins only to determine whether
the $i^{th}$ component of $Z_{t}$ is a relevant variable for the purpose of
factor estimation. Define $i\in \widehat{H}^{c}$ to indicate that $Z_{it}$
is a relevant variable and $i\in \widehat{H}$ to indicate that $Z_{it}$ is
an irrelevant variable, for factor estiamtion. Now, let $\mathbb{S}%
_{i,T}^{+} $ denote either the statistic $\max_{1\leq \ell \leq d}\left\vert
S_{i,\ell ,T}\right\vert $ or the statistic $\dsum\nolimits_{\ell
=1}^{d}\varpi _{\ell }\left\vert S_{i,\ell ,T}\right\vert $. The variable
selection procedure is based on the decision rule: 
\begin{equation}
i\in \left\{ 
\begin{array}{cc}
\widehat{H}^{c} & \text{ if }\mathbb{S}_{i,T}^{+}\geq \Phi ^{-1}\left( 1-%
\frac{\varphi }{2N}\right) \\ 
\widehat{H} & \text{if }\mathbb{S}_{i,T}^{+}<\Phi ^{-1}\left( 1-\frac{%
\varphi }{2N}\right)%
\end{array}%
\right. ,  \label{var selection decision rule}
\end{equation}%
where $\Phi ^{-1}\left( \cdot \right) $ denotes the quantile function or the
inverse of the cumulative distribution function of the standard normal
random variable, and where $\varphi $ is a tuning parameter which may depend
on $N$. Some conditions on $\varphi $ will be given in Assumptions 3-11 and
3-11* below. For a discussion of the use of the quantile function of the
standard normal as the threshold function, refer to CS (2022a), and note
that the threshold function used here is related to the one employed in
Belloni, Chen, Chernozhukov, and Hansen (2012).

In the sequel, we further require the following assumptions.

\noindent \textbf{Assumption 3-8: }There exists a positive constant, $%
\underline{c},$ such that for $T$ sufficiently large: 
\begin{equation*}
\min_{1\leq \ell \leq d}\min_{i\in H}\min_{r\in \left\{ 1,...,q\right\}
}E\left\{ \left[ \frac{1}{\sqrt{\tau _{1}}}\dsum\limits_{t=\left( r-1\right)
\tau +p}^{\left( r-1\right) \tau +\tau _{1}+p-1}y_{\ell ,t{\LARGE +}1}u_{it}%
\right] ^{2}\right\} \geq \underline{c},
\end{equation*}%
where, as defined earlier,%
\begin{equation*}
\tau _{1}=\left\lfloor T_{0}^{\alpha _{{\large 1}}}\right\rfloor \text{, }%
\tau _{2}=\left\lfloor T_{0}^{\alpha _{{\large 2}}}\right\rfloor \text{ for }%
1>\alpha _{1}\geq \alpha _{2}>0\text{ and }q=\left\lfloor \frac{T_{0}}{\tau
_{1}+\tau _{2}}\right\rfloor ,
\end{equation*}%
and $T_{0}=T-p+1$.

\noindent \textbf{Assumption 3-9: }Let $i\in H^{c}=\left\{ k\in \left\{
1,....,N\right\} :\gamma _{k}\neq 0\right\} .$Suppose that there exists a
positive constant, $\underline{c},$ such that, for all $N_{1},N_{2},$and $T$
sufficiently large:%
\begin{eqnarray*}
&&\min_{1\leq \ell \leq d}\min_{{\large i\in }H^{{\large c}}}\left\vert 
\frac{\mu _{i,\ell ,T}}{q\tau _{1}}\right\vert \\
&=&\min_{1\leq \ell \leq d}\min_{{\large i\in }H^{{\large c}}}\left\vert 
\frac{1}{q}\dsum\limits_{r=1}^{q}\frac{1}{\tau _{1}}\dsum\limits_{t=\left(
r-1\right) \tau +p}^{\left( r-1\right) \tau +\tau _{1}+p-1}\gamma
_{i}^{\prime }\left\{ E\left[ \underline{F}_{t}\right] \mu _{Y,\ell }+E\left[
\underline{F}_{t}\underline{Y}_{t}^{\prime }\right] \alpha _{YY,\ell }+E%
\left[ \underline{F}_{t}\underline{F}_{t}^{\prime }\right] \alpha _{YF,\ell
}\right\} \right\vert \\
&\geq &\underline{c}>0,
\end{eqnarray*}%
where $\mu _{Y,\ell }=e_{\ell ,d}^{\prime }\mu _{Y}$, $\alpha _{YY,\ell
}=A_{YY}^{\prime }e_{\ell ,d}$, and $\alpha _{YF,\ell }=A_{YF}^{\prime
}e_{\ell ,d}.$ Here, $e_{\ell ,d}$ is a $d\times 1$ elementary vector whose $%
\ell ^{th}$ component is $1$ and all other components are $0$.

\noindent \textbf{Assumption 3-10: }Suppose that, as $N_{1}$, $N_{2}$, and $%
T\rightarrow \infty $, the following rate conditions hold:

\begin{enumerate}
\item[(a)] 
\begin{equation*}
\frac{\sqrt{\ln N}}{T^{\min \left\{ \frac{{\large 1-\alpha }_{{\large 1}}}{%
{\large 6}},\frac{{\Large \alpha }_{{\large 2}}}{{\large 2}}\right\} }}%
\rightarrow 0\text{ }
\end{equation*}%
where $1>\alpha _{1}\geq \alpha _{2}>0$ and $N=N_{1}+N_{2}$.

\item[(b)] 
\begin{equation*}
\frac{N_{1}}{T^{3\alpha _{{\large 1}}}}\rightarrow 0\text{ where }1>\alpha
_{1}>0\text{.}
\end{equation*}
\end{enumerate}

\noindent \textbf{Assumption 3-11: }Let $\varphi $ satisfy the following two
conditions: (a) $\varphi \rightarrow 0$ as $N_{1},N_{2}\rightarrow \infty $,
and (b) there exists some constant $a>0,$ such that $\varphi \geq \frac{1}{%
N^{a}},$ for all $N_{1},N_{2}$ sufficiently large.

Note that Assumption 3-9 is a fairly mild condition which allows us to
differentiate the alternative hypothesis, $i\in H^{c},$ from the null
hypothesis, $i\in H.$ For further discussion of Assumptions 3-8 - 3-11,
refer to CS (2022a). Given the above assumptions, Theorem 1 of CS (2022a)
shows that the probability of a false positive, i.e., the probability that $%
i\in \widehat{H}^{c},$ even though $\gamma _{i}=0$, approaches zero, as $%
N,T\rightarrow \infty $, and Theorem 2 of the same paper shows that the
probability of a false negative, i.e., the probability that $i\in \widehat{H}
$ even though $\gamma _{i}\neq 0$, also approaches zero, as $N,T\rightarrow
\infty $. Together, these two theorems show that our variable selection
procedure is (completely) consistent in the sense that the probability of
committing a misclassification error vanishes as $N,T\rightarrow \infty $.
CS (2022a) also note that the above variable selection procedure provides us
with a consistent estimate $\widehat{N}_{1}$ of the unobserved quantity $%
N_{1}$, where the latter, in light of Assumption 3-6, can be interpreted as
giving the order of magnitude of $\Gamma ^{\prime }\Gamma $ and is, thus, a
measure of the overall pervasiveness of the factors in a given application.
Finally, note that knowledge of the number of factors is not needed to
implement the above variable selection procedure. Hence, in the case where
the number of factors needs to be determined empirically, an applied
researcher could first use our procedure to properly select the relevant
variables and then apply an information criterion such as that proposed in
Bai and Ng (2002) to estimate the number of factors.

Before presenting the main theoretical results proven in this paper, it is
worth making a final comment about variable selection. In particualr, note
that Bai and Ng (2008)\textbf{\ }address the important issue of choosing
predictor variables $Z_{it}$ based on their predictability for $Y_{t+1}$.
While we agree with this viewpoint, it is worth stressing that in our setup,
whether $Z_{it}$ helps to predict $Y_{t+h}$ depends on two things: (i)
whether $Z_{it}$ loads significantly on the underlying factors $\underline{F}%
_{t}$ (i.e., whether $\gamma _{i}\neq 0$ or not) and (ii) whether at least
some components of $\underline{F}_{t}$ are helpful for predicting certain
components of $Y_{t+h}$. The variable selection procedure which we propose
here focuses on the first issue but not the second. This is because, in our
view, it is important to first obtain good factor estimates with certain
desirable asymptotic properties before trying to assess which factor may or
may not be useful for predicting $Y_{t{\LARGE +}h}$. It is important to
distinguish between these two things because, if we try to do too much at
the variable selection stage and end up excluding a significant number of
(predictor) variables that load strongly on at least some of the factors,
then, this can lead to the factor vector $\underline{F}_{t}$ being
inconsistently estimated, and this is true even if the variables do not
individually help to predict $Y_{t+h}$, but instead are crucial for the
consistent estimation of the factor, which in turn is useful for predicting $%
Y_{t+h}$.

\section{\noindent Consistent Estimation of Factors and the h-Step Ahead
Predictor Based on the FAVAR Model}

In this section, we provide our main theoretical results on factor
estimation and on the estimation of the $h$-step predictor implied by the
FAVAR model. To obtain these results, we need to impose a further rate
condition on the tuning parameter, $\varphi $ (see part (c) of Assumption
3-11*).

\noindent \textbf{Assumption 3-11*: }Let $\varphi $ satisfy the following
three conditions: (a) $\varphi \rightarrow 0$ as $N_{1},N_{2}\rightarrow
\infty $, (b)\ there exists some constant $a>0,$ such that $\varphi \geq 
\frac{1}{N^{a}}$ for all $N_{1},N_{2}$ sufficiently large, and (c) 
\begin{equation*}
\max \left\{ \frac{N^{\frac{{\large 2}}{{\large 7}}}\varphi ^{\frac{{\large 5%
}}{{\large 7}}}}{N_{1}},\frac{N^{\frac{{\large 1}}{{\large 3}}}\varphi }{%
N_{1}T}\right\} \rightarrow 0\text{ as }N_{1},N_{2},T\rightarrow \infty .
\end{equation*}

\noindent \textbf{Remark 4.1: }Note that\textbf{\ }the rate condition given
in part (c) of Assumption 3-11* depends on $N_{1}$. However, if we choose $%
\varphi $ so that:%
\begin{equation*}
\varphi N^{\frac{{\large 2}}{{\large 5}}}=O\left( 1\right) \text{,}
\end{equation*}%
then 
\begin{equation*}
\frac{N^{\frac{{\large 2}}{{\large 7}}}\varphi ^{\frac{{\large 5}}{{\large 7}%
}}}{N_{1}}=O\left( \frac{1}{N_{1}}\right) =o\left( 1\right) \text{ and }%
\frac{N^{\frac{{\large 1}}{{\large 3}}}\varphi }{N_{1}T}=O\left( \frac{1}{%
N_{1}N^{\frac{{\large 1}}{{\large 15}}}T}\right) =o\left( \frac{1}{N_{1}}%
\right) \text{.}
\end{equation*}%
Hence, with this choice of $\varphi $, Assumption 3-11* part (c) will be
satisfied as long as $N_{1}\rightarrow \infty $, and there is no need to
impose any further condition on the rate at which $N_{1}$ grows. Requiring
that $N_{1}\rightarrow \infty $ is a minimal condition, since if $%
N_{1}\nrightarrow \infty $; then consistent factor estimation, even up to an
invertible matrix transformation, is impossible. Additionally, Monte Carlo
results reported in Section 3 of CS (2022a) show that the variable selection
procedure discussed above performs very well in finite samples, under the
tuning parameter choice $\varphi =N^{-\frac{{\large 2}}{{\large 5}}}$, both
in terms of controlling the probability of a false positive (or Type I)
error and in terms of controlling the probability of a false negative (or
Type II) error.

Next, consider the post-variable-selection principal component estimator

\noindent of $\underline{F}_{t}=\left( F_{t}^{\prime },F_{t-1}^{\prime
},...,F_{t-p{\LARGE +}1}^{\prime }\right) :$ 
\begin{equation}
\widehat{\underline{F}}_{t}=\frac{\widehat{\Gamma }^{\prime }Z_{t,N}\left( 
\widehat{H^{c}}\right) }{\widehat{N}_{1}}\text{,}  \label{factor estimator}
\end{equation}%
where%
\begin{equation*}
Z_{t,N}\left( \widehat{H^{c}}\right) =\left[ 
\begin{array}{cccc}
Z_{1,t}\mathbb{I}\left\{ 1\in \widehat{H^{c}}\right\} & Z_{2,t}\mathbb{I}%
\left\{ 2\in \widehat{H^{c}}\right\} & \cdots & Z_{N,t}\mathbb{I}\left\{
N\in \widehat{H^{c}}\right\}%
\end{array}%
\right] ^{\prime },
\end{equation*}%
with%
\begin{equation*}
\mathbb{I}\left\{ i\in \widehat{H^{c}}\right\} =\left\{ 
\begin{array}{cc}
1 & \text{if }i\in \widehat{H^{c}}\text{, i.e., if }\mathbb{S}%
_{i,T}^{+}>\Phi ^{-1}\left( 1-\frac{\varphi }{2N}\right) \\ 
0 & \text{if }i\in \widehat{H}\text{, i.e., if }\mathbb{S}_{i,T}^{+}\leq
\Phi ^{-1}\left( 1-\frac{\varphi }{2N}\right)%
\end{array}%
\right. ,
\end{equation*}%
and where $\widehat{N}_{1}=\#\left( \widehat{H^{c}}\right) $, i.e., the
cardinality of the set $\widehat{H^{c}}$. Here, $\widehat{\Gamma }$ denotes
the principal component estimator of the loading matrix $\Gamma $
constructed from taking $\sqrt{\widehat{N}_{1}}$ times the matrix whose
columns are the eigenvectors of the post-variable-selection sample
covariance matrix $\widehat{\Sigma }\left( \widehat{H^{c}}\right) $
associated with the $Kp$ largest eigenvalues of this matrix, where, in this
case,%
\begin{equation*}
\widehat{\Sigma }\left( \widehat{H^{c}}\right) =\frac{Z\left( \widehat{H^{c}}%
\right) ^{\prime }Z\left( \widehat{H^{c}}\right) }{\widehat{N}_{1}T_{0}}=%
\frac{1}{\widehat{N}_{1}T_{0}}\dsum\limits_{t=p}^{T}Z_{t,N}\left( \widehat{%
H^{c}}\right) Z_{t,N}\left( \widehat{H^{c}}\right) ^{\prime },\text{ }
\end{equation*}%
with $T_{0}=T-p+1$.

Our next result shows that the estimator given in expression (\ref{factor
estimator}) consistently estimates the unobserved factors $\underline{F}%
_{t}, $up to an invertible $Kp\times Kp$ matrix transformation.

\noindent \textbf{Theorem 4.1: }\textit{Suppose that Assumptions 3-1, 3-2,
3-3, 3-4, 3-5, 3-6, 3-7, 3-8, 3-9, and 3-10 hold. Let }$\widehat{\underline{F%
}}_{t}$\textit{\ be as defined in expression (\ref{factor estimator}).
Assume further that the specification of the tuning parameter, }$\varphi ,$%
\textit{\ in the decision rule (\ref{var selection decision rule}) satisfies
Assumption 3-11*. Then,}%
\begin{equation*}
\left\Vert \widehat{\underline{F}}_{t}-Q^{\prime }\underline{F}%
_{t}\right\Vert _{2}=o_{p}\left( 1\right) ,\text{ for all fixed }t\text{,}
\end{equation*}%
\textit{where}%
\begin{equation*}
Q=\left( \frac{\Gamma ^{\prime }\Gamma }{N_{1}}\right) ^{\frac{{\Large 1}}{%
{\Large 2}}}\Xi \widehat{V}\text{,}
\end{equation*}%
\textit{and where }$\widehat{V}$\textit{\ is the }$Kp\times Kp$\textit{\
orthogonal matrix given in Lemma D-14, and }$\Xi $\textit{\ is a }$Kp\times
Kp$\textit{\ orthogonal matrix whose columns are the eigenvectors of the
matrix }%
\begin{equation*}
M_{FF}^{\ast }=\left( \frac{\Gamma ^{\prime }\Gamma }{N_{1}}\right)
^{1/2}M_{FF}\left( \frac{\Gamma ^{\prime }\Gamma }{N_{1}}\right)
^{1/2}=\left( \frac{\Gamma ^{\prime }\Gamma }{N_{1}}\right) ^{1/2}\frac{1}{%
T_{0}}\dsum\limits_{t=p}^{T}E\left[ \underline{F}_{t}\underline{F}%
_{t}^{\prime }\right] \left( \frac{\Gamma ^{\prime }\Gamma }{N_{1}}\right)
^{1/2}\text{.}
\end{equation*}

\noindent \qquad If we examine the proof of Theorem 4.1 in the appendix and
the supporting arguments given in the proof of Lemma D-15 of Appendix D of
Chao and Swanson (2022c), we see that two of the key components of the proof
involve showing that:%
\begin{equation*}
\left\Vert \frac{\Gamma \left( \widehat{H^{c}}\right) -\Gamma }{\sqrt{N_{1}}}%
\right\Vert _{2}\overset{p}{\rightarrow }0
\end{equation*}%
and that 
\begin{equation*}
\frac{\widehat{N}_{1}-N_{1}}{N_{1}}\overset{p}{\rightarrow }0\text{.}
\end{equation*}%
This is one of the reasons why we argue that initial variable selection
should focus on determining which variables load strongly on the factors
without worrying specifically at that stage about the related issues of
predictability or, for that matter, any other issue. By contrast, if we make
our initial variable selection based on some more stringent criterion that
takes into consideration not only variable relevance but also other concerns
such as predictability, then, we may end up with a much smaller set $%
\widetilde{H}^{c}$ of selected variables relative to the set $\widehat{H^{c}}
$ selected under our procedure. In particular, in this case, it may be
possible that even in large samples a significant number of rows of $\Gamma
\left( \widetilde{H^{c}}\right) $ may contain only zero elements even though
the corresponding row of $\Gamma $ is not a zero vector, so that the result:%
\begin{equation*}
\left\Vert \frac{\Gamma \left( \widetilde{H^{c}}\right) -\Gamma }{\sqrt{N_{1}%
}}\right\Vert _{2}\overset{p}{\rightarrow }0
\end{equation*}%
may not hold. For the same reason, if we let $\widetilde{N}_{1}$ denote the
cardinality of the set of selected indices based on an alternative, more
stringent variable selection procedure, then, the result: 
\begin{equation*}
\frac{\widetilde{N}_{1}-N_{1}}{N_{1}}\overset{p}{\rightarrow }0
\end{equation*}%
also may not hold, since, by definition, $N_{1}$ is the number of rows of $%
\Gamma $ which have at least one non-zero element.

Although Theorem 4.1 shows that, without further identifying assumptions, we
can only estimate the factors $\underline{F}_{t}$ consistently up to an
invertible $Kp\times Kp$ matrix transformation, this result turns out to be
sufficient for us to estimate the $h$-step ahead predictor consistently.
More specifically, in Appendix D of Chao and Swanson (2022c) we show that
for $h$-step ahead forecasts associated with the (infeasible) forecasting
equation implied by the FAVAR model (\ref{FAVAR}), we have the form%
\begin{equation}
Y_{t{\LARGE +}h}=\beta _{0}+B_{1}^{\prime }\underline{Y}_{t}+B_{2}^{\prime }%
\underline{F}_{t}+\eta _{t{\LARGE +}h},
\label{infeasible h step forecast eqn}
\end{equation}%
where $\underline{Y}_{t}$ and $\underline{F}_{t}$ are as defined in
expression (\ref{Yunderscore and Funderscore}) above and where:%
\begin{eqnarray}
\beta _{0} &=&\dsum\limits_{j=0}^{h-1}J_{d}A^{j}\alpha \text{, }%
B_{1}^{\prime }=J_{d}A^{h}\mathcal{P}_{\left( d{\LARGE +}K\right) p}^{\prime
}S_{d}\text{, }B_{2}^{\prime }=J_{d}A^{h}\mathcal{P}_{\left( d{\LARGE +}%
K\right) p}^{\prime }S_{K}\text{ and}  \label{beta0 B1 and B2} \\
\text{ }\eta _{t{\LARGE +}h} &=&\dsum\limits_{j=0}^{h-1}J_{d}A^{j}J_{d%
{\LARGE +}K}^{\prime }\varepsilon _{t{\LARGE +}h-j}\text{.}  \notag
\end{eqnarray}%
Here, $\alpha $ and $A$ are, respectively, the intercept (vector) and the
coefficient matrix of the companion form defined in expression (\ref%
{companion form notations}) above, $\mathcal{P}_{\left( d{\LARGE +}K\right)
p}$ is a permutation matrix such that:%
\begin{equation*}
\mathcal{P}_{\left( d{\LARGE +}K\right) p}\underline{W}_{t}=\left( 
\begin{array}{c}
\underline{Y}_{t} \\ 
\underline{F}_{t}%
\end{array}%
\right) \text{,}
\end{equation*}%
and%
\begin{eqnarray*}
S_{d} &=&\left( 
\begin{array}{c}
I_{dp} \\ 
\underset{Kp\times dp}{0}%
\end{array}%
\right) \text{, }S_{K}=\left( 
\begin{array}{c}
\underset{dp\times Kp}{0} \\ 
I_{Kp}%
\end{array}%
\right) \text{,}\underset{d\times \left( d{\LARGE +}K\right) p}{J_{d}}=\left[
\begin{array}{cccc}
I_{d} & 0 & \cdots & 0%
\end{array}%
\right] \text{, and} \\
\text{ }\underset{\left( d{\LARGE +}K\right) \times \left( d{\LARGE +}%
K\right) p}{J_{d{\LARGE +}K}} &=&\left[ 
\begin{array}{cccc}
I_{d{\LARGE +}K} & 0 & \cdots & 0%
\end{array}%
\right] \text{.}
\end{eqnarray*}%
See\ the beginning of Appendix D of Chao and Swanson (2022c) for a
derivation of the equation given in expression (\ref{infeasible h step
forecast eqn}). The reason expression (\ref{infeasible h step forecast eqn})
is called an infeasible forecasting equation is, of course, because $%
\underline{F}_{t}$ is not observed, so to obtain a feasible version of this
forecasting equation, we must replace $\underline{F}_{t}$ in equation (\ref%
{infeasible h step forecast eqn}) with the estimate $\underline{\widehat{F}}%
_{t}$ given in expression (\ref{factor estimator}). Doing so, we arrive at a
feasible $h$-step ahead forecasting equation of the form: 
\begin{eqnarray}
Y_{t{\LARGE +}h} &=&\beta _{0}+\dsum\limits_{g=1}^{p}B_{1,g}^{\prime }Y_{t-g%
{\LARGE +}1}+\dsum\limits_{g=1}^{p}B_{2,g}^{\prime }\widehat{F}_{t-g{\LARGE +%
}1}+\widehat{\eta }_{t{\LARGE +}h}  \notag \\
&=&\beta _{0}+B_{1}^{\prime }\underline{Y}_{t}+B_{2}^{\prime }\underline{%
\widehat{F}}_{t}+\widehat{\eta }_{t{\LARGE +}h}\text{, }
\label{feasible h step forecast eqn}
\end{eqnarray}%
where $\widehat{\eta }_{t{\LARGE +}h}=\eta _{t{\LARGE +}h}-B_{2}^{\prime
}\left( \underline{\widehat{F}}_{t}-\underline{F}_{t}\right) ,$ with $\eta
_{t{\LARGE +}h}=\dsum\nolimits_{j=0}^{h-1}J_{d}A^{j}J_{d{\LARGE +}K}^{\prime
}\varepsilon _{t{\LARGE +}h-j}$.

One can interpret expression (\ref{feasible h step forecast eqn}) as a
\textquotedblleft reduced form\textquotedblright\ formulation of the
forecasting equation where the reduced form parameters $\beta _{0}$, $B_{1}$%
, and $B_{2}$ are nonlinear functions of the parameters $\left( \mu
,A_{1},....,A_{p}\right) $ of the FAVAR model, in the case where $h>1$. For
forecasting purposes, while it is possible to estimate the conditional mean
of the forecasting equation (\ref{feasible h step forecast eqn}) by
estimating the underlying parameters directly by nonlinear least squares,
here we choose instead to estimate the conditional mean by estimating the
reduced form parameters $\beta _{0}$, $B_{1}$, and $B_{2}$ via linear least
squares. An important reason why we choose this latter approach is due to
complications that arise both because we are forecasting with a FAVAR which
contains unobserved factors that must first be estimated and because we do
not make enough identifying assumptions so that the factors can only be
estimated consistently up to an invertible $Kp\times Kp$ matrix
transformation. In fact, it turns out that estimating the underlying
parameters $\mu ,A_{1},....,A_{p}$ by nonlinear least squares and
constructing an estimator of the conditional mean of the forecasting
equation based on these estimates will not lead to a consistently estimated $%
h$-step predictor, unless further identifying assumptions are made. On the
other hand, as we will show in Theorem 5 below, estimating the reduced form
parameters $\beta _{0}$, $B_{1}$, and $B_{2}$ by linear least squares does
allow us to construct a consistent estimator of the conditional mean, even
in the absence of additional identifying assumptions.

More precisely, let $\underline{\widehat{F}}_{t}$ denotes the factor
estimates given in expression (\ref{factor estimator}). Our procedure
minimizes the least squares criterion function:%
\begin{eqnarray}
Q\left( \beta _{0},B_{1},B_{2}\right) &=&\dsum\limits_{t=p}^{T-h}\left\Vert
Y_{t{\LARGE +}h}-\beta _{0}-B_{1}^{\prime }\underline{Y}_{t}-B_{2}^{\prime }%
\underline{\widehat{F}}_{t}\right\Vert _{2}^{2}  \notag \\
&=&\dsum\limits_{t=p}^{T-h}\left\Vert Y_{t{\LARGE +}h}-\beta
_{0}-\dsum\limits_{g=1}^{p}B_{1,g}^{\prime }Y_{t-g{\LARGE +}%
1}-\dsum\limits_{g=1}^{p}B_{2,g}^{\prime }\widehat{F}_{t-g{\LARGE +}%
1}\right\Vert _{2}^{2}  \label{least squares criterion}
\end{eqnarray}%
with respect to the parameters $\beta _{0}$, $B_{1}$, and $B_{2},$ and
delivers the OLS estimates $\widehat{\beta }_{0}$, $\widehat{B}_{1}$, and $%
\widehat{B}_{2}$. We then forecast $Y_{T{\LARGE +}h}$ using the $h$-step
predictor:%
\begin{equation}
\widehat{Y}_{T{\LARGE +}h}=\widehat{\beta }_{0}+\widehat{B}_{1}^{\prime }%
\underline{Y}_{T}+\widehat{B}_{2}^{\prime }\underline{\widehat{F}}_{T}\text{%
. }  \label{h step ahead predictor}
\end{equation}%
The following result shows that $\widehat{Y}_{T{\LARGE +}h}$ is a consistent
estimator of the conditional mean of the infeasible forecast equation (\ref%
{infeasible h step forecast eqn}).

\medskip

\noindent \textbf{Theorem 4.2: }\textit{Let }$\widehat{Y}_{T{\LARGE +}h}$%
\textit{\ be as defined in expression (\ref{h step ahead predictor}).
Suppose that Assumptions 3-1, 3-2, 3-3, 3-4, 3-5, 3-6, 3-7, 3-8, 3-9, 3-10,
and 3-11* hold. Then,}

\begin{equation*}
\widehat{Y}_{T{\LARGE +}h}-\left( \beta _{0}+B_{1}^{\prime }\underline{Y}%
_{T}+B_{2}^{\prime }\underline{F}_{T}\right) \overset{p}{\rightarrow }0\text{
as }N_{1},N_{2},T\rightarrow \infty \text{.}
\end{equation*}

\medskip

\section{\noindent Empirical Illustration}

To be completed.

\section{\noindent Conclusion}

In this paper, we study the problem of consistently estimating the
conditional mean of a factor-augmented forecasting equation based on the
FAVAR model. When the underlying dynamic factor model generating the latent
factors is high-dimensional, we show that it is important to pre-screen the
variables in terms of their association with the underlying factors prior to
estimation, particularly in cases where one suspects that the conventional
assumption of factor pervasiveness may not hold. For this purpose, we
utiluize a new variable selection procedure based on a self-normalized score
statistic (see Chao and Swanson (CS: 2022) that correctly identifies the set
of variables which load significantly on the underlying factors, with
probability approaching one, as the sample sizes go to infinity.
Furthermore, given that CS(2022) show that estimating the factors using only
those variables selected by their method allows factors to be consistently
estimated, up to an invertible matrix transformation, even if the standard
pervasiveness assumption does not hold, provided that the number of relevant
variables is sufficiently large. Using the factors estimated in such a
manner, we show that the conditional mean function of a factor-augmented
forecasting equation can be consistently estimated, even for the case of
multi-step ahead forecasts.

\section{\noindent Appendix}

All lemmas denoted C1-C17 and D1-D18 in this appendix are stated and proven
in an accompanying online appendix (see Chao and Swanson (2022c)).

\textbf{Proof of Theorem 2.1: }The proof of this theorem is rather long, and
is gathered in Appendix A of Chao and Swanson (2022c).

\textbf{Proof of Theorem 4.1: }

To proceed, note first that the principal component estimator of $\underline{%
F}_{t}$ can be written as%
\begin{equation*}
\widehat{\underline{F}}_{t}=\frac{\widehat{\Gamma }^{\prime }Z_{t,N}\left( 
\widehat{H^{c}}\right) }{\widehat{N}_{1}}
\end{equation*}%
where $\widehat{\Gamma }=\sqrt{\widehat{N}_{1}}\widehat{B}$ and where the
columns of the matrix $\widehat{B}$ are the eigenvectors associated with the 
$Kp$ largest eigenvalues of the (post-variable-selection) sample covariance
matrix 
\begin{equation*}
\widehat{\Sigma }\left( \widehat{H^{c}}\right) =\frac{Z\left( \widehat{H^{c}}%
\right) ^{\prime }Z\left( \widehat{H^{c}}\right) }{\widehat{N}_{1}T_{0}}.
\end{equation*}%
Moreover, by the result of part (d) of Lemma D-14, the matrix $\widehat{B}$
has the representation%
\begin{equation*}
\widehat{B}=\widehat{G}_{1}\widehat{V}
\end{equation*}%
where $\widehat{G}_{1}$ is an $N\times Kp$ matrix, whose columns define an
orthonormal basis for an invariant subspace of $\widehat{\Sigma }\left( 
\widehat{H^{c}}\right) $ and where $\widehat{V}$ is a $Kp\times Kp$
orthogonal matrix as defined in expression (\ref{V eigenvector matrix}) in
part (c) of Lemma D-14. (See Lemma D-14 and also Lemma D-13 for additional
discussion on the origin of this representation). Making use of this
representation, we can further write%
\begin{eqnarray*}
\widehat{\underline{F}}_{t}-Q^{\prime }\underline{F}_{t} &=&\frac{\sqrt{%
\widehat{N}_{1}}\widehat{V}^{\prime }\widehat{G}_{1}^{{\Large \prime }%
}Z_{t,N}\left( \widehat{H^{c}}\right) }{\widehat{N}_{1}}-Q^{\prime }%
\underline{F}_{t} \\
&=&\frac{\widehat{V}^{\prime }\widehat{G}_{1}^{{\Large \prime }}\Gamma
\left( \widehat{H^{c}}\right) \underline{F}_{t}}{\sqrt{\widehat{N}_{1}}}+%
\frac{\widehat{V}^{\prime }\widehat{G}_{1}^{{\Large \prime }}U_{t,N}\left( 
\widehat{H^{c}}\right) }{\sqrt{\widehat{N}_{1}}}-Q^{\prime }\underline{F}_{t}
\\
&=&\frac{\widehat{V}^{\prime }\widehat{G}_{1}^{{\Large \prime }}\Gamma
\left( \widehat{H^{c}}\right) \underline{F}_{t}}{\sqrt{\widehat{N}_{1}}}%
-Q^{\prime }\underline{F}_{t}+\frac{\widehat{V}^{\prime }\widehat{G}_{1}^{%
{\Large \prime }}U_{t,N}\left( \widehat{H^{c}}\right) }{\sqrt{\widehat{N}_{1}%
}} \\
&=&\left( \frac{\widehat{V}^{\prime }\widehat{G}_{1}^{{\Large \prime }%
}\Gamma \left( \widehat{H^{c}}\right) }{\sqrt{\widehat{N}_{1}}}-Q^{\prime
}\right) \underline{F}_{t}+\frac{\widehat{V}^{\prime }\widehat{G}_{1}^{%
{\Large \prime }}U_{t,N}\left( \widehat{H^{c}}\right) }{\sqrt{\widehat{N}_{1}%
}}
\end{eqnarray*}%
Next, note that%
\begin{eqnarray*}
\frac{\widehat{V}^{\prime }\widehat{G}_{1}^{{\Large \prime }}\Gamma }{\sqrt{%
\widehat{N}_{1}}}-Q^{\prime } &=&\frac{\widehat{V}^{\prime }\widehat{G}_{1}^{%
{\Large \prime }}\Gamma }{\sqrt{N_{1}}\sqrt{\left( \widehat{N}%
_{1}-N_{1}+N_{1}\right) /N_{1}}}-Q^{\prime } \\
&=&\left( 1+\frac{\widehat{N}_{1}-N_{1}}{N_{1}}\right) ^{-\frac{{\large 1}}{%
{\large 2}}}\frac{\widehat{V}^{\prime }\widehat{G}_{1}^{{\Large \prime }%
}\Gamma }{\sqrt{N_{1}}}-Q^{\prime } \\
&=&\left[ \left( 1+\frac{\widehat{N}_{1}-N_{1}}{N_{1}}\right) ^{-\frac{%
{\large 1}}{{\large 2}}}-1+1\right] \frac{\widehat{V}^{\prime }\widehat{G}%
_{1}^{{\Large \prime }}\Gamma }{\sqrt{N_{1}}}-Q^{\prime } \\
&=&\left[ \left( 1+\frac{\widehat{N}_{1}-N_{1}}{N_{1}}\right) ^{-\frac{%
{\large 1}}{{\large 2}}}-1\right] \frac{\widehat{V}^{\prime }\widehat{G}%
_{1}^{{\Large \prime }}\Gamma }{\sqrt{N_{1}}}+\frac{\widehat{V}^{\prime }%
\widehat{G}_{1}^{{\Large \prime }}\Gamma }{\sqrt{N_{1}}}-Q^{\prime }
\end{eqnarray*}%
and%
\begin{eqnarray*}
\frac{\Gamma \left( \widehat{H^{c}}\right) -\Gamma }{\sqrt{\widehat{N}_{1}}}
&=&\frac{\Gamma \left( \widehat{H^{c}}\right) -\Gamma }{\sqrt{N_{1}}\sqrt{%
\left( \widehat{N}_{1}-N_{1}+N_{1}\right) /N_{1}}} \\
&=&\left( 1+\frac{\widehat{N}_{1}-N_{1}}{N_{1}}\right) ^{-\frac{{\large 1}}{%
{\large 2}}}\left( \frac{\Gamma \left( \widehat{H^{c}}\right) -\Gamma }{%
\sqrt{N_{1}}}\right)
\end{eqnarray*}%
so that%
\begin{eqnarray*}
&&\frac{\widehat{V}^{\prime }\widehat{G}_{1}^{{\Large \prime }}\Gamma \left( 
\widehat{H^{c}}\right) \underline{F}_{t}}{\sqrt{\widehat{N}_{1}}} \\
&=&Q^{\prime }\underline{F}_{t}+\left( \frac{\widehat{V}^{\prime }\widehat{G}%
_{1}^{{\Large \prime }}\Gamma }{\sqrt{\widehat{N}_{1}}}-Q^{\prime }\right) 
\underline{F}_{t}+\widehat{V}^{\prime }\widehat{G}_{1}^{{\Large \prime }%
}\left( \frac{\Gamma \left( \widehat{H^{c}}\right) -\Gamma }{\sqrt{\widehat{N%
}_{1}}}\right) \underline{F}_{t} \\
&=&Q^{\prime }\underline{F}_{t}+\left( \frac{\widehat{V}^{\prime }\widehat{G}%
_{1}^{{\Large \prime }}\Gamma }{\sqrt{N_{1}}}-Q^{\prime }\right) \underline{F%
}_{t}+\left[ \left( 1+\frac{\widehat{N}_{1}-N_{1}}{N_{1}}\right) ^{-\frac{%
{\large 1}}{{\large 2}}}-1\right] \frac{\widehat{V}^{\prime }\widehat{G}%
_{1}^{{\Large \prime }}\Gamma }{\sqrt{N_{1}}}\underline{F}_{t} \\
&&+\left[ \left( 1+\frac{\widehat{N}_{1}-N_{1}}{N_{1}}\right) ^{-\frac{%
{\large 1}}{{\large 2}}}\right] \widehat{V}^{\prime }\widehat{G}_{1}^{%
{\Large \prime }}\left( \frac{\Gamma \left( \widehat{H^{c}}\right) -\Gamma }{%
\sqrt{N}_{1}}\right) \underline{F}_{t}
\end{eqnarray*}%
It follows that%
\begin{eqnarray*}
\widehat{\underline{F}}_{t}-Q^{\prime }\underline{F}_{t} &=&\left( \frac{%
\widehat{V}^{\prime }\widehat{G}_{1}^{{\Large \prime }}\Gamma \left( 
\widehat{H^{c}}\right) }{\sqrt{\widehat{N}_{1}}}-Q^{\prime }\right) 
\underline{F}_{t}+\frac{\widehat{V}^{\prime }\widehat{G}_{1}^{{\Large \prime 
}}U_{t,N}\left( \widehat{H^{c}}\right) }{\sqrt{\widehat{N}_{1}}} \\
&=&\left( \frac{\widehat{V}^{\prime }\widehat{G}_{1}^{{\Large \prime }%
}\Gamma }{\sqrt{N_{1}}}-Q^{\prime }\right) \underline{F}_{t}+\left[ \left( 1+%
\frac{\widehat{N}_{1}-N_{1}}{N_{1}}\right) ^{-\frac{{\large 1}}{{\large 2}}%
}-1\right] \frac{\widehat{V}^{\prime }\widehat{G}_{1}^{{\Large \prime }%
}\Gamma }{\sqrt{N_{1}}}\underline{F}_{t} \\
&&+\left[ \left( 1+\frac{\widehat{N}_{1}-N_{1}}{N_{1}}\right) ^{-\frac{%
{\large 1}}{{\large 2}}}\right] \widehat{V}^{\prime }\widehat{G}_{1}^{%
{\Large \prime }}\left( \frac{\Gamma \left( \widehat{H^{c}}\right) -\Gamma }{%
\sqrt{N}_{1}}\right) \underline{F}_{t}+\frac{\widehat{V}^{\prime }\widehat{G}%
_{1}^{{\Large \prime }}U_{t,N}\left( \widehat{H^{c}}\right) }{\sqrt{\widehat{%
N}_{1}}}
\end{eqnarray*}%
Hence, applying the triangle inequality as well as parts (a)-(c), (g), and
(i) of Lemma D-15 along with the Slutsky's theorem, we obtain 
\begin{eqnarray*}
&&\left\Vert \widehat{\underline{F}}_{t}-Q^{\prime }\underline{F}%
_{t}\right\Vert _{2} \\
&\leq &\left\Vert \frac{\widehat{V}^{\prime }\widehat{G}_{1}^{{\Large \prime 
}}\Gamma }{\sqrt{N_{1}}}-Q^{\prime }\right\Vert _{2}\left\Vert \underline{F}%
_{t}\right\Vert _{2}+\left\vert \left( 1+\frac{\widehat{N}_{1}-N_{1}}{N_{1}}%
\right) ^{-\frac{{\large 1}}{{\large 2}}}-1\right\vert \left\Vert \frac{%
\widehat{V}^{\prime }\widehat{G}_{1}^{{\Large \prime }}\Gamma }{\sqrt{N_{1}}}%
\right\Vert _{2}\left\Vert \underline{F}_{t}\right\Vert _{2} \\
&&+\left\vert \left( 1+\frac{\widehat{N}_{1}-N_{1}}{N_{1}}\right) ^{-\frac{%
{\large 1}}{{\large 2}}}\right\vert \left\Vert \widehat{V}^{\prime }\widehat{%
G}_{1}^{{\Large \prime }}\right\Vert _{2}\left\Vert \frac{\Gamma \left( 
\widehat{H^{c}}\right) -\Gamma }{\sqrt{N}_{1}}\right\Vert _{2}\left\Vert 
\underline{F}_{t}\right\Vert _{2}+\left\Vert \frac{\widehat{V}^{\prime }%
\widehat{G}_{1}^{{\Large \prime }}U_{t,N}\left( \widehat{H^{c}}\right) }{%
\sqrt{\widehat{N}_{1}}}\right\Vert _{2} \\
&=&\left\Vert \frac{\widehat{V}^{\prime }\widehat{G}_{1}^{{\Large \prime }%
}\Gamma }{\sqrt{N_{1}}}-Q^{\prime }\right\Vert _{2}\left\Vert \underline{F}%
_{t}\right\Vert _{2}+\left\vert \left( 1+\frac{\widehat{N}_{1}-N_{1}}{N_{1}}%
\right) ^{-\frac{{\large 1}}{{\large 2}}}-1\right\vert \left\Vert \frac{%
\widehat{V}^{\prime }\widehat{G}_{1}^{{\Large \prime }}\Gamma }{\sqrt{N_{1}}}%
\right\Vert _{2}\left\Vert \underline{F}_{t}\right\Vert _{2} \\
&&+\left\vert \left( 1+\frac{\widehat{N}_{1}-N_{1}}{N_{1}}\right) ^{-\frac{%
{\large 1}}{{\large 2}}}\right\vert \left\Vert \frac{\Gamma \left( \widehat{%
H^{c}}\right) -\Gamma }{\sqrt{N}_{1}}\right\Vert _{2}\left\Vert \underline{F}%
_{t}\right\Vert _{2}+\left\Vert \frac{\widehat{V}^{\prime }\widehat{G}_{1}^{%
{\Large \prime }}U_{t,N}\left( \widehat{H^{c}}\right) }{\sqrt{\widehat{N}_{1}%
}}\right\Vert _{2} \\
&&\left( \text{since }\left\Vert \widehat{V}^{\prime }\widehat{G}_{1}^{%
{\Large \prime }}\right\Vert _{2}=\lambda _{\max }\left( \widehat{G}_{1}%
\widehat{V}\widehat{V}^{\prime }\widehat{G}_{1}^{{\Large \prime }}\right)
=\lambda _{\max }\left( \widehat{V}^{\prime }\widehat{G}_{1}^{{\Large \prime 
}}\widehat{G}_{1}\widehat{V}\right) =\lambda _{\max }\left( I_{Kp}\right)
=1\right) \\
&=&o_{p}\left( 1\right) O_{p}\left( 1\right) +o_{p}\left( 1\right)
O_{p}\left( 1\right) O_{p}\left( 1\right) +O_{p}\left( 1\right) o_{p}\left(
1\right) O_{p}\left( 1\right) +o_{p}\left( 1\right) \\
&=&o_{p}\left( 1\right) \text{. }\square
\end{eqnarray*}

\medskip

\noindent

\noindent \textbf{Proof of Theorem 4.2:}

To proceed, note that for any $a\in \mathbb{R}^{d}$ such that $\left\Vert
a\right\Vert _{2}=1$, we have%
\begin{eqnarray*}
&&\left\vert a^{\prime }\widehat{Y}_{T{\LARGE +}h}-a^{\prime }\left( \beta
_{0}+B_{1}^{\prime }\underline{Y}_{T}+B_{2}^{\prime }\underline{F}%
_{T}\right) \right\vert \\
&=&\left\vert a^{\prime }\left( \widehat{\beta }_{0}+\widehat{B}_{1}^{\prime
}\underline{Y}_{T}+\widehat{B}_{2}^{\prime }\underline{\widehat{F}}%
_{T}\right) -a^{\prime }\left( \beta _{0}+B_{1}^{\prime }\underline{Y}%
_{T}+B_{2}^{\prime }\underline{F}_{T}\right) \right\vert \\
&=&\left\vert a^{\prime }\left( \widehat{\beta }_{0}-\beta _{0}\right)
+a^{\prime }\left( \widehat{B}_{1}-B_{1}\right) ^{\prime }\underline{Y}%
_{T}\right. \\
&&\left. +a^{\prime }\left( \widehat{B}_{2}-Q^{-1}B_{2}+Q^{-1}B_{2}\right)
^{\prime }\left( \widehat{\underline{F}}_{T}-Q^{\prime }\underline{F}%
_{T}+Q^{\prime }\underline{F}_{T}\right) -a^{\prime }B_{2}^{\prime }%
\underline{F}_{T}\right\vert \\
&\leq &\left\vert a^{\prime }\left( \widehat{\beta }_{0}-\beta _{0}\right)
\right\vert +\left\vert a^{\prime }\left( \widehat{B}_{1}-B_{1}\right)
^{\prime }\underline{Y}_{T}\right\vert +\left\vert a^{\prime }\left( 
\widehat{B}_{2}-Q^{-1}B_{2}\right) ^{\prime }\left( \widehat{\underline{F}}%
_{T}-Q^{\prime }\underline{F}_{T}\right) \right\vert \\
&&+\left\vert a^{\prime }B_{2}^{\prime }Q^{-1\prime }\left( \widehat{%
\underline{F}}_{T}-Q^{\prime }\underline{F}_{T}\right) \right\vert
+\left\vert a^{\prime }\left( \widehat{B}_{2}-Q^{-1}B_{2}\right) ^{\prime
}Q^{\prime }\underline{F}_{T}\right\vert +\left\vert a^{\prime
}B_{2}^{\prime }Q^{-1\prime }Q^{\prime }\underline{F}_{T}-a^{\prime
}B_{2}^{\prime }\underline{F}_{T}\right\vert \\
&=&\left\vert a^{\prime }\left( \widehat{\beta }_{0}-\beta _{0}\right)
\right\vert +\left\vert a^{\prime }\left( \widehat{B}_{1}-B_{1}\right)
^{\prime }\underline{Y}_{T}\right\vert +\left\vert a^{\prime }\left( 
\widehat{B}_{2}-Q^{-1}B_{2}\right) ^{\prime }\left( \widehat{\underline{F}}%
_{T}-Q^{\prime }\underline{F}_{T}\right) \right\vert \\
&&+\left\vert a^{\prime }B_{2}^{\prime }Q^{-1\prime }\left( \widehat{%
\underline{F}}_{T}-Q^{\prime }\underline{F}_{T}\right) \right\vert
+\left\vert a^{\prime }\left( \widehat{B}_{2}-Q^{-1}B_{2}\right) ^{\prime
}Q^{\prime }\underline{F}_{T}\right\vert
\end{eqnarray*}%
Lemma D-18 and Slutsky's theorem directly imply that%
\begin{equation*}
\left\vert a^{\prime }\left( \widehat{\beta }_{0}-\beta _{0}\right)
\right\vert =o_{p}\left( 1\right)
\end{equation*}%
Now, applying the CS inequality, we obtain%
\begin{eqnarray*}
\left\vert a^{\prime }\left( \widehat{B}_{1}-B_{1}\right) ^{\prime }%
\underline{Y}_{T}\right\vert &\leq &\sqrt{a^{\prime }\left( \widehat{B}%
_{1}-B_{1}\right) ^{\prime }\left( \widehat{B}_{1}-B_{1}\right) a}\sqrt{%
\underline{Y}_{T}^{\prime }\underline{Y}_{T}} \\
&=&\sqrt{a^{\prime }\left( \widehat{B}_{1}-B_{1}\right) ^{\prime }\left( 
\widehat{B}_{1}-B_{1}\right) a}\left\Vert \underline{Y}_{T}\right\Vert
_{2}^{2},
\end{eqnarray*}%
and%
\begin{eqnarray*}
&&\left\vert a^{\prime }\left( \widehat{B}_{2}-Q^{-1}B_{2}\right) ^{\prime
}Q^{\prime }\underline{F}_{T}\right\vert \\
&\leq &\sqrt{a^{\prime }\left( \widehat{B}_{2}-Q^{-1}B_{2}\right) ^{\prime
}\left( \widehat{B}_{2}-Q^{-1}B_{2}\right) a}\sqrt{\underline{F}_{T}^{\prime
}QQ^{\prime }\underline{F}_{T}} \\
&=&\sqrt{a^{\prime }\left( \widehat{B}_{2}-Q^{-1}B_{2}\right) ^{\prime
}\left( \widehat{B}_{2}-Q^{-1}B_{2}\right) a}\sqrt{\underline{F}_{T}^{\prime
}\left( \frac{\Gamma ^{\prime }\Gamma }{N_{1}}\right) ^{1/2}\Xi \widehat{V}%
\widehat{V}^{\prime }\Xi ^{\prime }\left( \frac{\Gamma ^{\prime }\Gamma }{%
N_{1}}\right) ^{1/2}\underline{F}_{T}} \\
&=&\sqrt{a^{\prime }\left( \widehat{B}_{2}-Q^{-1}B_{2}\right) ^{\prime
}\left( \widehat{B}_{2}-Q^{-1}B_{2}\right) a}\sqrt{\underline{F}_{T}^{\prime
}\left( \frac{\Gamma ^{\prime }\Gamma }{N_{1}}\right) \underline{F}_{T}} \\
&\leq &\sqrt{\lambda _{\max }\left( \frac{\Gamma ^{\prime }\Gamma }{N_{1}}%
\right) }\sqrt{a^{\prime }\left( \widehat{B}_{2}-Q^{-1}B_{2}\right) ^{\prime
}\left( \widehat{B}_{2}-Q^{-1}B_{2}\right) a}\left\Vert \underline{F}%
_{T}\right\Vert _{2}^{2} \\
&\leq &\overline{C}\sqrt{a^{\prime }\left( \widehat{B}_{2}-Q^{-1}B_{2}%
\right) ^{\prime }\left( \widehat{B}_{2}-Q^{-1}B_{2}\right) a}\left\Vert 
\underline{F}_{T}\right\Vert _{2}^{2}
\end{eqnarray*}%
Moreover, note that%
\begin{eqnarray*}
E\left[ \left\Vert \underline{Y}_{T}\right\Vert _{2}^{2}\right] &\leq
&\left( E\left\Vert \underline{Y}_{T}\right\Vert _{2}^{6}\right) ^{\frac{%
{\Large 1}}{{\Large 3}}}\text{ \ }\left( \text{by Liapunov's inequality}%
\right) \\
&\leq &\overline{C}^{\frac{{\Large 1}}{{\Large 3}}}=C<\infty \text{ \ }%
\left( \text{by Lemma C-5}\right)
\end{eqnarray*}%
and%
\begin{eqnarray*}
E\left[ \left\Vert \underline{F}_{T}\right\Vert _{2}^{2}\right] &\leq
&\left( E\left\Vert \underline{F}_{T}\right\Vert _{2}^{6}\right) ^{\frac{%
{\Large 1}}{{\Large 3}}}\text{ \ }\left( \text{by Liapunov's inequality}%
\right) \\
&\leq &\overline{C}^{\frac{{\Large 1}}{{\Large 3}}}=C<\infty \text{ \ }%
\left( \text{by Lemma C-5}\right)
\end{eqnarray*}%
Hence, for any $\epsilon >0$, set $C_{\epsilon }=\sqrt{C/\epsilon }$, and
Markov's inequality then implies that, for all $T>p-1$, 
\begin{equation*}
\Pr \left\{ \left\Vert \underline{Y}_{T}\right\Vert _{2}\geq C_{\epsilon
}\right\} =\Pr \left\{ \left\Vert \underline{Y}_{T}\right\Vert _{2}^{2}\geq
C_{\epsilon }^{2}\right\} \leq \frac{E\left[ \left\Vert \underline{Y}%
_{T}\right\Vert _{2}^{2}\right] }{C_{\epsilon }^{2}}=\frac{\epsilon E\left[
\left\Vert \underline{Y}_{T}\right\Vert _{2}^{2}\right] }{C}\leq \epsilon
\end{equation*}%
from which it follows that%
\begin{equation*}
\left\Vert \underline{Y}_{T}\right\Vert _{2}=O_{p}\left( 1\right) \text{.}
\end{equation*}%
In a similar way, we can also show that%
\begin{equation*}
\left\Vert \underline{F}_{T}\right\Vert _{2}=O_{p}\left( 1\right) \text{.}
\end{equation*}%
Application of the result given in Lemma D-18 then allows us to deduce that%
\begin{equation*}
\left\vert a^{\prime }\left( \widehat{B}_{1}-B_{1}\right) ^{\prime }%
\underline{Y}_{T}\right\vert \leq \sqrt{a^{\prime }\left( \widehat{B}%
_{1}-B_{1}\right) ^{\prime }\left( \widehat{B}_{1}-B_{1}\right) a}\left\Vert 
\underline{Y}_{T}\right\Vert _{2}^{2}=o_{p}\left( 1\right)
\end{equation*}%
and 
\begin{eqnarray*}
&&\left\vert a^{\prime }\left( \widehat{B}_{2}-Q^{-1}B_{2}\right) ^{\prime
}Q^{\prime }\underline{F}_{T}\right\vert \\
&\leq &\sqrt{a^{\prime }\left( \widehat{B}_{2}-Q^{-1}B_{2}\right) ^{\prime
}\left( \widehat{B}_{2}-Q^{-1}B_{2}\right) a}\sqrt{\underline{F}_{T}^{\prime
}QQ^{\prime }\underline{F}_{T}} \\
&\leq &\sqrt{a^{\prime }\left( \widehat{B}_{2}-Q^{-1}B_{2}\right) ^{\prime
}\left( \widehat{B}_{2}-Q^{-1}B_{2}\right) a}\sqrt{\lambda _{\max }\left(
QQ^{\prime }\right) }\left\Vert \underline{F}_{T}\right\Vert _{2} \\
&=&\sqrt{a^{\prime }\left( \widehat{B}_{2}-Q^{-1}B_{2}\right) ^{\prime
}\left( \widehat{B}_{2}-Q^{-1}B_{2}\right) a}\sqrt{\lambda _{\max }\left\{
\left( \frac{\Gamma ^{\prime }\Gamma }{N_{1}}\right) ^{\frac{{\Large 1}}{%
{\Large 2}}}\Xi \widehat{V}\widehat{V}^{\prime }\Xi ^{\prime }\left( \frac{%
\Gamma ^{\prime }\Gamma }{N_{1}}\right) ^{\frac{{\Large 1}}{{\Large 2}}%
}\right\} }\left\Vert \underline{F}_{T}\right\Vert _{2} \\
&=&\sqrt{a^{\prime }\left( \widehat{B}_{2}-Q^{-1}B_{2}\right) ^{\prime
}\left( \widehat{B}_{2}-Q^{-1}B_{2}\right) a}\sqrt{\lambda _{\max }\left\{
\left( \frac{\Gamma ^{\prime }\Gamma }{N_{1}}\right) \right\} }\left\Vert 
\underline{F}_{T}\right\Vert _{2} \\
&&\left( \text{since }\widehat{V}\widehat{V}^{\prime }=I_{Kp}\text{ and }\Xi
\Xi ^{\prime }=I_{Kp}\right) \\
&\leq &\sqrt{\overline{C}}\sqrt{a^{\prime }\left( \widehat{B}%
_{2}-Q^{-1}B_{2}\right) ^{\prime }\left( \widehat{B}_{2}-Q^{-1}B_{2}\right) a%
}\left\Vert \underline{F}_{T}\right\Vert _{2}\text{ \ }\left( \text{by
Assumption 3-6}\right) \\
&=&o_{p}\left( 1\right)
\end{eqnarray*}%
In addition, we can apply the CS inequality to get%
\begin{eqnarray*}
&&\left\vert a^{\prime }\left( \widehat{B}_{2}-Q^{-1}B_{2}\right) ^{\prime
}\left( \widehat{\underline{F}}_{T}-Q^{\prime }\underline{F}_{T}\right)
\right\vert \\
&\leq &\sqrt{a^{\prime }\left( \widehat{B}_{1}-B_{1}\right) ^{\prime }\left( 
\widehat{B}_{1}-B_{1}\right) a}\sqrt{\left( \widehat{\underline{F}}%
_{T}-Q^{\prime }\underline{F}_{T}\right) ^{\prime }\left( \widehat{%
\underline{F}}_{T}-Q^{\prime }\underline{F}_{T}\right) } \\
&\leq &\sqrt{a^{\prime }\left( \widehat{B}_{1}-B_{1}\right) ^{\prime }\left( 
\widehat{B}_{1}-B_{1}\right) a}\left\Vert \widehat{\underline{F}}%
_{T}-Q^{\prime }\underline{F}_{T}\right\Vert _{2} \\
&=&o_{p}\left( 1\right) \text{ \ }\left( \text{by Lemma D-18 and part (j) of
Lemma D-15 in Appendix D}\right)
\end{eqnarray*}%
and%
\begin{eqnarray*}
&&\left\vert a^{\prime }B_{2}^{\prime }Q^{-1\prime }\left( \widehat{%
\underline{F}}_{T}-Q^{\prime }\underline{F}_{T}\right) \right\vert \\
&\leq &\sqrt{a^{\prime }B_{2}^{\prime }Q^{-1\prime }Q^{-1}B_{2}a}\sqrt{%
\left( \widehat{\underline{F}}_{T}-Q^{\prime }\underline{F}_{T}\right)
^{\prime }\left( \widehat{\underline{F}}_{T}-Q^{\prime }\underline{F}%
_{T}\right) } \\
&=&\sqrt{a^{\prime }B_{2}^{\prime }Q^{-1\prime }Q^{-1}B_{2}a}\left\Vert 
\widehat{\underline{F}}_{T}-Q^{\prime }\underline{F}_{T}\right\Vert _{2} \\
&\leq &\sqrt{\left[ \lambda _{\min }\left( \frac{\Gamma ^{\prime }\Gamma }{%
N_{1}}\right) \right] ^{-1}\lambda _{\max }\left( B_{2}^{\prime
}B_{2}\right) }\left\Vert \widehat{\underline{F}}_{T}-Q^{\prime }\underline{F%
}_{T}\right\Vert _{2}\text{ } \\
&\leq &\sqrt{C^{\ast }}\left\Vert \widehat{\underline{F}}_{T}-Q^{\prime }%
\underline{F}_{T}\right\Vert _{2}\text{ \ }\left( \text{for some positive
constant }C^{\ast }\text{ as shown in}\right. \text{ } \\
&&\left. \text{expression (\ref{quad form HinvB2}) in Appendix D. See the
proof of part (d) of Lemma D-17}\right) \\
&=&o_{p}\left( 1\right) \text{ \ }\left( \text{by part (j) of Lemma D-15}%
\right)
\end{eqnarray*}%
Putting everything together and applying Slutsky's theorem, we then obtain 
\begin{eqnarray*}
&&\left\vert a^{\prime }\widehat{Y}_{T{\LARGE +}h}-a^{\prime }\left( \beta
_{0}+B_{1}^{\prime }\underline{Y}_{T}+B_{2}^{\prime }\underline{F}%
_{T}\right) \right\vert \\
&\leq &\left\vert a^{\prime }\left( \widehat{\beta }_{0}-\beta _{0}\right)
\right\vert +\left\vert a^{\prime }\left( \widehat{B}_{1}-B_{1}\right)
^{\prime }\underline{Y}_{T}\right\vert +\left\vert a^{\prime }\left( 
\widehat{B}_{2}-Q^{-1}B_{2}\right) ^{\prime }\left( \widehat{\underline{F}}%
_{T}-Q^{\prime }\underline{F}_{T}\right) \right\vert \\
&&+\left\vert a^{\prime }B_{2}^{\prime }Q^{-1\prime }\left( \widehat{%
\underline{F}}_{T}-Q^{\prime }\underline{F}_{T}\right) \right\vert
+\left\vert a^{\prime }\left( \widehat{B}_{2}-Q^{-1}B_{2}\right) ^{\prime
}Q^{\prime }\underline{F}_{T}\right\vert \\
&=&o_{p}\left( 1\right) \text{.}
\end{eqnarray*}%
Since the above argument holds for all $a\in \mathbb{R}^{d}$ such that $%
\left\Vert a\right\Vert _{2}=1$, we further deduce that%
\begin{equation*}
\widehat{Y}_{T{\LARGE +}h}-\left( \beta _{0}+B_{1}^{\prime }\underline{Y}%
_{T}+B_{2}^{\prime }\underline{F}_{T}\right) =o_{p}\left( 1\right) \text{.}
\end{equation*}%
as required. $\square $\noindent

\bigskip

\begin{thebibliography}{99}
\bibitem{} Anatolyev, S. and A. Mikusheva (2021): \textquotedblleft Factor
Models with Many Assets: Strong Factors, Weak Factors, and the Two-Pass
Procedure,\textquotedblright\ \textit{Journal of Econometrics}, forthcoming.

\bibitem{} Andrews, D.W.K. (1984): \textquotedblleft Non-strong Mixing
Autoregressive Processes,\textquotedblright\ \textit{Journal of Applied
Probability}, 21, 930-934.

\bibitem{} Bai, J. and S. Ng (2002): \textquotedblleft Determining the
Number of Factors in Approximate Factor Models,\textquotedblright\ \textit{%
Econometrica}, 70, 191-221.

\bibitem{} Bai, J. (2003): \textquotedblleft Inferential Theory for Factor
Models of Large Dimensions,\textquotedblright\ \textit{Econometrica}, 71,
135-171.

\bibitem{} Bai, J. and S. Ng (2008): \textquotedblleft Forecasting Economic
Time Series Using Targeted Predictors,\textquotedblright\ \textit{Journal of
Econometrics}, 146, 304-317.

\bibitem{} Bai, J. and S. Ng (2021): \textquotedblleft Approximate Factor
Models with Weaker Loading,\textquotedblright\ Working Paper, Columbia
University.

\bibitem{} Bai, Z. D. and Y. Q. Yin (1993): \textquotedblleft Limit of the
Smallest Eigenvalue of a Large Dimensional Sample Covariance Matrix," 
\textit{Annals of Probability}, 21, 1275-1294.

\bibitem{} Bair, E., T. Hastie, D. Paul, and R. Tibshirani (2006):
\textquotedblleft Prediction by Supervised Principal
Components,\textquotedblright\ \textit{Journal of the American Statistical
Association}, 101, 119-137.

\bibitem{} Belloni, A., D. Chen, V. Chernozhukov, and C. Hansen (2012):
\textquotedblleft Sparse Models and Methods for Optimal Instruments with an
Application to Eminent Domain,\textquotedblright\ \textit{Econometrica}, 80,
2369-2429.

\bibitem{} Billingsley, P. (1995): \textit{Probability and Measure}. New
York: John Wiley \& Sons.

\bibitem{} Borovkova, S., R. Burton, and H. Dehling (2001):
\textquotedblleft Limit Theorems for Functionals of Mixing Processes to
U-Statistics and Dimension Estimation,\textquotedblright\ \textit{%
Transactions of the American Mathematical Society}, 353, 4261-4318.

\bibitem{} Bryzgalova, S. (2016): \textquotedblleft Spurious Factors in
Linear Asset Pricing Models,\textquotedblright\ Working Paper, Stanford
Graduate School of Business.

\bibitem{} Burnside, C. (2016): \textquotedblleft Identification and
Inference in Linear Stochastic Discount Factor Models with Excess
Returns,\textquotedblright\ \textit{Journal of Financial Econometrics}, 14,
295-330.

\bibitem{} Chao, J. C. and N. R. Swanson (2022a): \textquotedblleft
Selecting the Relevant Variables for Factor Estimation in a Factor-Augmented
VAR Model,\textquotedblright\ Working Paper, Rutgers University and
University of Maryland.

\bibitem{} Chao, J. C. and N. R. Swanson (2022b): Technical Appendix to
\textquotedblleft Consistent Estimation, Variable Selection, and Forecasting
in Factor-Augmented VAR Models,\textquotedblright\ Working Paper, Rutgers
University and University of Maryland.

\bibitem{} Chao, J. C. and N. R. Swanson (2022c): Texchnical Appendix to
"Consistent Factor Estimation and Forecasting in Factor-Augmented VAR
Models,\textquotedblright\ Working Paper, Rutgers University and University
of Maryland.

\bibitem{} Chen, X., Q. Shao, W. B. Wu, and L. Xu (2016): \textquotedblleft
Self-normalized Cram\'{e}r-type Moderate Deviations under
Dependence,\textquotedblright\ \textit{Annals of Statistics}, 44, 1593-1617.

\bibitem{} Davidson. J. (1994): \textit{Stochastic Limit Theory: An
Introduction for Econometricians}. New York: Oxford University Press.

\bibitem{} Davidson, K. R. and S. J. Szarek (2001): \textquotedblleft Local
Operator Theory, Random Matrices and Banach Spaces.\textquotedblright\ In 
\textit{Handbook of the Geometry of Banach Spaces}, 1, 317-366. Amsterdam:
North-Holland.

\bibitem{} Fan, J., Y. Liao, and M. Mincheva (2011): \textquotedblleft
High-dimensional Covariance Matrix Estimation in Approximate Factor
Models,\textquotedblright\ \textit{Annals of Statistics}, 39, 3320-3356.

\bibitem{} Fan, J., Y. Liao, and M. Mincheva (2013): \textquotedblleft Large
Covariance Estimation by Thresholding Principal Orthogonal Complements," 
\textit{Journal of the Royal Statistical Society, Series B}, 75, 603-680.

\bibitem{} Freyaldenhoven, S. (2021a): \textquotedblleft Factor Models with
Local Factors - Determining the Number of Relevant
Factors,\textquotedblright\ \textit{Journal of Econometrics}, forthcoming.

\bibitem{} Freyaldenhoven, S. (2021b): \textquotedblleft Identification
through Sparsity in Factor Models: The $\ell _{1}$-Rotation
Criterion,\textquotedblright\ Working Paper, Federal Reserve Bank of
Philadelphia.

\bibitem{} Giglio, S., D. Xiu, and D. Zhang (2021): \textquotedblleft Test
Assets and Weak Factors,\textquotedblright\ Working Paper, Yale School of
Management and the Booth School of Business, University of Chicago.

\bibitem{} Golub, G. H. and C. F. van Loan (1996): \textit{Matrix
Computations}, 3rd Edition. Baltimore: The Johns Hopkins University Press.

\bibitem{} Goroketskii, V. V. (1977): \textquotedblleft On the Strong Mixing
Property for Linear Sequences,\textquotedblright\ \textit{Theory of
Probability and Applications}, 22, 411-413.

\bibitem{} Gospodinov, N., R. Kan, and C. Robotti (2017): \textquotedblleft
Spurious Inference in Reduced-Rank Asset Pricing Models,\textquotedblright\ 
\textit{Econometrica}, 85, 1613-1628.

\bibitem{} Harding, M. C. (2008): \textquotedblleft Explaining the Single
Factor Bias of Arbitrage Pricing Models in Finite
Samples,\textquotedblright\ \textit{Economics Letters}, 99, 85-88.

\bibitem{} Horn, R. and C. Johnson (1985): \textit{Matrix Analysis}.
Cambridge University Press.

\bibitem{} Jagannathan, R. and Z. Wang (1998): \textquotedblleft An
Asymptotic Theory for Estimating Beta-Pricing Models Using Cross-Sectional
Regression,\textquotedblright\ \textit{Journal of Finance}, 53, 1285-1309.

\bibitem{} Johnstone, I. M. and A. Lu (2009): \textquotedblleft On
Consistency and Sparsity for Principal Components Analysis in High
Dimensions,\textquotedblright\ \textit{Journal of the American Statistical
Association}, 104, 682-697.

\bibitem{} Johnstone, I. M. and D. Paul (2018): \textquotedblleft PCA in
High Dimensions: An Orientation,\textquotedblright\ \textit{Proceedings of
the IEEE}, 106, 1277-1292.

\bibitem{} Kan, R. and C. Zhang (1999): \textquotedblleft Two-Pass Tests of
Asset Pricing Models with Useless Factors,\textquotedblright\ \textit{%
Journal of Finance}, 54, 203-235.

\bibitem{} Kleibergen, F. (2009): \textquotedblleft Tests of Risk Premia in
Linear Factor Models,\textquotedblright\ \textit{Journal of Econometrics},
149, 149-173.

\bibitem{} L\"{u}tkepohl, H. (2005): \textit{New Introduction to Multiple
Time Series Analysis}. New York: Springer.

\bibitem{} Nadler, B. (2008): \textquotedblleft Finite Sample Approximation
Results for Principal Component Analysis: A Matrix Perturbation
Approach,\textquotedblright\ \textit{Annals of Statistics}, 36, 2791-2817.

\bibitem{} Onatski, A. (2012): \textquotedblleft Asymptotics of the
Principal Components Estimator of Large Factor Models with Weakly
Influential Factors,\textquotedblright\ \textit{Journal of Econometrics},
168, 244-258.

\bibitem{} Paul, D. (2007): \textquotedblleft Asymptotics of Sample
Eigenstructure for a Large Dimensional Spiked Covariance
Model,\textquotedblright\ \textit{Statistica Sinica}, 17, 1617-1642.

\bibitem{} Pham, T. D. and L. T. Tran (1985): \textquotedblleft Some Mixing
Properties of Time Series Models,\textquotedblright\ \textit{Stochastic
Processes and Their Applications}, 19, 297-303.

\bibitem{} Ruhe, A. (1975): \textquotedblleft On the Closeness of
Eigenvalues and Singular Values for Almost Normal
Matrices,\textquotedblright\ \textit{Linear Algebra and Its Applications},
11, 87-94.

\bibitem{} Shen, D., H. Shen, H. Zhu, J.S. Marron (2016): \textquotedblleft
The Statistics and Mathematics of High Dimension Low Sample Size
Asymptotics,\textquotedblright\ \textit{Statistica Sinica}, 26, 1747-1770.

\bibitem{} Stewart, G.W. (1973): \textquotedblleft Error and Perturbation
Bounds for Subspaces Associated with Certain Eigenvalue
Problems,\textquotedblright\ \textit{SIAM Review}, 15, 727-764.

\bibitem{} Stewart, G.W. and J. Sun (1990): \textit{Matrix Perturbation
Theory}. Boston: Academic Press.

\bibitem{} Stock, J. H. and M. W. Watson (2002a): \textquotedblleft
Forecasting Using Principal Components from a Large Number of
Predictors,\textquotedblright\ \textit{Journal of the American Statistical
Association}, 97, 1167-1179.

\bibitem{} Stock, J. H. and M. W. Watson (2002b): \textquotedblleft
Macroeconomic Forecasting Using Diffusion Indexes,\textquotedblright\ 
\textit{Journal of Business and Economic Statistics}, 20, 147-162.

\bibitem{} Vershynin, R. (2012): \textquotedblleft Introduction to the
Non-asymptotic Analysis of Random Matrices,\textquotedblright\ In \textit{%
Compressed Sensing}, \textit{Theory and Applications, }210-268. Cambridge
University Press.

\bibitem{} Wang, W. and J. Fan (2017): \textquotedblleft Asymptotics of
Empirical Eigenstructure for High Dimensional Spiked
Covariance,\textquotedblright\ \textit{Annals of Statistics}, 45, 1342-1374.
\end{thebibliography}

\end{document}
