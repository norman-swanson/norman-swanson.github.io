%2multibyte Version: 5.50.0.2960 CodePage: 936
%% This document created by Scientific Word (R) Version 3.5

\documentclass[a4paper,amstex,11pt]{article}%
\usepackage{multirow}
\usepackage{enumerate}
\usepackage{graphics}
\usepackage{epsfig}
\usepackage{verbatim}
\usepackage{bm}
\usepackage{latexsym}
\usepackage{amsmath}
\usepackage{natbib}
\usepackage{amsmath,amsthm,amsfonts,natbib,bm,latexsym,enumerate,url}
\usepackage{graphicx}
\usepackage{amsfonts}
\usepackage{setspace}
\usepackage{amssymb}%
\setcounter{MaxMatrixCols}{30}
%TCIDATA{OutputFilter=latex2.dll}
%TCIDATA{Version=5.50.0.2960}
%TCIDATA{Codepage=936}
%TCIDATA{CSTFile=LaTeX article (bright).cst}
%TCIDATA{Created=Tue Nov 04 20:07:01 2003}
%TCIDATA{LastRevised=Thursday, June 05, 2014 16:14:54}
%TCIDATA{<META NAME="GraphicsSave" CONTENT="32">}
%TCIDATA{<META NAME="SaveForMode" CONTENT="1">}
%TCIDATA{BibliographyScheme=Manual}
%TCIDATA{<META NAME="DocumentShell" CONTENT="General\Blank Document">}
%TCIDATA{Language=American English}
%BeginMSIPreambleData
\providecommand{\U}[1]{\protect\rule{.1in}{.1in}}
%EndMSIPreambleData
\providecommand{\U}[1]{\protect \rule{.1in}{.1in}}
\providecommand{\U}[1]{\protect \rule{.1in}{.1in}}
\providecommand{\U}[1]{\protect \rule{.1in}{.1in}}
\renewcommand{\baselinestretch}{0.9}
\textwidth=6.5in \textheight=9in \oddsidemargin=0in
\evensidemargin=0in \topmargin=-0.25in
\renewcommand {\baselinestretch}{1.3}
\newtheorem{theorem}{Theorem}
\newtheorem{acknowledgement}[theorem]{Acknowledgement}
\newtheorem{algorithm}[theorem]{Algorithm}
\newtheorem{axiom}[theorem]{Axiom}
\newtheorem{case}[theorem]{Case}
\newtheorem{claim}[theorem]{Claim}
\newtheorem{conclusion}[theorem]{Conclusion}
\newtheorem{condition}[theorem]{Condition}
\newtheorem{conjecture}[theorem]{Conjecture}
\newtheorem{corollary}{Corollary}
\newtheorem{criterion}[theorem]{Criterion}
\newtheorem{definition}[theorem]{Definition}
\newtheorem{example}[theorem]{Example}
\newtheorem{exercise}[theorem]{Exercise}
\newtheorem{lemma}{Lemma}
\newtheorem{assumption}{Assumption}
\newtheorem{notation}[theorem]{Notation}
\newtheorem{problem}[theorem]{Problem}
\newtheorem{proposition}{Proposition}
\theoremstyle{remark}
\newtheorem{remark}{Remark}
\newtheorem{solution}[theorem]{Solution}
\newtheorem{summary}[theorem]{Summary}
\begin{document}

\begin{center}
\bigskip\bigskip{\LARGE Consistent Pretesting for Jumps* }\bigskip\bigskip

\bigskip\bigskip

{\Large Valentina Corradi}$^{1}${\Large , Mervyn J. Silvapulle}$^{2}%
${\Large \ and Norman R. Swanson}$^{3}$

$^{1}${\Large University of Surrey,\ }$^{2}${\Large Monash University\ and
}$^{3}${\Large Rutgers University}{\large \bigskip\bigskip}

{\large June 2014}


\end{center}

\begin{spacing}{1.0}
{\footnotesize Abstract:     If the intensity parameter in a
jump diffusion model is identically
zero, then parameters characterizing the jump size density cannot be
identified. In general, this lack of identification precludes
consistent estimation of identified parameters.
Hence, it should be standard practice to consistently pretest for jumps, prior to estimating jump diffusions.  Many currently
available tests have power against the presence of jumps over a finite
time span (typically a day or a week); and, as already noted by various authors,
jumps may not be observed over finite time spans, even if the intensity
parameter is strictly positive. Such tests cannot be consistent against non-zero intensity.
Moreover, sequential application of finite time span tests
usually leads to sequential testing bias,
which in turn leads to jump discovery with probability one, in the limit, even if the true intensity is identically zero.
This paper introduces tests for jump intensity, based on both in-fill and long-span asymptotics,
which solve both the test consistency and the sequential testing bias problems discussed above, in turn facilitating
consistent estimation of jump diffusion models.
A \textquotedblleft self
excitement\textquotedblright \ test is also introduced, which is designed to have power against path
dependent intensity, thus providing a
direct test for the Hawkes diffusion model of Ait-Sahalia, Cacho-Diaz and
Laeven (2013). In a series of Monte Carlo experiments, the proposed tests are
evaluated, and are found to perform adequately in finite samples.}
\end{spacing}

\bigskip

\bigskip

\bigskip

\bigskip

\bigskip

\bigskip

\noindent\textit{Keywords}: diffusion model, jump intensity, jump size
density, sequential testing bias, bootstrap.

\noindent\textit{JEL classification}: C12, C22, C52, C55.

\renewcommand {\baselinestretch}{1.0}

\_ \_ \_ \_ \_ \_ \_ \_ \_ \_ \_ \_ \_ \_ \_ \_ \_ \_ \_ \_ \_ \_ \_ \_ \_ \_ \_

{\small \renewcommand {\baselinestretch}{1.0} }

\begin{spacing}{1.0}
{\small $^{\ast}$ }{\footnotesize Valentina Corradi, School of Economics,
University of Surrey, Guildford, Surrey, GU2 7XH, UK, v.corradi@surrey.ac.uk.\ Mervyn
Silvapulle, Department of Econometrics and Business Statistics, Monash
University Caulfield, 26 Sir John Monash Drive, Caulfield East, Victoria 3145,
Australia, mervyn.silvapulle@monash.edu. Norman Swanson, Department of
Economics, Rutgers University, 75 Hamilton Street, New Brunswick, NJ 08901,
USA, nswanson@econ.rutgers.edu. We are grateful to Francesco Bravo, Laura
Coroneo, Prosper Donovon, Rustam Ibragimov, Peter Spencer, and to
seminar participants at the University of York, the 2013 Conputational and
Financial Econometrics Conference in London, and the 2014 CIREQ
Time Series and Financial Econometrics conference in Montreal for useful comments
and suggestions.}
\end{spacing}


\renewcommand {\baselinestretch}{1.3}

\newpage

\clearpage


\allowdisplaybreaks


\section{Introduction}

Jump diffusions are widely used in the financial econometrics literature when
analyzing returns or exchange rates, for example, as discussed in Duffie, Pan
and Singleton (2000), Singleton (2001), Anderson, Benzoni and Lund (2002),
Jiang and Knight (2002), Chacko and Viceira (2003) and Eraker, Johannes and
Polson (2003), among others. Various estimation techniques have been
developed, and the common practice is to jointly estimate the parameters of
both the continuous time and the jump components of these diffusion models.
Thus, parameters characterizing the drift, variance, jump intensity and jump
size probability density are jointly estimated. However, an obvious
non-standard feature of this class of models is that the parameters
characterizing the jump size density are not identified when the jump
intensity is identically zero. This is an issue both when the intensity
parameter is constant, as in standard stochastic volatility models with jumps
(see, e.g. Andersen, Benzoni and Lund (2002)) as well as when the intensity
follows a diffusion process, as in the important case of the Hawkes diffusion
models analyzed by Ait-Sahalia, Cacho-Diaz and Laeven (2013). Clearly, when
one estimates a jump diffusion with jump intensity equal to zero, a subset of
the parameters is not identified. This in turn precludes consistent estimation
of other parameters in the model (see Andrews and Cheng (2012)).

The above estimation problem serves to underscore the importance of pretesting
for jumps. In the extant literature, there is a large variety of tests for the
null of no jumps versus the alternative of jumps. Tests include those based on
the comparison of two realized volatility measures, one which is robust, and
the other which is not robust to the presence of jumps (see, e.g.
Barndorff-Nielsen, Shephard and Winkel (2006) and Podolskji and Vetter
(2009a)), tests based on a thresholding approach (see, e.g. Corsi, Pirino, and
Reno (2010)), and tests based on power variation, as discussed in Ait-Sahalia
and Jacod (2009). All of these tests are carried out using a finite time span
(typically a day or a week), and limiting distributions are found using
in-fill asymptotic approximations. However, over a finite time span we may not
observe jumps, even if the intensity parameter is positive. Thus, these tests
are not consistent against the alternative of positive jump intensity, as
pointed out by Huang and Tauchen (2005) and Ait-Sahalia and Jacod (2009), for
example. Moreover, sequential application of these tests (again, typically
done daily or weekly) results in a failure to control the probability of false
jump discovery. This is because of the well known sequential testing size
distortion problem. This paper develops new and practical methods for solving
the above testing and associated estimation problems in the context of the
specification of jump diffusion models.

Consider solving the above problem of pretesting and subsequent estimation in
stages by first testing the null of zero versus positive intensity using a
score, Wald or likelihood ratio test, as in Andrews (2001), and subsequently
estimating the model using standard techniques. This would involve treating
the parameters of the jump size density as nuisance parameters unidentified
under the null. Furthermore, such an approach would require correct
specification of both the continuous and the jump components of the diffusion;
and misspecification of one or both components would in general affect the
overall outcome of the test. Just as importantly, the likelihood function of a
jump diffusion is not generally known in closed form, and therefore estimation
(which is needed for construction of these jump tests) is usually based on
either simulated GMM (see, e.g. Duffie and Singleton (1993) and Anderson,
Benzoni and Lund (2002)); Indirect Inference (see, e.g. Gourieroux and Monfort
(1993) and Gallant and Tauchen (1996)); or Nonparametric Simulated Maximum
Likelihood (see, e.g. Fermanian and Salanie (2004) and Corradi and Swanson
(2011)). However, it goes without saying that one cannot simulate a diffusion
with a negative intensity parameter. This, in turn, precludes the existence of
a quadratic approximation around the null parameters of the criterion function
to be maximized (minimized). Given that the existence of such quadratic
approximations is a necessary condition for estimation and inference about
parameters on the boundary (see, e.g. Andrews (1999,2001), Beg, Silvapulle and
Silvapulle (2001), and Chapter 4 in Silvapulle and Sen (2005)), we cannot rely
on simulation-based estimators if attempting to pretest using standard score,
Wald or likelihood ratio tests. A different variety of jump pretest is instead
required. The approach taken in this paper is to propose model free jump
pretests, and to subsequently estimate the jump diffusion using standard
estimation techniques, depending upon the outcome of the test(s).

In particular, this paper makes two key contributions to the literature.
First, we introduce model free \textquotedblleft jump\textquotedblright\ tests
for the null of zero intensity. The tests are based on both in-fill and
long-span asymptotics, thus addressing the issues of consistency and
sequential testing bias discussed above. Second, under the maintained
assumption of strictly positive intensity, we introduce a \textquotedblleft
self excitement\textquotedblright\ test for the null of constant intensity
against the alternative of path dependent intensity. The objective in this
context is the provision of a direct test for Hawkes diffusions (see
Ait-Sahalia, Cacho-Diaz and Laeven (2013)) in which jump intensity is modeled
as a mean-reverting diffusion process. When the tests are implemented prior to
model specification, standard estimation of jump diffusions can be
subsequently carried out, avoiding the identification problems discussed above.

The jump tests are based on sample third moments, and are constructed using a
long time span of high frequency observations. Two versions of these tests are
discussed. One version does not allow for leverage, in the sense that
rejection of the null may be due either to the presence of jumps or due to the
presence of leverage effects in the underlying data generating process. An
alternative version is robust to leverage effects, with the caveat that it is
a less powerful test than its non-robust counterpart. The limiting behavior of
the proposed statistics can be readily analyzed via use of a double asymptotic
scheme wherein the time span goes to infinity and the discrete interval
approaches zero. The tests are model free, except for a drift component, which
is assumed to be constant\footnote{Recall that drift terms can be ignored over
finite time spans, while they have a non-negligible impact on asymptotic
approximations over long time spans. Hence, this \textquotedblleft
trade-off\textquotedblright\ is not surprising.}. Under the null hypothesis of
zero intensity, the statistics are characterized by normal limiting
distributions. Under the alternative, it is necessary to distinguish between
the case in which the density of the jumps is asymmetric and the case in which
the density is symmetric. In the former case, the proposed tests have a well
defined Pitman drift and have power against $\sqrt{T}-$local alternatives,
where $T$ denotes the time span. In the latter case, the sample third moment
approaches zero, but the probability order of the statistics is larger than
that which obtains under the null, since the jump component does not
contribute to the mean, while it does contribute to the variance. To ensure
power under both types of alternatives, it follows that we cannot rescale the
test statistics by an estimator of the variance. We instead construct
bootstrap critical values, the first order validity of which is established in
the sequel.

Turning now to the self excitement test, note that if the null of zero
intensity is rejected, one can proceed with a second test that is carried out
in order to ascertain whether jump intensity is a constant, or follows a
diffusion process, as in the case of Hawkes diffusions. This test is based on
the sample autocorrelation of the (log) first differences of the data, and is
analyzed using asymptotic approximations closely related to those used in the
analysis of the jump test statistics.

As none of the tests proposed in this paper are robust to microstructure
noise, one might choose to build a dataset consisting of observations at the
highest frequency for which the noise is not binding. However, the assumptions
posited in order to analyze the tests herein simply require that the discrete
interval approaches zero; and this does not have to occur at a minimum speed.
Indeed, in our framework the time span can grow faster than the discrete interval.

The finite sample behavior of the suggested statistics is studied via Monte
Carlo experimentation. The jump tests exhibit empirical size very close to
nominal and empirical power close to unity, across various empirically
motivated parameterizations. The self excitement test likewise has very good
size and good power properties, whenever there are \textquotedblleft
enough\textquotedblright\ jumps and the degree of self excitation is not
\textquotedblleft too weak\textquotedblright. In an examination of the finite
sample behavior of the tests, when carried out in sequence (i.e., carry out
the self excitement test in all instances for which the jump test rejects the
null of no jumps), we also find evidence of adequate performance.

The rest of the paper is organized as follows. Section 2 describes the set-up.
Section 3 provides heuristic arguments for the testing approach taken in this
paper. Section 4 discusses the jump and self excitement tests, derives their
asymptotic properties, and discusses asymptotically valid bootstrap based
inference using an $m$ out of $n$ bootstrap procedure. Section 5 reports the
findings of a Monte Carlo study designed to examine the finite sample
properties of the tests, and concluding remarks are gathered in Section 6. All
proofs are collected in an Appendix.

\section{Set-Up}

\noindent Consider the following jump diffusion model,%
\begin{equation}
\mathrm{d\ln}X_{t}=\mu\mathrm{d}t+\sqrt{V_{t}}\mathrm{d}W_{1,t}+Z_{t}%
\mathrm{d}N_{t}, \label{X}%
\end{equation}
where volatility $V_{t}$ is defined according to either (i), (ii), (iii), or
(iv), as follows:

(i) a constant:
\begin{equation}
V_{t}=v\text{ for all }t; \label{va}%
\end{equation}


(ii) a measurable function of the state variable:%
\begin{equation}
V_{t}\text{ is }X_{t}-\text{measurable}; \label{vb}%
\end{equation}


(iii) a stochastic volatility process without leverage:%
\begin{equation}
\mathrm{d}V_{t}=\mu_{V,t}(\theta)\mathrm{d}t+g\left(  V_{t},\theta\right)
\mathrm{d}W_{2,t},\text{ \textrm{E}}\left(  W_{1,t}W_{2,t}\right)  =0;
\label{vc}%
\end{equation}


(iv) a stochastic volatility process with leverage:%
\begin{equation}
\mathrm{d}V_{t}=\mu_{V,t}(\theta)\mathrm{d}t+g\left(  V_{t},\theta\right)
\mathrm{d}W_{2,t},\text{ \textrm{E}}\left(  W_{1,t}W_{2,t}\right)  =\rho\neq0.
\label{vd}%
\end{equation}
Here,%
\begin{equation}
\Pr\left(  N_{t+\Delta}-N_{t}=1|\mathcal{F}_{t}\right)  =\lambda_{t}%
\Delta+o\left(  \Delta\right)  , \label{J1}%
\end{equation}%
\begin{equation}
\Pr\left(  N_{t+\Delta}-N_{t}=0|\mathcal{F}_{t}\right)  =1-\lambda_{t}%
\Delta+o\left(  \Delta\right)  , \label{J2}%
\end{equation}
and%
\begin{equation}
\Pr\left(  N_{t+\Delta}-N_{t}>1|\mathcal{F}_{t}\right)  =o\left(
\Delta\right)  , \label{J3}%
\end{equation}
where $\mathcal{F}_{t}=\sigma\left(  N_{s},\text{ }0\leq s\leq t\right)  ,$
and the jump size, $Z_{t},$ is identically and independently distributed with
density $f(z;\gamma).$

We consider two general cases. The first is that of Poisson jumps, in which
$\lambda_{t}=\lambda,$ for all $t.$ The second is that of Hawkes diffusions,
in which the intensity is an increasing function of past jumps (see Bowsher
(2007) and Ait-Sahalia, Cacho-Diaz and Laeven (2013)). In this case:%

\[
\lambda_{t}=\lambda_{\infty}+\beta\int_{0}^{t}\exp\left(  -a\left(
t-s\right)  \right)  \mathrm{d}N_{s},
\]
with $\lambda_{\infty}\geq0,$ $\beta\geq0,$ $a>0,$ and $a>\beta$ (in order to
ensure intensity mean reversion). Thus,%
\begin{equation}
\mathrm{d}\lambda_{t}=a\left(  \lambda_{\infty}-\lambda_{t}\right)
\mathrm{d}t+\beta\mathrm{d}N_{s} \label{lambda}%
\end{equation}
and \textrm{E}$\left(  \lambda_{t}\right)  =\frac{a\lambda_{\infty}}{a-\beta
}.$ If $\lambda_{\infty}=0,$ then \textrm{E}$\left(  \lambda_{t}\right)  =0;$
and since $\lambda_{t}$ can never be negative, this in turn implies that
$\lambda_{t}=0$ a.s., for all $t$ (i.e., $N_{t}=0$ a.s., for all $t).$ But, if
$N_{t}=0$ a.s., for all $t,$ then $\beta$ cannot be identified, and
consequently $a$ is not identified. Furthermore, if $N_{t}=0$ a.s., for all
$t,$ then $\gamma$ cannot be identified. In summary, if $\lambda_{\infty}=0,$
$\beta,\alpha,\gamma$ are not identified. By contrast, if $\lambda_{\infty
}>0,$ then $\gamma$ and $\beta$ are identified. However, if $\lambda_{\infty
}>0$ but $\beta=0,$ $a$ is not identified. These observations highlight the
importance of being very clear as to which of the two assumptions,
$\lambda_{\infty}=0$ or $\lambda_{\infty}>0,$ is made for statistical
inference in the foregoing Hawkes diffusion model. In practice, thus, we are
concerned with the following hypotheses $H_{0}:\lambda_{\infty}=0$ versus
$H_{A}:\lambda_{\infty}>0.$ This is a nonstandard inference problem because,
under $H_{0},$ some parameters are not identified and a parameter lies on the
boundary of the null parameter space. Additionally, depending upon the outcome
of tests of the above hypotheses, we are also interested in the following
hypotheses (i.e., self excitement pretests): $H_{0}:\beta=0$ versus
$H_{A}:\beta>0.$

At this juncture, we provide further heuristic motivation by discussing
various key differences between existing jump tests and tests based directly
on testing the intensity proposed in the sequel.

\section{Testing for Jumps or Jump Intensity - Heuristic Arguments}

In recent years, a large variety of tests for jumps have been developed. One
common feature of these tests is that they are all performed using high
frequency observations over a finite time span. We thus argue that none of
these tests is consistent against the alternative $\lambda_{\infty}>0.$ Many
of the extant tests can be broadly classified as belonging in one of three
groups: (i) Hausman type tests (ii); threshold type tests; and (iii) higher
order power variation tests.

Hausman type tests are based on the comparison of non-robust and robust
realized volatility measures (see, e.g. Barndorff-Nielsen and Shephard (2004),
Barndorff-Nielsen, Shephard and Winkel (2006), and Huang and Tauchen (2005)).
More recently, related tests have been proposed that are based on comparisons
using pre-averaged volatility measures, in order to obtain tests that are
robust to microstructure noise (see, e.g. Podolskij and Vetter (2009a)).
Hausman type tests are able to detect whether $\sum_{j=N_{t}}^{N_{t+1}}%
c_{j}^{2}=0$ or $\sum_{j=N_{t}}^{N_{t+1}}c_{j}^{2}>0,$ where $N_{t}$ denotes
the number of jumps up to time $t,$ and $c_{j}$ is the (random) size of the
jumps. However, $\lambda_{\infty}>0$ does not imply that $\sum_{j=N_{t}%
}^{N_{t+1}}c_{j}^{2}>0,$ given that $\Pr\left(  N_{t+1}-N_{t}>0\right)  <1$.

Threshold type tests are based on the difference between \textquotedblleft
standard\textquotedblright\ volatility measures and trimmed realized measures,
where the trimming is implemented at a threshold level which allows for the
separation of jump and continuous components. Such tests have power against
jump size, but not necessarily against jump intensity (see, e.g. Corsi, Pirino
and Reno (2010) and Lee and Mykland (2008)).

A more recent class of tests is based on higher order power variation, and is
motivated by the fact that for $p>2,$ $\sum_{i=1}^{n-1}\left\vert
X_{(t+(i+1)\Delta}-X_{t+i\Delta}\right\vert ^{p}$ converges to $\sum_{t\leq
s\leq t+1}\left\vert X_{s}-X_{s-}\right\vert ^{p},$ where $\sum_{t\leq s\leq
t+1}\left\vert X_{s}-X_{s-}\right\vert ^{p}$ is strictly positive if there are
jumps and zero otherwise (see, e.g. Ait-Sahalia and Jacod (2009) and
Ait-Sahalia, Jacod and Li (2012)). Even in this case, power obtains because of
jump size, and not because of jump probability.

More generally, for tests performed on a finite time span, are able to
distinguish between\footnote{Jump test inconsistency has been pointed out by
Huang and Tauchen (2005) and Ait-Sahalia and Jacod (2009), among others.}:%
\[%
\begin{array}
[c]{c}%
\Omega_{t}^{c}=\left\{  \omega:s\rightarrow X_{s}\text{ is continuous on
}[t,t+1)\right\} \\
\mathrm{and}\\
\Omega_{t}^{j}=\left\{  \omega:s\rightarrow X_{s}\text{ has jumps on
}[t,t+1)\right\}  .
\end{array}
\]
Hence, all of the tests discussed above are dependent upon pathwise behavior.
Clearly, one might decide in favor of $\Omega_{t}^{c},$ even if $\lambda
_{\infty}>0,$ simply because jumps are not observed over the interval
$[t,t+1).$ Lee, Loretan and Ploberger (2013) discuss the optimality properties
of jump tests against local alternative defined in terms of jump sizes. It
follows that in order to carry out a consistent jump test, one must test the
composite hypothesis:%
\[
\Omega_{T}^{c}=\cap_{t=0}^{T-1}\Omega_{t}^{c},
\]
versus it negation. Broadly speaking, one must test the composite null
hypothesis that none of the daily (or weekly, say) paths contain jumps. In
fact, under mild conditions on the degree of heterogeneity of the process,
failure to reject $\Omega_{\infty}^{c}=\lim_{T\rightarrow\infty}\cap
_{t=1}^{T-1}\Omega_{t}^{c}$ implies failure to reject $\lambda_{\infty}=0.$
The difficulty herein lies in how to implement a test for $\Omega_{T}^{c},$
when $T$ gets large. Needless to say, sequential application of finite time
span jump tests leads to sequential test bias, and for $T$ large $\Omega
_{T}^{c}$ is rejected with probability going to unity. At issue here is the
control of overall size when testing composite hypotheses. One common approach
to this problem is based on controlling the overall Family-Wise Error-Rate
(FWER), which ensures that no single hypothesis is rejected at a level larger
than a fixed value, say $\alpha$. This is typically accomplished by sorting
individual $p-$values, and using a rejection rule which depends on the overall
number of hypotheses. For further discussion, see Holm (1979), who develops
modified Bonferroni bounds, White (2000), who develops the so-called
\textquotedblleft reality check\textquotedblright, and Romano and Wolff
(2005), who provide a refinement of the reality check. However, when the
number of hypotheses in the composite grows with the sample size, the null
will (almost) never be rejected. In other words, approaches based on the FWER
are far too conservative for our purpose.

An alternative approach, which allows for the number of hypotheses in the
composite to grow to infinity, is based on the Expected False Discovery Rate
(E-FDR). When using this approach, one controls the expected number of false
discoveries (rejections). For further discussion, see Benjamini and Hochberg
(1995) and Storey (2003). Although the E-FDR approach applies to the case of a
growing number of hypotheses, it is very hard to implement in the presence of
generic dependences across $p$-values, as it is in our context.

In summary, a key advantage of jump tests based on high frequency observations
over finite time spans is that they are virtually model free, as minimal
regularity assumptions on the underlying process are required. A key
disadvantage is that they are not consistent against the alternative of
positive jump intensity. On the other hand, if more structure is imposed, and
most importantly if the transition density is known in closed form, then it is
easy to construct a consistent test for jumps, based only on a long time span
of discrete observations. In particular one can easily test $H_{0}%
:\lambda_{\infty}=0$ against $H_{A}:\lambda_{\infty}>0$. This fact can be
illustrated by considering a score test. Suppose that the skeleton of the
process $\ln X_{t}$ in $Eq.$\textit{ }(\ref{X}) is observed. Namely, $\ln
X_{1},\ln X_{2},...,\ln X_{T},$ is observed, with $V_{t}$ defined as in $Eq$.
(\ref{vc}) or $Eq.$ (\ref{vd}); and for sake of simplicity suppose that
$\lambda_{t}=\lambda_{\infty}.$ Now, using the notation in $Eqs.$
(\ref{X})-(\ref{J3}), let $\delta=\left(  \theta,\mu,\rho,\lambda_{\infty
},\gamma\right)  =\left(  \vartheta,\gamma\right)  .$ It immediately follows
that, provided the transition density is known in closed form, the likelihood
can be written as:%
\[
l_{T}(\vartheta,\gamma)=\frac{1}{T}\sum_{t=1}^{T-1}l_{t}(\vartheta
,\gamma)=\frac{1}{T}\sum_{t=1}^{T-1}\ln f_{y+1|y}\left(  Y_{t+1}%
|Y_{t},\vartheta,\gamma\right)  .
\]
The score statistic for testing $H_{0}$ is thus\footnote{If $\lambda_{\infty}$
is not scalar (for example, consider allowing for different up and down jump
intensities, as in Chacko and Viceira (2003)), then the score statistic can be
written as:%
\begin{align*}
K\left(  \gamma\right)   &  =U_{T}\left(  \gamma\right)  ^{\prime}\left(
R\widehat{\mathcal{I}}_{T}(\gamma)^{-1}\widehat{V}_{T}(\gamma
)\widehat{\mathcal{I}}_{T}(\gamma)^{-1}R^{\prime}\right)  ^{-1}U_{T}\left(
\gamma\right) \\
&  -\inf_{\lambda\geq0}\left(  U_{T}\left(  \gamma\right)  -\lambda\right)
^{\prime}\left(  R\widehat{\mathcal{I}}_{T}(\gamma)^{-1}\widehat{V}_{T}%
(\gamma)\widehat{\mathcal{I}}_{T}(\gamma)^{-1}R^{\prime}\right)  ^{-1}\left(
U_{T}\left(  \gamma\right)  -\lambda\right)
\end{align*}
\par
{}}:%
\[
K_{T}\left(  \gamma\right)  =\max\left\{  0,\left(  R\widehat{\mathcal{I}}%
_{T}(\gamma)^{-1}\widehat{V}_{T}(\gamma)\widehat{\mathcal{I}}_{T}(\gamma
)^{-1}R^{\prime}\right)  ^{-1/2}U_{T}\left(  \gamma\right)  \right\}  ,
\]
where $R$ is a $1\times p$ matrix, with $p$ denoting the dimension of
$\vartheta.$ Additionally,%

\[
U_{T}\left(  \gamma\right)  =\sqrt{T}\left(  R\widehat{\mathcal{I}}_{T}%
(\gamma)^{-1}\nabla_{\vartheta}l_{T}\left(  \widehat{\vartheta}_{T}%
,\gamma\right)  \right)  ,
\]%
\begin{equation}
\widehat{\mathcal{I}}_{T}(\gamma)=\frac{1}{T}\sum_{t=1}^{T}\nabla
_{\vartheta\vartheta}l_{t}\left(  \widehat{\vartheta}_{T},\gamma\right)  ,
\label{HAT-HES}%
\end{equation}%
\[
\widehat{\vartheta}_{T}=\arg\min_{\vartheta}l_{T}\left(  \vartheta
,\gamma\right)  \text{ s.t. }R\vartheta=\lambda_{\infty}=0,
\]
and%
\begin{equation}
\widehat{V}_{T}(\gamma)=\frac{1}{T}\sum_{j=-\tau_{T}}^{\tau_{T}}\sum
_{t=\tau_{T}}^{T-\tau_{T}}\omega_{j}\nabla_{\vartheta}l_{t}\left(
\widehat{\vartheta}_{T},\gamma\right)  \nabla_{\vartheta}l_{t+j}\left(
\widehat{\vartheta}_{T},\gamma\right)  ^{\prime},\text{ }\omega_{j}=1-\frac
{j}{1+\tau_{T}}. \label{HAT-HAC}%
\end{equation}
Now, given mild regularity assumptions controlling the smoothness of the
likelihood, under the null of $\lambda_{\infty}=0,$%
\[
\sup_{\gamma\in\Gamma}K\left(  \gamma\right)  \overset{d}{\rightarrow}%
\sup_{\gamma\in\Gamma}\max\left\{  0,\left(  R\mathcal{I}(\gamma)^{-1}%
V(\gamma)\mathcal{I}(\gamma)^{-1}R^{\prime}\right)  ^{-1/2}Z\left(
\gamma\right)  \right\}  ,
\]
where $\sup_{\gamma\in\Gamma}\left\vert \widehat{\mathcal{I}}_{T}%
(\gamma)-\mathcal{I}(\gamma)\right\vert =o_{p}(1),$ $\sup_{\gamma\in\Gamma
}\left\vert \widehat{V}_{T}(\gamma)-V(\gamma)\right\vert =o_{p}(1),$ and
$Z(\cdot)$ is a Gaussian process with covariance kernel,%
\[
C(\gamma_{1},\gamma_{2})=\left(
\begin{array}
[c]{cc}%
R\mathcal{I}(\gamma_{1})^{-1}V(\gamma_{1},\gamma_{1})\mathcal{I}(\gamma
_{1})^{-1}R^{\prime} & R\mathcal{I}(\gamma_{1})^{-1}V(\gamma_{1},\gamma
_{2})\mathcal{I}(\gamma_{2})^{-1}R^{\prime}\\
R\mathcal{I}(\gamma_{2})^{-1}V(\gamma_{1},\gamma_{2})\mathcal{I}(\gamma
_{1})^{-1}R^{\prime} & R\mathcal{I}(\gamma_{2})^{-1}V(\gamma_{2},\gamma
_{2})\mathcal{I}(\gamma_{2})^{-1}R^{\prime}%
\end{array}
\right)  ,
\]
where $V(\gamma_{1},\gamma_{2})=\mathrm{p\lim_{T\rightarrow\infty}}%
\widehat{V}_{T}(\gamma_{1},\gamma_{2}).$

Note also that $\sup_{\gamma\in\Gamma}K\left(  \gamma\right)  $ diverges to
infinity under the alternative$.$ This test has power against $\sqrt{T}-$local
alternatives. Additionally, the limiting behavior of the test depends on the
quadratic approximation of the likelihood around $\lambda_{\infty}=0$ (see
Andrews (2001)). Hence, if the likelihood is known in closed form, and if both
the continuous and the jump components of the model are correctly specified,
then inference can be easily carried out using this score test, or using
analogous Wald or likelihood ratio tests. However, it is well known that for
interesting models the likelihood is usually not known in closed form. In such
cases, as discussed in the introduction, one often relies on simulation based
estimation techniques such as simulated GMM, indirect inference, or
nonparametric simulated maximum likelihood. However, as one cannot simulate
observations with negative intensity, a quadratic approximation of the
criterion function cannot be constructed, and these sorts of tests are not
applicable. It is for this reason that we instead focus on simple moment based
jump and self-excitement tests.

\section{Long Time Span Jump Tests}

\subsection{Test of $\lambda_{\infty}=0$ (no leverage effects)}

From the above discussion, recall that tests based on high frequency
observations over a finite time span are model free, but are not consistent
against the alternative $\lambda_{\infty}>0$. On the other hand, tests based
on discrete observations over a long time span are consistent against
$\lambda_{\infty}>0,$ but require correct specification of both the continuous
and jump components, as well as knowledge of the transition density. This is
because long time spans ensure test consistency against non-zero intensities
(note that the intensity parameter cannot be identified on a given finite time
span), while the use of high frequency observations does not require knowledge
of the parametric specification of the diffusion. In order to have tests that
are consistent against $\lambda_{\infty}>0,$ but still (almost) model free, we
use functions of sample moments and rely on double in-fill and long-time span
asymptotic approximations. The only (small) price to pay is that the drift
component is assumed to be constant. This follows because the drift term can
be ignored over a finite time span, while it has a non-negligible impact on
asymptotic approximations over a long time spans.

In the sequel, assume the existence of a sample of $n$ observations over an
increasing time span $T$ and a shrinking discrete interval $\Delta,$ so that
$n=\frac{T}{\Delta},$ with $T\rightarrow\infty$ and $\Delta\rightarrow0.$
Proceed in two steps. First test for zero jump intensity ($\lambda_{\infty
}=0)$. Then, if the null is rejected, test for path dependence ($\beta=0)$. In
the first step, thus, interest lies in the following hypotheses:%
\[
H_{0}:\lambda_{\infty}=0
\]%
\begin{align*}
H_{A}  &  =H_{A}^{(1)}\cup H_{A}^{(2)}:\left(  \lambda_{\infty}>0\text{ and
\textrm{E}}\left(  \left(  Z_{t}-\mathrm{E}\left(  Z_{t}\right)  \right)
^{3}\right)  \neq0\right) \\
&  \cup\left(  \lambda_{\infty}>0\text{ and \textrm{E}}\left(  \left(
Z_{t}-\mathrm{E}\left(  Z_{t}\right)  \right)  ^{3}\right)  =0\right)  .
\end{align*}
Notice that the alternative hypothesis is the union of two different
alternatives, designed to allow for both symmetric and asymmetric jump size
density. Let $Y_{k\Delta}=\ln X_{k\Delta}-\frac{\Delta}{T}\sum_{k=1}^{n}\ln
X_{k\Delta},$ and $Y_{(k-1)\Delta}=\ln X_{(k-1)\Delta}-\frac{\Delta}{T}%
\sum_{k=2}^{n}\ln X_{(k-1)\Delta}.$ Also, let:%
\[
\widehat{\lambda}_{T,\Delta}=\frac{1}{T}\sum_{k=2}^{n}\left(  Y_{k\Delta
}-Y_{(k-1)\Delta}\right)  ^{3},
\]
and define the statistic:%
\begin{equation}
S_{T,\Delta}=\frac{T^{1/2}}{\Delta}\widehat{\lambda}_{T,\Delta}. \label{St}%
\end{equation}


\noindent The asymptotic behavior of $S_{T,\Delta}$ is analyzed under the
following set of assumptions.

\noindent\textbf{Assumption A: (i)} $\ln X_{t}$ is generated by $Eq.$
(\ref{X}) and $V_{t}$ is defined in $Eq.$ (\ref{va}), (\ref{vb}), or
(\ref{vc}). \textbf{(ii)} $\ln X_{t}$ is generated by $Eq.$ (\ref{X}) and
$V_{t}$ is defined in $Eq.$ (\ref{vd}). For $C$ a generic constant,
(\textbf{iii}) \textrm{E}$\left(  \left\vert V_{t}\right\vert ^{k}\right)
\leq C$, $k\geq3,$ (\textbf{iv}) $N_{t}$ satisfies $Eqs.$ (\ref{J1}%
)-(\ref{J3}), and $\lambda_{t}$ is either constant or it satisfies $Eq.$
(\ref{lambda}). (\textbf{v}) The jump size, $Z_{t},$ is independently and
identically distributed, and $\mathrm{E}\left(  \left\vert Z_{t}\right\vert
^{k}\right)  \leq C,$ for $k\geq6.$\medskip

\noindent\textbf{Theorem 1:}\textit{ Let Assumptions A(i) and A(iii)-(v) hold.
Also, assume that as }$n\rightarrow\infty,$\textit{ }$T\rightarrow\infty
$\textit{\ and }$\Delta\rightarrow0$\textit{.}

\noindent\textit{(i) Under }$H_{0}:$%
\[
S_{T,\Delta}\overset{d}{\rightarrow}N\left(  0,\omega_{0}\right)  ,
\]
\textit{with }$\omega_{0}=15E\left(  V_{k\Delta}^{3}\right)  +4\left(
\mathrm{E}\left(  V_{k,\Delta}\right)  \right)  ^{3}-12E\left(  V_{k,\Delta
}\right)  E\left(  V_{k,\Delta}^{2}\right)  $\textit{ and }$S_{T,\Delta}%
$\textit{ defined as in Eq. (\ref{St}).}

\noindent\textit{(ii) Under }$H_{A}^{(1)},$\textit{ there exists an
}$\varepsilon>0,$\textit{ such that:}%
\[
\lim_{T\rightarrow\infty,\Delta\rightarrow0}\Pr\left(  \frac{\Delta}{\sqrt{T}%
}\left\vert S_{T,\Delta}\right\vert >\varepsilon\right)  =1.
\]
\textit{(iii) Under }$H_{A}^{(2)},$\textit{ there exists an }$\varepsilon
>0,$\textit{ such that: \ \ }%
\[
\lim_{T\rightarrow\infty,\Delta\rightarrow0}\Pr\left(  \Delta\left\vert
S_{T,\Delta}\right\vert >\varepsilon\right)  =1.
\]


It follows immediately that $S_{T,\Delta}$ converges to a normal random
variable under the null hypothesis, diverges at rate $\frac{\sqrt{T}}{\Delta}$
under the alternative of asymmetric jumps, and diverges at the slower rate of
$\frac{1}{\Delta}$ under the alternative of symmetric jumps. As shown in the
Appendix, as $\Delta\rightarrow0$ and $T\rightarrow\infty,$ we have that
$\widehat{\lambda}_{T,\Delta}\overset{p}{\rightarrow}\lambda_{\infty
}\mathrm{E}\left(  \left(  Z_{t}-\mathrm{E}\left(  Z_{t}\right)  \right)
^{3}\right)  .$ Now, if $\mathrm{E}\left(  \left(  Z_{t}-\mathrm{E}\left(
Z_{t}\right)  \right)  ^{3}\right)  \neq0,$ then under $H_{A}^{(1)},$ the test
has power against $\frac{\sqrt{T}}{\Delta}-$alternatives. Nevertheless, under
$H_{0}$ the statistic has Pitman drift \textquotedblleft
only\textquotedblright\ against $\sqrt{T}-$alternatives. This is because the
limiting distribution of the statistic under $H_{A}^{(1)}$ differs from that
under $H_{0}$ both in terms of the location and scale. In fact, under
$H_{A}^{(1)}$ the statistic has a mean (Pitman drift) of order $\sqrt{T},$ and
has a standard deviation of order $\Delta^{-1},$ while under $H_{0}$ the
limiting distribution has mean zero and finite variance. On the other hand, if
$\mathrm{E}\left(  \left(  Z_{t}-\mathrm{E}\left(  Z_{t}\right)  \right)
^{3}\right)  =0,$ then $\lambda_{\infty}$ is not identified, and so under
$H_{A}^{(2)}$ the Pitman drift is zero. Indeed, $\widehat{\lambda}_{T,\Delta
}\overset{p}{\rightarrow}0$ regardless of whether $\lambda_{\infty}=0$ or
$\lambda_{\infty}>0.$ Although it is not possible to distinguish between
$H_{0}$ and $H_{A}^{(2)}$ based on the different locations of the limiting
distribution (Pitman drift), it is possible to distinguish between $H_{0}$ and
$H_{A}^{(2)}$ based on the different scales of the limiting distribution of
$\frac{T^{1/2}}{\Delta}\widehat{\lambda}_{T,\Delta}.$ This is because the
order of magnitude of the variance of $\frac{T^{1/2}}{\Delta}\widehat{\lambda
}_{T,\Delta}$ is larger when $\lambda_{\infty}>0$ and \textrm{E}$\left(
\left(  Z_{t}-\mathrm{E}\left(  Z_{t}\right)  \right)  ^{3}\right)  =0$ than
when $\lambda_{\infty}=0.$ Broadly speaking, under $H_{0},$ $S_{T,\Delta
}\overset{d}{\rightarrow}N\left(  0,\omega_{0}\right)  ,$ while under
$H_{A}^{(2)},$ $\Delta S_{T,\Delta}\overset{d}{\rightarrow}N\left(
0,\omega_{1}\right)  ,$ with $\omega_{1}\neq\omega_{0}$. This is what allows
one to distinguish between $H_{0}$ and $H_{A}^{(2)}.$

If the moments of $V_{t}$ were known$,$ then an estimator of the variance
which is consistent for the true variance under the null and bounded in
probability under the alternative could be constructed; and consequently one
could carry out inference on a simple $t-$statistic. However, spot
volatilities are not generally observed, and hence the moments are not
generally known. Heuristically, one may think of using $\widehat{\sigma
}_{\lambda,T,\Delta}^{2}=\frac{1}{T\Delta^{2}}\sum_{k=1}^{n}\left(
Y_{k\Delta}-\overline{Y}_{\Delta}\right)  ^{6}$ as an estimator of
\textrm{var}$\left(  \frac{\sqrt{T}}{\Delta}\widehat{\lambda}_{T,\Delta
}\right)  .$ However, while $\widehat{\sigma}_{\lambda,T,\Delta}$ is
consistent for the \textquotedblleft true\textquotedblright standard deviation
under the null, it is order $O_{p}\left(  \Delta^{-1}\right)  $ under either
of the alternatives. As a consequence, under $H_{A}^{(2)},$ $t_{\lambda
,T,\Delta}=\frac{\frac{\sqrt{T}}{\Delta}\widehat{\lambda}_{T,\Delta}%
}{\widehat{\sigma}_{\lambda,T,\Delta}}$ would remain bounded in probability.
Hence, $t_{\lambda,T,\Delta}$ does not have power against the alternative of
jumps characterized by a symmetric distribution. Needless to say, if one rules
out the possibility of symmetric jumps, then one could simply compare
$t_{\lambda,T,\Delta}$ with standard normal critical values. However, in order
to allow for the possibility of symmetric jumps, the statistic should not be
rescaled, and hence inference should be based on the use of the bootstrap.

Finally, not that $\frac{1}{T}\sum_{k=2}^{n}\left(  Y_{k\Delta}-Y_{(k-1)\Delta
}\right)  ^{4}\overset{p}{\rightarrow}\lambda_{\infty}\mathrm{E}\left(
\left(  Z_{t}-\mathrm{E}\left(  Z_{t}\right)  \right)  ^{4}\right)  ,$ and
hence a statistic based on the sample fourth moment has a well defined Pitman
drift against $\sqrt{T}-$alternatives, regardless whether the jump size
density is symmetric or not. We did not attempt to construct a statistic based
on the sample fourth moment as, under the null hypothesis it could not have a
limiting Gaussian distribution, because of the boundary issue and the
impossibility of having a quadratic approximation around $\lambda_{\infty}=0.$

\subsection{\noindent Bootstrap Critical Values}

\noindent Given that the variance is of a different order of magnitude under
the null and under each alternative, the \textquotedblleft
standard\textquotedblright\ nonparametric bootstrap is not asymptotically
valid. This issue arises because the variance of the bootstrap statistic
mimics the sample variance. This implies that the bootstrap statistic is of
order $\Delta^{-1}$ under the alternative. This is not be a problem under
$H_{A}^{(1)},$ since the statistic is of order $\sqrt{T}\Delta^{-1},$ but is a
problem under $H_{A}^{(2)}$, since the actual and bootstrap statistics would
be of the same order. To ensure power against $H_{A}^{(2)},$ it suffices to
ensure that the bootstrap statistic is of a smaller order than the actual
statistic. This can be accomplished by resampling observations over a rougher
grid, $\widetilde{\Delta}$, using the same time span, $T.$

Set the new discrete interval to be $\widetilde{\Delta},$ such that
$\Delta/\widetilde{\Delta}\rightarrow0,$ and resample, with replacement,

\noindent$\left(  Y_{k\widetilde{\Delta}}^{\ast}-Y_{(k-1)\widetilde{\Delta}%
}^{\ast},...,Y_{\widetilde{n}\widetilde{\Delta}}^{\ast}-Y_{(\widetilde{n}%
-1)\widetilde{\Delta}}^{\ast}\right)  $ from $\left(  Y_{k\widetilde{\Delta}%
}-Y_{(k-1)\widetilde{\Delta}},...,Y_{\widetilde{n}\widetilde{\Delta}%
}-Y_{(\widetilde{n}-1)\widetilde{\Delta}}\right)  $, where $\widetilde{n}%
=\frac{T}{\widetilde{\Delta}}.$ Now, let:
\[
\widetilde{\lambda}_{T,\widetilde{\Delta}}=\frac{1}{T}\sum_{k=2}%
^{\widetilde{n}}\left(  Y_{k\widetilde{\Delta}}-Y_{(k-1)\widetilde{\Delta}%
}\right)  ^{3},
\]
and%
\[
\widetilde{\lambda}_{T,\widetilde{\Delta}}^{\ast}=\frac{1}{T}\sum
_{k=2}^{\widetilde{n}}\left(  Y_{k\widetilde{\Delta}}^{\ast}%
-Y_{(k-1)\widetilde{\Delta}}^{\ast}\right)  ^{3}.
\]
Further, define the bootstrap statistic:%
\[
S_{T,\widetilde{\Delta}}^{\ast}=\frac{\sqrt{T}}{\widetilde{\Delta}}\left(
\widetilde{\lambda}_{T,\widetilde{\Delta}}^{\ast}-\widetilde{\lambda
}_{T,\widetilde{\Delta}}\right)  .
\]
Finally, let $c_{\alpha,B,\Delta,\widetilde{\Delta}}^{\ast}$ and
$c_{(1-\alpha),B,\Delta,\widetilde{\Delta}}^{\ast}$ be the ($\alpha/2)^{th}$
and $(1-\alpha/2)^{th}$ critical values of the empirical distribution of
$S_{T,\widetilde{\Delta}}^{\ast},$ constructed using $B$ bootstrap
replications.\medskip

\noindent\textbf{Theorem 2: }\textit{Let Assumptions A(i) and A(iii)-(v) hold.
Also, assume that as }$n\rightarrow\infty,$\textit{ }$B\rightarrow\infty
,$\textit{ }$T\rightarrow\infty,$\textit{ }$\Delta\rightarrow0$\textit{,
}$\widetilde{\Delta}\rightarrow0$\textit{ and }$\Delta/\widetilde{\Delta
}\rightarrow0.$

\noindent\textit{(i) Under }$H_{0}:$%
\[
\lim_{T,B\rightarrow\infty,\Delta,\widetilde{\Delta}\rightarrow0}\Pr\left(
c_{\alpha/2,B,\Delta,\widetilde{\Delta}}^{\ast}\leq S_{T,\Delta}\leq
c_{(1-\alpha/2),B,\Delta,\widetilde{\Delta}}^{\ast}\right)  =1-\alpha.
\]
\textit{\noindent(ii) Under }$H_{A}^{(1)}\cup H_{A}^{(2)}:$%
\[
\lim_{T,B\rightarrow\infty,\Delta,\widetilde{\Delta}\rightarrow0}\Pr\left(
c_{\alpha/2,B,\Delta,\widetilde{\Delta}}^{\ast}\leq S_{T,\Delta}\leq
c_{(1-\alpha/2),B,\Delta,\widetilde{\Delta}}^{\ast}\right)  =0.
\]


It is immediate to see that rejecting the null whenever $\frac{\sqrt{T}%
}{\Delta}\widehat{\lambda}_{T,\Delta}<c_{\alpha/2,B,\Delta,\widetilde{\Delta}%
}^{\ast}$ or $\frac{\sqrt{T}}{\Delta}\widehat{\lambda}_{T,\Delta}%
>c_{(1-\alpha/2),B,\Delta,\widetilde{\Delta}}^{\ast}$ , and otherwise failing
to reject, delivers a test with asymptotic size equal to $\alpha$ and
asymptotic power equal to unity. Note that the bootstrap statistic is of
$P^{\ast}-$probability order $\frac{1}{\widetilde{\Delta}}$ under both
$H_{A}^{(1)}$ and $H_{A}^{(2)},$ while the actual statistic is of
$P-$probability order $\frac{\sqrt{T}}{\Delta}$ under $H_{A}^{(1)}$ and
$\frac{1}{\Delta}$ under $H_{A}^{(2)}.$ Hence, the condition that
$\frac{\Delta}{\widetilde{\Delta}}\rightarrow0$ ensures unit asymptotic power
under $H_{A}^{(2)}.$ As the suggested statistics are not robust to the
presence of microstructure noise, the optimal discrete interval, $\Delta,$ is
the highest frequency at which microstructure noise doesn't bind. Visual
inspection of the signature plots of Andersen, Bollerslev and Diebold (2000)
provides a useful tool for choice of interval. It should also be noted that
the statistic is constructed over an increasing time span; and hence it is not
straightforward to ascertain whether simple pre-averaging will make the
statistic robust to microstructure noise (as in the case of the realized
pre-average power variation discussed in Podolskji and Vetter (2009b)). Future
exploration of this issue is left to future research.

\subsection{Test of $\lambda_{\infty}=0$ (leverage effects)}

The statements in Theorems 1 and 2 require absence of leverage effects. In
particular, the results presented in these theorems rely on the fact that
under the null of no jumps, returns are symmetrically distributed. More
precisely, all results are derived under the assumption that \textrm{E}%
$\left(  \left(  Y_{k\Delta}-Y_{(k-1)\Delta}\right)  ^{3}\right)  =0,$
whenever there are no jumps. However, in the presence of leverage, if $V_{t}$
is generated as in $Eq.$ (\ref{vd}), \textrm{E}$\left(  \left(  \int%
_{(k-1)\Delta}^{k\Delta}V_{s}^{1/2}\mathrm{d}W_{1,s}\right)  ^{3}\right)
\neq0,$ and is instead of order $\Delta^{2}.$ For example, if $V_{t}$ is
generated by a square root process (i.e., $\mathrm{d}V_{t}=\kappa\left(
\theta-V_{t}\right)  \mathrm{d}t+\eta V_{t}^{1/2}\mathrm{d}W_{2,t}),$ then
\textrm{E}$\left(  \left(  Y_{k\Delta}-Y_{(k-1)\Delta}\right)  ^{3}\right)
=\lambda_{\infty}\mathrm{E}\left(  Z_{t}-\mathrm{E}\left(  Z_{t}\right)
\right)  ^{3}\Delta+\frac{\eta\theta\rho}{2\kappa}\Delta^{2}$ (see
Ait-Sahalia, Cacho-Diaz and Laeven (2013)). Although, the contribution to the
third moment of the asymmetric jump component is of a larger order than that
of the leverage component, inference based on the comparison of $S_{T,\Delta}$
with the bootstrap critical values $c_{\alpha,B,\Delta,\widetilde{\Delta}%
}^{\ast}$ and $c_{(1-\alpha),B,\Delta,\widetilde{\Delta}}^{\ast}$ will lead to
the rejection of the null of no jumps, even if the null is true. This is
established in the theorem below.

\noindent\textbf{Theorem 3: }\textit{Let Assumptions A(ii)-(v) hold. Also,
assume that as }$n\rightarrow\infty,$\textit{ }$T\rightarrow\infty,$\textit{
}$\Delta\rightarrow0$\textit{, }$\widetilde{\Delta}\rightarrow0$\textit{ and
}$\Delta/\widetilde{\Delta}\rightarrow0.$ \textit{Then}$,$\textit{ under both
}$H_{0}$\textit{ and }$H_{A}^{(1)}\cup H_{A}^{(2)}:$%

\[
\lim_{T,B\rightarrow\infty,\Delta,\widetilde{\Delta}\rightarrow0}\Pr\left(
c_{\alpha/2,B,\Delta,\widetilde{\Delta}}^{\ast}\leq S_{T,\Delta}\leq
c_{(1-\alpha/2),B,\Delta,\widetilde{\Delta}}^{\ast}\right)  =0.
\]
\medskip

It follows that, in the presence of leverage, we always reject the null of no
jumps, regardless as to whether it is true or false. To avoid spurious
rejection due to the presence of leverage, use the following modified
statistic:%
\begin{equation}
\widetilde{S}_{T,\Delta}=\frac{1}{T^{1/2+\varepsilon}}S_{T,\Delta},
\label{Stl}%
\end{equation}
with $\varepsilon>0,$ arbitrarily small.

\noindent\textbf{Theorem 4: }\textit{Let Assumption A(ii)-(v) hold. Also,
assume that as }$n\rightarrow\infty,$\textit{ }$T\rightarrow\infty,$\textit{
}$\Delta\rightarrow0$\textit{, }$\widetilde{\Delta}\rightarrow0,$\textit{ and
}$(T^{1/2+\varepsilon}\Delta)/\widetilde{\Delta}\rightarrow0.$

\noindent\textit{(i) Under }$H_{0}:$%

\[
\lim_{T,B\rightarrow\infty,\Delta,\widetilde{\Delta}\rightarrow0}\Pr\left(
c_{\alpha/2,B,\Delta,\widetilde{\Delta}}^{\ast}\leq\widetilde{S}_{T,\Delta
}\leq c_{(1-\alpha/2),B,\Delta,\widetilde{\Delta}}^{\ast}\right)  =1.
\]
\textit{\noindent(ii) Under }$H_{A}^{(1)}\cup H_{A}^{(2)}:$%
\[
\lim_{T,B\rightarrow\infty,\Delta,\widetilde{\Delta}\rightarrow0}\Pr\left(
c_{\alpha/2,B,\Delta,\widetilde{\Delta}}^{\ast}\leq\widetilde{S}_{T,\Delta
}\leq c_{(1-\alpha/2),B,\Delta,\widetilde{\Delta}}^{\ast}\right)  =0.
\]
\medskip

It follows that inference based on the comparison of $\widetilde{S}_{T,\Delta
}$ with the bootstrap critical values $c_{\alpha,B,\Delta,\widetilde{\Delta}%
}^{\ast}$ and $c_{(1-\alpha),B,\Delta,\widetilde{\Delta}}^{\ast}$ delivers a
test with zero asymptotic size and unit asymptotic power. Needless to say, the
statements in Theorem 4 are also valid when there is no leverage. However, in
the this case tests should be based on $S_{T,\Delta}$, in order to maximize power.

\subsection{Test of $\beta=0$}

If the null hypothesis of zero intensity is rejected, one can proceed to test
the null of no self-excitation or path dependence. The null in this case is
$\beta=0,$ and the alternative is $\beta>0,$ with $\beta$ defined as in $Eq.$
(\ref{lambda}). As shown by Ait-Sahalia, Cacho-Diaz and Laeven (2013), $\beta$
can be identified from the autocorrelation function. In particular, they show
that given $Eq.$ (\ref{lambda}),%
\begin{align}
&  \mathrm{E}\left(  \left(  Y_{k\Delta}-Y_{(k-1)\Delta}\right)  \left(
Y_{(k+\tau)\Delta}-Y_{(k+\tau-1)\Delta}\right)  \right) \nonumber\\
&  =\frac{\beta\lambda_{\infty}\left(  2a-\beta\right)  }{2\left(
a-\beta\right)  }\exp\left(  -\left(  a-\beta\right)  \tau\right)  \left(
\mathrm{E}\left(  Z\right)  \right)  ^{2}\Delta^{2}+o\left(  \Delta
^{2}\right)  . \label{cross}%
\end{align}
Given that $\lambda_{\infty}>0,$ it follows that $\mathrm{E}\left(  \left(
Y_{k\Delta}-Y_{(k-1)\Delta}\right)  \left(  Y_{(k+\tau)\Delta}-Y_{(k+\tau
-1)\Delta}\right)  \right)  =0$ if and only if $\beta=0.$ Our objective is to
test the following hypotheses:%
\begin{align*}
H_{0}  &  :\beta=0\text{ }\\
H_{A}  &  :\beta>0.
\end{align*}
Define the statistic:%
\[
Z_{T,\Delta}=\max\left\{  0,t_{\beta,T,\Delta}\right\}  ,
\]
where%
\begin{equation}
t_{\beta,T,\Delta}=\frac{\sqrt{\frac{T}{\Delta}}\widehat{\tau}_{T,\Delta}%
}{\widehat{\sigma}_{\beta,T,\Delta}}, \label{t-beta}%
\end{equation}
with%
\begin{equation}
\widehat{\tau}_{T,\Delta}=\frac{1}{T}\sum_{k=2}^{n-1}\left(  Y_{k\Delta
}-Y_{(k-1)\Delta}\right)  \left(  Y_{(k+1)\Delta}-Y_{k\Delta}\right)
\label{beta}%
\end{equation}
and%
\[
\widehat{\sigma}_{\beta,T,\Delta}^{2}=\frac{1}{T\Delta}\sum_{k=2}^{n-1}\left(
Y_{k\Delta}-Y_{(k-1)\Delta}\right)  ^{2}\left(  Y_{(k+1)\Delta}-Y_{k\Delta
}\right)  ^{2}.
\]
From $Eq.$ (\ref{cross}), and recalling that $a>0,$ $\beta\geq0,$ and
$a>\beta,$ it follows immediately that the autocorrelation can never be
negative. This is why the test is one-sided.\medskip

\noindent\textbf{Theorem 5: }\textit{Let Assumption A(i) or A(ii) and
A(iii)-(v) hold, and let }$\lambda_{t}$\textit{ be the solution to Eq.
(\ref{lambda}). Also, assume that }$E\left(  Z\right)  \neq0,$\textit{ and as
}$n\rightarrow\infty,$\textit{ }$T\rightarrow\infty$\textit{ and }%
$\Delta\rightarrow0$\textit{.}

\noindent\textit{(i) Under }$H_{0}:$%
\[
Z_{T,\Delta}\overset{d}{\rightarrow}\max\left\{  0,Z\right\}  ,
\]
\textit{where }$Z$\textit{ is a standard normal random variable.}

\noindent\textit{(ii) Under }$H_{A},$\textit{ there exists an }$\varepsilon
>0$\textit{ such that: }%
\[
\lim_{T\rightarrow\infty,\Delta\rightarrow0}\Pr\left(  \sqrt{\frac{\Delta}{T}%
}Z_{T,\Delta}>\varepsilon\right)  =1.
\]


It follows that $Z_{T,\Delta}$ converges to an half-normal random variable
under the null, and diverges at rate $\sqrt{\frac{\Delta}{T}}$ under the alternative.

\noindent\textbf{Remark 1: }The test statistic is only a function of the first
autocovariance term. It follows immediately that one can construct a test
based on an increasing number of autocovariance terms, with the number of
terms chosen adaptively (see, e.g. Escanciano and Lobato (2009)).

\noindent\textbf{Remark 2: }If the nulls of zero intensity and no
self-excitation are both rejected, then one can proceed to estimate the full
Hawkes diffusion using GMM, as in Ait-Sahalia, Cacho-Diaz and Laeven (2013).
Of course, there is still a positive probability that the nulls have been
falsely rejected. Because of this, one should consider carrying out somewhat
conservative inference, using a \textquotedblleft small\textquotedblright%
\ significance level.

\noindent\textbf{Remark 3: }In this paper, we only derive model free tests for
the null of zero jump intensity in asset returns. However, the same approach
can be used for testing equivalent hypotheses for volatility. Such tests would
require estimators of the spot volatility, say $V_{k\Delta}^{2},$ which can be
constructed using a finer grid of observations than used in the above tests,
such as if there are $M$ observations over each interval of order $\Delta.$
The order of magnitude of the error due to the estimation of the spot
volatility is derived in Bandi and Reno (2012), under various settings.

\noindent\textbf{Remark 4: }In this section, we consider the case of
self-exciting intensity. However, from an empirical point of view, an
interesting case is that of financial contagion, where the contagion is due to
\textquotedblleft common\textquotedblright\ jumps. In this case, the jump
intensity is an increasing function not only of its own past jumps but also of
past jumps in other assets. In order to test for (no) cross-excitation, it
suffices to construct a statistic based on cross correlations instead of
autocorrelations (see Theorem 4 in Ait-Sahalia, Cacho-Diaz and Laeven (2013)).
For example, let:
\[
\widehat{\tau}_{T,\Delta}^{(I,II)}=\frac{1}{T}\sum_{k=2}^{n-1}\left(
Y_{k\Delta}^{(I)}-Y_{(k-1)\Delta}^{(I)}\right)  \left(  Y_{(k+1)\Delta}%
^{(II)}-Y_{k\Delta}^{(II)}\right)  ,
\]
and note that if the jump intensity in asset $II$ does not depend on past
jumps in asset $I$, then $\widehat{\tau}_{T,\Delta}^{(I,II)}\rightarrow0.$ On
the other hand, if the intensity in asset $II$ increases when there is a jump
in asset $I$, then $\widehat{\tau}_{T,\Delta}^{(I,II)}$ has a strictly
positive probability limit.

\section{Monte Carlo Results}

In this section we carry out a set of experiments designed to evaluate the
finite sample properties of (i) the test for the null of zero intensity, based
on $S_{T,\Delta},$ as defined in $Eq.$ (\ref{St}) and, for the case of
leverage, based on $\widetilde{S}_{T,\Delta},$ as defined in $Eq.$
(\ref{Stl}); (ii) the test for the null of no jump path dependence based on
$Z_{T,\Delta},$ as discussed in the previous section; and (iii) the overall
procedure according to which, if we reject the null of no jumps, according to
either $S_{T,\Delta}$ or $\widetilde{S}_{T,\Delta},$ we then proceed to test
the null of no path dependence, using $Z_{T,\Delta}.$ We now outline the data
generating processes (DGPs) used in the simulation experiments, namely:%
\[
\mathrm{d\ln}X_{t}=\mu\mathrm{d}t+\sqrt{V_{t}}\mathrm{d}W_{1,t}+Z_{t}%
\mathrm{d}N_{t},
\]
where volatility is modeled as a square-root process:%
\[
\mathrm{d}V_{t}=\kappa_{v}(\theta_{v}-V_{t})\mathrm{d}t+\zeta\sqrt{V_{t}%
}\mathrm{d}W_{2,t},
\]
with \textrm{E}$\left(  W_{1,t}W_{2,t}\right)  =\rho.$ We have set $\mu=0.5$,
$\rho=\{0,-0.5\}$, $\kappa_{v}=5,$ $\theta_{v}=0.04,$ and $\zeta=0.5.$
Additionally, $N_{t}$ satisfies the conditions in $Eqs.$ (\ref{J1})-(\ref{J3})
and for the jump size density we consider two cases, (a) $Z_{t}$ an $iid$
$N(0.5,0.01)$ random variable, and (b) $Z_{t}$ an exponential random variable
with parameter equal to $5.$ The jump intensity evolves according to:%
\begin{equation}
\lambda_{t}=\lambda_{\infty}+\beta\int_{0}^{t}\exp\left(  -a\left(
t-s\right)  \right)  \mathrm{d}N_{s}, \label{lambdaB}%
\end{equation}
where $\lambda_{\infty}=\left\{  1/20,1/10,1/5,3/10,2/5\right\}  ,$ and
$(a,\beta)=\{(0,0),(0.2,0.1),$ $(2,1),$ and $(5,4)\}.$ Note that the case
where $(a,\beta)=\{(0,0)\}$ is consistent with both the case of no jumps
(i.e., $\lambda_{\infty}=0)$ and constant jump intensity (i.e., $\lambda
_{t}=\lambda_{\infty}).$

We simulate observations using a Milstein discretization scheme, with discrete
interval $h=1/100.$ For DGPs with $\rho=0,$ we sample the simulated
observations using $\Delta=1/60$ when constructing the test statistics and
$\widetilde{\Delta}=1/20$ when constructing bootstrap statistics. For DGPs
with $\rho=-0.5,$ we set $\Delta=1/100$ and $\widetilde{\Delta}=1/10$ for test
statistics and bootstrap statistics, respectively. In all experiments, we
perform $1000$ Monte Carlo replications. Finally, recall that only the jump
intensity test uses bootstrap critical values (see the statements in Theorems
2 and 4).

Conducting Monte Carlo experiments involving bootstrap estimators are always
quite computationally demanding. In our experiments the computational burden
is potentially even higher than usual,\ since we rely on a joint in-fill and
long-span asymptotics, and since we need to control for the discretization
error when simulating according to diffusion processes. To cope with this
computational cost, we construct bootstrap critical values using the
Warp-Speed approach of Giacomini, Politis and White (2013). This approach
involves carrying out only one bootstrap replication for each simulated
sample, and then averaging the bootstrap statistics over all Monte Carlo
replications. Hence, the overall number of bootstrap replications is equal to
the number of Monte Carlo replications, which is $1000$ in our case. Under
mild regularity conditions, the accuracy of the Warp-Speed bootstrap
approaches is the same as that of the usual bootstrap, as both the sample size
and the number of Monte Carlo replications go to infinity (see Corollary 5 in
Giacomini, Politis and White (2013)).

The findings from our simulation studies are reported in Tables 1-4. Table 1
reports the empirical size, with the nominal size set at $10\%$. In Table 1
the first three columns consider DGPs with no leverage, while the last three
columns consider DGPs with leverage. The first row displays the rejection
frequencies for the test of $H_{0}:\lambda_{\infty}=0$ vs $H_{A}%
:\lambda_{\infty}>0,$ with the first three entries reporting results for the
$S_{T,\Delta}$ and the last three corresponding to $\widetilde{S}_{T,\Delta}.$
For $S_{T,\Delta}$ the empirical size is very accurate, and is effectively
equal to the nominal size, for $T\geq500,$ while, as stated in Theorem 4, the
rejection frequencies under the null for $\widetilde{S}_{T,\Delta}$ are
identically equal to zero. The first and second quadrants in Table 1 contain
rejection frequencies for $H_{0}:\beta=0$ vs $H_{A}:\beta>0,$ when data are
generated with jumps characterized by constant intensity and jump sizes
normally or exponentially distributed. Here, each row denotes a different
intensity, from the lowest ($1/20)$ in the top row to the highest ($2/5)$ in
the bottom row. The third and fourth quadrants report rejection frequencies
for the full sequential procedure. In particular, they report how many times
$H_{0}:\lambda_{\infty}=0$ vs $H_{A}:\lambda_{\infty}>0$ has been rejected and
how many times $H_{0}:\beta=0$ vs $H_{A}:\beta>0$ has not been rejected.
Overall, the empirical size is quite close to the nominal, though slightly
smaller for the rows characterized by a large intensity parameter.

Table 2 contains the results of power experiments using $S_{T,\Delta}$
(non-leverage case), and using $\widetilde{S}_{T,\Delta}$ (leverage case).
Panels A and B report rejection frequencies for DGPs generated with Poisson
jumps characterized by constant intensity ranging from $1/20$ (top row in each
quadrant) to $2/5$ (bottom row in each quadrant). The empirical power is
essentially unity$,$ even for the lowest intensity, and hence the test is
powerful even in the presence of relatively few jumps. This is true also for
the case of normal (symmetric) jumps, despite of the fact that the Pitman
drift is zero, so that rejections are due only to the different order of
magnitude of the variance. Interestingly, when $\Delta/\widetilde{\Delta}$ is
small enough, the power in the leverage cases is as high as in the
non-leverage case. Panels C and D report rejection frequencies in the case of
no-leverage and self-exciting jumps (i.e., when $\lambda_{t}$ is generated as
in $Eq.$ (\ref{lambdaB})), with exponential and normal jump densities,
respectively. Panels E and F report analogous results for the leverage case.
For cases where the mean intensity is low, rejection frequencies in the
leverage case (i.e., tests based on $\widetilde{S}_{T,\Delta})$ are now
slightly lower than corresponding rejection frequencies in the non-leverage
case (i.e., tests based on $\widetilde{S}_{T,\Delta}).$ However, for
$\lambda_{\infty}\geq1/5,$ rejection frequencies are rather close to unity.

Table 3 contains results from power experiments (i.e., $H_{0}:\beta=0$ vs
$H_{A}:\beta>0),$ under the maintained assumption that $\lambda_{\infty}>0.$
Note that for the test for no-path dependent intensity, we use the same
statistic $\max\left\{  0,t_{\beta,T,\Delta}\right\}  ,$ as defined in
$Eq.$\ (\ref{t-beta}), regardless of the presence of leverage or not. However,
for the sake of completeness, we still report result for the no-leverage case
in Panels A and B, and for the leverage case in Panels C and D. From $Eqs.$
(\ref{cross}) and (\ref{lambdaB}) is immediate to see that the smaller is $a$
and the smaller is $(a-\beta),$ the higher is the level of self-excitation.
For example, when $a=0.1$ and $\beta=0.2,$ rejection frequencies are above
$0.9,$ regardless of mean intensity. However, the case where $a=0.1$ and
$\beta=0.2$ imply a somewhat implausibly high level of path dependence. For an
intermediate degree of self-excitation, we consider $a=2$ and $\beta=1.$ In
this case, the rejection frequencies are reasonably high, from $0.65$ to
$0.80,$ when there are \textquotedblleft enough\textquotedblright\ jumps
(i.e., when $\lambda_{\infty}\geq3/10).$ Finally, in the case of low
self-excitation (i.e., $a=5,\beta=4)$ the power is slightly below $0.50,$ even
for the highest mean intensity.

Table 4 summarizes experimental findings based on implementation of the full
(sequential) procedure. Namely, whenever we reject $H_{0}:\lambda_{\infty}=0$
vs $\lambda_{\infty}>0$, we proceed to test $H_{0}:\beta=0$ vs $H_{A}%
:\beta>0.$ Entries in the table denote that rejection frequencies indicating
that $both$ null hypotheses are rejected (sequentially). As the power of the
jump intensity test against path-dependent jumps is very close to 1 (see
Panels C-F in Table 2), it is not surprising to note that the entries in Table
4 are very close to those in Table 3.

In summary, the test for zero jump intensity has excellent empirical size and
power across all cases. On the other hand, the test for no path dependence and
the sequential procedure have the very good empirical size but good power only
when jumps are frequent enough and the degree of self-excitation is not too low.

\section{Concluding Remarks}

If the intensity parameter in a jump diffusion model is identically zero, then
parameters characterizing the jump size density cannot be identified. In
general, this lack of identification precludes consistent estimation of
identified parameters. Hence, consistent estimation of jump diffusions
requires consistent pretesting for the null of zero jump intensity. Currently
available tests, which are based on high frequency observations over a finite
time-span, are model free but are not consistent. On the other hand, tests
based on discrete observations over a long time-span are consistent, but
require full specification of the model as well as knowledge of a closed form
expression for the transition density. This paper introduces novel (almost)
model free tests which are consistent against the alternative of positive
intensity. They are based on sample third moments, and make use of high
frequency observations over a long time-span. Inference is based on $m$ out
$n$ type bootstrap critical values, whose first order validity is established.
A "self-excitement" test is also introduced, which is designed to have power
against path dependent intensity, thus providing a direct test for the Hawkes
diffusion model of Ait-Sahalia, Cacho-Diaz and Laeven (2013). The finite
sample behavior of the suggested statistics is studied via Monte Carlo
experimentation. The jump tests exhibit empirical size very close to nominal
and empirical power close to unity. The self excitement test likewise has very
good size and good power properties, whenever there are \textquotedblleft
enough\textquotedblright\ jumps and the degree of self excitation is not
\textquotedblleft too weak\textquotedblright.\pagebreak

\section{Appendix}

\noindent\textbf{Proof of Theorem 1:}

\noindent(i) Under $H_{0}$ $N_{t}=0,$ since the drift term is constant,%
\begin{align}
&  S_{T,\Delta}\nonumber\\
&  =\frac{1}{\sqrt{T}\Delta}\sum_{k=2}^{n}\left(  \left(  \ln X_{k\Delta
}-\frac{\Delta}{T}\sum_{k=1}^{n}\ln X_{k\Delta}\right)  -\left(  \ln
X_{(k-1)\Delta}-\frac{\Delta}{T}\sum_{k=2}^{n}\ln X_{(k-1)\Delta}\right)
\right)  ^{3}\nonumber\\
&  =\frac{1}{\sqrt{T}\Delta}\sum_{k=2}^{n}\left(  \int_{(k-1)\Delta}^{k\Delta
}\sqrt{V_{s}}\mathrm{d}W_{1,s}-\frac{\Delta}{T}\sum_{k=1}^{n}\int%
_{(k-1)\Delta}^{k\Delta}\sqrt{V_{s}}\mathrm{d}W_{1,s}\right)  ^{3}\left(
1+o_{p}(1)\right) \nonumber\\
&  =\frac{1}{\sqrt{T}\Delta}\sum_{k=2}^{n}\left(  \Delta^{1/2}\sqrt
{V_{(k-1)\Delta}}\epsilon_{k\Delta}\right)  ^{3}\left(  1+o_{p}(1)\right)
\nonumber\\
&  -\frac{\sqrt{T}}{\Delta^{2}}\left(  \frac{\Delta}{T}\sum_{k=2}^{n}%
\Delta^{1/2}\sqrt{V_{(k-1)\Delta}}\epsilon_{k\Delta}\right)  ^{3}\left(
1+o_{p}(1)\right) \nonumber\\
&  -\frac{2}{\sqrt{T}\Delta}\sum_{k=2}^{n}\Delta V_{(k-1)\Delta}%
\epsilon_{k\Delta}^{2}\left(  \frac{\Delta}{T}\sum_{k=1}^{n}\Delta^{1/2}%
\sqrt{V_{(k-1)\Delta}}\epsilon_{k\Delta}\right)  \left(  1+o_{p}(1)\right)
\nonumber\\
&  +\frac{2}{\sqrt{T}\Delta}\sum_{k=2}^{n}\Delta^{1/2}\sqrt{V_{(k-1)\Delta}%
}\epsilon_{k\Delta}\left(  \frac{\Delta}{T}\sum_{k=2}^{n}\Delta^{1/2}%
\sqrt{V_{(k-1)\Delta}}\epsilon_{k\Delta}\right)  ^{2}\left(  1+o_{p}(1)\right)
\nonumber\\
&  =I_{T,\Delta}+II_{T,\Delta}+III_{T,\Delta}+IV_{T,\Delta}, \label{S_T}%
\end{align}
where $\epsilon_{k\Delta}$ is an \textit{iid}\textrm{ N}$\left(  0,1\right)  $
random variable$,$ and where the $o_{p}(1)$ terms denote terms approaching
zero as $\Delta\rightarrow0,$ uniformly in $T.$

\noindent Let $\mathcal{F}_{(k-1)\Delta}=\sigma\left(  V_{\Delta
},...,V_{(k-1)\Delta}\right)  $ and note that under A(i),

\noindent\textrm{E}$\left(  V_{(k-1)\Delta}^{3/2}\epsilon_{k\Delta}%
^{3}\left\vert \mathcal{F}_{(k-1)\Delta}\right.  \right)  =V_{(k-1)\Delta
}^{3/2}$\textrm{E}$\left(  \epsilon_{k\Delta}^{3}\left\vert \mathcal{F}%
_{(k-1)\Delta}\right.  \right)  =0,$ and hence $V_{(k-1)\Delta}^{3/2}%
\epsilon_{k\Delta}^{3}$ is a martingale difference sequence. It follows that,%
\begin{align}
&  \mathrm{var}\left(  \frac{1}{\sqrt{T}\Delta}\sum_{k=1}^{n}\left(
\Delta^{1/2}\sqrt{V_{(k-1)\Delta}}\epsilon_{k\Delta}\right)  ^{3}\right)
\nonumber\\
&  =\frac{1}{T\Delta^{2}}\frac{T}{\Delta}\Delta^{3}\mathrm{E}\left(
V_{(k-1)\Delta}^{3}\epsilon_{k\Delta}^{6}\right)  =15\mathrm{E}\left(
V_{(k-1)\Delta}^{3}\right)  . \label{var}%
\end{align}
\noindent We first show that $II_{T,\Delta}$ and $IV_{T,\Delta}$ are
$o_{p}(1).$%
\begin{align*}
II_{T,\Delta}  &  =\frac{2\sqrt{T}}{\Delta^{2}}\frac{\Delta^{3}}{T^{3/2}%
}\left(  \sqrt{\frac{\Delta}{T}}\sum_{k=1}^{n}\left(  \sqrt{V_{(k-1)\Delta}%
}\epsilon_{k\Delta}\right)  ^{3}\right)  \left(  1+o_{p}(1)\right) \\
&  =\frac{\Delta}{T}O_{p}(1)=o_{p}(1),
\end{align*}
since, recalling $Eq.$ (\ref{var}), $\sqrt{\frac{\Delta}{T}}\sum_{k=1}%
^{n}\left(  \sqrt{V_{(k-1)\Delta}}\epsilon_{k\Delta}\right)  ^{3}$ satisfies a
CLT for martingale difference sequences. Additionally,%
\begin{align*}
IV_{T,\Delta}  &  =2\frac{\Delta}{T}\sqrt{\frac{\Delta}{T}}\sum_{k=1}%
^{n}\left(  \sqrt{V_{(k-1)\Delta}}\epsilon_{k\Delta}\right)  \left(
\sqrt{\frac{\Delta}{T}}\sum_{k=1}^{n}\left(  \sqrt{V_{(k-1)\Delta}}%
\epsilon_{k\Delta}\right)  \right)  ^{2}\left(  1+o_{p}(1)\right) \\
&  =\frac{\Delta}{T}O_{p}\left(  1\right)  =o_{p}(1).
\end{align*}
Now, $\mathrm{var}\left(  I_{T,\Delta}\right)  =15\mathrm{E}\left(
V_{(k-1)\Delta}^{3}\right)  ,$ given $Eq.$ (\ref{var}), and%
\begin{align*}
\mathrm{var}\left(  III_{T,\Delta}\right)   &  =4\left(  \mathrm{E}\left(
V_{k,\Delta}\right)  \right)  ^{2}\mathrm{var}\left(  \sqrt{\frac{\Delta}{T}%
}\sum_{k=1}^{n}\sqrt{V_{(k-1)\Delta}}\epsilon_{k\Delta}\right) \\
&  =4\left(  \mathrm{E}\left(  V_{k,\Delta}\right)  \right)  ^{2}%
\mathrm{E}\left(  V_{k,\Delta}\right)  =4\left(  \mathrm{E}\left(
V_{k,\Delta}\right)  \right)  ^{3},
\end{align*}
with%
\[
\mathrm{cov}\left(  I_{T,\Delta},III_{T,\Delta}\right)  =-4\mathrm{E}\left(
V_{k,\Delta}\right)  \mathrm{E}\left(  V_{k,\Delta}^{2}\right)  \mathrm{E}%
\left(  \epsilon_{k\Delta}^{4}\right)  .
\]
Thus,%
\[
\left(  I_{T,\Delta}+III_{T,\Delta}\right)  \overset{d}{\rightarrow}N\left(
0,\omega_{0}\right)  ,
\]
with $\omega_{0}=15\mathrm{E}\left(  V_{(k-1)\Delta}^{3}\right)  +4\left(
\mathrm{E}\left(  V_{k,\Delta}\right)  \right)  ^{3}-12\mathrm{E}\left(
V_{k,\Delta}\right)  \mathrm{E}\left(  V_{k,\Delta}^{2}\right)  .$ The
statement in (i) then follows.\medskip

\noindent(ii) Under $H_{A}^{(1)},$ there are additional jump components,
including:%
\[
\frac{1}{\sqrt{T}}\sum_{k=1}^{n}\frac{1}{\Delta}\left(  Z_{(k-1)\Delta}\left(
N_{k\Delta}-N_{(k-1)\Delta}\right)  -\frac{\Delta}{T}\sum_{k=1}^{n}%
Z_{(k-1)\Delta}\left(  N_{k\Delta}-N_{(k-1)\Delta}\right)  \right)  ^{3},
\]
plus related cross-terms. Now,%
\begin{align*}
&  \frac{\Delta}{T}\sum_{k=1}^{n}\frac{1}{\Delta}\left(  Z_{(k-1)\Delta
}\left(  N_{k\Delta}-N_{(k-1)\Delta}\right)  -\frac{\Delta}{T}\sum_{k=1}%
^{n}Z_{(k-1)\Delta}\left(  N_{k\Delta}-N_{(k-1)\Delta}\right)  \right)  ^{3}\\
&  \overset{pr}{\rightarrow}\lambda\mathrm{E}\left(  Z-\mathrm{E}\left(
Z\right)  \right)  ^{3}.
\end{align*}
Thus, whenever $\mathrm{E}\left(  Z-\mathrm{E}\left(  Z\right)  \right)
^{3}\neq0$, $S_{T,\Delta}$ is of probability order $\frac{\sqrt{T}}{\Delta}$, and

\noindent$\lim_{T\rightarrow\infty,\Delta\rightarrow0}\Pr\left(  \frac{\Delta
}{\sqrt{T}}\left\vert S_{T,\Delta}\right\vert >\varepsilon\right)  =1.$ The
statement in (ii) then follows.\medskip

\noindent(iii) Under $H_{A}^{(2)},$ by the law of large numbers,%
\begin{align*}
&  \frac{\Delta}{T}\sum_{k=1}^{n}\frac{1}{\Delta}\left(  Z_{(k-1)\Delta
}\left(  N_{k\Delta}-N_{(k-1)\Delta}\right)  -\frac{\Delta}{T}\sum_{k=1}%
^{n}Z_{(k-1)\Delta}\left(  N_{k\Delta}-N_{(k-1)\Delta}\right)  \right)  ^{3}\\
&  \overset{pr}{\rightarrow}0.
\end{align*}
Since $\mathrm{E}\left(  Z-\mathrm{E}\left(  Z\right)  \right)  ^{3}=0,$ and
given the central limit theorem,%
\begin{align*}
&  \sqrt{\frac{\Delta}{T}}\sum_{k=1}^{n}\frac{1}{\Delta}\left(  Z_{(k-1)\Delta
}\left(  N_{k\Delta}-N_{(k-1)\Delta}\right)  -\frac{\Delta}{T}\sum_{k=1}%
^{n}Z_{(k-1)\Delta}\left(  N_{k\Delta}-N_{(k-1)\Delta}\right)  \right)  ^{3}\\
&  =O_{p}(1).
\end{align*}
Moreover, if $\beta=0$ (no path dependent intensity), then:%
\begin{align*}
&  \mathrm{var}\left(  S_{T,\Delta}\right) \\
&  =\mathrm{var}\left(  \frac{1}{\sqrt{T}\Delta}\sum_{k=1}^{n}\left(
Z_{(k-1)\Delta}\left(  N_{k\Delta}-N_{(k-1)\Delta}\right)  \right)
^{3}\right)  \left(  1+o(1)\right) \\
&  =\frac{1}{T\Delta^{2}}\sum_{k=1}^{n}\mathrm{var}\left(  \left(
Z_{(k-1)\Delta}\left(  N_{k\Delta}-N_{(k-1)\Delta}\right)  \right)
^{3}\right)  \left(  1+o(1)\right) \\
&  =\frac{1}{\Delta^{3}}\mathrm{var}\left(  \left(  Z_{(k-1)\Delta}\left(
N_{k\Delta}-N_{(k-1)\Delta}\right)  \right)  ^{3}\right)  \left(
1+o(1)\right) \\
&  =O\left(  \frac{1}{\Delta^{2}}\right)  .
\end{align*}
Alternatively, if $\beta>0,$ one must take autocovariance terms into account
when carrying out similar calculations. However, given A(iv), the order of
magnitude of the variance is still $O\left(  \frac{1}{\Delta^{2}}\right)  $.
Hence, $S_{T,\Delta}$ is of probability order $\Delta^{-1}$ and the statement
in (iii) follows.

\bigskip

\noindent\textbf{Proof of Theorem 2: }Hereafter, let $\mathrm{E}^{\ast}$ and
$\mathrm{var}^{\ast}$ denote the mean and variance operators under the
bootstrap probability measure $P^{\ast},$ conditional on the sample, and let
$d^{\ast}$ denote convergence in distribution under $P^{\ast}.$

\noindent(i) Note that:%
\[
\mathrm{E}^{\ast}\left(  S_{T,\widetilde{\Delta}}^{\ast}\right)  =\frac
{1}{\sqrt{T}\widetilde{\Delta}}\sum_{k=1}^{\widetilde{n}}\mathrm{E}^{\ast
}\left(  \left(  Y_{k\widetilde{\Delta}}^{\ast}-Y_{(k-1)\widetilde{\Delta}%
}^{\ast}\right)  ^{3}-\left(  Y_{k\widetilde{\Delta}}%
-Y_{(k-1)\widetilde{\Delta}}\right)  ^{3}\right)  .
\]
Now, since:%
\[
\mathrm{E}^{\ast}\left(  Y_{k\widetilde{\Delta}}^{\ast}%
-Y_{(k-1)\widetilde{\Delta}}^{\ast}\right)  ^{3}=\frac{\widetilde{\Delta}}%
{T}\sum_{k=1}^{\widetilde{n}}\left(  Y_{k\widetilde{\Delta}}%
-Y_{(k-1)\widetilde{\Delta}}\right)  ^{3},
\]
it follows immediately that $\mathrm{E}^{\ast}\left(  S_{T,\widetilde{\Delta}%
}^{\ast}\right)  =0.$ Now, consider $\mathrm{var}^{\ast}\left(
S_{T,\widetilde{\Delta}}^{\ast}\right)  .$ First note that, by the same
argument as that used in the proof of Theorem 1, part (i),%
\begin{align*}
&  S_{T,\widetilde{\Delta}}^{\ast}\\
&  =\frac{1}{\sqrt{T}\widetilde{\Delta}}\sum_{k=1}^{\widetilde{n}}\left(
V_{(k-1)\widetilde{\Delta}}^{\ast3/2}\widetilde{\Delta}^{3/2}\epsilon
_{k\widetilde{\Delta}}^{\ast3}-2V_{(k-1)\widetilde{\Delta}}^{\ast
}\widetilde{\Delta}\epsilon_{k\widetilde{\Delta}}^{\ast2}\frac
{\widetilde{\Delta}}{T}\sum_{k=1}^{\widetilde{n}}V_{(k-1)\widetilde{\Delta}%
}^{\ast1/2}\widetilde{\Delta}^{1/2}\epsilon_{k\widetilde{\Delta}}^{\ast
}\right) \\
&  -\frac{1}{\sqrt{T}\widetilde{\Delta}}\sum_{k=1}^{\widetilde{n}}\left(
V_{(k-1)\widetilde{\Delta}}^{3/2}\widetilde{\Delta}^{3/2}\epsilon
_{k\widetilde{\Delta}}^{3}-2V_{(k-1)\widetilde{\Delta}}\widetilde{\Delta
}\epsilon_{k\widetilde{\Delta}}^{2}\frac{\widetilde{\Delta}}{T}\sum
_{k=1}^{\widetilde{n}}V_{(k-1)\widetilde{\Delta}}^{1/2}\widetilde{\Delta
}^{1/2}\epsilon_{k\widetilde{\Delta}}\right)  +o_{p}^{\ast}(1)+o_{p}(1)\\
&  =\sqrt{\frac{\widetilde{\Delta}}{T}}\sum_{k=1}^{\widetilde{n}}\left(
V_{(k-1)\widetilde{\Delta}}^{\ast3/2}\epsilon_{k\widetilde{\Delta}}^{\ast
3}-V_{(k-1)\widetilde{\Delta}}^{3/2}\epsilon_{k\widetilde{\Delta}}^{3}\right)
-2\left(  \frac{\widetilde{\Delta}}{T}\sum_{k=1}^{\widetilde{n}}%
V_{(k-1)\widetilde{\Delta}}^{\ast}\epsilon_{k\widetilde{\Delta}}^{\ast2}%
\sqrt{\frac{\widetilde{\Delta}}{T}}\sum_{k=1}^{\widetilde{n}}%
V_{(k-1)\widetilde{\Delta}}^{\ast1/2}\epsilon_{k\widetilde{\Delta}}^{\ast
}\right. \\
&  \left.  -\frac{\widetilde{\Delta}}{T}\sum_{k=1}^{\widetilde{n}%
}V_{(k-1)\widetilde{\Delta}}\epsilon_{k\widetilde{\Delta}}^{2}\sqrt
{\frac{\widetilde{\Delta}}{T}}\sum_{k=1}^{\widetilde{n}}%
V_{(k-1)\widetilde{\Delta}}^{1/2}\epsilon_{k\widetilde{\Delta}}\right)
+o_{p}^{\ast}(1)+o_{p}(1)\\
&  =\sqrt{\frac{\widetilde{\Delta}}{T}}\sum_{k=1}^{\widetilde{n}}\left(
V_{(k-1)\widetilde{\Delta}}^{\ast3/2}\epsilon_{k\widetilde{\Delta}}^{\ast
3}-V_{(k-1)\widetilde{\Delta}}^{3/2}\epsilon_{k\widetilde{\Delta}}^{3}\right)
-2\frac{\widetilde{\Delta}}{T}\sum_{k=1}^{\widetilde{n}}%
V_{(k-1)\widetilde{\Delta}}\epsilon_{k\widetilde{\Delta}}^{2}\left(
\sqrt{\frac{\widetilde{\Delta}}{T}}\sum_{k=1}^{\widetilde{n}}%
V_{(k-1)\widetilde{\Delta}}^{\ast1/2}\epsilon_{k\widetilde{\Delta}}^{\ast
}\right. \\
&  \left.  -\sqrt{\frac{\widetilde{\Delta}}{T}}\sum_{k=1}^{\widetilde{n}%
}V_{(k-1)\widetilde{\Delta}}^{1/2}\epsilon_{k\widetilde{\Delta}}\right)
+o_{p}^{\ast}(1)+o_{p}(1).
\end{align*}
Hence,%
\begin{align*}
&  \mathrm{var}^{\ast}\left(  S_{T,\widetilde{\Delta}}^{\ast}\right) \\
&  =\mathrm{var}^{\ast}\left(  \sqrt{\frac{\widetilde{\Delta}}{T}}\sum
_{k=1}^{\widetilde{n}}V_{(k-1)\widetilde{\Delta}}^{\ast3/2}\epsilon
_{k\widetilde{\Delta}}^{\ast3}\right) \\
&  +9\left(  \frac{\widetilde{\Delta}}{T}\sum_{k=1}^{\widetilde{n}%
}V_{(k-1)\widetilde{\Delta}}\epsilon_{k\widetilde{\Delta}}^{2}\right)
^{2}\mathrm{var}^{\ast}\left(  \sqrt{\frac{\widetilde{\Delta}}{T}}\sum
_{k=1}^{\widetilde{n}}V_{(k-1)\widetilde{\Delta}}^{\ast1/2}\epsilon
_{k\widetilde{\Delta}}^{\ast}\right) \\
&  -6\left(  \frac{\widetilde{\Delta}}{T}\sum_{k=1}^{\widetilde{n}%
}V_{(k-1)\widetilde{\Delta}}\epsilon_{k\widetilde{\Delta}}^{2}\right)
\mathrm{cov}^{\ast}\left(  \sqrt{\frac{\widetilde{\Delta}}{T}}\sum
_{k=1}^{\widetilde{n}}V_{(k-1)\widetilde{\Delta}}^{\ast3/2}\epsilon
_{k\widetilde{\Delta}}^{\ast3},\sqrt{\frac{\widetilde{\Delta}}{T}}\sum
_{k=1}^{\widetilde{n}}V_{(k-1)\widetilde{\Delta}}^{\ast1/2}\epsilon
_{k\widetilde{\Delta}}^{\ast}\right)
\end{align*}
and so,%
\begin{align*}
&  \mathrm{var}^{\ast}\left(  S_{T,\widetilde{\Delta}}^{\ast}\right) \\
&  =\frac{\widetilde{\Delta}}{T}\sum_{k=1}^{\widetilde{n}}%
V_{(k-1)\widetilde{\Delta}}^{3}\epsilon_{k\widetilde{\Delta}}^{6}+4\left(
\frac{\widetilde{\Delta}}{T}\sum_{k=1}^{\widetilde{n}}%
V_{(k-1)\widetilde{\Delta}}\epsilon_{k\widetilde{\Delta}}^{2}\right)
^{2}\frac{\widetilde{\Delta}}{T}\sum_{k=1}^{\widetilde{n}}%
V_{(k-1)\widetilde{\Delta}}\epsilon_{k\widetilde{\Delta}}^{2}\\
&  -4\left(  \frac{\widetilde{\Delta}}{T}\sum_{k=1}^{\widetilde{n}%
}V_{(k-1)\widetilde{\Delta}}\epsilon_{k\widetilde{\Delta}}^{2}\right)
\frac{\widetilde{\Delta}}{T}\sum_{k=1}^{\widetilde{n}}%
V_{(k-1)\widetilde{\Delta}}^{2}\epsilon_{k\widetilde{\Delta}}^{4}\\
&  =\mathrm{var}\left(  S_{T,\widetilde{\Delta}}\right)  +o_{p}(1).
\end{align*}
As $S_{T,\widetilde{\Delta}}^{\ast}\overset{d^{\ast}}{\rightarrow}N\left(
0,\mathrm{var}^{\ast}\left(  S_{T,\widetilde{\Delta}}^{\ast}\right)  \right)
,$ the statement in (i) follows.

\noindent(ii) $S_{T,\Delta}^{\ast}$ now contains the following additional
term:%
\[
J_{T,\widetilde{\Delta}}^{\ast}=\frac{1}{\sqrt{T}\widetilde{\Delta}}\sum
_{k=1}^{\widetilde{n}}\left(  Z_{(k-1)\widetilde{\Delta}}^{\ast3}\left(
N_{(k-1)\widetilde{\Delta}}^{\ast}-N_{(k-1)\widetilde{\Delta}}^{\ast}\right)
^{3}-Z_{(k-1)\widetilde{\Delta}}^{3}\left(  N_{(k-1)\widetilde{\Delta}%
}-N_{(k-1)\widetilde{\Delta}}\right)  ^{3}\right)  ,
\]
plus additional cross product terms, which cannot be of larger $P^{\ast}%
-$order than $J_{T,\widetilde{\Delta}}^{\ast}.$ Now, \textrm{E}$^{\ast}\left(
J_{T,\widetilde{\Delta}}^{\ast}\right)  =0$, and%
\begin{align*}
&  \mathrm{var}^{\ast}\left(  J_{T,\widetilde{\Delta}}^{\ast}\right) \\
&  =\frac{1}{\widetilde{\Delta}^{2}}\frac{\widetilde{\Delta}}{T}\sum
_{k=1}^{\widetilde{n}}\frac{1}{\widetilde{\Delta}}\left(
Z_{(k-1)\widetilde{\Delta}}^{6}\left(  N_{(k-1)\widetilde{\Delta}%
}-N_{(k-1)\widetilde{\Delta}}\right)  ^{6}\right) \\
&  =\frac{1}{\widetilde{\Delta}^{2}}O_{p}(1),
\end{align*}
and so $S_{T,\widetilde{\Delta}}^{\ast}$ is of $P^{\ast}-$order $\frac
{1}{\widetilde{\Delta}}.$ Recalling that $S_{T,\widetilde{\Delta}}$ is of
$P-$order $\frac{1}{\Delta},$ with $\Delta/\widetilde{\Delta}\rightarrow0,$
the statement in (ii) follows.

\noindent\textbf{Proof of Theorem 3: }Suppose that \thinspace$H_{0}$ is true,
and so $\lambda_{\infty}=0.$ In this case:

\noindent\textrm{E}$\left(  \left(  Y_{k\Delta}-Y_{(k-1)\Delta}\right)
^{3}\right)  =O\left(  \Delta^{2}\right)  ,$ and by the same argument as in
the proof of Theorem 1,%
\begin{align*}
&  \frac{1}{T}\sum_{k=1}^{n}\left(  Y_{k\Delta}-Y_{(k-1)\Delta}\right)  ^{3}\\
&  =\frac{1}{T}\sum_{k=1}^{n}\left(  \left(  Y_{k\Delta}-Y_{(k-1)\Delta
}\right)  ^{3}-\mathrm{E}\left(  \left(  Y_{k\Delta}-Y_{(k-1)\Delta}\right)
^{3}\right)  \right)  +\mathrm{E}\left(  \left(  Y_{k\Delta}-Y_{(k-1)\Delta
}\right)  ^{3}\right) \\
&  =O_{p}\left(  \frac{\Delta}{\sqrt{T}}\right)  +\mathrm{E}\left(  \left(
Y_{k\Delta}-Y_{(k-1)\Delta}\right)  ^{3}\right)  \Delta^{-1}=O_{p}\left(
\frac{\Delta}{\sqrt{T}}\right)  +O(\Delta).
\end{align*}
Hence,%
\[
S_{T,\Delta}=\frac{1}{\sqrt{T}\Delta}\sum_{k=1}^{n}\left(  Y_{k\Delta
}-Y_{(k-1)\Delta}\right)  ^{3}=O_{p}\left(  1\right)  +O(\sqrt{T}).
\]
On the other hand, it is immediate to see from the proof of Theorem 2, part
(i), that $\mathrm{E}\left(  S_{T,\widetilde{\Delta}}^{\ast}\right)  =0,$
regardless of the presence of leverage. This is because the mean of the
bootstrap statistic is always zero. Hence, the comparison of $S_{T,\Delta}$
with the critical values of $S_{T,\widetilde{\Delta}}^{\ast}$ will lead to a
rejection of the null, with probability approaching one.

\noindent\textbf{Proof of Theorem 4:}

\noindent\textbf{(i) }Under $H_{0},$ $S_{T,\Delta}=O_{p}(\sqrt{T}),$ so that
$\widetilde{S}_{T,\Delta}=O_{p}(T^{-\varepsilon})=o_{p}(1).$ As
$S_{T,\widetilde{\Delta}}^{\ast}$ has a well defined, zero mean, normal
limiting distribution (regardless the presence of leverage), the statement in
(i) follows.

\noindent\textbf{(ii) }Under $H_{A}^{(1)},$ $S_{T,\Delta}$ is of probability
order $\frac{\sqrt{T}}{\Delta},$ and so $\widetilde{S}_{T,\Delta}$ is of
probability order $\frac{1}{T^{\varepsilon}\Delta}.$ Under $H_{A}^{(2)},$
$S_{T,\Delta}$ is of probability order $\frac{1}{\Delta},$ and so
$\widetilde{S}_{T,\Delta}$ is of probability order $\frac{1}%
{T^{1/2+\varepsilon}\Delta}.$ Now, from the proof of Theorem 2, we have that
under both $H_{A}^{(1)}$ and $H_{A}^{(2)},$ $S_{T,\widetilde{\Delta}}^{\ast}$
diverges at rate $\frac{1}{\widetilde{\Delta}}.$ As $\frac{T^{1/2+\varepsilon
}\Delta}{\widetilde{\Delta}}\rightarrow0,$ the statement follows.

\noindent\textbf{Proof of Theorem 5:}

\noindent(i) Recall $Eq.$ (\ref{beta}), which can be written as follows:%
\begin{align*}
&  \sqrt{\frac{T}{\Delta}}\widehat{\tau}_{T,\Delta}\\
&  =\frac{1}{\sqrt{T\Delta}}\sum_{k=1}^{n-1}\left(  \Delta^{1/2}%
V_{(k-1)\Delta}^{1/2}\epsilon_{k\Delta}-\frac{\Delta}{T}\sum_{k=1}^{n-1}%
\Delta^{1/2}V_{(k-1)\Delta}^{1/2}\epsilon_{k\Delta}+Z_{(k-1)\Delta}\left(
N_{k\Delta}-N_{(k-1)\Delta}\right)  \right. \\
&  \left.  -\frac{\Delta}{T}\sum_{k=1}^{n-1}Z_{(k-1)\Delta}\left(  N_{k\Delta
}-N_{(k-1)\Delta}\right)  \right)  \left(  \Delta^{1/2}V_{k\Delta}%
^{1/2}\epsilon_{(k+1)\Delta}-\frac{\Delta}{T}\sum_{k=1}^{n-1}\Delta
^{1/2}V_{k\Delta}^{1/2}\epsilon_{(k+1)\Delta}\right. \\
&  \left.  +Z_{k\Delta}\left(  N_{(k+1)\Delta}-N_{k\Delta}\right)
-\frac{\Delta}{T}\sum_{k=1}^{n-1}Z_{k\Delta}\left(  N_{(k+1)\Delta}%
-N_{k\Delta}\right)  \right)  \left(  1+o_{p}(1)\right) \\
&  =\frac{1}{\sqrt{T\Delta}}\sum_{k=1}^{n-1}\left(  \left(  \Delta
^{1/2}V_{(k-1)\Delta}^{1/2}\epsilon_{k\Delta}+Z_{(k-1)\Delta}\left(
N_{k\Delta}-N_{(k-1)\Delta}\right)  -\mathrm{E}\left(  Z\right)
\mathrm{E}\left(  \lambda\right)  \Delta\right)  \right. \\
&  \left.  \left(  \Delta^{1/2}V_{k\Delta}^{1/2}\epsilon_{(k+1)\Delta
}+Z_{k\Delta}\left(  N_{(k+1)\Delta}-N_{k\Delta}\right)  -\mathrm{E}\left(
Z\right)  \mathrm{E}\left(  \lambda\right)  \Delta\right)  \right)  +o_{p}(1).
\end{align*}
Let $\overline{Z}_{k\Delta}\left(  N_{(k+1)\Delta}-N_{k\Delta}\right)
=Z_{k\Delta}\left(  N_{(k+1)\Delta}-N_{k\Delta}\right)  -\mathrm{E}\left(
Z\right)  \mathrm{E}\left(  \lambda\right)  \Delta,$ and note that under the
null of constant intensity,%
\begin{align*}
&  \mathrm{var}\left(  \frac{1}{\sqrt{T\Delta}}\sum_{k=1}^{n-1}\overline
{Z}_{k\Delta}\left(  N_{(k+1)\Delta}-N_{k\Delta}\right)  \overline
{Z}_{(k-1)\Delta}\left(  N_{k\Delta}-N_{(k-1)\Delta}\right)  \right) \\
&  =\Delta^{-2}\mathrm{E}\left(  \left(  \overline{Z}_{k\Delta}\left(
N_{(k+1)\Delta}-N_{k\Delta}\right)  \right)  ^{2}\left(  \overline
{Z}_{(k-1)\Delta}\left(  N_{k\Delta}-N_{(k-1)\Delta}\right)  \right)
^{2}\right) \\
&  =\Delta^{-2}\mathrm{E}\left(  \left(  \overline{Z}_{k\Delta}\left(
N_{(k+1)\Delta}-N_{k\Delta}\right)  \right)  ^{2}\right)  \mathrm{E}\left(
\left(  \overline{Z}_{(k-1)\Delta}\left(  N_{k\Delta}-N_{(k-1)\Delta}\right)
\right)  ^{2}\right) \\
&  =\left(  \mathrm{E}\left(  \lambda\right)  \mathrm{E}\left(  \left(
Z-\mathrm{E}\left(  Z\right)  \right)  ^{2}\right)  \right)  ^{2},
\end{align*}%
\begin{align*}
&  \mathrm{var}\left(  \frac{1}{\sqrt{T\Delta}}\sum_{k=1}^{n-1}\Delta
V_{(k-1)\Delta}^{1/2}V_{k\Delta}^{1/2}\epsilon_{k\Delta}\epsilon_{(k+1)\Delta
}\right) \\
&  =\mathrm{E}\left(  V_{(k-1)\Delta}V_{k\Delta}\right)  ,
\end{align*}
and%
\[
\mathrm{cov}\left(  \frac{1}{\sqrt{T\Delta}}\sum_{k=1}^{n-1}\overline
{Z}_{k\Delta}\left(  N_{(k+1)\Delta}-N_{k\Delta}\right)  ,\frac{1}%
{\sqrt{T\Delta}}\sum_{k=1}^{n-1}\Delta^{1/2}V_{(k-1)\Delta}^{1/2}\right)  =0.
\]
Now, consider $\widehat{\sigma}_{\beta,T,\Delta}^{2},$ the expression for
which can be written as follows:%
\begin{align*}
\widehat{\sigma}_{\beta,T,\Delta}^{2}  &  =\frac{\Delta}{T}\sum_{k=1}%
^{n-1}V_{(k-1)\Delta}^{2}V_{k\Delta}^{2}\epsilon_{k\Delta}^{2}\epsilon
_{(k+1)\Delta}^{2}\\
&  +\frac{\Delta}{T}\sum_{k=1}^{n-1}\Delta^{-2}Z_{(k-1)\Delta}^{2}\left(
N_{k\Delta}-N_{(k-1)\Delta}\right)  ^{2}Z_{k\Delta}^{2}\left(  N_{(k-1)\Delta
}-N_{k\Delta}\right)  ^{2}+o_{p}(1)\\
&  =\left(  \mathrm{E}\left(  V_{(k-1)\Delta}^{2}\epsilon_{k\Delta}%
^{2}\right)  \right)  ^{2}+\left(  \mathrm{E}\left(  \lambda\right)
\mathrm{E}\left(  \left(  Z-\mathrm{E}\left(  Z\right)  \right)  ^{2}\right)
\right)  ^{2}+o_{p}(1).
\end{align*}
The statement then follows immediately from the central limit theorem for
$iid$ random variables and from the continuous mapping theorem.

\noindent(ii) Note also that $\widehat{\tau}_{T,\Delta}$ can be written as
follows:
\begin{align*}
&  \widehat{\tau}_{T,\Delta}\\
&  =\Delta\frac{\Delta}{T}\sum_{k=1}^{n-1}\Delta^{-2}\overline{Z}%
_{(k-1)\Delta}\left(  N_{k\Delta}-N_{(k-1)\Delta}\right)  \overline
{Z}_{k\Delta}\left(  N_{(k+1)\Delta}-N_{k\Delta}\right)  +o_{p}(1),
\end{align*}
and%
\begin{align*}
&  \frac{\Delta}{T}\sum_{k=1}^{n-1}\Delta^{-2}\overline{Z}_{(k-1)\Delta
}\overline{Z}_{k\Delta}\left(  N_{k\Delta}-N_{(k-1)\Delta}\right)  \left(
N_{(k+1)\Delta}-N_{k\Delta}\right) \\
&  \overset{p}{\rightarrow}\frac{\beta\lambda\left(  2a-\beta\right)
}{2\left(  a-\beta\right)  }\exp\left(  -\left(  a-\beta\right)  \tau\right)
\mathrm{E}\left(  Z\right)  ^{2}\\
&  >0.
\end{align*}
The statement in the theorem follows by noting that:%
\[
\frac{\Delta}{T}\sum_{k=1}^{n-1}\Delta^{-2}Z_{(k-1)\Delta}^{2}\left(
N_{k\Delta}-N_{(k-1)\Delta}\right)  ^{2}Z_{k\Delta}^{2}\left(  N_{(k-1)\Delta
}-N_{k\Delta}\right)  ^{2}=O_{p}(1)
\]
under both hypotheses, given that for $\beta<a,$ the time dependence of jumps
declines at an exponential rate.

\pagebreak

\section{Reference}

\noindent Ait-Sahalia, Y., J. Cacho-Diaz and R. Laeven (2013). Modeling
Financial Contagion Using Mutually Exciting Jump Processes. Working Paper,
Princeton University.\smallskip

\noindent Ait-Sahalia, Y., J. Jacod (2009). Testing for Jumps in a Discretely
Observed Process. \textit{Annals of Statistics, }37, 2202-2244.\smallskip

\noindent Ait-Sahalia, Y., J. Jacod and J. Li (2012). Testing for Jumps in
Noisy High Frequency Data. \textit{Journal of Econometrics, }168,
207-222.\smallskip

\noindent Andersen, T.G., T. Bollerslev and F.X. Diebold (2000). Great
Realizations. \textit{Risk, }105-108.\smallskip

\noindent Andersen, Benzoni and Lund (2002). An Empirical Investigation of
Continuous Time Equity Return Models. \textit{Journal of Finance, }62,
1239-1283\smallskip.

\noindent Andrews, D.W.K. (1999). Estimation When a Parameter is on the
Boundary. \textit{Econometrica, }67, 1341-1383.\smallskip

\noindent Andrews, D.W.K. (2001). Testing When a Parameter is on the Boundary
of the Maintained Hypothesis. \textit{Econometrica, }69, 683-734.\smallskip

\noindent Andrews, D.W.K. and X. Cheng (2012). Estimation and Inference with
Weak, Semi-strong and Strong Identification. \textit{Econometrica, }80,
2153-2211.\smallskip

\noindent Bandi, F.M. and R. Reno (2012). Time Varying Leverage Effects.
\textit{Journal of Econometrics, }169, 94-113.\smallskip

\noindent Barndorff-Nielsen, O.E. and N. Shephard (2004). Power and Bipower
Variation with Stochastic Volatility and Jumps. \textit{Journal of Financial
Econometrics, }2, 1-48.\smallskip

\noindent Barndorff-Nielsen, O.E., N. Shephard and M. Winkel (2006). Limit
Theorem for Multipower Variation in the Presence of Jumps. \textit{Stochastic
Processes and Their Applications, }116, 796-806.\smallskip

\noindent Beg, A.B.M.R.A., M.J. Silvapulle and P. Silvapulle (2001). Testing
Against Inequality Constraints When Some Nuisance Parameters are Present Only
Under the Alternative: Test of ARCH in ARCH-M Models. \textit{Journal of
Business and Economic Statistics, }19, 245-253.\smallskip

\noindent Benjamini, Y., and Y. Hochberg (1995). Controlling the False
Discovery Rate: A Practical and Powerful Approach to Multiple Testing.
\textit{Journal of the Royal Statistical Society, B}, 57, 289-300.\smallskip

\noindent Bowsher, C.G. (2007). Modeling Security Market Event in Continuous
Time: Intensity-Based Multivariate, Point Process Models. \textit{Journal of
Econometrics, }141, 876-912.\smallskip

\noindent Chacko, G., and L.M. Viceira (2003). Spectral GMM\ estimation of
continuous-time Processes. \textit{Journal of Econometrics,} 116,
259-292.\smallskip

\noindent Corradi, V. and N.R. Swanson, \textquotedblleft Predictive Density
Construction and Testing with Multiple Possibly Misspecified Diffusion
Models\textquotedblright, \textit{Journal of Econometrics, }161, 304-324,
2011.\smallskip

\noindent Corsi, F., D. Pirino and R. Reno (2010). Threshold Bipower Variation
and the Impact of Jumps on Volatility Forecasting. \textit{Journal of
Econometrics,} 159, 276-288.\smallskip

\noindent Duffie, D. and K. Singleton (1993). Simulated Moment Estimation of
Markov Models of Asset Prices.\textit{\ Econometrica,} 61, 929-952.\smallskip

\noindent Duffie, D., J. Pan and K. Singleton (2000). Transform Analysis and
Asset Pricing for Affine Jump Diffusion. \textit{Econometrica, }68,
1343-1376.\smallskip

\noindent Fermanian, J.-D. and B. Salani\'{e} (2004). A Nonparametric
Simulated Maximum Likelihood Estimation Method. \textit{Econometric Theory,}
20, 701-734.\smallskip

\noindent Eraker, B., M. Johannes, and N. Polson (2003). The Impact of Jumps
in Volatility and Returns. \textit{Journal of Finance} 58,
1269-1300.\smallskip

\noindent Escanciano, J.C. and I.N. Lobato (2009). An Automatic Data-Driven
Portmanteau Test for Testing Serial Autocorrelation. \textit{Journal of
Econometrics, }151, 140-149.\smallskip

\noindent Gallant, A.R. and G. Tauchen (1996). Which Moments to Match.
\textit{Econometric Theory,} 12, 657-681.\smallskip

\noindent Giacomini, R., D. Politis and H. White (2013). A Warp-Speed Method
for Conducting Monte Carlo Experiments Involving Bootstrap Estimators.
\textit{Econometric Theory, }29, 567-589.\smallskip

\noindent Gourieroux, C., A. Monfort, and E. Renault (1993). Indirect
Inference. \textit{Journal of Applied Econometrics,} 8, 203-227.\smallskip

\noindent Holm, S. (1979). A Simple Sequentially Rejective Multiple Test
Procedure. \textit{Scandinavian Journal of Statistics}, 6, 65--70.\smallskip

\noindent Huang, X. and G.E. Tauchen (2005). The Relative Contribution of
Jumps to Total Price Variance. \textit{Journal of Financial Econometrics, }3,
456-499.\smallskip

\noindent Jiang, G.J. and J.L. Knight (2002). Estimation of Continuous-time
Processes via the Empirical Characteristic Function. \textit{Journal of
Business and Economic Statistics,} 20, 198-212.\smallskip

\noindent Lee, S. and P.A. Mykland (2008). Jumps in Financial Markets: A New
Nonparametric Test and Jump Dynamics. \textit{Review of Financial Studies,
}21, 2535-2563.\smallskip

\noindent Lee, T., M. Loretan and W. Ploberger (2013). Rate-Optimal Tests for
Jumps in Diffusion Processes. \textit{Statistical Papers. }54,
1009-1041.\smallskip

\noindent Podolskij, M. and M. Vetter (2009a). Bipower Type Estimation in a
Noisy Diffusion Setting. \textit{Stochastic Processes and Their Applications.
}119, 2803-2832.\smallskip

\noindent Podolskij, M. and M. Vetter (2009b). Estimation of Volatility
Functionals in the Simultaneous Presence of Microstructure Noise and Jumps.
\textit{Bernoulli, }15, 634-658.\smallskip

\noindent Romano, J.P. and M. Wolf (2005). Stepwise Multiple Testing as
Formalized Data Snooping. \textit{Econometrica}, 73, 1237-1282.\smallskip

\noindent Silvapulle, M.J. and P.K. Sen (2005). \textit{Constrained
Statistical Inference: Inequality, Order and Shape Restrictions.
}Wiley.\smallskip

\noindent Singleton, K.J. (2001). Estimation of Affine Asset Pricing Models
Using Empirical Characteristic Function. \textit{Journal of Econometrics,}
102, 111-141.\smallskip

\noindent Storey, J.D. (2003). The Positive False Discovery Rate: a Bayesian
Interpretation and the $q$-value. \textit{Annals of Statistics}, 31,
2013-2035.\smallskip

\noindent White, H. (2000). A Reality Check For Data Snooping.
\textit{Econometrica}, 68, 1097-1127.

\textwidth=6.5in \textheight=9in \oddsidemargin=0in \evensidemargin=0in
\topmargin=-0.25in \renewcommand {\baselinestretch}{1.0}

\pagebreak

\begin{table}[ptb]
\begin{center}
{Table 1: Experimental Results - Empirical Size *}
\par
{\textit{Data Generated According to a Stochastic Volatility Process}}
\end{center}
\par
\label{tb11}
\par
\begin{center}%
\begin{tabular}
[c]{cccccc}\hline\hline
\multicolumn{3}{c}{No Leverage in DGP} & \multicolumn{3}{c}{Leverage in DGP}\\
T=300 & T=500 & T=1000 & T=300 & T=500 & T=1000\\
&  &  &  &  & \\\hline
\multicolumn{6}{c}{{\textit{Jump Test}}}\\\hline
0.110 & 0.104 & 0.102 & 0.000 & 0.000 & 0.000\\\hline
\multicolumn{6}{c}{{\textit{Self Excitement Test}}}\\
\multicolumn{6}{c}{{\textit{Jumps Generated Using Exponential Dist.}}}\\\hline
0.119 & 0.102 & 0.103 & 0.111 & 0.108 & 0.117\\
0.090 & 0.100 & 0.088 & 0.111 & 0.116 & 0.118\\
0.080 & 0.076 & 0.081 & 0.088 & 0.075 & 0.083\\
0.080 & 0.081 & 0.070 & 0.083 & 0.072 & 0.074\\
0.058 & 0.067 & 0.067 & 0.072 & 0.064 & 0.066\\\hline
\multicolumn{6}{c}{{\textit{Self Excitement Test}}}\\
\multicolumn{6}{c}{{\textit{Jumps Generated Using Normal Dist.}}}\\\hline
0.117 & 0.101 & 0.101 & 0.111 & 0.111 & 0.110\\
0.086 & 0.110 & 0.100 & 0.109 & 0.116 & 0.124\\
0.090 & 0.094 & 0.102 & 0.096 & 0.092 & 0.097\\
0.097 & 0.102 & 0.086 & 0.088 & 0.094 & 0.100\\
0.078 & 0.080 & 0.088 & 0.081 & 0.083 & 0.080\\\hline
\multicolumn{6}{c}{{\textit{Sequential Jump and Self Excitement Test}}}\\
\multicolumn{6}{c}{{\textit{Jumps Generated Using Exponential Dist.}}}\\\hline
0.111 & 0.102 & 0.103 & 0.107 & 0.099 & 0.114\\
0.090 & 0.100 & 0.088 & 0.110 & 0.116 & 0.118\\
0.080 & 0.076 & 0.081 & 0.088 & 0.075 & 0.083\\
0.080 & 0.081 & 0.070 & 0.083 & 0.072 & 0.074\\
0.058 & 0.067 & 0.067 & 0.072 & 0.064 & 0.066\\\hline
\multicolumn{6}{c}{{\textit{Sequential Jump and Self Excitement Test}}}\\
\multicolumn{6}{c}{{\textit{Jumps Generated Using Normal Dist.}}}\\\hline
0.114 & 0.101 & 0.101 & 0.100 & 0.105 & 0.107\\
0.086 & 0.110 & 0.100 & 0.109 & 0.116 & 0.124\\
0.090 & 0.094 & 0.102 & 0.096 & 0.092 & 0.097\\
0.097 & 0.102 & 0.086 & 0.088 & 0.094 & 0.100\\
0.078 & 0.080 & 0.088 & 0.081 & 0.083 & 0.080\\\hline\hline
\end{tabular}
\end{center}
\par
* Notes: Entries denote rejection frequencies based on comparing $S_{T,\Delta
}$ with 10\% critical values calculated using the bootstrap ({\textit{Jump
Test}}) and based on comparing $Z_{T,\Delta}$ with 10\% critical values from
the normal distribution ({\textit{Self Excitement Test}}). Additionally,
results are reported for the sequential test whereby $Z_{T,\Delta}$ is
constructed whenever implementation of $S_{T,\Delta}$ results in a rejection
of the null of no jumps. Results are reported in "blocks" of 5 rows. Each of
these 5 rows of entries corresponds to a different jump intensity. As
discussed above, "average jump arrival times" are assumed to be every
\textit{\{20 days, 10 days, 5 days, 10/3 days, 5/2 days\}}, and the rows
correspond to these arrivals, in order from least frequent to most frequent.
All experiments are based on 1,000 Monte Carlo iterations. For complete
details see above.\end{table}

\newpage

\begin{table}[ptb]
\begin{center}
{Table 2: Experimental Results - Jump Test Empirical Power *}
\par
{\textit{Data Generated According to a Stochastic Volatility Process}}
\end{center}
\par
\label{tb12}
\par
\begin{center}%
\begin{tabular}
[c]{ccccccccc}\hline\hline
&  & T=300 & T=500 & T=1000 & T=300 & T=500 & T=1000 & \\\hline
\multicolumn{9}{c}{{\textit{Panel A: No Leverage, No Self-Excitement in DGP}}%
}\\\hline
\multicolumn{2}{c}{} & \multicolumn{3}{c}{Exponential Dist.} &
\multicolumn{3}{c}{Normal Dist.} & \multicolumn{1}{c}{}\\
&  & 0.938 & 0.991 & 1.000 & 0.973 & 1.000 & 1.000 & \\
&  & 0.999 & 1.000 & 1.000 & 1.000 & 1.000 & 1.000 & \\
&  & 1.000 & 1.000 & 1.000 & 1.000 & 1.000 & 1.000 & \\
&  & 1.000 & 1.000 & 1.000 & 1.000 & 1.000 & 1.000 & \\
&  & 1.000 & 1.000 & 1.000 & 1.000 & 1.000 & 1.000 & \\
\multicolumn{9}{c}{{\textit{Panel B: Leverage, No Self-Excitement in DGP}}%
}\\\hline
\multicolumn{2}{c}{} & \multicolumn{3}{c}{Exponential Dist.} &
\multicolumn{3}{c}{Normal Dist.} & \multicolumn{1}{c}{}\\
&  & 0.933 & 0.960 & 0.976 & 0.903 & 0.931 & 0.985 & \\
&  & 0.981 & 0.999 & 1.000 & 0.994 & 1.000 & 1.000 & \\
&  & 1.000 & 1.000 & 1.000 & 1.000 & 1.000 & 1.000 & \\
&  & 1.000 & 1.000 & 1.000 & 1.000 & 1.000 & 1.000 & \\
&  & 1.000 & 1.000 & 1.000 & 1.000 & 1.000 & 1.000 & \\\hline\hline
\multicolumn{3}{c}{$a=0.1$,$\beta=0.2$} & \multicolumn{3}{c}{$a=2$,$\beta=1$}
& \multicolumn{3}{c}{$a=5$,$\beta=4$}\\
T=300 & T=500 & T=1000 & T=300 & T=500 & T=1000 & T=300 & T=500 &
T=1000\\\hline
\multicolumn{9}{c}{{\textit{Panel C: No Leverage, Self-Excitement in DGP,
Jumps Generated Using Exponential Dist.}}}\\\hline
0.917 & 0.939 & 0.958 & 0.545 & 0.511 & 0.797 & 0.811 & 0.917 & 1.000\\
0.975 & 0.995 & 1.000 & 0.682 & 0.865 & 0.996 & 0.921 & 0.978 & 1.000\\
0.995 & 1.000 & 1.000 & 0.993 & 0.999 & 1.000 & 0.998 & 1.000 & 1.000\\
1.000 & 1.000 & 1.000 & 0.999 & 1.000 & 0.999 & 1.000 & 1.000 & 1.000\\
1.000 & 1.000 & 1.000 & 1.000 & 0.999 & 1.000 & 0.999 & 1.000 & 0.999\\
\multicolumn{9}{c}{{\textit{Panel D: No Leverage, Self-Excitement in DGP,
Jumps Generated Using Normal Dist.}}}\\\hline
0.952 & 0.978 & 0.983 & 0.919 & 0.986 & 0.995 & 0.974 & 0.998 & 0.995\\
0.984 & 0.983 & 0.972 & 0.986 & 0.996 & 0.985 & 0.988 & 0.989 & 0.992\\
0.986 & 0.969 & 0.966 & 0.980 & 0.983 & 0.955 & 0.969 & 0.974 & 0.955\\
0.973 & 0.968 & 0.960 & 0.964 & 0.960 & 0.929 & 0.956 & 0.952 & 0.941\\
0.952 & 0.952 & 0.968 & 0.952 & 0.936 & 0.916 & 0.963 & 0.940 & 0.903\\
\multicolumn{9}{c}{{\textit{Panel E: Leverage, Self-Excitement in DGP, Jumps
Generated Using Exponential Dist.}}}\\\hline
0.906 & 0.917 & 0.944 & 0.345 & 0.261 & 0.288 & 0.370 & 0.812 & 0.729\\
0.974 & 0.987 & 0.999 & 0.458 & 0.471 & 0.471 & 0.725 & 0.699 & 0.509\\
0.995 & 0.999 & 1.000 & 0.767 & 0.757 & 0.720 & 0.802 & 0.834 & 0.956\\
1.000 & 1.000 & 1.000 & 0.941 & 0.953 & 0.964 & 0.976 & 0.943 & 1.000\\
1.000 & 1.000 & 1.000 & 0.998 & 0.997 & 0.998 & 0.996 & 0.999 & 1.000\\
\multicolumn{9}{c}{{\textit{Panel F: Leverage, Self-Excitement in DGP, Jumps
Generated Using Normal Dist.}}}\\\hline
0.948 & 0.939 & 0.929 & 0.592 & 0.668 & 0.697 & 0.784 & 0.921 & 0.949\\
0.974 & 0.963 & 0.957 & 0.915 & 0.939 & 0.967 & 0.985 & 0.988 & 0.978\\
0.966 & 0.940 & 0.899 & 0.964 & 0.963 & 0.932 & 0.981 & 0.965 & 0.915\\
0.957 & 0.930 & 0.908 & 0.950 & 0.937 & 0.903 & 0.960 & 0.951 & 0.887\\
0.951 & 0.932 & 0.870 & 0.942 & 0.922 & 0.880 & 0.952 & 0.934 &
0.851\\\hline\hline
\end{tabular}
\end{center}
\par
* See notes to Table 1. In results reported in Panels A-B, jump intensity
follows a Poisson distribution, with constant intensity, while in Panels C-F,
intensity follows a Hawkes diffusion, and $(a,\beta)$ parameterize path
dependence strength. For complete details see above.\end{table}

\newpage

\begin{table}[ptb]
\begin{center}
{Table 3: Experimental Results - Self-Excitement Test Empirical Power *}
\par
{\textit{Data Generated According to a Stochastic Volatility Process}}
\end{center}
\par
\label{tb13}
\par
\begin{center}%
\begin{tabular}
[c]{ccccccccc}\hline\hline
\multicolumn{3}{c}{$a=0.1$,$\beta=0.2$} & \multicolumn{3}{c}{$a=2$,$\beta=1$}
& \multicolumn{3}{c}{$a=5$,$\beta=4$}\\
T=300 & T=500 & T=1000 & T=300 & T=500 & T=1000 & T=300 & T=500 &
T=1000\\\hline
\multicolumn{9}{c}{{\textit{Panel A: No Leverage, Self-Excitement in DGP,
Jumps Generated Using Exponential Dist.}}}\\\hline
0.889 & 0.885 & 0.861 & 0.203 & 0.224 & 0.215 & 0.151 & 0.124 & 0.130\\
0.965 & 0.963 & 0.937 & 0.316 & 0.343 & 0.282 & 0.172 & 0.167 & 0.164\\
0.952 & 0.932 & 0.919 & 0.456 & 0.457 & 0.448 & 0.223 & 0.200 & 0.181\\
0.916 & 0.906 & 0.874 & 0.576 & 0.555 & 0.530 & 0.283 & 0.298 & 0.250\\
0.882 & 0.867 & 0.839 & 0.610 & 0.624 & 0.567 & 0.360 & 0.311 & 0.264\\\hline
\multicolumn{9}{c}{{\textit{Panel B: No Leverage, Self-Excitement in DGP,
Jumps Generated Using Normal Dist.}}}\\\hline
0.898 & 0.894 & 0.875 & 0.204 & 0.246 & 0.220 & 0.169 & 0.144 & 0.139\\
0.969 & 0.973 & 0.970 & 0.322 & 0.368 & 0.327 & 0.194 & 0.186 & 0.176\\
0.973 & 0.959 & 0.945 & 0.480 & 0.496 & 0.482 & 0.258 & 0.236 & 0.236\\
0.962 & 0.941 & 0.932 & 0.618 & 0.614 & 0.595 & 0.331 & 0.339 & 0.322\\
0.935 & 0.924 & 0.911 & 0.672 & 0.675 & 0.642 & 0.402 & 0.408 & 0.322\\\hline
\multicolumn{9}{c}{{\textit{Panel C: Leverage, Self-Excitement in DGP, Jumps
Generated Using Exponential Dist.}}}\\\hline
0.921 & 0.905 & 0.900 & 0.242 & 0.248 & 0.241 & 0.185 & 0.153 & 0.144\\
0.968 & 0.960 & 0.965 & 0.372 & 0.373 & 0.344 & 0.164 & 0.184 & 0.166\\
0.949 & 0.950 & 0.905 & 0.530 & 0.527 & 0.517 & 0.294 & 0.288 & 0.231\\
0.943 & 0.925 & 0.910 & 0.663 & 0.624 & 0.629 & 0.326 & 0.321 & 0.295\\
0.932 & 0.912 & 0.860 & 0.738 & 0.693 & 0.654 & 0.428 & 0.384 & 0.353\\\hline
\multicolumn{9}{c}{{\textit{Panel D: Leverage, Self-Excitement in DGP, Jumps
Generated Using Normal Dist.}}}\\\hline
0.920 & 0.909 & 0.906 & 0.256 & 0.250 & 0.262 & 0.183 & 0.161 & 0.135\\
0.979 & 0.974 & 0.980 & 0.385 & 0.386 & 0.380 & 0.176 & 0.207 & 0.191\\
0.976 & 0.966 & 0.949 & 0.560 & 0.550 & 0.557 & 0.314 & 0.304 & 0.282\\
0.970 & 0.958 & 0.951 & 0.680 & 0.654 & 0.666 & 0.360 & 0.369 & 0.342\\
0.966 & 0.952 & 0.925 & 0.766 & 0.745 & 0.718 & 0.470 & 0.431 &
0.411\\\hline\hline
\end{tabular}
\end{center}
\par
* See notes to Table 2.\end{table}

\newpage

\begin{table}[ptb]
\begin{center}
{Table 4: Experimental Results - Sequential Jump and Self Excitement Test
Empirical Power *}
\par
{\textit{Data Generated According to a Stochastic Volatility Process}}
\end{center}
\par
\label{tb14}
\par
\begin{center}%
\begin{tabular}
[c]{ccccccccc}\hline\hline
\multicolumn{3}{c}{$a=0.1$,$\beta=0.2$} & \multicolumn{3}{c}{$a=2$,$\beta=1$}
& \multicolumn{3}{c}{$a=5$,$\beta=4$}\\
T=300 & T=500 & T=1000 & T=300 & T=500 & T=1000 & T=300 & T=500 &
T=1000\\\hline
\multicolumn{9}{c}{{\textit{Panel A: No Leverage, Self-Excitement in DGP,
Jumps Generated Using Exponential Dist.}}}\\\hline
0.855 & 0.869 & 0.859 & 0.167 & 0.167 & 0.194 & 0.127 & 0.118 & 0.130\\
0.941 & 0.960 & 0.937 & 0.283 & 0.330 & 0.282 & 0.164 & 0.164 & 0.164\\
0.947 & 0.932 & 0.919 & 0.455 & 0.457 & 0.448 & 0.223 & 0.200 & 0.181\\
0.916 & 0.906 & 0.874 & 0.575 & 0.555 & 0.529 & 0.283 & 0.298 & 0.250\\
0.882 & 0.867 & 0.839 & 0.610 & 0.623 & 0.567 & 0.359 & 0.311 & 0.263\\\hline
\multicolumn{9}{c}{{\textit{Panel B: No Leverage, Self-Excitement in DGP,
Jumps Generated Using Normal Dist.}}}\\\hline
0.887 & 0.888 & 0.868 & 0.197 & 0.244 & 0.219 & 0.164 & 0.144 & 0.139\\
0.967 & 0.964 & 0.963 & 0.320 & 0.366 & 0.321 & 0.189 & 0.185 & 0.175\\
0.968 & 0.954 & 0.939 & 0.477 & 0.491 & 0.474 & 0.252 & 0.230 & 0.226\\
0.955 & 0.930 & 0.928 & 0.614 & 0.608 & 0.569 & 0.323 & 0.329 & 0.307\\
0.922 & 0.910 & 0.905 & 0.663 & 0.659 & 0.622 & 0.397 & 0.388 & 0.297\\\hline
\multicolumn{9}{c}{{\textit{Panel C: Leverage, Self-Excitement in DGP, Jumps
Generated Using Exponential Dist.}}}\\\hline
0.869 & 0.875 & 0.892 & 0.170 & 0.165 & 0.167 & 0.101 & 0.133 & 0.114\\
0.945 & 0.951 & 0.965 & 0.306 & 0.318 & 0.291 & 0.137 & 0.154 & 0.121\\
0.944 & 0.949 & 0.905 & 0.513 & 0.501 & 0.487 & 0.274 & 0.270 & 0.227\\
0.943 & 0.925 & 0.910 & 0.655 & 0.619 & 0.623 & 0.325 & 0.314 & 0.295\\
0.932 & 0.912 & 0.860 & 0.737 & 0.693 & 0.653 & 0.427 & 0.384 & 0.353\\\hline
\multicolumn{9}{c}{{\textit{Panel D: Leverage, Self-Excitement in DGP, Jumps
Generated Using Normal Dist.}}}\\\hline
0.913 & 0.897 & 0.881 & 0.205 & 0.204 & 0.224 & 0.158 & 0.154 & 0.132\\
0.967 & 0.955 & 0.949 & 0.372 & 0.373 & 0.367 & 0.175 & 0.205 & 0.185\\
0.962 & 0.938 & 0.899 & 0.543 & 0.539 & 0.514 & 0.309 & 0.292 & 0.241\\
0.951 & 0.927 & 0.905 & 0.665 & 0.627 & 0.623 & 0.346 & 0.351 & 0.293\\
0.948 & 0.923 & 0.864 & 0.746 & 0.717 & 0.662 & 0.453 & 0.405 &
0.336\\\hline\hline
\end{tabular}
\end{center}
\par
* See notes to Table 2.\end{table}


\end{document}