%2multibyte Version: 5.50.0.2960 CodePage: 936


\documentclass[12pt]{article}
%%%%%%%%%%%%%%%%%%%%%%%%%%%%%%%%%%%%%%%%%%%%%%%%%%%%%%%%%%%%%%%%%%%%%%%%%%%%%%%%%%%%%%%%%%%%%%%%%%%%%%%%%%%%%%%%%%%%%%%%%%%%%%%%%%%%%%%%%%%%%%%%%%%%%%%%%%%%%%%%%%%%%%%%%%%%%%%%%%%%%%%%%%%%%%%%%%%%%%%%%%%%%%%%%%%%%%%%%%%%%%%%%%%%%%%%%%%%%%%%%%%%%%%%%%%%
\usepackage{amssymb}
\usepackage{amsfonts}
\usepackage{amsmath}
\usepackage{geometry}
\usepackage[onehalfspacing]{setspace}
\usepackage{setspace}
\usepackage{graphics}
\usepackage{setspace}

\setcounter{MaxMatrixCols}{10}
%TCIDATA{OutputFilter=LATEX.DLL}
%TCIDATA{Version=5.00.0.2606}
%TCIDATA{Codepage=936}
%TCIDATA{<META NAME="SaveForMode" CONTENT="1">}
%TCIDATA{BibliographyScheme=Manual}
%TCIDATA{Created=Sunday, July 18, 2004 16:10:34}
%TCIDATA{LastRevised=Friday, July 28, 2023 19:28:21}
%TCIDATA{<META NAME="GraphicsSave" CONTENT="32">}
%TCIDATA{<META NAME="DocumentShell" CONTENT="General\Blank Document">}
%TCIDATA{Language=American English}
%TCIDATA{CSTFile=article.cst}
%TCIDATA{PageSetup=72,72,72,72,0}
%TCIDATA{AllPages=
%F=36,\PARA{038<p type="texpara" tag="Body Text" >\hfill \thepage}
%}


\newtheorem{theorem}{Theorem}
\newtheorem{acknowledgement}[theorem]{Acknowledgement}
\newtheorem{algorithm}[theorem]{Algorithm}
\newtheorem{axiom}[theorem]{Axiom}
\newtheorem{case}[theorem]{Case}
\newtheorem{claim}[theorem]{Claim}
\newtheorem{conclusion}[theorem]{Conclusion}
\newtheorem{condition}[theorem]{Condition}
\newtheorem{conjecture}[theorem]{Conjecture}
\newtheorem{corollary}[theorem]{Corollary}
\newtheorem{criterion}[theorem]{Criterion}
\newtheorem{definition}[theorem]{Definition}
\newtheorem{example}[theorem]{Example}
\newtheorem{exercise}[theorem]{Exercise}
\newtheorem{lemma}[theorem]{Lemma}
\newtheorem{notation}[theorem]{Notation}
\newtheorem{problem}[theorem]{Problem}
\newtheorem{proposition}[theorem]{Proposition}
\newtheorem{remark}[theorem]{Remark}
\newtheorem{solution}[theorem]{Solution}
\newtheorem{summary}[theorem]{Summary}
\newenvironment{proof}[1][Proof]{\noindent\textbf{#1.} }{\ \rule{0.5em}{0.5em}}
\renewcommand{\baselinestretch}{1.0} 
\textwidth=7.0in
\textheight=9.2in
\oddsidemargin=0in
\evensidemargin=0in
\topmargin=0in
\baselineskip=10pt
\linespread{1.75}
\input{tcilatex}
\geometry{left=0.9in,right=0.9in,top=0.9in,bottom=0.9in}

\begin{document}


\begin{center}
\vspace{0.22in}{\Large {\ }}

{\Large {Selecting the Relevant Variables for Factor Estimation in FAVAR
Models, With An Application to Forecasting the Yield Curve$^{\ast }$}}

\bigskip \vspace{0.5in}

July 28, 2023

\bigskip \bigskip

Abstract
\end{center}

\begin{spacing}{1.01}
\noindent When specifying and estimating latent factor models, factor pervasiveness is 
often assumed, which requires that 
$\Gamma^{\prime }\Gamma /N$\ converges to a positive definite matrix, as 
$N\rightarrow \infty $, where $\Gamma $\ denotes the  factor model loading matrix. 
We show that consistent factor estimation can be feasible, even under factor nonpervasiveness, if one prescreens
the available variables. For this purpose, we introduce a variable selection procedure that, with probability approaching 
one, correctly distinguishes between relevant and irrelevant variables. This in turn enables consistent estimation of 
conditional mean functions of factor-augmented forecast equations, even in certain situations where the factor pervasiveness assumption is violated. 
Our procedure is designed to assess whether the usual strong factor assumption holds or not; 
and so does not rule out the possibility that the strong factor assumption actually holds for a given dataset.
In addition, even if a particular dataset is such that most of the variables are actually relevant, 
so that inconsistent factor estimation does not occur, it may still be beneficial to use our procedure, since empirical 
researchers can prune out irrelevant variables and improve the finite sample performance of their factor estimator. 
Monte Carlo and empirical experiments are presented that indicate the empirical relevance of our procedure.

\end{spacing}

\bigskip \bigskip \bigskip

\noindent \textit{Keywords: }Factor analysis, forecasting, variable
selection.

\noindent

\bigskip \bigskip

\newpage

\noindent \noindent \setcounter{page}{2}

\section{Introduction}

\noindent \qquad As a result of the astounding rate at which raw information
is currently being accumulated, there is a clear need for variable
selection, dimension reduction and shrinkage techniques when analyzing big
data using machine learning methodologies. This has led to a profusion of
novel research in areas ranging from the analysis of high dimensional and/or
high frequency datasets to the development of new statistical learning
methods. Needless to say, there are many critical unanswered questions in
this burgeoning literature. One such question, which we address in this
paper stems from the work due to Bai and Ng (2002), Stock and Watson
(2002a,b), Forni, Hallin, Lippi, and Reichlin (2005), and Bai and Ng (2008).
In these papers, the authors develop methods for constructing forecasts
based on factor-augmented regression models. An obvious appeal of using
factor analytical methods for this problem is the capacity for dimension
reduction, so that in terms of the specification of the forecasting
equation, employment of a factor structure allows the parsimonious
representation of information embedded in a possibly high-dimensional vector
of predictor variables.

Within this context, we note that a key assumption commonly used in the
literature to obtain consistent factor estimation is the so-called factor
pervasiveness assumption, which requires that $\Gamma ^{\prime }\Gamma /N$\
converges to a positive definite matrix, as $N\rightarrow \infty $, where $%
\Gamma $\ denotes the loading matrix of the factor model. Since this
assumption imposes certain conditions on how the variables in a given
dataset load on the underlying latent factors, it is of interest to have
statistical tools which allow researchers to check the empirical content of
this assumption for the particular datasets they are using. Along these
lines, our paper explores situations where the pervasiveness assumption may
not hold because one is working with a dataset where some of the variables
are irrelevant, in the sense that they do not load on the underlying latent
factors. If a sufficient number of such irrelevant variables exist,
inconsistency in factor estimation may result if one naively includes all
available variables when estimating the underlying factors, without regard
to whether they are relevant or not. See Chao, Liu, and Swanson (2023), for
a particularly pathological example where an estimated factor, $\widehat{f}%
_{t},$\ approaches $0$\ in probability, regardless of what the true value of 
$f_{t}$\ happens to be - a situation which can arise when the underlying
factors are nonpervasive.\footnote{%
The example given in Chao, Liu, and Swanson (2023) is related to results
that have been obtained in the statistics literature showing the
inconsistency of sample eigenvectors as estimators of population
eigenvectors in certain high dimensional situations (see, e.g. Paul (2007)
and Johnstone and Paul (2018).} Not being able to obtain consistent
estimates of the underlying factors will clearly cause problems for
empirical researchers, such as when the objective is to estimate forecast
functions that incorporate estimated factors. On the other hand, if one
pre-screens the variables and successfully prunes out the irrelevant ones,
then consistent estimation can be achieved, under appropriate conditions.
For this reason, a main contribution of this paper is to introduce a novel
variable selection procedure which allows empirical researchers to correctly
distinguish the relevant from the irrelevant variables prior to factor
estimation, with probability approaching one. We study this problem within a
factor-augmented VAR (FAVAR) framework - a setup which has the advantage
that it allows time series forecasts to be made using information sets much
richer than those used in traditional VAR models. While the present paper
focuses on the development of a variable selection procedure and the
analysis of its asymptotic properties; we show in Chao, Liu, and Swanson
(2023) that the use of our methodology will allow the conditional mean
function of a factor-augmented forecast equation to be consistently
estimated in a wide range of situations, including cases where violation of
factor pervasiveness is such that consistent estimation is precluded in the
absence of variable pre-screening. Overall, the results detailed in this
paper can be viewed as adding to a nascent literature which considers the
problem of factor estimation under various relaxations of the conventional
factor pervasiveness assumption (see, for example, the interesting papers by
Giglio, Xiu, and Zhang (2021), Freyaldenhoven (2021a,b), and Bai and Ng
(2021)).

The variable selection procedure reported here is related to the well-known
supervised principal components method proposed by Bair, Hastie, Paul, and
Tibshirani (2006). Additionally, our procedure is related to recent work by
Giglio, Xiu, and Zhang (2021), who propose a method for selecting test
assets, with the objective of estimating risk premia in a Fama-MacBeth type
framework. A crucial difference between the variable selection method
proposed in our paper and those proposed in these papers is that we use a
score statistic that is self-nomalized, whereas the aforementioned papers do
not make use of statistics that involve self-normalization. An important
advantage of self-normalized statistics is their ability to accommodate a
much wider range of possible tail behavior in the underlying distributions,
relative to their non-self-normalized counterparts. This makes
self-normalized statistics better suited for some economic and financial
applications, where the distribution of the data is known to exhibit certain
thick-tailed behavior. In addition, the type of models studied in Bair,
Hastie, Paul, and Tibshirani (2006) and Giglio, Xiu, and Zhang (2021) differ
significantly from the FAVAR model studied here. In particular, Bair,
Hastie, Paul, and Tibshirani (2006) study a one-factor model in an $i.i.d.$
Gaussian framework, thus, precluding complications associated with the
introduction of dependence and non-normality. Giglio, Xiu, and Zhang (2021),
on the other hand, make certain high-level assumptions which can accommodate
some dependence both cross-sectionally and intertemporally, but the model
that they consider is very different from the dynamic vector time series
model studied in the sequel.\footnote{%
Another interesting recent paper on factor estimation is Ahn and Bae (2022).
This paper uses partial least squares instead of principal component methods
to estimate a factor-based forecasting equation, and thus utilizes an
approach that differs from the one taken in this paper. In addition, Ahn and
Bae (2022) assume factor pervasiveness so that issues of variable selection,
which are the main focus of this paper, do not arise in their paper.}For all
of the above reasons, the research reported in the sequel is meant to add to
the suite of tools available to empirical researchers for variable selection
in high dimensional data analysis.

It is also worth pointing out that our variable selection procedure differs
substantially from the approach to variable/model selection taken in much of
the traditional econometrics literature. In particular, we show that
important moderate deviation results obtained recently by Chen, Shao, Wu,
and Xu (2016) can be used to help control the probability of a Type I error,
i.e., the error that an irrelevant variable which is not informative about
the underlying factors is falsely selected as a relevant variable. This is
so even in situations where the number of irrelevant variables is very large
and even if the underlying distribution does not satisfy the kind of
sub-Gaussian tail behavior typically assumed in high-dimensional statistical
analysis. Hence, we are able to design a variable selection procedure where
the probability of a Type I error goes to zero, as the sample sizes grow to
infinity. This fact, taken together with the fact that the probability of a
Type II error for our procedure also goes to zero asymptotically, allows us
to establish that our variable selection procedure is completely consistent,
in the sense that the probabilities of both Type I and Type II errors go to
zero in the limit. This property of complete consistency is important
because if we try simply to control the probability of a Type I error at
some predetermined non-zero level, which is the typical approach in multiple
hypothesis testing, then we will not in general be able to estimate the
factors consistently, even up to an invertible matrix transformation, and in
consequence, we will have fallen short of our ultimate goal of obtaining a
consistent estimate of the conditional mean function of the factor-augmented
forecasting equation.

In order to assess the practical usefulness of our method, we carry out a
series of Monte Carlo experiments as well as an empirical application. In
our Monte Carlo experiments, we show that the probability of a false
positive and the probability of a false negative, when applying our methods
using a number\ of data generating processes and admissible tuning parameter
values, both approach zero, as expected, even for relatively small values of 
$T$ and $N$. In our empirical application, we forecast the U.S. yield curve
using a large macroeconomic dataset, with models constructed using our
method, various other PCA, least absolute shrinkage operator (LASSO) and
elastic net (EN), methods, a strawman autoregressive (AR), and the so-called
dynamic Nelson-Siegel (DNS) models that is based on rational expectations.
Interestingly, for 1-month ahead predictions of interest rates at 3-month
and 1-year maturities, our method yields statistically superior forecasts
relative to all other methods analyzed, and is approximately
\textquotedblleft tied\textquotedblright\ with PCA for longer maturities.
Additionally, for 3-month ahead predictions, our method yields models with
the lowest or second lowest mean-square forecast errors of any model, at all
maturities. Finally, for our longest horizon forecasts of 1-year, the
theoretically derived DNS model that includes no big data elements (i.e.,
only utilizes interest rates) yields superior forecasts. This suggests that
markets may be relatively inefficient in the short-run, but are more
efficient in the longer run.

The rest of the paper is organized as follows. In Section 2, we discuss the
FAVAR model and the assumptions that we impose on this model. We also
describe our variable selection procedure and provide theoretical results
establishing the complete consistency of this procedure. Section 3 presents
the results of a promising Monte Carlo study on the finite sample
performance of our variable selection method. Section 4 presents the
findings of our empirical application, and Section 5 offers some concluding
remarks. Due to space considerations, proofs of the main theorems and all
supporting lemmas as well as some additional technical details are provided
in a not-for-publication online supplement.

Before proceeding, we first say a few words about some of the frequently
used notation in this paper. Throughout, let $\lambda _{\left( j\right)
}\left( A\right) $, $\lambda _{\max }\left( A\right) $, and $\lambda _{\min
}\left( A\right) $ denote, respectively, the $j^{th}$ largest eigenvalue,
the maximal eigenvalue, and the minimal eigenvalue of a square matrix $A$.
Similarly, let $\sigma _{\left( j\right) }\left( B\right) $, $\sigma _{\max
}\left( B\right) $, and $\sigma _{\min }\left( B\right) $ denote,
respectively, the $j^{th}$ largest singular value, the maximal singular
value, and the minimal singular value of a matrix $B$, which is not
restricted to be a square matrix. In addition, let $\left\Vert a\right\Vert
_{2}$ denote the usual Euclidean norm when applied to a (finite-dimensional)
vector $a$. Also, for a matrix $A$, $\left\Vert A\right\Vert _{2}\equiv \max
\left\{ \sqrt{\lambda \left( A^{\prime }A\right) }:\lambda \left( A^{\prime
}A\right) \text{ is an eigenvalue of }A^{\prime }A\right\} $ denotes the
matrix spectral norm. For two sequences, $\left\{ x_{T}\right\} $ and $%
\left\{ y_{T}\right\} $, write $x_{T}\sim y_{T}$ if $x_{T}/y_{T}=O\left(
1\right) $ and $y_{T}/x_{T}=O\left( 1\right) $, as $T\rightarrow \infty $.
Furthermore, let $\left\vert z\right\vert $ denote the absolute value or the
modulus of the number $z$; let $\left\lfloor \cdot \right\rfloor $ denote
the floor function, so that $\left\lfloor x\right\rfloor $ gives the integer
part of the real number $x$, and let $\iota _{p}=\left( 1,1,...,1\right)
^{\prime }$ denote a $p\times 1$ vector of ones. Finally, for a sequence of
random variables $u_{i,t+m},u_{i,t+m+1},u_{i,t+m+2},....$; we let $\sigma
\left( u_{i,t+m},u_{i,t+m+1},u_{i,t+m+2},....\right) $ denote the $\sigma $%
-field generated by this sequence of random variables.

\section{\noindent Model, Assumptions, and Variable Selection}

\noindent \qquad Consider the following $p^{th}$-order factor-augmented
vector autoregression (FAVAR):%
\begin{equation}
W_{t+1}=\mu +A_{1}W_{t}+\cdot \cdot \cdot +A_{p}W_{t-p+1}+\varepsilon _{t+1}%
\text{ where,}  \label{FAVAR}
\end{equation}%
\begin{eqnarray*}
\underset{\left( d+K\right) \times 1}{W_{t+1}} &=&\left( 
\begin{array}{c}
\underset{d\times 1}{Y_{t+1}} \\ 
\underset{K\times 1}{F_{t+1}}%
\end{array}%
\right) \text{, }\underset{\left( d+K\right) \times 1}{\varepsilon _{t+1}}%
=\left( 
\begin{array}{c}
\underset{d\times 1}{\varepsilon _{t+1}^{Y}} \\ 
\underset{K\times 1}{\varepsilon _{t+1}^{F}}%
\end{array}%
\right) \text{, }\underset{\left( d+K\right) \times 1}{\mu }=\left( 
\begin{array}{c}
\underset{d\times 1}{\mu _{Y}} \\ 
\underset{K\times 1}{\mu _{F}}%
\end{array}%
\right) ,\text{ and} \\
\text{ }\underset{\left( d+K\right) \times \left( d+K\right) }{A_{g}}
&=&\left( 
\begin{array}{cc}
\underset{d\times d}{A_{YY,g}} & \underset{d\times K}{A_{YF,g}} \\ 
\underset{K\times d}{A_{FY,g}} & \underset{K\times K}{A_{FF,g}}%
\end{array}%
\right) ,\text{ for }g=1,...,p.
\end{eqnarray*}%
Here, $Y_{t}$ denotes the vector of observable economic variables, and $%
F_{t} $ is a vector of unobserved (latent) factors. In our analysis of this
model, it will often be convenient to rewrite the FAVAR in several
alternative forms, which will facilitate writing down assumptions and
conditions used in the sequel. We thus briefly outline two alternative
representations of the above model. It is easy to see that the system of
equations given in (\ref{FAVAR}) can be written as:%
\begin{eqnarray}
Y_{t+1} &=&\mu _{Y}+A_{YY}\underline{Y}_{t}+A_{YF}\underline{F}%
_{t}+\varepsilon _{t+1}^{Y},  \label{Y component FAVAR} \\
F_{t+1} &=&\mu _{F}+A_{FY}\underline{Y}_{t}+A_{FF}\underline{F}%
_{t}+\varepsilon _{t+1}^{F},  \label{F component FAVAR}
\end{eqnarray}%
where $\underset{d\times dp}{A_{YY}}=\left( 
\begin{array}{cccc}
A_{YY,1} & A_{YY,2} & \cdots & A_{YY,p}%
\end{array}%
\right) $, $\underset{d\times Kp}{A_{YF}}=\left( 
\begin{array}{cccc}
A_{YF,1} & A_{YF,2} & \cdots & A_{YF,p}%
\end{array}%
\right) $,

\noindent $\underset{K\times dp}{A_{FY}}=\left( 
\begin{array}{cccc}
A_{FY,1} & A_{FY,2} & \cdots & A_{FY,p}%
\end{array}%
\right) $, $\underset{K\times Kp}{A_{FF}}=\left( 
\begin{array}{cccc}
A_{FF,1} & A_{FF,2} & \cdots & A_{FF,p}%
\end{array}%
\right) $,

\noindent $\underset{dp\times 1}{\underline{Y}_{t}}=\left( 
\begin{array}{cccc}
Y_{t}^{\prime } & Y_{t-1}^{\prime } & \cdots & Y_{t-p{\LARGE +}1}^{\prime }%
\end{array}%
\right) ^{\prime }$, and $\underset{Kp\times 1}{\underline{F}_{t}}=\left( 
\begin{array}{cccc}
F_{t}^{\prime } & F_{t-1}^{\prime } & \cdots & F_{t-p{\LARGE +}1}^{\prime }%
\end{array}%
\right) ^{\prime }$. Another useful representation of the FAVAR model is the
so-called companion form, wherein the $p^{th}$-order model given in
expression (\ref{FAVAR}) is written in terms of a first-order model:

$\underset{\left( d+K\right) p\times 1}{\underline{W}_{t}}=\alpha +A%
\underline{W}_{t-1}+E_{t}\text{,}$ where $\underline{W}_{t}=(W_{t}^{\prime
},W_{t-1}^{\prime },\cdots ,W_{t-p{\LARGE +}2}^{\prime },W_{t-p{\LARGE +}%
1}^{\prime })^{\prime }$, and where

\begin{equation}
\alpha =\left( 
\begin{array}{c}
\mu \\ 
0 \\ 
\vdots \\ 
0 \\ 
0%
\end{array}%
\right) \text{, }A=\left( 
\begin{array}{ccccc}
A_{1} & A_{2} & \cdots & A_{p-1} & A_{p} \\ 
I_{d+K} & 0 & \cdots & 0 & 0 \\ 
0 & I_{d+K} & \ddots & \vdots & 0 \\ 
\vdots & \ddots & \ddots & 0 & \vdots \\ 
0 & \cdots & 0 & I_{d+K} & 0%
\end{array}%
\right) \text{, and }E_{t}=\left( 
\begin{array}{c}
\varepsilon _{t} \\ 
0 \\ 
\vdots \\ 
0 \\ 
0%
\end{array}%
\right) \text{.}  \label{companion form notations}
\end{equation}

In addition to observations on $Y_{t}$, suppose that the data set available
to researchers includes a vector of time series variables which are related
to the unobserved factors in the following manner:%
\begin{equation}
Z_{t}=\text{ }\Gamma \underline{F}_{t}+u_{t}\text{,}
\label{overspecified factor model}
\end{equation}%
where $\underset{N\times 1}{Z_{t}}=\left( Z_{1t},Z_{2t},...,Z_{Nt}\right)
^{\prime }$. Assume, however, that not all components of $Z_{t}$ provide
useful information for estimating the unobserved vector $\underline{F}_{t}$,
so that the $N\times Kp$ parameter matrix $\Gamma $ may have some rows whose
elements are all zero. More precisely, let the $1\times Kp$ vector $\gamma
_{i}^{\prime }$ denote the $i^{th}$ row of $\Gamma $, and assume that the
rows of the matrix $\Gamma $ can be divided into two classes: 
\begin{equation}
H=\left\{ k\in \left\{ 1,....,N\right\} :\gamma _{k}=0\right\} \text{ and }%
H^{c}=\left\{ k\in \left\{ 1,....,N\right\} :\gamma _{k}\neq 0\right\} .
\label{H and Hc}
\end{equation}%
Now, let $\mathcal{P}$ be a permutation matrix which reorders the components
of $Z_{t}$ such that $\mathcal{P}Z_{t}=\left( 
\begin{array}{cc}
Z_{t}^{\left( 1\right) \prime } & Z_{t}^{\left( 2\right) \prime }%
\end{array}%
\right) ^{\prime }$, where%
\begin{eqnarray}
\underset{N_{1}\times 1}{Z_{t}^{\left( 1\right) }} &=&\Gamma _{1}\underline{F%
}_{t}+u_{t}^{\left( 1\right) }  \label{Z(1)} \\
\underset{N_{2}\times 1}{Z_{t}^{\left( 2\right) }} &=&u_{t}^{\left( 2\right)
}\text{.}  \label{Z(2)}
\end{eqnarray}%
The above representation suggests that the components of $Z_{t}^{\left(
1\right) }$ can be interpreted as the relevant variables for the purpose of
factor estimation, as the information that they supply will be helpful in
estimating $\underline{F}_{t}$. On the other hand, the components of the
subvector $Z_{t}^{\left( 2\right) }$ are irrelevant variables (or pure
\textquotedblleft noise\textquotedblright\ variables), as they do not load
on the underlying factors and only add noise if they are included in the
factor estimation process. Given that an empirical researcher will typically
not have prior knowledge as to which variables are elements of $%
Z_{t}^{\left( 1\right) }$ and which are elements of $Z_{t}^{\left( 2\right)
} $, it will be nice to have a variable selection procedure which will allow
us to properly identify the components of $Z_{t}^{\left( 1\right) }$ and to
use only these variables when we try to estimate $\underline{F}_{t}$. On the
other hand, if we unknowingly include too many components of $Z_{t}^{\left(
2\right) }$ in the estimation process, then inconsistent factor estimation
can arise. This is demonstrated in an example analyzed recently in Chao,
Liu, and Swanson (2023) which considers a setting similar to the
specification given in expressions (\ref{overspecified factor model})-(\ref%
{Z(2)}) above, but for the case of a simple one-factor model. More
precisely, an example is given in which, without variable pre-screening, the
usual principal-component-based factor estimator $\widehat{f}_{t}\overset{p}{%
\rightarrow }$ $0$ regardless of the true value $f_{t}$ under the additional
rate condition that $N/\left( TN_{1}^{\left( 1+\kappa \right) }\right)
=c+o\left( N_{1}^{-1}\right) $, where $c$ and $\kappa $ are constants such
that $0<c<\infty $ and $0<\kappa <1$ and where $N_{1}$ is the number of
relevant variables, $N_{2}$ is the number of irrelevant variables, and $%
N=N_{1}+N_{2}$. This example shows the kind of severe inconsistency in
factor estimation that could result if the commonly assumed condition of
factor pervasiveness (which essentially requires that $N_{1}\sim N$) does
not hold.\footnote{%
The reason why we refer to the result given in Chao, Liu, and Swanson (2023)
as a severe form of inconsistency in factor estimation is because
inconsistency of this type will preclude the consistent estimation of the
conditional mean function of a factor-augmented forecast equation. This is
different from the case where the factors may be estimated consistently up
to a non-zero scalar multiplication or, more generally, up to an invertible
matrix transformation. In the latter case, consistent estimation of the
conditional mean function of a factor-augmented forecast equation can still
be attained.}

It should be noted that, in an important recent paper, Bai and Ng (2021)
provide results which show that factors can still be estimated consistently
in certain situations where factor loadings are weaker than implied by the
conventional pervasiveness assumption; although, as might be expected, in
such cases the rate of convergence of the factor estimator is slower and
additional assumptions are needed. To understand the relationship between
their results and our setup, note that a key condition for the consistency
result given in their paper, when expressed in terms of our setup, is the
assumption that $N/\left( TN_{1}\right) \rightarrow 0$. When violation of
the factor pervasiveness condition is more severe than that characterized by
this rate condition (i.e., if $N/\left( TN_{1}\right) \rightarrow c_{1},$
for some positive constant $c_{1}$ or if $N/\left( TN_{1}\right) \rightarrow
\infty )$, then factors will be estimated inconsistently unless there is
some method which can correctly identify the relevant variables, and only
these variables are used to estimate the factors. Indeed, in Thoerem 4.1 of
Chao, Liu, and Swanson (2023), we add to the results given in Bai and Ng
(2021) by giving a result which shows that if one pre-screens variables
using the variable selection method proposed below, then consistent factor
estimation can be achieved, even if the rate condition that $N/\left(
TN_{1}\right) \rightarrow 0$ is not satisfied. In general, knowledge about
the severity with which the conventional factor pervasiveness assumption may
be violated must ultimately be gathered on a case-by-case basis, and depends
on the dataset used for a particular study. Along these lines, various
authors have already documented cases where the empirical evidence shows
that the underlying factors are quite weak, suggesting that there may be
rather severe violation of the assumption of factor pervasiveness. For
example, see Jagannathan and Wang (1998), Harding (2008), Kleibergen (2009),
Onatski (2012), Bryzgalova (2016), Burnside (2016), Gospodinov, Kan, and
Robotti (2017), Anatolyev and Mikusheva (2021), and Freyaldenhoven
(2021a,b). In such cases, it is of interest to explore the possibility that
weakness in loadings is not uniform across all variables, but rather is due
to the fact that only a fraction of the $Z_{it}$ variables loads
significantly on the underlying factors. Furthermore, even if the empirical
situation of interest is one where, strictly speaking, the condition $%
N/\left( TN_{1}\right) \rightarrow 0$ does hold, it may still be beneficial
in some such instances to do variable pre-screening. This is particularly
true in situations where the condition $N/\left( TN_{1}\right) \rightarrow 0$
is \textquotedblleft barely\textquotedblright\ satisfied, in which case one
would expect to pay a rather hefty finite sample price for not pruning out
variables that do not load significantly on the underlying factors, since
these variables may add unwanted noise to the estimation process. For these
reasons, we believe that there is a need to develop methods which will
enable empirical researchers to pre-screen the components of $Z_{t},$ so
that variables which are informative and helpful to the estimation process
can be properly identified. In summary, our paper aims to build on the
results developed by Bai and Ng (2021) and others by introducing additional
tools for situations where factor estimator properties may be impacted by
failure of the conventional pervasiveness assumption.

To provide a variable selection procedure with provable guarantees, we must
first specify a number of conditions on the FAVAR model defined above.

\noindent \textbf{Assumption 2-1: }Suppose that: $\det \left\{ I_{\left(
d+K\right) }-A_{1}z-\cdot \cdot \cdot-A_{p}z^{p}\right\} =0,\text{ implies
that }\left\vert z\right\vert >1$. \noindent \textbf{Assumption 2-2: }Let $%
\varepsilon _{t}$ satisfy the following set of conditions: (a) $\left\{
\varepsilon _{t}\right\} $ is an independent sequence of random vectors with 
$E\left[ \varepsilon _{t}\right] =0$ $\forall t$; (b) there exists a
positive constant $C$ such that $\sup_{t}E\left\Vert \varepsilon
_{t}\right\Vert _{2}^{6}\leq C<\infty $; and (c) $\varepsilon _{t}$ admits a
density $g_{\varepsilon _{t}}$ such that, for some positive constant $%
M<\infty $, $\sup_{t}\dint \left\vert g_{\varepsilon _{t}}\left( \upsilon
-u\right) -g_{\varepsilon _{t}}\left( \upsilon \right) \right\vert d\upsilon
\leq M\left\Vert u\right\Vert $, whenever $\left\Vert u\right\Vert \leq 
\overline{\kappa }$ for some constant $\overline{\kappa }>0$.

\noindent \textbf{Assumption 2-3: }Let $u_{i,t}$ be the $i^{th}$ element of
the error vector $u_{t}$ in expression (\ref{overspecified factor model}),
and we assume that it satisfies the following conditions: (a) $E\left[
u_{i,t}\right] =0$ for all $i$ and $t$; (b) there exists a positive constant 
$\overline{C}$ such that $\sup_{i,t}E\left\vert u_{i,t}\right\vert ^{7}\leq 
\overline{C}<\infty $, and there exists a constant $\underline{C}>0$ such
that $\inf_{i,t}E\left[ u_{i,t}^{2}\right] \geq \underline{C}$; and (c)
define $\mathcal{F}_{i,-\infty }^{t}=\sigma \left(
....,u_{i,t-2},u_{i,t-1},u_{t}\right) $, $\mathcal{F}_{i,t+m}^{\infty
}=\sigma \left( u_{i,t+m},u_{i,t+m+1},u_{i,t+m+2},....\right) $, and $\beta
_{i}\left( m\right) =\sup_{t}E\left[ \sup \left\{ \left\vert P\left( B|%
\mathcal{F}_{i,-\infty }^{t}\right) -P\left( B\right) \right\vert :B\in 
\mathcal{F}_{i,t+m}^{\infty }\right\} \right] $. Assume $\exists $ constants 
$a_{1}>0$ and $a_{2}>0$ such that $\beta _{i}\left( m\right) \leq a_{1}\exp
\left\{ -a_{2}m\right\} ,$ for all $i$.

\noindent \textbf{Assumption 2-4: }$\varepsilon _{t}$ and $u_{i,s}$ are
independent, for all $i,t,$ and $s$.

\noindent \textbf{Assumption 2-5: }There exists a positive constant $%
\overline{C},$ such that $\sup_{i\in H^{{\large c}}}\left\Vert \gamma
_{i}\right\Vert _{2}\leq \overline{C}<\infty $ and $\left\Vert \mu
\right\Vert _{2}\leq \overline{C}<\infty $, where $\mu =\left( \mu
_{Y}^{\prime },\mu _{F}^{\prime }\right) ^{\prime }$.

\noindent \textbf{Assumption 2-6: }Let $A$ be as defined in expression (\ref%
{companion form notations}) above, and let the modulus of the eigenvalues of
the matrix $I_{\left( d+K\right) p}-A$ be sorted so that: $\left\vert
\lambda ^{\left( 1\right) }\left( I_{\left( d+K\right) p}-A\right)
\right\vert \geq \left\vert \lambda ^{\left( 2\right) }\left( I_{\left(
d+K\right) p}-A\right) \right\vert \geq \cdot \cdot \cdot \geq \left\vert
\lambda ^{\left( \left( d+K\right) p\right) }\left( I_{\left( d+K\right)
p}-A\right) \right\vert =\overline{\phi }_{\min }$. Suppose that there is a
constant $\underline{C}>0$ such that:%
\begin{equation}
\sigma _{\min }\left( I_{\left( d+K\right) p}-A\right) \geq \underline{C}%
\overline{\phi }_{\min }  \label{lower bd I-A}
\end{equation}%
In addition, there exists a positive constant $\overline{C}<\infty $ such
that, for all positive integer $j$, 
\begin{equation}
\sigma _{\max }\left( A^{j}\right) \leq \overline{C}\max \left\{ \left\vert
\lambda _{\max }\left( A^{j}\right) \right\vert ,\left\vert \lambda _{\min
}\left( A^{j}\right) \right\vert \right\} .  \label{upper bd A}
\end{equation}

\noindent \textbf{Remark 2.1:} \textbf{(a)} Note that Assumption 2-1 is the
stability condition that one typically assumes for a stationary VAR process.
One difference is that we allow for possible heterogeneity in the
distribution of $\varepsilon _{t}$ across time, so that our FAVAR process is
not necessarily a strictly stationary process. Under Assumption 2-1, there
exists a vector moving average representation for the FAVAR process. \textbf{%
(b) }It is well known that $\det \left\{ I_{\left( d+K\right) }-Az\right\}
=\det \left\{ I_{\left( d+K\right) }-A_{1}z-\cdot \cdot \cdot
-A_{p}z^{p}\right\} ,$ where $A$ is the coefficient matrix of the companion
form given in expression (\ref{companion form notations}). It follows that
Assumption 2-1 is equivalent to the condition that $\det \left\{ I_{\left(
d+K\right) }-Az\right\} =0$ implies that $\left\vert z\right\vert >1$. In
addition, Assumption 2-1 is also, of course, equivalent to the assumption
that all eigenvalues of $A$ have modulus less than $1$. \textbf{(c)}
Assumption 2-6 imposes a condition whereby the extreme singular values of
the matrices $A^{j}$ and $I_{\left( d+K\right) p}-A$ have bounds that depend
on the extreme eigenvalues of these matrices. More primitive conditions for
such a relationship between the singular values and the eigenvalues of a
(not necessarily symmetric) matrix have been studied in the linear algebra
literature. In fact, it is easy to show that Assumption 2-6 holds
automatically if the matrix $A$ is diagonalizable, even if it is not
symmetric. Assumptions 2-6, on the other hand, takes into account other
situations where expressions (\ref{lower bd I-A}) and (\ref{upper bd A}) are
valid even though the matrix $A$ is not diagonalizable. \textbf{(d)} Note
that Assumptions 2-1, 2-2, and 2-6 together imply that the process $\left\{
W_{t}\right\} $ generated by the FAVAR model given in expression (\ref{FAVAR}%
) is a $\beta $-mixing process with $\beta $-mixing coefficient satisfying $%
\beta _{W}\left( m\right) \leq a_{1}\exp \left\{ -a_{2}m\right\} $, for some
positive constants $a_{1}$ and $a_{2}$, with $\beta _{W}\left( m\right)
=\sup_{t}E\left[ \sup \left\{ \left\vert P\left( B|\mathcal{A}_{-\infty
}^{t}\right) -P(B)\right\vert :B\in \mathcal{A}_{t+m}^{\infty }\right\} %
\right] $, and with

\noindent $\mathcal{A}_{-\infty }^{t}=\sigma \left(
...,W_{t-2},W_{t-1},W_{t}\right) $ and $\mathcal{A}_{t+m}^{\infty }=\sigma
\left( W_{t+m},W_{t+m+1},W_{t+m+2},....\right) .$\footnote{%
This can be shown by applying Theorem 2.1 of Pham and Tran (1985). This
result is given as Lemma OA-11 in Chao, Qiu, and Swanson (2023).} Note, in
addition, that Assumption 2-2 (c) rules out situations such as that given in
the famous counterexample presented by Andrews (1984) which shows that a
first-order autoregression with errors having a discrete Bernoulli
distribution is not $\alpha $-mixing, even if it satisfies the stability
condition. Conditions similar to Assumption 2-2(c) have also appeared in
previous papers, such as Gorodetskii (1977) and Pham and Tran (1985), which
seek to provide sufficient conditions for establishing the $\alpha $ or $%
\beta $ mixing properties of linear time series processes.

Our variable selection procedure is based on a self-normalized statistic and
makes use of some pathbreaking moderate deviation results for weakly
dependent processes recently obtained by Chen, Shao, Wu, and Xu (2016). An
advantage of using a self-normalized statistic is that doing so allows us to
impose much weaker moment conditions, even when $N$ is much larger than $T$.
In particular, as can be seen from Assumptions 2-2 and 2-3 above, we only
make moment conditions that are of a polynomial order on the errors
processes $\left\{ \varepsilon _{t}\right\} $ and $\left\{ u_{it}\right\} $.
Such conditions are substantially weaker than assumptions of Gaussianity or
of sub-exponential tail behavior, which has been made in various papers
studying high-dimensional factor models and/or high-dimensional covariance
matrices, without employing statistics that are self-normalized.\footnote{%
See, for example, Bickel and Levina (2008) and Fan, Liao, and Mincheva
(2011, 2013).}

To accommodate data dependence, we consider self-nomalized statistics that
are constructed from observations which are first split into blocks in a
manner similar to the kind of construction one would employ in implementing
a block bootstrap or in proving a central limit theorem using the blocking
technique. Two such statistics are proposed in this paper. The first of
these statistics has the form of an $\ell _{\infty }$ norm and is given by: 
\begin{equation}
\max_{1\leq \ell \leq d}\left\vert S_{i,\ell ,T}\right\vert =\max_{1\leq
\ell \leq d}\left\vert \frac{\overline{S}_{i,\ell ,T}}{\sqrt{\overline{V}%
_{i,\ell ,T}}}\right\vert ,\text{ where}  \label{max statistic}
\end{equation}
\begin{equation}
\overline{S}_{i,\ell ,T}=\dsum\limits_{r=1}^{q}\dsum\limits_{t=\left(
r-1\right) \tau +p}^{\left( r-1\right) \tau +\tau _{1}+p-1}Z_{it}y_{\ell ,t%
{\LARGE +}1}\text{ and }\overline{V}_{i,\ell ,T}=\dsum\limits_{r=1}^{q}\left[
\dsum\limits_{t=\left( r-1\right) \tau +p}^{\left( r-1\right) \tau +\tau
_{1}+p-1}Z_{it}y_{\ell ,t{\LARGE +}1}\right] ^{2}\text{.}
\label{num and denom max stat}
\end{equation}%
Here, $Z_{it}$ denotes the $i^{th}$ component of $Z_{t}$ , $y_{\ell ,t+1}$
denotes the $\ell ^{th}$ component of $Y_{t+1}$, $\tau _{1}=\left\lfloor
T_{0}^{\alpha _{{\large 1}}}\right\rfloor $, and $\tau _{2}=\left\lfloor
T_{0}^{\alpha _{{\large 2}}}\right\rfloor $, where $1>\alpha _{1}\geq \alpha
_{2}>0$, $\tau =\tau _{1}+\tau _{2}$, $q=\left\lfloor T_{0}/\tau
\right\rfloor $, and $T_{0}=T-p+1$. Note that the statistic given in
expression (\ref{max statistic}) can be interpreted as the maximum of the
(self-normalized) sample covariances between the $i^{th}$ component of $%
Z_{t} $ and the components of $Y_{t+1}$. Our second statistic has the form
of a pseudo-$L_{1}$ norm and is given by: $\dsum\limits_{\ell =1}^{d}\varpi
_{\ell }\left\vert S_{i,\ell ,T}\right\vert =\dsum\limits_{\ell
=1}^{d}\varpi _{\ell }\left\vert \frac{\overline{S}_{i,\ell ,T}}{\sqrt{%
\overline{V}_{i,\ell ,T}}}\right\vert$, where $\overline{S}_{i,\ell ,T}$ and 
$\overline{V}_{i,\ell ,T}$ are as defined in (\ref{num and denom max stat})
above and where $\left\{ \varpi _{\ell }:\ell =1,..,d\right\} $ denotes
pre-specified weights, such that $\varpi _{\ell }\geq 0,$ for every $\ell
\in \left\{ 1,...,d\right\} $ and $\dsum\nolimits_{\ell =1}^{d}\varpi _{\ell
}=1$. Both of these statistics employ a blocking scheme similar to that
proposed in Chen, Shao, Wu, and Xu (2016), where, in order to keep the
effects of dependence under control, the construction of these statistics is
based only on observations in every other block. To see this, note that if
we write out the \textquotedblleft numerator\textquotedblright\ term $%
\overline{S}_{i,\ell ,T}$ in greater detail, we have that: 
\begin{equation}
\overline{S}_{i,\ell ,T} = \dsum\limits_{t=p}^{\tau _{1}+p-1}Z_{it}y_{\ell,t%
{\LARGE +}1}+\dsum\limits_{t=\tau +p}^{\tau +\tau _{1}+p-1}Z_{it}y_{\ell ,t%
{\LARGE +}1}+\dsum\limits_{t=2\tau +p}^{2\tau +\tau _{1}+p-1}Z_{it}y_{\ell ,t%
{\LARGE +}1}+\cdot \cdot \cdot +\dsum\limits_{t=\left( q-1\right) \tau
+p}^{\left(q-1\right) \tau +\tau _{1}+p-1}Z_{it}y_{\ell ,t{\LARGE +}1}
\label{sum of Z times y}
\end{equation}
Comparing the first term and the second term on the right-hand side of
expression (\ref{sum of Z times y}), we see that the observations $%
Z_{it}y_{\ell ,t{\LARGE +}1}$, for $t=\tau _{1}+p,...,\tau +p-1$, have not
been included in the construction of the sum. Similar observations hold when
comparing the second and the third terms, and so on.

It should also be pointed out that although we make use of some of their
fundamental results on moderate deviation, both the model studied in our
paper and the objective of our paper are very different from that of Chen,
Shao, Wu, and Xu (2016). Whereas Chen, Shao, Wu, and Xu \textbf{(}2016%
\textbf{)} focus their analysis on problems of testing and inference for the
mean of a scalar weakly dependent time series using self-normalized
Student-type test statistics, our paper applies the self-normalization
approach to a variable selection problem in a FAVAR setting. Indeed, the
problem which we study is in some sense more akin to a model selection
problem rather than a multiple hypothesis testing problem. In order to
consistently estimate the factors (at least up to an invertible matrix
transformation), we need to develop a variable selection procedure whereby
both the probability of a false positive and the probability of a false
negative converge to zero as $N_{1}$, $N_{2}$, $T\rightarrow \infty .$%
\footnote{%
Here, a false positive refers to mis-classifying a variable, $Z_{it},$ as a
relevant variable for the purpose of factor estimation when its factor
loading $\gamma _{i}^{\prime }=0$, whereas a false negative refers to the
opposite case, where $\gamma _{i}^{\prime }\neq 0,$ but the variable $Z_{it}$
is mistakenly classified as irrelevant.} This is different from the typical
multiple hypothesis testing approach whereby one tries to control the
familywise error rate (or, alternatively, the false discovery rate), so that
it is no greater than $0.05,$ say, but does not try to ensure that this
probability goes to zero as the sample size grows.

To determine whether the $i^{th}$ component of $Z_{t}$ is a relevant
variable for the purpose of factor estimation, we propose the following
procedure. Define $i\in \widehat{H}^{c}$ to indicate that the procedure has
classified $Z_{it}$ to be a relevant variable for the purpose of factor
estimation. Similarly\textbf{,} define $i\in \widehat{H}$ to indicate that
the procedure has classified $Z_{it}$ to be an irrelevant variable. Now, let 
$\mathbb{S}_{i,T}^{+}$ denote either the statistic $\max_{1\leq \ell \leq
d}\left\vert S_{i,\ell ,T}\right\vert $ or the statistic $%
\dsum\nolimits_{\ell =1}^{d}\varpi _{\ell }\left\vert S_{i,\ell
,T}\right\vert .$\footnote{%
It should be noted that the denominator of the statistic $S_{i,\ell ,T}=%
\overline{S}_{i,\ell ,T}/\sqrt{\overline{V}_{i,\ell ,T}}$ does not
correspond to the use of an HAR standard error constructed using the fixed $%
b $ (or fixed smoothing) approach pioneered by Kiefer and Vogelsang (2002),
even in the case without any truncation. Hence, our statistic differs from
the usual Studentized statistic that is normalized by an HAR estimator. This
can be shown by straightforward calculations for the case of the Bartlett
kernel, for example. For interesting discussions of different approaches to
self-normalization in the statistics and probability literature, refer to Z.
Zhou and X. Shao (2013), X. Chen, Q-M. Shao, W.B. Wu, and L. Xu (2016), and
the references cited therein.} Our variable selection procedure is based on
the decision rule: 
\begin{equation}
i\in \left\{ 
\begin{array}{cc}
\widehat{H}^{c} & \text{ if }\mathbb{S}_{i,T}^{+}\geq \Phi ^{-1}\left( 1-%
\frac{\varphi }{2N}\right) \\ 
\widehat{H} & \text{if }\mathbb{S}_{i,T}^{+}<\Phi ^{-1}\left( 1-\frac{%
\varphi }{2N}\right)%
\end{array}%
\right. ,  \label{var selection decision rule}
\end{equation}%
where $\Phi ^{-1}\left( \cdot \right) $ denotes the quantile function or the
inverse of the cumulative distribution function of the standard normal
random variable, and where $\varphi $ is a tuning parameter which may depend
on $N$. Some conditions on $\varphi $ will be given in Assumption 2-10 below.

\noindent \textbf{Remark 2.2:} To understand why using the quantile function
of the standard normal as the threshold function for our procedure is a
natural choice, note first that, by a slight modification of the arguments
given in the proof of Lemma A2\footnote{%
The statement and proof of Lemma A2 are provided in the online supplement to
this paper.}, we can show that, as $T\rightarrow \infty $, \noindent $%
P\left( \left\vert S_{i,\ell ,T}\right\vert \geq z\right) =2\left[ 1-\Phi
\left( z\right) \right] \left( 1+o\left( 1\right) \right) $, which holds for
all $i$ and $\ell $ and for all $z$ such that: $0\leq z\leq c_{0}\min
\left\{ T^{\left( 1-{\large \alpha }_{1}\right) /6}/L\left( T\right)
,T^{\alpha _{2}/2}\right\} $, where $L\left( T\right) $ denotes a slowly
varying function such that $L\left( T\right) \rightarrow \infty $, as $%
T\rightarrow \infty $. In view of the above expression, we can interpret
moderate deviation as providing an asymptotic approximation of the
(two-sided) tail behavior of the statistic, $S_{i,\ell ,T},$ based on the
tails of the standard normal distribution. Now, suppose initially that we
wish simply to control the probability of a Type I error for testing the
null hypothesis $H_{0}:\gamma _{i}=0$ (i.e., the $i^{th}$ variable does not
load on the underlying factors) at some fixed significance level $\alpha $.
Then, the above expression suggests that a natural way to do this is to set $%
z=\Phi ^{-1}\left( 1-\alpha /2\right) $. This is because, given that the
quantile function $\Phi ^{-1}\left( \cdot \right) $ is, by definition, the
inverse function of the cdf $\Phi \left( \cdot \right) $, we have that: 
\begin{equation*}
P\left( \left\vert S_{i,\ell ,T}\right\vert \geq \Phi ^{-1}\left( 1-\alpha
/2\right) \right) =2\left[ 1-\Phi \left( \Phi ^{-1}\left( 1-\alpha /2\right)
\right) \right] \left( 1+o\left( 1\right) \right) =\alpha \left( 1+o\left(
1\right) \right) ,
\end{equation*}%
so that the probability of a Type I error is controlled at the desired level 
$\alpha $ asymptotically. Note also that an advantage of moderate deviation
theory is that it gives a characterization of the relative approximation
error, as opposed to the absolute approximation error. As a result, the
approximation given is useful and meaningful even when $\alpha $ is very
small, which is of importance to us since we are interested in situations
where we might want to let $\alpha $ go to zero, as sample size approaches
infinity.

We give the above example to provide some intuition concerning the form of
the threshold function that we have specified. The variable selection
problem that we actually consider is more complicated than what is
illustrated by this example, since we need to control the probability of a
Type I error (or of a false positive) not just for a single test involving
the $i^{th}$ variable but for all variables simultaneously. Moreover, as
noted previously, we also need the probability of a false positive to go to
zero asymptotically, if we want to be able to estimate the factors
consistently, even up to an invertible matrix transformation. We show in
Theorem 1 below that these objectives can all be accomplished using the
threshold function specified in expression (\ref{var selection decision rule}%
), since a threshold function of this form makes it easy for us to properly
control the probability of a false positive in large samples. The threshold
function used here is reminiscent of the one employed in Belloni,
Chernozhukov, and Hansen (2014). The latter paper focuses on developing a
variable screening methodology for a partially linear treatment effects
model. In that paper, a threshold function that is similar to ours is used
to set the penalty level for a lasso-based procedure for selecting the terms
in a series expansion of the nonlinear component of their model under
conditions of sparsity. In spite of the similarity in the form of the
threshold function used, the nature of the variable selection problem
studied in Belloni, Chernozhukov, and Hansen (2014) is quite different from
that investigated in this paper. In particular, these authors do not require
their variable selection procedure to be completely consistent, nor do they
provide a result showing that the probability of both Type I and Type II
error vanishes asymptotically as sample sizes approach infinity. They also
stress that perfect variable selection is not needed in the type of
regression settings considered in their paper if the goal is to approximate
the nonlinear functions in their model sufficiently well so that the
post-selection estimators of the treatment effect parameter will have good
asymptotic properties. Here, we instead argue that having a variable
selection procedure that is completely consistent is quite useful given our
objective of ensuring that good factor estimates can be obtained in a
high-dimensional latent factor model. This is because, as noted earlier, if
the probability of a Type I error is only controlled at some fixed nonzero
level asymptotically, then consistent factor estimation may not be possible.
In addition, the precision with which the latent factors are estimated will
be reduced if we have a variable selection procedure where the probability
of a Type II error does not go to zero. As a result of these differences in
setup and objectives, the conditions that we specify for setting the tuning
parameter $\varphi $\ will also be quite different from that in Belloni,
Chernozhukov, and Hansen (2014).

Under appropriate conditions, the variable selection procedure described
above can be shown to be consistent, in the sense that both the probability
of a false positive, i.e. $P\left( i\in \widehat{H}^{c}|i\in H\right) $, and
the probability of a false negative, i.e., $P\left( i\in \widehat{H}|i\in
H^{c}\right) $, approach zero as $N_{1},N_{2},T\rightarrow \infty $. To show
this result, we must first state a number of additional assumptions.

\noindent \textbf{Assumption 2-7: }There exists a positive constant $%
\underline{c}$ such that for all $\tau \geq 1$ and $\tau _{1}\geq 1$: 
\begin{equation*}
\min_{1\leq \ell \leq d}\min_{i\in H}\min_{r\in \left\{ 1,...,q\right\}
}E\left\{ \left[ \frac{1}{\sqrt{\tau _{1}}}\dsum\limits_{t=\left( r-1\right)
\tau +p}^{\left( r-1\right) \tau +\tau _{1}+p-1}y_{\ell ,t{\LARGE +}1}u_{it}%
\right] ^{2}\right\} \geq \underline{c},
\end{equation*}%
where, as defined earlier, $\tau _{1}=\left\lfloor T_{0}^{\alpha _{{\large 1}%
}}\right\rfloor $, $\tau _{2}=\left\lfloor T_{0}^{\alpha _{{\large 2}%
}}\right\rfloor $ for $1>\alpha _{1}\geq \alpha _{2}>0$ and $q=\left\lfloor 
\frac{T_{0}}{\tau _{1}+\tau _{2}}\right\rfloor $, and $T_{0}=T-p+1$.

\noindent \textbf{Assumption 2-8: }Let $i\in H^{c}=\left\{ k\in \left\{
1,....,N\right\} :\gamma _{k}\neq 0\right\} $. Suppose that there exists a
positive constant, $\underline{c},$ such that, for all $N_{1},N_{2},$and $T$
sufficiently large:%
\begin{eqnarray*}
&&\min_{1\leq \ell \leq d}\min_{{\large i\in }H^{{\large c}}}\left\vert 
\frac{\mu _{i,\ell ,T}}{q\tau _{1}}\right\vert \\
&=&\min_{1\leq \ell \leq d}\min_{{\large i\in }H^{{\large c}}}\left\vert 
\frac{1}{q}\dsum\limits_{r=1}^{q}\frac{1}{\tau _{1}}\dsum\limits_{t=\left(
r-1\right) \tau +p}^{\left( r-1\right) \tau +\tau _{1}+p-1}\gamma
_{i}^{\prime }\left\{ E\left[ \underline{F}_{t}\right] \mu _{Y,\ell }+E\left[
\underline{F}_{t}\underline{Y}_{t}^{\prime }\right] \alpha _{YY,\ell }+E%
\left[ \underline{F}_{t}\underline{F}_{t}^{\prime }\right] \alpha _{YF,\ell
}\right\} \right\vert \\
&\geq &\underline{c}>0,
\end{eqnarray*}%
where $\mu _{Y,\ell }=e_{\ell ,d}^{\prime }\mu _{Y}$, $\alpha _{YY,\ell
}=A_{YY}^{\prime }e_{\ell ,d}$, and $\alpha _{YF,\ell }=A_{YF}^{\prime
}e_{\ell ,d}.$ Here, $e_{\ell ,d}$ is a $d\times 1$ elementary vector whose $%
\ell ^{th}$ component is $1$ and all other components are $0$.

\noindent \textbf{Assumption 2-9: }Suppose that, as $N_{1}$, $N_{2}$, and $%
T\rightarrow \infty $, the following rate conditions hold:

\begin{enumerate}
\item[(a)] $\sqrt{\ln N}/\min \left\{ T^{\left( 1-{\large \alpha }%
_{1}\right) /6},T^{\alpha _{2}/2}\right\} \rightarrow 0$, where $1>\alpha
_{1}\geq \alpha _{2}>0$ and $N=N_{1}+N_{2}$.

\item[(b)] $N_{1}/T^{3\alpha _{{\large 1}}}\rightarrow 0$ where $\alpha _{1}$
is as defined in part (a) above.
\end{enumerate}

\noindent \textbf{Assumption 2-10: }Let $\varphi $ satisfy the following two
conditions: (a) $\varphi \rightarrow 0$ as $N_{1},N_{2}\rightarrow \infty $,
and (b) there exists some constant $a>0,$ such that $\varphi \geq 1/N^{a},$
for all $N_{1},N_{2}$ sufficiently large.

\noindent \textbf{Remark 2.3:} \noindent \textbf{(a) }Assumption 2-8 imposes
the condition that there exists a positive constant, $\underline{c},$ such
that, for all $N_{1},N_{2},$ and $T$ sufficiently large: 
\begin{eqnarray*}
&&\min_{1\leq \ell \leq d}\min_{{\large i\in }H^{{\large c}}}\left\vert 
\frac{1}{q}\dsum\limits_{r=1}^{q}\frac{1}{\tau _{1}}\dsum\limits_{t=\left(
r-1\right) \tau +p}^{\left( r-1\right) \tau +\tau _{1}+p-1}\gamma
_{i}^{\prime }\left\{ E\left[ \underline{F}_{t}\right] \mu _{Y,\ell }+E\left[
\underline{F}_{t}\underline{Y}_{t}^{\prime }\right] \alpha _{YY,\ell }+E%
\left[ \underline{F}_{t}\underline{F}_{t}^{\prime }\right] \alpha _{YF,\ell
}\right\} \right\vert \\
&\geq &\underline{c}>0\text{.}
\end{eqnarray*}%
This is a fairly mild condition which allows us to differentiate the
alternative hypothesis, $i\in H^{c},$ from the null hypothesis, $i\in H,$
since if $i\in H$, then it is clear that:%
\begin{equation*}
\frac{\mu _{i,\ell ,T}}{q\tau _{1}}=\frac{1}{q}\dsum\limits_{r=1}^{q}\frac{1%
}{\tau _{1}}\dsum\limits_{t=\left( r-1\right) \tau +p}^{\left( r-1\right)
\tau +\tau _{1}+p-1}\gamma _{i}^{\prime }\left\{ E\left[ \underline{F}_{t}%
\right] \mu _{Y,\ell }+E\left[ \underline{F}_{t}\underline{Y}_{t}^{\prime }%
\right] \alpha _{YY,\ell }+E\left[ \underline{F}_{t}\underline{F}%
_{t}^{\prime }\right] \alpha _{YF,\ell }\right\} =0,
\end{equation*}%
given that $\gamma _{i}=0$. Note that this assumption does rule out certain
specialized situations, such as the case when $\mu _{Y,\ell }=0$, $\alpha
_{YY,\ell }=0$, and $\alpha _{YF,\ell }=0,$ for some $\ell \in \left\{
1,...,d\right\} $. However, we do not consider such cases to be of much
practical interest since, for example, if $\mu _{Y,\ell }=0$, $\alpha
_{YY,\ell }=0$, and $\alpha _{YF,\ell }=0$ for some $\ell $ then expression (%
\ref{Y component FAVAR}) implies that the $\ell ^{th}$ component of $Y_{t%
{\LARGE +}1}$ will have the representation $y_{\ell ,t{\LARGE +}1}=\mu
_{Y,\ell }+\underline{Y}_{t}^{\prime }\alpha _{YY,\ell }+\underline{F}%
_{t}^{\prime }\alpha _{YF,\ell }+\varepsilon _{\ell ,t{\LARGE +}%
1}^{Y}=\varepsilon _{\ell ,t{\LARGE +}1}^{Y}$, so that, in this case, $%
y_{\ell ,t{\LARGE +}1}$ depends neither on $\underline{Y}_{t}=\left(
Y_{t}^{\prime },Y_{t-1}^{\prime },...,Y_{t-p{\LARGE +}1}^{\prime }\right)
^{\prime }$ nor on $\underline{F}_{t}=\left( F_{t}^{\prime },F_{t-1}^{\prime
},...,F_{t-p{\LARGE +}1}^{\prime }\right) $. This is, of course, an
unrealistic model for $y_{\ell ,t{\LARGE +}1}$ since it would not even be a
dependent process in this case. \textbf{(b) }Bai and Ng (2008)\textbf{\ }%
address the important issue of pre-selecting variables $Z_{it}$ based on
their predictability for $Y_{t+1}$. Our selection approach is related to
theirs. However, it is worth stressing that for the FAVAR model considered
here, whether $Z_{it}$ helps predict future values of $Y_{t}$ (say, $Y_{t+h}$%
) depends on two things: (i) whether $Z_{it}$ loads significantly on the
underlying factors $\underline{F}_{t}$ (i.e., whether $\gamma _{i}\neq 0$ or
not) and (ii) whether at least some components of $\underline{F}_{t}$ are
helpful for predicting certain components of $Y_{t+h}$. The variable
selection procedure which we propose focuses on the first issue but not the
second. Thus, we focus on obtaining factor estimates with desirable
asymptotic properties before trying to assess which factor(s) may or may not
be useful for predicting $Y_{t{\LARGE +}h}$. Note that, for a given $t$, the
precision with which $\underline{F}_{t}$ is estimated depends primarily on
the size of the cross-sectional dimension, and the exclusion of any relevant 
$Z_{it}$ (with $\gamma _{i}\neq 0$) will have the negative effect of
reducing the sample size used for this estimation. More importantly, if we
exclude a significant number of variables (at the variable selection stage)
that load strongly on at least some of the factors, this can result in $%
\underline{F}_{t}$ being inconsistently estimated. While the question of
predictability is certainly an important one, the answer we get for this
question can, in some situations, be at odds with the objective of achieving
consistent factor estimation. This is because while $\gamma _{i}^{\prime }=0$
does imply that $Z_{i\cdot }$ will not be helpful for predicting future
values of $Y$, the reverse is not necessarily true. On the other hand, to
ensure consistent estimation of the factors, we would like to use every data
point $Z_{it},$ for which $\gamma _{i}^{\prime }\neq 0$. Furthermore, if it
is true that some of the factors load primarily on variables which are
uninformative predictors for certain components of $Y_{t+h}$, then that will
show up in the form of certain parameter restrictions on the forecasting
equation, in which case the best way to address this problem is to perform
hypothesis testing or model selection on the forecasting equation itself,
after the unobserved factors have first been properly estimated.

The following two theorems give our main theoretical results on the variable
selection procedure described above.

\noindent \textbf{Theorem 1: }\textit{Let\ }$H=\left\{ k\in \left\{
1,....,N\right\} :\gamma _{k}=0\right\} $\textit{. Suppose that Assumptions
2-1, 2-2, 2-3, 2-4, 2-5, 2-6, 2-7, 2-9 (a) and 2-10 hold. Let }$\Phi
^{-1}\left( \cdot \right) $\textit{\ denote the inverse of the cumulative
distribution function of the standard normal random variable, or,
alternatively, the quantile function of the standard normal distribution.
Then the following statements are true:}

\begin{enumerate}
\item[(a)] \textit{Let }$\left\{ \varpi _{\ell }:\ell =1,..,d\right\} $%
\textit{\ be pre-specified weights such that }$\varpi _{\ell }\geq 0$\textit{%
\ for every }$\ell \in \left\{ 1,...,d\right\} $\textit{\ and }$%
\dsum\nolimits_{\ell =1}^{d}\varpi _{\ell }=1$\textit{, then:}%
\begin{equation*}
P\left( \max_{{\large i\in }H}\dsum\limits_{\ell =1}^{d}\varpi _{\ell
}\left\vert S_{i,\ell ,T}\right\vert \geq \Phi ^{-1}\left( 1-\frac{\varphi }{%
2N}\right) \right) =O\left( \frac{N_{2}\varphi }{N}\right) =o\left( 1\right) 
\text{,}
\end{equation*}%
\textit{where }$N=N_{1}+N_{2}$\textit{.}

\item[(b)] 
\begin{equation*}
P\left( \max_{{\large i\in }H}\max_{1\leq \ell \leq d}\left\vert S_{i,\ell
,T}\right\vert \geq \Phi ^{-1}\left( 1-\frac{\varphi }{2N}\right) \right)
=O\left( \frac{N_{2}\varphi }{N}\right) =o\left( 1\right) \text{. }
\end{equation*}
\end{enumerate}

\noindent \textbf{Theorem 2: }\textit{Let }$H^{c}=\left\{ k\in \left\{
1,....,N\right\} :\gamma _{k}\neq 0\right\} $\textit{. Suppose that
Assumptions 2-1, 2-2, 2-3, 2-5, 2-6, 2-8, 2-9, and 2-10 hold. Then the
following statements are true.}

\begin{enumerate}
\item[(a)] \textit{Let }$\left\{ \varpi _{\ell }:\ell =1,..,d\right\} $%
\textit{\ be pre-specified weights such that }$\varpi _{\ell }\geq 0$\textit{%
\ for every }$\ell \in \left\{ 1,...,d\right\} $\textit{\ and }$%
\dsum\nolimits_{\ell =1}^{d}\varpi _{\ell }=1$\textit{, then: }%
\begin{equation*}
P\left( \min_{{\large i\in }H^{{\large c}}}\dsum\limits_{\ell =1}^{d}\varpi
_{\ell }\left\vert S_{i,\ell ,T}\right\vert \geq \Phi ^{-1}\left( 1-\frac{%
\varphi }{2N}\right) \right) \rightarrow 1\text{.}
\end{equation*}

\item[(b)] 
\begin{equation*}
P\left( \min_{{\large i\in }H^{{\large c}}}\max_{1\leq \ell \leq
d}\left\vert S_{i,\ell ,T}\right\vert \geq \Phi ^{-1}\left( 1-\frac{\varphi 
}{2N}\right) \right) \rightarrow 1\text{.}
\end{equation*}
\end{enumerate}

\noindent \textbf{Remark 2.4:} \textbf{(a)} Theorem 1 shows that, for both
of our statistics, the probability of a false positive approaches zero
uniformly over all ${\large i\in }$ $H$ as $N_{1},N_{2},T\rightarrow \infty $%
. The results of Theorem 2 further imply that, for both of our statistics,
the probability of a false negative also approaches zero, uniformly over all 
${\large i\in }$ $H^{{\large c}}$ as $N_{1},N_{2},T\rightarrow \infty $.
Together, these two theorems show that our procedure is (completely)
consistent in the sense that the probability of committing a
misclassification error vanishes as $N_{1},N_{2},T\rightarrow \infty $. 
\textbf{(b) }Note that our variable selection procedure also delivers a
consistent estimate of $N_{1}$ (i.e., $\widehat{N}_{1})$; this is shown in
Lemma D-15 part (a) of Chao, Liu, and Swanson (2023), where we establish
that $\widehat{N}_{1}/N_{1}\overset{p}{\rightarrow }1$. The estimator $%
\widehat{N}_{1}$ is useful to applied researchers implementing the
methodology developed in this paper, and also to empiricists interested in
assessing the rate condition for consistent factor estimation, given in
Assumption A4 of Bai and Ng (2021). This is another way in which the methods
developed in this paper built upon the work of Bai and Ng (2021). \textbf{%
(c) }In addition, note that knowledge of the number of factors is not needed
to implement our variable selection procedure. In the case where the number
of factors needs to be determined empirically, an applied researcher can
first use our procedure to select the relevant variables and then apply an
information criterion such as that proposed in Bai and Ng (2002) to estimate
the number of factors.

\section{\noindent Monte Carlo Study}

In this section, we report some simulation results on the finite sample
performance of our variable selection procedure. The model used in the Monte
Carlo study is the following tri-variate FAVAR(1) process: 
\begin{eqnarray}
W_{t} &=&\mu +AW_{t-1}+\varepsilon _{t},  \label{W eqn} \\
Z_{t} &=&\gamma F_{t}+u_{t}\text{,}  \label{Z eqn}
\end{eqnarray}%
where%
\begin{equation*}
W_{t}=\left( 
\begin{array}{c}
Y_{1t} \\ 
Y_{2t} \\ 
F_{t}%
\end{array}%
\right) \text{, }\mu =\left( 
\begin{array}{c}
2 \\ 
1 \\ 
2%
\end{array}%
\right) \text{, }A=\left( 
\begin{array}{ccc}
0.9 & 0.3 & 0.5 \\ 
0 & 0.7 & 0.1 \\ 
0 & 0.6 & 0.7%
\end{array}%
\right) \text{, and }\gamma =\left( 
\begin{array}{c}
\iota _{N_{1}} \\ 
\underset{N_{2}\times 1}{0}%
\end{array}%
\right) ,
\end{equation*}%
with $\iota _{N_{1}}$ denoting an $N_{1}\times 1$ vector of ones. We
consider different configurations of $N$, $N_{1}$, and $T,$ as given below.
For the error process in equation (\ref{W eqn}), we take $\left\{
\varepsilon _{t}\right\} \equiv i.i.d.N\left( 0,\Sigma _{\varepsilon
}\right) $, where: 
\begin{equation*}
\Sigma _{\varepsilon }=\left( 
\begin{array}{ccc}
1.3 & 0.99 & 0.641 \\ 
0.99 & 0.81 & 0.009 \\ 
0.641 & 0.009 & 5.85%
\end{array}%
\right) \text{.}
\end{equation*}%
The error process, $\left\{ u_{it}\right\} ,$ in equation (\ref{Z eqn}) is
allowed to exhibit both temporal and cross-sectional dependence and also
conditional heteroskedasticity. More specifically, we let $%
u_{it}=0.8u_{it-1}+\zeta _{it}$, and following the approach for modeling
cross-sectional dependence given in the Monte Carlo design of Stock and
Watson (2002a), we specify: $\zeta _{it}=\left( 1+b^{2}\right) \eta
_{it}+b\eta _{i+1,t}+b\eta _{i-1,t}$, and set $b=1$. In addition, $\eta
_{it}=\omega _{it}\xi _{it},$ with $\left\{ \xi _{it}\right\} \equiv
i.i.d.N\left( 0,1\right) $ independent of $\left\{ \varepsilon _{t}\right\} $%
, and $\omega _{it}$ follows a GARCH(1,1) process given by: $\omega
_{it}^{2}=1+0.9\omega _{it-1}^{2}+0.05\eta _{it-1}^{2}$. To study the
effects of varying the tuning parameter, we consider specifications where $%
\varphi =\left( \ln \ln N\right) ^{-\vartheta }$ for $\vartheta =0.1,0.5,1$
and also $\varphi =N^{-\vartheta }$ for $\vartheta =0.2,0.4,0.6$.\footnote{%
We have also obtained simulation results for the cases where $\varphi
=\left( \ln N\right) ^{-\vartheta }$ for $\vartheta =0.1,0.5,1$ and where $%
\varphi =N^{-\vartheta }$ for $\vartheta =0.3,0.5,0.7$. The results obtained
for these cases are qualitatively similar to the results reported in this
paper. Hence, due to space considerations, we do not report these results
here, but they are available from the authors upon request.} We also attempt
to shed light on the effects of using blocks of different sizes on the
performance of our procedure. To do this, for $T=100$, we set $\tau _{1}=2$, 
$3$, $4$, and $5$; for $T=200$, we set $\tau _{1}=5$, $6$, $8$, and $10$;
and for $T=600$, we set $\tau _{1}=6$, $8$, $10$, and $12$. Due to space
considerations, we only report Monte Carlo results for the statistic $%
\dsum\nolimits_{\ell =1}^{d}\varpi _{\ell }\left\vert S_{i,\ell
,T}\right\vert $. Simulation results for the statistic $\max_{1\leq \ell
\leq d}\left\vert S_{i,\ell ,T}\right\vert $ have also been obtained by the
authors and are qualitatively similar to the results reported here for $%
\dsum\nolimits_{\ell =1}^{d}\varpi _{\ell }\left\vert S_{i,\ell
,T}\right\vert $. The results for $\max_{1\leq \ell \leq d}\left\vert
S_{i,\ell ,T}\right\vert $ are available from the authors upon request. In
addition, since $d=2$ in our Monte Carlo setup, we set $\varpi _{1}=\varpi
_{2}=1/2$. Results are gathered in in Table 1, where FPR denotes the
\textquotedblleft False Positive Rate\textquotedblright\ or the
\textquotedblleft Type I\textquotedblright\ error rate, i.e., the proportion
of cases where an irrelevant variable $Z_{it}$, with associated coefficient $%
\gamma _{i}=0$ is erroneously selected as a relevant variable. FNR denotes
the \textquotedblleft False Negative Rate\textquotedblright\ or the
\textquotedblleft Type II\textquotedblright\ error rate, i.e., the
proportion of cases where a relevant variable is erroneously identified as
being irrelevant.

Looking across each row of the table, note that FPRs decrease when moving
from left to right, whereas FNRs increase. This is not surprising, because
moving from $\varphi =\left( \ln \ln N\right) ^{-0.1}$ to $\varphi =N^{-0.6}$
for a given $N$ results in smaller values of the tuning parameter $\varphi $%
, and the specified threshold $\Phi ^{-1}\left( 1-\frac{\varphi }{2N}\right) 
$ thus becomes larger. Overall, these results indicate that choosing $%
\varphi $ in the range between $\left( \ln \ln N\right) ^{-0.1}$ and $%
N^{-0.4}$\ leads to very good performance, since within this range, neither
FPR nor FNR exceeds $0.1$ in any of the cases studied here. In fact, both
are smaller than $0.05$ in a vast majority of the cases. In contrast,
choosing $\varphi =N^{-0.6}$ can lead to high FNRs, as such a choice of $%
\varphi $ can set our threshold at such a high level that our procedure ends
up having very little power.

Looking down the columns of the table, note that FPR tends to increase as $%
\tau _{1}$ increases, whereas FNR tends to decrease as $\tau _{1}$
increases. As an explanation for this result, note first that the smaller is 
$\tau _{1}$ relative to $\tau $, the larger is $\tau _{2}$ (since $\tau
=\tau _{1}+\tau _{2}$), and thus the larger is the number of observations
removed when constructing the self-normalized block sums. Intuitively, this
can lead to better accommodation of the effects of dependence and better
moderate deviation approximations under the null hypothesis, resulting in a
lower FPR. However, removal of a larger number of observations can also lead
to a reduction in power, when the alternative hypothesis is correct, so that
a negative consequence of having a smaller $\tau _{1}$ relative to $\tau $
is that FNR will tend to be higher in this case. The opposite, of course,
occurs when we try to specify a larger $\tau _{1}$ relative to $\tau $.

Our results also show that when the sample sizes are large enough such as
the cases presented in the last panel of the table, where $T=600$ and $%
N=1000 $, then both FPR and FNR are small for all of the cases that we
consider. This is in accord with the results of our theoretical analysis,
which shows that our variable selection procedure is completely consistent
in the sense that both the probability of a false positive and the
probability of a false negative approach zero, as the sample sizes go to
infinity.

\section{\noindent Empirical Application}

In this section, we construct forecasts of 3-month, 1-year, 3-year, 5-year,
and 10-year maturity interest rates, at $h=$1-month, 3-month, and 12-month
ahead horizons. The interest rates that we analyze are contained in the
so-called U.S. Treasury yield curve dataset, which is discussed in
Gurkaynak, Sack, and Wright (GSW: 2007). In order to assess the empirical
usefulness of our new variable selection method, we construct our forecasts
using a variety of \textquotedblleft big-data\textquotedblright\ as well as
\textquotedblleft small-data\textquotedblright\ models. Our big data models
utilize the GSW dataset in conjunction with the so-called FRED-MD real-time
macroeconomic dataset, which is available at the St. Louis federal reserve
bank website, and includes a broad array of 130 economic variables (see GSW
(2007) for further discussion of this dataset). All of these time series
variables are \textquotedblleft real-time\textquotedblright\ in the sense
that for each calendar date there may be multiple observations,
corresponding to the \textquotedblleft first-release\textquotedblright\ of
an observation for that calendar date, a \textquotedblleft second
release\textquotedblright , often a month or more later, for that same
calendar date, and so on. Many macroeconomic variables are subject to these
sorts of revisions, due to the data collection methodology used by the
relevant government reporting agencies. These different releases are called
vintages. In all of our experiments that utilize such real-time data, we
ensure that only vintages truly available prior to the construction of each
of our time series forecasts are actually used in model estimation and
forecast construction. Needless to say, this requires us to re-estimate all
models at each point in time, prior to the construction of each new
forecast. In our analysis, we use rolling (fixed) windows of 120 months for
all estimations. The sample period of our interest rate dataset, as well as
our real-time macroeconomic dataset is August 1988 - December 2021. Our
forecasting sample period is April 2001 - December 2021.

\begin{table}[tbp]
{{\textbf{Table 1: Monte Carlo Results for $\mathbb{S}_{i,T}^{+}=\dsum%
\nolimits_{\ell =1}^{d}\varpi _{\ell }\left\vert S_{i,\ell ,T}\right\vert $}}
\newline
\newline
\newline
\hspace{0.75in}} \vspace{0.5in} {\small 
\begin{tabular}{|l|l|c|c|c|c|c|c|}
\hline
&  & $N=100$ & $N_{1}=50$ & $T=100$ & $\tau =5$ &  &  \\ \hline
&  & \multicolumn{1}{|l|}{{\scriptsize {$\varphi =\left( \ln \ln N\right)
^{-0.1}$}}} & \multicolumn{1}{|l|}{{\scriptsize {$\varphi =\left( \ln \ln
N\right) ^{-0.5}$}}} & \multicolumn{1}{|l|}{{\scriptsize {$\varphi =\left(
\ln \ln N\right) ^{-1}$}}} & \multicolumn{1}{|l|}{$\varphi =N^{-0.2}$} & $%
\varphi =N^{-0.4}$ & $\varphi =N^{-0.6}$ \\ \hline\hline
$\tau _{1}=2$ & FPR & 0.03916 & 0.03350 & 0.02678 & 0.01460 & 0.00382 & 
0.00076 \\ \hline
& FNR & 0.00046 & 0.00068 & 0.00104 & 0.00284 & 0.01674 & 0.09412 \\ \hline
$\tau _{1}=3$ & FPR & 0.04544 & 0.03902 & 0.03110 & 0.01810 & 0.00526 & 
0.00092 \\ \hline
& FNR & 0.00022 & 0.00032 & 0.00052 & 0.00172 & 0.01100 & 0.06942 \\ \hline
$\tau _{1}=4$ & FPR & 0.05408 & 0.04650 & 0.03756 & 0.02224 & 0.00702 & 
0.00162 \\ \hline
& FNR & 0.00016 & 0.00024 & 0.00034 & 0.00118 & 0.00828 & 0.05194 \\ \hline
$\tau _{1}=5$ & FPR & 0.06332 & 0.05462 & 0.04558 & 0.02796 & 0.00924 & 
0.00232 \\ \hline
& FNR & 0.00014 & 0.00018 & 0.00034 & 0.00084 & 0.00574 & 0.03948 \\ \hline
&  & $N=200$ & $N_{1}=100$ & $T=100$ & $\tau =5$ &  &  \\ \hline
$\tau _{1}=2$ & FPR & 0.01913 & 0.01470 & 0.01068 & 0.00486 & 0.00064 & 
0.00002 \\ \hline
& FNR & 0.00206 & 0.00282 & 0.00449 & 0.01415 & 0.09966 & 0.48356 \\ \hline
$\tau _{1}=3$ & FPR & 0.02341 & 0.01842 & 0.01365 & 0.00657 & 0.00098 & 
0.00005 \\ \hline
& FNR & 0.00143 & 0.00190 & 0.00315 & 0.00921 & 0.07372 & 0.40894 \\ \hline
$\tau _{1}=4$ & FPR & 0.02869 & 0.02306 & 0.01733 & 0.00841 & 0.00133 & 
0.00004 \\ \hline
& FNR & 0.00111 & 0.00145 & 0.00224 & 0.00661 & 0.05564 & 0.34279 \\ \hline
$\tau _{1}=5$ & FPR & 0.03506 & 0.02903 & 0.02194 & 0.01124 & 0.00213 & 
0.00017 \\ \hline
& FNR & 0.00086 & 0.00112 & 0.00172 & 0.00477 & 0.04258 & 0.28620 \\ \hline
&  & $N=400$ & $N_{1}=200$ & $T=200$ & $\tau =10$ &  &  \\ \hline
\multicolumn{1}{|c|}{$\tau _{1}=5$} & FPR & 0.00214 & 0.00148 & 0.00090 & 
0.00030 & 2.5$\times $10$^{-5}$ & 0.00000 \\ \hline
\multicolumn{1}{|c|}{} & FNR & 7.5$\times $10$^{-5}$ & 0.00016 & 0.00040 & 
0.00231 & 0.06894 & 0.67266 \\ \hline
\multicolumn{1}{|c|}{$\tau _{1}=6$} & FPR & 0.00249 & 0.00166 & 0.00104 & 
0.00034 & 0.00002 & 0.00000 \\ \hline
\multicolumn{1}{|c|}{} & FNR & 0.00004 & 0.00009 & 0.00025 & 0.00148 & 
0.05058 & 0.60968 \\ \hline
\multicolumn{1}{|c|}{$\tau _{1}=8$} & FPR & 0.00337 & 0.00235 & 0.00142 & 
0.00046 & 0.00004 & 0.00000 \\ \hline
\multicolumn{1}{|c|}{} & FNR & 0.00001 & 0.00002 & 0.00008 & 0.00068 & 
0.02712 & 0.48133 \\ \hline
\multicolumn{1}{|c|}{$\tau _{1}=10$} & FPR & 0.00484 & 0.00350 & 0.00220 & 
0.00079 & 7.5$\times $10$^{-5}$ & 5.0$\times $10$^{-6}$ \\ \hline
\multicolumn{1}{|c|}{} & FNR & 0.00001 & 0.00001 & 0.00002 & 0.00034 & 
0.01535 & 0.36382 \\ \hline
&  & $N=1000$ & $N_{1}=500$ & $T=600$ & $\tau =12$ &  &  \\ \hline
\multicolumn{1}{|c|}{$\tau _{1}=6$} & FPR & 0.00155 & 0.00121 & 0.00086 & 
0.00038 & 0.00006 & 0.00001 \\ \hline
\multicolumn{1}{|c|}{} & FNR & 0.00000 & 0.00000 & 0.00000 & 0.00000 & 
0.00000 & 0.00000 \\ \hline
\multicolumn{1}{|c|}{$\tau _{1}=8$} & FPR & 0.00201 & 0.00153 & 0.00106 & 
0.00049 & 8.2$\times $10$^{-5}$ & 1.4$\times $10$^{-5}$ \\ \hline
\multicolumn{1}{|c|}{} & FNR & 0.00000 & 0.00000 & 0.00000 & 0.00000 & 
0.00000 & 0.00000 \\ \hline
\multicolumn{1}{|c|}{$\tau _{1}=10$} & FPR & 0.00274 & 0.00216 & 0.00155 & 
0.00072 & 0.00016 & 3.2$\times $10$^{-5}$ \\ \hline
\multicolumn{1}{|c|}{} & FNR & 0.00000 & 0.00000 & 0.00000 & 0.00000 & 
0.00000 & 0.00000 \\ \hline
\multicolumn{1}{|c|}{$\tau _{1}=12$} & FPR & 0.00421 & 0.00332 & 0.00242 & 
0.00115 & 0.00028 & 6.0$\times $10$^{-5}$ \\ \hline
\multicolumn{1}{|c|}{} & FNR & 0.00000 & 0.00000 & 0.00000 & 0.00000 & 
0.00000 & 0.00000 \\ \hline
\end{tabular}%
} {\footnotesize 
\begin{minipage}{1\columnwidth}
{\noindent Notes: False positive and negative rates are reported for various values of 
$N, N_1,$ and $T$. Results are based on 1000 simulations. See Section 3 for complete details.}
\end{minipage}
}
\end{table}

The forecasting models that we evaluate are summarized in Table 2, and all
have the following function form: 
\begin{equation}
y_{t+h}(\tau )=\alpha +\sum_{i=1}^{p}\beta _{i}y_{t-i+1}(\tau )+\gamma
_{1}^{\prime }W_{t}+\gamma _{2}^{\prime }F_{t}+\epsilon _{t+h},
\end{equation}%
where $y_{t+h}(\tau )$, is an annual yield interest rate, $\tau $ denotes
the maturity of the bond (bill) being forecast, $F_{t}$ is an $r-$%
dimensional vector of estimated factors, $W_{t}$ $\ $includes additional
variables from our real-time dataset, $\epsilon _{t+h}$ is a stochastic
disturbance term, and $p$ is the number of lags, selected using the Schwarz
information criterion (SIC). All variables were differenced to stationarity,
using stationarity test results reported in the documentation that
accompanies the FRED-MD dataset. Additionally, we considered $r=\{1,2,3\}.$
In all experiments, setting $r=1$ yieled more precise predictions. Thus,
results in the sequel are for the case where $r=1$. In this setup, we
consider a number of models with $\gamma _{2}=0.$ These include: the simple
AR(SIC) benchmark, where we additionally impose that $\gamma _{1}=0$; a
model called AR+LASSO, where we choose the elements of $W_{t}$ using the
LASSO; a model called AR+EN, where we choose the elements of $W_{t}$ using
the EN; and a model called AR+CS, where we choose the elements of $W_{t}$
using the the \textquotedblleft CS\textquotedblright\ method discussed in
Section 2.\footnote{%
Note that for this model we do not use the variables selected using the CS
method to estimate factors. Rather, we simply use the $S_{i,T}^{+}=\dsum%
\limits_{\ell =1}^{d}\varpi _{\ell }\left\vert S_{i,\ell ,T}\right\vert $
statistics to select a subset of variables for inclusion in the forecasting
model. Also, we set $\tau ,$ $\tau _{1},$ and $\varphi $ equal to 5, 4, and (%
$\ln \ln N)^{-0.1}$ in the sequel.} We also consider a number of models with 
$\gamma _{1}=0.$ These include: AR+PCA, where factors are estimated using
PCA applied to our entire real-time dataset; AR+LASSO+PCA, where factors are
estimated using PCA applied to a subset of our real-time dataset that is
selected using the LASSO; AR+EN+PCA, where factors are estimated using PCA
applied to a subset of our real-time dataset that is selected using the EN;
and AR+CS+PCA, where factors are estimated using PCA applied to a subset of
our real-time dataset that is selected using the CS.\footnote{%
For the CS method, we again use the $S_{i,T}^{+}=\dsum\limits_{\ell
=1}^{d}\varpi _{\ell }\left\vert S_{i,\ell ,T}\right\vert $ statistic, and
the tuning parameters for the LASSO and EN were estimated anew prior to the
construction of each forecast, using 10-fold cross validation.}

In addition to our \textquotedblleft small-data\textquotedblright\ AR(SIC)
model, we include another parsimonious model that utilizes only interest
rates when specifying forecasting models. This is the dynamic Nelson-Siegel
(DNS) model that is widely used in industry and government for forecasting
interest rates, as discussed in Diebold, Rudebusch and Aruoba (2006), and
Swanson and Xiong (2018). The Nelson and Siegel (1987) model specifies the
relationship between spot interest rates and instantaneous forward rates by
applying rational expectation theory. Specifically, the yield of a bond with
maturity $m$ can be expressed by the averaged forward rates: $y_{t}(\tau )=%
\frac{1}{m}\int_{0}^{m}f_{t}(m)d\tau ,$ where $y_{t}(\tau )$ is the yield at
time $t$ for a bond with maturity $\tau $, and $f_{t}(m)$ denotes the
instantaneous forward rate at time $t$ for a bond with time-to-maturity $m$.
Based on this setup, the dynamic Nelson-Siegel model approximates the term
structure of interest rates using a parsimonious three-factor model: 
\begin{equation}
y_{t}(\tau )=\beta _{0,t}+\beta _{1,t}[\frac{1-exp(-\frac{\tau }{\theta _{t}}%
)}{\frac{\tau }{\theta _{t}}}]+\beta _{2,t}[\frac{1-exp(-\frac{\tau }{\theta
_{t}})}{\frac{\tau }{\theta _{t}}}-exp(-\frac{\tau }{\theta _{t}})].
\end{equation}%
In this model, $\beta _{0,t}$, $\beta _{1,t}$, and $\beta _{2,t}$ are latent
factors representing the level, slope, and curvature of the yield curve,
respectively, and $\lambda =\frac{1}{\theta _{t}}$ is a decay parameter. For
a complete discussion of this model, refer to Swanson and Xiong (2018),
where the latent factors are modeled using a vector autoregression, and are
estimated either solely using the GSW dataset (our \textquotedblleft small
data\textquotedblright\ DNS model), or by including extra variables and/or
latent factors constructed via the FRED-MD dataset along with LASSO, EN, or
CS methods (see Table 2 for details).

Our empirical findings are gathered in Table 3. Inspection of the mean
square forecast errors (MSFEs) in this table reveal that many of our
\textquotedblleft big data\textquotedblright\ models are \textquotedblleft
MSFE-best\textquotedblright\ (i.e. exhibit lower point MSFEs), for shorter
horizon forecasts of $h=1$ and 3.\footnote{%
Many \textquotedblleft big data\textquotedblright\ models also deliver
significantly superior predictions to the AR(SIC) model (as well as the DNS
model), based on application of the Diebold-Mariano (1995) predictive
accuracy test, as denoted by entries that are starred.} Interestingly, this
is not the case for $h=12$. For our longest horizon forecasts our
\textquotedblleft small data\textquotedblright\ DNS model \textquotedblleft
wins\textquotedblright\ at almost all maturities. Thus, big data appears to
be particularly useful for shorter horizon forecasts. With regard to the
performance of the CS method, note that for 1-month ahead predictions of
interest rates at 3-month and 1-year maturities, our method yields
statistically superior forecasts relative to all other methods analyzed, and
is approximately \textquotedblleft tied\textquotedblright\ with PCA for
longer maturities. For 3-month ahead predictions, our method yields models
with the lowest or second lowest mean-square forecast errors of any model,
at all maturities. In this sense, our first evidence suggests that the CS
method is useful when selecting variables prior to constructing factors for
use in forecasting models.

\section{\noindent Concluding Remarks}

In this paper, we propose a new variable selection procedure based on two
alternative self-normalized score statistics and provide asymptotic analyses
showing that our procedure, based on either of these statistics, correctly
identify the set of variables which load significantly on the underlying
factors, with probability approaching one as the sample sizes go to
infinity. Our research is motivated by the observation that inconsistency in
factor estimation could result in high dimensional settings when the
conventional assumption of factor pervasiveness does not hold. Hence, in
such settings, it is particularly important to pre-screen the variables in
terms of their association with the underlying factors prior to estimation.
We conduct a small Monte Carlo study which yields encouraging evidence about
the finite sample properties of our variable selection procedure. It is also
worth noting that in a companion paper (Chao, Liu, and Swanson, 2023), we
prove that consistent estimation of factors (up to an invertible matrix
transformation) can be achieved by estimating factors using only those
variables selected by our method, and this is so even in certain situations
where the standard pervasiveness assumption does not hold. In addition, in
the same paper, we further show that by plugging factors estimated in such a
manner into the factor-augmented forecasting equation implied by the FAVAR
model, the conditional mean function of the forecasting equation can be
consistently estimated, even for the case of multi-step ahead forecasts.
Finally, we present a series Monte Carlo and empirical experiments
underscoring the potential usefulness of our procedure in empirical
settings. In sum, the collective body of results, including theoretical,
Monte Carlo,

\renewcommand{\baselinestretch}{1}

\begin{table}[tbp]
\noindent{{\textbf{Table 2: Models Used in Prediction Experiments}}}
\par
\begin{center}
{\footnotesize \renewcommand{\arraystretch}{1.5} 
% Increase the space between lines (use a factor greater than 1)
\begin{tabular}{|p{0.2\linewidth}|p{0.7\linewidth}|}
\hline
Model & Description \\ \hline
AR(SIC) & Autoregressive model with lag(s) selected by the Schwarz
information criterion. \\ 
AR+LASSO & AR(SIC) model augmented to include a subset of variables selected
from our real-time dataset using the LASSO. \\ 
AR+EN & AR(SIC) model augmented to include a subset of variables selected
from our real-time dataset using the Elastic Net. \\ 
AR+CS & AR(SIC) model augmented to include a subset of variables selected
from our real-time dataset using the ``Chao-Swanson'' (CS) variable
selection method discussed in Section 2. \\ 
AR+PCA & AR(SIC) model augmented with PCA type factors constructed using our
entire real-time dataset. \\ 
AR+LASSO+PCA & AR(SIC) model augmented with PCA type factors constructed
using a subset of variables from our real-time dataset, with the subset
selected using the LASSO. \\ 
AR+EN+PCA & AR(SIC) model augmented with PCA type factors constructed using
a subset of variables from our real-time dataset, with the subset selected
using the EN. \\ 
AR+CS+PCA & AR(SIC) model augmented with PCA type factors constructed using
a subset of variables from our real-time dataset, with the subset selected
using the CS method. \\ 
DNS & Dynamic Nelson-Siegel (DNS) model with underlying VAR(1) factors
fitted using our twelve different maturity yield data, and with a static
rate of decay parameter of $\lambda$ = 0.0609 \\ 
DNS+LASSO & DNS model augmented to include a subset of variables selected
from our real-time dataset using the LASSO. \\ 
DNS+EN & DNS model augmented to include a subset of variables selected from
our real-time dataset using the Elastic Net. \\ 
DNS+CS & DNS model augmented to include a subset of variables selected from
our real-time dataset using the ``Chao-Swanson'' (CS) variable selection
method. \\ 
DNS+PCA & DNS model augmented with PCA type factors constructed using our
entire real-time dataset. \\ 
DNS+LASSO+PCA & DNS model augmented with PCA type factors constructed using
a subset of variables from our real-time dataset, with the subset selected
using the LASSO. \\ 
DNS+LASSO+EN & DNS model augmented with PCA type factors constructed using a
subset of variables from our real-time dataset, with the subset selected
using the EN. \\ 
DNS+LASSO+CS & DNS model augmented with PCA type factors constructed using a
subset of variables from our real-time dataset, with the subset selected
using the CS method. \\ \hline
\end{tabular}%
}
\end{center}
\par
{\footnotesize 
\begin{minipage}{1\columnwidth}
{\noindent Notes: This table includes brief descriptions of the 16 different forecasting models used in our empirical experiment in which we predict monthly interest rates for bonds of  
maturity 3-months, as well as 1, 3, 5, and 10-years, at 1, 3-, and 12-month ahead forecast horizons. See Section 5 for complete details.}
\end{minipage}}
\end{table}

\begin{table}[tbp]
\noindent{{\textbf{Table 3: Point MSFEs for Various Maturities of U.S.
Interest Rates}}}
\par
\begin{center}
{\scriptsize 
\begin{tabular}{lccccc}
\hline
Forecast Horizon/Model & \multicolumn{5}{c}{Interest Rate Maturity} \\ 
h=1 & 3mo & 1yr & 3yr & 5yr & 10yr \\ \hline
AR(SIC) & 0.0367 & 0.0320 & 0.0435 & 0.0477 & 0.0491 \\ 
AR+LASSO & 0.0303 & 0.0335 & 0.0467 & 0.0601** & 0.0595** \\ 
AR+EN & 0.0344* & 0.0378 & 0.0652*** & 0.0730** & 0.0686*** \\ 
AR+CS & 0.0374 & 0.0455 & 0.0978 & 0.1100 & 0.1095 \\ 
AR+PCA & 0.0356 & 0.0320 & 0.0438 & 0.0480 & 0.0494 \\ 
AR+LASSO+PCA & 0.0373 & 0.0313 & 0.0459 & 0.0492 & 0.0501 \\ 
AR+EN+PCA & 0.0357** & 0.0324 & 0.0436*** & 0.0488 & 0.0501 \\ 
AR+CS+PCA & 0.0315 & 0.0294 & 0.0441 & 0.0488 & 0.0500 \\ 
DNS & 0.1680*** & 0.2895*** & 0.1416*** & 0.0955*** & 0.1887*** \\ 
DNS+LASSO & 0.0346*** & 0.0415*** & 0.0553*** & 0.0667*** & 0.0558*** \\ 
DNS+EN & 0.0376* & 0.0508 & 0.0602 & 0.0880*** & 0.0576 \\ 
DNS+CS & 0.0401 & 0.0433 & 0.0527 & 0.0703 & 0.0715*** \\ 
DNS+PCA & 0.0493 & 0.0473 & 0.0532 & 0.0702 & 0.0559** \\ 
DNS+LASSO+PCA & 0.0472 & 0.0461 & 0.0504 & 0.0651* & 0.0570 \\ 
DNS+EN+PCA & 0.0436* & 0.0442 & 0.0504 & 0.0663 & 0.0559 \\ 
DNS+CS+PCA & 0.0440 & 0.0343*** & 0.0481 & 0.0684 & 0.0546 \\ \hline
h=3 &  &  &  &  &  \\ \hline
AR(SIC) & 0.2209 & 0.2124 & 0.2301 & 0.2159 & 0.1909 \\ 
AR+LASSO & 0.1761* & 0.2208 & 0.2977** & 0.2995*** & 0.2259** \\ 
AR+EN & 0.1821 & 0.2380 & 0.2972 & 0.3400 & 0.2766*** \\ 
AR+CS & 0.1870 & 0.2381 & 0.3547 & 0.3546 & 0.3164 \\ 
AR+PCA & 0.2096 & 0.2078 & 0.2423** & 0.2351*** & 0.2055*** \\ 
AR+LASSO+PCA & 0.2363 & 0.2539** & 0.2675 & 0.2318 & 0.2231 \\ 
AR+EN+PCA & 0.2242* & 0.2390* & 0.2608 & 0.2508** & 0.2053* \\ 
AR+CS+PCA & 0.1925 & 0.2172 & 0.2659 & 0.2453 & 0.2038 \\ 
DNS & 0.2186** & 0.3444*** & 0.1975 & 0.1641 & 0.2704** \\ 
DNS+LASSO & 0.1350*** & 0.1469*** & 0.1872*** & 0.2165 & 0.1654*** \\ 
DNS+EN & 0.1273 & 0.1546 & 0.2039 & 0.2274 & 0.1716 \\ 
DNS+CS & 0.1370 & 0.1448 & 0.1901 & 0.2092 & 0.1819 \\ 
DNS+PCA & 0.1927** & 0.1795* & 0.2116 & 0.2216 & 0.1767 \\ 
DNS+LASSO+PCA & 0.2056* & 0.1901 & 0.2441* & 0.2202 & 0.1791 \\ 
DNS+EN+PCA & 0.2108 & 0.1888 & 0.2126*** & 0.2325 & 0.1725 \\ 
DNS+CS+PCA & 0.1667*** & 0.1517*** & 0.2011 & 0.2159 & 0.1725 \\ \hline
h=12 &  &  &  &  &  \\ \hline
AR(SIC) & 2.7069 & 2.5697 & 1.9402 & 1.3847 & 0.8476 \\ 
AR+LASSO & 1.6320*** & 1.1118*** & 1.6501 & 1.5874 & 1.4823*** \\ 
AR+EN & 1.4464 & 1.0408 & 1.1346*** & 1.1303*** & 1.2279*** \\ 
AR+CS & 1.0182*** & 1.4160 & 1.6360* & 1.2730 & 0.9431** \\ 
AR+PCA & 3.5114*** & 2.9847*** & 1.9703 & 1.4047 & 0.9173 \\ 
AR+LASSO+PCA & 2.9536** & 2.4207*** & 1.9940 & 1.4123 & 1.3234*** \\ 
AR+EN+PCA & 3.4974* & 2.6272 & 1.9954 & 1.5801 & 1.3021 \\ 
AR+CS+PCA & 3.0519 & 3.3013** & 2.5987*** & 1.7962* & 1.1204 \\ 
DNS & 0.6586*** & 0.7340*** & 0.5146*** & 0.4720*** & 0.5409*** \\ 
DNS+LASSO & 1.7234*** & 1.3490*** & 0.9325*** & 0.7391*** & 0.5374*** \\ 
DNS+EN & 1.7606* & 1.3172 & 0.9019 & 0.7612*** & 0.5265 \\ 
DNS+CS & 1.7807*** & 1.3727*** & 0.9808*** & 0.8312*** & 0.6337*** \\ 
DNS+PCA & 1.8072 & 1.4837*** & 1.0360 & 0.8345 & 0.5687*** \\ 
DNS+LASSO+PCA & 2.0002*** & 1.6740*** & 1.1609*** & 0.9063*** & 0.5864*** \\ 
DNS+EN+PCA & 2.0345 & 1.6647 & 1.1531 & 0.8863 & 0.5845 \\ 
DNS+CS+PCA & 1.7879*** & 1.4277*** & 0.9647*** & 0.7774*** & 0.5390** \\ 
\hline
\end{tabular}
}
\end{center}
\par
{\footnotesize 
\begin{minipage}{1\columnwidth}
{\noindent Notes: See notes to Table 2. Entries are MSFEs for predictions of 3-month, 
1, 3, 5, and 10-year maturity yields, at $h=$1, 3, and 12-month ahead horizons. 
The entire sample period of the yield and real-time macroeconomic datasets used is 1988:08-2021:12, while the forecast sample 2001:04-2021:12. 
Forecasting models, denoted in column one, are defined in Table 2. 
Entries superscripted with *,**,or *** denote rejection, based on application of 
the Diebold-Mariano (1995) predictive accuracy test, of the null hypothesis of equal predictive accuracy, relative to the AR(SIC) 
benchmark model, on average, at 10 \%,5\%,or 1\% significance levels, respectively. See Section 5 for complete details.}
\end{minipage}
}
\end{table}

%\pagenumbering{gobble}

\noindent and empirical discussed in this paper indicates that the proposed
variable selection methodology can be useful to empirical researchers as
they engage in the important tasks of factor estimation and the construction
of point forecasts based on factor-augmented forecasting equations.

\noindent

\noindent

\linespread{1.5}

{\singlespacing
}

\begin{thebibliography}{99}
\bibitem{} Ahn, S. C. and J. Bae (2022): \textquotedblleft Forecasting with
Partial Least Squares When a Large Number of Predictors Are Available,"
Working Paper, Arizona State University and University of Glasgow.

\bibitem{} Anatolyev, S. and A. Mikusheva (2021): \textquotedblleft Factor
Models with Many Assets: Strong Factors, Weak Factors, and the Two-Pass
Procedure,\textquotedblright\ \textit{Journal of Econometrics}, forthcoming.

\bibitem{} Andrews, D.W.K. (1984): \textquotedblleft Non-strong Mixing
Autoregressive Processes,\textquotedblright\ \textit{Journal of Applied
Probability}, 21, 930-934.

\bibitem{} Bai, J. and S. Ng (2002): \textquotedblleft Determining the
Number of Factors in Approximate Factor Models,\textquotedblright\ \textit{%
Econometrica}, 70, 191-221.

\bibitem{} Bai, J. and S. Ng (2008): \textquotedblleft Forecasting Economic
Time Series Using Targeted Predictors,\textquotedblright\ \textit{Journal of
Econometrics}, 146, 304-317.

\bibitem{} Bai, J. and S. Ng (2021): \textquotedblleft Approximate Factor
Models with Weaker Loading,\textquotedblright\ Working Paper, Columbia
University.

\bibitem{} Bair, E., T. Hastie, D. Paul, and R. Tibshirani (2006):
\textquotedblleft Prediction by Supervised Principal
Components,\textquotedblright\ \textit{Journal of the American Statistical
Association}, 101, 119-137.

\bibitem{} Belloni, A., V. Chernozhukov, and C. Hansen (2014):
\textquotedblleft Inference on Treatment Effects after Selection among
High-Dimensional Controls," \textit{Review of Economic Studies}, 81, 608-650.

\bibitem{} Bickel, P. J. and E. Levina (2008): \textquotedblleft Covariance
Regularization by Thresholding," \textit{Annals of Statistics}, 36,
2577-2604.

\bibitem{} Bryzgalova, S. (2016): \textquotedblleft Spurious Factors in
Linear Asset Pricing Models,\textquotedblright\ Working Paper, Stanford
Graduate School of Business.

\bibitem{} Burnside, C. (2016): \textquotedblleft Identification and
Inference in Linear Stochastic Discount Factor Models with Excess
Returns,\textquotedblright\ \textit{Journal of Financial Econometrics}, 14,
295-330.

\bibitem{} Chao, J. C., Y. Liu, and N. R. Swanson (2023): \textquotedblleft
Consistent Factor Estimation and Forecasting in Factor-Augmented VAR
Models,\textquotedblright\ Working Paper, Rutgers University and University
of Maryland.

\bibitem{} Diebold, F. X., G. D. Rudebusch, and S. B. Aruoba (2006):
\textquotedblleft The macroeconomy and the Yield Curve: A Dynamic Latent
Factor Approach,\textquotedblright\ \textit{Journal of Econometrics}, 131,
309--338.

\bibitem{} Chen, X., Q. Shao, W. B. Wu, and L. Xu (2016): \textquotedblleft
Self-normalized Cram\'{e}r-type Moderate Deviations under
Dependence,\textquotedblright\ \textit{Annals of Statistics}, 44, 1593-1617.

\bibitem{} Davidson. J. (1994): \textit{Stochastic Limit Theory: An
Introduction for Econometricians}. New York: Oxford University Press.

\bibitem{} Fan, J., Y. Liao, and M. Mincheva (2011): \textquotedblleft
High-dimensional Covariance Matrix Estimation in Approximate Factor
Models,\textquotedblright\ \textit{Annals of Statistics}, 39, 3320-3356.

\bibitem{} Fan, J., Y. Liao, and M. Mincheva (2013): \textquotedblleft Large
Covariance Estimation by Thresholding Principal Orthogonal Complements," 
\textit{Journal of the Royal Statistical Society, Series B}, 75, 603-680.

\bibitem{} Forni, M., M. Hallin, M. Lippi, and L. Reichlin (2005):
\textquotedblleft The Generalized Dynamic Factor Model, One-Sided Estimation
and Forecasting," \textit{Journal of the American Statistical Association},
100, 830-840.

\bibitem{} Freyaldenhoven, S. (2021a): \textquotedblleft Factor Models with
Local Factors - Determining the Number of Relevant
Factors,\textquotedblright\ \textit{Journal of Econometrics}, forthcoming.

\bibitem{} Freyaldenhoven, S. (2021b): \textquotedblleft Identification
through Sparsity in Factor Models: The $\ell _{1}$-Rotation
Criterion,\textquotedblright\ Working Paper, Federal Reserve Bank of
Philadelphia.

\bibitem{} Giglio, S., D. Xiu, and D. Zhang (2021): \textquotedblleft Test
Assets and Weak Factors,\textquotedblright\ Working Paper, Yale School of
Management and the Booth School of Business, University of Chicago.

\bibitem{} Goroketskii, V. V. (1977): \textquotedblleft On the Strong Mixing
Property for Linear Sequences,\textquotedblright\ \textit{Theory of
Probability and Applications}, 22, 411-413.

\bibitem{} Gospodinov, N., R. Kan, and C. Robotti (2017): \textquotedblleft
Spurious Inference in Reduced-Rank Asset Pricing Models,\textquotedblright\ 
\textit{Econometrica}, 85, 1613-1628.

\bibitem{} Gurkaynak, R.S., B. Sack, and J.H. Wright(2007):
\textquotedblleft The US Treasury Yield Curve: 1961 to the
Present,\textquotedblright \textit{Journal of Monetary Economics}, 54,
2291-2304.

\bibitem{} Harding, M. C. (2008): \textquotedblleft Explaining the Single
Factor Bias of Arbitrage Pricing Models in Finite
Samples,\textquotedblright\ \textit{Economics Letters}, 99, 85-88.

\bibitem{} Jagannathan, R. and Z. Wang (1998): \textquotedblleft An
Asymptotic Theory for Estimating Beta-Pricing Models Using Cross-Sectional
Regression,\textquotedblright\ \textit{Journal of Finance}, 53, 1285-1309.

\bibitem{} Johnstone, I. M. and D. Paul (2018): \textquotedblleft PCA in
High Dimensions: An Orientation,\textquotedblright\ \textit{Proceedings of
the IEEE}, 106, 1277-1292.

\bibitem{} Kiefer, N. M. and T. J. Vogelsang (2002): \textquotedblleft
Heteroskedasticity-Autocorrelation Robust Standard Errors Using the Bartlett
Kernel without Truncation," \textit{Econometrica}, 70, 2093-2095.

\bibitem{} Kleibergen, F. (2009): \textquotedblleft Tests of Risk Premia in
Linear Factor Models,\textquotedblright\ \textit{Journal of Econometrics},
149, 149-173.

\bibitem{} Onatski, A. (2012): \textquotedblleft Asymptotics of the
Principal Components Estimator of Large Factor Models with Weakly
Influential Factors,\textquotedblright\ \textit{Journal of Econometrics},
168, 244-258.

\bibitem{} Paul, D. (2007): \textquotedblleft Asymptotics of Sample
Eigenstructure for a Large Dimensional Spiked Covariance
Model,\textquotedblright\ \textit{Statistica Sinica}, 17, 1617-1642.

\bibitem{} Pham, T. D. and L. T. Tran (1985): \textquotedblleft Some Mixing
Properties of Time Series Models,\textquotedblright\ \textit{Stochastic
Processes and Their Applications}, 19, 297-303.

\bibitem{} Stock, J. H. and M. W. Watson (2002a): \textquotedblleft
Forecasting Using Principal Components from a Large Number of
Predictors,\textquotedblright\ \textit{Journal of the American Statistical
Association}, 97, 1167-1179.

\bibitem{} Stock, J. H. and M. W. Watson (2002b): \textquotedblleft
Macroeconomic Forecasting Using Diffusion Indexes,\textquotedblright\ 
\textit{Journal of Business and Economic Statistics}, 20, 147-162.

\bibitem{} Swanson, N.R. and W. Xiong (2018): \textquotedblleft Big Data
Analytics in Economics: What Have We Learned So Far, and Where Should We Go
From Here?\textquotedblright\ \textit{Canadian Journal of Economics}, 51,
695--746.

\bibitem{} Zhou, Z. and X. Shao (2013): \textquotedblleft Inference for
Linear Models with Dependent Errors," \textit{Journal of the Royal
Statistical Society Series B}, 75, 323-343.
\end{thebibliography}

\end{document}
