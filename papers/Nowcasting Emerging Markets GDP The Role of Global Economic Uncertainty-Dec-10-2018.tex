% Template for articles submitted to the International Journal of Forecasting
% Further instructions are available at www.ctan.org/pkg/elsarticle
% You only need to submit the pdf file, not the source files.
% If your article is accepted for publication, you will be asked for the source files.

\documentclass[11pt,3p,review,authoryear]{elsarticle}
\journal{}
\usepackage{longtable}
\usepackage{amsthm}
\usepackage{amsmath}
\usepackage{amssymb}
\usepackage{amsfonts}
\usepackage{pifont}
\usepackage{natbib}
\usepackage{geometry}
\usepackage{fleqn}
\usepackage{graphicx}
\usepackage{txfonts}
\usepackage{hyperref}
\usepackage{pdflscape}
\usepackage{pdfpages}
\usepackage{fancybox,epsf,latexsym,amssymb}
\usepackage{booktabs}
\usepackage{threeparttable}
\usepackage{caption}
\usepackage{textcomp}
\bibliographystyle{model5-names}
\biboptions{longnamesfirst}
% Please use \citet and \citep for citations.


\begin{document}

\begin{center}

{\Large Forecasting and Nowcasting Emerging Market GDP Growth Rates: The Role of Latent Global Economic Policy Uncertainty and Macroeconomic Data Surprise Factors$^{\ast }$}

\bigskip
\bigskip

{\large Oguzhan Cepni$^1$, I. Ethem Guney$^1$, and Norman R. Swanson$^2$}

{{\large $^{1}$Central Bank of the Republic of Turkey and $^{2}$Rutgers University}}

\bigskip

{\large December 2018}

\bigskip
\bigskip
\bigskip


{\large Abstract}
\end{center}

\noindent {\footnotesize In this paper, we assess the predictive content of latent economic policy uncertainty and data surprises factors for forecasting and nowcasting GDP using factor-type econometric models. Our analysis focuses on five emerging market economies, including Brazil, Indonesia, Mexico, South Africa, and Turkey; and we carry out a forecasting horse-race in which predictions from various different models are compared. These models may (or may not) contain latent uncertainty and surprise factors constructed using both local and global economic datasets. The set of models that we examine in our experiments includes both simple benchmark linear econometric models as well as dynamic factor models (DFMs) that are estimated using a variety of frequentist and Bayesian data shrinkage methods based on the least absolute shrinkage operator (LASSO). We find that the inclusion of our new uncertainty and surprise factors leads to superior predictions of GDP growth, particularly when these latent factors are constructed using Bayesian variants of the LASSO. Overall, our findings point to the importance of spillover effects from global uncertainty and data surprises, when predicting GDP growth in emerging market economies.}

\bigskip
\bigskip

\noindent \textit{Keywords:}{\normalsize \ Economic policy uncertainty, Emerging markets, Factor model, Forecasting, Lasso, Shrinkage.}

\noindent \textit{JEL Classification}{\normalsize : C53, G17.\smallskip
	\smallskip }

\noindent \_\_\_\_\_\_\_\_\_\_\_\_\_\_\_\_\_\_\_\_\_\_\_

$^{\ast }${\scriptsize \ Oguzhan Cepni (Oguzhan.Cepni@tcmb.gov.tr) and I. Ethem Guney (Ethem.Guney@tcmb.gov.tr), Central Bank of the Republic of Turkey, Anafartalar Mah. Istiklal Cad. No:10 06050 Ulus, Altındag, Ankara, Turkey. Norman R. Swanson (nswanson@econ.rutgers.edu), Department of Economics, 75 Hamilton Street, New Brunswick, NJ, 08901 USA. The authors wish to thank Hyun Hak Kim, Mingmian Cheng, Eric Ghysels, Massimiliano Marcellino, Christian Schumacher, and Xiye Yang for useful discussions related to the content of this paper on modeling and forecasting using diffusion indexes.}

\setcounter{page}{0} \thispagestyle{empty}

\newpage


\section{Introduction}
In many countries, initial real GDP estimates are released at least three weeks after the calendar  quarter to which the data pertains. For example, in the Euro area and the U.S., GDP reporting lags are  three and four weeks, respectively; and in Turkey, first release GDP data are available only after as many as 10 to 12 weeks. At the same time, tracking economic activity in real-time is crucial to the decision-making process of macroeconomic policymakers. Fortunately, there is now an abundance of both real-time and big datasets available to researchers, allowing for the construction of ever more accurate early forecasts and nowcasts (i.e., signals) of the current state of the economy. In this context, as pointed out by Giannone et al. (2008), dynamic factor models (DFMs) have become one of the workhorses for short-term forecasting, and are now widely used in central banks and research institutions for both forecasting and nowcasting. For further discussion, see, Artis et al. (2005) - for the UK; Schumacher (2010) - for Germany; Bessec (2013) - for France, Girardi et al. (2015) - the Euro Area; Modugno et al. (2016) - Turkey; Bragoli (2017) - Japan; Kim and Swanson (2018a) - for the USA, Kim and Swanson (2018b) - for Korea; Luciani et al. (2018) - for Indonesia; and Bragoli and Fosten (2018) - for India).

In this paper, we contribute to the literature on nowcasting and forecasting real GDP growth in emerging economies by empirically assessing the importance of economic policy uncertainty and data surprises in factor-type econometric forecasting models for five emerging market economies, including Brazil, Indonesia, Mexico, South Africa, and Turkey. Our analysis centers around the use of DFMs for constructing GDP predictions, although we also evaluate benchmark linear autoregressive (AR) models. Importantly, our DFMs are specified both with and without uncertainty and data surprise factors constructed using both local and global economic variables. Moreover, in addition to standard econometric estimation methods, we estimate factors using a variety of data shrinkage methods including the standard LASSO, the adaptive LASSO, the Bayesian LASSO, and the Bayesian adaptive LASSO.

As mentioned above, we utilize both local and global datasets when constructing the factors used in our prediction models. Although there are many empirical papers that focus on using only local macroeconomic data in the context of GDP forecasting, the importance of uncertainty regarding policymakers' decisions on international economic policies has received increasing attention since the beginning of 2018. For example, concerns over US-China trade tensions, Brexit negotiations with the EU, Italy's' fiscal planning, and how the Federal Reserve Board of the USA will determine the timing and pace of policy normalization all weigh heavily on the global economy. These sorts of international spillovers are particularly important for emerging markets, in particular those with high foreign portfolio ownership and weak macroeconomic balance sheets. For this reason, it is crucial to assess the relevance of uncertainty and data surprises in the context of forecasting emerging market GDP. Needless to say, the impact of uncertainty on economic activity has received considerable attention in the economics literature in recent years (see, e.g. Bloom (2014)). For example, Bontempi et al. (2016) propose uncertainty indicators based on Google Trends. Their results suggest that online search data can provide early signals of uncertainty, and can be used in macroeconomic forecasting. Baker et al. (2016) construct an index of economic policy uncertainty (EPU) based on newspaper coverage frequency. They find that economic policy innovations foreshadow declines in investment, output, and employment, using a panel vector autoregressive (VAR) model for 12 major economies. Thorsrud (2018) develops a new coincident index of business cycle activity based on quarterly GDP and textual information contained in a daily business newspaper.

In addition to the above proxies for economic policy ``uncertainty'', a growing strand of the literature uses consensus forecasts to disentangle macroeconomic uncertainty from more ``general'' uncertainty. In particular, it is argued in this literature that professional forecasters (e.g. those forecasters contributing to the Survey of Professional Forecasters in the USA) closely monitor macroeconomic data and often base their predictions on sophisticated econometric models. Thus, departures of their (consensus) predictions from actual realizations can be viewed as data ``surprises'', which are themselves measures of macroeconomic uncertainty. There are different proxies for this sort of macroeconomic uncertainty that are proposed in the empirical literature. For example, Rossi and Sekhposyan (2015) construct a macroeconomic uncertainty index based on comparing the realized forecast error of a variable of interest with the sample distribution of the forecast errors of that variable. If the realization is in the tail of the distribution, they conclude that the macroeconomic environment is more uncertain. Carriero et al. (2016) develop a model to identify uncertainty by modeling the common component of the volatility of the forecast errors of a large set of macroeconomic and financial variables. Finally, Scotti (2016) proposes a macroeconomic surprise index that exploits the difference between actual releases of data and Bloomberg forecasts to capture economic agents' expectations about the state of the economy. In this paper, we construct global economic policy uncertainty and surprise indices based on a variety of different local and global datasets. More specifically, we incorporate uncertainty into our prediction models in three different ways. First, as our benchmark we utilize only local macroeconomic data. Using this approach, we estimate both DFMs and simple AR models; but do not explicitly include any uncertainty or surprise indexes. Second, we augment our DFMs with ``surprise'' indices constructed using professional forecasters' expectations. Finally, we additionally augment our DFMs with factors extracted from a wide variety of uncertainty indices of economic policy, trade policy, monetary policy, and migration.

It should be noted that we focus on the prediction of GDP growth in emerging markets (EM) for two main reasons. First, official releases of EM GDP figures are subject to significant publication lags and data revision, as discussed above. Second, again as discussed above, it is likely that the effects of uncertainty on economic activity are particularly significant in environments characterized by large budget deficits, high current account/GDP ratios, and high external funding needs. As a case in point, note that Carriere et al. (2013) investigate the effects of an uncertainty shock from the USA on developed and developing countries. They find that emerging markets suffer much more severe falls in investment and private consumption when there are credit constraints that lead to increases in the uncertainty index. Gauvin et al. (2014) also point out that elevated policy uncertainty in advanced countries may lead to an increase in capital outflows from emerging markets, because of rising global risk aversion. At the same time, it is important to note that low-quality emerging market data presents distinct challenges when forecasting GDP growth. However, our dataset contains a very large number of variables, allowing us to mitigate this effect to some degree by ``pre-selection'' of key variables for use in our DFMs. Namely, we argue that a small set indicators from our dataset may be sufficient to identify the most informative uncertainty indexes. Indeed, the idea that a small set of indicators, when chosen appropriately, can improve forecasting performance of factor models is supported by evidence presented in various papers, including Boivin and Ng (2006), Schumacher (2007), Bai and Ng (2008), Banbura and R\"{u}nstler (2011), Kim and Swanson (2014, 2018a), Bulligan et al. (2015), and references cited therein. For this reason, we utilize variable selection methods to pre-select indicators before the construction of factors. To this end, we apply a number of LASSO methods for data shrinkage, as discussed above. Of course, we also use standard DFM methods where the entire dataset is used in factor construction. In this way, our analysis adds to our understanding not only of GDP forecasting using uncertainty indices, but also to the usefulness of shrinkage methods in DFM modeling.

Our empirical findings can be summarized as follows. First, as expected, there is a substantial reduction in mean square forecast error (MSFE), as more data related to the current quarter become available and is incorporated into our models. Thus, the forecast accuracy of our DFM models generally increases by incorporating the latest information. Second, uncertainty indexes are quite useful for predicting GDP growth in emerging market economies. More specifically, benchmark AR models, as well as dynamic factor models that utilize only local macroeconomic data yield inferior predictions for Indonesia, South Africa, and Turkey. For these three countries, both ``surprise'' and ``uncertainty'' indexes matter. Third, constructing surprise and uncertainty indexes using LASSO shrinkage methods leads to more accurate forecasts than when factors are constructed without variable pre-selection. Indeed, across all ten forecast horizons (including forecasts, nowcasts, and backcasts), and across all five countries the ``globally best'' models include factors constructed using targeted variable pre-selection in 47 of the 50 cases.\footnote{A globally best model is one that yields the lowest MSFE across all five data selection methods including (i) Use all data in DFM construction; (ii) use the standard LASSO for dataset reduction; (iii) use the adaptive LASSO for dataset reduction; (iv) use the Bayesian LASSO for dataset reduction; and (v) use the Bayesian adaptive LASSO for dataset reduction.} Finally, augmenting the DFMs utilized in our analysis regression with lags of the dependent variable (i.e., including an AR component in the DFM) yields forecasting gains relative to models without AR terms only for Indonesia and South Africa. This result is interesting given the preponderance of evidence in the time series forecasting literature concerning the importance of including AR components when forecasting economic variables, and serves to underscore the importance of uncertainty and surprise variables when predicting EM GDP.

Summarizing, we find strong evidence that using targeted predictors in combination with global uncertainty indices as well as global surprise indices based on expectations of professional forecasters leads to more precise GDP predictions for emerging market economies. These findings suggest that when assessing macroeconomic conditions, policymakers and central bankers may increasingly need to take into account the level of economic policy uncertainty originating from other countries, as well as related macroeconomic forecasts formulated by professional forecasters.

The rest of the paper is structured as follows. In Section 2, we present the main features of the datasets used in our empirical investigation. In Section 3, we outline the econometric methodology used in the papers. In Section 4, we discuss our empirical findings. Concluding remarks are gathered in Section 5. Finally, the Appendices include supplementary tables as well as additional details describing the datasets analyzed in the sequel.

\section{Data}


We analyze a relatively large set of economic indicators consists of 97, 87, 116, 109, 102 economic variables for Brazil, Indonesia, Mexico, South Africa, and Turkey, respectively. The dataset is composed of both ``hard indicators'' and survey data. Among the hard indicators, we have both supply-side variables, such as industrial production indexes, and demand-side variables, such as electricity consumption. Among the survey variables, we have the Market PMI survey, one of the most watched indicators of the business cycle. Given the sensitivity of EM economies to external conditions, we also include current account balance and volume indices of exports and imports, as well as real effective exchange rates. The dataset can be divided into six categories: \emph{Housing and Order Variables}: House price index, real estate units sold and new orders. \emph{Labor Market Variables}: Employment and unemployment. \emph{Prices}: Producers prices and consumer prices. \emph{Financial Variables}: Treasury bond yields, credit default swaps, exchange rates, and stock prices. \emph{Money, Credit and Quantity Aggregates}: Money supply, commercial bank loans, time and sight deposits. \emph{Real Economic Activity}: PMI survey, industrial production, retail sales, vehicle production and capacity utilization. In general, the survey variables and nominal indicators are released during the reference month (i.e., the calendar month to which the data pertains), whereas real and labor variables are announced with a publication lags of 1-3 months.

In addition to the above collection of macroeconomic and financial indicators, which are used in our baseline DFM model (this DFM model is called Specification 1 in Section 3.3), we examine an uncertainty dataset that contains a wide variety of uncertainty indices pertaining to economic policy, trade policy, monetary policy and migration. These variables are largely the same as those constructed by Baker et al. (2016) for major economies and are called economic policy uncertainty (EPU) indices. Their data construction approach is based on computing the proportion of newspaper articles referring to a specific type of uncertainty over a given period. In particular, the EPU indices reflect the frequency of articles that include terms related to three categories, i.e., economy (E), policy (P) and uncertainty (U)\footnote{More details on the EPU indices can be found at http://www.policyuncertainty.com/index.html.}. In total, we utilize 45 ``uncertainty'' indices from various fully and less developed countries when constructing our ``uncertainty'' factors for use in our DFM forecasting models (see Specifications 2 and 4 in Section 3.3).

We now turn to a discussion of our ``surprise'' indices. Understanding how economic data evolve, relative to consensus expectations, is important for gauging potential shifts in macroeconomic sentiment, for a given country. For this purpose, we collect a large set of surprise indices across key regions and countries. Our dataset consists of Citi and Goldman Sachs surprise indices, which are constructed based on similar methodologies. These indices are designed to summarize the degree of surprise inherent in a data release, relative to the associated Bloomberg median forecast of the variable in question. Multiplying the so-called relevance score for a variable by the surprise score allows us to track whether the economic data in a country are ``outperforming'' or ``under-performing'' consensus expectations. Since units of measurement vary across variables, the surprise scores are normalized by dividing by sample standard deviation of the corresponding variables.\footnote{Note that relevance scores are defined as the absolute value of the contemporaneous correlation between each variable and real GDP growth.} In total, we have 74 surprise indices across different regions and countries, including the USA, UK, Euro Area, Japan, Asia Pacific region, and Latin America. These indices are then used to construct our ``surprise'' factors for use in our DFM forecasting models (see Specifications 3 and 4 in Section 3.3).

All datasets were collected from Bloomberg, and are for the period January 2003 - June 2018. The complete list of Blomberg tickers is provided in Tables B1-B7 in Appendix B. Finally, all series are transformed to stationarity by differencing or log-differencing, as needed.


\section{ Econometric Methodology}
\subsection{Dynamic Factor Models}

In our analysis, we separately extract potentially useful forecasting information from our three datasets (i.e., our macroeconomic indicator dataset, surprise dataset, and uncertainty index dataset). To do this, we employ the widely used dynamic factor model (DFM) of Giannone et al. (2008). In this framework, the dynamics of individual variables is represented as the sum of a component that is common to all variables in the economy and an orthogonal idiosyncratic residual. Formally, the DFM can be written as a system with two types of equation: a measurement equation (Eq. (1)) that links the observed variables to the unobserved common factor to be estimated, and the transition equations ((Eq. (2) and Eq.(3)) that describe the dynamics of the common factors and the residuals of the measurement equation. Once Eqs. (1)-(3) are written in state space form, the Kalman filter and smoother are applied to extract common factors and generate forecasts for all of the variables in the model.

More specifically, we consider a panel of observable economic variables, $ X_{i,t} $, where $i$ indicates the cross-section unit, $ i= 1,....,N$, and $t$ denotes the time index, $t= 1,....,T$. Each variable in the dataset can be decomposed into common part and idiosyncratic components, where the common component capture comovement in the data, and is driven by a small number of shocks. The DFM model can be written as:

\begin{equation}
X_t =\Lambda F_t +\xi_t,      \qquad \qquad\qquad           \xi_t \sim N(0,\Sigma_e),
\end{equation}

\begin{equation}
F_t=\sum\limits_{i=1}^k\Psi_i F_{t-i} +u_t,     \qquad \qquad u_t   \sim N(0,Q),
\end{equation}

\noindent where $ F_t $  is an $ r\times1 $ vector of unobserved common factors with zero mean and unit variance, $ \Lambda $ is a corresponding $ N\times r $ factor loading matrix, and the idiosyncratic disturbance,  $\xi_t$, is uncorrelated with $ F_t $ at all leads and lags, and has a diagonal covariance matrix $\Sigma_e$. It is assumed that the common factors, $ F_t $,   follow a stationary VAR(p) process driven by the common shocks, $ u_t \sim N(0,Q)$, and that the $ \Psi_i $ are  $ r\times r $ matrices of autoregressive coefficients. Also, the common shocks ($u_t$)  and idiosyncratic components ($\xi_t$) are orthogonal. To handle missing observations at the end of the sample due to the non-synchronous flow of data, we characterize the variance of the idiosyncratic component as an extremely large variance. In this way, we ensure that the Kalman filter will put no weight on missing observations in the extraction of the common factors. Finally, in order to construct forecasts of our quarterly GDP target series,
say $ y_t $, in our monthly DFM framework, we express each quarterly variable in terms of a partially observed monthly counterpart following the approach of Mariano and Murasawa (2003). Put differently, the general form of the forecasting model can be described as follows:

\begin{equation}
y_t= 	\mu+ \beta'F_t+\varepsilon_t,  \qquad \qquad \varepsilon_t    \sim N(0,\sigma^{2}_{\varepsilon}),
\end{equation}

\noindent where $\varepsilon$ is a stochastic disturbance term. In order to select the optimal number of factors (r), one may use various methods suggested in the the literature. A non-exhaustive list of possible methods are discussed in the following papers: Bai and Ng (2002), Onatski (2009), Alessi et al. (2010), and Ahn and Horenstein (2013). Although the Bai and Ng (2002) criteria is frequently used in the empirical literature, we find that it generally chooses too many factors, resulting in deterioration in forecast accuracy. Hence, we adopt the Onastki (2009) approach that is based on testing the null of
$r-1 $ common factors against the alternative of $r$ common factors. Optimal lag length is selected using the Schwarz information criterion (SIC)\footnote{The list of selected r and p pairs are (2,2), (1,2), (1,3), (3,3), (1,2) for Brazil, Indonesia, Mexico, South Africa, and Turkey, respectively.}. In summary, we find that simple model specifications, with one or two factors and one or two lags often yield best out of sample forecast performance. Specification of a parsimonious one or two factor model is also consistent with the literature on factor models, which has shown that heavily parametrized models with many factors usually lead to poor forecasting performance (see Forni et al. (2000), Stock and Watson (2002), and Bragoli (2017)).

\subsection{Identifying targeted predictors: LASSO - based approaches}

Since dynamic factor models do not explicitly incorporate knowledge of the target variable being forecast, factor extraction is called ``un-targeted''. Needless to say, targeted forecasting, in which factors with information specific to the target variable are constructed and used in subsequent model specification may yield superior predictions, relative to predictions constructed using factors based on un-targeted methods. These issues are discussed at length in Boivin and Ng (2006), Bai and Ng (2008), Schumacher (2010), Caggiano et al. (2011),  Medeiros and Vasconcelos (2016), Kapetanios et al. (2016), Kim and Swanson (2014, 2018b), and many others. In the sequel, we utilize both un-targeted and targeted methods to construct factors. For un-targeted forecasting, we simply use the DFM model outlined in the previous section. For targeted forecasting, we note that one of the most commonly used variable selection and parameter shrinkage method is the LASSO. This method can be thought of as a type of penalized regression, related to classical ridge regression. However, in the case of ridge regression, an $\ell_2 $-norm penalty is imposed, while under the LASSO, an $\ell_1 $-norm penalty function is added to the usual least squares minimization problem. Interestingly, the $\ell_1 $-norm penalty function induces shrinkage to 0 of coefficients associated with ``irrelevant'' variables.\footnote{As an example of the coefficients referred to in the previous statement, consider coefficients that are defined to be the weights in a linear combination of variables forming a latent factor, when principal component analysis (PCA) is used to construct $F_t$. When using the LASSO, various of these factor loadings may be identically zero. On the other hand, the classical ridge regression penalty function results in all coefficients being nonzero, for each latent factor, under PCA.} For our purposes, since we are interested in selecting a targeted subset of the original variables in our dataset when constructing shrinkage type factors, we use the LASSO. In particular, we analyze factors constructed using the standard LASSO, the adaptive LASSO, the Bayesian LASSO and the Bayesian adaptive LASSO.

Turning again to our empirical setup, we consider a panel of observable economic variables, $ X_{i,t} $, where $i$ indicates the cross-section unit, $ i= 1,...,N$, and $t$ denotes the time index, $ i= 1,...,T$, as discussed above. Following the notation of Hastie et al. (2009), we consider the problem of selecting a subset of $ X$, where $ X $ is a $ T\times N $ matrix to be used for forecasting quarterly GDP growth, say $ Y $, for $ i= 1,...,T$.

\subsubsection{Least absolute shrinkage operator (LASSO)}

We implement the LASSO, as developed in Tibshirani (1996). This shrinkage operator performs both variable selection and regularization, using a regularization parameter, $\lambda$. The idea is to impose an $\ell_1 $-norm penalty on the regression coefficients, thus allowing for cases where $N>T$. This penalty also results in (possible) shrinkage of coefficients (called $\hat{\beta}^{LASSO}$) to zero, as discussed above. The LASSO estimator is defined as:

\begin{equation}
\begin{aligned}
\hat{\beta}^{LASSO} = & \underset{\beta}{\text{arg min}}
& &  \lVert Y-X\beta \rVert_2  +\lambda \sum\limits_{j=1}^N |\beta_j|,
\end{aligned}
\end{equation}
where $ \lambda $ is a tuning (regularization) parameter that controls the strength of the  $\ell_1 $-norm penalty. We choose the tuning parameter $ \lambda $ via cross-validation, which is a data-driven method that is designed to maximize expected out of sample forecasting performance. Since the objective function in the LASSO is not differentiable, numerical optimization must be used when constructing $\hat{\beta}^{LASSO}$. For example, an efficient iterative algorithm called the ``shooting algorithm" is proposed in Fu (1998). One of the limitations of the LASSO approach is that the sample size bounds the number of selected variables. For example, if $N > T$,  the LASSO yields at most $T$ non-zero coefficients (see Swanson (2016) for further discussion). The variables associated with these non-zero coefficients constitute our set of targeted predictors when using the LASSO method.

\subsubsection{Adaptive LASSO (AdaLASSO)}

Although the LASSO can perform automatic variable selection because of the $\ell_1 $-norm penalty, it may yield biased estimates when coefficients are large. Fan and Li (2001) conjecture that LASSO does not have oracle properties and may yield an inconsistent set of selected variables in some cases. In order to address these issues, Zou (2006) introduces a new version of the LASSO, called the adaptive LASSO, where adaptive weights are used for penalizing different coefficients in the $\ell_1 $-norm penalty. They show that adaptive LASSO enjoys oracle properties. Moreover, Medeiros and Mendes (2016) show that the adaptive LASSO estimator maintains its' consistency, under very general conditions; and performs well even when the number of variables increases faster than the number of observations.

The adaptive LASSO estimator can be written as follows:
\begin{equation}
\begin{aligned}
\hat{\beta}^{AdaLASSO} = & \underset{\beta}{\text{arg min}}
& &  \lVert Y-X\beta \rVert_2  +\lambda \sum\limits_{j=1}^N w_j|\beta_j|,
\end{aligned}
\end{equation}

\noindent where $ w_j=\hat{|{\beta_j}^*|}^{-\tau}$ represents different weights on the penalization of each variable, $\hat{|{\beta_j}^*|}$ is an initial estimator, such as the OLS or ridge estimator (which is estimated in a first step), and ${\tau} > 0$ controls the difference in weights. Since the adaptive LASSO is still an $\ell_1 $-norm penalization method, the algorithms for solving the LASSO can be employed for constructing adaptive LASSO estimates.

\subsubsection{Bayesian LASSO (BLASSO)}
Tibshirani (1996) suggested that LASSO estimates can be interpreted as the posterior mode of the Bayes estimate, under the Laplace priors. While the LASSO attributes a value of exactly zero to regression coefficients of irrelevant variables, redundant Bayesian LASSO coefficient modes are not precisely zero. Instead, the BLASSO provides a posterior distribution of coefficients. In our BLASSO estimator, we implement a conditional Laplace prior of the form:

\begin{equation}
\begin{aligned}
\pi\left( \beta |\sigma^{2}\right)= \prod ^{p}_{j=1} \dfrac{\lambda}{2\sqrt{\sigma^{2}}}  e^{\dfrac{-\lambda |\beta_j|} {\sqrt{\sigma^{2}}}}.
\end{aligned}
\end{equation}
%where the non-informative scale-invariant marginal prior $\pi\left( \sigma^{2}\right)= 1 / {\sqrt{\sigma^{2}}} $.

Park and Casella (2008) show that conditioning on error variance, ${\sigma^{2}} $, ensures a unimodal full posterior. Otherwise, expensive simulation methods lead to slow convergence of the Gibbs sampler and result in less meaningful point estimates. Since the Laplace distribution can be represented as a scale mixture of normal densities with an exponential mixing density, Park and Casella (2008) propose the following hierarchical BLASSO model:
\begin{equation}
\begin{aligned}
y | X,\beta, \sigma^{2} \sim N_n(X\beta,\sigma^{2}I_n) \\
\beta |\sigma^{2}, \{{ \tau_1}^2,...,{\tau_p}^2\} \sim N_p(0_p,\sigma^{2}D_{\tau}) \\
D_{\tau}=diag \{{ \tau_1}^2,...,{\tau_p}^2\} \\
\end{aligned}
\end{equation}
with the following priors on $\sigma^{2}$ and $\tau=\{{ \tau_1}^2,...,{\tau_p}^2\}$:
\begin{equation}
\sigma^{2}, { \tau_1}^2,...,{\tau_p}^2 \sim \pi\left( \sigma^{2}\right)d\sigma^{2} \prod ^{p}_{j=1}\dfrac{\lambda^{2}}{2} e^{\dfrac{-\lambda^{2} \tau_j^{2}}{2}} d\tau_j^{2}, \\
\sigma^{2}, { \tau_1}^2,...,{\tau_p}^2 >0,
\end{equation}
\noindent where $D_{\tau}$ is the prior covariance matrix, and $\lambda$ is a rate parameter of the exponential distribution. Korobilis (2013) compares the forecasting performance of hierarchical Bayesian shrinkage and factor models, and finds that Bayesian shrinkage serves as a valuable addition to existing methods, in the presence of many predictor variables. Our approach is to use this BLASSO method and store the posterior distributions of all coefficients. Thereafter we calculate the  $ 95\%$ confidence intervals, and set a coefficient equal to zero if its interval includes zero. The remaining variables constitute our set of targeted predictors, when using the Bayesian LASSO.

\subsubsection{Bayesian Adaptive LASSO (BaLASSO)}
As discussed in Section 3.2.2, the AdaLASSO uses weighted shrinkage for consistent estimation of regression coefficients, while retaining the attractive convexity property of the LASSO. However, the AdaLASSO requires consistent and informative initial estimates of the regression coefficients, which are generally not available when the number of regressors is larger than the number of observations. Since Bayesian Adaptive LASSO does not require any informative initial estimates of the regression coefficients, this motivates us to replace equation (8) in the Bayesian LASSO section to allow for a more adaptive penalty, as follows:

\begin{equation}
\sigma^{2}, { \tau_1}^2,...,{\tau_p}^2 \sim \pi\left( \sigma^{2}\right)d\sigma^{2} \prod ^{p}_{j=1}\dfrac{\lambda_j^{2}}{2} e^{\dfrac{-\lambda_j^{2} \tau_j^{2}}{2}} d\tau_j^{2} \\
\end{equation}

Similar to the BLASSO, we select variables such that their corresponding  $ 95 \%$ confidence intervals do not include zero.

\subsection{Factor Augmented Prediction Models}

To evaluate the forecasting performance of dynamic factor models based on different factor specification types, we run a recursive pseudo out of sample forecasting exercise over the period July 2008 to June 2018. For each reference quarter, we produce a sequence of ten forecasts, starting with the forecast based on the information available in the first month of the two previous quarters, and stopping on the first of the month of the subsequent quarter, before GDP is released. Put differently, we construct three monthly forecasts (for quarterly forecast horizon, $h=1,2$), three monthly nowcasts (for quarterly forecast horizon, $h=0$), and one monthly backcast (for quarterly forecast horizon, $h=-1$).

We apply the dynamic factor model to extract the leading common factors in our large set of uncertainty and surprise indices. We called these new factors our ``Global Economic Policy Uncertainty" (GEPU) and ``Global Macro-Surprise'' (GMS) factors. Earlier, we referred to these variables as our ``uncertainty'' and ``surprise'' variables, respectively. These factors can be interpreted as measures of common variation in economic policy uncertainty (uncertainty) and macroeconomic data uncertainty (surprise) across countries. These variables are individually "added" to our DFM model. In particular, noting that ``Local'' refers to factors that are constructed using only ``own'' country variables, we construct predictions using the following specifications, where Specification 1 is the model used in Giannone et al. (2008), and Specifications 2-4 are extensions that incorporate our new uncertainty and surprise factors.\footnote{In addition to estimating Specifications 1-4 using the methods of Section 3.1, variations are estimated using the shrinkage approaches discussed in Sections 3.2.1-3.2.4.}
\begin{itemize}
  \item \textbf{Specification 1}: Local factor model\\
  $ y_{t+h}= \mu+ \beta'F_t^{Local}+\varepsilon_{t+h} $
  \item \textbf{Specification 2}: Uncertainty factor model \\
   $ y_{t+h}= \mu+ \beta'F_t^{Local}+ \vartheta'F_t^{GEPU}+\varepsilon_{t+h} $
  \item \textbf{Specification 3}: Macro-Surprise factor model\\
    $ y_{t+h}= 	\mu+ \beta'F_t^{Local}+\theta'F_t^{GMS}+\varepsilon_{t+h} $
  \item \textbf{Specification 4}: Global factor model\\
   $ y_{t+h}= 	\mu+ \beta'F_t^{Local}+\theta'F_t^{GEPU}+\delta'F_t^{GMS}+\varepsilon_{t+h} $
\end{itemize}

An additional set of models is also used to construct predictions, in which each of the above DFM models is augmented with lags of the dependent variable (where the lag number is selected using the SIC). Finally, we also construct forecasts using a straw-man AR model, with lags selected using the SIC. We assess the precision of the different sequences of forecasts using mean square forecast error (MSFE), which is measured as the average of the squared differences between predicted and actual GDP values for the ten year period from July 2008 to June 2018. In order to assess the statistical significance of differences in MSFE across specification types, pairwise Diebold-Mariano (DM: 1995) tests of equal predictive accuracy are implemented. In this test, the null hypothesis that of equal predictive accuracy. For a discussion of the test, which is normally distributed under non-nestedness, and has a nonstandard distribution if models are nested, see Kim and Swanson (2018a,b). When carrying out DM tests, the benchmark against which each of our Specification Type 1-4 models are compared with our straw-man AR model.

\section{Empirical Results}

\subsection{Global Economic Policy Uncertainty and Macroeconomic Surprise Factors}

Before describing our prediction results, it is of interest to investigate how global economic policy uncertainty and macroeconomic surprise factors are related to macroeconomic fluctuations across our EM economies. As seen in Figure 1, the global uncertainty factor captures the crucial events and spikes near elections in the USA, effects likely associated with Brexit, global financial crises, and major uncertainty surrounding fiscal policy in the USA. Given the pivotal role of uncertainty on world trade volume, spill-overs from elevated global uncertainty to emerging markets likely affect economic activity through a number of channels. For example, the uncertainty about whether the USA will place tariffs on steel or other goods that are imported from China is likely to undermine export oriented investments that are allocated to these sectors. This in turn puts pressure on Chinese GDP growth. Also, in the era of Brexit, we have seen that many firms have stopped hiring and are restricting production until the fallout from Brexit becomes more clear. This has a negative impact on GDP growth in countries with high exports to the UK. These sorts of linkages are precisely what we are trying to capture with our uncertainty and surprise factors.

As shown in Figure 2, the global macroeconomic surprise factor tracks the world GDP growth quite closely and captures the sharp contraction in economic activity over the crisis period around 2008. While we might indeed expect markets to move in response to the ``surprise'' factor, it is noteworthy that the global macroeconomic surprise factor fails to track world GDP growth rates since the beginning of 2017. One reason might be that over this period, post-US election, economic agents' expectations were more pessimistic about the economy than warranted, given elevated global uncertainty in this period.

The below table reports correlation coefficients between the global economic uncertainty factor (called ``Uncertainty''), the global macroeconomic surprise factor (called ``Surprise'') and country-specific (local) factors. The correlation coefficients between ``Uncertainty'' and country-specific factors range from -0.17 to -0.57, showing that the global economic uncertainty factor and economic activity is negatively correlated, except for Indonesia. A reason for this might be that an increase in uncertainty regarding economic policy is likely to trigger recessionary effects, including delays in investment and increases in unemployment. Furthermore, Carriere et al. (2013) show that the impact of exogenous uncertainty shocks on emerging economies is to increase the burden of foreign currency denominated debt, due to credit constraints. This may also have an adverse effect on economic growth. On the other hand, the positive relation between ``Surprise'' and country-specific factors shows that, in general, the macroeconomic data and market expectations move synchronously, as expected. Finally, note that country-specific factors are positively correlated, across countries, likely because of trade linkages.

{\centering  Correlation Coefficients Between Local and Global Factors

}

{	
	\centering
	\scriptsize
	
	\begin{tabular}{lrrrrrrrr}\hline
		& \multicolumn{1}{c}{\textit{Brazil}} & \multicolumn{1}{c}{\textit{Indonesia}} & \multicolumn{1}{c}{\textit{Mexico}} & \multicolumn{1}{c}{\textit{S.Africa}} & \multicolumn{1}{c}{\textit{Turkey}} & \multicolumn{1}{c}{\textit{Surprise}}& \multicolumn{1}{c}{\textit{Uncertainty}}\\\hline
		\textit{Brazil} & 1 & 0.52& 0.34 & 0.67 & 0.38&0.48&-0.27 \\
		\textit{Indonesia} & 0.52 & 1 & 0.18 & 0.32 & -0.02& 0.04&0.14\\
		\textit{Mexico} & 0.34 & 0.18 & 1 & 0.63 & 0.70&0.40&-0.17 \\
		\textit{S.Africa} & 0.67 & 0.32 & 0.63 & 1 & 0.51 &0.31&-0.57\\
		\textit{Turkey} & 0.38 & -0.02 & 0.70 & 0.51 & 1&0.51&-0.20 \\
		\textit{Surprise} & 0.43 & 0.04 & 0.40 & 0.31 & 0.51 &1&-0.25\\
        \textit{Uncertainty} & -0.27 & 0.14 & -0.17 & -0.57 & -0.20&-0.25 &1\\\hline
	\end{tabular}%

}

\bigskip
In order to provide insight into the evolution of country-specific factors (local factors), Figure 3 plots GDP growth against estimated common factors. As is evident from inspection of the plots in this figure, local common factors track GDP growth quite well; and also capture the GDP dynamics in these countries during the global financial crisis.


\subsection{Forecasting Experiment Results}

Tables 1-5 summarize the results of our prediction experiments, for each of Brazil, Indonesia, Mexico, South Africa, and Turkey, respectively. As discussed above, there are eight main DFM varieties in our experiment, including Specifications 1-4 (called ``Local'', ``Uncertainty'', ``Surprise'', and ``Global'', respectively, in the tables), as well as Specifications 1-4 with AR terms (called ``Local-AR'', ``Uncertainty-AR,'' ``Surprise-AR'', and ``Global-AR'', respectively). Additionally, each of these eight DFM models is estimated using five different types of shrinkage, including ``All Sample'', in which all country-specific variables are used in the DFM specification, ``LASSO", ``AdaLASSO'', ``Bayesian LASSO'', and ``Bayesian AdaLASSO''. For complete details, refer to Section 3. In each of the tables, entries in the first row correspond to MSFEs associated with forecasts constructed using our straw-man AR model. All other entries in the tables are relative MSFEs, where the numeraire is MSFE of the AR model. Thus, entries that are less than unity indicate point MSFEs that are lower than that of the AR model. For each of two quarterly $h$-step ahead forecast horizons, (under the headers ``$h=1$'' and ``$h=2$''), MSFEs from three monthly forecasts (denoted as months ``1'', ``2'', and ``3'') are reported. Results are also reported for three monthly nowcasts, under the header ``$h=0$'', and one monthly backcast (under the header ``$h=-1$''). For each country, entries denoted in bold indicate the MSFE-``best" model, across all specification types, for a given forecast horizon and shrinkage method. Additionally, for each country, entries denoted in bold and subscripted with "GB" (for ``globally-best") indicate the MSFE-best models across all specification types and shrinkage methods, for a given forecast horizon.

The results in Table 1-5 reveal various interesting insights. First, there is a substantial reduction in MSFE as one moves from left to right along each row in the tables (i.e., as more data related to the current quarter become available). This increase in forecast accuracy is as expected, and indicates that the DFM models are able to correctly revise GDP prediction by effectively exploiting the flow of monthly data releases (see, Giannone et al. (2008), Banbura and Runstler (2011) and Li and Chen (2014) for further discussion).

Second, virtually all of the entries in Tables 1-5 are less than unity, indicating that the DFM model generally produces smaller point MSFEs than our straw-man or benchmark AR model. Additionally, noting that starred entries indicate rejection of the null of equal predictive accuracy (when comparing a given model against the AR benchmark), it is apparent upon inspection of the tabulated results that many of these MSFEs are significantly lower than that of the AR model.

Third, recall that there are ten forecast horizons and five shrinkage methods so that there are a total of 50 specification types for each country. Our results indicate that there are notable decreases in MSFE for a number of countries when ``uncertainty'' and ``surprise'' factors extracted from our uncertainty and surprise index datasets are included in the DFM.{\footnote{Recall that entries labeled ``Uncertainty'' in the tables correspond to DFMs with ``uncertainty'' factors (i.e. Specification 2), entries labeled ``Surprise'' correspond to DFMs with ``surprise'' factors (i.e. Specification 3), and entries labeled ``Global'' correspond to DFMs with ``uncertainty'' and ``surprise'' factors (i.e. Specification 4).} Thus, latent factors that capture uncertainty and macro data surprise appear to contain significant marginal predictive content for country-specific growth prediction. For example, note that in Table 1 (i.e., the case of Brazil) the inclusion of factors extracted from uncertainty and/or surprise index datasets results in the globally MSFE-best models for 4 of 10 forecast horizons, across all shrinkage methods. For Indonesia (see Table 2), we see that for 7 of 10 forecast horizons, the inclusion of uncertainty and/or surprise factors results in the globally MSFE-best models. For Mexico, South Africa, and Turkey, the number of analogous ``wins'' for our global factors are 3 of 10, 10 of 10, and 9 of 10, respectively. Overall, thus globally MSFE-best models include our global uncertainty and surprise factors in 33 of 50 cases, across all five countries. In summary, there is strong evidence that global factors play a key role in GDP growth for Indonesia, South Africa, and Turkey; while the evidence is mixed for Brazil and Mexico.
	
Fourth, if one digs more deeply into the findings concerning Brazil, a further pattern emerges. In particular, note that for Brazil, 3 of the 4 ``wins'' discussed above occur when forecasting, while only 1 of 4 ``wins'' occur when nowcasting or backcasting. Thus, for this country, global information appears to lose its predictive content, relative to local information, as the calendar date of available monthly data becomes closer to (or surpasses) the calendar date of the GDP value being predicted. For Mexico, the story is different, as local information remains the most useful, regardless of forecast horizon. One reason for this might be that our global factors do not adequately account for shocks that are important to Mexico. This may in turn be due, in part, to the fact that Mexico has approximately the 15th largest economy in the world, in nominal terms, and exhibits a surprising level of macroeconomic stability. Moreover, Mexico has a largely export oriented economy with manufactured products accounting for approximately 90$\%$ of all exports.

In order to determine whether the above findings are apparent upon visual inspection of DFM predictions, we have plotted nowcasts against actual GDP growth for the 5 countries (see Figure 4). In this figure, the ``Actual GDP'' plot is of actual GDP growth rates, ``AR'' corresponds to nowcasts made with our AR model using all available data (i.e., no shrinkage), and ``Local corresponds to DFM nowcasts made using all available data. Finally, for each country, we also include a ``Local'', ``Surprise'', or ``Global'' plot, corresponding to nowcasts made with the MSFE-best DFM model that includes ``uncertainty'' and/or ``surprise'' factors. Examination of these plots indicates that specification types based on uncertainty and macroeconomic data surprises tend to predict turning points relatively well, and outperform the ``Local'' factor model in volatile episodes, particularly for Brazil, South Africa, and Turkey. However, broad rankings of the variety available by summarizing the data in Tables 1-5 cannot be made by inspecting the plots in Figure 4. For this reason, we provide an alternative summary of the information given in Tables 1-5. This summary is given in Table A1 in Appendix A. In this table, the number of model ``wins'' is tabulated by model specification type, across all forecast horizons. Results in this table corroborate the findings above concerning the usefulness of our global factors for predicting GDP growth in EM countries.

Summarizing, our findings with respect to the importance of ``uncertainty'' factors are consistent with business cycle synchronization studies that focus on the growing financial and trade integration of world economies, and note that this integration is likely to result in stronger spillovers of shocks across economies. Additionally, our findings with regard to the importance of ``surprise'' factors  underscore the importance of expectations of economic agents, which in turn reflect the importance of the ever greater amount of ``soft" news that is available to agents, and that may not be reflected in the timely release of macroeconomic data.

We now turn to a discussion of the usefulness of data shrinkage in our experiments. Even casual inspection of the findings in Tables 1-5 indicates that pre-selection of indicators using shrinkage delivers models with more accurate forecasts than when DFMs are estimated without pre-selection. In particular, when comparing the ``globally MSFE-best'' models in Table 1-5, which we have defined to be the MSFE-best models for each country across all specification and shrinkage methods, we see that 47 of the 50 ``globally best'' models utilize shrinkage\footnote{Recall that there are ten forecast horizons and five countries, so that we have total of 50 cases.}. This suggests that the selection of relevant predictors from a large dataset mitigates data noisiness, and model (and coefficient estimator) imprecision due to multicollinearity, as might be expected. This result corroborates that of Boivin and Ng (2006), where it is suggested that the possibility of correlated errors increase as more series from the same ``category'' are included in a dataset, and creates a situation where more data might not be desirable. Among the different shrinkage methods that we utilize in our experiment, the Bayesian methods (Bayesian LASSO and Bayesian AdaLASSO) perform surprisingly well, as they attain the top rank in 36 out of 50 cases. This evidence strongly supports the use of Bayesian shrinkage methods for variable selection.\footnote{See De Mol et al. (2008), where it is found that a few appropriately selected variables often capture the bulk of the covariation in large macroeconomic dataset that are characterized by collinearity, and use of these small groups of variables often yields comparable forecasting performance, relative to the case where all variables are used when constructing predictions.} These results are summarized in Table A2 of Appendix A.

\section{Concluding Remarks}

Dynamic factor models (DFMs) are widely used in the forecasting literature. In this paper, we add to this literature by utilizing these models to predict emerging market (EM) GDP growth rates. Moreover, we augment the standard DFM type models that we examine with global ``uncertainty'' and ``surprise'' factors that are meant to capture the deepening interdependencies among all countries in the world. To do this, we construct three new datasets, one focusing on country specific data, and two focusing on worldwide uncertainty, sentiment, and so-called ``surprise'' data, from which global latent factors are extracted. These factors are constructed using both standard estimation methods as well as via implementation of a number of LASSO type shrinkage methods. We find that our new global economic ``uncertainty'' and macroeconomic data ``surprise'' factors are indeed useful, in the sense that they contain substantial marginal predictive content for real GDP growth in numerous EM economies, as shown through a series of real-time forecasting experiments. Moreover, the data shrinkage methods employed in our experiments are found to significantly improve the predictive content of the latent factors used in our DFMs.

This paper is meant as a starting point, as many questions remain unanswered. For example, it will be of interest to investigate the regime-dependent impact of uncertainty shocks on growth by distinguishing effects associated with phases of the business cycle. Also, it remains to analyze the impacts of possible changes in information rigidities in consensus forecasts (such as those used in the construction of our global ``surprise'' factors) when the economy moves from one phase of the business cycle to another. It also remains to asymptotically analyze extant shrinkage methods, with an eye to the development of new and improved estimation algorithms.


\newpage

\section*{References}

Ahn, S. C., and Horenstein, A. R. (2013). Eigenvalue ratio test for the number of factors. Econometrica, 81(3), 1203-1227.

Alessi, L., Barigozzi, M., and Capasso, M. (2010). Improved penalization for determining the number of factors in approximate factor models. Statistics and Probability Letters, 80(23-24), 1806-1813.

Artis, M. J., Banerjee, A., and Marcellino, M. (2005). Factor forecasts for the UK. Journal of forecasting, 24(4), 279-298.

Bai, J., and Ng, S. (2002). Determining the number of factors in approximate factor models. Econometrica, 70, 191-221.

Bai, J., and Ng, S. (2008). Forecasting economic time series using targeted predictors. Journal of Econometrics, 146, 304-317.

Baker, S. R., Bloom, N., and Davis, S. J. (2016). Measuring economic policy uncertainty. The Quarterly Journal of Economics, 131(4), 1593-1636.

Banbura, M.,  and R\"{u}nstler, G. (2011). A look into the factor model black box: publication lags and the role of hard and soft data in forecasting GDP. International Journal of Forecasting, 27, 333-346.

Bessec, M. (2013). Short-Term Forecasts of French GDP: A Dynamic Factor Model with Targeted Predictors. Journal of Forecasting, 32(6), 500-511.

Bloom, N. (2014). Fluctuations in uncertainty. Journal of Economic Perspectives, 28(2), 153-76.

Bontempi, M. E., Golinelli, R., and Squadrani, M. (2016). A New Index of Uncertainty Based on Internet Searches: A Friend or Foe of Other Indicators?, Quaderni, Working Paper DSE No 1062.

Boivin, J., and Ng, S. (2006). Are more data always better for factor analysis? Journal of Econometrics, 132, 169-194.

Bragoli, D. (2017). Now-casting the Japanese economy. International Journal of Forecasting, 33(2), 390-402.

Bragoli, D., and Fosten, J. (2018). Nowcasting Indian GDP. Oxford Bulletin of Economics and Statistics, 80(2), 259-282.

Bulligan, G., Marcellino, M., and Venditti, F. (2015). Forecasting economic activity with targeted predictors. International Journal of Forecasting, 31, 188-206.

Caggiano, G., Kapetanios, G., and Labhard, V. (2011). Are more data always better for factor analysis? Results for the euro area, the six largest euro area countries and the UK. Journal of Forecasting, 30(8), 736-752.

Carriere-Swallow, Y., and Cespedes, L. F. (2013). The impact of uncertainty shocks in emerging economies. Journal of International Economics, 90(2), 316-325.

Carriero, A., Clark, T. E., and Marcellino, M. (2016). Common drifting volatility in large Bayesian VARs. Journal of Business and Economic Statistics, 34(3), 375-390.

De Mol, C., Giannone, D., and Reichlin, L. (2008). Forecasting using a large number of predictors: Is Bayesian shrinkage a valid alternative to principal components?. Journal of Econometrics, 146(2), 318-328.

Diebold, F. X., and  Mariano, R. S. (1995). Comparing predictive accuracy. Journal of Business and Economic Statistics, 20, 134-144.

Fan, J., and Li, R. (2001). Variable selection via non-concave penalized likelihood and its oracle properties. Journal of the American statistical Association, 96(456), 1348-1360.

Forni, M., Hallin, M., Lippi, M., and Reichlin, L. (2000). The generalized dynamic-factor model: Identification and estimation. Review of Economics and Statistics, 82, 540-554.

Fu, W. J. (1998). Penalized regressions: the bridge versus the LASSO. Journal of Computational and Graphical Statistics, 7, 397-416.

Gauvin, L., McLoughlin, C., and Reinhardt, D. (2014). Policy uncertainty spillovers to emerging markets–evidence from capital flows. Bank of England Working Paper No. 512

Giannone, D., Reichlin, L., and Small, D. (2008). Nowcasting: The real-time informational content of macroeconomic data. Journal of Monetary Economics, 55(4), 665-676.

Girardi, A., Gayer, C., and Reuter, A. (2015). The role of survey data in nowcasting euro area GDP growth. Journal of Forecasting, 35(5), 400-418.

Hastie, T., Tibshirani, R., and Friedman, J. (2009). Unsupervised learning. In The elements of statistical learning (pp. 485-585). Springer, New York, NY.

Kapetanios, G., Marcellino, M., and Papailias, F. (2016). Forecasting inflation and GDP growth using heuristic optimisation of information criteria and variable reduction methods. Computational Statistics and Data Analysis, 100, 369-382.

Kim, H. H., and Swanson, N. R. (2014). Forecasting financial and macroeconomic variables using data reduction methods: New empirical evidence. Journal of Econometrics, 178, 352-367.

Kim, H. H.,  and Swanson, N. R. (2018a). Mining big data using parsimonious factor, machine learning, variable selection and shrinkage methods. International Journal of Forecasting, 34, 339-354.

Kim, H. H., and Swanson, N. R. (2018b). Methods for backcasting, nowcasting and forecasting using factor MIDAS: With an application to Korean GDP. Journal of Forecasting, 37(3), 281-302.

Korobilis, D. (2013). Hierarchical shrinkage priors for dynamic regressions with many predictors. International Journal of Forecasting, 29(1), 43-59.

Li, J., and Chen, W. (2014). Forecasting macroeconomic time series: LASSO-based approaches and their forecast combinations with dynamic factor models. International Journal of Forecasting, 30(4), 996-1015.

Luciani, M., Pundit, M., Ramayandi, A., and Veronese, G. (2018). Nowcasting Indonesia. Empirical Economics, 55(2), 597-619.

Mariano, R. S., and Murasawa, Y. (2003). A new coincident index of business cycles based on monthly and quarterly series. Journal of applied Econometrics, 18(4), 427-443.

Medeiros, M. C., and Mendes, E. F. (2016). L1-regularization of high-dimensional time-series models with non-Gaussian and heteroskedastic errors. Journal of Econometrics, 191(1), 255-271.

Medeiros, M. C., and Vasconcelos, G. F. (2016). Forecasting macroeconomic variables in data-rich environments. Economics Letters, 138, 50-52.

Modugno, M., Soybilgen, B., and Yazgan, E. (2016). Nowcasting Turkish GDP and news decomposition. International Journal of Forecasting, 32, 1369-1384.

Onatski, A. (2009). Testing hypotheses about the number of factors in large factor models. Econometrica, 77(5), 1447-1479.

Park, T., and Casella, G. (2008). The bayesian LASSO. Journal of the American Statistical Association, 103(482), 681-686.

Rossi, B., and Sekhposyan, T. (2015). Macroeconomic uncertainty indices based on nowcast and forecast error distributions. American Economic Review, 105(5), 650-55.

Schumacher, C. (2007). Forecasting German GDP using alternative factor models based on large datasets. Journal of Forecasting, 26(4), 271-302.

Schumacher, C. (2010). Factor forecasting using international targeted predictors: the case of German GDP. Economics Letters, 107, 95-98.

Scotti, C. (2016). Surprise and uncertainty indexes: Real-time aggregation of real-activity macro-surprises. Journal of Monetary Economics, 82, 1-19.

Stock, J. H., and Watson, M. W. (2002). Macroeconomic forecasting using diffusion indexes. Journal of Business and Economic Statistics, 20, 147-162.

Swanson, N.R., (2016). Comment On: In Sample Inference and Forecasting in Misspecified Factor Models, Journal of Business and Economic Statistics, 34, 348-353.

Thorsrud, L. A. (2018). Words are the New Numbers: A Newsy Coincident Index of the Business Cycle. Journal of Business and Economic Statistics, (just-accepted), 1-35.

Tibshirani, R. (1996). Regression shrinkage and selection via the LASSO. Journal of the Royal Statistical Society. Series B, 267-288.

Zou, H. (2006). The adaptive LASSO and its oracle properties. Journal of the American statistical association, 101(476), 1418-1429.

\pagebreak

\begin{figure}
  \centering
  \caption{Global economic uncertainty factor}
  \includegraphics[scale=0.60 ]{globalfactor1.jpg}
\end{figure}

\begin{figure}[hbt]
  \centering
  \caption{World GDP growth rates plotted against global macro-surprise factor}
  \includegraphics[scale=0.60 ]{globalfactor2.jpg}
\end{figure}
%

\begin{landscape}
\begin{figure}
  \caption{GDP growth rates plotted against local (country-specific) factors}
  \includegraphics[scale=0.85 ]{factors.pdf}
\end{figure}
\end{landscape}


\newpage
\begin{landscape}
\begin{figure}
	\caption{Comparison of actual GDP growth rates with nowcasts based on an AR benchmark, local factor model and MSFE-best models}
	\begin{center}
		\includegraphics[scale=0.83 ]{Nowcasts_new.pdf}
	\end{center}
\end{figure}
\end{landscape}

\clearpage
% Table generated by Excel2LaTeX from sheet 'Brazil'
\begin{table}[!htb]
  \centering
   	\caption{MSFEs based on the use of different dimension reduction and shrinkage methods with added global diffusion indexes \\
		Panel A: Brazil}
   \tiny
    \begin{tabular*}{\textwidth}{lc @{\extracolsep{\fill}}ccccccccccc}\hline
		\multicolumn{1}{l}{\textbf{All Sample}} &   & Forecast (h=2) &   &   & Forecast (h=1) &   &   & Nowcast (h=0) &   & Backcast (h=-1) \\\hline
		& 1 & 2 & 3 & 1 & 2 & 3 & 1 & 2 & 3 & 1 \\\hline
    AR    & 4.43  & 4.43  & 4.20  & 3.85  & 3.85 & 3.52  & 2.99  & 2.99  & 2.62 & 2.62 \\
    Local & 0.96  & 0.88  & 0.85  & 0.77  & 0.67*  & 0.62* & 0.50*  & 0.39*  & 0.33*  & 0.43** \\
    Uncertainty & 0.94  & 0.85  & 0.83  & 0.72*  & 0.60*  & 0.57*  & 0.44*  & 0.30*  & 0.30*  & 0.46** \\
    Suprise & 0.96  & 0.89  & 0.85  & 0.78  & 0.68  & 0.60*  & 0.51*  & 0.39*  & 0.28*  & 0.39** \\
    Global & 0.92  & \textbf{0.84} & \textbf{0.83} & 0.72*  & 0.60*  & 0.57*  & 0.45*  & 0.30*  & 0.28*  & 0.44** \\
    Local-AR & 0.96  & 0.89  & 0.86  & 0.78  & 0.68  & 0.63*  & 0.51*  & 0.39*  & 0.31*  & 0.38** \\
    Uncertainty-AR & 0.93  & 0.85  & 0.84  & 0.72*  & 0.60*  & 0.57*  & 0.44*  & \textbf{0.29*} & \textbf{0.27*} & 0.43** \\
    Suprise-AR & 0.96  & 0.89  & 0.86  & 0.78  & 0.68  & 0.60*  & 0.51*  & 0.38*  & 0.27*  & \textbf{0.38**} \\
    Global-AR & \textbf{0.91} & 0.84  & 0.84  & \textbf{0.71*} & \textbf{0.60*} & \textbf{0.57*} & \textbf{0.44*} & 0.29*  & 0.27*  & 0.41** \\\hline
    \multicolumn{1}{l}{\textbf{LASSO}} &   &   &   &   &   &   &   &   &   &  \\\hline
    Local & \textbf{0.90} & 0.81  & \textbf{0.76} & 0.72  & 0.64  & \textbf{0.53**} & 0.49*  & 0.42*  & 0.32*  & 0.44** \\
    Uncertainty & 0.98  & 0.81  & 0.81  & 0.72*  & \textbf{0.58*} & 0.56**  & 0.44*  & 0.32*  & 0.33*  & 0.51** \\
    Suprise & 0.92  & 0.82  & 0.83  & 0.74  & 0.66  & 0.54**  & 0.49*  & 0.41*  & 0.33*  & 0.41** \\
    Global & 0.92  & \textbf{0.80} & 0.87  & \textbf{0.70*} & 0.59*  & 0.57**  & 0.44*  & 0.32*  & 0.34*  & 0.53* \\
    Local-AR & 0.95  & 0.85  & 0.85  & 0.77  & 0.68  & 0.66* & 0.50*  & 0.41*  & 0.35*  & \textbf{0.34**} \\
    Uncertainty-AR & 1.00  & 0.83  & 0.85  & 0.74*  & 0.59*  & 0.57**  & 0.43*  & \textbf{0.28*} & 0.25*  & 0.42** \\
    Suprise-AR & 0.96  & 0.84  & 0.84  & 0.75*  & 0.67*  & 0.65*  & 0.48*  & 0.39*  & 0.35*  & 0.35** \\
    Global-AR & 0.95  & 0.82  & 0.85  & 0.71*  & 0.59*  & 0.57*  & \textbf{0.42*} & 0.29*  & \textbf{0.25*} & 0.39** \\\hline
    \multicolumn{1}{l}{\textbf{AdaLASSO}} &   &   &   &   &   &   &   &   &   &  \\\hline
    Local & 0.91  & 0.86  & 0.85  & 0.75  & 0.67  & 0.63*  & 0.51*  & 0.40*  & 0.33*  & 0.38** \\
    Uncertainty & 0.87  & 0.82  & 0.83  & 0.71*  & 0.61*  & 0.59*  & 0.46*  & 0.33*  & 0.31*  & 0.40** \\
    Suprise & 0.91  & 0.86  & 0.85  & 0.77  & 0.68  & 0.62*  & 0.53*  & 0.40*  & 0.30*  & 0.35** \\
    Global & \textbf{0.85} & \textbf{0.81} & \textbf{0.83} & \textbf{0.69*} & \textbf{0.60*} & \textbf{0.58*} & 0.46*  & 0.32*  & 0.29*  & 0.37** \\
    Local-AR & 0.93  & 0.87  & 0.87  & 0.77  & 0.68  & 0.65*  & 0.52*  & 0.39*  & 0.31*  & \textbf{0.34**}$_\textbf{{GB}}$ \\
    Uncertainty-AR & 0.89  & 0.83  & 0.85  & 0.72*  & 0.61*  & 0.59*  & 0.46*  & 0.31*  & 0.27*  & 0.37** \\
    Suprise-AR & 0.93  & 0.87  & 0.86  & 0.77  & 0.68  & 0.63*  & 0.52*  & 0.40*  & 0.29*  & 0.34** \\
    Global-AR & 0.87  & 0.83  & 0.84  & 0.70*  & 0.60*  & 0.58*  & \textbf{0.45*} & \textbf{0.31*} & \textbf{0.27*} & 0.39* \\\hline
    \multicolumn{1}{l}{\textbf{Bayesian LASSO}} &   &   &   &   &   &   &   &   &   &  \\\hline
    Local & 0.80  & 0.75  & 0.76  & \textbf{0.62*}$_\textbf{{GB}}$ & 0.54*& 0.53** & \textbf{0.28*}$_\textbf{{GB}}$ & \textbf{0.14*}$_\textbf{{GB}}$ & 0.13*  & 0.47* \\
    Uncertainty & 0.81  & 0.76  & 0.77  & 0.64*  & 0.56*  & 0.54**  & 0.31*  & 0.16*  & \textbf{0.12*}$_\textbf{{GB}}$ & 0.45* \\
    Suprise & \textbf{0.79} & \textbf{0.74} & \textbf{0.75} & 0.63*  & \textbf{0.54*}$_\textbf{{GB}}$ & \textbf{0.53**}  & 0.30*  & 0.16*  & 0.14*  & 0.46* \\
    Global & 0.80  & 0.75  & 0.76  & 0.64*  & 0.56*  & 0.54*  & 0.31*  & 0.17*  & 0.13*  & 0.43** \\
    Local-AR & 0.88  & 0.82  & 0.82  & 0.70*  & 0.61*  & 0.58*  & 0.39*  & 0.24*  & 0.14*  & 0.36** \\
    Uncertainty-AR & 0.89  & 0.83  & 0.83  & 0.71  & 0.62*  & 0.59*  & 0.41*  & 0.26*  & 0.14*  & \textbf{0.35**} \\
    Suprise-AR & 0.88  & 0.82  & 0.82  & 0.70*  & 0.61*  & 0.58*  & 0.40*  & 0.24*  & 0.14*  & 0.36** \\
    Global-AR & 0.88  & 0.82  & 0.82  & 0.71*  & 0.62*  & 0.58*  & 0.40*  & 0.25*  & 0.13*  & 0.35** \\\hline
    \multicolumn{1}{l}{\textbf{Bayesian AdaLASSO}} &   &   &   &   &   &   &   &   &   &  \\\hline
    Local & 0.79  & \textbf{0.73}$_\textbf{{GB}}$ & \textbf{0.71*}$_\textbf{{GB}}$ & 0.66*  & \textbf{0.58*} & 0.53**  & \textbf{0.48*} & 0.41*  & 0.37* & 0.48** \\
    Uncertainty & 0.80  & 0.74  & 0.72  & 0.67*  & 0.59*  & \textbf{0.52**}$_\textbf{{GB}}$ & 0.48*  & 0.42*  & 0.35*  & 0.47** \\
    Suprise & 0.80  & 0.79  & 0.75  & 0.70*  & 0.65*  & 0.56**  & 0.53*  & 0.48*  & 0.37*  & 0.48** \\
    Global & \textbf{0.76}$_\textbf{{GB}}$ & 0.79  & 0.74  & \textbf{0.65**} & 0.62  & 0.53*  & 0.49*  & 0.44*  & 0.33*  & 0.45** \\
    Local-AR & 0.86  & 0.79  & 0.79  & 0.73*  & 0.65*  & 0.61**  & 0.52*  & 0.43*  & 0.34*  & \textbf{0.35**} \\
    Uncertainty-AR & 0.83  & 0.76  & 0.75  & 0.70**  & 0.60*  & 0.54**  & 0.49*  & \textbf{0.39*} & 0.26*  & 0.36** \\
    Suprise-AR & 0.85  & 0.80  & 0.78  & 0.73*  & 0.66*  & 0.59**  & 0.52*  & 0.46*  & 0.31*  & 0.38** \\
    Global-AR & 0.81  & 0.79  & 0.77  & 0.68**  & 0.62*  & 0.54*  & 0.49*  & 0.40*  & \textbf{0.26*} & 0.41** \\\hline
    \end{tabular*}%
\begin{tablenotes}
		\tiny
		\item[a]{Entries in this table are MSFEs. Models that yield the smallest MSFE are denoted in bold, for each estimation method and forecast horizon; and models that are denoted in bold with a ``GB'' subscript denote models that are ``globally'' MSFE-``best'', across all models and estimation methods, for a given forecast horizon. Entries in the first row of the table are point MSFEs based our benchmark AR(SIC) model, while the rest of the entries in the table are relative MSFEs (i.e., relative to the AR(SIC) benchmark model). Thus, a value of less than unity indicates that a particular model and estimation method is more accurate than that based on the AR(SIC) benchmark, for a particular forecast horizon. Quarterly forecast horizons are denoted by $h$=-1,0,1, or 2; and monthly forecasts within each of these quarters are denoted by month 1, 2, or 3. Entries superscripted with asterisks (** = $5\% $ level; * = $10\%$ level) are significantly superior than the AR(SIC) benchmark model, based on application of the Diebold-Mariano (1995) predictive accuracy test. All models are listed in the first column of the table, and Local, Uncertainty, Surprise, and Global correspond to Specifications 1-4 from Section 3.3, respectively; while the models appended with ``-AR" are the same as Specifications 1-4, but with additional lagged dependent variables added as regressors. For complete details, refer to Section 3.}
	\end{tablenotes}
\end{table}%
\newpage

\clearpage
% Table generated by Excel2LaTeX from sheet 'Indonesia'
\begin{table}[!htb]
  \centering
   	\caption{MSFEs based on the use of different dimension reduction and shrinkage methods with added global diffusion indexes \\
		Panel B: Indonesia}
   \tiny
    \begin{tabular*}{\textwidth}{lc @{\extracolsep{\fill}}ccccccccccc}\hline
		\multicolumn{1}{l}{\textbf{All Sample}} &   & Forecast (h=2) &   &   & Forecast (h=1) &   &   & Nowcast (h=0) &   & Backcast (h=-1) \\\hline
		& 1 & 2 & 3 & 1 & 2 & 3 & 1 & 2 & 3 & 1 \\\hline
    AR    & 1.07  & 1.07  & 1.01  & 1.05  & 1.05  & 0.97  & 0.80  & 0.80  & 0.73  & 0.73 \\
    Local & \textbf{0.83} & \textbf{0.83}$_\textbf{{GB}}$ & \textbf{0.92} & \textbf{0.72*} & \textbf{0.75*} & \textbf{0.82*} & 0.76*  & 0.79  & 0.88  & 1.16 \\
    Uncertainty & 1.35  & 1.56  & 1.33  & 1.25  & 1.39  & 1.32  & 1.21  & 1.30  & 1.31  & 1.40 \\
    Suprise & 1.13  & 1.18  & 1.28  & 0.98  & 1.01  & 1.11  & 0.94  & 0.93  & 1.02  & 1.07 \\
    Global & 1.13  & 1.19  & 1.24  & 1.09  & 1.13  & 1.23  & 1.11  & 1.13  & 1.23  & 1.32 \\
    Local-AR & 0.86  & 0.86  & 0.97  & 0.76*  & 0.76*  & 0.85  & \textbf{0.75*} & \textbf{0.71*} & \textbf{0.84} & \textbf{0.94} \\
    Uncertainty-AR & 1.38  & 1.55  & 1.33  & 1.24  & 1.33  & 1.27  & 1.12  & 1.17  & 1.16  & 1.25 \\
    Suprise-AR & 1.26  & 1.21  & 1.30  & 1.10  & 1.00  & 1.11  & 1.01  & 0.89  & 0.97  & 1.01 \\
    Global-AR & 1.28  & 1.21  & 1.26  & 1.21  & 1.13  & 1.23  & 1.17  & 1.09  & 1.18  & 1.30 \\\hline
    \multicolumn{1}{l}{\textbf{LASSO}} &   &   &   &   &   &   &   &   &   &  \\\hline
    Local & 0.95  & 0.90  & 0.93  & 0.87*  & 0.87  & \textbf{0.82*} & 0.89*  & 0.86  & 0.88  & 1.03 \\
    Uncertainty & 1.60  & 1.63  & 1.29  & 1.43  & 1.45  & 1.30  & 1.36  & 1.38  & 1.31  & 1.45 \\
    Suprise & 1.26  & 1.17  & 1.23  & 1.15  & 1.08  & 1.06  & 1.13  & 1.08  & 0.99  & 1.11 \\
    Global & 1.33  & 1.23  & 1.19  & 1.28  & 1.19  & 1.17  & 1.30  & 1.25  & 1.18  & 1.36 \\
    Local-AR & \textbf{0.90} & \textbf{0.87} & \textbf{0.93} & \textbf{0.81*} & \textbf{0.83*} & 0.83  & \textbf{0.81*} & \textbf{0.79*} & \textbf{0.86} & \textbf{0.98} \\
    Uncertainty-AR & 1.63  & 1.70  & 1.12  & 1.41  & 1.44  & 1.15  & 1.26  & 1.27  & 1.12  & 1.34 \\
    Suprise-AR & 1.31  & 1.22  & 1.20  & 1.16  & 1.09  & 1.06  & 1.11  & 1.00  & 0.98  & 1.07 \\
    Global-AR & 1.36  & 1.25  & 1.09  & 1.29  & 1.21  & 1.13  & 1.29  & 1.21  & 1.16  & 1.38 \\\hline
    \multicolumn{1}{l}{\textbf{AdaLASSO}} &   &   &   &   &   &   &   &   &   &  \\\hline
    Local & 0.89  & 0.89  & 0.92  & 0.77  & 0.74  & 0.76*  & 0.73*  & 0.65*  & 0.63*  & 0.96 \\
    Uncertainty & 0.90  & 1.01  & 0.96  & 0.83  & 0.91  & 0.85*  & 0.91  & 0.92  & 0.83*  & 1.09 \\
    Suprise & 0.93  & 1.01  & 1.00  & 0.83  & 0.86  & 0.84*  & 0.79  & 0.74*  & 0.68*  & 0.96 \\
    Global & 0.90  & 0.94  & 0.89  & 0.82  & 0.85  & 0.79*  & 0.88  & 0.85  & 0.73*  & 1.05 \\
    Local-AR & 0.95  & 0.97  & 0.97  & 0.79  & 0.80  & 0.75  & \textbf{0.65*} & \textbf{0.61*} & \textbf{0.55*} & 1.00 \\
    Uncertainty-AR & \textbf{0.82}$_\textbf{{GB}}$ & \textbf{0.89} & \textbf{0.84}$_\textbf{{GB}}$ & \textbf{0.67*}$_\textbf{{GB}}$ & \textbf{0.68*}$_\textbf{{GB}}$ & \textbf{0.71*} & 0.72*  & 0.72  & 0.67*  & 1.11 \\
    Suprise-AR & 1.04  & 1.04  & 1.04  & 0.86  & 0.85  & 0.81  & 0.71*  & 0.70*  & 0.63*  & \textbf{0.95} \\
    Global-AR & 0.88  & 0.91  & 0.85  & 0.71*  & 0.70*  & 0.73*  & 0.74*  & 0.76  & 0.70*  & 1.08 \\\hline
   \multicolumn{1}{l}{\textbf{Bayesian LASSO}} &   &   &   &   &   &   &   &   &   &  \\\hline
    Local & 1.32  & 1.22  & 1.24  & 0.92  & 0.78  & 0.68*  & 0.55*  & 0.56*  & 0.39*  & 0.79 \\
    Uncertainty & 1.25  & 1.20  & 1.21  & 0.85  & 0.76  & 0.69*  & 0.48*  & 0.57*  & 0.41*  & 0.86 \\
    Suprise & 1.40  & 1.31  & 1.30  & 0.99  & 0.86  & 0.78  & 0.54*  & 0.54*  & 0.37*  & 0.73 \\
    Global & 1.36  & 1.28  & 1.28  & 0.94  & 0.83  & 0.80  & 0.46*  & 0.51*  & 0.39*  & 0.78 \\
    Local-AR & 1.21  & 1.13  & 1.06  & 0.85  & \textbf{0.77} & \textbf{0.67*}$_\textbf{{GB}}$ & 0.43*  & 0.38*  & \textbf{0.29*}$_\textbf{{GB}}$ & 0.49 \\
    Uncertainty-AR & 1.23  & 1.16  & 1.17  & \textbf{0.82} & 0.77  & 0.75  & \textbf{0.35*}$_\textbf{{GB}}$ & \textbf{0.36*}$_\textbf{{GB}}$ & 0.32*  & 0.56 \\
    Suprise-AR & 1.16  & 1.09  & 1.12  & 0.88  & 0.82  & 0.77*  & 0.53*  & 0.46*  & 0.42*  & \textbf{0.43}$_\textbf{{GB}}$ \\
    Global-AR & 1.21  & 1.16  & 1.20  & 0.89  & 0.84  & 0.82  & 0.50*  & 0.46*  & 0.44*  & 0.50 \\\hline
    \multicolumn{1}{l}{\textbf{Bayesian AdaLASSO}} &   &   &   &   &   &   &   &   &   &  \\\hline
    Local & \textbf{0.95} & \textbf{0.94} & 1.03  & 0.83*  & \textbf{0.80} & \textbf{0.87} & 0.81*  & 0.73*  & 0.82  & 0.99 \\
    Uncertainty & 1.04  & 0.97  & 1.13  & 1.05  & 0.97  & 1.06  & 1.20  & 1.16  & 1.17  & 1.43 \\
    Suprise & 1.15  & 1.11  & 1.21  & 1.02  & 0.97  & 1.04  & 0.99  & 0.88  & 0.92  & 1.08 \\
    Global & 1.28  & 0.99  & 1.12  & 1.19  & 0.95  & 1.04  & 1.24  & 1.10  & 1.09  & 1.33 \\
    Local-AR & 1.00  & 1.00  & 1.10  & \textbf{0.83} & 0.83  & 0.92  & \textbf{0.71*} & \textbf{0.63*} & \textbf{0.75} & \textbf{0.88} \\
    Uncertainty-AR & 1.09  & 0.98  & \textbf{0.99} & 1.03  & 0.88  & 0.90  & 1.09  & 1.02  & 0.97  & 1.36 \\
    Suprise-AR & 1.22  & 1.14  & 1.18  & 1.04  & 0.96  & 1.00  & 0.91  & 0.77*  & 0.83  & 1.00 \\
    Global-AR & 1.42  & 1.16  & 1.08  & 1.26  & 1.01  & 0.96  & 1.24  & 1.09  & 0.97  & 1.36 \\\hline
    \end{tabular*}%
 	\begin{tablenotes}
		\tiny
		\item[a]{See notes to Table 1.}
	\end{tablenotes}
\end{table}%
\newpage

% Table generated by Excel2LaTeX from sheet 'Mexico'
\begin{table}[!htb]
  \centering
   	\caption{MSFEs based on the use of different dimension reduction and shrinkage methods with added global diffusion indexes \\
		Panel C: Mexico}
   \tiny
    \begin{tabular*}{\textwidth}{lc @{\extracolsep{\fill}}ccccccccccc}\hline
		\multicolumn{1}{l}{\textbf{All Sample}} &   & Forecast (h=2) &   &   & Forecast (h=1) &   &   & Nowcast (h=0) &   & Backcast (h=-1) \\\hline
		& 1 & 2 & 3 & 1 & 2 & 3 & 1 & 2 & 3 & 1 \\\hline
     AR    & 4.38  & 4.38  & 3.98  & 3.72  & 3.72  & 3.26  & 2.73  & 2.73  & 2.25  & 2.25 \\
    Local & \textbf{0.63} & \textbf{0.53} & 0.52  & \textbf{0.45*} & 0.38**  & \textbf{0.44**} & 0.35**  & 0.33**  & \textbf{0.60*} & \textbf{0.66} \\
    Uncertainty & 0.70  & 0.63  & 0.54  & 0.50*  & 0.45*  & 0.48**  & 0.37**  & 0.36**  & 0.62*  & 0.72 \\
    Suprise & 0.65  & 0.53  & 0.57  & 0.48*  & 0.39**  & 0.52**  & 0.36**  & 0.33**  & 0.68  & 0.78 \\
    Global & 0.70  & 0.61  & 0.57  & 0.50*  & 0.43*  & 0.52**  & 0.37**  & 0.34**  & 0.68  & 0.79 \\
    Local-AR & 0.72  & 0.59  & 0.55  & 0.49*  & 0.40**  & 0.45**  & 0.34**  & 0.32**  & 0.62*  & 0.70 \\
    Uncertainty-AR & 0.75  & 0.65  & 0.55  & 0.51*  & 0.45*  & 0.48**  & 0.36**  & 0.35**  & 0.63*  & 0.74 \\
    Suprise-AR & 0.67  & 0.53  & \textbf{0.51} & 0.47*  & \textbf{0.37**} & 0.47**  & \textbf{0.33**} & \textbf{0.30**} & 0.65  & 0.79 \\
    Global-AR & 0.70  & 0.58  & 0.52  & 0.49*  & 0.40*  & 0.49**  & 0.35**  & 0.32**  & 0.65  & 0.79 \\\hline
    \multicolumn{1}{l}{\textbf{LASSO}} &   &   &   &   &   &   &   &   &   &  \\\hline
    Local & \textbf{0.68} & 0.58  & 0.53  & 0.47*  & 0.39**  & 0.46**  & 0.32**  & 0.28**  & 0.60*  & \textbf{0.71} \\
    Uncertainty & 0.73  & 0.69  & 0.55  & 0.48*  & 0.43**  & 0.48**  & 0.31**  & 0.29**  & 0.61*  & 0.77 \\
    Suprise & 0.69  & \textbf{0.55} & 0.53  & \textbf{0.45*} & \textbf{0.37**} & 0.48**  & 0.30**  & 0.26**  & 0.64  & 0.77 \\
    Global & 0.71  & 0.65  & 0.56  & 0.46*  & 0.37**  & 0.48**  & 0.29**  & 0.26**  & 0.64  & 0.80 \\
    Local-AR & 0.78  & 0.64  & 0.55  & 0.52*  & 0.42*  & 0.46**  & 0.31**  & 0.27**  & \textbf{0.60*} & 0.74 \\
    Uncertainty-AR & 0.81  & 0.73  & 0.57  & 0.52*  & 0.45*  & 0.50**  & 0.31**  & 0.28**  & 0.60*  & 0.79 \\
    Suprise-AR & 0.77  & 0.61  & \textbf{0.52} & 0.50*  & 0.39*  & \textbf{0.46**} & 0.30**  & 0.25**  & 0.61*  & 0.79 \\
    Global-AR & 0.78  & 0.70  & 0.56  & 0.49*  & 0.39*  & 0.48**  & \textbf{0.29**} & \textbf{0.25**} & 0.62*  & 0.81 \\\hline
    \multicolumn{1}{l}{\textbf{AdaLASSO}} &   &   &   &   &   &   &   &   &   &  \\\hline
    Local & \textbf{0.53}$_\textbf{{GB}}$ & \textbf{0.48}$_\textbf{{GB}}$ & \textbf{0.48*}$_\textbf{{GB}}$ & \textbf{0.41*}$_\textbf{{GB}}$ & \textbf{0.40*} & \textbf{0.42**}$_\textbf{{GB}}$ & 0.34**  & 0.33**  & 0.52**  & \textbf{0.64*}$_\textbf{{GB}}$ \\
    Uncertainty & 0.63  & 0.56  & 0.61  & 0.45*  & 0.44*  & 0.53**  & 0.34**  & 0.32**  & 0.57*  & 0.79 \\
    Suprise & 0.55  & 0.52  & 0.52  & 0.42*  & 0.42*  & 0.49**  & 0.34**  & 0.34**  & 0.62*  & 0.78 \\
    Global & 0.61  & 0.54  & 0.62  & 0.45*  & 0.42*  & 0.56**  & 0.33**  & 0.31**  & 0.64*  & 0.87 \\
    Local-AR & 0.64  & 0.56  & 0.54  & 0.46*  & 0.41*  & 0.44**  & 0.31**  & \textbf{0.29**} & \textbf{0.52*} & 0.67 \\
    Uncertainty-AR & 0.71  & 0.64  & 0.67  & 0.49*  & 0.47*  & 0.58**  & 0.32**  & 0.32**  & 0.56*  & 0.79 \\
    Suprise-AR & 0.63  & 0.57  & 0.52  & 0.44*  & 0.43*  & 0.46**  & 0.31**  & 0.32**  & 0.58*  & 0.81 \\
    Global-AR & 0.67  & 0.59  & 0.64  & 0.46*  & 0.43*  & 0.56**  & \textbf{0.31**} & 0.31**  & 0.60*  & 0.87 \\\hline
   \multicolumn{1}{l}{\textbf{Bayesian LASSO}} &   &   &   &   &   &   &   &   &   &  \\\hline
    Local & 0.91  & \textbf{0.74} & \textbf{0.82} & 0.47*  & \textbf{0.41*} & \textbf{0.57*} & \textbf{0.19**}$_\textbf{{GB}}$ & 0.17** & 0.18**  & 0.78 \\
    Uncertainty & \textbf{0.90} & 0.80  & 0.83  & \textbf{0.47*} & 0.43*  & 0.58*  & 0.20**  & 0.21**  & 0.18**  & 0.78 \\
    Suprise & 0.91  & 0.75  & 0.83  & 0.47*  & 0.41*  & 0.58*  & 0.19**  & \textbf{0.16**}$_\textbf{{GB}}$ & \textbf{0.18**}$_\textbf{{GB}}$ & 0.77 \\
    Global & 0.90  & 0.80  & 0.84  & 0.47*  & 0.43*  & 0.59*  & 0.21**  & 0.22**  & 0.18**  & 0.77 \\
    Local-AR & 1.18  & 0.87  & 0.99  & 0.66  & 0.52*  & 0.72  & 0.23**  & 0.26**  & 0.31**  & 0.70 \\
    Uncertainty-AR & 1.25  & 0.97  & 1.03  & 0.71  & 0.56*  & 0.75  & 0.24**  & 0.20**  & 0.33**  & 0.71 \\
    Suprise-AR & 1.17  & 0.88  & 0.99  & 0.65  & 0.52*  & 0.72  & 0.23**  & 0.25**  & 0.32**  & \textbf{0.70} \\
    Global-AR & 1.23  & 0.96  & 1.04  & 0.70  & 0.55*  & 0.76  & 0.24**  & 0.22**  & 0.34** & 0.70 \\\hline
    \multicolumn{1}{l}{\textbf{Bayesian AdaLASSO}} &   &   &   &   &   &   &   &   &   &  \\\hline
    Local & \textbf{0.61} & 0.53  & \textbf{0.48} & 0.45*  & 0.40*  & \textbf{0.43**} & 0.35**  & 0.33**  & \textbf{0.58*} & \textbf{0.69} \\
    Uncertainty & 0.66  & 0.56  & 0.50  & 0.47*  & 0.41*  & 0.47**  & 0.35**  & 0.34**  & 0.61*  & 0.78 \\
    Suprise & 0.62  & \textbf{0.51} & 0.48  & \textbf{0.42*} & 0.36**  & 0.44**  & 0.32**  & 0.31**  & 0.62*  & 0.74 \\
    Global & 0.65  & 0.52  & 0.51  & 0.44*  & \textbf{0.36**}$_\textbf{{GB}}$ & 0.47**  & 0.33**  & 0.31**  & 0.63*  & 0.80 \\
    Local-AR & 0.73  & 0.62  & 0.53  & 0.52*  & 0.44*  & 0.46**  & 0.35**  & 0.32**  & 0.59*  & 0.74 \\
    Uncertainty-AR & 0.76  & 0.62  & 0.54  & 0.52*  & 0.44*  & 0.49**  & 0.35**  & 0.33**  & 0.60*  & 0.80 \\
    Suprise-AR & 0.74  & 0.60  & 0.52  & 0.49*  & 0.41*  & 0.46**  & \textbf{0.33**} & 0.31**  & 0.62*  & 0.78 \\
    Global-AR & 0.75  & 0.60  & 0.54  & 0.49*  & 0.38*  & 0.49**  & 0.33**  &\textbf{ 0.30**}  & 0.62*  & 0.81 \\\hline
   \end{tabular*}%
 	\begin{tablenotes}
		\tiny
		\item[a]{See notes to Table 1.}
	\end{tablenotes}
\end{table}%
\newpage




% Table generated by Excel2LaTeX from sheet 'S.Africa'
\begin{table}[!htb]
  \centering
   	\caption{MSFEs based on the use of different dimension reduction and shrinkage methods with added global diffusion indexes \\
		Panel D: South Africa}
   \tiny
    \begin{tabular*}{\textwidth}{lc @{\extracolsep{\fill}}ccccccccccc}\hline
		\multicolumn{1}{l}{\textbf{All Sample}} &   & Forecast (h=2) &   &   & Forecast (h=1) &   &   & Nowcast (h=0) &   & Backcast (h=-1) \\\hline
		& 1 & 2 & 3 & 1 & 2 & 3 & 1 & 2 & 3 & 1 \\\hline
    AR    & 2.55  & 2.55  & 2.38  & 2.10  & 2.10  & 1.90  & 1.44  & 1.44  & 1.30  & 1.30 \\
    Local & 0.94  & 0.91  & 1.10  & 0.93  & 0.77  & 0.90  & 0.85  & 0.77  & 1.01  & 1.04 \\
    Uncertainty & 1.04  & 0.94  & \textbf{0.94} & 0.86  & 0.74  & \textbf{0.74*} & 0.76*  & 0.78  & 0.84  & 0.88 \\
    Suprise & 0.91  & 0.82  & 1.18  & 0.91  & 0.76  & 0.97  & 0.85  & 0.73*  & 0.99  & 1.03 \\
    Global & \textbf{0.78} & \textbf{0.76} & 0.99  & \textbf{0.79*} & \textbf{0.71*} & 0.80  & \textbf{0.72*} & \textbf{0.72**} & \textbf{0.80} & \textbf{0.80} \\
    Local-AR & 1.28  & 1.13  & 1.27  & 1.04  & 1.00  & 1.04  & 0.89  & 0.88  & 1.09  & 1.14 \\
    Uncertainty-AR & 1.07  & 1.14  & 1.12  & 0.95  & 0.98  & 1.05  & 0.78*  & 0.89  & 1.08  & 0.98 \\
    Suprise-AR & 1.25  & 1.06  & 1.47  & 1.01  & 0.94  & 1.21  & 0.88  & 0.82  & 1.09  & 1.12 \\
    Global-AR & 1.09  & 1.04  & 1.05  & 0.88  & 0.94  & 0.93  & 0.73**  & 0.88  & 0.99  & 0.91 \\\hline
    \multicolumn{1}{l}{\textbf{LASSO}} &   &   &   &   &   &   &   &   &   &  \\\hline
    Local & 0.82  & 0.85  & 1.10  & 0.81**  & 0.72**  & 1.04  & 0.70**  & 0.67**  & 0.86  & 0.77 \\
    Uncertainty & 0.92  & 0.85  & 1.02  & 0.82**  & 0.69**  & \textbf{0.95} & 0.69**  & 0.65**  & \textbf{0.80} & 0.96 \\
    Suprise & \textbf{0.75} & \textbf{0.80} & 1.18  & 0.79**  & \textbf{0.68**} & 1.18  & 0.68**  & 0.65  & 0.93  & 0.77 \\
    Global & 0.92  & 0.87  & 1.06  & 0.89  & 0.75**  & 1.06  & 0.80  & 0.73**  & 0.82  & 1.01 \\
    Local-AR & 0.87  & 0.86  & 1.18  & 0.83**  & 0.72**  & 1.16  & 0.71**  & \textbf{0.64**} & 0.90  & 0.68 \\
    Uncertainty-AR & 0.85  & 0.80  & 1.07  & 0.82*  & 0.68*  & 1.01  & 0.72**  & 0.67**  & 0.83  & 0.75 \\
    Suprise-AR & 0.77  & 0.87  & 1.28  & \textbf{0.76**} & 0.72**  & 1.35  & \textbf{0.65**} & 0.66**  & 0.99  & \textbf{0.67*}$_\textbf{{GB}}$ \\
    Global-AR & 0.90  & 0.85  & 1.18  & 0.84  & 0.72**  & 1.18  & 0.81  & 0.67**  & 0.88  & 0.88 \\\hline
    \multicolumn{1}{l}{\textbf{AdaLASSO}} &   &   &   &   &   &   &   &   &   &  \\\hline
    Local & 0.99  & 0.99  & 1.09  & 0.87*  & 0.89*  & 1.09  & 0.86*  & 0.89  & 1.50  & 0.87 \\
    Uncertainty & \textbf{0.90} & \textbf{0.92} & \textbf{0.83} & \textbf{0.79*} & \textbf{0.82*} & \textbf{0.70*} & \textbf{0.78**} & 0.86*  & \textbf{0.71*} & 0.84 \\
    Suprise & 0.95  & 0.98  & 0.93  & 0.85*  & 0.89  & 0.78*  & 0.85*  & \textbf{0.84**} & 0.82  & 0.85 \\
    Global & 0.91  & 0.97  & 0.99  & 0.80*  & 0.89  & 1.03  & 0.80*  & 0.89  & 1.47  & 0.86 \\
    Local-AR & 1.20  & 1.08  & 1.07  & 0.99  & 1.02  & 0.95  & 0.88**  & 1.00  & 0.94  & 0.80 \\
    Uncertainty-AR & 1.10  & 0.98  & 0.98  & 0.90  & 0.92  & 0.85  & 0.82**  & 0.90  & 0.94  & \textbf{0.70} \\
    Suprise-AR & 1.29  & 1.24  & 1.12  & 1.15  & 1.18  & 1.21  & 1.09  & 1.24  & 1.88  & 0.91 \\
    Global-AR & 1.03  & 1.02  & 0.95  & 0.91  & 1.03  & 0.77  & 0.81**  & 0.94  & 0.73*  & 0.70 \\\hline
    \multicolumn{1}{l}{\textbf{Bayesian LASSO}} &   &   &   &   &   &   &   &   &   &  \\\hline
    Local & 0.97  & 0.71  & 0.79  & 0.95  & 0.68**  & 0.79*  & 0.81**  & 0.89*  & 1.05  & 1.16 \\
    Uncertainty & 0.87  & 0.78  & 0.79  & 0.85*  & 0.71**  & 0.82*  & 0.83**  & 0.92*  & 1.09  & 1.17 \\
    Suprise & 0.97  & \textbf{0.69*}$_\textbf{{GB}}$ & 0.76  & 0.89  & 0.66**  & 0.77*  & 0.79**  & 0.85*  & 1.01  & 1.16 \\
    Global & 0.82  & 0.75  & 0.77  & 0.79*  & 0.69** & 0.77*  & 0.83*  & 0.92*  & 1.05  & 1.18 \\
    Local-AR & 0.80  & 0.74  & 0.68  & 0.61**  & 0.58**  & 0.53**  & 0.52**  & 0.57**  & 0.70**  & 0.91 \\
    Uncertainty-AR & 0.79  & 0.75  & 0.68  & 0.62**  & 0.57**  & 0.53**  & 0.53**  & 0.61**  & 0.73**  & 0.94 \\
    Suprise-AR & 0.74  & 0.72  & \textbf{0.68*}$_\textbf{{GB}}$ & \textbf{0.58**}$_\textbf{{GB}}$ & 0.56**  & \textbf{0.52**}$_\textbf{{GB}}$ & \textbf{0.50**}$_\textbf{{GB}}$ & \textbf{0.56**}$_\textbf{{GB}}$ & \textbf{0.67**} & \textbf{0.88} \\
    Global-AR & \textbf{0.74}$_\textbf{{GB}}$ & 0.75  & 0.69*  & 0.60**  & \textbf{0.55**}$_\textbf{{GB}}$ & 0.53**  & 0.51**  & 0.61**  & 0.70**  & 0.92 \\\hline
   \multicolumn{1}{l}{\textbf{Bayesian AdaLASSO}} &   &   &   &   &   &   &   &   &   &  \\\hline
    Local & \textbf{0.92} & 0.81  & 0.91  & 0.79*  & 0.65**  & 0.81  & 0.73*  & 0.81  & 0.77*  & 0.85 \\
    Uncertainty & 1.11  & 0.97  & \textbf{0.85} & 0.98  & 0.82  & \textbf{0.77} & 0.91  & 1.00  & \textbf{0.64*}$_\textbf{{GB}}$ & 0.92 \\
    Suprise & 0.94  & \textbf{0.75} & 1.04  & \textbf{0.79*} & \textbf{0.55**} & 0.97  & \textbf{0.71*} & \textbf{0.77*} & 0.80  & \textbf{0.82} \\
    Global & 1.12  & 0.95  & 1.01  & 0.96  & 0.81  & 0.97  & 0.95  & 0.83  & 0.74*  & 1.12 \\
    Local-AR & 1.12  & 1.04  & 1.15  & 0.95  & 0.93  & 1.04  & 0.78*  & 1.05  & 0.86  & 0.93 \\
    Uncertainty-AR & 1.29  & 1.17  & 1.04  & 1.18  & 1.06  & 1.01  & 1.03  & 1.25  & 0.87  & 1.10 \\
    Suprise-AR & 1.13  & 1.00  & 1.33  & 0.93  & 0.88  & 1.29  & 0.73*  & 1.09  & 0.98  & 0.87 \\
    Global-AR & 1.10  & 1.20  & 1.16  & 0.98  & 1.09  & 1.16  & 0.94  & 0.95  & 0.94  & 1.08 \\\hline
   \end{tabular*}%
 	\begin{tablenotes}
		\tiny
		\item[a]{See notes to Table 1.}
	\end{tablenotes}
\end{table}%
\newpage


% Table generated by Excel2LaTeX from sheet 'Sheet2'
\begin{table}[!htb]
  \centering
   	\caption{MSFEs based on the use of different dimension reduction and shrinkage methods with added global diffusion indexes \\
		Panel E: Turkey}
   \tiny
    \begin{tabular*}{\textwidth}{lc @{\extracolsep{\fill}}ccccccccccc}\hline
		\multicolumn{1}{l}{\textbf{All Sample}} &   & Forecast (h=2) &   &   & Forecast (h=1) &   &   & Nowcast (h=0) &   & Backcast (h=-1) \\\hline
		& 1 & 2 & 3 & 1 & 2 & 3 & 1 & 2 & 3 & 1 \\\hline
    AR    & 7.21  & 7.21  & 6.58  & 6.63  & 6.63  & 5.94  & 5.48  & 5.48  & 4.69  & 4.69 \\
    Local & 0.68  & 0.65  & 0.72  & 0.62**  & 0.56**  & 0.60  & 0.56  & 0.55  & 0.66**  & 0.67 \\
    Uncertainty & 0.69  & 0.69  & 0.75  & \textbf{0.60} & 0.55  & 0.58  & 0.56  & 0.55  & 0.66**  & 0.72 \\
    Suprise & \textbf{0.68}$_\textbf{{GB}}$ & \textbf{0.64}$_\textbf{{GB}}$ & \textbf{0.69} & 0.60**  & \textbf{0.54**} & 0.56*  & 0.53  & 0.53  & 0.64**  & 0.74 \\
    Global & 0.68  & 0.67  & 0.73  & 0.60**  & 0.55*  & 0.57  & 0.55**  & 0.55  & 0.64**  & 0.71 \\
    Local-AR & 0.71  & 0.68  & 0.73  & 0.62  & 0.58  & 0.58  & 0.51  & 0.50  & 0.52  & 0.58* \\
    Uncertainty-AR & 0.71  & 0.71  & 0.75  & 0.60  & 0.56  & 0.56*  & 0.51*  & 0.50  & 0.52  & 0.56* \\
    Suprise-AR & 0.73  & 0.69  & 0.73  & 0.62**  & 0.58**  & 0.56*  & \textbf{0.50*} & 0.49  & 0.48  & 0.54* \\
    Global-AR & 0.72  & 0.71  & 0.74  & 0.62*  & 0.59*  & \textbf{0.55*} & 0.51*  & \textbf{0.50} & \textbf{0.48} & \textbf{0.51*} \\\hline
    \multicolumn{1}{l}{\textbf{LASSO}} &   &   &   &   &   &   &   &   &   &  \\\hline
    Local & 0.73  & 0.66  & 0.70  & 0.64*  & 0.57**  & 0.58  & 0.52*  & 0.51  & 0.58*  & 0.61* \\
    Uncertainty & 0.71  & 0.70  & 0.73  & 0.61  & 0.57  & 0.58*  & 0.54  & 0.53  & 0.58**  & 0.66 \\
    Suprise & 0.73  & \textbf{0.65} & \textbf{0.68*}$_\textbf{{GB}}$ & 0.63**  & 0.56**  & 0.56  & 0.50*  & 0.51  & 0.57  & 0.71 \\
    Global & \textbf{0.70} & 0.67  & 0.71  & \textbf{0.59*} & \textbf{0.56*} & \textbf{0.55*} & 0.53  & 0.55  & 0.57*  & 0.65 \\
    Local-AR & 0.79  & 0.72  & 0.73  & 0.66**  & 0.61**  & 0.59  & 0.49*  & 0.48  & 0.47  & 0.56* \\
    Uncertainty-AR & 0.75  & 0.72  & 0.75  & 0.63*  & 0.59  & 0.57*  & 0.49*  & \textbf{0.48} & 0.48  & 0.54* \\
    Suprise-AR & 0.78  & 0.71  & 0.73  & 0.65*  & 0.60**  & 0.58  & \textbf{0.48**} & 0.49  & \textbf{0.45} & 0.54* \\
    Global-AR & 0.74  & 0.70  & 0.74  & 0.62*  & 0.59  & 0.56*  & 0.50  & 0.50  & 0.47  & \textbf{0.51*} \\\hline
    \multicolumn{1}{l}{\textbf{AdaLASSO}} &   &   &   &   &   &   &   &   &   &  \\\hline
    Local & 0.74  & 0.70  & 0.81  & 0.67  & 0.62  & 0.70  & 0.60  & 0.58  & 0.70**  & 0.72 \\
    Uncertainty & 0.75  & 0.79  & 0.82  & 0.64  & 0.62  & 0.64  & 0.63  & 0.61  & 0.69**  & 0.76 \\
    Suprise & \textbf{0.71} & \textbf{0.68} & \textbf{0.75} & 0.64  & 0.60  & 0.64  & 0.56  & 0.56  & 0.67**  & 0.78 \\
    Global & 0.72  & 0.76  & 0.79  & 0.60  & 0.60  & 0.61  & 0.62  & 0.62  & 0.68**  & 0.74 \\
    Local-AR & 0.75  & 0.75  & 0.80  & 0.66  & 0.64  & 0.66  & 0.54*  & 0.52  & 0.51  & 0.56* \\
    Uncertainty-AR & 0.76  & 0.77  & 0.79  & 0.62  & 0.61  & 0.61  & 0.55  & 0.52  & 0.54  & 0.58* \\
    Suprise-AR & 0.76  & 0.75  & 0.79  & 0.66  & 0.63  & 0.64  & \textbf{0.53*} & \textbf{0.52} & \textbf{0.48*} & 0.53* \\
    Global-AR & 0.73  & 0.75  & 0.77  & \textbf{0.60*} & \textbf{0.59*} & \textbf{0.58*} & 0.55  & 0.54  & 0.51*  & \textbf{0.53*} \\\hline
    \multicolumn{1}{l}{\textbf{Bayesian LASSO}} &   &   &   &   &   &   &   &   &   &  \\\hline
    Local & 0.76  & 0.73  & 0.78  & 0.62*  & 0.58  & 0.63  & 0.40**  & 0.39*  & 0.40  & \textbf{0.46}$_\textbf{{GB}}$ \\
    Uncertainty & 0.74  & 0.72  & 0.78  & 0.58*  & 0.52  & 0.59  & 0.37**  & 0.36*  & 0.37*  & 0.61* \\
    Suprise & 0.75  & 0.72  & \textbf{0.76} & 0.61*  & 0.58  & 0.61  & 0.39**  & 0.38**  & 0.38*  & 0.65* \\
    Global & \textbf{0.74} & \textbf{0.72} & 0.78  & \textbf{0.57*} & \textbf{0.52} & \textbf{0.59} & \textbf{0.36**}$_\textbf{{GB}}$ & 0.35*  & 0.36*  & 0.62* \\
    Local-AR & 0.94  & 0.82  & 0.84  & 0.69  & 0.61  & 0.62  & 0.38**  & 0.36*  & 0.32**  & 0.55* \\
    Uncertainty-AR & 0.93  & 0.82  & 0.83  & 0.67  & 0.59  & 0.61  & 0.37**  & 0.35*  & 0.30**  & 0.54* \\
    Suprise-AR & 0.96  & 0.83  & 0.83  & 0.69  & 0.62  & 0.62  & 0.38**  & 0.36*  & 0.31**  & 0.55* \\
    Global-AR & 0.94  & 0.82  & 0.83  & 0.67  & 0.59  & 0.60  & 0.36**  & \textbf{0.35*}$_\textbf{{GB}}$ & \textbf{0.30**}$_\textbf{{GB}}$ & 0.53* \\\hline
   \multicolumn{1}{l}{\textbf{Bayesian AdaLASSO}} &   &   &   &   &   &   &   &   &   &  \\\hline
    Local & 0.76  & 0.67  & 0.72  & 0.62*  & 0.55*  & 0.58  & 0.46**  & 0.45  & 0.50  & \textbf{0.53*} \\
    Uncertainty & 0.73  & 0.70  & 0.74  & 0.54  & 0.50  & 0.54**  & 0.42**  & 0.43  & 0.48  & 0.65 \\
    Suprise & 0.76  & \textbf{0.66} & \textbf{0.69} & 0.62*  & 0.54*  & 0.56  & 0.44**  & 0.44  & 0.49  & 0.72 \\
    Global & \textbf{0.72} & 0.69  & 0.71  & \textbf{0.54*}$_\textbf{{GB}}$ & \textbf{0.50*}$_\textbf{{GB}}$ & 0.52**  & 0.42** & 0.44  & 0.48  & 0.65 \\
    Local-AR & 0.85  & 0.71  & 0.73  & 0.64  & 0.56*  & 0.57*  & 0.42**  & 0.42*  & 0.43  & 0.61 \\
    Uncertainty-AR & 0.81  & 0.72  & 0.75  & 0.58  & 0.52  & 0.54**  & 0.39**  & \textbf{0.39*} & 0.42*  & 0.58* \\
    Suprise-AR & 0.84  & 0.69  & 0.72  & 0.63  & 0.55*  & 0.55*  & 0.41**  & 0.41*  & 0.42*  & 0.61* \\
    Global-AR & 0.78  & 0.70  & 0.72  & 0.57  & 0.51  & \textbf{0.52**}$_\textbf{{GB}}$ & \textbf{0.39**} & 0.40  & \textbf{0.41*} & 0.56* \\\hline
    \end{tabular*}%
 	\begin{tablenotes}
		\tiny
		\item[a]{See notes to Table 1.}
	\end{tablenotes}
\end{table}%
\pagebreak
\newpage
\section*{Appendix A: Supplementary Tables}
\setcounter{table}{0}
\renewcommand{\thetable}{A\arabic{table}}
\setcounter{figure}{0}
\renewcommand{\thefigure}{A\arabic{figure}}
\begin{table}[h]
  \centering
  \caption{The number of times the each specification types ranked MSFE-best model across all dimension reduction methods}
  \tiny
    \begin{tabular}{c|ccccc}
    \multicolumn{1}{c}{} & Brazil & Indonesia & Mexico & S.Africa & Turkey \\
    \midrule
    AR    & 0     & 3     & 0     & 1     & 0 \\
    Local & 10    & 11    & 25    & 1     & 2 \\
    Uncertainty & 3     & 0     & 2     & 15    & 1 \\
    Suprise & 5     & 0     & 7     & 11    & 12 \\
    Global & 12    & 0     & 1     & 8     & 13 \\
    Local-AR & 3     & 24    & 3     & 1     & 0 \\
    Uncertainty-AR & 5     & 10    & 0     & 1     & 2 \\
    Suprise-AR & 1     & 2     & 8     & 10    & 6 \\
    Global-AR & 11    & 0     & 4     & 2     & 14 \\\hline
    \end{tabular}%
  \label{tab:addlabel}%
\end{table}%

\begin{table}[h]
  \centering
  \caption{Summary of the ``globally-best" model across all dimension reduction methods and specification types}
    \tiny
    \begin{tabular*}{\textwidth}{lc @{\extracolsep{\fill}}cccccccccccc}\hline
          & \multicolumn{3}{c}{Forecast (h=2)} & \multicolumn{3}{c}{Forecast (h=1)} & \multicolumn{3}{c}{Nowcast (h=0)} & Backcast (h=-1) \\\hline
          & 1     & 2     & 3     & 1     & 2     & 3     & 1     & 2     & 3     & 1 \\\hline
    Brazil & BaLASSO-4 & BaLASSO-1  & BaLASSO-1   & BLASSO-1 & BLASSO-2   & BaLASSO-2   & BLASSO-1 & BLASSO-1   & BLASSO-2 & AdaLASSO-5 \\
    Indonesia & AdaLASSO-6  & ALL-1   & AdaLASSO-6  & AdaLASSO-6   & AdaLASSO-6   & BLASSO-5   & BLASSO-6  & BLASSO-6   & BLASSO-5  & BLASSO-7  \\
    Mexico & AdaLASSO-1  & AdaLASSO-1    & AdaLASSO-1    & AdaLASSO-1    & BaLASSO-4  & AdaLASSO-1   & BLASSO-1   & BLASSO-3 & BLASSO-3   & AdaLASSO-1 \\
    S.Africa & BLASSO-8  & BLASSO-3   & BLASSO-7   & BLASSO-7   & BLASSO-8   & BLASSO-7   & BLASSO-7   & BLASSO-7  & BaLASSO-2  & LASSO-7 \\
    Turkey & ALL-3 & ALL-3  & LASSO-3   & BaLASSO-4   & BaLASSO-4   & BaLASSO-8   & BLASSO-4   & BLASSO-8  & BLASSO-8   & BLASSO-1  \\\hline
    \end{tabular*}%
     \begin{tablenotes}
    	\tiny
    	\item[a]{See notes to Table 3. The following abbreviations; Local = "1", Uncertainty = "2", Surprise = "3", Global = "4", Local-AR = "5", Uncertainty-AR = `"6", Surprise-AR = ``7", and Global-AR = "8". For example, BLASSO-4 means that the Global factor model given as Specification 4 in Section 3.3 yields the lowest MSFE across all different models and different targeted predictor selection methods for a given country, as listed in column 1 of the table.}
    \end{tablenotes}
\end{table}%


\section*{Appendix B: Data}
\setcounter{table}{0}
\renewcommand{\thetable}{B\arabic{table}}
\setcounter{figure}{0}
\renewcommand{\thefigure}{B\arabic{figure}}

\scriptsize
\begin{longtable}{|c|c|c|}
\caption{Dataset-Brazil}\\
\hline
\textbf{Number} & \textbf{Ticker}  & \textbf{Description} \\
\hline
\endfirsthead
\multicolumn{3}{c}%
{\tablename\ \thetable\ -- \textit{Continued from previous page}} \\
\hline
\textbf{Number} & \textbf{Ticker}  & \textbf{Description} \\
\hline
\endhead
\hline \multicolumn{3}{r}{\textit{Continued on next page}} \\
\endfoot
\hline
\endlastfoot
      1     & 2236689 Index & IMF Brazil Unemployment Rate \\
    2     & BZJCYTOT Index & Brazil Government Registered Job Creation Total NSA YTD \\
    3     & BZMW Index & Brazil Minimum Wage \\
    4     & BZGDWGSL Index & Brazil Wages and Salaries                                                        \\
    5     & BRMWRL Index & Brazil Real Minimum Wage \\
    6     & BZCCEXUN Index & Brazil CNI Consumer Confidence Expectations on Unemployment                      \\
    7     & BZCCEXIN Index & Brazil CNI Consumer Confidence Income Expectations                               \\
    8     & BZUETAYL Index & BR Unemployment Rate - Taylor Rule \\
    9     & BZREOFFR Index & Brazil Sao Paulo Secovi Real Estate Units Offered \\
    10    & BZRESTRT Index & Brazil Sao Paulo Secovi Real Estate Units Started \\
    11    & BZREPERD Index & Secovi Brazil Real Estate Units Average Sale Time Period \\
    12    & BZRESOLD Index & Brazil Sao Paulo Secovi Real Estate Units Sold \\
    13    & IBREINCM Index & FGV Brazil IGP-M Construction Prices INCC-M \\
    14    & USDBRL Curncy & USDBRL Spot Exchange Rate - Price of 1 USD in BRL \\
    15    & EURBRL Curncy & EURBRL Spot Exchange Rate - Price of 1 EUR in BRL \\
    16    & JPYBRL Curncy & JPYBRL Spot Exchange Rate - Price of 1 JPY in BRL \\
    17    & USDBRLV1M Index & USDBRL 1 Month ATM Implied Volatility \\
    18    & USDBRL25R3M Index & USDBRL 3 Month 25 Delta Risk Reversal \\
    19    & BISBBRR Index & Brazil Real Effective Exchange Rate Broad \\
    20    & IBOV Index & Ibovespa Brasil Sao Paulo Stock Exchange Index \\
    21    & WCAUBRAZ Index & Bloomberg Brazil Exchange Market Capitalization USD \\
    22    & IFNCBV Index & Brazil Financial Index \\
    23    & BZLIQDTY Index & Bovespa Volume Brazil Settlement \\
    24    & IBOVIEE Index & Sao Paulo Stock Exchange Electrical Energy Index \\
    25    & CBRZ1U5 Curncy & Federative Republic of Brazil \\
    26    & BZSTSETA Index & Brazil Selic Target Rate \\
    27    & BZTJLP Index & Brazil BNDES Long Term Interest Rate TJLP \\
    28    & BZAD1Y Index & Anbima Brazil Govt Bond Fixed Rate 1 Year \\
    29    & BZAD2Y Index & Anbima Brazil Govt Bond Fixed Rate 2 Years \\
    30    & GEBR5Y Index & Brazil Government Generic Bond 5 Year \\
    31    & GEBR10Y Index & Brazil Government Generic Bond 10 Year \\
    32    & GEBU10Y Index & Brazil Government Generic Bond 10 Year USD \\
    33    & BZLNTOTA Index & Brazil Financial System Loans \\
    34    & BZLNPTOT Index & Brazil Financial Private System Loans \\
    35    & BZMBMB Index & Brazil Monetary Base \\
    36    & BZMS1 Index & Brazil Money Supply M1 Brazil M1 \\
    37    & BZMS2 Index & Brazil Money Supply M2 Brazil M2 \\
    38    & BZMS3 Index & Brazil Money Supply M3 Brazil M3 \\
    39    & BZMS4 Index & Brazil Money Supply M4 Brazil M4 \\
    40    & BRCDDEFT Index & Brazil Personal Loans More Than 90 Days Late \\
    41    & BZIDINTL Index & Brazil International Daily Reserves \\
    42    & BRCCVEHB Index & Brazil Consumer Credit Operations for Vehicle Acquisition \\
    43    & BZPIIPCA Index & Brazil CPI IPCA Dec 1993=100 \\
    44    & BZCILIVE Index & Brazil FIPE CPI Sao Paulo Living \\
    45    & BZCIFOOD Index & Brazil FIPE CPI Sao Paulo Food Main \\
    46    & BZCIPERS Index & Brazil FIPE CPI Sao Paulo Personal \\
    47    & BZCITRAN Index & Brazil CIPE CPI Sao Paulo Transportation \\
    48    & IPEAEXIN Index & Brazil IPEA Export Price Index \\
    49    & IPEAIMIN Index & Brazil IPEA Import Price Index \\
    50    & IBREIPAM Index & FGV Brazil IGP-M Wholesale Prices IPA-M \\
    51    & IBREIPA1 Index & FGV Brazil Wholesale Prices IPA-10 \\
    52    & BZICINDX Index & CNI Brazil Industrial Confidence General \\
    53    & BZCCI Index & CNI Brazil Consumer Confidence \\
    54    & OLEDBRAZ Index & OECD Brazil Composite Leading Ind. Total Trend Restored Stck \\
    55    & OEBRI003 Index & OECD Brazil Cons. Opin. Confidence Composite \& OECD Indicators SA amp adj \\
    56    & MPMIBRMA Index & Markit Brazil Manufacturing PMI SA \\
    57    & BZTBBALM Index & Brazil Trade Balance FOB Balance NSA \\
    58    & BZTBEXPM Index & Brazil Trade Balance FOB Exports \\
    59    & BZTBIMPM Index & Brazil Trade Balance FOB Imports NSA \\
    60    & BZTWBALW Index & Brazil Trade Balance Weekly Balance \\
    61    & BZDPGOD Index & Brazil General Government Net Debt \\
    62    & BZDPNDTL Index & Brazil Public Net Debt \\
    63    & BZPBPRDM Index & Brazil Public Primary Budget Result \\
    64    & BZCACURR Index & Brazil Current Account Monthly \\
    65    & BZEDTLEX Index & Brazil External Debt Brazil External Gross Debt \\
    66    & BZCA\%GDP Index & Brazil Current Account \% of GDP Last 12 Months Accumulated \\
    67    & BZFDTMON Index & Brazil Foreign Direct Investment Net \\
    68    & BZDPNDT\% Index & Brazil Public Net Debt \% of GDP \\
    69    & BSRFTOFD Index & Brazil Total Federal Revenue \\
    70    & BZBGEXPN Index & Brazil Central Government Total Expenditures \\
    71    & BZPBNODM Index & Brazil Public Nominal Budget Result \\
    72    & BZBGPRIM Index & Brazil Central Government Primary Budget Surplus/Deficit \\
    73    & BZBGNOMI Index & Brazil Central Government Nominal Budget Surplus/Deficit \\
    74    & BZIPTLSA Index & Brazil Real Industrial Production SA 2012=100 \\
    75    & BZIXEXTR Index & Brazil Industrial Production Activity Extractive Industry2012 \\
    76    & BZASSUBT Index & Brazil Auto Sales Subtotal \\
    77    & BZCNCNIS Index & CNI Brazil Manufacture Industry Capacity Utilization SA \\
    78    & BZCNSALS Index & CNI Brazil Manufacture Industry Real Sales SA 2006=100 \\
    79    & BZCNEMPS Index & CNI Brazil Manufacture Industry Employment SA 2006=100 \\
    80    & BZCNHOUS Index & CNI Brazil Manufacture Industry Working Hours SA 2006=100 \\
    81    & BZVPTLVH Index & Anfavea Brazil Vehicle Production \\
    82    & BZVLTLVH Index & Anfavea Brazil Vehicle Sales Licensed \\
    83    & BZVXEXTL Index & Anfavea Brazil Vehicle Exports \\
    84    & BZVLTOTL Index & Anfavea Brazil Vehicle Sales Licensed Cars \\
    85    & BZRTRTSA Index & Brazil Retail Sales Volume SA \\
    86    & BZRTCOSA Index & Brazil Retail Sales Volume Construction Materials Index SA \\
    87    & BZRTFDSA Index & Brazil Retail Sales Volume Supermarket Food Beverages \& Tobacco SA \\
    88    & BZRTFURN Index & Brazil Retail Sales Volume Furniture \& Domestic Appliance \\
    89    & OEBRV008 Index & OECD Brazil Prod. Manufacturing Total Manufacturing  SA 2010=100 \\
    90    & BZEASA Index & Economic Activity GDP SA IBC-BR \\
    91    & BZGDCAPX Index & Brazil GDP Qtrly Gross Formation of Fixed Capital SA 1995=100 \\
    92    & BZGDFAMX Index & Brazil GDP Qtrly Family Consumption SA 1995=100 \\
    93    & BZGDAGRX Index & Brazil GDP Qtrly Agriculture SA 1995=100 \\
    94    & BZGDIDTX Index & Brazil GDP Qtrly Industry SA 1990=100 \\
    95    & BZGDTRNX Index & Brazil GDP Qtrly  Transformation Industry SA 1995=100 \\
    96    & BZGDSERX Index & Brazil GDP Qtrly Services SA 1995=100 \\
    97    & EHGDBR Index & Brazil Real GDP (Annual YoY \%) \\


\end{longtable}


\scriptsize
\begin{longtable}{|c|c|c|}
\caption{Dataset-Indonesia}\\
\hline
\textbf{Number} & \textbf{Ticker}  & \textbf{Description} \\
\hline
\endfirsthead
\multicolumn{3}{c}%
{\tablename\ \thetable\ -- \textit{Continued from previous page}} \\
\hline
\textbf{Number} & \textbf{Ticker}  & \textbf{Description} \\
\hline
\endhead
\hline \multicolumn{3}{r}{\textit{Continued on next page}} \\
\endfoot
\hline
\endlastfoot

    1     & IDEMUNE\% Index & Indonesia Unemployment Rate \\
    2     & IDEMEMPL Index & Indonesia Number of People Emp \\
    3     & CPNFIDCU Index & BIS Indonesia Credit to Non Fi \\
    4     & EHPIID Index & Indonesia Consumer Price Index \\
    5     & IDTIBINC Index & Indonesia Business Tendency In \\
    6     & IDTIBPRD Index & Indonesia Business Tendency In \\
    7     & IDTIBTOT Index & Indonesia Business Tendency In \\
    8     & IDTINCGR Index & Indonesia Consumer Tendency In \\
    9     & IDTIIRCG Index & Indonesia Consumer Tendency In \\
    10    & IDTIHINC Index & Indonesia Consumer Tendency In \\
    11    & IDCABAL Index & Indonesia Balance of Payments \\
    12    & IDCAPORT Index & Indonesia BOP Financial Accoun \\
    13    & IDPUMANU Index & Indonesia Production Utilizati \\
    14    & IDPUTOTL Index & Indonesia Production Utilizati \\
    15    & IDCGRFY Index & Indonesia GDP Current Prices E \\
    16    & IDCGREXY Index & Indonesia GDP Current Prices E \\
    17    & IDCGRIMY Index & Indonesia GDP Current Prices E \\
    18    & IDCGRGY Index & Indonesia GDP Current Prices E \\
    19    & IDCGRHY Index & Indonesia GDP Current Prices P \\
    20    & IDWGCDN Index & Indonesia Wage for Constructio \\
    21    & IDWGSMN Index & Indonesia Wage for Household S \\
    22    & IDELHOUS Index & Indonesia Property Loans House \\
    23    & IDANITEM Index & ANZ Roy Morgan Indonesia Consu \\
    24    & USDIDR Index & USD-IDR X-RATE \\
    25    & JPYIDR Index & JPY-IDR X-RATE \\
    26    & CADIDR Curncy & CAD-IDR X-RATE \\
    27    & USDIDRV1M Index & USD-IDR OPT VOL 1M \\
    28    & USDIDR25R3M Index & USD-IDR RR 25D 3M \\
    29    & BISBIDR Index & Indonesia Real Effective Excha \\
    30    & WCAUINDO Index & Bloomberg Indonesia Exchange M \\
    31    & MXID Index & MSCI INDONESIA \\
    32    & JCI Index & JAKARTA COMPOSITE INDEX \\
    33    & IDBIRATE Index & Bank Indonesia Reference Inter \\
    34    & INDON CDS USD SR 5Y D14 Corp & INDON CDS USD SR 5Y D14 \\
    35    & BIASINVP Index & BI Indonesian Bank Val \\
    36    & IDBRLDR Index & Indonesia Bank Ratio - Loan to \\
    37    & IDBRCAR Index & Indonesia Bank Ratio - Capital \\
    38    & IDBRNIM Index & Indonesia Bank Ratio - Net Int \\
    39    & IDBRROA Index & Indonesia Bank Ratio Return on \\
    40    & GTIDR1Y Govt & INDONESIA GOVERNMENT \\
    41    & GTIDR5Y Govt & INDONESIA GOVERNMENT \\
    42    & GTIDR10Y Govt & INDONESIA GOVERNMENT \\
    43    & GTUSDID5Y Govt & REPUBLIC OF INDONESIA \\
    44    & IDGBFRGN Index & Indonesia Govt Bond Outstandin \\
    45    & ELI GIND Index & J.P. Morgan EMBIG Indonesia So \\
    46    & IDBLHOUS Index & Indonesia Outstanding Loans by \\
    47    & IDBLOTHR Index & Indonesia Outstanding Loans by \\
    48    & IDBLSHOP Index & Indonesia Outstanding Loans by \\
    49    & IDWCTOTL Index & Indonesia Working Capital Loan \\
    50    & IDWCLIO Index & ID Total Working Capital Loans \\
    51    & IDWCCONS Index & Indonesia Working Capital Loan \\
    52    & IDWCMANU Index & Indonesia Working Capital Loan \\
    53    & IDBRNPLG Index & Indonesia Bank Ratio - Non Per \\
    54    & IDBDTD Index & Indonesia All Commercial Banks \\
    55    & IDBDALLR Index & Indonesia All Commercial Banks \\
    56    & IDBDALLF Index & Indonesia All Commercial Banks \\
    57    & IDM2YOY Index & Indonesia Money Supply M2 YoY \\
    58    & IDM1YOY Index & Indonesia Money Supply M1 YoY \\
    59    & IDRMR Index & Indonesia Reserve Base Money \\
    60    & IDPGDEBT Index & Indonesia Government Portfolio \\
    61    & IDNRIRR Index & Indonesia Net Foreign Assets I \\
    62    & IDGFA Index & Indonesia Net International Re \\
    63    & IDGFFORC Index & Indonesia International Reserv \\
    64    & IDCPIY Index & Indonesia CPI YoY \\
    65    & IDCCI Index & Bank Indonesia Consumer Confid \\
    66    & MPMIIDMA Index & Nikkei Indonesia Manufacturing \\
    67    & IDANCCT Index & ANZ Roy Morgan Indonesia Consu \\
    68    & IDANFL1Y Index & ANZ Roy Morgan Indonesia Consu \\
    69    & IDANFN1Y Index & ANZ Roy Morgan Indonesia Consu \\
    70    & IDEXPY Index & Indonesia Exports YoY \\
    71    & IDBALTOL Index & Indonesia Trade Balance \\
    72    & IDEXPEGY Index & Indonesia Export Oil \& Gas YoY \\
    73    & IDIMPTLY Index & Indonesia Import Total YoY \\
    74    & IDIMPEGY Index & Indonesia Import Oil \& Gas YoY \\
    75    & IDEDTOTL Index & Indonesia External Debt Total \\
    76    & IDEDGI Index & Indonesia External Debt Govern \\
    77    & IDMPIYOY Index & Indonesia Industrial/Manufactu \\
    78    & ASEAINDO Index & Automotive Production by Indon \\
    79    & ASEAINDS Index & Automotive Sales for Indonesia \\
    80    & IDVHCLOC Index & Gaikindo Indonesia Motor Vehic \\
    81    & IDVHMTLC Index & Asosiasi Industri Sepedamotor \\
    82    & IDRSTOTY Index & Indonesia Retail Sales Survey \\
    83    & IDCETOTL Index & Indonesia Cement Consumption \\
    84    & IDTOTOT Index & Indonesia Tourist Arrivals \\
    85    & IDHTTOTL Index & Indonesia Hotel Occupancy Rate \\
    86    & OLE3INDO Index & OECD Indonesia Composite Leadi \\
    87    & EHGDID Index & Indonesia Real GDP (Annual YoY \\


\end{longtable}



\scriptsize
\begin{longtable}{|c|c|c|}
\caption{Dataset-Mexico}\\
\hline
\textbf{Number} & \textbf{Ticker}  & \textbf{Description} \\
\hline
\endfirsthead
\multicolumn{3}{c}%
{\tablename\ \thetable\ -- \textit{Continued from previous page}} \\
\hline
\textbf{Number} & \textbf{Ticker}  & \textbf{Description} \\
\hline
\endhead
\hline \multicolumn{3}{r}{\textit{Continued on next page}} \\
\endfoot
\hline
\endlastfoot
    1     & MXR4TTSA Index & Mexico Real GDP by Industry Total SA \\
    2     & MXR4CNSA Index & Mexico Real GDP by Industry Construction SA \\
    3     & MXR4MFSA Index & Mexico Real GDP by Industry Manufacturing SA \\
    4     & MXR4RESA Index & Mexico Real GDP by Industry Wholesale and Retail Trade SA \\
    5     & MXGNTTAL Index & Mexico Nominal GDP Total SA \\
    6     & MXCACUAC Index & Mexico Nominal Current Account Balance \\
    7     & MXSDPRYO Index & Mexico Supply \& Demand Private Consumption YoY \\
    8     & MXSDPUYO Index & Mexico Supply \& Demand Public Consumption YoY \\
    9     & MXSDGCFY Index & Mexico Supply \& Demand  Total SA Annual Change 2008 Pesos \\
    10    & MEXHMEXY Index & Mexico House Price Index YoY \\
    11    & MXUEUNSA Index & Mexico Unemployment Rate SA for Workers 15 and Older ENOE \\
    12    & MXWICONS Index & Mexico Formal Job Temporary \& Permanent Workers Construction \\
    13    & MXWIRETL Index & Mexico Formal Job Temporary \& Permanent Workers Retail \\
    14    & MXWITRCO Index & Mexico Formal Job Temporary \& Permanent Workers Transportation \& Communication \\
    15    & MXWICOSV Index & Mexico Formal Job Temporary \& Permanent Workers Commercial Services \\
    16    & MXWIMANU Index & Mexico Formal Job Temporary \& Permanent Workers Manufacturing \\
    17    & MXWITOTL Index & Mexico Formal Job Temporary \& Permanent Workers Total \\
    18    & MXUETEPT Index & Mexico Employment Rate \\
    19    & MXMIMITO Index & Mexico Wages by Manufacturing Industry Total \\
    20    & IMEFMNOR Index & Mexico Manufacturing Index New Orders SA \\
    21    & IMEFNMNO Index & Mexico Non Manufacturing Index New Orders SA \\
    22    & MXBLMORT Index & Mexico Bank Lending Mortgages \\
    23    & MXCSBUIL Index & Mexico Construction Spending Buildings \\
    24    & USDMXN Index & USDMXN Spot Exchange Rate - Price of 1 USD in MXN \\
    25    & JPYMXN Index & JPYMXN Spot Exchange Rate - Price of 1 JPY in MXN \\
    26    & CADMXN Curncy & CADMXN Spot Exchange Rate - Price of 1 CAD in MXN \\
    27    & USDMXNV1M Index & USDMXN 1 Month ATM Implied Volatility \\
    28    & USDMXN25R3M Index & USDMXN 3 Month 25 Delta Risk Reversal \\
    29    & BISBMXR Index & Mexico Real Effective Exchange Rate Broad \\
    30    & WCAUMEX Index & Bloomberg Mexico Exchange Market Capitalization USD \\
    31    & MEXBOL Index & Mexican Stock Exchange Mexican Bolsa IPC Index \\
    32    & MXONBR Index & Bank of Mexico Official Overnight Rate \\
    33    & MEX CDS USD SR 5Y D14 Corp & United Mexican States \\
    34    & MXFRCINR Index & BOM Government Funding Rate Closing Interest Rate \\
    35    & MPTBF CMPN Curncy & MXN T-BILL          6 MO \\
    36    & MPTB1 CMPN Curncy & MXN T-BILL          1 YR \\
    37    & GMXN02YR Index & Mexico Generic 2 Year \\
    38    & GMXN05YR Index & Mexico Generic 5 Year \\
    39    & MXLCLFCB Index & Mexico Loans from Commercial Banks \\
    40    & MXLCMOLO Index & Mexico Mortgage Loans \\
    41    & MXLCCOLO Index & Mexico Consumption Loans \\
    42    & MXBDNPLR Index & Mexico Non-Performing Loans as \% of Total Loans \\
    43    & MXLCEXSE Index & Mexico External Sector \\
    44    & MXBLCNST Index & Mexico Bank Lending Construction \\
    45    & MXBLMAIN Index & Mexico Bank Lending Manufacturing Industry \\
    46    & MXBLFFAF Index & Mexico Bank Lending Farming Forestry and Fishing \\
    47    & MXBLSAOA Index & Mexico Bank Lending Services and Other Activities \\
    48    & MXDFCONS Index & Mexico Private Sector Direct Financing Total \\
    49    & MXBLPERF Index & Mexico Bank Lending Performing Loans \\
    50    & MXBLNONB Index & Mexico Bank Lending Performing Loans for Non Bank Financial \\
    51    & MXMB Index & Mexico Monetary Base Money Base \\
    52    & MXMSM1 Index & Mexico Money Supply M1-M4 M1 Total \\
    53    & MXMSM2 Index & Mexico Money Supply M1-M4 M2 Total \\
    54    & MXMSM3 Index & Mexico Money Supply M1-M4 M3 Total \\
    55    & MXMSM4 Index & Mexico Money Supply M1-M4 M4 Total \\
    56    & MXDEINT Index & Mexico Federal Government Net Domestic Debt in Millions of Mexican Pesos \\
    57    & MXDEEXT Index & Mexico Public Sector Net External Debt in Millions of U.S. Dollars \\
    58    & MXERBUDD Index & Mexico Public Rev \& Expend Budgetary Deficit YTD \\
    59    & MXIRINUS Index & Mexico International Reserves in USD \\
    60    & MXDBPDDV Index & Mexico Development Banks Total Public Demand Deposits Volume \\
    61    & MXDBPTDV Index & Mexico Development Banks Total Public Time Deposits Volume \\
    62    & 2735E55 Index & IMF Mexico Financial Corp Deposits \\
    63    & MXPII Index & Mexico Producer Price Index \\
    64    & MFGSMANU Index & Mexico Fin Gds \& Srvs Secondary Sector Manufacturing 2012 \\
    65    & MFGSCONS Index & Mexico Fin Gds \& Srvs Secondary Sector Construction 2012 \\
    66    & MXPIIXO Index & Mexico Producer Price Index Ex Oil \\
    67    & MFGSMINE Index & Mexico Fin Gds \& Srvs Primary Sector Mining 2012 \\
    68    & MFGSELGA Index & Mexico Fin Gds \& Srvs Tertiary Water  Electricity and Gas 2012 \\
    69    & MPPRIMPT Index & Mexico International Trade Import Price NSA 1980=100 \\
    70    & MPPREXPT Index & Mexico International Trade Export Price NSA 1980=100 \\
    71    & MXCPI Index & Mexico CPI \\
    72    & MXCCCORE Index & Mexico Core CPI \\
    73    & MXCNFDAT Index & Mexico CPI Index 2010=100 Food Drinks and Tobacco \\
    74    & MXCNNFGD Index & Mexico CPI Index 2010=100 Non Food Goods \\
    75    & MXCNSERV Index & Mexico CPI Index 2010=100 Services \\
    76    & MXCNAGRI Index & Mexico CPI Index 2010=100 Agriculture \\
    77    & MXCNERAG Index & Mexico CPI Index 2010=100 Energy Rates Auth by Govt \\
    78    & MXCIHOUS Index & Mexico CPI by Expenditure Housing \\
    79    & IMEFMAIN Index & Mexico Manufacturing Index SA \\
    80    & IMEFNMIN Index & Mexico Non Manufacturing Index SA \\
    81    & IMEFMPRO Index & Mexico Manufacturing Index Production SA \\
    82    & SCMXPROI Index & Mexico Producer Confidence Indicator SA \\
    83    & MXMAAITR Index & Mexico Manufacturing Aggregate Trend Indicator \\
    84    & MXMAEXPT Index & Mexico Manufacturing Aggregate Trend Indicator Exports \\
    85    & MXMAMNOR Index & Mexico Manufacturing Aggregate Orders Indicator Manufacturing Orders SA \\
    86    & CSMXCONU Index & MX Consumer Confidence Index SA \\
    87    & CSMXPOSU Index & Mexico Compared Economic Situation with a Year Ago at Present SA \\
    88    & MXCLYLEA Index & Mexico Leading Indicator YoY \\
    89    & MXCLSALE Index & Mexico Seasonally Adjusted Leading Indicator \\
    90    & MXCLSACO Index & Mexico Seasonally Adjusted Coincident Indicator \\
    91    & MXTBBEXP Index & Mexico Trade Balance Exports Monthly Total USD Million \\
    92    & MXOTAMER Index & Petroleos Mexicanos (Pemex) Crude Oil Mexico Trade Data/Americas \\
    93    & MXOTEURO Index & Petroleos Mexicanos (Pemex) Crude Oil Mexico Trade Data/Europe \\
    94    & MXEXPETR Index & Mexico Nominal Current Account Balance \\
    95    & MXEXNONP Index & Mexico Exports by Sector Non Petroleum Mexico Exports Monthly Total USD Million \\
    96    & MXRETOT\$ Index & Mexican Remittances Money Sent from Workers Outside Mexico \\
    97    & IGAEINDX Index & Mexico Indicator of Economic Activity Index SA \\
    98    & IGAEPADI Index & Mexico Economic Activity Primary Activities Series Index SA \\
    99    & MINVCNST Index & Mexico Capital Investment Construction \\
    100   & MXPSTOTL Index & Mexico Industrial Production Total Seasonally Adjusted \\
    101   & MXPSOGSA Index & Mexico Industrial Production Oil and Gas Extraction Seasonally Adjusted \\
    102   & MXPSELEC Index & Mexico Industrial Production Utilities Seasonally Adjusted \\
    103   & MXPSCONS Index & Mexico Industrial Production Construction Seasonally Adjusted \\
    104   & MXPSMANF Index & Mexico Industrial Production Manufacturing Seasonally Adjusted \\
    105   & MXSATOTL Index & Mexico Antad Same-Store Sales Overall YoY\% \\
    106   & MXSLMOGA Index & Mexico Gasoline Sales Monthly \\
    107   & MXSLDIES Index & Mexico Diesel Sales Monthly \\
    108   & MXMNMCEQ Index & Mexico Capacity Utilization Manufacture of Machinery and Equipment \\
    109   & MXMNPECO Index & Mexico Capacity Utilization Manufacture of Petroleum Products and Coal \\
    110   & MXVPTOTL Index & Mexico Vehicle Production Total Production \\
    111   & MXWRTWHO Index & Mexico Wholesale/Retail Sale Totl Whole \\
    112   & MXVHTOTL Index & Mexican Vehicle Sales Auto+truck NSA \\
    113   & MXVETOTL Index & Mexican Vehicle Exports Total \\
    114   & MDPCSAIN Index & Mexico Total Season Adjusted Index Base 2008 \\
    115   & MINVTOSA Index & Mexico Gross Fixed Inv Total Seasonally Adjusted \\
    116   & EHGDMX Index & Mexico Real GDP (Annual YoY \%) \\


\end{longtable}


\scriptsize
\begin{longtable}{|c|c|c|}
\caption{Dataset-South Africa}\\
\hline
\textbf{Number} & \textbf{Ticker}  & \textbf{Description} \\
\hline
\endfirsthead
\multicolumn{3}{c}%
{\tablename\ \thetable\ -- \textit{Continued from previous page}} \\
\hline
\textbf{Number} & \textbf{Ticker}  & \textbf{Description} \\
\hline
\endhead
\hline \multicolumn{3}{r}{\textit{Continued on next page}} \\
\endfoot
\hline
\endlastfoot

     1     & SAUEQEMP Index & South Africa Labour- Employed \\
    2     & SAUEQABS Index & South Africa Labour - Labor Absorption Rate \\
    3     & SAUEQPRT Index & South Africa Labour - Labor Force Participation Rate \\
    4     & SAUEQNLF Index & South Africa Labour - Not in the Labor Force \\
    5     & EHUPZA Index & South Africa Unemployment Rate (\%) \\
    6     & SACWC Index & South Africa Consumer Confidence \\
    7     & SACWE Index & South Africa Consumer Confidence Economic Position in Next 12m \\
    8     & SACWF Index & South Africa Consumer Confidence Financial Position During Next 12m. \\
    9     & SACTLVL Index & South Africa Current Account SA \\
    10    & SACTMEX Index & South Africa Current Account SA - Merchandise Exports Free on Board \\
    11    & SACTGEX Index & South Africa Current Account SA - Net Gold Exports \\
    12    & SACTLMI Index & South Africa Current Account SA - Less Merchandise Imports \\
    13    & SACTCTR Index & South Africa Current Account SA - Current Transfers Net Receipts \\
    14    & SACUI Index & South Africa Utilization of Production Capacity \\
    15    & SABTHDIQ Index & South Africa Household Debt to Disposable Income of Households \\
    16    & SAGNDISA Index & South Africa Nominal Household Disposable Income SA \\
    17    & SADXFCFR Index & South Africa Real GDP Gross Fixed Capital Formation SA \\
    18    & SASGAGR Index & South Africa Agriculture SA Constant Prices \\
    19    & SASGMINE Index & South Africa Mining SA Constant Prices \\
    20    & SASGMANU Index & South Africa Manufacturing SA Constant Prices \\
    21    & SASGELEC Index & South Africa Electricity SA Constant Prices \\
    22    & SASGCON Index & South Africa Construction sa constant 2000 prices \\
    23    & SASGWRH Index & South Africa Wholesale Retail Hotels SA Constant Prices \\
    24    & SADXRGSA Index & South Africa Real GDP Expenditure on GDP \\
    25    & SATCTREM Index & Trade Activity Index Employment \\
    26    & SAPME Index & South Africa Barclays PMI Employment SA \\
    27    & SATCTRBL Index & Trade Activity Index Backlog on Orders \\
    28    & SATCTEBL Index & Trade Expectations Index  Backlog on Orders \\
    29    & SATCTRNO Index & Trade Activity Index New Orders \\
    30    & SATCTENO Index & Trade Expectations Index  New Orders \\
    31    & SACSPSTO Index & SA Recorded Building Plans Total SA \\
    32    & SACSPSRB Index & SA Recorded Building Plans Residentual Buildings SA \\
    33    & SACSPSNR Index & SA Recorded Building Plans Non-Residentual Buildings SA \\
    34    & SACSPSAA Index & SA Recorded Building Plans Additions and Alterations SA \\
    35    & SACSCSTO Index & SA Completed Buildings Recorded Total SA \\
    36    & SACSCSRB Index & SA Completed Buildings Recorded Residentual Buildings SA \\
    37    & SACSCSNR Index & SA Completed Buildings Recorded Non-Residentual Buildings SA \\
    38    & SACSCSAA Index & SA Completed Buildings Recorded Additions and Alterations SA \\
    39    & ZAR Curncy & USDZAR Spot Exchange Rate - Price of 1 USD in ZAR \\
    40    & EURZAR Curncy & EURZAR Spot Exchange Rate - Price of 1 EUR in ZAR \\
    41    & GBPZAR Curncy & GBPZAR Spot Exchange Rate - Price of 1 GBP in ZAR \\
    42    & JPYZAR Curncy & JPYZAR Spot Exchange Rate - Price of 1 JPY in ZAR \\
    43    & TRYZAR Curncy & TRYZAR Spot Exchange Rate - Price of 1 TRY in ZAR \\
    44    & USDZARV1M Index & USDZAR 1 Month ATM Implied Volatility \\
    45    & USDZAR25R3M Index & USDZAR 3 Month 25 Delta Risk Reversal \\
    46    & BISBZAR Index & South Africa Real Effective Exchange Rate Broad \\
    47    & TOP40 Index & FTSE/JSE Africa Top40 Tradeable Index \\
    48    & JFINX Index & FTSE/JSE Africa Financials Index \\
    49    & JBIND Index & FTSE/JSE Africa Basic Materials Index \\
    50    & JGIND Index & FTSE/JSE Africa Industrials Index \\
    51    & JGOLD Index & FTSE/JSE Africa Gold Mining Index \\
    52    & WCAUSAF Index & Bloomberg South Africa Exchange Market Capitalization USD \\
    53    & JALSH Index & FTSE/JSE Africa All Share Index \\
    54    & REPSOU CDS USD SR 5Y D14 Corp & Republic of South Africa \\
    55    & SARPRT Index & South Africa Repo Avg Rate \\
    56    & GSAB2YR Index & South Africa Govt Bonds 2 Year Note Generic Bid Yield \\
    57    & GSAB3YR Index & South Africa Govt Bonds 3 Year Note Generic Bid Yield \\
    58    & GSAB5YR Index & South Africa Govt Bonds 5 Year Note Generic Bid Yield \\
    59    & GSAB10YR Index & South Africa Govt Bonds 10 Year Note Generic Bid Yield \\
    60    & SALQCMPN Index & South Africa Liquidations Cos \\
    61    & SACEI Index & South Africa Private Credit Extension \\
    62    & SACEINV Index & South Africa Private Credit Extension Investments \\
    63    & SACEMORT Index & South Africa Private Credit Extension Mortgage Advances \\
    64    & SACELEAS Index & South Africa Private Credit Extension Leasing Finance \\
    65    & SACELOAN Index & South Africa Private Credit Extension Total Loans and Advances \\
    66    & SACESALE Index & South Africa Private Credit Extension Installment Sales Credit \\
    67    & SACEHOUS Index & South Africa Private Credit Extension Of Which To Households \\
    68    & SAMYSAM3 Index & South Africa Money Supply M3 Seasonally Adjusted \\
    69    & SAMYM1 Index & South Africa Money Supply M1 \\
    70    & SAMYM2 Index & South Africa Money Supply M2 \\
    71    & SAMYM0 Index & South Africa Money Supply M0 \\
    72    & 199.055 Index & IMF South Africa Foreign Exchange Reserves in Millions of USD \\
    73    & SANOGOL\$ Index & South Africa Gold Reserves \\
    74    & SANOGR\$ Index & South Africa Gross Reserves \\
    75    & 1995E55 Index & IMF South Africa Deposits in Rand \\
    76    & SACPI Index & South Africa CPI 2012=100 \\
    77    & SABCI Index & SACCI South Africa Business Confidence \\
    78    & SAPMI Index & South Africa Barclays SA \\
    79    & SAPMIPP Index & South Africa Barclays PMI Prices NSA \\
    80    & SCP8COUN Index & South Africa CPI For Total Country NSA \\
    81    & SCP8EPNY Index & South Africa Ex Food NAB Petrol \& Energy YoY \\
    82    & SCP8EENR Index & South Africa Ex Energy \\
    83    & MPMIZAWA Index & Standard Bank South Africa PMI SA                                                \\
    84    & SACBLI Index & Composite Business Cycle Indicator - Leading Indicator \\
    85    & SACBLG Index & Composite Business Cycle Indicator - Lagging Indicator \\
    86    & SACBCI Index & Composite Business Cycle Indicator - Coincident Indicator \\
    87    & SANOFP\$ Index & South Africa Net Open Foreign Currency Position \\
    88    & SABBBAL Index & South Africa Budget Summary National Budget Balance \\
    89    & SATBAL Index & South Africa Trade Balance Incl Oil Arms \& Bullion \\
    90    & SATBEX Index & South Africa Trade Balance Exports Incl Oil Arms \& Bullion \\
    91    & SATBEOTH Index & South Africa Trade Export Other Gd \\
    92    & SATBIM Index & South Africa Trade Balance Imports Incl Oil Arms \& Bullion \\
    93    & SABBEXP Index & South Africa Budget Summary National Expenditures \\
    94    & SABBREV Index & South Africa Budget Summary National Revenue \\
    95    & NAAMTTMS Index & NAAMSA South Africa Total Market Sales Level \\
    96    & SARSTCSA Index & South Africa Retail Sales Total Sales Constant Prices SA 2012=100 \\
    97    & SASRGEN Index & South Africa Retail Trade Sales Constant 2012 Prices General \\
    98    & SATWCOS Index & South Africa Wholesale Trade Constant 2000 Prices SA \\
    99    & SFMPPET Index & South Africa Manufacturing Production SA 2005=100 Petroleum Chemical Prod \\
    100   & SFPMI Index & South Africa Manufacturing Production SA 2010=100 \\
    101   & SFMPFB Index & South Africa Manufacturing Production SA 2005=100 Food \& Beverages \\
    102   & SFMPTCF Index & South Africa Manufacturing Production SA 2005=100 Textile Leather Footwear \\
    103   & SFMPMVP Index & South Africa Manufacturing Production SA 2005=100 Parts \& Other Transport Equip \\
    104   & SAMSTGSA Index & South Africa Mining Sales Total Including Gold SA \\
    105   & SAMPGDSY Index & South Africa Mining Production Volume Gold SA YoY \\
    106   & SAMPTTSY Index & South Africa Mining Production Volume Total Inc Gold SA YoY \\
    107   & SAPW09Y Index & South Africa Electricity Production Index Year on Year \% \\
    108   & SAPW08Y Index & South Africa Electricity Consumption Year on Year \% \\
    109   & EHGDZA Index & South Africa Real GDP (Annual YoY \%) \\

\end{longtable}


\scriptsize
\begin{longtable}{|c|c|c|}
\caption{Dataset-Turkey}\\
\hline
\textbf{Number} & \textbf{Ticker}  & \textbf{Description} \\
\hline
\endfirsthead
\multicolumn{3}{c}%
{\tablename\ \thetable\ -- \textit{Continued from previous page}} \\
\hline
\textbf{Number} & \textbf{Ticker}  & \textbf{Description} \\
\hline
\endhead
\hline \multicolumn{3}{r}{\textit{Continued on next page}} \\
\endfoot
\hline
\endlastfoot

    1     & TULSUR Index & Turkey Labor Statistics Unemployment Rate SA \\
    2     & TULSER Index & Turkey Labor Statistics Employment Rate SA \\
    3     & TULSCO Index & Turkey Labor Statistics Employment in Construction SA \\
    4     & TULSSER Index & Turkey Labor Statistics Employment in Services SA \\
    5     & TULSIN Index & Turkey Labor Statistics Employment in Industry SA \\
    6     & TULSAGRI Index & Turkey Labor Statistics Employment in Agriculture SA \\
    7     & TULSLPAR Index & Turkey Labor Statistics Labor Participation Rate SA \\
    8     & TULSNO Index & Turkey Labor Statistics Non Agricultural Unemployment Rate SA \\
    9     & TULSYOU Index & Turkey Labor Statistics Youth Unemployment Rate SA \\
    10    & TUPYUK1 Index & Turkey Non-Residents Holdings of Equity Stock \\
    11    & TUPYUK2 Index & Turkey Non-Residents Holdings Government Domestic Debt Securities (GDSS) Stock \\
    12    & USDTRY Index & USDTRY Spot Exchange Rate - Price of 1 USD in TRY \\
    13    & EURTRY Index & EURTRY Spot Exchange Rate - Price of 1 EUR in TRY \\
    14    & JPYTRY Curncy & JPYTRY Spot Exchange Rate - Price of 100 JPY in TRY \\
    15    & USDTRYV1M Index & USDTRY 1 Month ATM Implied Volatility \\
    16    & USDTRY25R3M Index & USDTRY 3 Month 25 Delta Risk Reversal \\
    17    & CPIXBREX Index & Turkey Real Effective Exchange Rate (2003=100) CPI \\
    18    & XU100 Index & Borsa Istanbul 100 Index \\
    19    & XBANK Index & Borsa Istanbul Banks Sector Index \\
    20    & XUSIN Index & Borsa Istanbul Industrials Sector Index \\
    21    & WCAUTURK Index & Bloomberg Turkey Exchange Market Capitalization USD \\
    22    & CTURK1U5 Curncy & Republic of Turkey \\
    23    & TUBRONRA Index & Turkey Overnight Lending Rate Announcement \\
    24    & TUBROBRA Index & Turkey Overnight Borrowing Rate Announcement \\
    25    & IECM2Y Index & Turkish Government Bond 2Y Compound Yield \\
    26    & TUBOL54 Index & Turkey Banks Balance Sheet Deposits - Residents in Dollars (\$) \\
    27    & WAIRCASH Index & Weighted Average Interest Rates for Turkish Banks Loans - Cash \\
    28    & WAIRVEHI Index & Weighted Average Interest Rates for Banks Loans - Vehicles \\
    29    & WAIRHOUS Index & Weighted Average Interest Rates for Banks Loans - Housing \\
    30    & WAIRCOMM Index & Weighted Average Interest Rates for Banks Loans - Commercial \\
    31    & GTRU2YR Index & USD Turkey Govt Bond Generic Bid Yield 2 Year \\
    32    & GTRU5YR Index & USD Turkey Govt Bond Generic Bid Yield 5 Year \\
    33    & GTRU10YR Index & USD Turkey Govt Bond Generic Bid Yield 10 Year \\
    34    & TBRDELT Index & Export Loans - Total \\
    35    & TBRDWCLT Index & Working Capital Loans - Total \\
    36    & TBRDTLTL Index & Total Loans \\
    37    & TUCRTOTL Index & Turkey Consumer Loans Total \\
    38    & DPMLAUTO Index & Deposit Money Banks Loans Private Sector - Automobile \\
    39    & DPMLINCC Index & Deposit Money Banks Loans Private Sector - Individual Credit Cards \\
    40    & DPMLHOUS Index & Deposit Money Banks Loans Private Sector - Housing \\
    41    & DPMLCOOT Index & Deposit Money Banks Loans Private Sector - Consumer \& Other \\
    42    & TUNMM1 Index & Turkey Money Supply M1 \\
    43    & TUNMM2 Index & Turkey New Money Supply M2 \\
    44    & TUNMM3 Index & Turkey New Money Supply M3 \\
    45    & TUNMTDTR Index & Turkey Money Supply Time Deposits TRY \\
    46    & TUNMSDFX Index & Turkey Money Supply Sight Deposits FX \\
    47    & TUNMSDTR Index & Turkey Money Supply Sight Deposits TRY \\
    48    & TUNMTDFX Index & Turkish Money Supply Time Deposits FX \\
    49    & TBRDLOAN Index & Turkey SME Loans Total \\
    50    & TURWL Index & Turkey Gross Foreign Exchange Reserves (Weekly) \\
    51    & TUDPPI Index & Turkey Domestic PPI  \\
    52    & TUDPC Index & Turkey Domestic PPI Manufacturing  \\
    53    & TUDPB Index & Turkey Domestic PPI Mining \& Quarrying  \\
    54    & TUDP6 Index & Turkey Domestic PPI Crude Petroleum \& Natural Gas \\
    55    & TUDP10 Index & Turkey Domestic PPI Food Products YoY \\
    56    & TUDP29 Index & Turkey Domestic PPI Motor Vehicles Trailers \& Semi-Trailers   \\
    57    & TUDP25 Index & Turkey Domestic PPI Fabricated Metal Products Except Machinery \& Equipment   \\
    58    & TUCPI Index & Turkey CPI    \\
    59    & TUCPF Index & Turkey CPI Food \& Non Alcoholic Beverages   \\
    60    & TUCPH Index & Turkey CPI Housing Water Electricity Gas \& Other Fuels   \\
    61    & TUCPHO Index & Turkey CPI Hotels Cafes \& Restaurants   \\
    62    & TUCPFH Index & Turkey CPI Furnishings Household Equipment \& Routine House Maintenance   \\
    63    & TUCPR Index & Turkey CPI Recreation \& Culture   \\
    64    & TUCXSG Index & Turkey CPI Ex Seasonal Goods   \\
    65    & TUCXEF Index & Turkey CPI Ex Energy Food Non Alcoholic Bev Alcoholic Bev Tobacco \& Gold   \\
    66    & TUCOGY2S Index & Turkey Real Sector Confidence Index Volume of Orders (Current Situation) SA \\
    67    & TUCOGY3S Index & Turkey Real Sector Confidence Stocks of Finished Goods (Current Situation) SA \\
    68    & TUCOGY7S Index & Turkey Real Sector Confidence Index Export Orders (Next 3 Months) SA \\
    69    & TUCOREAL Index & Turkey Conf IndxReal Sect \\
    70    & TUCDCONF Index & Consumer Confidence \\
    71    & TUCOGY1S Index & TU Real Sector Confidence SA \\
    72    & TUCOGY9S Index & TU Business Situation SA \\
    73    & MPMITRMA Index & Markit Turkey Manufacturing PMI  \\
    74    & TUCALVLP Index & Turkey Balance of Payments Portfolio Investment 12M YoY Level Change USD \\
    75    & TUCADIT Index & Turkey Balance of Payments Direct Investment in Turkey \\
    76    & TUDDTOTL Index & Turkey Domestic Debt Position Total \\
    77    & TUBTREV Index & Turkey Budget Deficit Revenue \\
    78    & TUTBEX Index & Turkey Trade Exports WDA  \\
    79    & TUTBIM Index & Turkey Trade Imports WDA  \\
    80    & ECOCTRN Index & Turkey Current Account Balance (Billion USD) NSA \\
    81    & TUKVDB17 Index & Turkey Short Term External Debt Stock \\
    82    & TUCSET Index & Turkey Motor Vehicle Industry Export Total \\
    83    & TUCSEP Index & Turkey Motor Vehicle Industry Export Passenger Cars \\
    84    & E50DGTR Index & EU Ind Prod Durable Consumer Goods Turkey SWDA \\
    85    & E50IGTR Index & EU Ind Prod Intermediate Goods Turkey SWDA \\
    86    & E50KGTR Index & EU Ind Prod Capital Goods Turkey SWDA \\
    87    & TUIOMT Index & Turkey Industrial Production Manufacturing 2010=100 \\
    88    & TUINTURN Index & Turkey Industry Turnover 2010=100 \\
    89    & TUIOSA Index & Turkey Industrial Production SWDA 2010=100 \\
    90    & TUIOST Index & Turkey Industrial Production Mining 2010=100 \\
    91    & TUIOET Index & Turkey Industrial Production Electricity 2010=100 \\
    92    & TYCOLEVS Index & Turkey Capacity Utilization SA \\
    93    & TUCSPT Index & Turkey Motor Vehicle Industry Production Total \\
    94    & TUCSMT Index & Turkey Motor Vehicle Industry Sales Total \\
    95    & TUCSMP Index & Turkey Motor Vehicle Industry Sales Passenger Cars \\
    96    & TUTOARTO Index & Turkey Tourism Arriving Visitors Total \\
    97    & EHBBTR Index & Turkey Budget Balance (\% GDP) \\
    98    & EHCATR Index & Turkey Current Account Balance (\% GDP) \\
    99    & TUQRRESY Index & Turkey GDP at Constant Prices Final Consumption Expenditure of Residents YoY \\
    100   & TUQRGFY Index & Turkey Real GDP Imports of Goods and Services WDA YoY \\
    101   & TUQRIMY Index & Turkey GDP Transportation \& Storage Constant Prices SWDA \\
    102   & TUGPIGDY Index & Turkey Real GDP (Annual YoY \%) \\

\end{longtable}



\scriptsize
\begin{longtable}{|c|c|c|}
\caption{Uncertainty indices}\\
\hline
\textbf{Number} & \textbf{Ticker}  & \textbf{Description} \\
\hline
\endfirsthead
\multicolumn{3}{c}%
{\tablename\ \thetable\ -- \textit{Continued from previous page}} \\
\hline
\textbf{Number} & \textbf{Ticker}  & \textbf{Description} \\
\hline
\endhead
\hline \multicolumn{3}{r}{\textit{Continued on next page}} \\
\endfoot
\hline
\endlastfoot
    1     & EPUCCUSM Index & US Economic Policy Uncertainty Composite Index                                   \\
    2     & EPUCTRAD Index & US Categorical Economic Policy Uncertainty Trade Policy                          \\
    3     & EPUCCEUM Index & European Economic Policy Uncertainty Composite Index                             \\
    4     & EPUCNUSM Index & US Economic Policy Uncertainty Index                                             \\
    5     & EPUCUK Index & United Kingdom Economic Policy Uncertainty Index                                 \\
    6     & EPUCMONE Index & US Categorical Economic Policy Uncertainty Monetary Policy                       \\
    7     & EPUCIT Index & Italy Economic Policy Uncertainty Index                                          \\
    8     & CONSHOMS Index & UMich Buying Conditions for Houses: Uncertain                                    \\
    9     & EPUCJP Index & Japan Economic Policy Uncertainty                                                \\
    10    & EPUCBRAZ Index & Brazil Economic Policy Uncertainty                                               \\
    11    & EPUCAUST Index & Australia Economic Policy Uncertainty \\
    12    & EPUCDE Index & Germany Economic Policy Uncertainty \\
    13    & EPUCFR Index & France Economic Policy Uncertainty \\
    14    & SBOIUNCR Index & NFIB Small Business Uncertainty \\
    15    & EPUCFINR Index & US Categorical Economic Policy Uncertainty Financial Regulation                  \\
    16    & EPUCNINM Index & India Economic Policy Uncertainty Index                                          \\
    17    & EPUCTAX Index & US Categorical Economic Policy Uncertainty Taxes                                 \\
    18    & EPUCGOVT Index & US Categorical Economic Policy Uncertainty Government Spending                   \\
    19    & EPUCSP Index & Spain Economic Policy Uncertainty Index                                          \\
    20    & EPUCCTGR Index & US Categorical Economic Policy Uncertainty                                       \\
    21    & EPUCNATL Index & US Categorical Economic Policy Uncertainty National Security                     \\
    22    & EPUCFISC Index & US Categorical Economic Policy Uncertainty Fiscal Policy                         \\
    23    & EPUCCAND Index & Canada Economic Policy Uncertainty Index                                         \\
    24    & EPUCSWED Index & Sweden Economic Policy Uncertainty                                               \\
    25    & FEPUUSE Index & USA Migration Economic Policy Uncertainty Index                                  \\
    26    & FEPUGEE Index & Germany Migration Economic Policy Uncertainty Index                              \\
    27    & EPUCSKOR Index & South Korea Economic Policy Uncertainty \\
    28    & EPUCENTI Index & US Categorical Economic Policy Uncertainty Entitlement Programs                  \\
    29    & EPUCREGU Index & US Categorical Economic Policy Uncertainty Regulation                            \\
    30    & EPUCRU Index & Russia Economic Policy Uncertainty \\
    31    & EPUCNL Index & Netherlands Economic Policy Uncertainty \\
    32    & CONSDURS Index & UMich Buying Conditions for Large Household Durables: Uncertain                  \\
    33    & EPUCCHIL Index & Chile Economic Policy Uncertainty \\
    34    & EPUCIREL Index & Ireland Economic Policy Uncertainty \\
    35    & FEPUFRE Index & France Migration Economic Policy Uncertainty Index                               \\
    36    & FEPUGBE Index & UK Migration Economic Policy Uncertainty Index                                   \\
    37    & http://www.policyuncertainty.com/ & Japan Categorical Fiscal Policy Uncertainty Index \\
    38    & http://www.policyuncertainty.com/ & Japan Categorical Monetary Policy Uncertainty Index \\
    39    & http://www.policyuncertainty.com/ & Japan Categorical Trade Policy Uncertainty Index \\
    40    & http://www.policyuncertainty.com/ & Japan Categorical Exchange Rate Policy Uncertainty Index \\
    41    & http://www.policyuncertainty.com/ & US Equity Market Uncertainty Index \\
    42    & http://www.policyuncertainty.com/ & Mexico Economic Policy Uncertainty \\
    43    & http://www.policyuncertainty.com/ & Hong Kong Economic Policy Uncertainty \\
    44    & http://www.policyuncertainty.com/ & China Economic Policy Uncertainty \\
    45    & http://www.policyuncertainty.com/ & Singapore Economic Policy Uncertainty \\


\end{longtable}


\scriptsize
\begin{longtable}{|c|c|c|}
\caption{Surprise indices}\\
\hline
\textbf{Number} & \textbf{Ticker}  & \textbf{Description} \\
\hline
\endfirsthead
\multicolumn{3}{c}%
{\tablename\ \thetable\ -- \textit{Continued from previous page}} \\
\hline
\textbf{Number} & \textbf{Ticker}  & \textbf{Description} \\
\hline
\endhead
\hline \multicolumn{3}{r}{\textit{Continued on next page}} \\
\endfoot
\hline
\endlastfoot
      1     & GSERMEM Index & Goldman Sachs MAP Economic Surprise Index - EM \\
    2     & CESICNY Index & Citi Economic Surprise Index - China \\
    3     & GSERMUS Index & Goldman Sachs MAP Economic Surprise Index - US \\
    4     & GSERMWD Index & Goldman Sachs MAP Economic Surprise Index - Global \\
    5     & CESIG10 Index & Citi Economic Surprise Index - Major Economies \\
    6     & CESIJPY Index & Citi Economic Surprise - Japan \\
    7     & CESIUSD Index & Citi Economic Surprise - United States \\
    8     & CESIEUR Index & Citi Economic Surprise Index - Eurozone \\
    9     & CESIEM Index & Citi Economic Surprise Index - Emerging Markets \\
    10    & ECSURPUS Index & Bloomberg ECO US Surprise Index \\
    11    & CESIGBP Index & Citi Economic Surprise - United Kingdom \\
    12    & CESIGL Index & Citi Economic Surprise Index - Global \\
    13    & CESICAD Index & Citi Economic Surprise - Canada \\
    14    & CESIAUD Index & Citi Economic Surprise Index - Australia \\
    15    & CSIIUSD Index & Citi Inflation Surprise Index - United States \\
    16    & CSIIGL Index & Citi Inflation Surprise Index - Global \\
    17    & ECSURPEA Index & Bloomberg ECO Euro Area Surprise Index \\
    18    & CESIAPAC Index & Citi Economic Surprise Index   - Asia Pacific \\
    19    & CESINZD Index & Citi Economic Surprise Index-New Zealand \\
    20    & CESISEK Index & Citi Economic Surprise Index - Sweden \\
    21    & CSIIEUR Index & Citi Inflation Surprise Index - Eurozone \\
    22    & ECSUHOUS Index & Bloomberg ECO US Housing and Real Estate Market Surprise Index \\
    23    & CSIIEM Index & Citi Inflation Surprise Index - Emerging Markets \\
    24    & CESICHF Index & Citi Economic Surprise Index - Switzerland \\
    25    & CESILTAM Index & Citi Economic Surprise Index - Latin America \\
    26    & CESIEMXP Index & Citi Economic Surprise Index - Emerging Markets Exports \\
    27    & CESIG10F Index & Citi Economic Surprise Index - Major Economies Fixed Weight \\
    28    & ECSUSUUS Index & Bloomberg ECO US Surveys \& Business Cycle Indicators Surprise Index \\
    29    & CSIIG10 Index & Citi Inflation Surprise Index - Major Economies \\
    30    & CSIIJPY Index & Citi Inflation Surprise Index - Japan \\
    31    & ECSURPGB Index & Bloomberg ECO UK Surprise Index \\
    32    & CSIICNY Index & Citi Inflation Surprise Index - China \\
    33    & GSERMEA Index & Goldman Sachs MAP Economic Surprise Index - Euro Area \\
    34    & CESIBRIC Index & Citi Economic Surprise Index - BRIC \\
    35    & CSIIGBP Index & Citi Inflation Surprise Index - United Kingdom \\
    36    & ECSUINUS Index & Bloomberg ECO US Industrial Sector Surprise Index \\
    37    & CSIIDE Index & Citi Inflation Surprise Index - Germany \\
    38    & ITMRBI Index & Brazil Itau Economics Surprise Index \\
    39    & CESIEMFW Index & Citi Economic Surprise Index - Emerging Markets Fixed Weight \\
    40    & CESICMEA Index & Citi Economic Surprise Index - CEEMEA \\
    41    & CSIIAUD Index & Citi Inflation Surprise Index - Australia \\
    42    & CSIILTAM Index & Citi Inflation Surprise Index - Latin America \\
    43    & CSIIAPAC Index & Citi Inflation Surprise Index - Asia Pacific \\
    44    & GSERMCA Index & Goldman Sachs MAP Economic Surprise Index - Canada \\
    45    & WSUREURP Index & Westpac Positive Surprise Euro \\
    46    & ITMRMI Index & Mexico Economic Itau Surprise Index \\
    47    & GSERMDM Index & Goldman Sachs MAP Economic Surprise Index - DM \\
    48    & CSIICAD Index & Citi Inflation Surprise Index - Canada \\
    49    & WSUREURS Index & Westpac Size of Surprise Euro \\
    50    & CSIIFR Index & Citi Inflation Surprise Index - France \\
    51    & CSIISEK Index & Citi Inflation Surprise Index - Sweden \\
    52    & GSERMCE Index & Goldman Sachs MAP Economic Surprise Index - CEEMEA \\
    53    & GSERMJP Index & Goldman Sachs MAP Economic Surprise Index - Japan \\
    54    & ITMRLAI Index & Latin America Itau Surprise Index \\
    55    & WSURCADP Index & Westpac Positive Surprise Canada \\
    56    & WSURDMRP Index & Westpac Positive Surprise Developed Markets \\
    57    & CSIINOK Index & Citi Inflation Surprise Index - Norway \\
    58    & GSERMUK Index & Goldman Sachs MAP Economic Surprise Index - UK \\
    59    & GSERMLA Index & Goldman Sachs MAP Economic Surprise Index - LatAm \\
    60    & CSIIBRIC Index & Citi Inflation Surprise Index - BRIC \\
    61    & CSIICMEA Index & Citi Inflation Surprise Index - CEEMEA \\
    62    & GSERMAJ Index & Goldman Sachs MAP Economic Surprise Index - AeJ \\
    63    & SCGRMYES Index & Standard Chartered Economic Surprise Index - Malaysia \\
    64    & WSURCHFS Index & Westpac Size of Surprise Switzerland \\
    65    & WSURJPYP Index & Westpac Positive Surprise Japan \\
    66    & WSURNOKP Index & Westpac Positive Surprise Norway \\
    67    & WSURSEKP Index & Westpac Positive Surprise Sweden \\
    68    & WSURUSAS Index & Westpac Size of Surprise US \\
    69    & CSIIES Index & Citi Inflation Surprise Index - Spain \\
    70    & CSIINZD Index & Citi Inflation Surprise Index - New Zealand \\
    71    & WSURCHFP Index & Westpac Positive Surprise Switzerland \\
    72    & WSURNZDP Index & Westpac Positive Surprise New Zealand \\
    73    & WSURAUDS Index & Westpac Size of Surprise Australia \\
    74    & WSURUKP Index & Westpac Positive Surprise UK \\


\end{longtable}

 \end{document}
