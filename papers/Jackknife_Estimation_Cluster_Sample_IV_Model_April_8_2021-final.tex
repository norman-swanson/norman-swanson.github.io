%2multibyte Version: 5.50.0.2960 CodePage: 936


\documentclass[12pt]{article}
%%%%%%%%%%%%%%%%%%%%%%%%%%%%%%%%%%%%%%%%%%%%%%%%%%%%%%%%%%%%%%%%%%%%%%%%%%%%%%%%%%%%%%%%%%%%%%%%%%%%%%%%%%%%%%%%%%%%%%%%%%%%%%%%%%%%%%%%%%%%%%%%%%%%%%%%%%%%%%%%%%%%%%%%%%%%%%%%%%%%%%%%%%%%%%%%%%%%%%%%%%%%%%%%%%%%%%%%%%%%%%%%%%%%%%%%%%%%%%%%%%%%%%%%%%%%
\usepackage{amssymb}
\usepackage{amsfonts}
\usepackage{amsmath}
\usepackage{geometry}
\usepackage[onehalfspacing]{setspace}
\usepackage{setspace}

\setcounter{MaxMatrixCols}{10}
%TCIDATA{OutputFilter=LATEX.DLL}
%TCIDATA{Version=5.00.0.2606}
%TCIDATA{Codepage=936}
%TCIDATA{<META NAME="SaveForMode" CONTENT="1">}
%TCIDATA{BibliographyScheme=Manual}
%TCIDATA{Created=Sunday, July 18, 2004 16:10:34}
%TCIDATA{LastRevised=Friday, April 09, 2021 21:00:19}
%TCIDATA{<META NAME="GraphicsSave" CONTENT="32">}
%TCIDATA{<META NAME="DocumentShell" CONTENT="General\Blank Document">}
%TCIDATA{Language=American English}
%TCIDATA{CSTFile=article.cst}
%TCIDATA{PageSetup=72,72,72,72,0}
%TCIDATA{AllPages=
%F=36,\PARA{038<p type="texpara" tag="Body Text" >\hfill \thepage}
%}


\newtheorem{theorem}{Theorem}
\newtheorem{acknowledgement}[theorem]{Acknowledgement}
\newtheorem{algorithm}[theorem]{Algorithm}
\newtheorem{axiom}[theorem]{Axiom}
\newtheorem{case}[theorem]{Case}
\newtheorem{claim}[theorem]{Claim}
\newtheorem{conclusion}[theorem]{Conclusion}
\newtheorem{condition}[theorem]{Condition}
\newtheorem{conjecture}[theorem]{Conjecture}
\newtheorem{corollary}[theorem]{Corollary}
\newtheorem{criterion}[theorem]{Criterion}
\newtheorem{definition}[theorem]{Definition}
\newtheorem{example}[theorem]{Example}
\newtheorem{exercise}[theorem]{Exercise}
\newtheorem{lemma}[theorem]{Lemma}
\newtheorem{notation}[theorem]{Notation}
\newtheorem{problem}[theorem]{Problem}
\newtheorem{proposition}[theorem]{Proposition}
\newtheorem{remark}[theorem]{Remark}
\newtheorem{solution}[theorem]{Solution}
\newtheorem{summary}[theorem]{Summary}
\newenvironment{proof}[1][Proof]{\noindent\textbf{#1.} }{\ \rule{0.5em}{0.5em}}
\renewcommand{\baselinestretch}{1.3} 
\textwidth=6.8in
\textheight=8.7in
\oddsidemargin=0in
\evensidemargin=0in
\topmargin=0in
\baselineskip=10pt
\linespread{1.3}
\input{tcilatex}
\geometry{left=1in,right=1in,top=1.25in,bottom=1.25in}

\begin{document}

\title{\textbf{Jackknife Estimation of a Cluster-Sample IV Regression Model
with Many Weak Instruments\thanks{%
Corresponding author: John C. Chao, Department of Economics, 7343 Preinkert
Drive, University of Maryland, chao@econ.umd.edu. Norman R. Swanson,
Department of Economics, 9500 Hamilton Street, Rutgers University,
nswanson@econ.rutgers.edu. Tiemen Woutersen, Department of Economics, 1130 E
Helen Street, University of Arizona, woutersen@arizona.edu. The authors owe
special thanks to Jerry Hausman and Whitney Newey for many discussions on
the topic of this paper over a number of years. In addition, thanks are owed
to the particpants of the 2019 MIT Conference Honoring Whitney Newey for
comments on and advice given about the topic of this paper. Finally, the
authors wish to thank Miriam Arden for excellent research assistance. Chao
thanks the University of Maryland for research support, and Woutersen's work
was supported by an Eller College of Management Research Grant.}}}
\author{John C. Chao, Norman R. Swanson, and Tiemen Woutersen}
\date{April 8, 2021}
\maketitle

\begin{spacing}{1.0}
\begin{abstract}
This paper proposes new jackknife IV estimators that are robust to the
effects of many weak instruments and error heteroskedasticity in a cluster
sample setting with cluster-specific effects and possibly many included
exogenous regressors. The estimators that we propose are designed to
properly partial out the cluster-specific effects and included exogenous
regressors while preserving the re-centering property of the jackknife
methodology. To the best of our knowledge, our proposed procedures provide
the first consistent estimators under many weak instrument asymptotics in
the setting considered. We also present results on the asymptotic normality
of our estimators and show that t-statistics based on our estimators are asymptotically normal under the
null and consistent under fixed alternatives. Our Monte Carlo results
further show that our t-statistics perform better in controlling size in
finite samples than those based on alternative jackknife IV procedures
previously introduced in the literature.
\end{abstract}
\end{spacing}

\noindent \textit{Keywords: }Cluster sample, instrumental variables,
heteroskedasticity, jackknife, many weak instruments, panel data

\medskip

\noindent \textit{JEL classification: }C12, C13, C23, C26, C38

\noindent \setcounter{page}{1}

\section{Introduction}

\noindent \qquad The problem of endogeneity remains central to economics,
despite the vast literature on the topic. One key reason for this is that
there are many different regression settings for which endogeneity is an
issue, but for which valid estimators are not currently available. One such
setting involves the case where the objective is to estimate an IV
regression with fixed effects using panel or cluster-sampled data in
situations where the number of available instruments may be large, but where
the instruments themselves are all only weakly correlated with the
endogenous regressors. There is now a substantial literature on estimation
and inference under many weak instruments, including Chao and Swanson
(2005), Stock and Yogo (2005), Hansen, Hausman, and Newey (2008), Hausman et
al. (2012), Chao et al. (2012, 2014), Bekker and Crudu (2015), Crudu,
Mellace, and S\'{a}ndor (2020), and Mikusheva and Sun (2020). However, the
analyses given in these papers are for cross-sectional data, thus precluding
panel data or cluster sampling settings where there is additional unobserved
heterogeneity modeled by fixed or cluster-specific effects. Moreover, even
in the cross-sectional context, 2SLS and the LIML estimators are not well
behaved under many weak instruments. In particular, Chao and Swanson (2005)
and Stock and Yogo (2005) show that the 2SLS estimator is inconsistent under
many weak instrument asymptotics, even when the errors are homoskedastic. In
addition, Hausman et al. (2012) and Chao et al. (2012) both point out that
LIML is also inconsistent under many weak instruments, when there is error
heteroskedasticity. Estimators that are robust to the effects of many weak
instruments in cross sectional settings with error heteroskedasticity turn
out to have a jackknife form, as discussed in Chao and Swanson (2004). These
include the JIVE1 and JIVE2 estimators studied in Angrist, Imbens, and
Krueger (1999), for example. For further discussion, see Phillips and Hale
(1977), Blomquist and Dahlberg (1999), Ackerberg and Devereux (2009), and
Bekker and Crudu (2015). These papers again only study various versions of
the jackknife IV estimator in a cross-sectional setup without fixed effects.

The goal of this paper is to consider the problem of many weak instruments
in a panel data or cluster-sampling framework with fixed or cluster specific
effects. In addition to the presence of unobserved heterogeneity, our setup
allows the structural equation of interest to have a partially linear form
so that additional exogenous regressors can enter the equation nonlinearly.
In this sense, our paper is also related to recent work by Cattaneo,
Jansson, and Newey (2018a,b) on the partially linear model. However, the
focus of these papers differs from ours, as they do not consider the problem
of endogeneity. Thus, rather than employing IV estimators, estimation is
done using an OLS estimator, with the nonlinear component being first
approximated nonparametrically by a set of basis functions.

To consistently estimate the parameters of an IV regression with fixed or
cluster-specific effects, we propose three new estimators, which we refer to
by the acronyms FEJIV, FELIM, and FEFUL. These estimators are so named as
they are modified versions and generalizations, respectively, of the
jackknife IV (JIV), the LIML, and the Fuller (1977) estimators. In contrast
to the original JIV, LIML, and Fuller estimators, our new estimators are
designed to be robust to the effects of many weak instruments and error
heteroskedasticity, even in the presence of additional complications caused
by having fixed or cluster-specific effects and many included exogenous
regressors. To achieve consistency in our setting requires an estimator that
not only properly partials out additional covariates and cluster-specific
effects, but at the same time is also properly centered in a form similar to
a degenerate U-statistic. It turns out that accomplishing both of these
objectives simultaneously is quite challenging. While a number of innovative
JIV-type estimators have been proposed recently (see, for example, the
improved jackknife estimators, IJIVE1 and IJIVE2, of Ackerberg and Devereux
(2009), and the UJIVE estimator of Koles\'{a}r (2013)), due to the
aforementioned difficulties, these estimators are not consistent when
applied to our setting under many weak instrument asymptotics, as we shall
elaborate in greater detail in Section 2. On the other hand, the estimation
procedures that we introduce here are carefully designed to properly partial
out the presence of fixed or cluster-specific effects and included exogenous
regressors, while preserving the re-centering property of the jackknife
methodology. To the best of our knowledge, the estimators presented here are
the first consistent estimators under many weak instrument asymptotics in an
IV regression model with fixed or cluster-specific effects and possibly many
included exogenous regressors. In addition to consistency, we also establish
the asymptotic normality of the FELIM and FEFUL estimators\footnote{%
We do not provide a formal proof of the asymptotic normality of the FEJIV
estimator because the results of our Monte Carlo study, as reported in
Section 5, show that FELIM and FEFUL tend to have better finite sample
properties than FEJIV. For this reason, we shall focus the presentation of
our theoretical results on FELIM\ and FEFUL only. However, one can easily
show, by slightly modifying the arguments that we give for FELIM and FEFUL,
that FEJIV is also asymptotically normal, under many weak instrument
asymptotics. Note also that our simulation finding regarding the properties
of FEJIV are consistent with the findings of Davidson and MacKinnon (2006).}.

This paper also provides a number of results showing that hypothesis testing
procedures based on FELIM and FEFUL are robust to the effects of many weak
instruments. In particular, we construct t-statistics based on these two
estimators and show that, when the null hypothesis is true, these
t-statistics converge to an asymptotic standard normal distribution under
both standard (strong but fixed number of instruments) asymptotics and also
under many weak instrument asymptotics. Moreover, our t-statistics are shown
to be consistent in the sense that under fixed alternatives they diverge,
with probability approaching one, in the direction of the alternative
hypothesis.

The many-weak-instrument asymptotic framework used in this paper to analyze
the performance of FELIM and FEFUL was first proposed in Chao and Swanson
(2005). This framework extends earlier work by Morimune (1983) and Bekker
(1994) on what has become known in the IV literature as the many-instrument
asymptotics or \textquotedblleft Bekker asymptotics\textquotedblright ,
whereby a large sample approximation is carried out by considering an
alternative sequence where the number of instruments is allowed to approach
infinity as the sample size grows to infinity. A key difference between the
Bekker asymptotic framework and the many-weak-instrument asymptotic
framework is the rate of growth of the so-called concentration parameter. As
has been pointed out by Phillips (1983) and Rothenberg (1984), among others,
the concentration parameter is the natural measure of instrument strength in
a linear IV model. In the original papers by Morimune (1983) and Bekker
(1994), the concentration parameter is assumed to grow at the same rate as
the sample size, which is also what is assumed under standard (strong but
fixed number of instruments) asymptotics, whereas the many-weak-instrument
asymptotic framework allows the concentration parameter to grow at a rate
much slower than the sample size, thus allowing for much weaker instruments.
Let $\mu _{n}^{2}$ be a sequence that gives the rate of growth of the
concentration parameter, and let $K_{2,n}$ denote the number of instruments.
Chao and Swanson (2005) show that for consistent point estimation to be
possible, a sufficient condition is $\sqrt{K_{2,n}}/\mu _{n}^{2}\rightarrow
0 $, as $K_{2,n}$, $\mu _{n}^{2},n\rightarrow \infty $. This allows for the
possibility that $\mu _{n}^{2}$ is of an order smaller than $K_{2,n}$ which,
in turn, can be of an order much smaller than the sample size $n$. The
original Bekker framework, on the other hand, requires $K_{2,n},\mu
_{n}^{2}, $ and $n$ to all be of the same order of magnitude. Recent work by
Mikusheva and Sun (2020) indicates that the condition $\sqrt{K_{2,n}}/\mu
_{n}^{2}\rightarrow 0$, as $K_{2,n}$, $\mu _{n}^{2},n\rightarrow \infty $ is
not only sufficient but also necessary for consistency in point estimation
and hypothesis testing.

It should be noted that the hypothesis testing procedures considered in this
paper will not be valid under the weak instrument asymptotics proposed in
Staiger and Stock (1997). However, we do not find this to be a problem. The
reason is that the Staiger-Stock framework considers a setting where $\mu
_{n}^{2}=O\left( 1\right) $; hence, the instruments are so weak that the IV
model is not point identified. As a result, two of the properties that we
are particularly interested in, and that we think are most important for
reliable practical applications (namely, consistency in point estimation and
test consistency), will both be unachievable in their framework. In
addition, the results of a survey of the applied literature reported in
Hansen, Hausman, and Newey (2008) suggest that instruments used in empirical
microeconomic applications are typically not so weak that one must resort to
the type of large sample approximations given by the Staiger-Stock weak
instrument asymptotics.

The rest of the paper is organized as follows. Section 2 states the model,
defines the FELIM, FEFUL, and FEJIV estimators, and provides an explanation
of how our estimators improve upon various alternative jackknife IV
estimators that have previously been proposed in the literature. Analytical
results presented in Section 3 establish that our estimators are consistent
and asymptotically normally distributed. Section 4 shows how to estimate the
variances of the estimators and also provides asymptotic results for
t-statistics based on our estimators. Section 5 contains the results of a
series of Monte Carlo experiments in which the relative performance of our
estimators is compared with that of extant estimators in the literature.
Section 6 concludes. Proofs of Theorem 1, Corollary 1, and Theorems 4-6 are
presented in the Appendix to this paper. The proofs of Theorems 2 and 3 are
longer and are given in a supplemental Appendix, along with the proofs of a
number of supporting lemmas.

Before proceeding, we will first say a few words about some of the commonly
used notations in this paper. In what follows, we use $\lambda _{\min
}\left( A\right) $, $\lambda _{\max }\left( A\right) $, and $tr\left(
A\right) $ to denote, respectively, the minimal eigenvalue, the maximal
eigenvalue, and the trace of a square matrix $A,$ whereas $A^{\prime }$
denotes the transpose of a (not necessarily square) matrix $A$. $\left\Vert
a\right\Vert _{2}$ denotes the usual Euclidean norm when applied to a
(finite-dimensional) vector $a$. On the other hand, for a matrix $A$, $%
\left\Vert A\right\Vert _{2}\equiv \max \left\{ \sqrt{\lambda \left(
A^{\prime }A\right) }:\lambda \left( A^{\prime }A\right) \text{ is an
eigenvalue of }A^{\prime }A\right\} $ denotes the matrix spectral norm,
while $\left\Vert A\right\Vert _{F}\equiv \sqrt{tr\left\{ A^{\prime
}A\right\} }$ denotes the Frobenius norm and $\left\Vert A\right\Vert
_{\infty }\equiv \max_{1\leq i\leq
m_{n}}\dsum\nolimits_{j=1}^{m_{n}}\left\vert a_{ij}\right\vert $ (i.e., the
maximal row sum of an $m_{n}\times m_{n}$ matrix). In addition, we use $%
A\circ B$ to denote the Hadamard product of two conformable matrices $A$ and 
$B$ (i.e., $A\circ B\equiv \left[ a_{ij}b_{ij}\right] ,$ for $A=\left[ a_{ij}%
\right] $ and $B=\left[ b_{ij}\right] ).$ We take $D\left( a\right) $ to be
a diagonal matrix whose diagonal elements correspond with the elements of
the vector $a,$ while $D\left( A\right) $ is taken to be a diagonal matrix
whose diagonal elements are the same as the diagonal elements of the square
matrix $A$. Furthermore, we will let $\iota _{p}=\left( 1,1,...,1\right)
^{\prime }$ denote a $p\times 1$ vector of ones. Finally, we use CS and T,
respectively, to denote the Cauchy-Schwarz and the triangle inequality, and
the abbreviation w.p.a.1 stands for \textquotedblleft with probability
approaching one\textquotedblright .

\section{Model, Assumptions, and Estimation Procedures}

The model that we consider is a cluster-sample IV regression model

\begin{eqnarray}
\underset{1\times 1}{y_{\left( i,t\right) }} &=&X_{\left( i,t\right)
}^{\prime }\delta _{0}+\varphi _{n}\left( W_{1,\left( i,t\right) }\right)
+\alpha _{i}+\varepsilon _{\left( i,t\right) },  \label{structural eqn} \\
X_{\left( i,t\right) } &=&\Phi _{n}\left( W_{1,\left( i,t\right) }\right)
+\Upsilon _{n}\left( W_{2,\left( i,t\right) }\right) +\xi _{i}+U_{\left(
i,t\right) },  \label{1st stage eqn}
\end{eqnarray}%
where $i=1,...,n,$ $t=1,...,T_{i},$ and the total sample size is given by $%
m_{n}=\dsum\nolimits_{i=1}^{n}T_{i}$. The notation $\left( i,t\right) :%
\mathbb{N\times N\rightarrow }$ $\mathbb{N}$ denotes a pairing function
which maps an ordered pair of natural numbers into a natural number, so
that, in particular, we have $\left( 1,1\right) =1$, $\left( 1,T_{1}\right)
=T_{1}$, $\left( 2,1\right) =T_{1}+1$, and $\left( n,T_{n}\right) =m_{n}$.
This is just a notational device used to convert a double index into a
single index, thus, facilitating certain vectorization and summation
operations while still allowing one to keep track of both $i$ and $t$. In
this setup, we take $X_{\left( i,t\right) }$ to be a $d\times 1$ vector of
endogenous regressors, and we let $W_{1,\left( i,t\right) }$ and $%
W_{2,\left( i,t\right) }$ denote, respectively, a $p_{1}\times 1$ vector and
a $p_{2}\times 1$ vector of exogenous variables, for $i=1,2,...,n$ and $%
t=1,...,T_{i}$ (or, equivalently, for $\left( i,t\right) =1,...,m_{n}$).
Note that $\varphi _{n}\left( \cdot \right) $, $\Phi _{n}\left( \cdot
\right) $, and $\Upsilon _{n}\left( \cdot \right) $ are allowed to be
nonlinear functions, so that the structural equation (\ref{structural eqn})
can be taken to be a partially linear equation, and the system of
first-stage equations given by (\ref{1st stage eqn}) may be interpreted as a
generalized additive model in the sense of Hastie and Tibshirani (1990). In
addition, $\alpha _{i}$ and $\xi _{i}$ in the above equations are unobserved
or individual effects interpreted as \textquotedblleft fixed
effects\textquotedblright\ in the sense that although we do not necessarily
require $\alpha _{i}$ and $\xi _{i}$ to be (non-random) constants, they are
allowed to be correlated with the exogenous variables $W_{1,\left(
i,t\right) }$ and $W_{2,\left( i,t\right) }$, unlike the typical assumptions
specified in a traditional \textquotedblleft random
effects\textquotedblright\ model. More precise assumptions on the model
given by equations (\ref{structural eqn}) and (\ref{1st stage eqn}) are
given below.

We will develop some additional notations before proceeding. First, let

\noindent $W_{\left( i,t\right) }=\left( W_{1,\left( i,t\right) }^{\prime
},W_{2,\left( i,t\right) }^{\prime }\right) ^{\prime },$ for $\left(
i,t\right) =1,...,m_{n}$, and define

\noindent $W_{n}=\left( W_{\left( 1,1\right) },..,W_{\left( 1,T_{1}\right)
},...,W_{\left( n,1\right) },..,W_{\left( n,T_{n}\right) }\right) ^{\prime }$%
. Now, stacking the observations $\left( i,t\right) =1,...,m_{n}$, we can
write the model given by equations (\ref{structural eqn}) and (\ref{1st
stage eqn}) more succinctly as%
\begin{eqnarray}
\underset{m_{n}\times 1}{y} &=&\underset{m_{n}\times d}{X}\underset{d\times 1%
}{\delta _{0}}+\underset{m_{n}\times 1}{\varphi _{n}}+\underset{m_{n}\times n%
}{Q}\underset{n\times 1}{\alpha }+\underset{m_{n}\times 1}{\varepsilon }%
\text{,}  \label{structural eqn stacked} \\
\underset{m_{n}\times d}{X} &=&\underset{m_{n}\times d}{\Phi _{n}}+\underset{%
m_{n}\times d}{\Upsilon _{n}}+\underset{m_{n}\times n}{Q}\underset{n\times d}%
{\Xi }+\underset{m_{n}\times d}{U}\text{,}  \label{1st stage eqn stacked}
\end{eqnarray}%
where $\alpha =\left( \alpha _{1},...,\alpha _{n}\right) ^{\prime }$, $\Xi
=\left( \xi _{1},...,\xi _{n}\right) ^{\prime }$, and

\noindent\ 
\begin{equation*}
\underset{m_{n}\times n}{Q}=\left( 
\begin{array}{cccc}
\iota _{T_{1}} & 0 & \cdots & 0 \\ 
0 & \iota _{T_{2}} & \ddots & \vdots \\ 
\vdots & \ddots & \ddots & 0 \\ 
0 & \cdots & 0 & \iota _{T_{n}}%
\end{array}%
\right) \text{.}
\end{equation*}%
and where the other vectors and matrices are stacked similar to $W_{n}$. For
notational convenience, we have suppressed the dependence of $\varphi _{n}$, 
$\Phi _{n}$, and $\Upsilon _{n}$ on $W_{n}$. Note that our setup allows the
clusters to be of possibly different sizes, so that our model can also be
interpreted as a possibly unbalanced panel data model.

Making use of these notations, we can write down the following assumptions
for our model.

\medskip

\noindent \textbf{Assumption 1: }Let $\mathcal{F}_{n}^{W}=\sigma \left(
W_{n}\right) $ (i.e., the $\sigma $-algebra generated by $W_{n})$. Suppose
that the following conditions are satisfied (i) Conditional on $\mathcal{F}%
_{n}^{W}$, $\left( \varepsilon _{\left( 1,1\right) },U_{\left( 1,1\right)
}^{\prime }\right) ,...,\left( \varepsilon _{\left( 1,T_{1}\right)
},U_{\left( 1,T_{1}\right) }^{\prime }\right) ,$

\noindent $.....,\left( \varepsilon _{\left( n,1\right) },U_{\left(
n,1\right) }^{\prime }\right) ,...,\left( \varepsilon _{\left(
n,T_{n}\right) },U_{\left( n,T_{n}\right) }^{\prime }\right) $ are mutually
independent. (ii) $E\left[ \varepsilon _{\left( i,t\right) }|\mathcal{F}%
_{n}^{W}\right] =0$ and $E\left[ U_{\left( i,t\right) }|\mathcal{F}_{n}^{W}%
\right] =0$ $a.s.,$ for $\left( i,t\right) =1,...,m_{n}$.

\medskip

\noindent \textbf{Assumption 2: }Suppose that there exists a constant $C\geq
1$ such that for all $n$

\noindent (i) $\max_{1\leq \left( i,t\right) \leq m_{n}}E\left[ \varepsilon
_{\left( i,t\right) }^{8}|\mathcal{F}_{n}^{W}\right] \leq C<\infty $ $a.s.$%
and $\max_{1\leq \left( i,t\right) \leq m_{n}}E\left[ \left\Vert U_{\left(
i,t\right) }\right\Vert _{2}^{8}|\mathcal{F}_{n}^{W}\right] \leq C<\infty $ $%
a.s.$ and (ii) $\inf_{1\leq \left( i,t\right) \leq m_{n}}\lambda _{\min
}\left( \Omega _{\left( i,t\right) }\right) \geq 1/C>0$ $\ a.s.$, where $%
\Omega _{\left( i,t\right) }=E\left[ \nu _{\left( i,t\right) }\nu _{\left(
i,t\right) }^{\prime }|\mathcal{F}_{n}^{W}\right] $ with $\nu _{\left(
i,t\right) }=\left( 
\begin{array}{cc}
\varepsilon _{\left( i,t\right) } & U_{\left( i,t\right) }^{\prime }%
\end{array}%
\right) ^{\prime }$.\bigskip

\noindent \textbf{Assumption 3: }Suppose that $\Upsilon _{n}\left(
W_{2,\left( i,t\right) }\right) =D_{\mu }\gamma \left( W_{2,\left(
i,t\right) }\right) /\sqrt{n},$ for $\left( i,t\right) =1,...,m_{n}$, where $%
D_{\mu }=diag\left( \mu _{1,n},..,\mu _{d,n}\right) $. The following
conditions are assumed on the diagonal elements $\mu _{1,n},..,\mu _{d,n},$
as $n\rightarrow \infty $. (i) Either $\mu _{k,n}=\sqrt{n}$ or $\mu _{k,n}/%
\sqrt{n}\rightarrow 0,$ for $k\in \left\{ 1,...,d\right\} $. (ii) Let $\mu
_{n}^{\min }=\min_{1\leq k\leq d}$ $\mu _{k,n}$ , and suppose that $\mu
_{n}^{\min }\rightarrow \infty ,$ as $n\rightarrow \infty $, such that$\sqrt{%
K_{2,n}}/\left( \mu _{n}^{\min }\right) ^{2}\rightarrow 0$. (iii) $\lambda
_{\min }\left( H_{n}\right) \geq 1/C>0$ and $\lambda _{\max }\left( \Gamma
^{\prime }\Gamma /n\right) \leq C<\infty $ $a.s.,$ for all $n$ sufficiently
large, where $H_{n}=\Gamma ^{\prime }M^{\left( Z_{1},Q\right) }\Gamma /n$ and

\noindent $\underset{m_{n}\times d}{\Gamma }=\left( 
\begin{array}{ccccccc}
\gamma \left( W_{2,\left( 1,1\right) }\right) & \cdots & \gamma \left(
W_{2,\left( 1,T_{1}\right) }\right) & \cdots & \gamma \left( W_{2,\left(
n,1\right) }\right) & \cdots & \gamma \left( W_{2,\left( n,T_{n}\right)
}\right)%
\end{array}%
\right) ^{\prime }$.\bigskip

\noindent \textbf{Assumption 4: }Suppose that $\Phi _{n}\left( W_{1,\left(
i,t\right) }\right) =D_{\kappa }f\left( W_{1,\left( i,t\right) }\right) /%
\sqrt{n}$ and $\varphi _{n}\left( W_{1,\left( i,t\right) }\right) $

\noindent $=\tau _{n}g\left( W_{1,\left( i,t\right) }\right) /\sqrt{n},$ for 
$\left( i,t\right) =1,...,m_{n}$, where $\underset{d\times d}{D_{\kappa }}%
=diag\left( \kappa _{1,n},..,\kappa _{d,n}\right) $ and $\tau _{n}$ is a
sequence of positive real numbers. The following conditions are assumed on $%
\kappa _{1,n},..,\kappa _{d,n}$ and on $\tau _{n}$ as $n\rightarrow \infty $%
: (i) either $\kappa _{\ell ,n}=\sqrt{n}$ or $\kappa _{\ell ,n}/\sqrt{n}%
\rightarrow 0,$ for $\ell \in \left\{ 1,...,d\right\} $; (ii) either $\tau
_{n}=\sqrt{n}$ or $\tau _{n}/\sqrt{n}\rightarrow 0.$\medskip 

\medskip

We shall consider approximating the function $\gamma \left( W_{2,\left(
i,t\right) }\right) $ in Assumption 3 by the series expansion $%
\dsum\nolimits_{k=1}^{K_{2,n}}\pi _{k}z_{2,k}\left( W_{2,\left( i,t\right)
}\right) ,$ for some family of approximating functions $z_{2,1}\left( \cdot
\right) ,z_{2,2}\left( \cdot \right) ,.....$\footnote{%
The approximating functions $z_{2,k}\left( \cdot \right) $, for $%
k=1,..,K_{2,n}$, may be polynomials, splines, or some other family of
functions as discussed in Newey (1997).}. Stacking the observed values of
these functions into a matrix, we obtain the $m_{n}\times K_{2,n}$ matrix, $%
Z_{2}=\left[ Z_{2}\left( W_{2,\left( 1,1\right) }\right) ,..,Z_{2}\left(
W_{2,\left( 1,T_{1}\right) }\right) ,...,Z_{2}\left( W_{2,\left( n,1\right)
}\right) ,..,Z_{2}\left( W_{2,\left( n,T_{n}\right) }\right) \right] $,
where $Z_{2}\left( W_{2,\left( i,t\right) }\right) =\left( z_{2,1}\left(
W_{2,\left( i,t\right) }\right) ,....,z_{2,K_{2,n}}\left( W_{2,\left(
i,t\right) }\right) \right) ^{\prime }$ is a $K_{2,n}\times 1$ vector, for
each $\left( i,t\right) \in \left\{ 1,...,m_{n}\right\} $. Similarly, we
shall approximate the functions $f\left( W_{1,\left( i,t\right) }\right) $
and $g\left( W_{1,\left( i,t\right) }\right) $ given in Assumption 4 by,
respectively, the series expansions $\dsum\nolimits_{k=1}^{K_{1,n}}\Theta
_{k}z_{1,k}\left( W_{2,\left( i,t\right) }\right) $ and

\noindent $\dsum\nolimits_{k=1}^{K_{1,n}}\theta _{k}z_{1,k}\left(
W_{2,\left( i,t\right) }\right) ,$ \ for a family of approximating functions 
$z_{1,1}\left( \cdot \right) ,z_{1,2}\left( \cdot \right) ,.....$ Stacking
the observed values of these functions, we get the $m_{n}\times K_{1,n}$
matrix,

\noindent $Z_{1}=\left[ Z_{1}\left( W_{1,\left( 1,1\right) }\right)
,..,Z_{1}\left( W_{1,\left( 1,T_{1}\right) }\right) ,...,Z_{1}\left(
W_{1,\left( n,1\right) }\right) ,..,Z_{1}\left( W_{1,\left( n,T_{n}\right)
}\right) \right] $, with $Z_{1}\left( W_{1,\left( i,t\right) }\right)
=\left( z_{1,1}\left( W_{1,\left( i,t\right) }\right)
,....,z_{1,K_{1,n}}\left( W_{1,\left( i,t\right) }\right) \right) ^{\prime }$
being a $K_{1,n}\times 1$ vector, for each $\left( i,t\right) \in \left\{
1,...,m_{n}\right\} $. For notational convenience, we shall suppress the
dependence of $Z_{1}$ and $Z_{2}$ on $W_{n}$. In addition, let $\underset{%
m_{n}\times K_{n}}{Z}=\left[ 
\begin{array}{cc}
Z_{1} & Z_{2}%
\end{array}%
\right] $, with $K_{n}=K_{1,n}+K_{2,n}$, and let $\Pi ^{K_{2,n}}=\left( \pi
_{1},...,\pi _{K_{2,n}}\right) ^{\prime }$, $\Theta ^{K_{1,n}}=\left( \Theta
_{1},...,\Theta _{K_{1,n}}\right) ^{\prime }$, and $\theta ^{K_{1,n}}=\left(
\theta _{1},...,\theta _{K_{1,n}}\right) ^{\prime }$.

Analogous to the linear IV model, we could interpret $Z_{2}$ as the matrix
of observations on the instruments of the model and $Z_{1}$ as the matrix of
observations on the additional covariates or included exogenous regressors.
Viewed in this light, we see that Assumption 3 is general enough to
accommodate a range of situations including both cases where there are
strong instruments and cases where the instruments are weaker. In
particular, when $\mu _{1,n}=\cdot \cdot \cdot =\mu _{d,n}=\mu _{n}^{\min }=%
\sqrt{n}$, our model specializes to the more classical situation where the
instruments are strong. On the other hand, the cases where some of the $\mu
_{j,n}$'s $\left( j=1,..,d\right) $ grow at a rate slower than $\sqrt{n}$
correspond to cases where at least some of the components of the parameter
vector of interest $\delta $ are weakly identified. By allowing for the
possibility that different $\mu _{j,n}$'s may grow at different rates, our
setup also allows for heterogeneity in how strongly the different components
of $\delta $ are identified. Note, however, that we do require that $\sqrt{%
K_{2,n}}/\left( \mu _{n}^{\min }\right) ^{2}\rightarrow 0$, since (as
discussed in Chao and Swanson (2005), Hausman et al. (2012), and Chao et al.
(2012)), if this condition is not fulfilled, then consistent estimation of
at least some of the components of $\delta $ may not be possible; and, in
this paper, we focus only on situations where we can consistently estimate $%
\delta $. To interpret this condition, it is easiest to consider the special
case where $\mu _{1,n}=\cdot \cdot \cdot =\mu _{d,n}=\mu _{n}^{\min }$. In
this case, $\left( \mu _{n}^{\min }\right) ^{2}$ can be interpreted as
giving the order of magnitude of the signal component of the IV model,
whereas $\sqrt{K_{2,n}}$ measures the order of magnitude of a leading noise
term, so that, in order for consistent estimation to be possible, the
signal-to-noise ratio $\left( \mu _{n}^{\min }\right) ^{2}/\sqrt{K_{2,n}}$
must diverge to infinity. A consequence of imposing this condition is that
our framework does not include the Staiger and Stock (1997) type weak
instrument situation, where $\mu _{1,n}=\cdot \cdot \cdot =\mu _{d,n}=\mu
_{n}^{\min }=O\left( 1\right) $, under which consistent point estimation
cannot be achieved.

Likewise, Assumption 4 allows for possible local-to-zero modeling of the
nonlinear components $g\left( W_{1,\left( i,t\right) }\right) $ and $f\left(
W_{1,\left( i,t\right) }\right) $. In the special case where $\kappa
_{1,n}=\cdot \cdot \cdot =\kappa _{d,n}=\tau _{n}=\sqrt{n}$ and $\mu
_{1,n}=\cdot \cdot \cdot =\mu _{d,n}=\mu _{n}^{\min }=\sqrt{n}$, our
structural equation of interest becomes a standard partially linear model,
whereas the system of first-stage equations becomes a standard multivariate
generalized additive model . However, by allowing for the possibility that
some of the $\kappa _{j,n}$'s and/or $\tau _{n}$ may grow at a rate slower
than $\sqrt{n}$, we also accommodate situations where the additional
covariates may only be weakly correlated with $y_{\left( i,t\right) }$
and/or with some elements of \ $X_{\left( i,t\right) }$.

\medskip

\noindent \textbf{Assumption 5: }(i) $m_{n}\rightarrow \infty $ as $%
n\rightarrow \infty ,$ such that $m_{n}\sim n$. (ii) $K_{1,n},K_{2,n}%
\rightarrow \infty ,$ as $n\rightarrow \infty ,$ such that $%
K_{1,n}^{2}/n=O\left( 1\right) $ and $K_{2,n}^{2}/n=o\left( 1\right) $.
(iii) Let $M^{Q}=I_{m_{n}}-Q\left( Q^{\prime }Q\right) ^{-1}Q^{\prime }$.
There exists a positive constant $\underline{C}$ such that $\lambda _{\min
}\left( Z^{\prime }M^{Q}Z\right) \geq \underline{C}$ $>0$ $a.s.,$ for all $n$
sufficiently large. (iv) Let $P^{\perp }=P^{\left( Z,Q\right) }-P^{\left(
Z_{1},Q\right) }=M^{\left( Z_{1},Q\right) }Z_{2}\left( Z_{2}^{\prime
}M^{\left( Z_{1},Q\right) }Z_{2}\right) ^{-1}Z_{2}^{\prime }M^{\left(
Z_{1},Q\right) }$ and let $P^{Z_{1}^{\perp }}=M^{Q}Z_{1}\left( Z_{1}^{\prime
}M^{Q}Z_{1}\right) ^{-1}Z_{1}^{\prime }M^{Q}$, where $M^{\left(
Z_{1},Q\right) }=M^{Q}-M^{Q}Z_{1}\left( Z_{1}^{\prime }M^{Q}Z_{1}\right)
^{-1}Z_{1}^{\prime }M^{Q}$ with $M^{Q}$ as defined in part (iii) above.
Suppose that $\max_{1\leq \left( i,t\right) \leq m_{n}}P_{\left( i,t\right)
,\left( i,t\right) }^{Z_{1}^{\perp }}=O_{a.s.}\left( K_{1,n}/n\right) $ and $%
\max_{1\leq \left( i,t\right) \leq m_{n}}P_{\left( i,t\right) ,\left(
i,t\right) }^{\perp }=O_{a.s.}\left( K_{2,n}/n\right) $.

\medskip

Note that part (ii) of Assumption 5 requires that both $K_{1,n}$ and $%
K_{2,n} $ grow at a rate slower than the sample size $n$. Hence, our setup
does not include the type of many-instrument setup of Morimune (1983) and
Bekker (1994), where the number of instruments grows at the same rate as $n$%
, nor the type of many-regressor setup of Cattaneo, Jansson, and Newey
(2018a,b), where the number of exogenous regressors grows to infinity on the
same order as $n$. However, note that our assumptions are in accord with the
interpretation of the structural equation as a partially linear model with
the nonlinear component being estimated nonparametrically by a series
estimator, as Newey (1997) has given results which show that consistent
series estimation requires $\kappa _{n}$, the number of approximating
functions, to grow slower than the sample size $n$. In fact, for consistent
series estimation using regression spline functions, Newey (1997) provides
an explicit rate restriction, where $\kappa _{n}^{2}/n\rightarrow 0$ as $%
n\rightarrow \infty $, and our assumption is in accord with this rate
condition. Moreover, note that our setup does allow the number of fixed or
cluster-specific effects to be on the order of $n$, so that the number of
overall covariates in the structural equation of interest will be on the
order of $n$. Hence, we believe that our framework is general enough to
accommodate a wide range of empirical problems of interest.

\medskip

\noindent \textbf{Assumption 6: }(i)\textbf{\ }$\min_{1\leq i\leq
n}T_{i}\geq 3$ for all $n$; (ii)\textbf{\ }There exists a positive integer $%
\overline{T}\geq 3,$ such that $\max_{1\leq i\leq n}T_{i}\leq \overline{T}%
<\infty ,$ for all $n$.

\medskip

\noindent \textbf{Assumption 7: }Let $\mathcal{W}_{1}\subseteq \mathbb{R}%
^{p_{1}}$ and $\mathcal{W}_{2}\subseteq \mathbb{R}^{p_{2}}$ denote the
support of $W_{1,\left( i,t\right) }$ and $W_{2,\left( i,t\right) }$,
respectively. The following rates of approximation are assumed. (i) There
exists a positive real number $\varrho _{g}$ and a vector of coefficients $%
\theta ^{K_{1,n}},$ such that 
\begin{equation*}
\left\Vert g\left( \cdot \right) -\theta ^{K_{1,n}\prime }Z_{1}\left( \cdot
\right) \right\Vert _{\infty }=O_{a.s.}\left( K_{1,n}^{-\varrho _{g}}\right) 
\text{, as }K_{1,n}\rightarrow \infty \text{,}
\end{equation*}%
where $\tau _{n}/K_{1,n}^{\varrho _{g}}=o\left( 1\right) $ and $\left\Vert
g\left( \cdot \right) -\theta ^{K_{1,n}\prime }Z_{1}\left( \cdot \right)
\right\Vert _{\infty }=\sup_{w_{1}\in \mathcal{W}_{1}}\left\vert g\left(
w_{1}\right) -\theta ^{K_{1,n}\prime }Z_{1}\left( w_{1}\right) \right\vert $.

\noindent (ii) There exists a positive real number $\varrho _{f}$ and a
matrix of coefficients $\Theta ^{K_{1,n}},$ such that 
\begin{equation*}
\left\Vert f\left( \cdot \right) -\Theta ^{K_{1,n}\prime }Z_{1}\left( \cdot
\right) \right\Vert _{\infty ,d}=O_{a.s.}\left( K_{1,n}^{-\varrho
_{f}}\right) \text{, as }K_{1,n}\rightarrow \infty \text{,}
\end{equation*}%
where $\kappa _{n}^{\max }/K_{1,n}^{\varrho _{f}}=O\left( 1\right) ,$ with $%
\kappa _{n}^{\max }=\max_{1\leq \ell \leq d}$ $\kappa _{\ell ,n}$ and $%
\left\Vert f\left( \cdot \right) -\Theta ^{K_{1,n}\prime }Z_{1}\left( \cdot
\right) \right\Vert _{\infty ,d}=\max_{\ell \in \left\{ 1,2,...,d\right\}
}\sup_{w_{1}\in \mathcal{W}_{1}}\left\vert f_{\ell }\left( w_{1}\right)
-e_{\ell ,d}^{\prime }\Theta ^{K_{1,n}\prime }Z_{1}\left( w_{1}\right)
\right\vert ,$ with $e_{\ell ,d}$ denoting a $d\times 1$ elementary vector
with $1$ in the $\ell ^{th}$ component and $0$ in all other components.

\noindent (iii) There exists a positive real number $\varrho _{\gamma }$ and
a matrix of coefficients $\Pi ^{K_{2,n}},$ such that 
\begin{equation*}
\left\Vert \gamma \left( \cdot \right) -\Pi ^{K_{2,n}\prime }Z_{2}\left(
\cdot \right) \right\Vert _{\infty ,d}=O_{a.s.}\left( K_{2,n}^{-\varrho
_{\gamma }}\right) \text{, as }K_{2,n}\rightarrow \infty \text{,}
\end{equation*}%
where $\mu _{n}^{\max }/K_{2,n}^{\varrho _{\gamma }}=O\left( 1\right) ,$
with $\mu _{n}^{\max }=\max_{1\leq \ell \leq d}$ $\mu _{\ell ,n}$ and $%
\left\Vert \gamma \left( \cdot \right) -\Pi ^{K_{2,n}\prime }Z_{2}\left(
\cdot \right) \right\Vert _{\infty ,d}=\max_{\ell \in \left\{
1,2,...,d\right\} }\sup_{w_{2}\in \mathcal{W}_{2}}\left\vert \gamma _{\ell
}\left( w_{2}\right) -e_{\ell ,d}^{\prime }\Pi ^{K_{2,n}\prime }Z_{2}\left(
w_{2}\right) \right\vert $. (iv) Assume that

\noindent $\max_{1\leq \left( i,t\right) \leq m_{n}}\left\Vert \Gamma
^{\prime }M^{\left( Z_{1},Q\right) }e_{\left( i,t\right) }\right\Vert _{2}/%
\sqrt{n}=o_{p}\left( 1\right) $.

\medskip

A few comments about Assumption 7 are in order. Parts (i)-(iii) of this
assumption place conditions on the rate at which the error in approximating
the functions $g\left( \cdot \right) $, $f\left( \cdot \right) $, $\gamma
\left( \cdot \right) $, and $\mu _{\Gamma }\left( \cdot \right) $ must
vanish uniformly. A similar assumption \ has been specified in Newey (1997)
in studying convergence rates for nonparametric series estimators (see
Assumption 3 of that paper). As noted in that paper, the size of the
exponents, such as $\varrho _{g}$, $\varrho _{f}$, $\varrho _{\gamma }$, and 
$\varrho _{\mu },$ depends both on the degree of smoothness of the function
to be approximated (i.e., the number of continuous derivatives that the
functions has) and on the dimension of the argument of the function (i.e.,
the dimension of $W_{1,\left( i,t\right) }$ or $W_{2,\left( i,t\right) }$ in
our case). For example, under Assumption 7(i), if the approximating
functions used are splines or polynomials, then $\left\Vert g\left( \cdot
\right) -\theta ^{K_{1,n}\prime }Z_{1}\left( \cdot \right) \right\Vert
_{\infty }=O_{a.s.}\left( K_{1,n}^{-\varrho _{g}}\right) ,$ with $\varrho
_{g}=s/p_{1}$, where $s$ is the number of continuous derivatives of the
function $g\left( \cdot \right) $ and $p_{1}$ is the dimension of $%
W_{1,\left( i,t\right) }$. Since our results require that these
approximation errors vanish sufficiently fast, this, in turn, places certain
requirements on the smoothness of the functions $g\left( \cdot \right) $, $%
f\left( \cdot \right) $, $\gamma \left( \cdot \right) $, and $\mu _{\Gamma
}\left( \cdot \right) $ and on the dimension of $W_{1,\left( i,t\right) }$
or $W_{2,\left( i,t\right) }$, with some trade-offs between the two.
Finally, note that part (iv) of Assumption 7 is similar to a condition given
in Assumption 3 of Cattaneo, Jansson, and Newey (2018b). As noted in that
paper, this condition comes close to being minimal for the central limit
theorem to hold.

Although our specification allows the structural equation (\ref{structural
eqn}) and the system of first-stage equations (\ref{1st stage eqn}) to have
nonlinear components, the results that we give in this paper will also, of
course, hold under a linear specification with (possibly) many weak
instruments and/or many weak covariates taking the form 
\begin{eqnarray}
y_{\left( i,t\right) } &=&X_{\left( i,t\right) }^{\prime }\delta _{0}+\frac{%
\tau _{n}\theta ^{K_{1,n}\prime }Z_{1,\left( i,t\right) }}{\sqrt{n}}+\alpha
_{i}+\varepsilon _{\left( i,t\right) },  \label{linear structural eqn} \\
X_{\left( i,t\right) } &=&\frac{D_{\kappa }\Theta ^{K_{1,n}\prime
}Z_{1,\left( i,t\right) }}{\sqrt{n}}+\frac{D_{\mu }\Pi ^{K_{2,n}\prime
}Z_{2,\left( i,t\right) }}{\sqrt{n}}+\xi _{i}+U_{\left( i,t\right) },
\label{linear 1st stage eqn}
\end{eqnarray}%
where $i=1,...,n$ and $t=1,...,T_{i}$. In this case, $Z_{1,\left( i,t\right)
}$ and $Z_{2,\left( i,t\right) }$ will be exogenous regressors/instruments
that need not depend on other variables such as $W_{1,\left( i,t\right) }$
and $W_{2,\left( i,t\right) }$. In addition, in a strictly linear setup,
parts (i)-(iii) of Assumption 7 that impose conditions on the approximation
errors by series estimation will no longer be needed.

\medskip

\noindent \textbf{Assumption 8: }Let $\rho _{n}=E\left[ U^{\prime
}M^{Q}\varepsilon \right] /E\left[ \varepsilon ^{\prime }M^{Q}\varepsilon %
\right] $. Suppose that the limit of $\rho _{n}$ exists, so that $\rho
_{n}\rightarrow \rho $ ,\ as $n\rightarrow \infty ,$ for some fixed $d\times
1$ vector $\rho \in \mathcal{S}_{\rho }$, where $\mathcal{S}_{\rho }$
denotes some compact subset of $\mathbb{R}^{d}$.\bigskip

To estimate the parameter (vector) of interest $\delta $ in equation (\ref%
{structural eqn}), we propose three new jackknife-type IV estimators. We
shall use the acronyms FEJIV, FELIM, and FEFUL to denote, respectively, the
Fixed Effect Jackknife IV, the Fixed Effect LIML, and the Fixed Effect
Fuller estimator.

\begin{enumerate}
\item \textbf{FEJIV: }%
\begin{equation*}
\widehat{\delta }_{J}=\left( X^{\prime }AX\right) ^{-1}X^{\prime }Ay\text{,}
\end{equation*}
where $A=P^{\perp }-M^{\left( Z,Q\right) }D_{\widehat{\vartheta }}M^{\left(
Z,Q\right) }$, $P^{\perp }=P^{\left( Z,Q\right) }-P^{\left( Z_{1},Q\right) }$%
, and $M^{\left( Z,Q\right) }=I_{m_{n}}-P^{\left( Z,Q\right) }$, with $%
P^{\left( Z,Q\right) }$ and $P^{\left( Z_{1},Q\right) }$ being projection
matrices that project into the column space of $\left[ 
\begin{array}{cc}
Z & Q%
\end{array}%
\right] $ and $\left[ 
\begin{array}{cc}
Z_{1} & Q%
\end{array}%
\right] $, respectively. In addition, $D_{\widehat{\vartheta }}$ denotes an $%
m_{n}\times m_{n}$ diagonal matrix, whose diagonal elements $\widehat{%
\vartheta }=\left( 
\begin{array}{cccc}
\widehat{\vartheta }_{1} & \widehat{\vartheta }_{2} & \cdots & \widehat{%
\vartheta }_{N}%
\end{array}%
\right) ^{\prime }$, when stacked into a vector, correspond to the solution
of the system of linear equations $d_{P^{\perp }}=\left( M^{\left(
Z,Q\right) }\circ M^{\left( Z,Q\right) }\right) \vartheta $, where $%
d_{P^{\perp }}$ is an $m_{n}\times 1$ vector containing the diagonal
elements of the projection matrix $P^{\perp }$.

\item \textbf{FELIM: }The FELIM estimator $\widehat{\delta }_{L}$ is the
estimator that minimizes the objective function 
\begin{equation}
\widehat{Q}_{FELIM}\left( \delta \right) =\frac{\left( y-X\delta \right)
^{\prime }A\left( y-X\delta \right) }{\left( y-X\delta \right) ^{\prime
}M^{\left( Z_{1},Q\right) }\left( y-X\delta \right) },  \label{FELIM obj fn}
\end{equation}%
where $A$ is as defined above in the definition of FEJIV and $M^{\left(
Z_{1},Q\right) }=I_{m_{n}}-P^{\left( Z_{1},Q\right) },$ with $P^{\left(
Z_{1},Q\right) }$ as defined above. $\widehat{\delta }_{L}$ has the explicit
representation%
\begin{equation}
\widehat{\delta }_{L}=\left( X^{\prime }\left[ A-\widehat{\ell }%
_{L}M^{\left( Z_{1},Q\right) }\right] X\right) ^{-1}\left( X^{\prime }\left[
A-\widehat{\ell }_{L}M^{\left( Z_{1},Q\right) }\right] y\right) ,
\label{FELIM}
\end{equation}%
where $\widehat{\ell }_{L}$ is the smallest root of the determinantal
equation $\det \left\{ \overline{X}^{\prime }A\overline{X}-\ell \overline{X}%
^{\prime }M^{\left( Z_{1},Q\right) }\overline{X}\right\} $

$=0$ with $\overline{X}=\left[ 
\begin{array}{cc}
y & X%
\end{array}%
\right] $.

\item \textbf{FEFUL: }The FEFUL estimator $\widehat{\delta }_{F}$ is defined
as follows\textbf{\ }%
\begin{equation*}
\widehat{\delta }_{F}=\left( X^{\prime }\left[ A-\widehat{\ell }%
_{F}M^{\left( Z_{1},Q\right) }\right] X\right) ^{-1}\left( X^{\prime }\left[
A-\widehat{\ell }_{F}M^{\left( Z_{1},Q\right) }\right] y\right) ,
\end{equation*}%
where $M^{\left( Z_{1},Q\right) }=I_{m_{n}}-P^{\left( Z_{1},Q\right) }$ and $%
\widehat{\ell }_{F}=\left[ \widehat{\ell }_{L}-\left( 1-\widehat{\ell }%
_{L}\right) C/m_{n}\right] /\left[ 1-\left( 1-\widehat{\ell }_{L}\right)
C/m_{n}\right] $ for some constant $C$. Here, $\widehat{\ell }_{L}$ is the
smallest root of equation

$\det \left\{ \overline{X}^{\prime }A\overline{X}-\ell \overline{X}^{\prime
}M^{\left( Z_{1},Q\right) }\overline{X}\right\} =0$. For the Monte Carlo
results reported in section 5, we shall take $C=1$.\medskip
\end{enumerate}

To help develop some intuition for these new estimators, it is easiest if we
focus the discussion on FEJIV. To proceed, note first that, under our setup,
it is not difficult to show that%
\begin{equation*}
\widehat{\delta }_{J}-\delta _{0}=\left( X^{\prime }AX\right) ^{-1}X^{\prime
}A\varepsilon +o_{p}\left( 1\right) =\left( X^{\prime }AX\right) ^{-1}\left(
\Upsilon _{n}^{\prime }A\varepsilon +U^{\prime }A\varepsilon \right)
+o_{p}\left( 1\right) ,
\end{equation*}%
so that, at least in large samples, the \textquotedblleft
numerator\textquotedblright\ of the right-hand side of this equation has a
familiar form (i.e., it is in terms of a linear form $\Upsilon _{n}^{\prime
}A\varepsilon $ plus a bilinear form $U^{\prime }A\varepsilon )$. Next, note
that an elementary result from linear algebra states that if $A=MDM$, where $%
A$ is a square matrix, $D$ is a diagonal matrix, and $M$ is a symmetric
matrix, then $a=\left( M\circ M\right) d$, where $a=\left(
a_{11},a_{22},...,a_{m_{n},m_{n}}\right) ^{\prime }$ and $d=\left(
d_{11},d_{22},...,d_{m_{n},m_{n}}\right) ^{\prime }$. Put in words, this
result states that the vector of diagonal elements of $A$ is a linear
transformation of the vector of diagonal elements of $D$, with the
transformation matrix given by $\left( M\circ M\right) $. Since in the
definition of $\widehat{\delta }_{J}$, we have specified $A=P^{\perp
}-M^{\left( Z,Q\right) }D_{\widehat{\vartheta }}M^{\left( Z,Q\right) }$, it
follows that by choosing the diagonal elements of $D_{\widehat{\vartheta }}$
to satisfy the system of linear equations $d_{P^{\perp }}=\left( M^{\left(
Z,Q\right) }\circ M^{\left( Z,Q\right) }\right) \vartheta $, where $%
d_{P^{\perp }}=\left( P_{11}^{\perp },P_{22}^{\perp
},...,P_{m_{n},m_{n}}^{\perp }\right) ^{\prime }$, we would, by
construction, end up with a matrix $A$ whose diagonal elements $%
A_{11},...,A_{m_{n},m_{n}}$ are all zero. This, in turn, leads to the
bilinear form $U^{\prime }A\varepsilon $ having the characteristics of a
degenerate U-statistic, with expectation that is properly centered at zero.
This proper centering, in turn, allows $\widehat{\delta }_{J}$ to be both
consistent and asymptotically normal under many weak instrument asymptotics.
In addition, write $\widehat{\delta }_{J}-\delta _{0}=\left( X^{\prime
}AX\right) ^{-1}X^{\prime }A\left( \varphi _{n}+Q\alpha +\varepsilon \right) 
$, and note that 
\begin{eqnarray*}
X^{\prime }A\left( \varphi _{n}+Q\alpha +\varepsilon \right) &=&\left( \Phi
_{n}+\Upsilon _{n}+Q\Xi +U\right) \left[ P^{\perp }-M^{\left( Z,Q\right) }D_{%
\widehat{\vartheta }}M^{\left( Z,Q\right) }\right] \left( \varphi
_{n}+\varepsilon \right) \\
&=&\left( \Phi _{n}+\Upsilon _{n}+U\right) \left[ P^{\perp }-M^{\left(
Z,Q\right) }D_{\widehat{\vartheta }}M^{\left( Z,Q\right) }\right] \left(
\varphi _{n}+\varepsilon \right) .
\end{eqnarray*}%
Looking at the equation above, we see that the matrix $A$ is designed to not
only partial out the fixed effects, but also to make all the
\textquotedblleft projection residues\textquotedblright\ $M^{\left(
Z,Q\right) }\varphi _{n}$, $M^{\left( Z,Q\right) }\Upsilon _{n}$, and $%
M^{\left( Z,Q\right) }\Phi _{n}$ sufficiently small so as not to cause a
bias problem even in the presence of many weak instruments. For this
purpose, it is important that our specification uses $M^{\left( Z,Q\right)
}. $ This matrix projects into the orthogonal complement of the full set of
exogenous variables/approximating functions, $\left( Z,Q\right) =\left(
Z_{1},Z_{2},Q\right) $, and not $M^{\left( Z_{1},Q\right) },$ whose use may
still leave the projection residue $M^{\left( Z_{1},Q\right) }\Upsilon _{n}$
relatively large. In addition, note that $M^{\left( Z,Q\right) }$ appears on
both sides of the jackknife correction matrix $M^{\left( Z,Q\right) }D_{%
\widehat{\vartheta }}M^{\left( Z,Q\right) }$ so that fixed effects and
nonlinear exogenous components are taken out on both sides of the
(multivariate) bilinear form, not just on one side. FELIM and FEFUL are a
bit more complicated than FEJIV, but they share the same basic design as
FEJIV; and, in consequence, they will also be consistent and asymptotically
normal under many weak instrument asymptotics, as we will show in the
theorems below.

On the other hand, jackknife IV estimators currently available in the
literature do not fully accomplish the dual goals of being both properly
centered and of having all cluster-specific effects and additional
covariates properly partialed out. To be more specific, we will briefly
discuss a number of jackknife IV estimators that have been proposed in the
literature. The paper by Angrist, Imbens, and Krueger (1999) consider the
JIVE1 and JIVE2 estimators of the parameter vector $\delta ,$ but in a
cross-sectional setup without either fixed effects or included exogenous
regressors. Hence, these authors do not explicitly study the more general
version of these estimators that partials out additional covariates. Hausman
et al. (2012) introduce jackknife versions of LIML\ and Fuller estimators
called HLIM\ and HFUL, but they do so in a cross-sectional context where
there are no fixed effects and where only a small number of included
exogenous regressors is allowed, so that the problem of having to partial
out fixed effects and a potentially large number of included exogenous
variables is not studied in that paper. In addition, the symmetric jackknife
IV (SJIVE) estimator proposed by Bekker and Crudu (2015) is formulated in a
setting without fixed effects and with no included exogenous regressors.
Hence, that paper also does not consider issues related to having to partial
out additional covariates.

A recent paper, Evdokimov and Koles\'{a}r (2018), does examine a number of
interesting jackknife IV estimators that allow for partialing out of
additional covariates. In the following discussion we discuss how these
estimators might perform if applied to our setting under many weak
instrument asymptotics. Consider first the IJIVE1 estimator studied in that
paper. This estimator was originally proposed by Ackerberg and Devereux
(2009) and is further analyzed in the grouped data setting by Evdokimov and
Koles\'{a}r (2018). Using our notation, the estimator can be written in the
form%
\begin{eqnarray*}
\widehat{\delta }_{IJIVE1} &=&\left( X^{\prime }M^{\left( Z_{1},Q\right) }%
\left[ P^{\perp }-D\left( P^{\perp }\right) \right] \left[ I_{m_{n}}-D\left(
P^{\perp }\right) \right] ^{-1}M^{\left( Z_{1},Q\right) }X\right) ^{-1} \\
&&\times \left( X^{\prime }M^{\left( Z_{1},Q\right) }\left[ P^{\perp
}-D\left( P^{\perp }\right) \right] \left[ I_{m_{n}}-D\left( P^{\perp
}\right) \right] ^{-1}M^{\left( Z_{1},Q\right) }y\right) .
\end{eqnarray*}%
It follows that we can further write the deviation of this estimator from
the true value $\delta _{0}$ as%
\begin{eqnarray}
\widehat{\delta }_{IJIVE1}-\delta _{0} &=&\left( X^{\prime }A_{IJ1}X\right)
^{-1}\left( X^{\prime }A_{IJ1}\varphi _{n}+X^{\prime }A_{IJ1}\varepsilon
\right)  \notag \\
&=&\left( X^{\prime }A_{IJ1}X\right) ^{-1}\left( X^{\prime }A_{IJ1}\varphi
_{n}+\Phi _{n}^{\prime }A_{IJ1}\varepsilon +\Upsilon _{n}^{\prime
}A_{IJ1}\varepsilon +U^{\prime }A_{IJ1}\varepsilon \right) \text{,}
\label{IJIVE1}
\end{eqnarray}%
where $A_{IJ1}=M^{\left( Z_{1},Q\right) }\left[ P^{\perp }-D\left( P^{\perp
}\right) \right] \left[ I_{m_{n}}-D\left( P^{\perp }\right) \right]
^{-1}M^{\left( Z_{1},Q\right) }$. By straightforward

\noindent calculation, it is easy to see that the $\left( i,t\right) ^{th}$
diagonal element of the matrix $A_{IJ1}$ is given by%
\begin{equation*}
A_{IJ1,\left( i,t\right) ,\left( i,t\right) }=\dsum\limits_{\left(
j,s\right) =1}^{m_{n}}\frac{M_{\left( j,s\right) ,\left( i,t\right)
}^{\left( Z_{1},Q\right) }}{1-P_{\left( j,s\right) ,\left( j,s\right)
}^{\perp }}\left[ P_{\left( i,t\right) ,\left( j,s\right) }^{\perp
}-M_{\left( i,t\right) ,\left( j,s\right) }^{\left( Z_{1},Q\right)
}P_{\left( j,s\right) ,\left( j,s\right) }^{\perp }\right] \neq 0,
\end{equation*}%
for $\left( i,t\right) =1,...,m_{n}$, so that $U^{\prime }A_{IJ1}\varepsilon 
$, the bilinear form on the right-hand side of equation (\ref{IJIVE1})
above, will not be a degenerate U-statistic and will not be properly
centered at zero. Another way of looking at this issue is that although the
matrix $\left[ P^{\perp }-D\left( P^{\perp }\right) \right] \left[
I_{m_{n}}-D\left( P^{\perp }\right) \right] ^{-1}$ does have a
\textquotedblleft jackknife form\textquotedblright\ in the sense that the
elements of its main diagonal are all zero, it defines a bilinear form not
with respect to $u$ and $\varepsilon $ but with respect to the projected
vectors $\widehat{u}=M^{\left( Z_{1},Q\right) }u$ and $\widehat{\varepsilon }%
=M^{\left( Z_{1},Q\right) }\varepsilon $. Note, however, that in general the 
$i^{th}$ element of $\widehat{u}$ will contain not just the $i^{th}$ element
of $u$ but other elements as well, and similarly for $\widehat{\varepsilon }$%
. In consequence, merely having the diagonal elements zeroed out in this
case is not sufficient for the bilinear form $u^{\prime }A_{IJ1}\varepsilon =%
\widehat{u}^{\prime }\left[ P^{\perp }-D\left( P^{\perp }\right) \right] %
\left[ I_{m_{n}}-D\left( P^{\perp }\right) \right] ^{-1}\widehat{\varepsilon 
}$ to be properly centered at zero. In some sense, the process of partialing
out the covariates has interfered with the process of jackknife recentering
in the way this estimator is constructed. We can use a similar argument to
also show that the bilinear form for IJIVE2 is not properly centered at zero.

Now consider the UJIVE estimator, which was first introduced in Koles\'{a}r
(2013) and is further analyzed in the grouped data setting by Evdokimov and
Koles\'{a}r (2018). This estimator takes the form%
\begin{eqnarray*}
\widehat{\delta }_{UJIVE} &=&\left( X^{\prime }\left[ \widetilde{P}^{\left(
Z,Q\right) }D\left( M^{\left( Z,Q\right) }\right) ^{-1}-\widetilde{P}%
^{\left( Z_{1},Q\right) }D\left( M^{\left( Z_{1},Q\right) }\right) ^{-1}%
\right] X\right) ^{-1} \\
&&\times \left( X^{\prime }\left[ \widetilde{P}^{\left( Z,Q\right) }D\left(
M^{\left( Z,Q\right) }\right) ^{-1}-\widetilde{P}^{\left( Z_{1},Q\right)
}D\left( M^{\left( Z_{1},Q\right) }\right) ^{-1}\right] y\right) ,
\end{eqnarray*}%
where $Z=\left[ 
\begin{array}{cc}
Z_{1} & Z_{2}%
\end{array}%
\right] $, $\widetilde{P}^{\left( Z,Q\right) }=P^{\left( Z,Q\right)
}-D\left( P^{\left( Z,Q\right) }\right) $, and $\widetilde{P}^{\left(
Z_{1},Q\right) }=P^{\left( Z_{1},Q\right) }-D\left( P^{\left( Z_{1},Q\right)
}\right) $. To discuss this estimator, it is most convenient to consider the 
$d=1$ case (i.e., the case where there is only one endogenous regressor). In
this case, the diagonal matrix $D_{\mu }$ defined in Assumption 3 reduces to
the scalar $\mu _{n}=\mu _{n}^{\min }$. Now, we can write the deviation of
this estimator from the true value $\delta _{0}$ as%
\begin{equation*}
\widehat{\delta }_{UJIVE}-\delta _{0}=\left( \frac{X^{\prime }A_{UJ}X}{\mu
_{n}^{2}}\right) ^{-1}\left( \frac{X^{\prime }A_{UJ}\varphi _{n}+X^{\prime
}A_{UJ}Q\alpha +\Phi _{n}^{\prime }A_{UJ}\varepsilon +\Upsilon _{n}^{\prime
}A_{UJ}\varepsilon +U^{\prime }A_{UJ}\varepsilon }{\mu _{n}^{2}}\right) ,
\end{equation*}%
where $A_{UJ}=\left[ P^{\left( Z,Q\right) }-D\left( P^{\left( Z,Q\right)
}\right) \right] D\left( M^{\left( Z,Q\right) }\right) ^{-1}-\left[
P^{\left( Z_{1},Q\right) }-D\left( P^{\left( Z_{1},Q\right) }\right) \right]
D\left( M^{\left( Z_{1},Q\right) }\right) ^{-1}$. Note first that the
diagonal elements of the matrix $A_{UJ}$ are all equal to zero, so the
bilinear term for this estimator, $U^{\prime }A_{UJ}\varepsilon $, is
properly centered. However, this estimator has a bias problem that arises
from the presence of the term $X^{\prime }A_{UJ}\varphi _{n}/\mu _{n}^{2}$,
which can be nonnegligible and even large in order of magnitude. To see
this, observe first that simple manipulation shows that $A_{UJ}=M^{\left(
Z_{1},Q\right) }D\left( M^{\left( Z_{1},Q\right) }\right) ^{-1}-M^{\left(
Z,Q\right) }D\left( M^{\left( Z,Q\right) }\right) ^{-1}$. Using this
identity, we can write%
\begin{eqnarray}
\frac{X^{\prime }A_{UJ}\varphi _{n}}{\mu _{n}^{2}} &=&\frac{\Upsilon
_{n}^{\prime }M^{\left( Z_{1},Q\right) }D\left( M^{\left( Z_{1},Q\right)
}\right) ^{-1}\varphi _{n}}{\mu _{n}^{2}}-\frac{\Upsilon _{n}^{\prime
}M^{\left( Z,Q\right) }D\left( M^{\left( Z,Q\right) }\right) ^{-1}\varphi
_{n}}{\mu _{n}^{2}}  \notag \\
&&+\frac{\Phi _{n}^{\prime }M^{\left( Z_{1},Q\right) }D\left( M^{\left(
Z_{1},Q\right) }\right) ^{-1}\varphi _{n}}{\mu _{n}^{2}}-\frac{\Phi
_{n}^{\prime }M^{\left( Z,Q\right) }D\left( M^{\left( Z,Q\right) }\right)
^{-1}\varphi _{n}}{\mu _{n}^{2}}  \notag \\
&&+\frac{U^{\prime }M^{\left( Z_{1},Q\right) }D\left( M^{\left(
Z_{1},Q\right) }\right) ^{-1}\varphi _{n}}{\mu _{n}^{2}}-\frac{U^{\prime
}M^{\left( Z,Q\right) }D\left( M^{\left( Z,Q\right) }\right) ^{-1}\varphi
_{n}}{\mu _{n}^{2}}.  \label{XAphi}
\end{eqnarray}%
Note that the term on the right-hand side of (\ref{XAphi}) which can be
particularly large in order of magnitude is $\Upsilon _{n}^{\prime
}M^{\left( Z_{1},Q\right) }D\left( M^{\left( Z_{1},Q\right) }\right)
^{-1}\varphi _{n}/\mu _{n}^{2}$. In fact, one can show that%
\begin{equation*}
\frac{\Upsilon _{n}^{\prime }M^{\left( Z_{1},Q\right) }D\left( M^{\left(
Z_{1},Q\right) }\right) ^{-1}\varphi _{n}}{\mu _{n}^{2}}=\frac{\mu
_{n}\Gamma _{n}^{\prime }M^{\left( Z_{1},Q\right) }D\left( M^{\left(
Z_{1},Q\right) }\right) ^{-1}}{\sqrt{n}\mu _{n}^{2}}\frac{\tau _{n}g_{n}}{%
\sqrt{n}}
\end{equation*}%
\begin{equation*}
=\frac{\tau _{n}}{\mu _{n}}\frac{\Gamma _{n}^{\prime }M^{\left(
Z_{1},Q\right) }D\left( M^{\left( Z_{1},Q\right) }\right) ^{-1}g_{n}}{n}%
=O_{a.s.}\left( \frac{\tau _{n}}{\mu _{n}}\right) \text{.}
\end{equation*}%
Hence, this estimator will be inconsistent as long as $\mu _{n}=O\left( \tau
_{n}\right) $. This will certainly be true in weak instrument cases where $%
\mu _{n}=o\left( \tau _{n}\right) $, but can also occur even in strong
instrument cases where $\mu _{n}\sim \sqrt{n}$ if the included exogenous
regressors enter significantly into the structural equation of interest, in
which case $\tau _{n}\sim \sqrt{n}$. Our simulation results, given in
Section 5, confirm that UJIVE tends to do much less well in terms of bias
when there are included exogenous regressors that enter significantly into
the structural equation of interest.

It should be noted, however, that, in the context of a linear IV model such
as that given by (\ref{linear structural eqn}) and (\ref{linear 1st stage
eqn}), UJIVE can be shown to be consistent under many weak instrument
asymptotics in the special case where the equation of interest contains no
included exogenous regressors and only fixed effects. This is not only
because in this case there is no term of the form $X^{\prime }A_{UJ}\varphi
_{n}/\mu _{n}^{2}=\tau _{n}X^{\prime }A_{UJ}Z_{1}\theta ^{K_{1,n}}/\left(
\mu _{n}^{2}\sqrt{n}\right) $, but also because, in a linear model with no
included exogenous regressors, $\Upsilon _{n}^{\prime }A_{UJ}Q\alpha /\mu
_{n}^{2}$

\noindent $=\Pi ^{K_{2,n}\prime }Z_{2}^{\prime }\left[ M^{Q}D\left(
M^{Q}\right) ^{-1}-M^{\left( Z_{2},Q\right) }D\left( M^{\left(
Z_{2},Q\right) }\right) ^{-1}\right] Q\alpha /\left( \mu _{n}\sqrt{n}\right)
=0$ so that, without the contaminating effects of the included exogenous
regressors, UJIVE does properly partial out the fixed effects.

Since our setup essentially has a panel data structure, one may also wonder
if it is possible to simply first difference away the fixed effects and then
do a jackknife-type recentering. A problem with this strategy occurs if the
IV regression contains, in addition to fixed effects, other included
exogenous regressors which cannot be eliminated by first-differencing. In
that case, one will have to do a projection to partial out these included
exogenous regressors, leading to the same problem as we have discussed
previously with regard to IJIVE1. In fact, the problem will be worse in this
case due to the serial correlation in the errors induced by the
first-differencing. Moreover, even if there are no additional included
exogenous regressors, the serial correlation induced by first differencing
causes additional complications. In particular, let $P^{Z}=Z\left( Z^{\prime
}Z\right) ^{-1}Z^{\prime }$ denote the projection matrix of the instruments%
\footnote{%
Here, we let $Z$ denote the matrix of observations on the instruments
because we are referring to a case where there are no included exogenous
variables, $Z_{1}$.}. Then, to achieve proper jackknife recentering in this
case requires the removal not only of the elements on the main diagonal of $%
P^{Z}$ but also the elements on the superdiagonal and the subdiagonal of $%
P^{Z}$, so that with serial correlation proper recentering is attained only
at the cost of greater information loss. Finally, the presence of serial
correlation also makes the large sample covariance matrix of a jackknife IV
estimator under many weak instrument asymptotics both more complicated and
more difficult to estimate. Hence, we believe that our approach for removing
fixed or cluster-specific effects has certain advantages over any
alternative procedure that is based on first-differencing. It should be
noted that a recent panel data paper by Hsiao and Zhou (2018) does take the
approach of constructing a jackknife IV estimator after first-differencing
the data. However, the objective and focus of that paper differs greatly
from ours. First of all, the panel data simultaneous equations model
specified in Hsiao and Zhou (2018) does not allow for the degree of
instrument weakness that we consider. In addition, the model that they
consider does not have error heteroskedasticity or included exogenous
regressors. If we apply their estimator to our setting, the estimator will
not be consistent in the case where $K_{2,n}\sim \left( \mu _{n}^{\min
}\right) ^{2}$ or in the case where $K_{2,n}/\left( \mu _{n}^{\min }\right)
^{2}\rightarrow \infty ,$ but $\sqrt{K_{2,n}}/\left( \mu _{n}^{\min }\right)
^{2}\rightarrow 0$. Still, it should be stressed that in their setting with
strong instruments and error homoskedasticity their estimator has good
asymptotic properties.

Turning our attention back to the equation $d_{P^{\perp }}=\left( M^{\left(
Z,Q\right) }\circ M^{\left( Z,Q\right) }\right) \vartheta $, note that in
order for this system of linear equations to have a unique solution, we need
the matrix $\left( M^{\left( Z,Q\right) }\circ M^{\left( Z,Q\right) }\right) 
$ to be invertible. The following lemma provides sufficient conditions for
the invertibility of $\left( M^{\left( Z,Q\right) }\circ M^{\left(
Z,Q\right) }\right) $.

\medskip

\noindent \textbf{Lemma 1: }Suppose that Assumptions 5 and 6(i) are
satisfied. Then, there exists a positive constant $C$ such that $\lambda
_{\min }\left( M^{\left( Z,Q\right) }\circ M^{\left( Z,Q\right) }\right)
\geq C>0$ $a.s.$, for all $n$ sufficiently large\footnote{%
The proof of Lemma 1 follows in a manner similar to the argument given in
Section 3 of the Supplemental Appendix to Cattaneo, Jansson, and Newey
(2018b). Hence, to save space, we have not included the proof here. An
explicit proof of this lemma, however, is available from us upon request.}.

\smallskip

It should be noted that a more general result on conditions for the
invertibility of Hadamard products has been given previously in Cattaneo,
Jansson, and Newey (2018b)\footnote{%
See, in particular, the analysis given in Section 3 of their Supplemental
Appendix.}. However, we choose to present a specialization of their result
because it shows that, in the context of our cluster-sampling setup, a key
condition for ensuring the invertibility of $\left( M^{\left( Z,W,Q\right)
}\circ M^{\left( Z,W,Q\right) }\right) $ is $\min_{1\leq i\leq n}T_{i}\geq 3$%
, which we explicitly assume in Assumption 6 part (i) above.

A further observation is that, in analyzing estimators that are obtained
from minimizing a variance ratio (e.g., FELIM), it is often convenient to
first consider the objective function in the form $Q\left( \beta \right)
=\left( \beta ^{\prime }\overline{X}^{\prime }A\overline{X}\beta \right)
/\left( \beta ^{\prime }\overline{X}^{\prime }M^{\left( Z_{1},Q\right) }%
\overline{X}\beta \right) $, where $\overline{X}=\left[ y,X\right] $ and
where $\beta $ is a $\left( d+1\right) \times 1$ vector, not initially
normalized to identify the dependent variable from the regressors. Here,
ones performs the minimization problem on $Q\left( \beta \right) $ in order
to obtain a minimizer $\widetilde{\beta }=\left( 
\begin{array}{cc}
\widetilde{\beta }_{1} & \widetilde{\beta }_{2}^{\prime }%
\end{array}%
\right) ^{\prime }$, with $\widetilde{\beta }_{1}$ a scalar and $\widetilde{%
\beta }_{2}$ a $d\times 1$ vector, and subsequently normalize the last $d$
components of $\widetilde{\beta }$ to obtain an estimator $\widetilde{\delta 
}=-\widetilde{\beta }_{2}/\widetilde{\beta }_{1},$ for the coefficients of
the endogenous regressors $X$. The following assumption ensures that this
subsequent normalization is well-defined. Moreover, in the proof of Lemma
S2-11 given in the Supplemental Appendix to this paper, we show that, by
following this procedure, we end up with exactly the FELIM estimator $%
\widehat{\delta }_{L}$, that satisfies the first-order conditions of the
objective function given by (\ref{FELIM obj fn}) and that also has explicit
representation given by equation (\ref{FELIM}) above.

\medskip

\noindent \textbf{Assumption 9: }Consider the variance-ratio objective
function

\noindent $Q\left( \beta \right) =\left( \beta ^{\prime }\overline{X}%
^{\prime }A\overline{X}\beta \right) /\left( \beta ^{\prime }\overline{X}%
^{\prime }M^{\left( Z_{1},Q\right) }\overline{X}\beta \right) $, where $%
\beta \in \overline{B}=\left\{ \beta \in \mathbb{R}^{d+1}:\left\Vert \beta
\right\Vert _{2}=1\right\} $. Let $\widetilde{\beta }$ be a $\left(
d+1\right) \times 1$ vector that minimizes the objective function $Q\left(
\beta \right) ,$ among all $\beta \in \overline{B}$ (i.e., $\widetilde{\beta 
}=\arg \min_{\beta \in \overline{B}}Q\left( \beta \right) )$. Partition $%
\widetilde{\beta }=\left( \widetilde{\beta }_{1},\widetilde{\beta }%
_{2}^{\prime }\right) ^{\prime }$ as defined above and assume that there
exists a positive constant $\underline{C}$ such that 
\begin{equation}
\left\vert \widetilde{\beta }_{1}\right\vert \geq \underline{C}>0\text{ \ }%
a.s.\text{ for all }n\text{ sufficiently large.}  \label{Cond 9}
\end{equation}

\medskip

\noindent Note that constraining $\beta $ (so that $\left\Vert \beta
\right\Vert _{2}=1)$ is not restrictive since we are dealing with an
objective function $Q\left( \beta \right) $ that is a ratio of quadratic
forms in $\beta $. More precisely, let $\overline{\beta }=\arg \min_{\beta
\in \mathbb{R}^{d+1}}Q\left( \beta \right) $, where $\overline{\beta }\neq 0$%
, and let $\widetilde{\beta }=\overline{\beta }/\left\Vert \overline{\beta }%
\right\Vert _{2}$ so that $\left\Vert \widetilde{\beta }\right\Vert _{2}=1$.
Then, $Q\left( \overline{\beta }\right) =\left( \overline{\beta }^{\prime }%
\overline{X}^{\prime }A\overline{X}\overline{\beta }\right) /\left( 
\overline{\beta }^{\prime }\overline{X}^{\prime }M^{\left( Z_{1},Q\right) }%
\overline{X}\overline{\beta }\right) =\left( \left\Vert \overline{\beta }%
\right\Vert _{2}^{-1}\overline{\beta }^{\prime }\overline{X}^{\prime }A%
\overline{X}\overline{\beta }\left\Vert \overline{\beta }\right\Vert
_{2}^{-1}\right) /\left( \left\Vert \overline{\beta }\right\Vert _{2}^{-1}%
\overline{\beta }^{\prime }\overline{X}^{\prime }M^{\left( Z_{1},Q\right) }%
\overline{X}\overline{\beta }\left\Vert \overline{\beta }\right\Vert
_{2}^{-1}\right) $

\noindent $=Q\left( \widetilde{\beta }\right) $, so any minimal value of $%
Q\left( \beta \right) $ obtained by minimizing $\beta $ over all $\beta \in 
\mathbb{R}^{d+1}$ can also be achieved by some $\widetilde{\beta }$ such
that $\left\Vert \widetilde{\beta }\right\Vert _{2}=1$.

\section{\noindent Consistency and Asymptotic Normality \newline
of Point Estimators}

\noindent \textbf{Theorem 1: }Suppose that Assumptions 1-7 are satisfied. Let

\noindent $\overline{\delta }_{n}=\left( X^{\prime }\left[ A-\overline{\ell }%
_{n}M^{\left( Z_{1},Q\right) }\right] X\right) ^{-1}\left( X^{\prime }\left[
A-\overline{\ell }_{n}M^{\left( Z_{1},Q\right) }\right] y\right) $, for some
sequence $\overline{\ell }_{n},$ such that $\overline{\ell }_{n}=o_{p}\left( %
\left[ \mu _{n}^{\min }\right] ^{2}/n\right) =o_{p}\left( 1\right) $. Then,
as $n\rightarrow \infty $, $\left\Vert D_{\mu }\left( \overline{\delta }%
_{n}-\delta _{0}\right) /\mu _{n}^{\min }\right\Vert _{2}\overset{p}{%
\rightarrow }0$ and $\left\Vert \overline{\delta }_{n}-\delta
_{0}\right\Vert _{2}\overset{p}{\rightarrow }0$.

\medskip

Special cases of the class of estimators that satisfy the conditions of
Theorem 1, and are thus consistent in the sense described in the theorem,
include FEJIV $\widehat{\delta }_{J,n}$, FELIM $\widehat{\delta }_{L,n}$,
and FEFUL $\widehat{\delta }_{F,n}$. Evidently, the main difference between
these estimators is the different specifications of $\overline{\ell }_{n}$. $%
\widehat{\delta }_{J,n}$ takes $\overline{\ell }_{n}=0,$ for all $n$; $%
\widehat{\delta }_{L,n}$ takes $\overline{\ell }_{n}=\widehat{\ell }_{L,n},$
where $\widehat{\ell }_{L,n}$ is the smallest root of the determinantal
equation $\det \left\{ \overline{X}^{\prime }A\overline{X}-\ell \overline{X}%
^{\prime }M^{\left( Z_{1},Q\right) }\overline{X}\right\} =0$; and $\widehat{%
\delta }_{F,n}$ takes $\overline{\ell }_{n}=\widehat{\ell }_{F,n}$ $=\left[ 
\widehat{\ell }_{L}-\left( 1-\widehat{\ell }_{L}\right) C/m_{n}\right] /%
\left[ 1-\left( 1-\widehat{\ell }_{L}\right) C/m_{n}\right] $, as described
earlier. Hence, by verifying that, in all three cases, $\overline{\ell }_{n}$
satisfies the condition $\overline{\ell }_{n}=o_{p}\left( \left[ \mu
_{n}^{\min }\right] ^{2}/n\right) =o_{p}\left( 1\right) $, we can easily
specialize the consistency result of Theorem 1 to establish the consistency
of FEJIV, FELIM, and FEFUL. These results are given in the following
corollary.

\medskip

\noindent \textbf{Corollary 1: }Under Assumptions 1-7 and 9, the following
results hold as $n\rightarrow \infty $. (a) $\left\Vert D_{\mu }\left( 
\widehat{\delta }_{J,n}-\delta _{0}\right) /\mu _{n}^{\min }\right\Vert _{2}%
\overset{p}{\rightarrow }0$ and $\left\Vert \widehat{\delta }_{J,n}-\delta
_{0}\right\Vert _{2}\overset{p}{\rightarrow }0$. (b) $\left\Vert D_{\mu
}\left( \widehat{\delta }_{L,n}-\delta _{0}\right) /\mu _{n}^{\min
}\right\Vert _{2}\overset{p}{\rightarrow }0$ and $\left\Vert \widehat{\delta 
}_{L,n}-\delta _{0}\right\Vert _{2}\overset{p}{\rightarrow }0$. (c) $%
\left\Vert D_{\mu }\left( \widehat{\delta }_{F,n}-\delta _{0}\right) /\mu
_{n}^{\min }\right\Vert _{2}\overset{p}{\rightarrow }0$ and $\left\Vert 
\widehat{\delta }_{F,n}-\delta _{0}\right\Vert _{2}\overset{p}{\rightarrow }%
0 $.

\medskip

The next two results establish asymptotic normality for the FELIM and FEFUL
estimators, under two different cases: (i) Case I: $K_{2,n}/\left( \mu
_{n}^{\min }\right) ^{2}=O\left( 1\right) $ and (ii) Case II: $%
K_{2,n}/\left( \mu _{n}^{\min }\right) ^{2}\rightarrow \infty ,$ but $\sqrt{%
K_{2,n}}/\left( \mu _{n}^{\min }\right) ^{2}\rightarrow 0$. The FEJIV
estimator can also be shown to have an asymptotic normal distribution under
both Cases I and II. However, we choose to focus our theoretical analysis on
FELIM\ and FEFUL\ because, as noted previously, the results of our Monte
Carlo study indicate that FELIM and FEFUL have better finite sample
properties than FEJIV.

To facilitate the statement of the next two results, define%
\begin{eqnarray}
\Lambda _{I,n} &=&H_{n}^{-1}\left( \Sigma _{1,n}+\Sigma _{2,n}\right)
H_{n}^{-1}=H_{n}^{-1}\Sigma _{n}H_{n}^{-1}\text{,}  \label{Asy Var I} \\
\Lambda _{II,n} &=&\frac{\left( \mu _{n}^{\min }\right) ^{2}}{K_{2,n}}%
H_{n}^{-1}\Sigma _{2,n}H_{n}^{-1}\text{,}  \label{Asy Var II}
\end{eqnarray}%
where $H_{n}=\Gamma ^{\prime }M^{\left( Z_{1},Q\right) }\Gamma /n$, $\Sigma
_{1,n}=VC\left( \Gamma ^{\prime }M^{\left( Z_{1},Q\right) }\varepsilon /%
\sqrt{n}|\mathcal{F}_{n}^{W}\right) =\Gamma ^{\prime }M^{\left(
Z_{1},Q\right) }D_{\sigma ^{2}}M^{\left( Z_{1},Q\right) }\Gamma /n$, and%
\begin{eqnarray*}
\Sigma _{2,n} &=&D_{\mu }^{-1}VC\left( \underline{U}^{\prime }A\varepsilon |%
\mathcal{F}_{n}^{W}\right) D_{\mu }^{-1} \\
&=&\dsum\limits_{\substack{ \left( i,t\right) ,\left( j,s\right) =1  \\ %
\left( i,t\right) \neq \left( j,s\right) }}^{m_{n}}A_{\left( i,t\right)
,\left( j,s\right) }^{2}E\left[ \varepsilon _{\left( i,t\right) }^{2}|%
\mathcal{F}_{n}^{W}\right] D_{\mu }^{-1}E\left[ \underline{U}_{\left(
j,s\right) }\underline{U}_{\left( j,s\right) }^{\prime }|\mathcal{F}_{n}^{W}%
\right] D_{\mu }^{-1} \\
&&+\dsum\limits_{\substack{ \left( i,t\right) ,\left( j,s\right) =1  \\ %
\left( i,t\right) \neq \left( j,s\right) }}^{m_{n}}A_{\left( i,t\right)
,\left( j,s\right) }^{2}D_{\mu }^{-1}E\left[ \underline{U}_{\left(
i,t\right) }\varepsilon _{\left( i,t\right) }|\mathcal{F}_{n}^{W}\right] E%
\left[ \varepsilon _{\left( j,s\right) }\underline{U}_{\left( j,s\right)
}^{\prime }|\mathcal{F}_{n}^{W}\right] D_{\mu }^{-1}\text{,}
\end{eqnarray*}%
with $\Sigma _{n}=\Sigma _{1,n}+\Sigma _{2,n}$ and $\underline{U}_{\left(
i,t\right) }=U_{\left( i,t\right) }-\rho \varepsilon _{\left( i,t\right) }$
for $\left( i,t\right) =1,...,m_{n}$. Here, for any random vector $x$, $%
VC\left( x|\mathcal{F}_{n}^{W}\right) $ denotes the conditional
variance-covariance matrix of $x$ given $\mathcal{F}_{n}^{W}$. In addition,
let $D_{\sigma ^{2}}=diag\left( \sigma _{\left( 1,1\right) }^{2},....,\sigma
_{\left( n,T_{n}\right) }^{2}\right) =diag\left( \sigma _{1}^{2},....,\sigma
_{m_{n}}^{2}\right) $, where $\sigma _{\left( i,t\right) }^{2}=\left[
\varepsilon _{\left( i,t\right) }^{2}|\mathcal{F}_{n}^{W}\right] ,$ for $%
\left( i,t\right) =1,...,m_{n}$ and where, for notational convenience, we
suppress the dependence of $\sigma _{\left( i,t\right) }^{2}$ on $\mathcal{F}%
_{n}^{W}$.

As evident from the results given below, $\Lambda _{I,n}$ and $\Lambda
_{II,n}$ are the (conditional) variance-covariance matrices of FELIM\ (and
also of FEFUL) in large samples under Cases I and II, respectively.

\medskip

\noindent \textbf{Theorem 2: }Suppose that Assumptions 1-9 are satisfied. In
addition, suppose that Case I holds so that $K_{2,n}/\left( \mu _{n}^{\min
}\right) ^{2}=O\left( 1\right) $. Then, $\Lambda _{I,n}$ is positive
definite $a.s.$ for all $n$ sufficiently large; and, as $n\rightarrow \infty 
$, $\Lambda _{I,n}^{-1/2}D_{\mu }\left( \widehat{\delta }_{L,n}-\delta
_{0}\right) \overset{d}{\rightarrow }N\left( 0,I_{d}\right) $ and $\Lambda
_{I,n}^{-1/2}D_{\mu }\left( \widehat{\delta }_{F,n}-\delta _{0}\right) 
\overset{d}{\rightarrow }N\left( 0,I_{d}\right) $.

\medskip

\noindent \textbf{Theorem 3: }Suppose that Assumptions 1-9 are satisfied,
and suppose that Case II holds, so that $\left( \mu _{n}^{\min }\right)
^{2}/K_{2,n}=o\left( 1\right) ,$ but $\sqrt{K_{2,n}}/\left( \mu _{n}^{\min
}\right) ^{2}\rightarrow 0$. In addition, let $\widetilde{L}_{n}$ be a $%
q\times d$ matrix with $1\leq q\leq d$, and suppose that there exists a
positive constant $C$ such that $\left\Vert \widetilde{L}_{n}\right\Vert
_{2}\leq C<\infty $ and $\lambda _{\min }\left( \widetilde{L}_{n}\Lambda
_{II,n}\widetilde{L}_{n}^{\prime }\right) \geq 1/C>0$ $a.s.n$. Then,

\noindent $\left( \mu _{n}^{\min }/\sqrt{K_{2,n}}\right) \left( \widetilde{L}%
_{n}\Lambda _{II,n}\widetilde{L}_{n}^{\prime }\right) ^{-1/2}\widetilde{L}%
_{n}D_{\mu }\left( \widehat{\delta }_{L,n}-\delta _{0}\right) \overset{d}{%
\rightarrow }N\left( 0,I_{q}\right) $ and

\noindent $\left( \mu _{n}^{\min }/\sqrt{K_{2,n}}\right) \left( \widetilde{L}%
_{n}\Lambda _{II,n}\widetilde{L}_{n}^{\prime }\right) ^{-1/2}\widetilde{L}%
_{n}D_{\mu }\left( \widehat{\delta }_{F,n}-\delta _{0}\right) \overset{d}{%
\rightarrow }N\left( 0,I_{q}\right) $.

\medskip

As alluded to earlier, the asymptotic results for FEJIV, FELIM, and FEFUL
given in the above theorems can be specialized to obtain results for the
linear IV regression case with (possibly) many weak instruments and/or many
weak covariates, as specified in equations (\ref{linear structural eqn}) and
(\ref{linear 1st stage eqn}) above.

\section{Covariance Matrix Estimation and Hypothesis\newline
Testing}

\noindent \qquad To consistently estimate the asymptotic variance-covariance
matrix of FELIM and FEFUL, we propose the following estimators%
\begin{equation}
\widehat{V}_{L}=\widehat{H}_{L}^{-1}\widehat{\Sigma }_{L}\widehat{H}_{L}^{-1}%
\text{ and }\widehat{V}_{F}=\widehat{H}_{F}^{-1}\widehat{\Sigma }_{F}%
\widehat{H}_{F}^{-1}\text{,}  \label{Var Est LIM and FUL}
\end{equation}%
where%
\begin{eqnarray*}
\widehat{H}_{L} &=&X^{\prime }\left[ A-\widehat{\ell }_{L,n}M^{\left(
Z_{1},Q\right) }\right] X\text{, }\widehat{H}_{F}=X^{\prime }\left[ A-%
\widehat{\ell }_{F,n}M^{\left( Z_{1},Q\right) }\right] X \\
\widehat{\Sigma }_{L} &=&X^{\prime }AD\left( J\left[ \widehat{\varepsilon }%
_{L}\circ \widehat{\varepsilon }_{L}\right] \right) AX-\widehat{\rho }%
_{L}\left( \widehat{\varepsilon }_{L}\circ \widehat{\varepsilon }_{L}\right)
^{\prime }J\left( A\circ A\right) J\left( \widehat{\varepsilon }_{L}\iota
_{d}^{\prime }\circ M^{\left( Z,Q\right) }X\right) \\
&&-\left( \widehat{\varepsilon }_{L}\iota _{d}^{\prime }\circ M^{\left(
Z,Q\right) }X\right) ^{\prime }J\left( A\circ A\right) J\left( \widehat{%
\varepsilon }_{L}\circ \widehat{\varepsilon }_{L}\right) \widehat{\rho }%
_{L}^{\prime }+\widehat{\rho }_{L}\widehat{\rho }_{L}^{\prime }\left( 
\widehat{\varepsilon }_{L}\circ \widehat{\varepsilon }_{L}\right) ^{\prime
}J\left( A\circ A\right) J\left( \widehat{\varepsilon }_{L}\circ \widehat{%
\varepsilon }_{L}\right) \\
&&+\left( \widehat{\varepsilon }_{L}\iota _{d}^{\prime }\circ \widehat{%
\underline{U}}_{L}\right) ^{\prime }J\left( A\circ A\right) J\left( \widehat{%
\varepsilon }_{L}\iota _{d}^{\prime }\circ \widehat{\underline{U}}%
_{L}\right) \text{,} \\
\widehat{\Sigma }_{F} &=&X^{\prime }AD\left( J\left[ \widehat{\varepsilon }%
_{F}\circ \widehat{\varepsilon }_{F}\right] \right) AX-\widehat{\rho }%
_{F}\left( \widehat{\varepsilon }_{F}\circ \widehat{\varepsilon }_{F}\right)
^{\prime }J\left( A\circ A\right) J\left( \widehat{\varepsilon }_{F}\iota
_{d}^{\prime }\circ M^{\left( Z,Q\right) }X\right) \\
&&-\left( \widehat{\varepsilon }_{F}\iota _{d}^{\prime }\circ M^{\left(
Z,Q\right) }X\right) ^{\prime }J\left( A\circ A\right) J\left( \widehat{%
\varepsilon }_{F}\circ \widehat{\varepsilon }_{F}\right) \widehat{\rho }%
_{F}^{\prime }+\widehat{\rho }_{F}\widehat{\rho }_{F}^{\prime }\left( 
\widehat{\varepsilon }_{F}\circ \widehat{\varepsilon }_{F}\right) ^{\prime
}J\left( A\circ A\right) J\left( \widehat{\varepsilon }_{F}\circ \widehat{%
\varepsilon }_{F}\right) \\
&&+\left( \widehat{\varepsilon }_{F}\iota _{d}^{\prime }\circ \widehat{%
\underline{U}}_{F}\right) ^{\prime }J\left( A\circ A\right) J\left( \widehat{%
\varepsilon }_{F}\iota _{d}^{\prime }\circ \widehat{\underline{U}}%
_{F}\right) \text{.}
\end{eqnarray*}%
and where $J=\left[ M^{Q}\circ M^{Q}\right] ^{-1}$, $\widehat{\varepsilon }%
_{L}=M^{\left( Z,Q\right) }\left( y-X\widehat{\delta }_{L}\right) $, $%
\widehat{\varepsilon }_{F}=M^{\left( Z,Q\right) }\left( y-X\widehat{\delta }%
_{F}\right) $,

\noindent $\widehat{\underline{U}}_{L}=M^{\left( Z,Q\right) }X-\widehat{%
\varepsilon }_{L}\widehat{\rho }_{L}^{\prime }$, and $\widehat{\underline{U}}%
_{F}=M^{\left( Z,Q\right) }X-\widehat{\varepsilon }_{F}\widehat{\rho }%
_{F}^{\prime }$. In addition, let

\noindent $\widehat{\rho }_{L}=\left[ X^{\prime }M^{\left( Z,Q\right)
}\left( y-X\widehat{\delta }_{L}\right) \right] /\left[ \left( y-X\widehat{%
\delta }_{L}\right) ^{\prime }M^{\left( Z,Q\right) }\left( y-X\widehat{%
\delta }_{L}\right) \right] $ and

\noindent $\widehat{\rho }_{F}=\left[ X^{\prime }M^{\left( Z,Q\right)
}\left( y-X\widehat{\delta }_{F}\right) \right] /\left( y-X\widehat{\delta }%
_{F}\right) ^{\prime }M^{\left( Z,Q\right) }\left( y-X\widehat{\delta }%
_{F}\right) $ denote estimators of the parameter $\rho =\lim_{n\rightarrow
\infty }E\left[ U^{\prime }M^{Q}\varepsilon \right] /E\left[ \varepsilon
^{\prime }M^{Q}\varepsilon \right] ,$ based on $\widehat{\delta }_{L}$ and $%
\widehat{\delta }_{F}$, respectively.

Our next result shows the consistency of the covariance matrix estimators
given in equation (\ref{Var Est LIM and FUL}) under both Cases I and II%
\footnote{%
It can be shown that an estimator of the asymptotic covariance matrix of
FEJIV, which will be consistent under both Case I and II, is given by%
\begin{equation*}
\widehat{V}_{J,n}=\widehat{H}^{-1}\widehat{\Sigma }_{J}\widehat{H}%
^{-1}=\left( X^{\prime }AX\right) ^{-1}\left[ X^{\prime }AD_{\widehat{%
\varsigma }_{J}}AX+\left( \widehat{\varepsilon }_{J}\circ \widehat{U}\right)
^{\prime }J\left( A\circ A\right) J\left( \widehat{\varepsilon }_{J}\circ 
\widehat{U}\right) \right] \left( X^{\prime }AX\right) ^{-1},
\end{equation*}%
where $D_{\widehat{\varsigma }_{J}}=diag\left( \widehat{\varsigma }%
_{J,\left( 1,1\right) },..,\widehat{\varsigma }_{J,\left( 1,T_{1}\right)
},..,\widehat{\varsigma }_{J,\left( n,1\right) },..,\widehat{\varsigma }%
_{J,\left( n,T_{n}\right) }\right) $, $\widehat{\varsigma }_{J,\left(
i,t\right) }=e_{\left( i,t\right) }^{\prime }J\left( \widehat{\varepsilon }%
_{J}\circ \widehat{\varepsilon }_{J}\right) $, $\widehat{\varepsilon }%
_{J}=M^{\left( Z,Q\right) }\left( y-X\widehat{\delta }_{J}\right) $, and $%
\widehat{U}=M^{\left( Z,Q\right) }X$. Note also that the standard error used
for FEJIV in our Monte Carlo study given in section 5 is based on the above
formula.}.

\bigskip

\noindent \textbf{Theorem 4: }Suppose that Assumptions 1-9 are satisfied.
Then, the following statements are true.

\begin{enumerate}
\item[(a)] For Case I, where $K_{2,n}/\left( \mu _{n}^{\min }\right)
^{2}=O\left( 1\right) $, $D_{\mu }\widehat{V}_{L}D_{\mu }=\Lambda
_{I,n}+o_{p}\left( 1\right) $ and $D_{\mu }\widehat{V}_{F}D_{\mu }=\Lambda
_{I,n}+o_{p}\left( 1\right) $, where $\Lambda _{I,n}$ is as defined in
equation (\ref{Asy Var I}).

\item[(b)] For Case II, where $K_{2,n}/\left( \mu _{n}^{\min }\right)
^{2}\rightarrow \infty $, but $\sqrt{K_{2,n}}/\left( \mu _{n}^{\min }\right)
^{2}\rightarrow 0$, $\left[ \left( \mu _{n}^{\min }\right) ^{2}/K_{2,n}%
\right] D_{\mu }\widehat{V}_{L}D_{\mu }=\Lambda _{II,n}+o_{p}\left( 1\right) 
$ and $\left[ \left( \mu _{n}^{\min }\right) ^{2}/K_{2,n}\right] D_{\mu }%
\widehat{V}_{F}D_{\mu }=\Lambda _{II,n}+o_{p}\left( 1\right) $, where $%
\Lambda _{II,n}$ is as defined in equation (\ref{Asy Var II}).
\end{enumerate}

\bigskip

Theorem 5 below provides asymptotic normality results for t-statistics
associated with the FELIM and FEFUL estimators in the case where $\mu
_{1,n}=\cdot \cdot \cdot =\mu _{d,n}=\mu _{n}^{\min }$. The case where the
degree of instrument weakness is homogeneous and does not vary across the
different first-stage equations is one which is often assumed in previous
papers on weak and/or many instruments. In this case, we show that, without
any additional side conditions that may restrict the form of the linear
hypothesis tested, the t-ratio based on our estimators has an asymptotic
standard normal distribution under the null hypothesis, as long as $\sqrt{%
K_{2,n}}/\left( \mu _{n}^{\min }\right) ^{2}=\sqrt{K_{2,n}}/\left( \mu
_{n}\right) ^{2}\rightarrow 0$. Moreover, the results show that, under these
same rate conditions, the tests are also consistent, as the test statistics
diverge under fixed alternatives.

\bigskip

\noindent \textbf{Theorem 5: }Suppose that Assumptions 1-9 are satisfied.
Suppose further that the diagonal matrix $D_{\mu }$ in Assumption 3 takes
the form $D_{\mu }=\mu _{n}^{\min }\cdot I_{d}$ (i.e., $\mu _{1,n}=\cdot
\cdot \cdot =\mu _{d,n}=\mu _{n}^{\min })$. Then, the following statements
are true for the t-statistics $\mathbb{T}_{L}=\left( c^{\prime }\widehat{%
\delta }_{L}-r\right) /\sqrt{c^{\prime }\widehat{V}_{L}c}$ and $\mathbb{T}%
_{F}=\left( c^{\prime }\widehat{\delta }_{F}-r\right) /\sqrt{c^{\prime }%
\widehat{V}_{F}c}$.

\begin{enumerate}
\item[a.] For Case I, where $K_{2,n}/\left( \mu _{n}^{\min }\right)
^{2}=O\left( 1\right) $:

\begin{enumerate}
\item[(i)] Under $H_{0}:c^{\prime }\delta _{0}=r$, $\mathbb{T}_{L}\overset{d}%
{\rightarrow }N\left( 0,1\right) $ and $\mathbb{T}_{F}\overset{d}{%
\rightarrow }N\left( 0,1\right) .$

\item[(ii) ] Under $H_{1}:c^{\prime }\delta _{0}\neq r$, with probability
approaching one, as $n\rightarrow \infty $, the following results hold: $%
\mathbb{T}_{L}\rightarrow +\infty $ \ if $c^{\prime }\delta _{0}>r$; $%
\mathbb{T}_{L}\rightarrow -\infty $ \ if $c^{\prime }\delta _{0}<r$; $%
\mathbb{T}_{F}\rightarrow +\infty $\ if $c^{\prime }\delta _{0}>r$; and $%
\mathbb{T}_{F}\rightarrow -\infty $ \ if $c^{\prime }\delta _{0}<r$.
\end{enumerate}

\item[b.] For Case II, where $K_{2,n}/\left( \mu _{n}^{\min }\right)
^{2}\rightarrow \infty $ but $\sqrt{K_{2,n}}/\left( \mu _{n}^{\min }\right)
^{2}\rightarrow 0$:

\begin{enumerate}
\item[(i)] Under $H_{0}:c^{\prime }\delta _{0}=r$, $\mathbb{T}_{L}\overset{d}%
{\rightarrow }N\left( 0,1\right) $ and $\mathbb{T}_{F}\overset{d}{%
\rightarrow }N\left( 0,1\right) .$

\item[(ii)] Under $H_{1}:c^{\prime }\delta _{0}\neq r$, with probability
approaching one, as $n\rightarrow \infty $, the following results hold: $%
\mathbb{T}_{L}\rightarrow +\infty $ \ if $c^{\prime }\delta _{0}>r$; $%
\mathbb{T}_{L}\rightarrow -\infty $ \ if $c^{\prime }\delta _{0}<r$; $%
\mathbb{T}_{F}\rightarrow +\infty $\ if $c^{\prime }\delta _{0}>r$; and $%
\mathbb{T}_{F}\rightarrow -\infty $ \ if $c^{\prime }\delta _{0}<r$.
\end{enumerate}
\end{enumerate}

Our next result considers cases where we test a null hypothesis involving
only one coefficient, such as testing the significance of a particular
parameter. We choose to analyze this case because this seems to be the most
frequent use of the t-statistic by empirical researchers. In these cases, we
establish that, under mild additional conditions, the t-test based on our
proposed estimators will be robust to many weak instruments, even if there
is heterogeneity in the degree of instrument weakness across the different
first-stage equations. Moreover, our test will be robust to many weak
instruments even if the empirical researcher using our test has no knowledge
of how the degree of instrument weakness varies across the different
first-stage equations. For this result, we introduce a modification of
Assumption 3.

\medskip

\noindent \textbf{Assumption 3*: }Suppose that $\Upsilon _{n}\left(
W_{2,\left( i,t\right) }\right) =D_{\mu }\gamma \left( W_{2,\left(
i,t\right) }\right) /\sqrt{n},$ for $\left( i,t\right) =1,...,m_{n}$, where $%
D_{\mu }$ has the form%
\begin{equation}
D_{\mu }=\left( 
\begin{array}{cc}
\underset{d_{1}\times d_{1}}{D_{1}} & 0 \\ 
0 & \left( \mu _{n}^{\min }\right) \cdot I_{d_{2}}%
\end{array}%
\right) \text{,}  \label{Dmu}
\end{equation}%
with $D_{1}=diag\left( \mu _{1,n},..,\mu _{d_{1},n}\right) ,$ and where $%
d_{1}$ and $d_{2}$ are positive integers, with $d_{1}+d_{2}=d$. The
following conditions are assumed. (i) Either $\mu _{g,n}=\sqrt{n}$ or $\mu
_{g,n}/\sqrt{n}\rightarrow 0,$ for $g\in \left\{ 1,...,d\right\} $. (ii) Let 
$\mu _{n}^{\min }=\min_{1\leq g\leq d}$ $\mu _{g,n},$and suppose that $\mu
_{n}^{\min }\rightarrow \infty ,$ as $n\rightarrow \infty ,$ such that$\sqrt{%
K_{2,n}}/\left( \mu _{n}^{\min }\right) ^{2}\rightarrow 0$. (iii) $\left(
\mu _{n}^{\min }\right) /\mu _{g,n}\rightarrow 0,$ as $n\rightarrow \infty ,$
for $g\in \left\{ 1,...,d_{1}\right\} $. (iv) Let $H_{n}$ be as defined in
Assumption 3(iii) above, and suppose that there exists a positive constant $%
C,$ such that $\lambda _{\max }\left( H_{n}\right) \leq C<\infty $ and $%
\lambda _{\min }\left( H_{n}\right) \geq 1/C>0$ $a.s.,$ for all $n$
sufficiently large. (v) Let $e_{g}$ denote a $d\times 1$ elementary vector
whose $g^{th}$ element is $1,$ with all other elements (or components) equal
to $0$. Partition $H_{n}^{-1}$ as $H_{n}^{-1}=\overline{H}_{n}=\left( 
\begin{array}{cc}
\overline{H}_{1\cdot }^{\prime } & \overline{H}_{2\cdot }^{\prime }%
\end{array}%
\right) ^{\prime }$, where $\overline{H}_{1\cdot }$ is $d_{1}\times d$ and $%
\overline{H}_{2\cdot }$ is $d_{2}\times d$. Suppose that there exists a
positive constant $C_{\ast }$ such that $e_{g}^{\prime }\overline{H}_{2\cdot
}^{\prime }\overline{H}_{2\cdot }e_{g}\geq C_{\ast }>0$ $a.s.$for all $n$
sufficiently large and for $g\in \left\{ 1,...,d\right\} $.

\medskip

Note that writing the matrix $D_{\mu }$ in the form given by equation (\ref%
{Dmu}) may appear to require a particular ordering of the diagonal elements $%
\mu _{1n},..,\mu _{d_{1},n},\mu _{n}^{\min }$ of $D_{\mu }$, where $\mu
_{n}^{\min }$ is placed in the last $d_{2}$ diagonal position. However, it
is easily seen that the way $D_{\mu }$ is specified in equation (\ref{Dmu})
does not really lead to any loss of generality. In fact, a more general $%
D_{\mu }$ matrix, where not all of the diagonal elements grow at the same
rate, as $n\rightarrow \infty $, can always be put in the form given in
equation (\ref{Dmu}), via repermutation of the rows and columns of $D_{\mu }$%
. To see this, suppose that $\mu _{1n},..,\mu _{d_{1},n},\mu _{n}^{\min }$
are not ordered as in equation (\ref{Dmu}), so that we have some diagonal
matrix $D_{\mu }^{\ast },$ whose diagonal elements are $\mu _{1n},..,\mu
_{d_{1},n},\mu _{n}^{\min },$ but in some other ordering. Then, there exists
some permutation matrix $P$ such that $D_{\mu }=PD_{\mu }^{\ast }P^{\prime }$%
, where $D_{\mu }$ is the diagonal matrix given in equation (\ref{Dmu}).
Moreover, let the elements of $\widehat{\delta }^{\ast },\delta _{0}^{\ast }$%
, $c^{\ast }$, and $\widehat{V}^{\ast }$ be ordered in a way that is
conformable with $D_{\mu }^{\ast },$ and let $\widehat{\delta },\delta _{0}$%
, $c$, and $\widehat{V}$ be the corresponding vectors and matrix but with
elements ordered conformably with $D_{\mu }$. Then, it is easy to see that $%
\widehat{\delta }=P\widehat{\delta }^{\ast }$, $\delta _{0}=P\delta
_{0}^{\ast }$, $c=Pc^{\ast }$, $\widehat{V}=P\widehat{V}^{\ast }P^{\prime }$%
. Hence, by making use of these relations and of the fact that $P$ is an
orthogonal matrix, we further obtain that $\mathbb{T}_{L}^{\ast }=c^{\ast
\prime }\left( \widehat{\delta }^{\ast }-\delta _{0}^{\ast }\right) /\sqrt{%
c^{\ast \prime }\widehat{V}^{\ast }c^{\ast }}=c^{\ast \prime }P^{\prime
}P\left( \widehat{\delta }^{\ast }-\delta _{0}^{\ast }\right) /\sqrt{c^{\ast
\prime }P^{\prime }P\widehat{V}^{\ast }P^{\prime }Pc^{\ast }}=c^{\prime
}\left( \widehat{\delta }-\delta _{0}\right) /\sqrt{c^{\prime }\widehat{V}c}=%
\mathbb{T}_{L}$. It follows that the value of the t-statistic is invariant
to repermutation of the order of the elements of $\widehat{\delta }$, $%
\delta _{0}$, $c$, and $\widehat{V}$, so that the asymptotic distribution
which we derive for $\mathbb{T}_{L}$, under an assumed ordering of the
elements of $\widehat{\delta }$, $\delta _{0}$, $c$, and $\widehat{V}$ that
is conformable with equation (\ref{Dmu}) will still apply, even if the
t-statistic computed by the empirical researcher is based on some other
ordering.

Here, we let $D_{1}=diag\left( \mu _{1,n},..,\mu _{d_{1},n}\right) ,$ such
that $\left( \mu _{n}^{\min }\right) /\mu _{g,n}\rightarrow 0,$ as $%
n\rightarrow \infty ,$ for $g\in \left\{ 1,...,d_{1}\right\} $, where $d_{1}$
and $d_{2}$ are positive integers with $d_{1}+d_{2}=d$. This specification
excludes the case where $d_{1}=0$, or $d_{2}=d$, because this case has
already been covered by Theorem 5.

\medskip

\noindent \textbf{Theorem 6: }Suppose that Assumptions 1, 2, 3$^{\text{*}}$,
4-9 are satisfied; and, in what follows, let $e_{g}$ denote a $d\times 1$
elementary vector whose $g^{th}$ element is $1,$ with all other elements (or
components) equal to $0$, and define the t-statistics $\mathbb{T}_{L}=\left(
e_{g}^{\prime }\widehat{\delta }_{L}-r\right) /\sqrt{e_{g}^{\prime }\widehat{%
V}_{L}e_{g}}$ and $\mathbb{T}_{F}=\left( e_{g}^{\prime }\widehat{\delta }%
_{F}-r\right) /\sqrt{e_{g}^{\prime }\widehat{V}_{F}e_{g}}$.

\begin{enumerate}
\item[a.] For Case I, where $K_{2,n}/\left( \mu _{n}^{\min }\right)
^{2}=O\left( 1\right) $, the following results hold for any $g\in \left\{
1,...,d\right\} $.

\begin{enumerate}
\item[(i)] Under $H_{0}:e_{g}^{\prime }\delta _{0}=r$, $\mathbb{T}_{L}%
\overset{d}{\rightarrow }N\left( 0,1\right) $ and $\mathbb{T}_{F}\overset{d}{%
\rightarrow }N\left( 0,1\right) $.

\item[(ii) ] Under $H_{1}:e_{g}^{\prime }\delta _{0}\neq r$, with
probability approaching one, as $n\rightarrow \infty $, the following
results hold: $\mathbb{T}_{L}\rightarrow +\infty $ \ if $e_{g}^{\prime
}\delta _{0}>r$; $\mathbb{T}_{L}\rightarrow -\infty $ \ if $e_{g}^{\prime
}\delta _{0}<r$; $\mathbb{T}_{F}\rightarrow +\infty $\ if $e_{g}^{\prime
}\delta _{0}>r$; and $\mathbb{T}_{F}\rightarrow -\infty $ \ if $%
e_{g}^{\prime }\delta _{0}<r$.
\end{enumerate}

\item[b.] For Case II, where $K_{2,n}/\left( \mu _{n}^{\min }\right)
^{2}\rightarrow \infty $ but $\sqrt{K_{2,n}}/\left( \mu _{n}^{\min }\right)
^{2}\rightarrow 0$, the following results hold, for any $g\in \left\{
1,...,d\right\} $.

\begin{enumerate}
\item[(i)] Under $H_{0}:e_{g}^{\prime }\delta _{0}=r$, $\mathbb{T}_{L}%
\overset{d}{\rightarrow }N\left( 0,1\right) $ and $\mathbb{T}_{F}\overset{d}{%
\rightarrow }N\left( 0,1\right) $.

\item[(ii)] Under $H_{1}:e_{g}^{\prime }\delta _{0}\neq r$, with probability
approaching one, as $n\rightarrow \infty $, the following results hold: $%
\mathbb{T}_{L}\rightarrow +\infty $ \ if $e_{g}^{\prime }\delta _{0}>r$; $%
\mathbb{T}_{L}\rightarrow -\infty $ \ if $e_{g}^{\prime }\delta _{0}<r$; $%
\mathbb{T}_{F}\rightarrow +\infty $\ if $e_{g}^{\prime }\delta _{0}>r$; and $%
\mathbb{T}_{F}\rightarrow -\infty $ \ if $e_{g}^{\prime }\delta _{0}<r$.
\end{enumerate}
\end{enumerate}

\medskip

Comparing Assumption 3* with Assumption 3, we see that the one additional
side condition required for Theorem 6 is the condition placed on elements of 
$\overline{H}_{2\cdot }$ in part (v) of Assumption 3*. To test hypotheses
involving only the $g^{th}$ coefficient, this condition will be violated
only if the $g^{th}$ column of $\overline{H}_{2\cdot }$ does not have a
single nonzero entry, which seems unlikely in most practical applications.

To date, papers in the weak instrument literature have focused primarily on
size control, with little attention paid to test consistency under weak
identification. One exception is a recent paper by Mikusheva and Sun (2020),
which shows that a condition similar to $\sqrt{K_{2,n}}/\left( \mu
_{n}^{\min }\right) ^{2}\rightarrow 0$ is both necessary and sufficient for
the existence of a consistent test. Interpreted in light of their result,
the results presented in Theorems 5 and 6 above prove that t-tests based on
FELIM and FEFUL are consistent as long as instruments are strong enough so
that consistency in hypothesis testing is possible. In contrast, t-tests
based on estimators such as the 2SLS estimator will only be consistent if $%
K_{2,n}/\left( \mu _{n}^{\min }\right) ^{2}\rightarrow 0$ (i.e., under
stronger instruments). Test statistics based on LIML also have undesirable
properties under many weak instrument asymptotics, when there is error
heteroskedasticity. In addition, note that one advantage of t-tests is that
they are particularly easy to apply if one is interested in testing against
one-sided alternatives. The results of Theorems 5 and 6 show that, when the
null hypothesis is incorrect, t-tests based on FELIM and FEFUL diverge in
the direction of the true alternative, with probability approaching one,
even in situations where identification is weaker than that typically
assumed under standard large sample theory, provided of course that $\sqrt{%
K_{2,n}}/\left( \mu _{n}^{\min }\right) ^{2}\rightarrow 0$. Hence, the test
statistics proposed in this paper should be useful to empirical researchers
interested in testing whether an effect in a particular direction is
statistically significant.

\section{\noindent Monte Carlo Results}

In this section, we report some Monte Carlo results based on a setup similar
to that of Hausman et al. (2012), but extended to the cluster-sample/panel
data setting. In particular, we consider two closely related groups of
data-generating processes:

\noindent \textbf{DGP 1:}%
\begin{eqnarray*}
y_{\left( i,t\right) } &=&\underset{1\times 1}{\delta }\underset{1\times 1}{%
x_{\left( i,t\right) }}+\underset{1\times 9}{\gamma ^{\prime }}\underset{%
9\times 1}{Z_{1,\left( i,t\right) }}+\alpha _{i}+\varepsilon _{\left(
i,t\right) }\text{,} \\
x_{\left( i,t\right) } &=&\underset{1\times 1}{\pi }\underset{1\times 1}{%
z_{2,\left( i,t\right) }}+\underset{1\times 9}{\Phi ^{\prime }}\underset{%
9\times 1}{Z_{1,\left( i,t\right) }}+\xi _{i}+u_{\left( i,t\right) }\text{.}
\end{eqnarray*}%
In all experiments that utilize this DGP, we take $\gamma =\left( 
\begin{array}{cccc}
1 & 1 & \cdots & 1%
\end{array}%
\right) ^{\prime }$ and $\Phi =\left( 
\begin{array}{cccc}
1 & 1 & \cdots & 1%
\end{array}%
\right) ^{\prime }$.

\noindent \textbf{DGP 2:}%
\begin{eqnarray*}
y_{\left( i,t\right) } &=&\underset{1\times 1}{\delta }\underset{1\times 1}{%
x_{\left( i,t\right) }}+\underset{1\times 1}{\gamma _{1}}\underset{1\times 1}%
{z_{1,\left( i,t\right) }}+\alpha _{i}+\varepsilon _{\left( i,t\right) }%
\text{,} \\
x_{\left( i,t\right) } &=&\underset{1\times 1}{\pi }\underset{1\times 1}{%
z_{2,\left( i,t\right) }}+\underset{1\times 1}{\Phi _{1}}\underset{1\times 1}%
{z_{1,\left( i,t\right) }}+\xi _{i}+u_{\left( i,t\right) }\text{.}
\end{eqnarray*}%
In all experiments that utilize this DGP, we take $\gamma _{1}=1$ and $\Phi
_{1}=1$. Additionally, we set $n=300$ and $T_{i}=3,$ for each $i\in \left\{
1,2,...,300\right\} $, so that $m_{n}=900$. We also take $\left\{
z_{1,\left( i,t\right) }\right\} _{\left( i,t\right) =1}^{900}\equiv
i.i.d.N\left( 0,1\right) $, $\left\{ z_{2,\left( i,t\right) }\right\}
_{\left( i,t\right) =1}^{900}\equiv i.i.d.N\left( 0,1\right) $, and $\left\{
u_{\left( i,t\right) }\right\} _{\left( i,t\right) =1}^{900}\equiv
i.i.d.N\left( 0,1\right) $. Moreover, $z_{1,\left( i,t\right) }$, $%
z_{2,\left( i,t\right) }$, and $u_{\left( i,t\right) }$ are all mutually
independent. The $\left( i,t\right) ^{th}$ observation of the vector of
instruments is specified to be $Z_{2,\left( i,t\right) }=\left( 
\begin{array}{cccccc}
z_{2,\left( i,t\right) } & z_{2,\left( i,t\right) }^{2} & z_{2,\left(
i,t\right) }^{3} & z_{2,\left( i,t\right) }^{4} & z_{2,\left( i,t\right)
}D_{\left( i,t\right) ,1} & \cdots%
\end{array}%
\right. $

\noindent $\left. 
\begin{array}{cc}
\cdots & z_{2,\left( i,t\right) }D_{\left( i,t\right) ,5}%
\end{array}%
\right) ^{\prime }$, while the $\left( i,t\right) ^{th}$ observation of the
vector of included exogenous regressors, or covariates, is given by $%
Z_{1,\left( i,t\right) }=\left( 
\begin{array}{cccccc}
z_{1,\left( i,t\right) } & z_{1,\left( i,t\right) }^{2} & z_{1,\left(
i,t\right) }^{3} & z_{1,\left( i,t\right) }^{4} & z_{1,\left( i,t\right)
}D_{\left( i,t\right) ,1} & \cdots%
\end{array}%
\right. $

\noindent $\left. 
\begin{array}{cc}
\cdots & z_{1,\left( i,t\right) }D_{\left( i,t\right) ,5}%
\end{array}%
\right) ^{\prime }$, where $D_{\left( i,t\right) ,k}\in \left\{ 0,1\right\} $
for $k\in \left\{ 1,2,..,5\right\} $ is a binary variable such that $\Pr
\left( D_{\left( i,t\right) ,k}=1\right) =1/2,$ and where $\left\{ D_{\left(
i,t\right) ,k}\right\} $ is independent across both $\left( i,t\right) $ and 
$k$. The structural disturbance, $\varepsilon _{\left( i,t\right) },$ is
allowed to exhibit conditional heteroskedasticity in a manner similar to the
design given in Hausman et al. (2012). In particular, under DGP1, we take%
\begin{equation}
\varepsilon _{\left( i,t\right) }=\rho u_{\left( i,t\right) }+\sqrt{\frac{%
1-\rho ^{2}}{\phi ^{2}+\left( 0.86\right) ^{4}}}\left( \phi v_{1,\left(
i,t\right) }+0.86v_{2,\left( i,t\right) }\right) ,  \label{epsilon}
\end{equation}%
where $v_{1,\left( i,t\right) }|Z_{1,\left( i,t\right) },z_{2,\left(
i,t\right) }\sim N\left( 0,\kappa _{1}\left[ 1+\left( \iota _{9}^{\prime
}Z_{1,\left( i,t\right) }+z_{2,\left( i,t\right) }\right) ^{2}\right]
\right) $ and $v_{2,\left( i,t\right) }\sim N\left( 0,\left( 0.86\right)
^{2}\right) $. Both of these distributions are assumed to be independent
across the index $\left( i,t\right) $. Under DGP2, $\varepsilon _{\left(
i,t\right) }$ has a similar structure as given in equation (\ref{epsilon})
above, except that we take $v_{1,\left( i,t\right) }|Z_{1,\left( i,t\right)
},z_{2,\left( i,t\right) }\equiv v_{1,\left( i,t\right) }|z_{1,\left(
i,t\right) },z_{2,\left( i,t\right) }\sim N\left( 0,\kappa _{2}\left[
1+\left( z_{1,\left( i,t\right) }+z_{2,\left( i,t\right) }\right) ^{2}\right]
\right) $. Also, $\kappa _{1}$ and $\kappa _{2}$ are normalization constants
chosen so that under both DGP1 and DGP2 the unconditional variance, $%
Var\left( v_{1,\left( i,t\right) }\right) $, is equal to 1. For all
experiments reported below, we set $\rho =0.3$ and, under both DGP1 and
DGP2, we choose the parameter $\phi $, so that the R-squared for the
regression of $\varepsilon ^{2}$ on the instruments and the included
exogenous variables take the values $0$, $0.1$, and $0.2$.

Our simulation study examines the finite sample properties of our three
estimators (FEJIV, FELIM, and FEFUL) and their associated t-statistics.
Additionally, we compare the performance of our estimators with the 2SLS
estimator, the IJIVE1 estimator originally proposed in Ackerberg and
Devereux (2009), the IJIVE2 estimator introduced in Evdokimov and Koles\'{a}%
r (2018), and the UJIVE estimator originally proposed in Koles\'{a}r (2013)
and further studied in Evdokimov and Koles\'{a}r (2018). The comparison of
these point estimators is made on the basis of median bias and nine decile
range. We also evaluate the associated t-statistics for these estimators on
the basis of size control, as measured by their rejection frequencies under
the null hypothesis $H_{0}:\delta =0$.

\noindent \qquad The results of our Monte Carlo study are reported in Tables
1-6 below.

\begin{tabular}{|l|l|l|l|l|l|l|l|l|}
\hline\hline
\multicolumn{9}{|c|}{Table 1: Median Bias, DGP 1} \\ \hline\hline
$\mu ^{2}$ & $\mathcal{R}_{\varepsilon ^{2}|z_{1}^{2}}^{2}$ & 2SLS & IJIVE1
& IJIVE2 & UJIVE & FEJIV & FELIM & FEFUL \\ \hline
& \multicolumn{1}{|c|}{0} & \multicolumn{1}{|c|}{0.1056} & 
\multicolumn{1}{|c|}{0.0429} & \multicolumn{1}{|c|}{0.0417} & 
\multicolumn{1}{|c|}{0.3405} & \multicolumn{1}{|c|}{0.0012} & 
\multicolumn{1}{|c|}{0.0029} & \multicolumn{1}{|c|}{0.0157} \\ \cline{2-9}
24 & \multicolumn{1}{|c|}{0.1} & \multicolumn{1}{|c|}{0.1028} & 
\multicolumn{1}{|c|}{0.0417} & \multicolumn{1}{|c|}{0.0400} & 
\multicolumn{1}{|c|}{0.3474} & \multicolumn{1}{|c|}{-0.0002} & 
\multicolumn{1}{|c|}{0.0007} & \multicolumn{1}{|c|}{0.0122} \\ \cline{2-9}
& \multicolumn{1}{|c|}{0.2} & \multicolumn{1}{|c|}{0.1060} & 
\multicolumn{1}{|c|}{0.0423} & \multicolumn{1}{|c|}{0.0417} & 
\multicolumn{1}{|c|}{0.3517} & \multicolumn{1}{|c|}{0.0022} & 
\multicolumn{1}{|c|}{0.0083} & \multicolumn{1}{|c|}{0.0200} \\ \hline
& \multicolumn{1}{|c|}{0} & \multicolumn{1}{|c|}{0.0852} & 
\multicolumn{1}{|c|}{0.0321} & \multicolumn{1}{|c|}{0.0298} & 
\multicolumn{1}{|c|}{0.2241} & \multicolumn{1}{|c|}{-0.0059} & 
\multicolumn{1}{|c|}{0.0009} & \multicolumn{1}{|c|}{0.0105} \\ \cline{2-9}
32 & \multicolumn{1}{|c|}{0.1} & \multicolumn{1}{|c|}{0.0830} & 
\multicolumn{1}{|c|}{0.0273} & \multicolumn{1}{|c|}{0.0264} & 
\multicolumn{1}{|c|}{0.2376} & \multicolumn{1}{|c|}{-0.0131} & 
\multicolumn{1}{|c|}{-0.0010} & \multicolumn{1}{|c|}{0.0082} \\ \cline{2-9}
& \multicolumn{1}{|c|}{0.2} & \multicolumn{1}{|c|}{0.0842} & 
\multicolumn{1}{|c|}{0.0295} & \multicolumn{1}{|c|}{0.0285} & 
\multicolumn{1}{|c|}{0.2453} & \multicolumn{1}{|c|}{-0.0094} & 
\multicolumn{1}{|c|}{0.0026} & \multicolumn{1}{|c|}{0.0118} \\ \hline
& \multicolumn{1}{|c|}{0} & \multicolumn{1}{|c|}{0.0729} & 
\multicolumn{1}{|c|}{0.0247} & \multicolumn{1}{|c|}{0.0244} & 
\multicolumn{1}{|c|}{0.1622} & \multicolumn{1}{|c|}{-0.0062} & 
\multicolumn{1}{|c|}{-0.0004} & \multicolumn{1}{|c|}{0.0079} \\ \cline{2-9}
40 & \multicolumn{1}{|c|}{0.1} & \multicolumn{1}{|c|}{0.0714} & 
\multicolumn{1}{|c|}{0.0245} & \multicolumn{1}{|c|}{0.0240} & 
\multicolumn{1}{|c|}{0.1783} & \multicolumn{1}{|c|}{-0.0084} & 
\multicolumn{1}{|c|}{-0.0014} & \multicolumn{1}{|c|}{0.0058} \\ \cline{2-9}
& \multicolumn{1}{|c|}{0.2} & \multicolumn{1}{|c|}{0.0724} & 
\multicolumn{1}{|c|}{0.0270} & \multicolumn{1}{|c|}{0.0255} & 
\multicolumn{1}{|c|}{0.1802} & \multicolumn{1}{|c|}{-0.0050} & 
\multicolumn{1}{|c|}{0.0032} & \multicolumn{1}{|c|}{0.0104} \\ \hline
& \multicolumn{1}{|c|}{0} & \multicolumn{1}{|c|}{0.0625} & 
\multicolumn{1}{|c|}{0.0212} & \multicolumn{1}{|c|}{0.0200} & 
\multicolumn{1}{|c|}{0.1269} & \multicolumn{1}{|c|}{-0.0068} & 
\multicolumn{1}{|c|}{0.0004} & \multicolumn{1}{|c|}{0.0065} \\ \cline{2-9}
48 & \multicolumn{1}{|c|}{0.1} & \multicolumn{1}{|c|}{0.0605} & 
\multicolumn{1}{|c|}{0.0182} & \multicolumn{1}{|c|}{0.0176} & 
\multicolumn{1}{|c|}{0.1272} & \multicolumn{1}{|c|}{-0.0117} & 
\multicolumn{1}{|c|}{-0.0020} & \multicolumn{1}{|c|}{0.0042} \\ \cline{2-9}
& \multicolumn{1}{|c|}{0.2} & \multicolumn{1}{|c|}{0.0605} & 
\multicolumn{1}{|c|}{0.0190} & \multicolumn{1}{|c|}{0.0182} & 
\multicolumn{1}{|c|}{0.1414} & \multicolumn{1}{|c|}{-0.0082} & 
\multicolumn{1}{|c|}{0.0009} & \multicolumn{1}{|c|}{0.0073} \\ \hline
\end{tabular}

{\small Results based on 10,000 simulations}

\begin{tabular}{|l|l|l|l|l|l|l|l|l|}
\hline\hline
\multicolumn{9}{|c|}{Table 2: Nine Decile Range 0.05 to 0.95\QQfnmark{%
By nine decile range we mean the range between the $0.05$ and the $0.95$
quantiles.}, DGP 1} \\ \hline\hline
$\mu ^{2}$ & $\mathcal{R}_{\varepsilon ^{2}|z_{1}^{2}}^{2}$ & 2SLS & IJIVE1
& IJIVE2 & UJIVE & FEJIV & FELIM & FEFUL \\ \hline
& \multicolumn{1}{|c|}{0} & \multicolumn{1}{|c|}{0.5986} & 
\multicolumn{1}{|c|}{0.9415} & \multicolumn{1}{|c|}{0.9447} & 
\multicolumn{1}{|c|}{6.1610} & \multicolumn{1}{|c|}{1.5655} & 
\multicolumn{1}{|c|}{1.2073} & \multicolumn{1}{|c|}{1.0575} \\ \cline{2-9}
24 & \multicolumn{1}{|c|}{0.1} & \multicolumn{1}{|c|}{0.6032} & 
\multicolumn{1}{|c|}{0.9402} & \multicolumn{1}{|c|}{0.9428} & 
\multicolumn{1}{|c|}{6.1080} & \multicolumn{1}{|c|}{1.5857} & 
\multicolumn{1}{|c|}{1.2502} & \multicolumn{1}{|c|}{1.0800} \\ \cline{2-9}
& \multicolumn{1}{|c|}{0.2} & \multicolumn{1}{|c|}{0.5968} & 
\multicolumn{1}{|c|}{0.9392} & \multicolumn{1}{|c|}{0.9360} & 
\multicolumn{1}{|c|}{6.2265} & \multicolumn{1}{|c|}{1.5635} & 
\multicolumn{1}{|c|}{1.1811} & \multicolumn{1}{|c|}{1.0528} \\ \hline
& \multicolumn{1}{|c|}{0} & \multicolumn{1}{|c|}{0.5386} & 
\multicolumn{1}{|c|}{0.7618} & \multicolumn{1}{|c|}{0.7639} & 
\multicolumn{1}{|c|}{5.7060} & \multicolumn{1}{|c|}{1.1133} & 
\multicolumn{1}{|c|}{0.9156} & \multicolumn{1}{|c|}{0.8485} \\ \cline{2-9}
32 & \multicolumn{1}{|c|}{0.1} & \multicolumn{1}{|c|}{0.5367} & 
\multicolumn{1}{|c|}{0.7662} & \multicolumn{1}{|c|}{0.7657} & 
\multicolumn{1}{|c|}{6.2091} & \multicolumn{1}{|c|}{1.1017} & 
\multicolumn{1}{|c|}{0.9167} & \multicolumn{1}{|c|}{0.8526} \\ \cline{2-9}
& \multicolumn{1}{|c|}{0.2} & \multicolumn{1}{|c|}{0.5492} & 
\multicolumn{1}{|c|}{0.7763} & \multicolumn{1}{|c|}{0.7716} & 
\multicolumn{1}{|c|}{5.8831} & \multicolumn{1}{|c|}{1.1172} & 
\multicolumn{1}{|c|}{0.9282} & \multicolumn{1}{|c|}{0.8678} \\ \hline
& \multicolumn{1}{|c|}{0} & \multicolumn{1}{|c|}{0.4994} & 
\multicolumn{1}{|c|}{0.6591} & \multicolumn{1}{|c|}{0.6570} & 
\multicolumn{1}{|c|}{5.5880} & \multicolumn{1}{|c|}{0.8722} & 
\multicolumn{1}{|c|}{0.7608} & \multicolumn{1}{|c|}{0.7229} \\ \cline{2-9}
40 & \multicolumn{1}{|c|}{0.1} & \multicolumn{1}{|c|}{0.4961} & 
\multicolumn{1}{|c|}{0.6412} & \multicolumn{1}{|c|}{0.6415} & 
\multicolumn{1}{|c|}{5.1816} & \multicolumn{1}{|c|}{0.8585} & 
\multicolumn{1}{|c|}{0.7572} & \multicolumn{1}{|c|}{0.7170} \\ \cline{2-9}
& \multicolumn{1}{|c|}{0.2} & \multicolumn{1}{|c|}{0.4970} & 
\multicolumn{1}{|c|}{0.6528} & \multicolumn{1}{|c|}{0.6546} & 
\multicolumn{1}{|c|}{5.5664} & \multicolumn{1}{|c|}{0.8550} & 
\multicolumn{1}{|c|}{0.7454} & \multicolumn{1}{|c|}{0.7109} \\ \hline
& \multicolumn{1}{|c|}{0} & \multicolumn{1}{|c|}{0.4608} & 
\multicolumn{1}{|c|}{0.5790} & \multicolumn{1}{|c|}{0.5764} & 
\multicolumn{1}{|c|}{4.7318} & \multicolumn{1}{|c|}{0.7231} & 
\multicolumn{1}{|c|}{0.6490} & \multicolumn{1}{|c|}{0.6237} \\ \cline{2-9}
48 & \multicolumn{1}{|c|}{0.1} & \multicolumn{1}{|c|}{0.4586} & 
\multicolumn{1}{|c|}{0.5822} & \multicolumn{1}{|c|}{0.5822} & 
\multicolumn{1}{|c|}{5.0871} & \multicolumn{1}{|c|}{0.7322} & 
\multicolumn{1}{|c|}{0.6516} & \multicolumn{1}{|c|}{0.6288} \\ \cline{2-9}
& \multicolumn{1}{|c|}{0.2} & \multicolumn{1}{|c|}{0.4700} & 
\multicolumn{1}{|c|}{0.5873} & \multicolumn{1}{|c|}{0.5855} & 
\multicolumn{1}{|c|}{5.0922} & \multicolumn{1}{|c|}{0.7309} & 
\multicolumn{1}{|c|}{0.6636} & \multicolumn{1}{|c|}{0.6389} \\ \hline
\end{tabular}%
\QQfntext{0}{
By nine decile range we mean the range between the $0.05$ and the $0.95$
quantiles.}

{\small Results based on 10,000 simulations}

\begin{tabular}{|l|l|l|l|l|l|l|l|l|}
\hline\hline
\multicolumn{9}{|c|}{Table 3: 0.05 Rejection Frequencies\QQfnmark{%
See Ackerberg and Devereux (2009), Koles\'{a}r (2013), and Evdokimov and
Koles\'{a}r (2018) for formulae for the estimators IJIVE1, IJIVE2, and UJIVE
as well as for the standard errors used in constructing the t-statistics for
these estimators.}, DGP 1} \\ \hline\hline
$\mu ^{2}$ & $\mathcal{R}_{\varepsilon ^{2}|z_{1}^{2}}^{2}$ & 2SLS & IJIVE1
& IJIVE2 & UJIVE & FEJIV & FELIM & FEFUL \\ \hline
& \multicolumn{1}{|c|}{0} & \multicolumn{1}{|c|}{0.1861} & 
\multicolumn{1}{|c|}{0.1056} & \multicolumn{1}{|c|}{0.0961} & 
\multicolumn{1}{|c|}{0.5167} & \multicolumn{1}{|c|}{0.0322} & 
\multicolumn{1}{|c|}{0.0559} & \multicolumn{1}{|c|}{0.0588} \\ \cline{2-9}
24 & \multicolumn{1}{|c|}{0.1} & \multicolumn{1}{|c|}{0.1809} & 
\multicolumn{1}{|c|}{0.1013} & \multicolumn{1}{|c|}{0.0920} & 
\multicolumn{1}{|c|}{0.5333} & \multicolumn{1}{|c|}{0.0307} & 
\multicolumn{1}{|c|}{0.0583} & \multicolumn{1}{|c|}{0.0602} \\ \cline{2-9}
& \multicolumn{1}{|c|}{0.2} & \multicolumn{1}{|c|}{0.1894} & 
\multicolumn{1}{|c|}{0.1038} & \multicolumn{1}{|c|}{0.0940} & 
\multicolumn{1}{|c|}{0.5318} & \multicolumn{1}{|c|}{0.0317} & 
\multicolumn{1}{|c|}{0.0579} & \multicolumn{1}{|c|}{0.0610} \\ \hline
& \multicolumn{1}{|c|}{0} & \multicolumn{1}{|c|}{0.1716} & 
\multicolumn{1}{|c|}{0.1077} & \multicolumn{1}{|c|}{0.0962} & 
\multicolumn{1}{|c|}{0.5281} & \multicolumn{1}{|c|}{0.0313} & 
\multicolumn{1}{|c|}{0.0486} & \multicolumn{1}{|c|}{0.0515} \\ \cline{2-9}
32 & \multicolumn{1}{|c|}{0.1} & \multicolumn{1}{|c|}{0.1678} & 
\multicolumn{1}{|c|}{0.1061} & \multicolumn{1}{|c|}{0.0974} & 
\multicolumn{1}{|c|}{0.5307} & \multicolumn{1}{|c|}{0.0367} & 
\multicolumn{1}{|c|}{0.0556} & \multicolumn{1}{|c|}{0.0587} \\ \cline{2-9}
& \multicolumn{1}{|c|}{0.2} & \multicolumn{1}{|c|}{0.1767} & 
\multicolumn{1}{|c|}{0.1124} & \multicolumn{1}{|c|}{0.1033} & 
\multicolumn{1}{|c|}{0.5342} & \multicolumn{1}{|c|}{0.0373} & 
\multicolumn{1}{|c|}{0.0581} & \multicolumn{1}{|c|}{0.0620} \\ \hline
& \multicolumn{1}{|c|}{0} & \multicolumn{1}{|c|}{0.1662} & 
\multicolumn{1}{|c|}{0.1143} & \multicolumn{1}{|c|}{0.1055} & 
\multicolumn{1}{|c|}{0.5390} & \multicolumn{1}{|c|}{0.0371} & 
\multicolumn{1}{|c|}{0.0510} & \multicolumn{1}{|c|}{0.0536} \\ \cline{2-9}
40 & \multicolumn{1}{|c|}{0.1} & \multicolumn{1}{|c|}{0.1600} & 
\multicolumn{1}{|c|}{0.1084} & \multicolumn{1}{|c|}{0.0987} & 
\multicolumn{1}{|c|}{0.5573} & \multicolumn{1}{|c|}{0.0371} & 
\multicolumn{1}{|c|}{0.0519} & \multicolumn{1}{|c|}{0.0553} \\ \cline{2-9}
& \multicolumn{1}{|c|}{0.2} & \multicolumn{1}{|c|}{0.1643} & 
\multicolumn{1}{|c|}{0.1129} & \multicolumn{1}{|c|}{0.1033} & 
\multicolumn{1}{|c|}{0.5469} & \multicolumn{1}{|c|}{0.0387} & 
\multicolumn{1}{|c|}{0.0521} & \multicolumn{1}{|c|}{0.0555} \\ \hline
& \multicolumn{1}{|c|}{0} & \multicolumn{1}{|c|}{0.1551} & 
\multicolumn{1}{|c|}{0.1166} & \multicolumn{1}{|c|}{0.1039} & 
\multicolumn{1}{|c|}{0.5749} & \multicolumn{1}{|c|}{0.0348} & 
\multicolumn{1}{|c|}{0.0479} & \multicolumn{1}{|c|}{0.0508} \\ \cline{2-9}
48 & \multicolumn{1}{|c|}{0.1} & \multicolumn{1}{|c|}{0.1542} & 
\multicolumn{1}{|c|}{0.1126} & \multicolumn{1}{|c|}{0.1033} & 
\multicolumn{1}{|c|}{0.5695} & \multicolumn{1}{|c|}{0.0399} & 
\multicolumn{1}{|c|}{0.0531} & \multicolumn{1}{|c|}{0.0555} \\ \cline{2-9}
& \multicolumn{1}{|c|}{0.2} & \multicolumn{1}{|c|}{0.1643} & 
\multicolumn{1}{|c|}{0.1200} & \multicolumn{1}{|c|}{0.1102} & 
\multicolumn{1}{|c|}{0.5733} & \multicolumn{1}{|c|}{0.0409} & 
\multicolumn{1}{|c|}{0.0567} & \multicolumn{1}{|c|}{0.0603} \\ \hline
\end{tabular}%
\QQfntext{0}{
See Ackerberg and Devereux (2009), Koles\'{a}r (2013), and Evdokimov and
Koles\'{a}r (2018) for formulae for the estimators IJIVE1, IJIVE2, and UJIVE
as well as for the standard errors used in constructing the t-statistics for
these estimators.}

{\small Results based on 10,000 simulations}

\begin{tabular}{|l|l|l|l|l|l|l|l|l|}
\hline\hline
\multicolumn{9}{|c|}{Table 4: Median Bias, DGP 2} \\ \hline\hline
$\mu ^{2}$ & $\mathcal{R}_{\varepsilon ^{2}|z_{1}^{2}}^{2}$ & 2SLS & IJIVE1
& IJIVE2 & UJIVE & FEJIV & FELIM & FEFUL \\ \hline
& \multicolumn{1}{|c|}{0} & \multicolumn{1}{|c|}{0.1056} & 
\multicolumn{1}{|c|}{0.0429} & \multicolumn{1}{|c|}{0.0417} & 
\multicolumn{1}{|c|}{-0.0095} & \multicolumn{1}{|c|}{0.0012} & 
\multicolumn{1}{|c|}{0.0029} & \multicolumn{1}{|c|}{0.0157} \\ \cline{2-9}
24 & \multicolumn{1}{|c|}{0.1} & \multicolumn{1}{|c|}{0.1030} & 
\multicolumn{1}{|c|}{0.0403} & \multicolumn{1}{|c|}{0.0397} & 
\multicolumn{1}{|c|}{-0.0110} & \multicolumn{1}{|c|}{-0.0037} & 
\multicolumn{1}{|c|}{0.0034} & \multicolumn{1}{|c|}{0.0155} \\ \cline{2-9}
& \multicolumn{1}{|c|}{0.2} & \multicolumn{1}{|c|}{0.1076} & 
\multicolumn{1}{|c|}{0.0457} & \multicolumn{1}{|c|}{0.0445} & 
\multicolumn{1}{|c|}{-0.0047} & \multicolumn{1}{|c|}{0.0051} & 
\multicolumn{1}{|c|}{0.0135} & \multicolumn{1}{|c|}{0.0258} \\ \hline
& \multicolumn{1}{|c|}{0} & \multicolumn{1}{|c|}{0.0852} & 
\multicolumn{1}{|c|}{0.0321} & \multicolumn{1}{|c|}{0.0298} & 
\multicolumn{1}{|c|}{-0.0120} & \multicolumn{1}{|c|}{-0.0059} & 
\multicolumn{1}{|c|}{0.0009} & \multicolumn{1}{|c|}{0.0105} \\ \cline{2-9}
32 & \multicolumn{1}{|c|}{0.1} & \multicolumn{1}{|c|}{0.0837} & 
\multicolumn{1}{|c|}{0.0285} & \multicolumn{1}{|c|}{0.0281} & 
\multicolumn{1}{|c|}{-0.0137} & \multicolumn{1}{|c|}{-0.0102} & 
\multicolumn{1}{|c|}{-0.0007} & \multicolumn{1}{|c|}{0.0087} \\ \cline{2-9}
& \multicolumn{1}{|c|}{0.2} & \multicolumn{1}{|c|}{0.0869} & 
\multicolumn{1}{|c|}{0.0315} & \multicolumn{1}{|c|}{0.0305} & 
\multicolumn{1}{|c|}{-0.0138} & \multicolumn{1}{|c|}{-0.0079} & 
\multicolumn{1}{|c|}{0.0058} & \multicolumn{1}{|c|}{0.0156} \\ \hline
& \multicolumn{1}{|c|}{0} & \multicolumn{1}{|c|}{0.0729} & 
\multicolumn{1}{|c|}{0.0247} & \multicolumn{1}{|c|}{0.0244} & 
\multicolumn{1}{|c|}{-0.0135} & \multicolumn{1}{|c|}{-0.0062} & 
\multicolumn{1}{|c|}{-0.0004} & \multicolumn{1}{|c|}{0.0079} \\ \cline{2-9}
40 & \multicolumn{1}{|c|}{0.1} & \multicolumn{1}{|c|}{0.0715} & 
\multicolumn{1}{|c|}{0.0255} & \multicolumn{1}{|c|}{0.0246} & 
\multicolumn{1}{|c|}{-0.0137} & \multicolumn{1}{|c|}{-0.0069} & 
\multicolumn{1}{|c|}{-0.0002} & \multicolumn{1}{|c|}{0.0073} \\ \cline{2-9}
& \multicolumn{1}{|c|}{0.2} & \multicolumn{1}{|c|}{0.0752} & 
\multicolumn{1}{|c|}{0.0277} & \multicolumn{1}{|c|}{0.0271} & 
\multicolumn{1}{|c|}{-0.0077} & \multicolumn{1}{|c|}{-0.0030} & 
\multicolumn{1}{|c|}{0.0082} & \multicolumn{1}{|c|}{0.0148} \\ \hline
& \multicolumn{1}{|c|}{0} & \multicolumn{1}{|c|}{0.0625} & 
\multicolumn{1}{|c|}{0.0212} & \multicolumn{1}{|c|}{0.0200} & 
\multicolumn{1}{|c|}{-0.0096} & \multicolumn{1}{|c|}{-0.0068} & 
\multicolumn{1}{|c|}{0.0004} & \multicolumn{1}{|c|}{0.0065} \\ \cline{2-9}
48 & \multicolumn{1}{|c|}{0.1} & \multicolumn{1}{|c|}{0.0603} & 
\multicolumn{1}{|c|}{0.0196} & \multicolumn{1}{|c|}{0.0185} & 
\multicolumn{1}{|c|}{-0.0108} & \multicolumn{1}{|c|}{-0.0098} & 
\multicolumn{1}{|c|}{-0.0013} & \multicolumn{1}{|c|}{0.0048} \\ \cline{2-9}
& \multicolumn{1}{|c|}{0.2} & \multicolumn{1}{|c|}{0.0635} & 
\multicolumn{1}{|c|}{0.0208} & \multicolumn{1}{|c|}{0.0198} & 
\multicolumn{1}{|c|}{-0.0111} & \multicolumn{1}{|c|}{-0.0089} & 
\multicolumn{1}{|c|}{0.0026} & \multicolumn{1}{|c|}{0.0085} \\ \hline
\end{tabular}

{\small Results based on 10,000 simulations}

\begin{tabular}{|l|l|l|l|l|l|l|l|l|}
\hline\hline
\multicolumn{9}{|c|}{Table 5: Nine Decile Range 0.05 to 0.95, DGP 2} \\ 
\hline\hline
$\mu ^{2}$ & $\mathcal{R}_{\varepsilon ^{2}|z_{1}^{2}}^{2}$ & 2SLS & IJIVE1
& IJIVE2 & UJIVE & FEJIV & FELIM & FEFUL \\ \hline
& \multicolumn{1}{|c|}{0} & \multicolumn{1}{|c|}{0.5986} & 
\multicolumn{1}{|c|}{0.9415} & \multicolumn{1}{|c|}{0.9447} & 
\multicolumn{1}{|c|}{1.5682} & \multicolumn{1}{|c|}{1.5655} & 
\multicolumn{1}{|c|}{1.2073} & \multicolumn{1}{|c|}{1.0575} \\ \cline{2-9}
24 & \multicolumn{1}{|c|}{0.1} & \multicolumn{1}{|c|}{0.6153} & 
\multicolumn{1}{|c|}{0.9693} & \multicolumn{1}{|c|}{0.9704} & 
\multicolumn{1}{|c|}{1.6082} & \multicolumn{1}{|c|}{1.6216} & 
\multicolumn{1}{|c|}{1.3049} & \multicolumn{1}{|c|}{1.1267} \\ \cline{2-9}
& \multicolumn{1}{|c|}{0.2} & \multicolumn{1}{|c|}{0.6399} & 
\multicolumn{1}{|c|}{1.0004} & \multicolumn{1}{|c|}{0.9935} & 
\multicolumn{1}{|c|}{1.6862} & \multicolumn{1}{|c|}{1.6536} & 
\multicolumn{1}{|c|}{1.3127} & \multicolumn{1}{|c|}{1.1508} \\ \hline
& \multicolumn{1}{|c|}{0} & \multicolumn{1}{|c|}{0.5386} & 
\multicolumn{1}{|c|}{0.7618} & \multicolumn{1}{|c|}{0.7639} & 
\multicolumn{1}{|c|}{1.0812} & \multicolumn{1}{|c|}{1.1133} & 
\multicolumn{1}{|c|}{0.9156} & \multicolumn{1}{|c|}{0.8485} \\ \cline{2-9}
32 & \multicolumn{1}{|c|}{0.1} & \multicolumn{1}{|c|}{0.5574} & 
\multicolumn{1}{|c|}{0.7975} & \multicolumn{1}{|c|}{0.7949} & 
\multicolumn{1}{|c|}{1.1461} & \multicolumn{1}{|c|}{1.1600} & 
\multicolumn{1}{|c|}{0.9613} & \multicolumn{1}{|c|}{0.8939} \\ \cline{2-9}
& \multicolumn{1}{|c|}{0.2} & \multicolumn{1}{|c|}{0.5821} & 
\multicolumn{1}{|c|}{0.8192} & \multicolumn{1}{|c|}{0.8103} & 
\multicolumn{1}{|c|}{1.1639} & \multicolumn{1}{|c|}{1.1644} & 
\multicolumn{1}{|c|}{1.0028} & \multicolumn{1}{|c|}{0.9361} \\ \hline
& \multicolumn{1}{|c|}{0} & \multicolumn{1}{|c|}{0.4994} & 
\multicolumn{1}{|c|}{0.6591} & \multicolumn{1}{|c|}{0.6570} & 
\multicolumn{1}{|c|}{0.8578} & \multicolumn{1}{|c|}{0.8722} & 
\multicolumn{1}{|c|}{0.7608} & \multicolumn{1}{|c|}{0.7229} \\ \cline{2-9}
40 & \multicolumn{1}{|c|}{0.1} & \multicolumn{1}{|c|}{0.5119} & 
\multicolumn{1}{|c|}{0.6670} & \multicolumn{1}{|c|}{0.6634} & 
\multicolumn{1}{|c|}{0.8851} & \multicolumn{1}{|c|}{0.8864} & 
\multicolumn{1}{|c|}{0.7891} & \multicolumn{1}{|c|}{0.7481} \\ \cline{2-9}
& \multicolumn{1}{|c|}{0.2} & \multicolumn{1}{|c|}{0.5301} & 
\multicolumn{1}{|c|}{0.6920} & \multicolumn{1}{|c|}{0.6876} & 
\multicolumn{1}{|c|}{0.9098} & \multicolumn{1}{|c|}{0.9150} & 
\multicolumn{1}{|c|}{0.8068} & \multicolumn{1}{|c|}{0.7691} \\ \hline
& \multicolumn{1}{|c|}{0} & \multicolumn{1}{|c|}{0.4608} & 
\multicolumn{1}{|c|}{0.5790} & \multicolumn{1}{|c|}{0.5764} & 
\multicolumn{1}{|c|}{0.7159} & \multicolumn{1}{|c|}{0.7231} & 
\multicolumn{1}{|c|}{0.6490} & \multicolumn{1}{|c|}{0.6237} \\ \cline{2-9}
48 & \multicolumn{1}{|c|}{0.1} & \multicolumn{1}{|c|}{0.4762} & 
\multicolumn{1}{|c|}{0.6038} & \multicolumn{1}{|c|}{0.6019} & 
\multicolumn{1}{|c|}{0.7554} & \multicolumn{1}{|c|}{0.7649} & 
\multicolumn{1}{|c|}{0.6822} & \multicolumn{1}{|c|}{0.6560} \\ \cline{2-9}
& \multicolumn{1}{|c|}{0.2} & \multicolumn{1}{|c|}{0.4995} & 
\multicolumn{1}{|c|}{0.6237} & \multicolumn{1}{|c|}{0.6216} & 
\multicolumn{1}{|c|}{0.7731} & \multicolumn{1}{|c|}{0.7742} & 
\multicolumn{1}{|c|}{0.7101} & \multicolumn{1}{|c|}{0.6838} \\ \hline
\end{tabular}

{\small Results based on 10,000 simulations}

\begin{tabular}{|l|l|l|l|l|l|l|l|l|}
\hline\hline
\multicolumn{9}{|c|}{Table 6: 0.05 Rejection Frequencies, DGP 2} \\ 
\hline\hline
$\mu ^{2}$ & $\mathcal{R}_{\varepsilon ^{2}|z_{1}^{2}}^{2}$ & 2SLS & IJIVE1
& IJIVE2 & UJIVE & FEJIV & FELIM & FEFUL \\ \hline
& \multicolumn{1}{|c|}{0} & \multicolumn{1}{|c|}{0.1861} & 
\multicolumn{1}{|c|}{0.1056} & \multicolumn{1}{|c|}{0.0961} & 
\multicolumn{1}{|c|}{0.2236} & \multicolumn{1}{|c|}{0.0322} & 
\multicolumn{1}{|c|}{0.0559} & \multicolumn{1}{|c|}{0.0588} \\ \cline{2-9}
24 & \multicolumn{1}{|c|}{0.1} & \multicolumn{1}{|c|}{0.1783} & 
\multicolumn{1}{|c|}{0.1015} & \multicolumn{1}{|c|}{0.0924} & 
\multicolumn{1}{|c|}{0.2243} & \multicolumn{1}{|c|}{0.0304} & 
\multicolumn{1}{|c|}{0.0627} & \multicolumn{1}{|c|}{0.0646} \\ \cline{2-9}
& \multicolumn{1}{|c|}{0.2} & \multicolumn{1}{|c|}{0.1786} & 
\multicolumn{1}{|c|}{0.1018} & \multicolumn{1}{|c|}{0.0927} & 
\multicolumn{1}{|c|}{0.2306} & \multicolumn{1}{|c|}{0.0313} & 
\multicolumn{1}{|c|}{0.0631} & \multicolumn{1}{|c|}{0.0657} \\ \hline
& \multicolumn{1}{|c|}{0} & \multicolumn{1}{|c|}{0.1716} & 
\multicolumn{1}{|c|}{0.1077} & \multicolumn{1}{|c|}{0.0962} & 
\multicolumn{1}{|c|}{0.2563} & \multicolumn{1}{|c|}{0.0313} & 
\multicolumn{1}{|c|}{0.0486} & \multicolumn{1}{|c|}{0.0515} \\ \cline{2-9}
32 & \multicolumn{1}{|c|}{0.1} & \multicolumn{1}{|c|}{0.1718} & 
\multicolumn{1}{|c|}{0.1098} & \multicolumn{1}{|c|}{0.0992} & 
\multicolumn{1}{|c|}{0.2575} & \multicolumn{1}{|c|}{0.0356} & 
\multicolumn{1}{|c|}{0.0547} & \multicolumn{1}{|c|}{0.0579} \\ \cline{2-9}
& \multicolumn{1}{|c|}{0.2} & \multicolumn{1}{|c|}{0.1739} & 
\multicolumn{1}{|c|}{0.1155} & \multicolumn{1}{|c|}{0.1039} & 
\multicolumn{1}{|c|}{0.2607} & \multicolumn{1}{|c|}{0.0380} & 
\multicolumn{1}{|c|}{0.0600} & \multicolumn{1}{|c|}{0.0627} \\ \hline
& \multicolumn{1}{|c|}{0} & \multicolumn{1}{|c|}{0.1662} & 
\multicolumn{1}{|c|}{0.1143} & \multicolumn{1}{|c|}{0.1055} & 
\multicolumn{1}{|c|}{0.2770} & \multicolumn{1}{|c|}{0.0371} & 
\multicolumn{1}{|c|}{0.0510} & \multicolumn{1}{|c|}{0.0536} \\ \cline{2-9}
40 & \multicolumn{1}{|c|}{0.1} & \multicolumn{1}{|c|}{0.1600} & 
\multicolumn{1}{|c|}{0.1133} & \multicolumn{1}{|c|}{0.1029} & 
\multicolumn{1}{|c|}{0.2767} & \multicolumn{1}{|c|}{0.0380} & 
\multicolumn{1}{|c|}{0.0558} & \multicolumn{1}{|c|}{0.0590} \\ \cline{2-9}
& \multicolumn{1}{|c|}{0.2} & \multicolumn{1}{|c|}{0.1608} & 
\multicolumn{1}{|c|}{0.1167} & \multicolumn{1}{|c|}{0.1063} & 
\multicolumn{1}{|c|}{0.2851} & \multicolumn{1}{|c|}{0.0389} & 
\multicolumn{1}{|c|}{0.0578} & \multicolumn{1}{|c|}{0.0609} \\ \hline
& \multicolumn{1}{|c|}{0} & \multicolumn{1}{|c|}{0.1551} & 
\multicolumn{1}{|c|}{0.1166} & \multicolumn{1}{|c|}{0.1039} & 
\multicolumn{1}{|c|}{0.2958} & \multicolumn{1}{|c|}{0.0348} & 
\multicolumn{1}{|c|}{0.0479} & \multicolumn{1}{|c|}{0.0508} \\ \cline{2-9}
48 & \multicolumn{1}{|c|}{0.1} & \multicolumn{1}{|c|}{0.1579} & 
\multicolumn{1}{|c|}{0.1173} & \multicolumn{1}{|c|}{0.1057} & 
\multicolumn{1}{|c|}{0.2960} & \multicolumn{1}{|c|}{0.0384} & 
\multicolumn{1}{|c|}{0.0524} & \multicolumn{1}{|c|}{0.0555} \\ \cline{2-9}
& \multicolumn{1}{|c|}{0.2} & \multicolumn{1}{|c|}{0.1614} & 
\multicolumn{1}{|c|}{0.1240} & \multicolumn{1}{|c|}{0.1121} & 
\multicolumn{1}{|c|}{0.2988} & \multicolumn{1}{|c|}{0.0440} & 
\multicolumn{1}{|c|}{0.0605} & \multicolumn{1}{|c|}{0.0616} \\ \hline
\end{tabular}

{\small Results based on 10,000 simulations}

Looking over the results reported in Tables 1-6, note first that, in terms
of median bias, the performance of FEJIV, FELIM, and FEFUL are almost
uniformly better than 2SLS, IJIVE1, and IJIVE2, although our experiments do
show the latter three to be less dispersed than the three estimators studied
in this paper. Comparing FELIM and FEFUL in terms of the nine decile range,
we see that FEFUL is less dispersed than FELIM, which is in accord with the
motivation behind the original Fuller (1977) modification. Perhaps the most
notable difference in performance is that t-statistics based on FELIM\ and
FEFUL have much less size distortion than t-statistics constructed from any
of the other five estimators. The t-statistics based on the FEJIV estimator
tend to be undersized, but the empirical rejection frequencies are still
closer to the nominal level than t-statistics based on 2SLS, IJIVE1, IJIVE2,
or UJIVE. Finally, we note that UJIVE did much better under DGP2 than under
DGP1. This is due to the fact that UJIVE does not properly partial out the
included exogenous regressors; hence, it performs less well under DGP1,
where a larger number of included exogenous regressors enter significantly
into the structural equation of interest.

\section{Conclusion}

This paper considers an IV regression model with many weak instruments,
cluster specific effects, error heteroskedasticity, and possibly many
included exogenous regressors. To carry out point estimation in this setup,
we propose three new jackknife-type IV estimators, which we refer to by the
acronyms FEJIV, FELIM, and FEFUL. All three of these estimators are shown to
be robust to the effects of many weak instruments, in the sense that they
are shown to be consistent in a framework broad enough to include both the
standard situation with strong instruments and situations with many weak
instruments. ~To the best of our knowledge, the estimators proposed in this
paper are the first to be consistent under many weak instrument asymptotics
when the IV regression under consideration has both cluster specific effects
and possibly many included exogenous regressors. We establish asymptotic
normality for FELIM and FEFUL under both strong instrument and many weak
instrument asymptotics. In addition, we provide consistent standard errors
for our estimators and show that, when the null hypothesis is true,
t-statistics based on these standard errors are asymptotically normal under
both strong instrument and many weak instrument asymptotics. Finally, we
show that under both strong instrument and many weak instrument asymptotics,
the t-statistics based on these standard errors are consistent under fixed
alternatives. Thus, we underscore an interesting aspect of the many weak
instrument setup. Namely, test consistency is still possible under this
framework, as has been pointed out in a recent paper by Mikusheva and Sun
(2020). In a series of Monte Carlo experiments, we find that t-statistics
based on FELIM and FEFUL control size better in finite samples than
t-statistics based on alternative jackknife-type IV estimators that have
previously been proposed in the literature. Hence, based on the findings of
this paper, we recommend that either FELIM or FEFUL be used in settings
where there are many weak instruments. cluster specific effects, and
possibly many included exogenous regressors.

\begin{thebibliography}{99}
\bibitem{} Ackerberg, D.A. and P.J. Devereux (2009): Improved JIVE
Estimators for Overidentified Linear Models with and without
Heteroskedasticity,\ Review of Economics and Statistics, 91, 351-362.

\bibitem{} Angrist, J.D., G.W. Imbens, and A. Krueger (1999): Jackknife
Instrumental Variables Estimation,\ Journal of Applied Econometrics, 14,
57-67.

\bibitem{} Bekker, P. A. (1994): Alternative Approximations to the
Distributions of Instrumental Variables Estimators,\ Econometrica, 62,
657-681.

\bibitem{} Bekker, P. A. and F. Crudu (2015): Jackknife Instrumental
Variable Estimation with Heteroskedasticity,\ Journal of Econometrics, 185,
332-342.

\bibitem{} Blomquist, S. and M. Dahlberg (1999): Small Sample Properties of
LIML and Jackknife IV Estimators: Experiments with Weak Instruments,\
Journal of Applied Econometrics, 14, 69-88.

\bibitem{} Cattaneo, M. D., M. Jansson, and W.K. Newey (2018a): Alternative
Asymptotics and the Partially Linear Model with Many Regressors,\
Econometric Theory, 34, 277-301.

\bibitem{} Cattaneo, M. D., M. Jansson, and W.K. Newey (2018b): Inference \
in Linear Regression Models with Many Covariates and Heteroskedasticity,
Journal of the American Statistical\textit{\ }Association, 113, 1350-1361.

\bibitem{} Chao, J. C. and N. R. Swanson (2004): Estimation and Testing
Using Jackknife IV in Heteroskedastic Regressions with Many Weak
Instruments, Unpublished Manuscript, University of Maryland and Rutgers
University.

\bibitem{} Chao, J. C. and N. R. Swanson (2005): Consistent \ Estimation
with a Large Number of Weak Instruments,\ Econometrica, 73, 1673-1692.

\bibitem{} Chao, J.C., \ N. R. Swanson, J. A. Hausman, W. K. Newey, and T.
Woutersen (2012): Asymptotic Distribution of JIVE in a Heteroskedastic IV
Regression with Many Instruments,\ Econometric Theory, 28, 42-86.

\bibitem{} Crudu, F., G. Mellace. and Z. S\'{a}ndor (2020): Inference in
Instrumental Variables Models with Heteroskedasticity and Many Instruments,\
Econometric Theory, forthcoming.

\bibitem{} Davidson, R. and J. G. MacKinnon (2006): The Case Against JIVE,
Journal of Applied Econometrics, 21, 827--833.

\bibitem{} Evdokimov, K. S. and M. Koles\'{a}r (2018): Inference in
Instrumental Variables Analysis with Heterogeneous Treatment Effects,
Unpublished Manuscript,\ Princeton University.

\bibitem{} Fuller, W. A. (1977): Some Properties of a Modification of the
Limited Information Estimator, Econometrica, 45, 939-954.

\bibitem{} Hansen, C., J. A. Hausman, and W. K. Newey (2008): Estimation
with Many Instrumental Variables,\ Journal of Business \& Economic
Statistics, 26, 398-422.

\bibitem{} Hastie, T. and R. Tibshirani (1990): Generalized Additive Models.
London: Chapman and Hall.

\bibitem{} Hausman, J. A., W. K. Newey, T. Woutersen, J. C. Chao, and N. R.
Swanson (2012): Instrumental Variable Estimation with Heteroskedasticity and
Many Instruments, Quantitative\textit{\ }Economics, 3, 211-255.

\bibitem{} Hsiao, C. and Q. Zhou (2018): JIVE for Panel Dynamic Simultaneous
Equations Models,\ Econometric Theory, 34, 1325-1369.

\bibitem{} Koles\'{a}r, M. (2013): Estimation in Instrumental Variables
Models with Treatment Effect Heterogeneity, Unpublished Manuscript,\
Princeton University.

\bibitem{} Mikusheva, A. and L. Sun (2020): Inference with Many Weak
Instruments, Unpublished Manuscript,\ MIT.

\bibitem{} Morimune, K. (1983): Approximate Distributions of k-Class
Estimators When the Degree of Overidentifiability Is Large Compared with the
Sample Size,\ Econometrica, 51, 821-841.

\bibitem{} Newey, W. K. (1997): Convergence Rates and Asymptotic Normality
for Series Estimators, Journal of Econometrics, 79, 147-168.

\bibitem{} Phillips, G. D. A. and C. Hale (1977): The Bias of Instrumental
Variable Estimators of Simultaneous Equation Systems, International Economic
Review, 18, 219-228.

\bibitem{} Phillips, P. C. B. (1983). Exact Small Sample Theory in the
Simultaneous Equations Model, in Handbook of Econometrics, Vol. I, ed. by Z.
Griliches and M.D. Intriligator. Amsterdam: North-Holland, 449-516.

\bibitem{} Rothenberg, T. (1984): Approximating the Distributions of
Econometric Estimators and Test Statistics,\ in Handbook of Econometrics,
Vol. II, ed. by Z. Griliches and M.D. Intriligator. Amsterdam:
North-Holland, 881-935.

\bibitem{} Staiger, D. and J. H. Stock (1997): Instrumental Variables
Regression with Weak Instruments, Econometrica, 65, 557-586.

\bibitem{} Stock, J. \ H. and M. Yogo (2005): Asymptotic Distributions of
Instrumental Variables Statistics with Many Instruments,\ in Identification
and Inference for Econometric Models: Essays in\textbf{\ }Honor of Thomas
Rothenberg, ed. by D.W.K. Andrews and J. H. Stock, Chapter 6. Cambridge:
Cambridge University Press.
\end{thebibliography}

\section{Appendix: Proofs of Main Theorems and Other Key Results}

\noindent This appendix provides the proofs for Theorem 1, Corollary 1, and
Theorems 4-6 of the paper. The proofs of Theorems 2 and 3 are longer and,
thus, are given in Appendix S1 of a Supplemental Appendix to this paper. In
addition, the proofs provided below rely on a number of technical results
that are established in Appendix S2 of the Supplemental Appendix. These
results are designated in the derivations that follow by the use of the
prefix S. So, for example, Lemma S2-2 will refer to the second lemma in
Appendix S2 of the Supplemental Appendix.\vspace{0.08in}

\noindent \textbf{Proof of Theorem 1:}

To proceed, note first that, by parts (a) and (b) of Lemma S2-2 and by the
assumption on $\overline{\ell }_{n}$, we have $D_{\mu }^{-1}X^{\prime }\left[
A-\overline{\ell }_{n}M^{\left( Z_{1},Q\right) }\right] XD_{\mu
}^{-1}=D_{\mu }^{-1}X^{\prime }AXD_{\mu }^{-1}-\overline{\ell }_{n}D_{\mu
}^{-1}X^{\prime }M^{\left( Z_{1},Q\right) }XD_{\mu }^{-1}=H_{n}+o_{p}\left(
1\right) $, where $H_{n}=\Gamma ^{\prime }M^{\left( Z_{1},Q\right) }\Gamma
/n=O_{p}\left( 1\right) $. By Assumption 3(iii), we also have that $H_{n}$
is positive definite almost surely for $n$ sufficiently large, so that $%
D_{\mu }^{-1}X^{\prime }\left[ A-\overline{\ell }_{n}M^{\left(
Z_{1},Q\right) }\right] XD_{\mu }^{-1}$ is invertible w.p.a.1. Hence,
w.p.a.1., we can write%
\begin{eqnarray*}
\frac{1}{\mu _{n}^{\min }}D_{\mu }\left( \overline{\delta }_{n}-\delta
_{0}\right) &=&\left( D_{\mu }^{-1}X^{\prime }\left[ A-\overline{\ell }%
_{n}M^{\left( Z_{1},Q\right) }\right] XD_{\mu }^{-1}\right) ^{-1}\frac{1}{%
\mu _{n}^{\min }}D_{\mu }^{-1}X^{\prime }\left[ A-\overline{\ell }%
_{n}M^{\left( Z_{1},Q\right) }\right] \varphi _{n} \\
&&+\left( D_{\mu }^{-1}X^{\prime }\left[ A-\overline{\ell }_{n}M^{\left(
Z_{1},Q\right) }\right] XD_{\mu }^{-1}\right) ^{-1}\frac{1}{\mu _{n}^{\min }}%
D_{\mu }^{-1}X^{\prime }\left[ A-\overline{\ell }_{n}M^{\left(
Z_{1},Q\right) }\right] \varepsilon .
\end{eqnarray*}%
Moreover, by applying part (a) of Lemma S2-4 and part (a) of Lemma S2-5, we
obtain%
\begin{eqnarray*}
\frac{1}{\mu _{n}^{\min }}D_{\mu }^{-1}X^{\prime }\left[ A-\overline{\ell }%
_{n}M^{\left( Z_{1},Q\right) }\right] \varphi _{n} &=&\frac{1}{\mu
_{n}^{\min }}D_{\mu }^{-1}X^{\prime }A\varphi _{n}-\overline{\ell }_{n}\frac{%
1}{\mu _{n}^{\min }}D_{\mu }^{-1}X^{\prime }M^{\left( Z_{1},Q\right)
}\varphi _{n} \\
&=&O_{p}\left( \frac{\tau _{n}}{\left[ \mu _{n}^{\min }\right]
K_{1,n}^{\varrho _{g}}}\right) +o_{p}\left( \frac{\left[ \mu _{n}^{\min }%
\right] \tau _{n}}{nK_{1,n}^{\varrho _{g}}}\right) =o_{p}\left( 1\right) .
\end{eqnarray*}%
Applying part (b) of Lemma S2-4 and part (b) of Lemma S2-5, we get%
\begin{eqnarray*}
\frac{1}{\mu _{n}^{\min }}D_{\mu }^{-1}X^{\prime }\left[ A-\overline{\ell }%
_{n}M^{\left( Z_{1},Q\right) }\right] \varepsilon &=&\frac{1}{\mu _{n}^{\min
}}D_{\mu }^{-1}X^{\prime }A\varepsilon -\overline{\ell }_{n}\frac{1}{\mu
_{n}^{\min }}D_{\mu }^{-1}X^{\prime }M^{\left( Z_{1},Q\right) }\varepsilon \\
&=&O_{p}\left( \max \left\{ \frac{1}{\mu _{n}^{\min }},\frac{\sqrt{K_{2,n}}}{%
\left( \mu _{n}^{\min }\right) ^{2}}\right\} \right) +o_{p}\left( 1\right)
=o_{p}\left( 1\right) .
\end{eqnarray*}%
It follows by the triangle inequality and the Slutsky's Theorem that $%
\left\Vert D_{\mu }\left( \overline{\delta }_{n}-\delta _{0}\right) /\left(
\mu _{n}^{\min }\right) \right\Vert _{2}=o_{p}\left( 1\right) $, which gives
the first result. To show the second result, note that, by straightforward
calculations, we obtain $\left\Vert D_{\mu }\left( \overline{\delta }%
_{n}-\delta _{0}\right) /\left( \mu _{n}^{\min }\right) \right\Vert _{2}\geq 
\sqrt{\left( \mu _{n}^{\min }\right) ^{2}/\left( \mu _{n}^{\min }\right) ^{2}%
}\sqrt{\left( \overline{\delta }_{n}-\delta _{0}\right) ^{\prime }\left( 
\overline{\delta }_{n}-\delta _{0}\right) }=\left\Vert \overline{\delta }%
_{n}-\delta _{0}\right\Vert _{2}$, which implies that $\left\Vert \overline{%
\delta }_{n}-\delta _{0}\right\Vert _{2}\overset{p}{\rightarrow }0$, as
required. $\square $

\bigskip

\noindent \textbf{Proof of Corollary 1:}

In light of the results given in Theorem 1, it suffices that we verify the
condition $\overline{\ell }_{n}=o_{p}\left( \left[ \mu _{n}^{\min }\right]
^{2}/n\right) =o_{p}\left( 1\right) $ for all three estimators. For the
FEJIV estimator considered in part (a), $\overline{\ell }_{n}=0$ for all $n$%
, so this condition is trivially satisfied. Now, part (b) considers the
FELIM estimator. For this estimator, the result of Lemma S2-11 has shown
that we can take $\overline{\ell }_{n}=\widehat{\ell }_{L,n}=\min_{\beta \in 
\overline{B}}\left( \beta ^{\prime }\overline{X}^{\prime }A\overline{X}\beta
\right) /\left( \beta ^{\prime }\overline{X}^{\prime }M^{\left(
Z_{1},Q\right) }\overline{X}\beta \right) $

\noindent $=\left( y-X\widehat{\delta }_{L}\right) ^{\prime }A\left( y-X%
\widehat{\delta }_{L}\right) /\left[ \left( y-X\widehat{\delta }_{L}\right)
^{\prime }M^{\left( Z_{1},Q\right) }\left( y-X\widehat{\delta }_{L}\right) %
\right] $. By part (a)\ of Lemma S2-7, we then have $\widehat{\ell }%
_{L,n}=o_{p}\left( \left[ \mu _{n}^{\min }\right] ^{2}/n\right) $, so FELIM
also satisfies the needed condition. Finally, part (c) considers the FEFUL
estimator, which takes $\overline{\ell }_{n}=\widehat{\ell }_{F,n}$

\noindent $=\left[ \widehat{\ell }_{L,n}-\left( 1-\widehat{\ell }%
_{L,n}\right) \left( C/m_{n}\right) \right] /\left[ 1-\left( 1-\widehat{\ell 
}_{L,n}\right) \left( C/m_{n}\right) \right] $. By part (b) of Lemma S2-7,
we have that $\widehat{\ell }_{F,n}=o_{p}\left( \left[ \mu _{n}^{\min }%
\right] ^{2}/n\right) $, so the needed condition is satisfied again. The
consistency results given in parts (a)-(c) of this corollary then follow as
a consequence of Theorem 1. $\square $

\medskip

\noindent \textbf{Proof of Theorem 4:}

We shall prove this theorem for the FELIM case since the proof for FEFUL is
similar. To proceed, first define $S_{L,1}=X^{\prime }AD\left( J\left[ 
\widehat{\varepsilon }_{L}\circ \widehat{\varepsilon }_{L}\right] \right) AX$%
, $S_{L,2}=\left( \widehat{\varepsilon }_{L}\circ \widehat{\varepsilon }%
_{L}\right) ^{\prime }J\left( A\circ A\right) J\left( \widehat{\varepsilon }%
_{L}\iota _{d}^{\prime }\circ M^{\left( Z,Q\right) }X\right) $, $%
S_{L,3}=\left( \widehat{\varepsilon }_{L}\circ \widehat{\varepsilon }%
_{L}\right) ^{\prime }J\left( A\circ A\right) J\left( \widehat{\varepsilon }%
_{L}\circ \widehat{\varepsilon }_{L}\right) $, $\underline{S}_{L,4}=\left( 
\widehat{\varepsilon }_{L}\iota _{d}^{\prime }\circ \widehat{\underline{U}}%
_{L}\right) ^{\prime }J\left( A\circ A\right) J\left( \widehat{\varepsilon }%
_{L}\iota _{d}^{\prime }\circ \widehat{\underline{U}}_{L}\right) $, $%
\widehat{H}_{L}=X^{\prime }\left[ A-\widehat{\ell }_{L,n}M^{\left(
Z_{1},Q\right) }\right] X$, $\Sigma _{1,n}=\Gamma ^{\prime }M^{\left(
Z_{1},Q\right) }D_{\sigma ^{2}}M^{\left( Z_{1},Q\right) }\Gamma /n$. In
addition, also define $\sigma _{\left( i,t\right) }^{2}=E\left[ \varepsilon
_{\left( i,t\right) }^{2}|\mathcal{F}_{n}^{W}\right] $, $\phi _{\left(
i,t\right) }=E\left[ U_{\left( i,t\right) }\varepsilon _{\left( i,t\right) }|%
\mathcal{F}_{n}^{W}\right] $, $\Psi _{\left( i,t\right) }=E\left[ U_{\left(
i,t\right) }U_{\left( i,t\right) }^{\prime }|\mathcal{F}_{n}^{W}\right] $, $%
\underline{\phi }_{\left( i,t\right) }=E\left[ \underline{U}_{\left(
i,t\right) }\varepsilon _{\left( i,t\right) }|\mathcal{F}_{n}^{W}\right] $,
and $\underline{\Psi }_{\left( i,t\right) }=E\left[ \underline{U}_{\left(
i,t\right) }\underline{U}_{\left( i,t\right) }^{\prime }|\mathcal{F}_{n}^{W}%
\right] $ where $\underline{U}_{\left( i,t\right) }=U_{\left( i,t\right)
}-\rho \varepsilon _{\left( i,t\right) }$ and where for notational
convenience we suppress the dependence of $\sigma _{\left( i,t\right) }^{2}$%
, $\phi _{\left( i,t\right) }$, $\Psi _{\left( i,t\right) }$, $\underline{%
\phi }_{\left( i,t\right) }$, and $\underline{\Psi }_{\left( i,t\right) }$
on $\mathcal{F}_{n}^{W}=\sigma \left( W_{n}\right) $.

Using these notations, to show part (a), we first write $D_{\mu }\widehat{V}%
_{L}D_{\mu }=\widehat{V}_{L,1}+\widehat{V}_{L,2}+\widehat{V}_{L,3}+\widehat{V%
}_{L,4}$, where $\widehat{V}_{L,1}=\left( D_{\mu }^{-1}\widehat{H}_{L}D_{\mu
}^{-1}\right) ^{-1}D_{\mu }^{-1}S_{L,1}D_{\mu }^{-1}\left( D_{\mu }^{-1}%
\widehat{H}_{L}D_{\mu }^{-1}\right) ^{-1}$,

\noindent $\widehat{V}_{L,2}=-\left( D_{\mu }^{-1}\widehat{H}_{L}D_{\mu
}^{-1}\right) ^{-1}D_{\mu }^{-1}\left( \widehat{\rho }_{L}S_{L,2}+S_{L,2}^{%
\prime }\widehat{\rho }_{L}^{\prime }\right) D_{\mu }^{-1}\left( D_{\mu
}^{-1}\widehat{H}_{L}D_{\mu }^{-1}\right) ^{-1}$,

\noindent $\widehat{V}_{L,3}=\left( D_{\mu }^{-1}\widehat{H}_{L}D_{\mu
}^{-1}\right) ^{-1}D_{\mu }^{-1}\widehat{\rho }_{L}S_{L,3}\widehat{\rho }%
_{L}^{\prime }D_{\mu }^{-1}\left( D_{\mu }^{-1}\widehat{H}_{L}D_{\mu
}^{-1}\right) ^{-1}$, and $\widehat{V}_{L,4}=\left( D_{\mu }^{-1}\widehat{H}%
_{L}D_{\mu }^{-1}\right) ^{-1}$

\noindent $\times D_{\mu }^{-1}\underline{S}_{L,4}D_{\mu }^{-1}\left( D_{\mu
}^{-1}\widehat{H}_{L}D_{\mu }^{-1}\right) ^{-1}$. Now, consider $\widehat{V}%
_{L,1}$ first. Note that, by Lemma S2-17, 
\begin{equation*}
D_{\mu }^{-1}X^{\prime }AD\left( \varepsilon \circ \varepsilon \right)
AXD_{\mu }^{-1}=\Sigma _{1,n}+\dsum\limits_{\substack{ \left( i,t\right)
,\left( j,s\right) =1  \\ \left( i,t\right) \neq \left( j,s\right) }}%
^{m_{n}}A_{\left( i,t\right) ,\left( j,s\right) }^{2}\sigma _{\left(
i,t\right) }^{2}D_{\mu }^{-1}\Psi _{\left( j,s\right) }D_{\mu
}^{-1}+o_{p}\left( 1\right) \text{,}
\end{equation*}%
from which we deduce that $D_{\mu }^{-1}X^{\prime }AD\left( \varepsilon
\circ \varepsilon \right) AXD_{\mu }^{-1}=O_{p}\left( 1\right) $ using
Assumptions 2(i) and 3(iii), Lemma S2-1 part (a), and the assumption that $%
K_{2,n}/\left( \mu _{n}^{\min }\right) ^{2}=O\left( 1\right) $ under Case I.

Next, note that by Lemma S2-11,

\noindent $\widehat{\ell }_{L}=\left( y-X\widehat{\delta }_{L}\right)
^{\prime }A\left( y-X\widehat{\delta }_{L}\right) /\left( y-X\widehat{\delta 
}_{L}\right) ^{\prime }M^{\left( Z_{1},Q\right) }\left( y-X\widehat{\delta }%
_{L}\right) $. Moreover, by the result given in Lemma S2-10, we have that $%
D_{\mu }^{-1}\widehat{H}_{L}D_{\mu }^{-1}=H_{n}+o_{p}\left( 1\right) $,
where, by Assumption 3(iii), $H_{n}=\Gamma ^{\prime }M^{\left(
Z_{1},Q\right) }\Gamma /n$ is positive definite $a.s.n.$ In addition, we can
apply part (a) of Lemma S2-18 and Slutsky's theorem to deduce that%
\begin{eqnarray}
\widehat{V}_{L,1} &=&\left( D_{\mu }^{-1}\widehat{H}_{L}D_{\mu }^{-1}\right)
^{-1}D_{\mu }^{-1}S_{L,1}D_{\mu }^{-1}\left( D_{\mu }^{-1}\widehat{H}%
_{L}D_{\mu }^{-1}\right) ^{-1}  \notag \\
&=&H_{n}^{-1}\Sigma _{1,n}H_{n}^{-1}+H_{n}^{-1}\left( \dsum\limits 
_{\substack{ \left( i,t\right) ,\left( j,s\right) =1  \\ \left( i,t\right)
\neq \left( j,s\right) }}^{m_{n}}A_{\left( i,t\right) ,\left( j,s\right)
}^{2}\sigma _{\left( i,t\right) }^{2}D_{\mu }^{-1}\Psi _{\left( j,s\right)
}D_{\mu }^{-1}\right) H_{n}^{-1}+o_{p}\left( 1\right) \text{.}
\label{VhatL 1}
\end{eqnarray}%
Next, consider $\widehat{V}_{L,2}$. Here, note that we can further decompose 
$\widehat{V}_{L,2}$ as $\widehat{V}_{L,2}=\widehat{V}_{L,2,1}+\widehat{V}%
_{L,2,2}$, where $\widehat{V}_{L,2,1}=-\left( D_{\mu }^{-1}\widehat{H}%
_{L}D_{\mu }^{-1}\right) ^{-1}D_{\mu }^{-1}\widehat{\rho }_{L}S_{L,2}D_{\mu
}^{-1}\left( D_{\mu }^{-1}\widehat{H}_{L}D_{\mu }^{-1}\right) ^{-1}$ and

\noindent $\widehat{V}_{L,2,2}=-\left( D_{\mu }^{-1}\widehat{H}_{L}D_{\mu
}^{-1}\right) ^{-1}D_{\mu }^{-1}S_{L,2}^{\prime }\widehat{\rho }_{L}^{\prime
}D_{\mu }^{-1}\left( D_{\mu }^{-1}\widehat{H}_{L}D_{\mu }^{-1}\right) ^{-1}$%
. Noting that $K_{2,n}/\left( \mu _{n}^{\min }\right) ^{2}=O\left( 1\right) $
under Case I and applying the result of Lemma S2-10, as well as parts (d)
and (e) of Lemma S2-18 and Slutsky's theorem, we get%
\begin{eqnarray*}
\widehat{V}_{L,2,1} &=&-H_{n}^{-1}\frac{K_{2,n}}{\left( \mu _{n}^{\min
}\right) }\left\{ D_{\mu }^{-1}\rho +D_{\mu }^{-1}\left( \widehat{\rho }%
_{L}-\rho \right) \right\} \frac{\mu _{n}^{\min }}{K_{2,n}}S_{L,2}D_{\mu
}^{-1}H_{n}^{-1}\left( 1+o_{p}\left( 1\right) \right) \\
&=&-H_{n}^{-1}D_{\mu }^{-1}\rho \dsum\limits_{\substack{ \left( i,t\right)
,\left( j,s\right) =1  \\ \left( i,t\right) \neq \left( j,s\right) }}%
^{m_{n}}A_{\left( i,t\right) ,\left( j,s\right) }^{2}\sigma _{\left(
i,t\right) }^{2}\phi _{\left( j,s\right) }^{\prime }D_{\mu
}^{-1}H_{n}^{-1}+o_{p}\left( 1\right) .
\end{eqnarray*}%
Moreover, since $\widehat{V}_{L,2,2}=\widehat{V}_{L,2,1}^{\prime }$, we also
have

\noindent $\widehat{V}_{L,2,2}=-H_{n}^{-1}\dsum\nolimits_{\left( i,t\right)
,\left( j,s\right) =1:m_{n},\left( i,t\right) \neq \left( j,s\right)
}A_{\left( i,t\right) ,\left( j,s\right) }^{2}\sigma _{\left( i,t\right)
}^{2}D_{\mu }^{-1}\phi _{\left( j,s\right) }\rho ^{\prime }D_{\mu
}^{-1}H_{n}^{-1}+o_{p}\left( 1\right) $. Given that $\widehat{V}_{L,2}=%
\widehat{V}_{L,2,1}+\widehat{V}_{L,2,2}$, it follows from these calculations
that%
\begin{equation}
\widehat{V}_{L,2}=-H_{n}^{-1}\dsum\limits_{\substack{ \left( i,t\right)
,\left( j,s\right) =1  \\ \left( i,t\right) \neq \left( j,s\right) }}%
^{m_{n}}A_{\left( i,t\right) ,\left( j,s\right) }^{2}D_{\mu }^{-1}\left(
\rho \sigma _{\left( i,t\right) }^{2}\phi _{\left( j,s\right) }^{\prime
}+\sigma _{\left( i,t\right) }^{2}\phi _{\left( j,s\right) }\rho ^{\prime
}\right) D_{\mu }^{-1}H_{n}^{-1}+o_{p}\left( 1\right)  \label{VhatL 2}
\end{equation}%
Turning our attention to $\widehat{V}_{L,3}$, note that, in this case, we
can apply Lemma S2-10, parts (b) and (e) of Lemma S2-18, and Slutsky's
theorem to obtain%
\begin{eqnarray}
\widehat{V}_{L,3} &=&K_{2,n}H_{n}^{-1}\left[ D_{\mu }^{-1}\rho +D_{\mu
}^{-1}\left( \widehat{\rho }_{L}-\rho \right) \right] \frac{S_{L,3}}{K_{2,n}}%
\rho ^{\prime }D_{\mu }^{-1}H_{n}^{-1}\left( 1+o_{p}\left( 1\right) \right) 
\notag \\
&&+K_{2,n}H_{n}^{-1}\left[ D_{\mu }^{-1}\rho +D_{\mu }^{-1}\left( \widehat{%
\rho }_{L}-\rho \right) \right] \frac{S_{L,3}}{K_{2,n}}\left( \widehat{\rho }%
_{L}-\rho \right) ^{\prime }D_{\mu }^{-1}H_{n}^{-1}\left( 1+o_{p}\left(
1\right) \right)  \notag \\
&=&H_{n}^{-1}D_{\mu }^{-1}\rho \dsum\limits_{\substack{ \left( i,t\right)
,\left( j,s\right) =1  \\ \left( i,t\right) \neq \left( j,s\right) }}%
^{m_{n}}A_{\left( i,t\right) ,\left( j,s\right) }^{2}\sigma _{\left(
i,t\right) }^{2}\sigma _{\left( j,s\right) }^{2}\rho ^{\prime }D_{\mu
}^{-1}H_{n}^{-1}+o_{p}\left( 1\right) \text{.}  \label{Vhatl 3}
\end{eqnarray}%
Lastly, we consider $\widehat{V}_{L,4}$. Here, we can apply Lemma S2-10,
part (f) of Lemma S2-18, the fact that $K_{2,n}/\left( \mu _{n}^{\min
}\right) ^{2}=O\left( 1\right) $ under Case I, as well as Slutsky's theorem
to obtain 
\begin{eqnarray}
\widehat{V}_{L,4} &=&H_{n}^{-1}D_{\mu }^{-1}\underline{S}_{L,4}D_{\mu
}^{-1}H_{n}^{-1}\left( 1+o_{p}\left( 1\right) \right)  \notag \\
&=&H_{n}^{-1}\dsum\limits_{\substack{ \left( i,t\right) ,\left( j,s\right)
=1  \\ \left( i,t\right) \neq \left( j,s\right) }}^{m_{n}}A_{\left(
i,t\right) ,\left( j,s\right) }^{2}D_{\mu }^{-1}\underline{\phi }_{\left(
i,t\right) }\underline{\phi }_{\left( j,s\right) }^{\prime }D_{\mu
}^{-1}H_{n}^{-1}+o_{p}\left( 1\right) .  \label{Vhatl 4}
\end{eqnarray}%
It follows from equations (\ref{VhatL 1}), (\ref{VhatL 2}), (\ref{Vhatl 3}),
and (\ref{Vhatl 4}) that%
\begin{eqnarray*}
D_{\mu }\widehat{V}_{L}D_{\mu } &=&H_{n}^{-1}\Sigma
_{1,n}H_{n}^{-1}+H_{n}^{-1}\dsum\limits_{\substack{ \left( i,t\right)
,\left( j,s\right) =1  \\ \left( i,t\right) \neq \left( j,s\right) }}%
^{m_{n}}A_{\left( i,t\right) ,\left( j,s\right) }^{2}\sigma _{\left(
i,t\right) }^{2}D_{\mu }^{-1}\underline{\Psi }_{\left( j,s\right) }D_{\mu
}^{-1}H_{n}^{-1} \\
&&+H_{n}^{-1}\dsum\limits_{\substack{ \left( i,t\right) ,\left( j,s\right)
=1  \\ \left( i,t\right) \neq \left( j,s\right) }}^{m_{n}}A_{\left(
i,t\right) ,\left( j,s\right) }^{2}D_{\mu }^{-1}E\underline{\phi }_{\left(
i,t\right) }\underline{\phi }_{\left( j,s\right) }^{\prime }D_{\mu
}^{-1}H_{n}^{-1}+o_{p}\left( 1\right) \\
&=&H_{n}^{-1}\left( \Sigma _{1,n}+\Sigma _{2,n}\right)
H_{n}^{-1}+o_{p}\left( 1\right) =\Lambda _{I,n}+o_{p}\left( 1\right) \text{.}
\end{eqnarray*}

To show the same result for FEFUL, note that $\widehat{\delta }_{F}$
satisfies the conditions of both Lemma S2-12 and Lemma S2-18. Hence, we can
make the same argument as given above for FELIM, except that we use the
result of Lemma S2-12 in lieu of Lemma S2-10 to obtain $D_{\mu }\widehat{V}%
_{F}D_{\mu }=H_{n}^{-1}\left( \Sigma _{1,n}+\Sigma _{2,n}\right)
H_{n}^{-1}+o_{p}\left( 1\right) =\Lambda _{I,n}+o_{p}\left( 1\right) $.

To show part (b), we again only provide an explicit argument for $\widehat{V}%
_{L}$ since the proof of $\widehat{V}_{F}$ follows in a similar way. To
proceed, write $\left[ \left( \mu _{n}^{\min }\right) ^{2}/K_{2,n}\right]
D_{\mu }\widehat{V}_{L}D_{\mu }=\left[ \left( \mu _{n}^{\min }\right)
^{2}/K_{2,n}\right] \dsum\nolimits_{\ell =1}^{4}\widehat{V}_{L,\ell }$,
where $\widehat{V}_{L,1}$, $\widehat{V}_{L,2}$, $\widehat{V}_{L,3}$, and $%
\widehat{V}_{L,4}$ are as defined in the proof of part (a).

Considering $\widehat{V}_{L,1}$ first, note that, since $K_{2,n}/\left( \mu
_{n}^{\min }\right) ^{2}\rightarrow \infty $ but $\sqrt{K_{2,n}}/\left( \mu
_{n}^{\min }\right) ^{2}\rightarrow 0$ under Case II, we have, upon applying
the result of Lemma S2-10, part (a) of Lemma S2-18, and Slutsky's theorem,%
\begin{eqnarray}
\frac{\left( \mu _{n}^{\min }\right) ^{2}}{K_{2,n}}\widehat{V}_{L,1}
&=&H_{n}^{-1}\frac{\left( \mu _{n}^{\min }\right) ^{2}}{K_{2,n}}\left[
\Sigma _{1,n}+\dsum\limits_{\substack{ \left( i,t\right) ,\left( j,s\right)
=1  \\ \left( i,t\right) \neq \left( j,s\right) }}^{m_{n}}A_{\left(
i,t\right) ,\left( j,s\right) }^{2}\sigma _{\left( i,t\right) }^{2}D_{\mu
}^{-1}\Psi _{\left( j,s\right) }D_{\mu }^{-1}\right] H_{n}^{-1}\left(
1+o_{p}\left( 1\right) \right)  \notag \\
&=&H_{n}^{-1}\frac{\left( \mu _{n}^{\min }\right) ^{2}}{K_{2,n}}\dsum\limits 
_{\substack{ \left( i,t\right) ,\left( j,s\right) =1  \\ \left( i,t\right)
\neq \left( j,s\right) }}^{m_{n}}A_{\left( i,t\right) ,\left( j,s\right)
}^{2}\sigma _{\left( i,t\right) }^{2}D_{\mu }^{-1}\Psi _{\left( j,s\right)
}D_{\mu }^{-1}H_{n}^{-1}+o_{p}\left( 1\right) .  \label{VLhat 1}
\end{eqnarray}%
Now, consider $\widehat{V}_{L,2}$. Here, we write $\left[ \left( \mu
_{n}^{\min }\right) ^{2}/K_{2,n}\right] \widehat{V}_{L,2}=\left[ \left( \mu
_{n}^{\min }\right) ^{2}/K_{2,n}\right] \widehat{V}_{L,2,1}$

\noindent $+\left[ \left( \mu _{n}^{\min }\right) ^{2}/K_{2,n}\right] 
\widehat{V}_{L,2,2}$, where $\widehat{V}_{L,2,1}$ and $\widehat{V}_{L,2,2}$
are again as defined in the proof of part (a). Making use of the results of
Lemma S2-10, parts (d) and (e) of Lemma S2-18, and Slutsky's theorem while
noting that $K_{2,n}/\left( \mu _{n}^{\min }\right) ^{2}\rightarrow \infty $
under Case II, we get%
\begin{eqnarray*}
\frac{\left( \mu _{n}^{\min }\right) ^{2}}{K_{2,n}}\widehat{V}_{L,2,1}
&=&-H_{n}^{-1}\left( \mu _{n}^{\min }\right) \left\{ D_{\mu }^{-1}\rho
+D_{\mu }^{-1}\left( \widehat{\rho }_{L}-\rho \right) \right\} \frac{\mu
_{n}^{\min }}{K_{2,n}}S_{L,2}D_{\mu }^{-1}H_{n}^{-1}\left( 1+o_{p}\left(
1\right) \right) \\
&=&-H_{n}^{-1}\frac{\left( \mu _{n}^{\min }\right) ^{2}}{K_{2,n}}%
\dsum\limits _{\substack{ \left( i,t\right) ,\left( j,s\right) =1  \\ \left(
i,t\right) \neq \left( j,s\right) }}^{m_{n}}A_{\left( i,t\right) ,\left(
j,s\right) }^{2}D_{\mu }^{-1}\rho \sigma _{\left( i,t\right) }^{2}\phi
_{\left( j,s\right) }^{\prime }D_{\mu }^{-1}H_{n}^{-1}+o_{p}\left( 1\right)
\end{eqnarray*}%
Moreover, since $\widehat{V}_{L,2,2}=\widehat{V}_{L,2,1}^{\prime }$, we also
have

\noindent $\left[ \left( \mu _{n}^{\min }\right) ^{2}K_{2,n}^{-1}\right] 
\widehat{V}_{L,2,2}=-H_{n}^{-1}\left[ \left( \mu _{n}^{\min }\right)
^{2}K_{2,n}^{-1}\right] \dsum\nolimits_{\substack{ \left( i,t\right) ,\left(
j,s\right) =1  \\ \left( i,t\right) \neq \left( j,s\right) }}%
^{m_{n}}A_{\left( i,t\right) ,\left( j,s\right) }^{2}D_{\mu }^{-1}\phi
_{\left( j,s\right) }\sigma _{\left( i,t\right) }^{2}\rho ^{\prime }D_{\mu
}^{-1}H_{n}^{-1}+o_{p}\left( 1\right) $. It follows from these calculations
that%
\begin{equation}
\frac{\left( \mu _{n}^{\min }\right) ^{2}\widehat{V}_{L,2}}{K_{2,n}}%
=-H_{n}^{-1}\dsum\limits_{\substack{ \left( i,t\right) ,\left( j,s\right) =1 
\\ \left( i,t\right) \neq \left( j,s\right) }}^{m_{n}}\frac{\left( \mu
_{n}^{\min }\right) ^{2}A_{\left( i,t\right) ,\left( j,s\right) }^{2}}{%
K_{2,n}}D_{\mu }^{-1}\left( \rho \sigma _{\left( i,t\right) }^{2}\phi
_{\left( j,s\right) }^{\prime }+\phi _{\left( j,s\right) }\sigma _{\left(
i,t\right) }^{2}\rho ^{\prime }\right) D_{\mu }^{-1}H_{n}^{-1}+o_{p}\left(
1\right) .  \label{VLhat 2}
\end{equation}%
Next, consider $\widehat{V}_{L,3}$. Given that $K_{2,n}/\left( \mu
_{n}^{\min }\right) ^{2}\rightarrow \infty $ under Case II, we get, upon
applying the result given in Lemma S2-10, as well as parts (b) and (e) of
Lemma S2-18 and Slutsky's theorem,%
\begin{eqnarray}
\frac{\left( \mu _{n}^{\min }\right) ^{2}}{K_{2,n}}\widehat{V}_{L,3}
&=&\left( \mu _{n}^{\min }\right) ^{2}H_{n}^{-1}\left[ D_{\mu }^{-1}\rho
+D_{\mu }^{-1}\left( \widehat{\rho }_{L}-\rho \right) \right] \frac{S_{L,3}}{%
K_{2,n}}\rho ^{\prime }D_{\mu }^{-1}H_{n}^{-1}\left( 1+o_{p}\left( 1\right)
\right)  \notag \\
&&+\left( \mu _{n}^{\min }\right) ^{2}H_{n}^{-1}\left[ D_{\mu }^{-1}\rho
+D_{\mu }^{-1}\left( \widehat{\rho }_{L}-\rho \right) \right] \frac{S_{L,3}}{%
K_{2,n}}\left( \widehat{\rho }_{L}-\rho \right) ^{\prime }D_{\mu
}^{-1}H_{n}^{-1}\left( 1+o_{p}\left( 1\right) \right)  \notag \\
&=&H_{n}^{-1}\dsum\limits_{\substack{ \left( i,t\right) ,\left( j,s\right)
=1  \\ \left( i,t\right) \neq \left( j,s\right) }}^{m_{n}}\frac{\left( \mu
_{n}^{\min }\right) ^{2}A_{\left( i,t\right) ,\left( j,s\right) }^{2}}{%
K_{2,n}}D_{\mu }^{-1}\rho \sigma _{\left( i,t\right) }^{2}\sigma _{\left(
j,s\right) }^{2}\rho ^{\prime }D_{\mu }^{-1}H_{n}^{-1}+o_{p}\left( 1\right)
\label{VLhat 3}
\end{eqnarray}%
Finally, we consider $\widehat{V}_{L,4}$. Again, noting that $K_{2,n}/\left(
\mu _{n}^{\min }\right) ^{2}\rightarrow \infty $ under Case II, we have,
upon applying the result given in Lemma S2-10, as well as part (f) of Lemma
S2-18 and Slutsky's theorem,%
\begin{equation*}
\frac{\left( \mu _{n}^{\min }\right) ^{2}}{K_{2,n}}\widehat{V}%
_{L,4}=H_{n}^{-1}\frac{\left( \mu _{n}^{\min }\right) ^{2}}{K_{2,n}}D_{\mu
}^{-1}\underline{S}_{L,4}D_{\mu }^{-1}H_{n}^{-1}\left( 1+o_{p}\left(
1\right) \right)
\end{equation*}%
\begin{equation}
=H_{n}^{-1}\frac{\left( \mu _{n}^{\min }\right) ^{2}}{K_{2,n}}\dsum\limits 
_{\substack{ \left( i,t\right) ,\left( j,s\right) =1  \\ \left( i,t\right)
\neq \left( j,s\right) }}^{m_{n}}A_{\left( i,t\right) ,\left( j,s\right)
}^{2}D_{\mu }^{-1}\underline{\phi }_{\left( i,t\right) }\underline{\phi }%
_{\left( j,s\right) }^{\prime }D_{\mu }^{-1}H_{n}^{-1}+o_{p}\left( 1\right) .
\label{VLhat 4}
\end{equation}%
It follows from equations (\ref{VLhat 1}), (\ref{VLhat 2}), (\ref{VLhat 3}),
and (\ref{VLhat 4}) that%
\begin{eqnarray*}
\frac{\left( \mu _{n}^{\min }\right) ^{2}D_{\mu }\widehat{V}_{L}D_{\mu }}{%
K_{2,n}} &=&H_{n}^{-1}\frac{\left( \mu _{n}^{\min }\right) ^{2}}{K_{2,n}}%
\dsum\limits_{\substack{ \left( i,t\right) ,\left( j,s\right) =1  \\ \left(
i,t\right) \neq \left( j,s\right) }}^{m_{n}}A_{\left( i,t\right) ,\left(
j,s\right) }^{2}D_{\mu }^{-1}\left( \sigma _{\left( i,t\right) }^{2}%
\underline{\Psi }_{\left( j,s\right) }+\underline{\phi }_{\left( i,t\right) }%
\underline{\phi }_{\left( j,s\right) }^{\prime }\right) D_{\mu
}^{-1}H_{n}^{-1}+o_{p}\left( 1\right) \\
&=&\frac{\left( \mu _{n}^{\min }\right) ^{2}}{K_{2,n}}H_{n}^{-1}\Sigma
_{2,n}H_{n}^{-1}+o_{p}\left( 1\right) =\Lambda _{II,n}+o_{p}\left( 1\right) 
\text{. \ }
\end{eqnarray*}%
To show the same result for FEFUL, note again that $\widehat{\delta }_{F}$
satisfies the conditions of Lemmas S2-12 and S2-18. Hence, we can make the
same argument as given above for FELIM, except using Lemma S2-12 in lieu of
Lemma S2-10 to obtain $\left[ \left( \mu _{n}^{\min }\right) ^{2}/K_{2,n}%
\right] D_{\mu }\widehat{V}_{F}D_{\mu }=$

\noindent $\left[ \left( \mu _{n}^{\min }\right) ^{2}/K_{2,n}\right]
H_{n}^{-1}\Sigma _{2,n}H_{n}^{-1}+o_{p}\left( 1\right) =\Lambda
_{II,n}+o_{p}\left( 1\right) $. $\square $

\medskip

\noindent \textbf{Proof of Theorem 5:}

To show part (a), first note that since $\mu _{1,n}=\cdot \cdot \cdot =\mu
_{d,n}=\mu _{n}^{\min }$ here, we can take $\mu _{n}^{\min }=\mu _{n}$.
Moreover, by part (d) of Lemma S2-3 and Assumption 3(iii), $\Lambda _{I,n}$
is positive definite $a.s.n$. In addition, specializing the result of part
(a) of Theorem 4 to this case, we have $D_{\mu }\widehat{V}_{L}D_{\mu }=\mu
_{n}^{2}\widehat{V}_{L}=\Lambda _{I,n}+o_{p}\left( 1\right) $, so that $\mu
_{n}^{2}\widehat{V}_{L}$ is positive definite w.p.a.1. Hence, under $%
H_{0}:c^{\prime }\delta _{0}=r$, we can write%
\begin{equation*}
\mathbb{T}_{L}=\frac{c^{\prime }\widehat{\delta }_{L,n}-r}{\sqrt{c^{\prime }%
\widehat{V}_{L}c}}=\frac{c^{\prime }\left( \widehat{\delta }_{L,n}-\delta
_{0}\right) }{\sqrt{c^{\prime }\widehat{V}_{L}c}}=\frac{c^{\prime }\Lambda
_{I,n}^{1/2}\left[ \mu _{n}\Lambda _{I,n}^{-1/2}\left( \widehat{\delta }%
_{L,n}-\delta _{0}\right) \right] }{\sqrt{c^{\prime }\mu _{n}^{2}\widehat{V}%
_{L}c}}
\end{equation*}%
Now, specializing the result of Theorem 2 to this case, we have $\Lambda
_{I,n}^{-1/2}D_{\mu }\left( \widehat{\delta }_{L,n}-\delta _{0}\right) =\mu
_{n}\Lambda _{I,n}^{-1/2}\left( \widehat{\delta }_{L,n}-\delta _{0}\right) 
\overset{d}{\rightarrow }N\left( 0,I_{d}\right) $. It follows by the
continuous mapping theorem that%
\begin{equation}
\mathbb{T}_{L}=\frac{c^{\prime }\Lambda _{I,n}^{1/2}\left[ \mu _{n}\Lambda
_{I,n}^{-1/2}\left( \widehat{\delta }_{L,n}-\delta _{0}\right) \right] }{%
\sqrt{c^{\prime }\Lambda _{I,n}c}}\left[ 1+o_{p}\left( 1\right) \right] 
\overset{d}{\rightarrow }N\left( 0,1\right) \text{.}  \label{asy normality I}
\end{equation}%
On the other hand, under $H_{1}$, we have $c^{\prime }\delta _{0}=r+h$ for
some $h\in \mathbb{R}\backslash \left\{ 0\right\} $, and we can write $%
\mathbb{T}_{L}=\left( c^{\prime }\widehat{\delta }_{L,n}-r\right) /\sqrt{%
c^{\prime }\widehat{V}_{L}c}=c^{\prime }\left( \widehat{\delta }%
_{L,n}-\delta _{0}\right) /\sqrt{c^{\prime }\widehat{V}_{L}c}+h/\sqrt{%
c^{\prime }\widehat{V}_{L}c}$. The first term above is $O_{p}\left( 1\right)
,$ as shown in (\ref{asy normality I}) above, whereas application of part
(a) of Theorem 4 and Slutsky's theorem shows that $\mu _{n}^{2}c^{\prime }%
\widehat{V}_{L}c=c^{\prime }\Lambda _{I,n}c+o_{p}\left( 1\right) $, where $%
c^{\prime }\Lambda _{I,n}c>0$ for all $c\neq 0$ $a.s.n.$ in light of part
(d) of Lemma S2-3 and Assumption 3(iii). In addition, by parts (a) and (c)
of Lemma S2-3; Assumption 3(iii); and the fact that, under Case I, $%
K_{2,n}/\left( \mu _{n}^{\min }\right) ^{2}=K_{2,n}/\mu _{n}^{2}=O\left(
1\right) $; there exists a positive constant $C<\infty $ such that, almost
surely for all $n$ sufficiently large,%
\begin{equation}
\lambda _{\max }\left( \Lambda _{I,n}\right) \leq \frac{\lambda _{\max }%
\left[ VC\left( \Gamma ^{\prime }M^{\left( Z_{1},Q\right) }\varepsilon /%
\sqrt{n}\right) |\mathcal{F}_{n}^{W}\right] +\frac{K_{2,n}}{\mu _{n}^{2}}%
\lambda _{\max }\left[ VC\left( \underline{U}^{\prime }A\varepsilon /\sqrt{%
K_{2,n}}\right) |\mathcal{F}_{n}^{W}\right] }{\left[ \lambda _{\min }\left(
H_{n}\right) \right] ^{2}}\leq C\text{.}  \label{maxlam bd case I}
\end{equation}%
It follows that, in this case, $h/\sqrt{c^{\prime }\widehat{V}_{L}c}=\mu
_{n}h/\sqrt{c^{\prime }\mu _{n}^{2}\widehat{V}_{L}c}=\left( \mu _{n}h/\sqrt{%
c^{\prime }\Lambda _{I,n}c}\right) \left[ 1+o_{p}\left( 1\right) \right] $.
So, w.p.a.1, $h/\sqrt{c^{\prime }\widehat{V}_{L}c}\rightarrow +\infty $ \ if 
$h>0,$ whereas $h/\sqrt{c^{\prime }\widehat{V}_{L}c}\rightarrow -\infty $ \
if $h<0$, from which the stated result follows. Finally, note that the
results for $\mathbb{T}_{F}$ can be shown in the same way, so to avoid
redundancy, we omit the proof.

To show part (b), note that, setting $\widetilde{L}_{n}=c^{\prime }$ and $%
D_{\mu }=\mu _{n}\cdot I_{d}$ in Theorem 3, we have $\left( \mu _{n}/\sqrt{%
K_{2,n}}\right) \left( c^{\prime }\Lambda _{II,n}c\right) ^{-1/2}c^{\prime }%
\left[ \mu _{n}\left( \widehat{\delta }_{L,n}-\delta _{0}\right) \right] 
\overset{d}{\rightarrow }N\left( 0,1\right) $. Moreover, part (b) of Theorem
4 implies that $\left( \mu _{n}^{2}/K_{2,n}\right) D_{\mu }\widehat{V}%
_{L}D_{\mu }=\left( \mu _{n}^{4}/K_{2,n}\right) \widehat{V}_{L}=\Lambda
_{II,n}+o_{p}\left( 1\right) $. It follows that, under $H_{0}:c^{\prime
}\delta _{0}=r$,%
\begin{equation}
\mathbb{T}_{L}=\frac{\left( \mu _{n}/\sqrt{K_{2,n}}\right) c^{\prime }\left[
\mu _{n}\left( \widehat{\delta }_{L,n}-\delta _{0}\right) \right] }{\sqrt{%
c^{\prime }\left( \mu _{n}^{4}/K_{2,n}\right) \widehat{V}_{L}c}}=\frac{%
\left( \mu _{n}/\sqrt{K_{2,n}}\right) c^{\prime }\left[ \mu _{n}\left( 
\widehat{\delta }_{L,n}-\delta _{0}\right) \right] }{\sqrt{c^{\prime
}\Lambda _{II,n}c}}\left[ 1+o_{p}\left( 1\right) \right] \overset{d}{%
\rightarrow }N\left( 0,1\right) \text{.}  \label{asy normality II}
\end{equation}%
Under $H_{1}$, we again write $c^{\prime }\delta _{0}=r+h$ for some $h\in 
\mathbb{R}\backslash \left\{ 0\right\} $, and note that, in this case, by
part (b) of Theorem 4 and Slutsky's theorem, we have $\left( \mu
_{n}^{4}/K_{2,n}\right) c^{\prime }\widehat{V}_{L}c=c^{\prime }\Lambda
_{II,n}c+o_{p}\left( 1\right) $. Moreover, there exists a positive constant $%
\underline{C}$ such that $c^{\prime }\Lambda _{II,n}c=\mu _{n}^{2}c^{\prime
}H_{n}^{-1}\Sigma _{2,n}H_{n}^{-1}c/K_{2,n}=c^{\prime }H_{n}^{-1}VC\left( 
\underline{U}^{\prime }A\varepsilon /\sqrt{K_{2,n}}|\mathcal{F}%
_{n}^{W}\right) H_{n}^{-1}c\geq \underline{C}>0$ $a.s.n.$ for all $c\neq 0$,
by the almost sure positive definiteness of $VC\left( \underline{U}^{\prime
}A\varepsilon /\sqrt{K_{2,n}}|\mathcal{F}_{n}^{W}\right) $ as shown in part
(b) of Lemma S2-3. In addition, by part (c) of Lemma S2-3 and Assumption
3(iii), there exists a positive constant $C$ such that, almost surely for
all $n$ sufficiently large%
\begin{equation}
\lambda _{\max }\left( \Lambda _{II,n}\right) \leq \frac{\mu _{n}^{2}}{%
K_{2,n}}\frac{1}{\left[ \lambda _{\min }\left( H_{n}\right) \right] ^{2}}%
\frac{K_{2,n}}{\mu _{n}^{2}}\lambda _{\max }\left[ VC\left( \frac{\underline{%
U}^{\prime }A\varepsilon }{\sqrt{K_{2,n}}}\right) |\mathcal{F}_{n}^{W}\right]
\leq C<\infty \text{.}  \label{maxlam bd case II}
\end{equation}%
It follows that, for this case,%
\begin{equation*}
\frac{h}{\sqrt{c^{\prime }\widehat{V}_{L}c}}=\frac{\left( \mu _{n}^{2}/\sqrt{%
K_{2,n}}\right) h}{\sqrt{\left( \mu _{n}^{4}/K_{2,n}\right) c^{\prime }%
\widehat{V}_{L}c}}=\frac{\left( \mu _{n}^{2}/\sqrt{K_{2,n}}\right) h}{\sqrt{%
c^{\prime }\Lambda _{II,n}c}}\left[ 1+o_{p}\left( 1\right) \right] \text{.}
\end{equation*}%
Hence, w.p.a.1, $h/\sqrt{c^{\prime }\widehat{V}_{L}c}\rightarrow +\infty $ \
if $h>0$ whereas $h/\sqrt{c^{\prime }\widehat{V}_{L}c}\rightarrow -\infty $
\ if $h<0$, given the condition that $\mu _{n}^{2}/\sqrt{K_{2,n}}\rightarrow
\infty $. Finally, write 
\begin{equation*}
\mathbb{T}_{L}=\frac{c^{\prime }\widehat{\delta }_{L,n}-r}{\sqrt{c^{\prime }%
\widehat{V}_{L}c}}=\frac{c^{\prime }\left( \widehat{\delta }_{L,n}-\delta
_{0}\right) }{\sqrt{c^{\prime }\widehat{V}_{L}c}}+\frac{h}{\sqrt{c^{\prime }%
\widehat{V}_{L}c}}\text{.}
\end{equation*}%
Since the first term on the right-hand side above is $O_{p}\left( 1\right) $
as shown in (\ref{asy normality II}), we deduce that w.p.a.1, $\mathbb{T}%
_{L}\rightarrow +\infty $ \ if $h>0$ and $\mathbb{T}_{L}\rightarrow -\infty $
\ if $h<0$. The results for $\mathbb{T}_{F}$ can be shown in the same way,
so to avoid redundancy, we omit the proof. $\square $

\medskip

\noindent \textbf{Proof of Theorem 6:}

To show part (a), note first that, by part (d) of Lemma S2-3 and Assumption
3*(iv), $\Lambda _{I,n}$ is positive definite $a.s.n$. Hence, under $%
H_{0}:e_{g}^{\prime }\delta _{0}=r$, we can write%
\begin{equation}
\mathbb{T}_{L}=\frac{e_{g}^{\prime }\widehat{\delta }_{L}-r}{\sqrt{%
e_{g}^{\prime }\widehat{V}_{L}e_{g}}}=\frac{e_{g}^{\prime }\left( \mu
_{g,n}\right) D_{\mu }^{-1}\Lambda _{I,n}^{1/2}\left[ \Lambda
_{I,n}^{-1/2}D_{\mu }\left( \widehat{\delta }_{L}-\delta _{0}\right) \right] 
}{\sqrt{e_{g}^{\prime }\left( \mu _{g,n}\right) D_{\mu }^{-1}D_{\mu }%
\widehat{V}_{L}D_{\mu }D_{\mu }^{-1}\left( \mu _{g,n}\right) e_{g}}}=\frac{%
e_{g}^{\prime }\Lambda _{I,n}^{1/2}\left[ \Lambda _{I,n}^{-1/2}D_{\mu
}\left( \widehat{\delta }_{L}-\delta _{0}\right) \right] }{\sqrt{%
e_{g}^{\prime }D_{\mu }\widehat{V}_{L}D_{\mu }e_{g}}}
\label{Case I T stat repr}
\end{equation}%
where the last equality follows from the fact that

\noindent $e_{g}^{\prime }\left( \mu _{g,n}\right) D_{\mu }^{-1}=\left( \mu
_{g,n}\right) e_{g}^{\prime }\left[ 
\begin{array}{ccc}
\left( \mu _{1,n}\right) ^{-1}e_{1} & \cdots  & \left( \mu _{d,n}\right)
^{-1}e_{d}%
\end{array}%
\right] =e_{g}^{\prime }$. Now, applying the result of Theorem 2 to this
case, we have $\Lambda _{I,n}^{-1/2}D_{\mu }\left( \widehat{\delta }%
_{L}-\delta _{0}\right) \overset{d}{\rightarrow }N\left( 0,I_{d}\right) $.
In addition, applying the result of part (a) of Theorem 4 to this case, we
have $D_{\mu }\widehat{V}_{L}D_{\mu }=\Lambda _{I,n}+o_{p}\left( 1\right) $.
Let $a_{n}^{\prime }=e_{g}^{\prime }\Lambda _{I,n}^{1/2}/\sqrt{e_{g}^{\prime
}\Lambda _{I,n}e_{g}}$, and note that $a_{n}^{\prime }a_{n}=1$. It follows
from these intermediate results and the continuous mapping theorem that $%
\mathbb{T}_{L}=a_{n}^{\prime }\Lambda _{I,n}^{-1/2}D_{\mu }\left( \widehat{%
\delta }_{L}-\delta _{0}\right) \left[ 1+o_{p}\left( 1\right) \right] 
\overset{d}{\rightarrow }N\left( 0,1\right) $. On the other hand, under $%
H_{1}$, we can take $e_{g}^{\prime }\delta _{0}=r+h$ for some $h\in \mathbb{R%
}\backslash \left\{ 0\right\} $. Write%
\begin{equation*}
\mathbb{T}_{L}=\frac{e_{g}^{\prime }\left( \widehat{\delta }_{L}-\delta
_{0}\right) }{\sqrt{e_{g}^{\prime }\widehat{V}_{L}e_{g}}}+\frac{\left( \mu
_{g,n}\right) h}{\sqrt{e_{g}^{\prime }\left( \mu _{g,n}\right) D_{\mu
}^{-1}D_{\mu }\widehat{V}_{L}D_{\mu }D_{\mu }^{-1}\left( \mu _{g,n}\right)
e_{g}}}=\frac{e_{g}^{\prime }\left( \widehat{\delta }_{L}-\delta _{0}\right) 
}{\sqrt{e_{g}^{\prime }\widehat{V}_{L}e_{g}}}+\frac{\left( \mu _{g,n}\right)
h}{\sqrt{e_{g}^{\prime }D_{\mu }\widehat{V}_{L}D_{\mu }e_{g}}}\text{,}
\end{equation*}%
where the last line again follows from the identity $e_{g}^{\prime }\left(
\mu _{g,n}\right) D_{\mu }^{-1}=e_{g}^{\prime }$, as shown previously. Now,
the first term on the right-hand side of the last equality above has
previously been shown to converge to a $N\left( 0,1\right) $ distribution so
that, in particular, $e_{g}^{\prime }\left( \widehat{\delta }_{L}-\delta
_{0}\right) /\sqrt{e_{g}^{\prime }\widehat{V}_{L}e_{g}}=O_{p}\left( 1\right) 
$. Moreover, application of part (a) of Theorem 4 and Slutsky's theorem
shows that $e_{g}^{\prime }D_{\mu }\widehat{V}_{L}D_{\mu
}e_{g}=e_{g}^{\prime }\Lambda _{I,n}e_{g}+o_{p}\left( 1\right) $, where $%
e_{g}^{\prime }\Lambda _{I,n}e_{g}>0$  $a.s.n.$ since $\Lambda _{I,n}$ is
positive definite $a.s.n.$ by part (d) of Lemma S2-3 and Assumption 3*(iv).
In addition, by parts (a) and (c) of Lemma S2-3, Assumption 3*(iv), and the
fact that $K_{2,n}/\left( \mu _{n}^{\min }\right) ^{2}=K_{2,n}/\mu
_{n}^{2}=O\left( 1\right) $ under Case I, there exists a positive constant $C
$ such that, almost surely for all $n$ sufficiently large, $\lambda _{\max
}\left( \Lambda _{I,n}\right) \leq C<\infty $, as can be shown by following
an argument similar to that given previously in obtaining equation (\ref%
{maxlam bd case I}) in the proof of Theorem 5. Hence, in this case, $h/\sqrt{%
e_{g}^{\prime }\widehat{V}_{L}e_{g}}=\left( \mu _{g,n}\right) h/\sqrt{%
e_{g}^{\prime }D_{\mu }\widehat{V}_{L}D_{\mu }e_{g}}=\left( \left( \mu
_{g,n}\right) h/\sqrt{e_{g}^{\prime }\Lambda _{I,n}e_{g}}\right) \left[
1+o_{p}\left( 1\right) \right] $. Given that $\mu _{g,n}\rightarrow \infty $
as $n\rightarrow \infty $, w.p.a.1, $h/\sqrt{e_{g}^{\prime }\widehat{V}%
_{L}e_{g}}\rightarrow +\infty $ \ if $h>0,$ whereas $h/\sqrt{e_{g}^{\prime }%
\widehat{V}_{L}e_{g}}\rightarrow -\infty $ \ if $h<0$, from which the stated
result follows. The results for $\mathbb{T}_{F}$ can be shown in the same
way, so to avoid redundancy, we omit the proof.

To show part (b), note first that 
\begin{equation*}
\left( \mu _{n}^{\min }\right) D_{\mu }^{-1}=\left( \mu _{n}^{\min }\right)
\left( 
\begin{array}{cc}
D_{1}^{-1} & 0 \\ 
0 & \left( \mu _{n}^{\min }\right) ^{-1}\cdot I_{d_{2}}%
\end{array}%
\right) \rightarrow \left( 
\begin{array}{cc}
0 & 0 \\ 
0 & I_{d_{2}}%
\end{array}%
\right) =D_{0}\text{, \ }\left( say\right) \text{.}
\end{equation*}%
Moreover, note that there exists a positive constant $C$ such that $%
e_{g}^{\prime }\Lambda _{II,n}e_{g}\geq $ $C>0$ w.p.a.1 as $n\rightarrow
\infty $ since%
\begin{eqnarray*}
e_{g}^{\prime }\Lambda _{II,n}e_{g} &=&e_{g}^{\prime }H_{n}^{-1}\left( \mu
_{n}^{\min }\right) D_{\mu }^{-1}VC\left( \frac{\underline{U}^{\prime
}A\varepsilon }{\sqrt{K_{2,n}}}|\mathcal{F}_{n}^{W}\right) D_{\mu
}^{-1}\left( \mu _{n}^{\min }\right) H_{n}^{-1}e_{g} \\
&=&e_{g}^{\prime }H_{n}^{-1}D_{0}VC\left( \frac{\underline{U}^{\prime
}A\varepsilon }{\sqrt{K_{2,n}}}|\mathcal{F}_{n}^{W}\right)
D_{0}H_{n}^{-1}e_{g}\left[ 1+o_{p}\left( 1\right) \right] 
\end{eqnarray*}%
and since, by applying the result of part (b) of Lemma S2-3 and Assumption
3*(v), we have $e_{g}^{\prime }H_{n}^{-1}D_{0}V\left( \underline{U}^{\prime
}A\varepsilon /\sqrt{K_{2,n}}|\mathcal{F}_{n}^{W}\right)
D_{0}H_{n}^{-1}e_{g}\geq \underline{C}e_{g}^{\prime }\overline{H}_{2\cdot
}^{\prime }\overline{H}_{2\cdot }e_{g}\geq \underline{C}C_{\ast }=C>0$ $%
a.s.n.$ for $\overline{H}_{2\cdot }$ as defined in Assumption 3*(v) and for
positive constants $\underline{C}$ (defined in Lemma S2-3), $C_{\ast }$
(defined in Assumption 3*(v)), and $C=\underline{C}C_{\ast }$. Now, setting $%
\widetilde{L}_{n}=e_{g}^{\prime }$ in Theorem 3, we have $\left( \mu
_{n}^{\min }/\sqrt{K_{2,n}}\right) \left( e_{g}^{\prime }\Lambda
_{II,n}e_{g}\right) ^{-1/2}e_{g}^{\prime }D_{\mu }\left( \widehat{\delta }%
_{L}-\delta _{0}\right) \overset{d}{\rightarrow }N\left( 0,1\right) $, and,
by applying part (b) of Theorem 4 and Slutsky's theorem, we also obtain $%
\left[ \left( \mu _{n}^{\min }\right) ^{2}/K_{2,n}\right] e_{g}^{\prime
}D_{\mu }\widehat{V}_{L}D_{\mu }e_{g}=e_{g}^{\prime }\Lambda
_{II,n}e_{g}+o_{p}\left( 1\right) $. Hence, under $H_{0}:e_{g}^{\prime
}\delta _{0}=r$, we can apply the identity $e_{g}^{\prime }\left( \mu
_{g,n}\right) D_{\mu }^{-1}=e_{g}^{\prime }$ to obtain 
\begin{eqnarray*}
\mathbb{T}_{L} &=&\frac{\left( \mu _{n}^{\min }/\sqrt{K_{2,n}}\right) \left[
e_{g}^{\prime }\left( \mu _{g,n}\right) D_{\mu }^{-1}D_{\mu }\left( \widehat{%
\delta }_{L}-\delta _{0}\right) \right] }{\sqrt{\left[ \left( \mu _{n}^{\min
}\right) ^{2}/K_{2,n}\right] e_{g}^{\prime }\left( \mu _{g,n}\right) D_{\mu
}^{-1}D_{\mu }\widehat{V}_{L}D_{\mu }D_{\mu }^{-1}\left( \mu _{g,n}\right)
e_{g}}}=\frac{\left( \mu _{n}^{\min }/\sqrt{K_{2,n}}\right) \left[
e_{g}^{\prime }D_{\mu }\left( \widehat{\delta }_{L}-\delta _{0}\right) %
\right] }{\sqrt{\left[ \left( \mu _{n}^{\min }\right) ^{2}/K_{2,n}\right]
e_{g}^{\prime }D_{\mu }\widehat{V}_{L}D_{\mu }e_{g}}}\text{ \ } \\
&=&\frac{\left( \mu _{n}^{\min }/\sqrt{K_{2,n}}\right) \left[ e_{g}^{\prime
}D_{\mu }\left( \widehat{\delta }_{L}-\delta _{0}\right) \right] }{\sqrt{%
e_{g}^{\prime }\Lambda _{II,n}e_{g}}}\left[ 1+o_{p}\left( 1\right) \right] 
\overset{d}{\rightarrow }N\left( 0,1\right) \text{.}
\end{eqnarray*}%
Under $H_{1}$, write $e_{g}^{\prime }\delta _{0}=r+h$ for some $h\in \mathbb{%
R}\backslash \left\{ 0\right\} $. As shown above,

\noindent $\left[ \left( \mu _{n}^{\min }\right) ^{2}/K_{2,n}\right]
e_{g}^{\prime }D_{\mu }\widehat{V}_{L}D_{\mu }e_{g}=e_{g}^{\prime }\Lambda
_{II,n}e_{g}+o_{p}\left( 1\right) $, where there exists a positive constant $%
C$ such that $e_{g}^{\prime }\Lambda _{II,n}e_{g}\geq $ $C>0$ w.p.a.1 as $%
n\rightarrow \infty $. In addition, by part (c) of Lemma S2-3 and Assumption
3$^{\text{*}}$(iv), there exists a positive constant $C$ such that, almost
surely for all $n$ sufficiently large, $\lambda _{\max }\left( \Lambda
_{II,n}\right) \leq C<\infty $, as can be shown by following an argument
similar to that given previously in obtaining equation (\ref{maxlam bd case
II}) in the proof of Theorem 5. It follows from these results and from
making use of the identity $e_{g}^{\prime }\left( \mu _{g,n}\right) D_{\mu
}^{-1}=e_{g}^{\prime }$ that%
\begin{equation*}
\frac{h}{\sqrt{e_{g}^{\prime }\widehat{V}_{L}e_{g}}}=\frac{\left( \mu
_{n}^{\min }/\sqrt{K_{2,n}}\right) \left( \mu _{g,n}\right) h}{\sqrt{\frac{%
\left( \mu _{n}^{\min }\right) ^{2}}{K_{2,n}}e_{g}^{\prime }\left( \mu
_{g,n}\right) D_{\mu }^{-1}D_{\mu }\widehat{V}_{L}D_{\mu }D_{\mu
}^{-1}\left( \mu _{g,n}\right) e_{g}}}=\frac{\mu _{n}^{\min }}{\sqrt{K_{2,n}}%
}\frac{\left( \mu _{g,n}\right) h}{\sqrt{e_{g}^{\prime }\Lambda _{II,n}e_{g}}%
}\left[ 1+o_{p}\left( 1\right) \right] \text{.}
\end{equation*}%
Thus, w.p.a.1, $h/\sqrt{e_{g}^{\prime }\widehat{V}_{L}e_{g}}\rightarrow
+\infty $ \ if $h>0$ whereas $h/\sqrt{e_{g}^{\prime }\widehat{V}_{L}e_{g}}%
\rightarrow -\infty $ \ if $h<0$, given that $\left( \mu _{n}^{\min }\right)
^{2}/\sqrt{K_{2,n}}\rightarrow \infty $ and $\mu _{n}^{\min }/\mu
_{g,n}=O\left( 1\right) $ for any $g\in \left\{ 1,...,d\right\} $. Finally,
write 
\begin{equation}
\mathbb{T}_{L}=\frac{e_{g}^{\prime }\widehat{\delta }_{L}-r}{\sqrt{%
e_{g}^{\prime }\widehat{V}_{L}e_{g}}}=\frac{e_{g}^{\prime }\left( \widehat{%
\delta }_{L}-\delta _{0}\right) }{\sqrt{e_{g}^{\prime }\widehat{V}_{L}e_{g}}}%
+\frac{h}{\sqrt{e_{g}^{\prime }\widehat{V}_{L}e_{g}}}\text{.}
\label{T stat under H1}
\end{equation}%
Since the first term on the right-hand side of equation (\ref{T stat under
H1}) is $O_{p}\left( 1\right) ,$ as shown above, we deduce that w.p.a.1., $%
\mathbb{T}_{L}\rightarrow +\infty $ \ if $h>0$ and $\mathbb{T}%
_{L}\rightarrow -\infty $ \ if $h<0$. Finally, the results for $\mathbb{T}%
_{F}$ can be shown in the same way, so to avoid redundancy, we omit the
proof. $\square $

\end{document}
