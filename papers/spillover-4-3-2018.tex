%2multibyte Version: 5.50.0.2960 CodePage: 936
%% This document created by Scientific Word (R) Version 3.5

\documentclass[a4paper,amstex,11pt]{article}%
\usepackage{multirow}
\usepackage{enumerate}
\usepackage{graphics}
\usepackage{epsfig}
\usepackage{verbatim}
\usepackage{booktabs}
\usepackage{bm}
\usepackage{latexsym}
\usepackage{afterpage}
\usepackage{amsmath}
\usepackage{caption}
\usepackage{subcaption}
\usepackage{natbib}
\usepackage{amsmath,amsthm,amsfonts,natbib,bm,latexsym,enumerate,url}
\usepackage{graphicx}
\usepackage{amsfonts}
\usepackage{setspace}
\usepackage{amssymb}
\usepackage{lscape}
\usepackage{adjustbox}
\usepackage{afterpage}%
\setcounter{MaxMatrixCols}{30}
%TCIDATA{OutputFilter=latex2.dll}
%TCIDATA{Version=5.50.0.2960}
%TCIDATA{Codepage=936}
%TCIDATA{CSTFile=LaTeX article (bright).cst}
%TCIDATA{Created=Tue Nov 04 20:07:01 2003}
%TCIDATA{LastRevised=Thursday, March 22, 2018 19:30:22}
%TCIDATA{<META NAME="GraphicsSave" CONTENT="32">}
%TCIDATA{<META NAME="SaveForMode" CONTENT="1">}
%TCIDATA{BibliographyScheme=Manual}
%TCIDATA{<META NAME="DocumentShell" CONTENT="General\Blank Document">}
%TCIDATA{Language=American English}
%BeginMSIPreambleData
\providecommand{\U}[1]{\protect\rule{.1in}{.1in}}
%EndMSIPreambleData
\providecommand{\U}[1]{\protect \rule{.1in}{.1in}}
\providecommand{\U}[1]{\protect \rule{.1in}{.1in}}
\providecommand{\U}[1]{\protect \rule{.1in}{.1in}}
\renewcommand{\baselinestretch}{0.9}
\textwidth=6.5in \textheight=9in \oddsidemargin=0in
\evensidemargin=0in \topmargin=-0.25in
\renewcommand {\baselinestretch}{1.3}
\newtheorem{theorem}{Theorem}
\newtheorem{acknowledgement}[theorem]{Acknowledgement}
\newtheorem{algorithm}[theorem]{Algorithm}
\newtheorem{axiom}[theorem]{Axiom}
\newtheorem{case}[theorem]{Case}
\newtheorem{claim}[theorem]{Claim}
\newtheorem{conclusion}[theorem]{Conclusion}
\newtheorem{condition}[theorem]{Condition}
\newtheorem{conjecture}[theorem]{Conjecture}
\newtheorem{corollary}{Corollary}
\newtheorem{criterion}[theorem]{Criterion}
\newtheorem{definition}[theorem]{Definition}
\newtheorem{example}[theorem]{Example}
\newtheorem{exercise}[theorem]{Exercise}
\newtheorem{lemma}{Lemma}
\newtheorem{assumption}{Assumption}
\newtheorem{notation}[theorem]{Notation}
\newtheorem{problem}[theorem]{Problem}
\newtheorem{proposition}{Proposition}
\theoremstyle{remark}
\newtheorem{remark}{Remark}
\newtheorem{solution}[theorem]{Solution}
\newtheorem{summary}[theorem]{Summary}
\begin{document}
\bigskip\bigskip\bigskip

\begin{center}
\bigskip\bigskip{\LARGE Jump Spillover and Risk Effects on Excess Returns in
the United States During the Great Recession*}\bigskip

{\Large Jessica Schlossberg}$^{1}${\Large and Norman R. Swanson}$^{2}$

\bigskip

$^{1}${\Large Ernst and Young and }$^{1,2}${\Large Rutgers University}

{\large April 2018}

\bigskip

\bigskip Abstract

\bigskip
\end{center}

{\small In this paper, we review econometric methodology that is used to test for jumps and to decompose realized
volatility into continuous and jump components. In order to illustrate how to implement the methods discussed, 
we also present the results of an empirical analysis in which 
we separate continuous asset return variation and finite activity
jump variation from excess returns on various U.S. market sector exchange traded
funds (ETFs), during and around the Great Recession of 2008. Our objective is
to characterize the financial contagion that was present during one of the
greatest financial crises in U.S. history. In particular, we study how shocks,
as measured by jumps, propagate through nine different market sectors. One
element of our analysis involves the investigation of causal linkages
associated with jumps (via use of vector autoregressions), and
another involves the examination of the predictive content of jumps for excess
returns. We find that as early as 2006, jump spillover effects became more
pronounced in the markets. We also observe that
jumps had a significant effect on excess returns during 2008 and 2009; but not in the years before and after the recession.}

\bigskip

\noindent\textit{Keywords}: High-frequency jumps, jump spillover, jump risks,
excess returns, ETFs, Great Recession, high-frequency data.

\noindent\textit{JEL classification}: G12, C58, C22, G10.

\renewcommand {\baselinestretch}{1.0}

\_ \_ \_ \_ \_ \_ \_ \_ \_ \_ \_ \_ \_ \_ \_ \_ \_ \_ \_ \_ \_ \_ \_ \_ \_ \_ \_

{\small \renewcommand {\baselinestretch}{1.0} }

\begin{spacing}{1.0}
{\small $^{\ast}$ }{\footnotesize Jessica Schlossberg, Department of Economics, Rutgers University, New Brunswick, NJ 08901, USA, jessicaschlossbe@gmail.com. Norman Swanson, Department of Economics, Rutgers University, New Brunswick, NJ 08901, USA, nswanson@econ.rutgers.edu. This chapter has been prepared for the Handbook of Financial Econometrics, Mathematics, Statistics, and Technology; and we are grateful to the editor, Cheng-Few Lee, for useful comments. Additionally, we have
benefited from comments made by Valentina Corradi, Yuan Liao, John Landon-Lane, Bruce Mizrach, and Xiye Yang.}
\end{spacing}


\renewcommand {\baselinestretch}{1.3}

\newpage

\clearpage


\allowdisplaybreaks


\section{Introduction}

The so-called Great Recession of 2008-2009 has received considerable attention
in the economics and finance professions in recent years. Indeed, countless
academic papers have studied its causes, impact, and aftermath. This paper
provides a fresh perspective by looking at this important event through the
lens of high frequency trading data. First, we survey recent advances in the
econometric methodology of analyzing jumps using high frequency financial
data. Then, we utilize five-minute trading data and apply the aforementioned
econometric methods to analyze jump spillover effects and jump contributions
to excess returns in U.S. markets during and around the Great Recession.

The economic rationale for the paper draws on the idea that jumps are
associated with specific economic events. Andersen, Bollerslev, Diebold, and
Vega (2003) study foreign exchange markets and find that unexpected news
announcements result in conditional mean jumps; and that negative news has a
greater impact than positive news. Huang (2015) analyzes jumps using intra-day
high frequency data in equity and fixed-income markets, and finds that more
large jumps are present on days with news than on days without news. Evans
(2011) discovers that approximately one third of jumps between July 1998 and
June 2006 in the U.S. futures markets are connected with U.S. macroeconomic
news announcements, and that these news announcements lead to large jumps.
Jiang and Verdelhan (2011) find that pre-announcement liquidity shocks can be
used to predict jumps in treasury bond markets and are therefore useful for
asset pricing. Lee and Mykland (2008) apply nonparametric tests to search for
jumps in equity markets. Their results suggest that different pricing models
should be applied for individual equity options and index options, due to the
fact that jumps in individual stocks are associated with company-specific news
events. Lahaye, Laurent, and Neely (2011) focus on futures markets, and find
that the size, frequency and timing of jumps in futures markets are related to
economic shocks. Bollerslev, Law, and Tauchen (2008) examine jumps in both
individual stocks and an aggregate market index. They conclude that the
existence and pattern of co-jumps provides evidence of a relationship between
jumps and macroeconomic news announcements. Similar results can also be seen
in the currency markets. For example, Chatrath, Miao, Ramchander, and
Villupuram (2014) find that correlation exists between jumps and news
announcements. They also find evidence of co-jumps. Some authors focus on
international markets rather than just domestic markets. For example,
Asgharian and Bengtsson (2006) focus on the U.S. market and several European
markets and find that significant jump spillover effects exist in countries
that have features in common, such as industry structure or geographic
location. Asgharian and Nossman (2011) inspect jumps in equity markets in
several regions and conclude that local European markets are under the
influence of U.S. markets. Jawadi, Louhichi, and Cheffou (2015) use
nonparametric econometric methods to test contagion hypotheses, and provide
evidence of dependence between jumps in three European markets and U.S.
markets. Lahaye, Laurent, and Neely (2010) find that payroll announcements are
important in stock and bond futures markets, while trade related news often
creates co-jumps in exchange rate markets. A\"{\i}t-Sahalia and Xiu (2016)
provide strong evidence of correlation between financial crises and increase
in the quadratic variation of assets.

In this paper, we extend the findings of Asgharian and Bengtsson (2006),
Asgharian and Nossman (2011), Jawadi, Louhichi, and Cheffou (2015), and
A\"{\i}t-Sahalia and Xiu (2016) in three ways. First, our research centers on
the domestic jump spillover effects in the U.S. during the 2008 financial
crisis. Particularly, we look at jump spillover effects across nine market
sectors. Second, we decompose jumps based on their size and investigate
financial market interactions using different sized jumps. By using truncation
in order to identify (small and large) jumps, we are able to investigate how
different economic shocks affect U.S. markets. This is important, since
macroeconomics news events often cause large jumps, while many (asset) price
movements are associated with small jumps. Our approach is to remain agnostic
about the cause of jumps, and to instead focus on the relationship among
different jumps (in different market sectors, for example). Third, we focus
attention on the importance of jumps for explaining excess returns.

Following the methodology used in much of the extant literature on jumps in
financial markets, our approach to examining jump propagation is based on the
use of nonparametric tools. In particular, we apply nonparametric jump tests
and decomposition methods, which are discussed in detail in the sequel, in
order to characterize jumps. We then perform two regression analyses. In a
first analysis, we test the hypothesis that jump spillovers exists across
different market sectors. Our main findings are as follows. First, large jump
spillover effects that impact multiple markets seem to be correlated with the
major news and events and can be industry-specific. This is because large
jumps are known to be related to unexpected major news and \ events. Second,
total jump spillover effects are similar to large jump spillover effects, as
large jumps usually dominate the jump process. Third, strong large and total
jump spillover effects are observed prior to the onset of the 2008-2009
recession, and weakened in 2008; while small jump spillover effects
intensified as the recession unfolded. This can be explained by the different
origins of large jumps and small jumps. It is also consistent with a
hypothesis that that jumps are affected by trader's behavior in the markets.
Finally, jumps from the XLF (i.e., the financial sector) are not a major
player in our findings, as might be expected. this might be explained in part
by unmodelled nonlinear correlation across market sectors, for example.

In a second regression analysis we study the contribution of jumps to excess
returns. We find that jumps are statistically significant in models of excess
returns. Moreover, we observe a sharp increase in jump contribution to sector
excess returns in 2008 and 2009. This provides evidence that jumps are
important in asset pricing, especially in turbulent times.

The rest of the paper is organized as follows. Section 2 reviews nonparametric
jump tests and decomposition methods. Section 3 outlines the empirical
methodology used in our data analysis. Section 4 contains our empirical
findings. Finally, concluding remarks are gathered in Section 5.

\section{Jump Tests and Jump Decomposition Methods}

\subsection{Set-up}

\noindent Define log prices as $Y_{t}=log(P_{t}),$ and assume that they follow
an It\^{o} semimartingale process,
\begin{equation}
Y_{t}=Y_{0}+\int_{0}^{t}a_{u}d_{u}+\int_{0}^{t}\sigma_{u}dW_{u}+\int_{0}%
^{t}\int_{\{|y|\leq\epsilon\}}y(j-\nu)(du,dy)+\int_{0}^{t}\int_{\{|y|>\epsilon
\}}yj(du,dy),
\end{equation}
where $Y_{0}+\int_{0}^{t}a_{u}d_{u}+\int_{0}^{t}\sigma_{u}dW_{u}$ is a
Brownian semi-martingale. Here, $\int_{0}^{t}a_{u}d_{u}$ is the drift term,
with $a_{t}$ being the instantaneous drift, and $\int_{0}^{t}\sigma_{u}dW_{u}$
is the continuous part. with $\sigma_{t}$ being the spot volatility.
Additionally, $j$ is the jump measure of $Y_{t}$, and its predictable
compensator is the L\'{e}vy measure $\nu$. Finally, $\int_{0}^{t}%
\int_{\{|y|\leq\epsilon\}}y(j-\nu)(du,dy)$ is the so-called small jump
component, and $\int_{0}^{t}\int_{\{|y|>\epsilon\}}yj(du,dy)$ is the so-called
large jump component, with $\epsilon$ being an arbitrary cutoff level
specified in order to differentiate between small and large jumps.

Volatility is a latent variable, and realized measures are often employed to
consistently estimate it.\footnote{Sometimes, in financial econometrics, the
word variance is used interchangeably with volatility. Here we follow the
convention of equating volatility with sums of squared returns.} In the high
frequency literature, one of the most widely known measures is realized
volatility (RV). Suppose that $t>0$ is a fixed time period, for example, one
trading day, and the $i$th log-price of an asset observed during day $t$ is
$Y_{i,t}$. The intra-$i$th return on day $t$ is $r_{i,t}=Y_{i,t}-Y_{i-1,t}$,
where $i=1,2,...,t/\delta$ and $\delta$ is the sampling frequency. For one
trading day, we have the explicit expression for RV:
\begin{equation}
RV_{t}=\sum_{i=1}^{t/\delta}r_{i,t}^{2}.
\end{equation}


When sampling is at a high and fixed frequency (such as $N\rightarrow\infty$
or $\delta\rightarrow0$), then realized volatility converges to so-called
quadratic variation which is defined as follows:
\begin{equation}
\lbrack Y]_{t}=p\lim\limits_{\delta\rightarrow0}\sum_{i=0}^{t/\delta
-1}(Y_{t_{i}}-Y_{t_{i}})^{2},
\end{equation}
for any sequence of partitions $t_{0}=0<t_{1}<...<t_{n}=t,$ with
$sup_{i}\{t_{i+1}-t_{i}\}\rightarrow0$ for $\delta\rightarrow0$.
Thus
\[
RV_{t}\overset{\mathbb{P}}{\rightarrow}[Y]_{t}%
\]
where $\mathbb{P}$ denotes convergence in probability. Thus, realized quadratic
variation (QV) is expressed as:
\begin{equation}
QV=[Y_{\delta}]_{t}=\sum_{i=1}^{t/\delta}r_{i,t}^{2}%
\end{equation}


Another important measure is called integrated volatility, which is defined as
$\int_{0}^{t}\sigma_{u}^{2}du$. When asset prices are continuous on a fixed
interval $[0,T]$:
\begin{equation}
\lbrack Y]_{t}\overset{\mathbb{P}}{\rightarrow}\int_{0}^{t}\sigma_{u}^{2}du,
\end{equation}
and when asset prices also have a discontinuous component on $[0,T]$ (like in
Equation (1)):
\begin{equation}
\lbrack Y]_{t}\overset{\mathbb{P}}{\rightarrow}\int_{0}^{t}\sigma_{u}^{2}du+\sum_{u\leq
t}(\Delta Y_{u})^{2},
\end{equation}
where $\sum_{u\leq t}(\Delta Y_{u})$ is a pure jump process and a jump at time
$s$ is defined as $\Delta Y_{t}=Y_{u}-Y_{u-}$. Here, $\sum_{u\leq t}(\Delta
Y_{u})^{2}$ is the variation of the jump component.

\subsection{Jump Testing}

The literature on jump testing has been active since 2002. Testing whether or
not jumps are present in a process is particularly useful to do prior to
constructing realized measures of jump and continuous components of a
variable. For early relevant discussions in this area, see Andersen, Benzoni,
and Lund (2002) and Chernov et al. (2003), as well as A\"{\i}t-Sahalia (2002)
and Johannes (2004). In this paper, we discuss three different tests
including: the bipower-variation-based tests of Barndorff-Nielsen and Shephard
(2006a, 2006b, 2006c), Huang and Tauchen (2005), Andersen, Bollerslev and
Diebold (2007), and Lee and Mykland (2008); the swap-variance-based test due
to Jiang and Oomen (2008); and the truncated-power-variation based tests due
to A\"{\i}t-Sahalia and Jacod(2008, 2009a, 2009b) and Lee and Hanning (2010). We also discuss a so-called
long time span jump test dues to Corradi, Silvapulle and Swanson (2018), which is consistent (the above fixed time span tests are not consistent, in the sense that
power does not go to unity as the sample size increases)

\subsubsection{Bipower Variation Tests}

Under the assumption of Equation (1), Equation (6) shows that if the
theoretical integrated volatility can be properly estimated, jumps can be
measured using the difference between QV and realized integrated volatility.
This is the key idea underpinning bipower variation based tests.
Barndorff-Nielsen and Sharphard (2004) suggest using bipower variation to
estimate integrated volatility. Barndorff-Nielsen and Shephard (2006a) propose
various bipower variation based jump test statistics.

The quadratic variation defined in equation (3) is a special case of power
variation. Additionally, $s$th power variation is defined as:
\[
\{Y\}_{t}^{[s]}=p\lim\limits_{\delta\rightarrow0}\delta^{1-s/2}\sum
_{i=1}^{t/\delta}|r_{i,t}|^{s},
\]
where $s>0$. The bipower variation process is defined as:
\[
\{Y\}_{t}^{[s_{1},s_{2}]}=p\lim\limits_{\delta\rightarrow0}\delta
^{1-(s_{1}+s_{2})/2}\sum_{i=1}^{[t/\delta]-1}|r_{i,t}|^{s_{1}}|r_{i+1,t}%
|^{s_{2}},
\]
where $s_{1},s_{2}>0$. When $s_{1}=s_{2}=1$, $\{Y\}_{t}^{[1,1]}$ can be
consistently estimated using realized bipower variation (BV), defined as
follows:
\begin{equation}
BV_{t}=\{Y_{\delta}\}_{t}^{[1,1]}=\sum_{i=2}^{t/\delta}|r_{i-1,t}||r_{i,t}|.
\end{equation}


Barndorff-Nielsen and Shephard (2004) show that the power variation and
bipower variation can be expressed as:
\[
\mu^{-1}\{Y\}_{t}^{[s]}=\left\{
\begin{array}
[c]{lll}%
\int_{0}^{t}\sigma_{u}^{s}du &  & s\in(0,2)\\
\big[Y\big]_{t} &  & s=2\\
\infty &  & s>2
\end{array}
\right.
\]
and
\[
\mu_{s_{1}}^{-1}\mu_{s_{2}}^{-1}\{Y\}_{t}^{[s_{1},s_{2}]}=\left\{
\begin{array}
[c]{lll}%
\int_{0}^{t}\sigma_{u}^{s_{1}+s_{2}}du &  & max(s_{1},s_{2})\in(0,2)\\
x_{t}^{\ast} &  & max(s_{1},s_{2})=2\\
\infty &  & max(s_{1},s_{2})>2
\end{array}
\right.
\]
where $x_{t}^{\ast}$ is some stochastic process.

A special case is when $s_{1}=s_{2}=1$,
\[
\mu_{1}^{-2}\{Y\}_{t}^{[1,1]}=\int_{0}^{t}\sigma_{u}^{2}du.
\]
Thus, integrated volatility can be consistently estimated as:
\begin{equation}
\mu_{1}^{-2}BV\overset{\mathbb{P}}{\rightarrow}\int_{0}^{t}\sigma_{u}^{2}du\\
\end{equation}
where $\mu_{1}=E[u]=\sqrt{2}/\sqrt{\pi}\simeq0.79788$, and $u$ is $N(0,1)$
random variable.

The bipower jump test null hypothesis is that no jumps are present. Barndorff-Nielsen and Shephard (2006a) propose a linear jump test statistic $G$, and a ratio jump test statistic $H$:
\[
G=\frac{\delta^{-1/2}(\mu_{1}^{-2}BV_{t}-QV_{t})}{\sqrt{\int_{0}^{t}\eta
\sigma_{u}^{4}du}}\overset{d}{\rightarrow}N(0,1)
\]
and%
\[
H=\frac{\delta^{-1/2}(\frac{\mu_{1}^{-2}BV_{t}}{QV_{t}}-1)}{\sqrt{\eta
\frac{\int_{0}^{t}\sigma_{u}^{4}du}{\{\int_{0}^{t}\sigma_{u}^{2}du\}^{2}}}%
}\overset{d}{\rightarrow}N(0,1),
\]
where $\eta=(\pi^{2}/4)+\pi-5\simeq0.6090$ and $d$ means convergence in distribution. Here,$\int_{0}^{t}\sigma_{u}%
^{4}du$ is the integrated quarticity and can be estimated using realized
quadpower variation (QPV): \footnote{Barndorff-Nielsen et al. (2005) discuss a
more general case for realized multipower variation, and Barndorff-Nielsen,
Shephard, and Winkel (2006) analyze the case where the jump component is a
L\`{e}vy or non-Gaussian Ornstein-Uhlenbeck (OU) process.}
\begin{equation}
QPV_{t}=\{Y_{\delta}\}_{t}^{[1,1,1,1]}=\delta^{-1}\sum_{i=4}^{t/\delta
}|r_{i-3,t}||r_{i-2,t}||r_{i-1,t}||r_{i,t}|\overset{\mathbb{P}}{\rightarrow}\mu_{1}%
^{4}\int_{0}^{t}\sigma_{u}^{4}du.
\end{equation}
Additionally, $\int_{0}^{t}\sigma_{u}^{2}du$ can be estimated using BV. This
yields the following feasible linear jump and ratio jump statistics, $\hat{G}$
and $\hat{H}:$
\[
\hat{G}=\frac{\delta^{-1/2}(\mu_{1}^{-2}BV_{t}-QV_{t})}{\sqrt{\eta\mu_{1}%
^{-4}QPV_{t}}}\overset{d}{\rightarrow}N(0,1).
\]
and%
\[
\hat{H}=\frac{\delta^{-1/2}}{\sqrt{\eta QPV_{t}/BV_{t}^{2}}}(\frac{\mu
_{1}^{-2}BV_{t}}{QV_{t}}-1)\overset{d}{\rightarrow}N(0,1),
\]
Inference using these tests is straightforward, as both test statistics have limiting
standard normal distributions. Clearly, the ratio $\frac{\int_{0}^{t}%
\sigma_{u}^{4}du}{\mu_{1}^{-4}BV^{2}}\geq1/t$, and Barndorff-Nielsen and
Shephard (2006a) suggest replacing $\hat{H}$ by the adjusted ratio jump test
\begin{equation}
\hat{J}=\frac{\delta^{-1/2}}{\sqrt{\eta max(t^{-1},\frac{QPV_{t}}{BV_{t}^{2}%
})}}\big(\frac{\mu_{1}^{-2}BV_{t}}{QV_{t}}-1\big)\overset{d}{\rightarrow
}N(0,1).
\end{equation}


Huang and Tauchen (2005), Andersen, Bollerslev, and Diebold (2007) analyze the
statistical properties of bipower variation based jump tests using S\&P index
data, exchange rates, and bond yields; as well as via Monte Carlo simulation.
They suggest using a daily statistic, $z_{TP,t},$ to test for jumps on a daily
basis, where
\begin{equation}
z_{TP,t}=\frac{RV_{t}-BV_{t}}{\sqrt{(v_{bb}-v_{qq})\frac{1}{N}TP_{t}}%
}\overset{d}{\rightarrow}N(0,1),
\end{equation}
with $v_{qq}=2$, $v_{bb}=(\frac{\pi}{2})^{2}+\pi-3$, Here, realized tripower
quarticity (TP) is defined and estimated as follows:
\begin{equation}
TP_{t}=\delta^{-1}\mu_{4/3}^{-3}\sum_{j=3}^{1/\delta}|r_{i-2,t}|^{4/3}%
|r_{i-1,t}|^{4/3}|r_{i,t}|^{4/3}\overset{\mathbb{P}}{\rightarrow}\int_{0}^{t}\sigma
_{u}^{4}du
\end{equation}


Additionally, the asymptotic covariance of
\[
\delta^{-1/2}\big(%
\begin{array}
[c]{l}%
RV_{t}-\int_{0}^{t}\sigma_{u}^{2}du\\
BV_{t}-\int_{0}^{t}\sigma_{u}^{2}du\\
\end{array}
\big)
\]
is $\Pi\int_{0}^{t}\sigma_{u}^{4}du$, where
\begin{align*}
\Pi &  =\bigg(%
\begin{array}
[c]{ll}%
Var(u^{2}) & 2\mu_{1}^{-2}Cov(u^{2},|u||u^{\prime}|)\\
2\mu_{1}^{-2}Cov(u^{2},|u||u^{\prime}|) & \mu_{1}^{-4}(Var(|u||u^{\prime
}|)+2Cov(|u||u^{\prime}|,|u^{\prime}||u^{\prime\prime}|))\\
&
\end{array}
\bigg)\\
&  =\bigg(%
\begin{array}
[c]{ll}%
v_{qq} & v_{qb}\\
v_{qb} & v_{bb}%
\end{array}
\bigg)
\end{align*}
with $v_{qb}=2$. Inference is carried out by rejecting the null of no jumps if
$z_{TP,t}$ exceeds the critical value, $\Phi_{\alpha}$, leading to a conclusion
that there are jumps during the day. A common choice for the critical value is 1.96, equivalent to 5\% significant level.

Lee and Mykland (2008) focus on detecting jump at time $t$ without assuming
that there are (or are not) jumps before or after time $t$. Their objective is
to detect jumps over time. The main idea behind Lee and Mykland (2008) centers
around the difference between observed high returns caused by jumps and by
spot volatility. They standardize the return using instantaneous volatility
$\sigma(ti)$, which only includes the local variance from the continuous part
of the process. The instantaneous volatility is consistently measured using
realized bipower variation. The test statistic that they propose is
constructed as follows.
\[
LM(ti)=\frac{r_{i,t}}{\widehat{\sigma_{i,t}}},
\]
where
\[
\widehat{\sigma_{i,t}}=\frac{1}{K-2}\sum_{j=i-K+2}^{i-1}|r_{i,t}||r_{i-1,t}|,
\]
and $K$ is the window size of a local movement of the process, and is chosen
so that the effect of jumps on the volatility estimator disappears. They
suggest to choosing $K=10,$ when sampling at a 5-minute frequency.
Asymptotically, $LM(ti)$ follows a normal distribution. Namely:
\[
\sqrt{\frac{2}{\pi}}LM(ti)\overset{d}{\rightarrow}N(0,1).
\]


\subsubsection{Swap Variance Based Tests}

Inspired by the comparison between bipower variation and realized variance, as
proposed in Barndorff-Nielsen and Shephard (2004, 2006), Jiang and Oomen
(2008) propose comparing a jump sensitive variance measure and the realized
variance. Their idea comes from a well known observation about market
microstructure noise in the finance literature. Namely, in the absence of
jumps the accumulated difference between the simple return and the log return
captures one half of the integrated variance in the continuous-time limit.
Since this relation is the foundation of a variance swap replication strategy,
the accumulated difference between simple returns and log returns is called
the swap variance. They compare this value to the realized variance in order
to test for jumps.

Intuitively, when jumps are absent, the difference between the swap variance
and the realized variance should be indistinguishable from zero, while when
jumps are present, it will reflect the replication error of the variance swap,
which leads to jump detection. The swap variance is defined as:
\[
SwV_{t}=2\sum_{i=1}^{t/\delta}(R_{i,t}-r_{i,t}),
\]
where $R_{i,t}=\frac{P_{i,t}}{P_{i-1,t}}-1$, and $r_{i,t}=Y_{i,t}-Y_{i-1,t}$.

Three types of swap variance jump tests are developed by these authors.
Namely, they propose \newline the difference test
\[
\frac{t/\delta}{\sqrt{\Omega_{SwV}}}(SwV_{t}-RV_{t})\overset{d}{\rightarrow
}N(0,1),
\]
the logarithmic test
\[
\frac{BV\ast N}{\sqrt{\Omega_{SwV_{t}}}}(lnSwV_{t}-lnRV_{t}%
)\overset{d}{\rightarrow}N(0,1),
\]
and the ratio test
\[
\frac{BV\ast N}{\sqrt{\Omega_{SwV_{t}}}}(lnSwV_{t}-lnRV_{t}%
)\overset{d}{\rightarrow}N(0,1),
\]
where $\Omega_{SwV_{t}}=\frac{\mu_{6}}{9}\frac{N^{3}\mu_{6/s}^{-s}}{N-s+1}%
\sum_{i=0}^{N-s}\prod_{k=1}^{s}|r_{i+1}|^{6/s}$, $N=t/\delta$, and $\mu
_{s}=E(|x|^{s})$ for $x\sim N(0,1)$. Setting $s$ equal to either $4$ or $6$
(as a robust estimation of $\Omega_{SwV_{t}})$ is recommended.

Jiang and Oomen (2008) provide Monte Carlo simulation evidence that their SwV
test is more sensitive to jumps than the bipower variation tests discussed
above, but the requirement of estimating the sixticity can be challenging in
practice. They also provide a useful discussion of jumps when the sampling
frequency is ultra-high and market microstructure noise needs to be taken into
consideration when testing for jumps.

\subsubsection{Truncated Power Variation Tests}

The truncated $s$th realized power variation as defined in A\"{\i}t-Sahalia
and Jacod (2012) is expressed as follows.
\[
B(s,u,\delta)=\sum_{i=1}^{t/\delta}|r_{i,t}|^{s}I_{\{|r_{i,t}|\leq u\}}.
\]
Here, the truncation level $u$ is set equal to $b\delta^{\omega},$ for some
constant $\omega\in(0,1/2),$ with $b>0$, which results in $u$ shrinking to 0.
As above, $\delta$ is the sampling frequency. In this framework, $\omega<1/2$
ensures that all increments \textquotedblleft mainly" contain a Brownian
contribution. Note, when $u$ is set to infinity, the truncated realized power
variation becomes $B(s,\infty,\delta),$ in which case no truncation is applied.

When $\delta\rightarrow0$, $B(s,\infty,\delta)$ converges in probability as
follows.
\[
\left\{
\begin{array}
[c]{ll}%
s>2\text{ all }Y_{t} & \Rightarrow B(s,\infty,\delta)\overset{\mathbb{P}}{\rightarrow} J(s)\\
\text{all }s\text{ on }\Omega_{T}^{c} & \Rightarrow\frac{\delta^{1-s/2}}%
{\mu_{1}^{s}}B(s,\infty,\delta)\overset{\mathbb{P}}{\rightarrow}\int_{0}^{t}|\sigma
_{u}|^{s}d_{u}%
\end{array}
\right.
\]
where $\mu_{1}^{s}$ is the $s$th absolute moment of a standard normal random
variable, and $\Omega_{T}^{c}$ =\{$Y$ is continuous in $[0,T]\}$ is a set
defined pathwise on $[0,T]$. Also, define $\Omega_{T}^{W}$ =\{$Y$ has a Wiener
component in $[0,T]\},$ and $\Omega_{T}^{J}$ =\{$Y$ has jumps in
$[0,T]\},$which are additional sets defined pathwise on $[0,T]$. They
recommend using the following test statistic:
\[
AJ(s,k,\delta)=\frac{B(s,\infty,k\delta)}{B(s,\infty,\delta)},
\]
where $s>2$, and $k>2$ is an integer that controls the sampling frequency.
These authors show that:
\[
AJ(s,k,\delta)\rightarrow\left\{
\begin{array}
[c]{ll}%
1 & \text{ on }\Omega_{T}^{J}\\
k^{s/2-1} & \text{ on }\Omega_{T}^{c}\cap\Omega_{T}^{W}%
\end{array}
\right.
\]
$\Omega_{T}^{c}\cap\Omega_{T}^{W}$ means $Y_{t}$ is continuous and has a
Wiener component in $[0,T]$.

Thus, when jumps are present, the variation converges to a finite limit and so
the ratio, $AJ(s,k,\delta),$ tends to 1, while when there are no jumps, the
variation converges to 0, and so $AJ(s,k,\delta)$ tends to a limit that is
greater than 1, and depends on the choice of $k$. Essentially, this test
compares the estimator of integrated variance using different sampling
frequencies, and is motivated by the fact that sampling frequency should have
no influence on the estimator when there are jumps.

Lee and Hanning (2010) also utilize truncated power variation, and develop a
related test for jump detection that is robust to infinite activity jumps.
Their test is quite similar to the test developed by Lee and Mykland (2008),
although the Lee and Mykland test is designed to have power against
Poisson-type (finite activity) jumps. Namely, they propose using:
\[
LH(ti)=\frac{r_{i,t}}{\hat{\sigma}_{i,t}\delta^{1/2}}\overset{d}{\rightarrow
}N(0,1),
\]
with
\[
\hat{\sigma}_{i,t}^{1/2}=\frac{\delta^{-1}}{K}\sum_{j=i-K}^{i-1}%
r_{j-m+1,t}^{2}I_{\{|r_{j-m+1,t}|\leq g\delta^{\omega}\}}%
\]
where $\delta$ is the sampling frequency, $g>0$, $0<\omega<1/2$, and $K$ is
the window size, which is usually set to be $b\delta^{c},$ with $-1<c<0$, and
$b$ a constant. \ As recommended by the Lee and Manning, $g=1.2$,
$\omega=0.47$, $K=b\delta^{c}$ with $-1<c<0$ for some constant $b$.

\subsubsection{ Long Time Span Jump Tests}

Building on the work by A\"{\i}t-Sahalia (2002, 2012), Corradi, Silvapulle,
and Swanson (2018) construct a jump test to detect jumps in the data by
examining the intensity parameter in the data generating process. In
particular, they develop a jump test\ for the null hypothesis that the
probability of a jump is zero. Their test is based on realized third moments,
and uses observations over an increasing time span. The test offers an
alternative to the standard finite time span jump tests discussed above, and
is designed to detect jumps in the data generating process rather than
detecting realized jumps over a fixed time span. They also provide a test for
self-excitement (i.e., is the intensity parameter constant or does the
intensity follow a Hawkes diffusion process (as discussed in Andersen,
Benzoni, and Lund (2002), A\"{\i}t-Sahalia, Cacho-Diaz, and Laeven (2015)).

\bigskip Let%
\begin{align}
\widehat{\mu}_{3,T,\delta} &  =\frac{1}{T}\sum_{k=1}^{n-1}\left(
Y_{(k+1)\delta}-Y_{k\delta}-\frac{Y_{n\delta}-Y_{\delta}}{n}\right)
^{3}\nonumber\\
&  -\frac{1}{T^{+}}\sum_{k=1}^{n^{+}-1}\left(  Y_{(k+1)\delta}-Y_{k\delta
}-\frac{Y_{n^{+}\delta}-Y_{\delta}}{n^{+}}\right)  ^{3}1\left\{  \left\vert
Y_{(k+1)\delta}-Y_{k\delta}\right\vert \leq\tau\left(  \delta\right)
\right\}  ,\label{mu3}%
\end{align}
where $\tau\left(  \delta\right)  $ is a truncation parameter, $\delta$ is the
sampling frequency, $T$ and $T^{+\text{ }}$are time spans (with $T^{+}%
/T\rightarrow\infty${\small )} , and $n = \frac{T}{\delta}$ and $n^{+\text{ }}$are analogously
defined, but denote the number of observations, as discussed in CSS (2018).
Now, define the statistic for testing no null of no jumps as follows:%
\begin{equation}
S_{T,\delta}=\frac{T^{1/2}}{\delta}\widehat{\mu}_{3,T,\delta}%
\overset{d}{\rightarrow}N\left(  0,\omega_{0}\right)  .\label{St}%
\end{equation}

\noindent where $\omega_{0}$ is defined in CSS (2018).

The test has power not only against constant and self-exciting intensity, but
also against affine jump diffusions where the intensity is an affine function
of volatility, for example. As the variance of the statistic is of larger
order under the alternative of positive jump intensity, one cannot construct a
variance estimator which is consistent under all hypotheses. Thus, the authors
construct an estimator for the variance of $S_{T,\delta}$ which is consistent
under the null of no jumps and bounded in probability under the (union of)
alternatives. This is done by using a threshold variance estimator, which
filters out the contribution of the jump component. In particular, define:%
\begin{align}
&  \widehat{\sigma}_{\lambda,T,\delta}^{2}\nonumber\\
&  =\frac{1}{T\delta^{2}}\sum_{k=0}^{n-1}\left(  Y_{(k+1)\delta}-Y_{k\delta}-\frac{Y_{n\delta}-Y_{\delta}}{n}\right)  ^{3}I\left\{
\left\vert Y_{(k+1)\delta}-Y_{k\delta}\right\vert \leq\tau\left(
\delta\right)  \right\}  .\label{jsig2}%
\end{align}
It follows that the $t$-statistic version of this jump test is,
\[
t_{\lambda,T,\delta}=\frac{S_{T,\delta}}{\widehat{\sigma}_{\lambda,T,\delta}}.
\]

\subsection{Jump Decompositions}
In our empirical application, we utilize the jump decomposition methods
discussed in A\"{\i}t-Sahalia and Jacod (2012) in order to decompose quadratic
variation into continuous components and jump components. Furthermore, we
consider large jump and small jump components, as discussed above. When
considering truncated $s$th realized power variation, if the power, $s<2,$
then the continuous component in the process dominates, while if $s>2$ then
the jump component dominates. When $s=2$ both components have equal influence
on the process. Thus, we can obtain important information about quadratic
variation by decomposing realized power variation into continuous and jumps
components, as follows.%
\begin{equation}%
\begin{tabular}
[c]{c}%
Percentage of total QV due to continuous component ($QVC$) = $\frac
{B(2,u,\delta)}{B(2,\infty,\delta)}$\\
Percentage of total QV due to jump component ($QVJ$) = $1-\frac{B(2,u,\delta
)}{B(2,\infty,\delta)}$%
\end{tabular}
\end{equation}
In our empirical section, we use the value of $u$ used in code available from
A\"{\i}t-Sahalia and Jacod (2012). We denote the variation due to jumps (i.e.,
increments \textquotedblleft larger\textquotedblright\ than $u)$ as:
\begin{align*}
U(s,u,\delta) &  =\sum_{i=1}^{t/\delta}|r_{i,t}|^{s}I_{\{|r_{i,t}|>u\}}\\
&  =B(s,\infty,\delta)-B(s,u,\delta)
\end{align*}
Jump decompositions based on this metric can be calculated as:
\begin{equation}%
\begin{tabular}
[c]{c}%
Percentage of QV due to large jump component ($QVJL$) = $\frac{U(2,\epsilon
,\delta)}{B(2,\infty,\delta)}$\\
Percentage of QV due to small jump component ($QVJS$) = $\frac{B(2,\infty
,\delta)-B(2,u,\delta)-U(2,\epsilon,\delta)}{B(2,\infty,\delta)}$%
\end{tabular}
\end{equation}
The large jump cut-off level is $\epsilon=b\delta^{\omega},$ which is
arbitrarily chosen, by experimenting with multiple values of $\epsilon
$.\footnote{Recall that $u$ is set equal to $b\delta^{\omega}$. In our
calculations, we set $b=2$ when calculating $u.$} In our analysis, we set
$b=3$ and $b=5$. We consider the following variations: $QVJ$, $QVJL3$, $QVJL5$,
$QVJS3$, and $QVJS5$ (where the \textquotedblleft3\textquotedblright\ and
\textquotedblleft5\textquotedblright\ values correspond to the values of $b$
that we utilized in our empirical analysis).

\section{Empirical Methodology}

Two experiments are conducted in this paper. In the first experiment,
\textquotedblleft jump spillover effects" are examined by carrying out a
regression analysis in which the causal linkages between quadratic jump
variations in nine SPDR sector ETFs (see Section 4.1 for complete details) are
examined. In the second experiment, causal linkages between excess returns
from each of the sectors that we examine and jump variations from all nine
sectors are examined. Excess returns are defined to be the difference between
daily log-returns of an asset and the daily log-returns of the market. We use
an ETF based on S\&P500 called SPY to obtain the log-returns of the market.

We adopt the year over year (YoY) method from finance to compare our results,
which means results are compared based on each calendar year. More
specifically, for each experiment we fit vector autoregression (VAR) models
for each calendar year. Moreover, we categorize our analysis by jump types
(total jumps, large jumps, small jumps), as discussed above. To summarize,
there are five jump types ($QVJ$, $QVJL3$, $QVJL5$, $QVJS3$, $QVJS5$), nine
market sectors, and six calendar years in our dataset. Thus, we have 270
models for each experiment.

Table 1 summarizes the experimental setup used in this paper. First, we run
$\hat{J}$ tests for each trading day in our sample, and record the dates when
we reject the null of no jumps. Second, we use the methods described in
Section 2.3 in order to obtain $QVJ$,$QVJL3$, $QVJL5$,$QVJS3$ and $QVJS5$ on
trading days when we reject the null. On days when we do not reject the null,
$QVJ=QVJL3=QVJL5=QVJS3=QVJS5=0,$ as no jumps are present. Finally, we conduct
regression analysis for each calender year using daily data and the two VAR
models described below.

\subsection{Modeling Jump Spillover Effects}

Jump spillover effects measure whether or not jumps in a given sector
(Granger) cause jumps in other sectors. In our empirical experiment, we fit a
linear VAR model to test for such effects. In our tabulated results (i..e,
Tables 3 and 4), we collect coefficients on jumps variables in a given sector
that are significantly different from zero at a 95\% level of confidence
(based on application of $t$-tests), take the absolute value of these, and
report the sum thereof, of each regression in our VAR. This sum represents
jump spillover effects of a given sector on one of the other sectors. The VAR
model that we fit is the following:
\[
\left[
\begin{array}
[c]{c}%
Sector_{1,t,h}=\beta_{1,0,h}+\sum_{j=1}^{9}\sum_{k=1}^{k=22}\beta
_{1,j,k,h}Sector_{j,t-k,h}+\epsilon_{1,t,h}\\
Sector_{2,t,h}=\beta_{2,0,h}+\sum_{j=1}^{9}\sum_{k=1}^{k=22}\beta
_{2,j,k,h}Sector_{j,t-k,h}+\epsilon_{2,t,h}\\
Sector_{3,t,h}=\beta_{3,0,h}+\sum_{j=1}^{9}\sum_{k=1}^{k=22}\beta
_{3,j,k,h}Sector_{j,t-k,h}+\epsilon_{3,t,h}\\
Sector_{4,t,h}=\beta_{4,0,h}+\sum_{j=1}^{9}\sum_{k=1}^{k=22}\beta
_{4,j,k,h}Sector_{j,t-k,h}+\epsilon_{4,t,h}\\
Sector_{5,t,h}=\beta_{5,0,h}+\sum_{j=1}^{9}\sum_{k=1}^{k=22}\beta
_{5,j,k,h}Sector_{j,t-k,h}+\epsilon_{5,t,h}\\
Sector_{6,t,h}=\beta_{6,0,h}+\sum_{j=1}^{9}\sum_{k=1}^{k=22}\beta
_{6,j,k,h}Sector_{j,t-k,h}+\epsilon_{6,t,h}\\
Sector_{7,t,h}=\beta_{7,0,h}+\sum_{j=1}^{9}\sum_{k=1}^{k=22}\beta
_{7,j,k,h}Sector_{j,t-k,h}+\epsilon_{7,t,h}\\
Sector_{8,t,h}=\beta_{8,0,h}+\sum_{j=1}^{9}\sum_{k=1}^{k=22}\beta
_{8,j,k,h}Sector_{j,t-k,h}+\epsilon_{8,t,h}\\
Sector_{9,t,h}=\beta_{9,0,h}+\sum_{j=1}^{9}\sum_{k=1}^{k=22}\beta
_{9,j,k,h}Sector_{j,t-k,h}+\epsilon_{9,t,h}\\
\end{array}
\right]
\]
where $Sector_{i,t,h}$ is the variation of the jump component of the $i$th
market sector at time $t$ in year $h$, with $i=1,...,9$ representing our nine
market sectors. $Sector_{j,t-k,h}$ is the $k^{th}$ lagged variation of the
jump component of the $j^{th}$ market sector in year $h,$ with $j=1,...,9$
representing nine market sectors. Here,  $h=2005,...,2010$ denotes the
calendar year. Variations used as regressors in the above model are
$QVJ$,$QVJL3$, $QVJL5$,$QVJS3$ and $QVJS5$. $\beta_{i,0,h}$ is the intercept for
market sector $i$ in year $h$. $\beta_{i,j,k,h}$ denotes the coefficient on
the $k^{th}$ lagged jump in sector $j,$ in the regression of the $i^{th}$
sector in year $h$. Clearly, the $\beta$s quantify the causal or spillover
effects for a given year. The number of lags is chosen based on use of the
Akaike Information Crietion (AIC). Additionally, we believe that jump
spillover effects can last for a long period, and in particular at least one
month (i.e., 22 trading days). Our use of the AIC confirms our choice (i.e.,
we find that $k=22$ for most sectors). Augmented Dickey-Fuller tests were conducted to ensure
that variables are stationary. Maximum likelihood is used to estimate the
model. As discussed above, jump spillover effects of market sector $j$ on
market sector $i$ ($j \ne i$) is calculated as $\sum_{k=1}^{k=22}|\beta_{i,j,k,h}^{\ast}|$,
where $|\beta_{i,j\neq i,k,h}^{\ast}|$ is set to zero if not significantly different from zero based on application of a 5\% level $t$-test. The total of jump spillover effects from market sector $j$ in year $h$ is then $\sum_{i}\sum_{k=1}^{k=22}|\beta_{i,j,k,h}^{\ast}|$, and $j \ne i$.

Of note is that there are a total of 199 parameters in each equation of the VAR model discussed above. This does not pose a problem in our empirical analysis, since the number of daily observations used in each of the yearly regressions that we estimate is much greater than 199. However, informative interpretation of individual coefficient magnitudes in our analysis is not feasible, given multicollinearity across regressors, and given the sheer number of regressors. For this reason, our interpretation of these regressions is based on aggregation of coefficient magnitudes, as discussed above. Another approach to this problem is to utilize 
machine learning, dimension reduction, and shrinkage methods in order to reduce the dimension of the set of regressors used in the equations in the above VAR. Refer to Kim and Swanson (2014, 2018) for further discussion and references. This is left to future research.

\subsection{Modeling Jump Contributions to Excess Returns}

Our assessment of jump risk in excess returns measures the impact of jumps on
excess returns of an market sector return. As done above, we fit a linear VAR
model in order to quantify jump risk. Our tabulated results are presented in
the same fashion as results based on our jump spillover effect analysis. The
VAR model is also the same, except that dependent variables are now excess
market sector returns rather than jump variations.
\[
\left[
\begin{array}
[c]{c}%
SectorEX_{1,t,h}=\beta_{1,0,h}+\gamma_{1,1,h}SectorEX_{1,t-1,h}+\sum_{j=1}%
^{9}\sum_{k=0}^{k=22}\beta_{1,j,k,h}Sector_{j,t-k,h}+\epsilon_{1,t,h}\\
SectorEX_{2,t,h}=\beta_{2,0,h}+\gamma_{2,1,h}SectorEX_{2,t-1,h}+\sum_{j=1}%
^{9}\sum_{k=0}^{k=22}\beta_{2,j,k,h}Sector_{j,t-k,h}+\epsilon_{2,t,h}\\
SectorEX_{3,t,h}=\beta_{3,0,h}+\gamma_{3,1,h}SectorEX_{3,t-1,h}+\sum_{j=1}%
^{9}\sum_{k=0}^{k=22}\beta_{3,j,k,h}Sector_{j,t-k,h}+\epsilon_{3,t,h}\\
SectorEX_{4,t,h}=\beta_{4,0,h}+\gamma_{4,1,h}SectorEX_{4,t-1,h}+\sum_{j=1}%
^{9}\sum_{k=0}^{k=22}\beta_{4,j,k,h}Sector_{j,t-k,h}+\epsilon_{4,t,h}\\
SectorEX_{5,t,h}=\beta_{5,0,h}+\gamma_{5,1,h}SectorEX_{5,t-1,h}+\sum_{j=1}%
^{9}\sum_{k=0}^{k=22}\beta_{5,j,k,h}Sector_{j,t-k,h}+\epsilon_{5,t,h}\\
SectorEX_{6,t,h}=\beta_{6,0,h}+\gamma_{6,1,h}SectorEX_{6,t-1,h}+\sum_{j=1}%
^{9}\sum_{k=0}^{k=22}\beta_{6,j,k,h}Sector_{j,t-k,h}+\epsilon_{6,t,h}\\
SectorEX_{7,t,h}=\beta_{7,0,h}+\gamma_{7,1,h}SectorEX_{7,t-1,h}+\sum_{j=1}%
^{9}\sum_{k=0}^{k=22}\beta_{7,j,k,h}Sector_{j,t-k,h}+\epsilon_{7,t,h}\\
SectorEX_{8,t,h}=\beta_{8,0,h}+\gamma_{8,1,h}SectorEX_{8,t-1,h}+\sum_{j=1}%
^{9}\sum_{k=0}^{k=22}\beta_{8,j,k,h}Sector_{j,t-k,h}+\epsilon_{8,t,h}\\
SectorEX_{9,t,h}=\beta_{9,0,h}+\gamma_{9,1,h}SectorEX_{9,t-1,h}+\sum_{j=1}%
^{9}\sum_{k=0}^{k=22}\beta_{9,j,k,h}Sector_{j,t-k,h}+\epsilon_{9,t,h}\\
\end{array}
\right]
\]
where $SectorEX_{i,t,h}$ is the excess return of the $i^{th}$ market sector at
time $t$ in year $h$, and other variables and coefficients are discussed above. The jump contribution level of market sector $j$ on excess returns of
market sector $i$ is calculated as $C\sum_{k=0}^{k=22}|\beta_{i,j,k,h}^{\ast}|$,
where $|\beta_{i,j,k,h}^{\ast}|$ is set to zero if not significantly different from zero based on application of a 5\% level $t$-test and $C$ is a constant to adjust the contribution level, because the $\beta$s are very close to zero. The total jump contribution level of market sector $j$ in year $h$ is then $C\sum_{i}\sum_{k=0}^{k=22}|\beta_{i,j,k,h}^{\ast}|$, where $C=10^{17}$.

\section{Empirical Results}

\subsection{Data}

We obtain daily millisecond trading data for the period January 2005 -
December 2010 from the TAQ database through the Wharton Research Data Services
portal. To reduce the micro-structure noise effects, we follow convention and
choose a sampling frequency of 5 minutes, which yields roughly 78 observations
per day. When there is no price at an exact time stamp, we use the closest one available.

Our dataset consists of nine SPDR market sector ETFs. These nine sector ETFs
are XLY (consumer discretionary sector), XLP (consumer staples sector), XLE
(energy sector), XLF (financials sector), XLV (health care sector), XLI
(industrials sector), XLB (materials sector), XLK (technology sector), and XLU
(utilities sector). According to the SPDR website, XLY includes companies from
industries like: media, retail (specialty, multiline, internet and catalog),
hotels, restaurants and leisure, textiles, apparel and luxury goods, household
durables, automobiles, auto components, distributors, leisure products, and
diversified consumer services. XLP includes food and staples, retailing,
household products, food products, beverages, tobacco, and personal products.
XLE includes companies in oil, gas and consumable fuels, and energy equipment
and services. XLF includes diversified financial services, insurance, banks,
capital markets, mortgage real estate investment trusts (REITs), consumer
finance, and thrifts and mortgage finance. XLV includes companies in
pharmaceuticals, health care equipment and supplies, health care providers and
services, biotechnology, life sciences tools and services, and health care
technology. XLI includes a wide range of industries, such as aerospace and
defense, industrial conglomerates, marine, transportation infrastructure,
machinery, road and rail, air freight and logistics, commercial services and
supplies, professional services, electrical equipment, construction and
engineering, trading companies and distributors, airlines, and building
products. XLB includes a collection of companies in chemicals, metals and
mining, paper and forest products, containers and packaging, and construction
materials. XLK includes companies in technology hardware, storage, and
peripherals, software, diversified telecommunication services, communications
equipment, semiconductors and semiconductor equipment, internet software and
services, IT services, electronic equipment, instruments and components, and
wireless telecommunication services. Finally, XLU includes companies in
electric utilities, water utilities, multi-utilities, independent power
producers and energy traders, and gas utilities. In 2015, SPDR launched a new
ETF targeting real estate management and development and REITs, excluding
mortgage REITs, but since our analysis is between 2005 and 2010, we exclude
this new sector ETF from our data set.

For our the excess return calculations, we downloaded the S\&P 500 index based
ETF (SPY) and the nine market sector ETFs from \textit{Yahoo Finance} at a
daily frequency for the period January 2005 - December 2010.

\subsection{Empirical Findings}

See Sections 3.1 and 3.2 for a discussion of our empirical setup. As discussed
in that section, tabulated results in Tables 3 and 4 collect coefficients on
jumps variables in a given sector that are significantly different from zero
at a 95\% level of confidence (based on application of $t$-tests), take the
absolute value of these, and report the sum thereof, for each regression in
our VAR.\footnote{Complete regression findings are available upon request, and
are omitted here for the sake of brevity.} Thus, for each of our 9 market
sectors one can assess the impact each of the other 8 sectors has on that
sector. Our results based on $QVJL5$ and $QVJS5$ were found to be
un-informative, so that tabulated results are presented only for regressions
that include $QVJ$, $QVJL3$ and $QVJS3$ in this paper.\footnote{Complete
results are available upon request.} This is not surprising, given the
findings presented in Table 2, where it can be seen that $QVJL5$ is often 0,
suggesting that the cut-off level used in the calculation of $QVJL5$ is not
informative.\footnote{Table 2 only contains results for the first 6 weeks in
2005. Similar results were found when constructing $QVJL5$ for the the rest of
2005, and for other calendar years in our sample. } Interestingly, Table 2
also indicates that jumps can either contribute as much as 80\% of quadratic
variation or as little as 20\% on a given trading day. This suggests that
market sectors are frequently beset by shocks that cause jumps. However, it
should be noted that, large jumps usually dominate the quadratic variation,
with a few exceptions, such as on January 11 and February 8.

Before turning to our discussion of the results in Tables 3 and 4, note that
Figures 1 - 3 plot jump spillover effects by sector, by year, based on Table
3. Examination of these figures indicates that there are jump spillovers
across all sectors, broadly speaking. Interestingly, total jump spillovers was
greater in 2005, 2006, and 2007 than in 2008, large jump spillovers were
greatest in 2006, and small jump spillovers peaked in 2008. This suggests that
transmission of jumps of different magnitudes across sectors is asymmetric,
and dependent upon the state of the economy. Finally, notice that there were
no years where total jump spillover effects were notably fewer
than in other years. The same is not the case when one examines the
propagation of jumps through excess returns. Figures 4 - 6 plot jump
contribution levels to excess returns, by year, based on Table 4.
Interestingly, even cursory examination of these figures indicates that excess
returns are affected much more significantly by both large and small jumps
during 2008 and 2009, than during any other calendar years in our analysis.
Indeed, large jumps exhibit almost no correlation with excess returns during
2005, 2006, 2007, or 2010; whereas there are significant excess return-jump
spillovers during 2008 and 2009. Thus, the effects of jump variations on
excess returns are a clear indicator of the Great recession, while the same
cannot be said when considering jump spillover effects.

We now turn to a discussion of Tables 3 and 4. A number of clear-cut
conclusions emerge upon inspection of the results in these tables. First,
consider Table 3.

First, large jump spillover effects from each sector seem to coincide with
sector-related major events that happened around that time. For example, XLI,
XLK, and XLP had the strongest large jump spillover effects in 2006, and large
jumps spillovers in XLI and XLP might be related to the volatile housing
market at the time. According to a report published by RealtyTrac, the number
of total foreclosure filings nationwide rose from about 885,000 in 2005 to
1,259,118 in 2006, which is more than 42\% increase. For the same reason, the
large jump spillover effects for XLF in 2006 was quite strong as
well.\footnote{For more details see:
https://www.housingwire.com/articles/us-foreclosure-filings-42-percent-2006
for more details.} In terms of the XLK, 2006 is often called a
\textquotedblleft tech bubble\textquotedblright\ year. For example, Youtube
was sold for \$1.65 billion during 2006. Also, six prominent tech companies
filed their IPOs in 2006, but only one of them was profitable. Moreover, quite
a few tech companies experienced skyrocketing stock prices until early 2006,
and slid dramatically afterwards.\footnote{For more details see:
\par
http://www.nytimes.com/2006/10/10/technology/10deal.html,
\par
http://money.cnn.com/2006/05/16/technology/pluggedin\_fortuneipos0516/index.htm,
\par
and
\par
https://seekingalpha.com/article/308397-we-may-be-nearing-a-third-tech-bubble-collapse}
In 2009, XLF, XLV, and XLY exhibited their most spillovers. Similar news
events can be used to explain many of the other incidences of large spillover effects.

Second, small jump spillover effects are quite different from large jump
effects. Most sectors had their strongest spillover effects between 2007 and
2010. This discrepancy between the large jump case and the small jump case can
perhaps be best interpreted as a result of the different causes of large jumps
and small jumps: large jumps are associated with major news and events, while
small jumps are likely the result of things like high frequency trading and
company specific events.

Third, total jump spillover effects (both large and small) are interesting.
For example, it is worth noting that 2008 was a relatively quiet year for all
sectors as none of the sectors showed the strongest spillover effects in that
year. This may be related to the fact that 2008 was the peak of the recession
and fear dominated the market, which led to liquidity problems (Reavis
(2012)). These issues in turn may have affect the ease with which spillover
effects occurred.

Fourth, a YoY comparison indicates that large jump spillover effects and total
jump spillover effects in the whole U.S. market peaked in 2006, bottomed in
2008. Small jump spillover effects started to increase in 2006, peaked in
2008, dropped in 2009, and rose up again in 2010. This is intriguing, since
2006 was the year of the \textquotedblleft slowdown\textquotedblright.
According to the Center for American Progress, the U.S. economy experienced a
fall in both economic growth and consumption growth in 2006, for the first
time in more than three years, indicating high risks in certain areas in the
market. Figures 1 - 3 illustrate this pattern quite clearly.\footnote{For more
details see: https://www.americanprogress
.org/issues/economy/news/2006/12/21/2420/the-u-s-economy-in-review-2006,
\par
http://money.cnn.com/2008/09/15/markets/markets\_newyork2,
\par
and
\par
{}%
http://www.nytimes.com/2012/05/07/business/stock-trading-remains-in-a-slide-after-08-crisis.html}


Drilling down a bit further, the results in Table 3 does no show jumps from
XLF dominating the spillover effects prior to the recession. This is different
from what we expected, as the financial sector was the main cause of the
recession. What we instead observe is that the large jump spillover effects
and total jump spillover effects peaked in 2006 and bottomed in 2008. This
implies that prior to the Great Recession, the market was more volatile but
not necessarily concentrated only in the financial sector.

Now, consider the results contained in Table 4. Again, a number of clear-cut
conclusion emerge upon inspection of the results in this table.

\qquad First, there were scarcely any large jump and total jump contributions
to excess returns before and after the recession (see Table 4 and Figures 4
and 6). Additionally, large jump contributions were only prevalent during
2008 and 2009. This provides evidence that jumps, especially large jumps
should not be neglected in asset pricing, particularly in a volatile markets.
Second, small jump contribution levels were rather significant across all
sampling years, and became intensified between 2007 and 2009 (see Table 4 and
Figure 5). It is also worth noting that while large and total jump spillover
effects weakened during the recession, the impact of jumps on excess returns
escalated, as discussed above. Finally, and similar to the spillover case, we
do not observe jumps from XLF contributing to excess returns more than jumps
from other sectors.

\section{ Concluding Remarks}

This paper begins with a review of jump testing and variation decomposition
methodology. Thereafter, an empirical analysis is presented in which jump
spillover effects in nine market sectors over a six year period around the
Great Recession of 2008-2009 are examined. Broadly speaking, strong large and
total jump spillover effects (i.e., jumps from one sector (Granger) causing
jumps from another sector) were seen as early as in 2006, and weakened as the
recession unfolded. With small jumps, the opposite occurred. In particular,
2008 was the weakest year for large and total jump spillover effects and
strongest year for small jumps. This can be understood by examining the causes
of jumps of different sizes. Large jump spillover effects seem to correlate
with major news and events, while small jump spillover effects are harder to
interpret and seem more correlated with heterogeneous agent and firm specific
characteristics. \ With regard to the jump contributions to excess returns,
total jump and large jump contributions were close to zero in years other than
2008 and 2009. This provides strong evidence that jumps play an important role
in asset pricing during crisis times.\pagebreak

\section{References}

\noindent A\"{\i}t-Sahalia, Y. (2002). Telling from discrete data whether the
underlying continuous-time model is a diffusion. \textit{The Journal of
Finance}, 57, 2075-2112.

\noindent A\"{\i}t-Sahalia, Y., and Jacod, J. (2008). Fisher's information for
discretely sampled Levy processes. \textit{Econometrica}, 76, 727-761.

\noindent A\"{\i}t-Sahalia, Y., and Jacod, J. (2009a). Testing for jumps in a
discretely observed process. \textit{The Annals of Statistics}, 37, 184-222.

\noindent A\"{\i}t-Sahalia, Y., and Jacod, J. (2009b). Estimating the degree
of activity of jumps in high frequency data. \textit{The Annals of Statistics}, 37, 2202-2244.

\noindent A\"{\i}t-Sahalia, Y., and Jacod, J. (2011). Testing whether jumps
have finite or infinite activity. \textit{The Annals of Statistics}, 39, 1689-1719.

\noindent A\"{\i}t-Sahalia, Y., and Jacod, J. (2012). Analyzing the spectrum
of asset returns: Jump and volatility components in high frequency data.
\textit{Journal of Economic Literature}, 50, 1007-50.

\noindent A\"{\i}t-Sahalia, Y., Cacho-Diaz, J., and Laeven, R. J. (2015).
Modeling financial contagion using mutually exciting jump processes.
\textit{Journal of Financial Economics}, 117, 585-606.

\noindent A\"{\i}t-Sahalia, Y., and Xiu, D. (2016). Increased correlation
among asset classes: Are volatility or jumps to blame, or both?
\textit{Journal of Econometrics}, 194, 205-219.

\noindent Andersen, T. G., Benzoni, L., abd Lund, J. (2002). An empirical
investigation of continuous-time equity return models. \textit{The Journal of
Finance}, 57, 1239-1284.

\noindent Andersen, T. G., and Bollerslev, T. (1998). Answering the skeptics:
Yes, standard volatility models do provide accurate forecasts. \textit{International Economic Review}, 39, 885-905.

\noindent Andersen, T. G., Bollerslev, T., Diebold, F. X., and Vega, C. (2003). Micro effects of macro announcements: Real-time price discovery in foreign exchange. \textit{American Economic Review}, 93, 38-62.

\noindent Andersen, T. G., Bollerslev, T., Diebold, F. X., and Labys, P.
(2003). Modeling and forecasting realized volatility. \textit{Econometrica},
71, 579-625.

\noindent Andersen, T. G., Bollerslev, T., and Diebold, F. X. (2007). Roughing
it up: Including jump components in the measurement, modeling, and forecasting
of return volatility. \textit{The Review of Economics and Statistics}, 89, 701-720.

\noindent Andersen, T. G., Bollerslev, T., Diebold, F. X., and Labys, P.
(2010). Parametric and nonparametric volatility measurement. L.P. A\"{\i}t-Sahalia, Y., Hansen, L. P. (Eds.). (2009). \textit{Handbook of Financial Econometrics}, North-Holland Press, Amsterdam.

\noindent Asgharian, H., and Bengtsson, C. (2006). Jump spillover in
international equity markets. \textit{Journal of Financial Econometrics},
4, 167-203.

\noindent Asgharian, H., and Nossman, M. (2011). Risk contagion among
international stock markets. \textit{Journal of International Money and
Finance}, 30, 22-38.

\noindent Barndorff-Nielsen, O. E., and Shephard, N. (2001). Non-Gaussian
Ornstein--Uhlenbeck-based models and some of their uses in financial
economics. \textit{Journal of the Royal Statistical Society: Series B
(Statistical Methodology)}, 63, 167-241.

\noindent Barndorff-Nielsen, O. E., and Shephard, N. (2002a). Econometric
analysis of realized volatility and its use in estimating stochastic
volatility models. \textit{Journal of the Royal Statistical Society: Series B
(Statistical Methodology)}, 64, 253-280.

\noindent Barndorff-Nielsen, O. E., and Shephard, N. (2002b). Estimating
quadratic variation using realized variance. \textit{Journal of Applied
Econometrics}, 17, 457-477.

\noindent Barndorff-Nielsen, O. E., and Shephard, N. (2003). Realized power
variation and stochastic volatility models. \textit{Bernoulli}, 9, 243-265.

\noindent Barndorff-Nielsen, O. E., and Shephard, N. (2004a). Power and
bipower variation with stochastic volatility and jumps. \textit{Journal of
Financial Econometrics}, 2, 1-37.

\noindent Barndorff-Nielsen, O. E., and Shephard, N. (2006a). Econometrics of
testing for jumps in financial economics using bipower variation.
\textit{Journal of Financial Econometrics}, 4, 1-30.

\noindent Barndorff-Nielsen, O. E., and Shephard, N. (2006b). Impact of jumps
on returns and realised variances: econometric analysis of time-deformed Levy
processes. \textit{Journal of Econometrics}, 131, 217-252.

\noindent Barndorff-Nielsen, O. E., and Shephard, N. (2006c). Power variation
and time change. \textit{Theory of Probability and Its Applications}, 50, 1-15.

\noindent Barndorff-Nielsen, O. E., Graversen, S. E., and Shephard, N.
(2004b). Power variation and stochastic volatility: a review and some new
results. \textit{Journal of Applied Probability}, 41, 133-143.

\noindent Barndorff–Nielsen, O. E., Graversen, S. E., Jacod, J., Podolskij, M., and Shephard, N. (2006). A central limit theorem for realised power and bipower variations of continuous semimartingales. In \textit{From stochastic calculus to mathematical finance} (pp. 33-68). Springer, Berlin, Heidelberg.

\noindent Barndorff-Nielsen, O. E., Shephard, N., and Winkel, M. (2006). Limit
theorems for multipower variation in the presence of jumps. \textit{Stochastic
Processes and Their Applications}, 116, 796-806.

\noindent Bollerslev, T., Law, T. H., and Tauchen, G. (2008). Risk, jumps, and
diversification. \textit{Journal of Econometrics}, 144, 234-256.

\noindent Corradi, V., Silvapulle, M. J., and Swanson, N. R. (2018). Testing
for Jumps and Jump Intensity Path Dependence, \textit{Journal of Econometrics}, forthcoming.

\noindent Evans, K. P. (2011). Intraday jumps and US macroeconomic news
announcements, \textit{Journal of Banking and Finance}, 35, 2511-2527.

\noindent Huang, X. (2015). Macroeconomic news announcements, financial market
volatility and jumps. Mimeo, Working Paper 2015-097, Federal Reserve Board.

\noindent Huang, X., and Tauchen, G. (2005). The relative contribution of
jumps to total price variance. \textit{Journal of Financial Econometrics},
3, 456-499.

\noindent Jawadi, F., Louhichi, W., and Cheffou, A. I. (2015). Testing and
modeling jump contagion across international stock markets: A nonparametric
intraday approach. \textit{Journal of Financial Markets}, 26, 64-84.

\noindent Jiang, G. J., and Oomen, R. C. (2008). Testing for jumps when asset
prices are observed with noise-a "swap variance" approach. \textit{Journal of
Econometrics}, 144, 352-370.

\noindent Jiang, G. J., Lo, I., and Verdelhan, A. (2011). Information shocks,
liquidity shocks, jumps, and price discovery: Evidence from the US treasury
market. \textit{Journal of Financial and Quantitative Analysis}, 46, 527-551.

\noindent Kim, H. H., and Swanson, N. R. (2014). Forecasting financial and macroeconomic variables using data reduction methods: New empirical evidence. \textit{Journal of Econometrics}, 178, 352-367.

\noindent Kim, H. H.,  and Swanson, N. R. (2018). Mining big data using parsimonious factor, machine learning, variable selection and shrinkage methods. \textit{International Journal of Forecasting}, 34, 339-354.

\noindent Lahaye, J., Laurent, S., and Neely, C. J. (2011). Jumps, co-jumps
and macro announcements. \textit{Journal of Applied Econometrics}, 26, 893-921.

\noindent Lee, S. S., and Mykland, P. A. (2008). Jumps in financial markets: A
new nonparametric test and jump dynamics. \textit{Review of Financial
Studies}, 21, 2535-2563.

\noindent Lee, S. S., and Hannig, J. (2010). Detecting jumps from L\`{e}vy
jump diffusion processes. \textit{Journal of Financial Economics}, 96, 271-290.

\noindent Mancini, C. (2004). Estimation of the characteristics of the jumps
of a general Poisson-diffusion model. \textit{Scandinavian Actuarial Journal},
2004, 42-52.

\noindent Mancini, C. (2006). Estimating The Integrated Volatility in
Stochastic Volatility Models with Levy Type Jumps. Working Paper, University of Firenze. 

\noindent Reavis, C. (2012). The Global Financial Crisis of 2008: The Role of
Greed, Fear, and Oligarchs. \textit{MIT Sloan Management Review}, 16, 1-22.

\noindent Todorov, V., and Tauchen, G. (2011). Volatility jumps.
\textit{Journal of Business \& Economic Statistics}, 29, 356-371.

\clearpage
\allowdisplaybreaks


%% tables


\textwidth=6.5in \textheight=9.5in \topmargin=-1.2in \baselineskip=1pt \renewcommand{\baselinestretch}{1.2}

\newpage
%TABLE 1 Jump Intensity Test %%%%%%%%%%%%%%%%%%%%%%%%%%%%%%%%%%%%%%%%%%%%%%%%%%%%%%%%%%%%%%%%%%%%%%%%%%%%%%%%%%%%%%%%%%%%%%
\begin{table}[ptb]
\caption{\textbf{Experimental Setup}}%
\label{tb1}%
\begin{tabular}
[c]{p{4cm}|p{12cm}}\hline\hline
Sample Period: & Jan. 3, 20005 to Dec. 31, 2010\\\hline
Sampling Frequency: & 5 minutes.\\\hline
Regression Estimation Scheme: & VAR estimation with time span equal to one
calender year.\\\hline
Jump types: & Total jumps ($QVJ$), large jumps at cutoff level $b=3$
($QVJL3$), large jumps at cutoff level $b=3$ ($QVJL5$), small jumps at
cutoff level $b=3$ ($QVJS3$), small jumps at cutoff level $b=5$ ($QVJS5$).\\\hline
Evaluation Criterion: & Coefficients are summed that are significant using a
5\% level $t$-test.\\\hline
Step 1: Jump Test & Test for jumps on each trading day during sample
period. For this, the bipower variation based test $z_{TP,t}$ described in Section 2.2.1
is applied with significance level $\alpha$ = 5\%. The null hypothesis is that no jumps
are present.\\\hline
Step 2: Jump Decomposition & For trading days which reject the null in Step 1,
the decomposition method in Section 2.3 is applied to extract $QVJ$, $QVJL3$, $QVJL5$, $QVJS3$, and $QVJS5$ on that day. For trading days for which
the null is not rejected in Step 1, jump quadratic variation is set equal to
0.\\\hline
Step 3a: Jump Spillover Analysis & Fit the model in Section 3.1
by calender year, for different jump types.\\
Step 3b: Jump Contribution to Excess Returns & Ft the model in Section 3.2 by calender year, for different jump types.\\
\hline\hline
\end{tabular}
\end{table}
\newpage

%TABLE 1 Jump Intensity Test %%%%%%%%%%%%%%%%%%%%%%%%%%%%%%%%%%%%%%%%%%%%%%%%%%%%%%%%%%%%%%%%%%%%%%%%%%%%%%%%%%%%%%%%%%%%%%
\begin{table}[ptb]
\caption{\textbf{Disaggregate Quadratic Variation in the XLB Sector} *}%
\label{tb1}%
\centering
\begin{tabular}
[c]{cccccc}\hline\hline
Date & $QVJ$ & $QVJL3$ & $QVJL5$ & $QVJS3$ & $QVJS5$\\\hline
1/3/2005 & 0.4352 & 0.2521 & 0 & 0.1831 & 0.4352\\
1/4/2005 & 0.2594 & 0 & 0 & 0.2594 & 0.2594\\
1/5/2005 & 0.84 & 0.8006 & 0.7301 & 0.0394 & 0.1099\\
1/6/2005 & 0.3481 & 0.1494 & 0 & 0.1987 & 0.3481\\
1/7/2005 & 0.3668 & 0.2417 & 0 & 0.1251 & 0.3668\\
1/10/2005 & 0.5789 & 0.3394 & 0 & 0.2395 & 0.5789\\
1/11/2005 & 0.5809 & 0.2611 & 0 & 0.3198 & 0.5809\\
1/12/2005 & 0.577 & 0.4163 & 0.2563 & 0.1607 & 0.3207\\
1/13/2005 & 0.3918 & 0.2224 & 0 & 0.1694 & 0.3918\\
1/14/2005 & 0.5241 & 0.2987 & 0 & 0.2254 & 0.5241\\
1/18/2005 & 0.6349 & 0.3646 & 0 & 0.2703 & 0.6349\\
1/19/2005 & 0 & 0 & 0 & 0 & 0\\
1/20/2005 & 0.5473 & 0.419 & 0 & 0.1283 & 0.5473\\
1/21/2005 & 0.3142 & 0.1824 & 0 & 0.1318 & 0.3142\\
1/24/2005 & 0.5531 & 0.387 & 0.387 & 0.1661 & 0.1661\\
1/25/2005 & 0.7079 & 0.5185 & 0.5185 & 0.1894 & 0.1894\\
1/26/2005 & 0.4206 & 0.2352 & 0 & 0.1854 & 0.4206\\
1/27/2005 & 0.5888 & 0.3594 & 0.3594 & 0.2294 & 0.2294\\
1/28/2005 & 0.3973 & 0.2494 & 0 & 0.1479 & 0.3973\\
1/31/2005 & 0.4323 & 0.3689 & 0.3689 & 0.0634 & 0.0634\\
2/1/2005 & 0.4831 & 0.3397 & 0.3397 & 0.1434 & 0.1434\\
2/2/2005 & 0.45 & 0.1362 & 0 & 0.3138 & 0.45\\
2/3/2005 & 0 & 0 & 0 & 0 & 0\\
2/4/2005 & 0.3565 & 0.1941 & 0 & 0.1624 & 0.3565\\
2/7/2005 & 0.4154 & 0.1396 & 0 & 0.2758 & 0.4154\\
2/8/2005 & 0.505 & 0 & 0 & 0.505 & 0.505\\
2/9/2005 & 0.7931 & 0.7931 & 0.6389 & 0 & 0.1542\\
2/10/2005 & 0.5243 & 0 & 0 & 0.5243 & 0.5243\\
2/11/2005 & 0.5517 & 0.2765 & 0 & 0.2752 & 0.5517\\
2/14/2005 & 0.3892 & 0.3229 & 0.3229 & 0.0663 & 0.0663\\
2/15/2005 & 0.3822 & 0 & 0 & 0.3822 & 0.3822\\
2/16/2005 & 0.4734 & 0 & 0 & 0.4734 & 0.4734\\
2/17/2005 & 0.629 & 0.4126 & 0.4126 & 0.2164 & 0.2164\\
2/18/2005 & 0 & 0 & 0 & 0 & 0\\\hline\hline
\end{tabular}
\begin{minipage}{1\columnwidth}			
	$^{\ast}$ \footnotesize Notes: This table shows the percentage of quadratic variation (QV) that is due to total jumps, jumps at the $b=3$ cutoff level, and jumps at the $b=5$ cutoff level, for the period January 2005 - March 2005. Similar results for other time periods and market sectors are omitted for the sake of brevity, but are available upon request.
\end{minipage}
\end{table}

\afterpage{
	\centering
	\small
	\captionof{table}{\textbf{Jump Spillover Analysis of Nine SPDR Sector ETFs}}
	\captionof{subtable}{Results Based on Analysis of 2005 Jump Variation Data*}
	\begin{tabular}{c|ccccccccc}
		\hline\hline
		& \multicolumn{9}{c}{Lagged $QVJ$ from} \\		
		$QVJ$&XLB	&XLE&	XLF&	XLI&	XLK&	XLP&	XLU&	XLV&	XLY \\ \cmidrule{2-10}
		XLB&	NA&	0.000&	0.000&	0.440&	0.000&	0.000&	0.000&	0.000&	0.000 \\ 
		XLE&	0.336&	NA&	0.705&	1.474&	1.040&	0.359&	0.731&	0.538&	0.505 \\
		XLF&	2.040&	0.000&	NA&	0.000&	0.000&	0.436&	0.000&	0.000&	1.716 \\
		XLI&	0.472&	0.000&	0.000&	NA&	0.000&	0.000&	0.399&	0.000&	0.873 \\
		XLK&	0.757&	0.000&	0.000&	0.894&	NA&	0.431&	0.000&	0.365&	0.399 \\
		XLP&	0.000&	0.000&	0.000&	0.309&	0.348&	NA&	0.000&	0.000&	0.352 \\
		XLU&	0.000&	0.992&	0.000&	0.000&	0.000&	0.498&	NA&	0.501&	0.575 \\
		XLV&	0.324&	0.478&	0.000&	0.000&	0.000&	0.415&	0.478&	NA&	0.000 \\
		XLY&	0.000&	0.000&	0.000&	0.000&	0.000&	0.362&	0.000&	0.440&	NA \\
		\hline
		& \multicolumn{9}{c}{Lagged $QVJL3$ from} \\
		$QVJL3$&	XLB&	XLE&	XLF&	XLI&	XLK&	XLP&	XLU&	XLV&	XLY  \\ \cmidrule{2-10}
		XLB&	NA&	0.000&	0.000&	0.000&	0.000&	0.469&	0.430&	0.394&	0.000 \\
		XLE&	0.461&	NA&	0.000&	0.000&	0.254&	0.219&	0.000&	0.000&	0.465 \\
		XLF&	0.000&	0.000&	NA&	0.590&	0.000&	0.512&	0.000&	0.000&	0.000 \\
		XLI&	0.808&	1.300&	0.000&	NA&	0.000&	0.609&	0.977&	0.420&	0.000 \\
		XLK&	0.283&	1.020&	0.000&	0.257&	NA&	0.000&	0.317&	0.000&	0.486 \\
		XLP&	0.000&	0.796&	0.000&	0.273&	0.000&	NA&	0.000&	0.664&	0.000 \\
		XLU&	0.000&	0.649&	0.000&	0.390&	0.000&	0.463&	NA&	0.463&	0.664 \\
		XLV&	0.790&	2.259&	0.000&	0.677&	0.000&	1.652&	0.779&	NA&	0.454 \\
		XLY&	0.792&	0.000&	0.911&	0.000&	0.476&	0.866&	0.710&	0.000&	NA \\
		\hline
		& \multicolumn{9}{c}{Lagged $QVJS3$ from} \\
		$QVJS3$&	XLB&	XLE&	XLF&	XLI&	XLK&	XLP&	XLU&	XLV&	XLY \\ \cmidrule{2-10}
		XLB&	NA&	0.502&	0.865&	0.406&	0.443&	0.000&	1.378&	0.941&	0.000 \\
		XLE&	0.000&	NA&	0.703&	0.934&	0.000&	0.338&	0.656&	0.584&	1.032 \\
		XLF&	0.000&	0.000&	NA&	0.000&	0.382&	0.000&	0.000&	0.000&	0.000 \\
		XLI&	1.132&	2.847&	0.325&	NA&	0.394&	1.853&	0.437&	0.383&	0.496 \\
		XLK&	0.000&	0.000&	0.000&	0.000&	NA&	0.000&	0.000&	0.000&	0.000 \\
		XLP&	0.000&	0.525&	0.391&	0.000&	0.347&	NA&	0.415&	0.000&	0.000 \\
		XLU&	0.859&	0.000&	0.754&	1.168&	0.578&	0.748&	NA&	0.000&	0.735 \\
		XLV&	0.000&	0.416&	0.361&	0.000&	0.000&	0.000&	0.000&	NA&	0.365 \\
		XLY&	0.000&	0.000&	0.000&	0.000&	0.330&	0.000&	0.340&	1.229&	NA \\
		\hline\hline
	\end{tabular}
	\begin{minipage}{1\columnwidth}
		$^{\ast}$ \footnotesize Notes: Entries are ``aggregate spillover effects" of lagged jumps from a given sector (see first row of entries) on the jumps in each of the sectors listed in the first column of the table. Aggregate spillover effects are aggregated absolute coefficient magnitudes, summed for statistically significant (at a 5\% level, based on application of $t$-statistics) coefficients on the lags in the VAR associated with the regression pertaining to each sector listed in the first column of the table, for all lags in the regression pertaining to the sector listed in the first row of entries in the table. For further details refer to Section 3.
	\end{minipage}
	\pagebreak
	\captionof{subtable}{Results Based on Analysis of 2006 Jump Variation Data*}
	\begin{tabular}{c|ccccccccc}
		\hline\hline
		& \multicolumn{9}{c}{Lagged $QVJ$ from} \\	
		$QVJ$ &XLB	&XLE&	XLF&	XLI&	XLK&	XLP&	XLU&	XLV&	XLY \\ \cmidrule{2-10}
		XLB&	NA&	0.362&	0.000&	0.347&	0.508&	0.766&	0.929&	1.836&	0.772 \\
		XLE&	0.861&	NA&	0.000&	0.000&	0.000&	0.000&	0.000&	0.402&	0.608 \\
		XLF&	1.233&	0.768&	NA&	0.398&	0.588&	2.179&	0.328&	0.966&	0.780 \\
		XLI&	0.000&	0.338&	0.369&	NA&	0.000&	0.000&	0.819&	0.000&	1.107 \\
		XLK&	1.046&	0.522&	0.806&	0.000&	NA&	1.003&	0.806&	1.117&	1.633 \\
		XLP&	0.000&	0.000&	0.000&	0.000&	0.000&	NA&	0.000&	0.000&	0.000 \\
		XLU&	0.558&	0.000&	0.000&	0.000&	0.343&	0.000&	NA&	0.000&	0.000 \\
		XLV&	0.364&	1.707&	0.000&	0.000&	0.581&	1.679&	0.000&	NA&	0.396 \\
		XLY&	0.839&	0.000&	0.000&	0.281&	0.000&	0.379&	0.350&	0.000&	NA \\
		\hline
		& \multicolumn{9}{c}{Lagged $QVJL3$ from} \\
		$QVJL3$&	XLB&	XLE&	XLF&	XLI&	XLK&	XLP&	XLU&	XLV&	XLY  \\ \cmidrule{2-10}
		XLB&	NA&	0.000&	0.875&	0.404&	1.168&	0.420&	0.431&	0.000&	0.490 \\
		XLE&	0.000&	NA&	1.100&	0.000&	0.000&	0.000&	0.000&	0.000&	0.000 \\
		XLF&	0.000&	1.223&	NA&	1.816&	0.317&	2.329&	0.520&	0.849&	1.359 \\
		XLI&	0.000&	0.000&	0.348&	NA&	0.000&	0.354&	0.000&	0.000&	0.000 \\
		XLK&	0.000&	0.000&	1.028&	0.000&	NA&	0.000&	0.428&	0.000&	0.000 \\
		XLP&	0.000&	0.000&	0.000&	0.557&	0.000&	NA&	0.535&	0.000&	0.499 \\
		XLU&	3.211&	2.889&	2.807&	1.741&	4.453&	3.005&	NA&	1.136&	2.840 \\
		XLV&	0.000&	0.000&	0.000&	0.000&	0.000&	0.000&	0.000&	NA&	0.000 \\
		XLY&	0.291&	0.532&	1.352&	0.320&	0.000&	1.501&	0.000&	0.567&	NA \\
		\hline
		& \multicolumn{9}{c}{Lagged $QVJS3$ from} \\
		$QVJS3$ &	XLB&	XLE&	XLF&	XLI&	XLK&	XLP&	XLU&	XLV&	XLY \\ \cmidrule{2-10}
		XLB&	NA&	0.820&	0.000&	0.823&	0.694&	0.000&	0.703&	0.335&	1.205  \\
		XLE&	0.457&	NA&	0.000&	0.000&	0.964&	0.000&	0.424&	0.000&	1.308 \\
		XLF&	0.882&	0.500&	NA&	0.000&	0.000&	0.343&	0.000&	0.000&	0.000 \\
		XLI&	0.745&	0.000&	0.310&	NA&	0.000&	0.797&	0.000&	0.000&	0.000 \\
		XLK&	0.516&	0.962&	0.787&	0.325&	NA&	0.450&	0.835&	0.325&	0.500 \\
		XLP&	1.181&	0.714&	1.184&	0.391&	0.670&	NA&	0.415&	0.916&	0.405 \\
		XLU&	0.457&	0.000&	0.000&	0.000&	0.337&	0.364&	NA&	0.357&	1.140 \\
		XLV&	0.000&	0.000&	0.000&	0.000&	0.000&	0.000&	0.000&	NA&	0.000 \\
		XLY&	0.000&	0.503&	0.311&	0.000&	0.338&	0.000&	0.363&	0.465&	NA \\
		\hline\hline
	\end{tabular}
	\begin{minipage}{1\columnwidth}
		$^{\ast}$ \footnotesize Notes: See notes to Table 3a.
	\end{minipage}
	\pagebreak		
	\captionof{subtable}{Results Based on Analysis of 2007 Jump Variation Data*}
	\begin{tabular}{c|ccccccccc}
		\hline\hline
		& \multicolumn{9}{c}{Lagged $QVJ$ from} \\	
		$QVJ$ &XLB	&XLE&	XLF&	XLI&	XLK&	XLP&	XLU&	XLV&	XLY \\ \cmidrule{2-10}
		XLB&	NA&	0.000&	0.000&	0.000&	0.370&	0.403&	0.000&	0.000&	0.000  \\
		XLE&	0.458&	NA&	0.000&	0.000&	0.000&	0.360&	0.000&	0.363&	0.888 \\
		XLF&	1.168&	0.459&	NA&	0.334&	1.033&	0.732&	0.000&	0.000&	0.838 \\
		XLI&	1.557&	0.587&	0.000&	NA&	0.000&	0.599&	0.433&	0.902&	0.000 \\
		XLK&	0.000&	0.352&	0.445&	0.357&	NA&	0.465&	0.000&	0.372&	0.511 \\
		XLP&	0.000&	0.583&	0.525&	0.000&	0.481&	NA&	0.000&	0.459&	0.000 \\
		XLU&	1.037&	0.445&	0.544&	0.425&	1.316&	0.947&	NA&	0.000&	1.019 \\
		XLV&	0.468&	1.062&	0.548&	0.000&	0.377&	0.488&	0.000&	NA&	1.007 \\
		XLY&	0.342&	0.437&	0.532&	0.000&	0.346&	1.087&	0.739&	0.000&	NA \\
		\hline
		& \multicolumn{9}{c}{Lagged $QVJL3$ from} \\
		$QVJL3$&	XLB&	XLE&	XLF&	XLI&	XLK&	XLP&	XLU&	XLV&	XLY  \\ \cmidrule{2-10}
		XLB&	NA&	0.609&	0.000&	0.000&	0.000&	0.462&	0.373&	0.000&	0.000 \\
		XLE&	1.402&	NA&	1.075&	1.216&	0.336&	0.428&	0.802&	0.634&	0.788 \\
		XLF&	0.463&	0.000&	NA&	0.000&	0.000&	0.000&	0.470&	0.000&	0.000 \\
		XLI&	0.000&	1.232&	0.000&	NA&	0.815&	0.521&	0.000&	0.000&	0.573 \\
		XLK&	0.787&	0.691&	0.599&	1.037&	NA&	0.000&	0.551&	0.623&	0.000 \\
		XLP&	0.582&	1.412&	0.522&	0.000&	1.015&	NA&	0.446&	0.550&	0.598 \\
		XLU&	0.000&	1.601&	0.452&	0.000&	0.000&	2.060&	NA&	0.000&	0.000 \\
		XLV&	0.000&	0.000&	0.000&	0.000&	0.000&	0.993&	0.469&	NA&	0.000 \\
		XLY&	0.566&	0.994&	0.749&	0.889&	0.441&	0.000&	0.418&	0.321&	NA \\
		\hline
		& \multicolumn{9}{c}{Lagged $QVJS3$ from} \\
		$QVJS3$ &	XLB&	XLE&	XLF&	XLI&	XLK&	XLP&	XLU&	XLV&	XLY \\ \cmidrule{2-10}
		XLB&	NA&	0.827&	0.406&	0.364&	0.977&	0.325&	1.072&	0.364&	0.738  \\
		XLE&	0.429&	NA&	0.000&	0.000&	0.365&	0.406&	0.000&	0.000&	0.000 \\
		XLF&	0.858&	0.877&	NA&	0.248&	1.178&	0.805&	1.207&	1.474&	0.772 \\
		XLI&	0.496&	0.400&	0.545&	NA&	0.000&	0.000&	0.000&	0.000&	0.954 \\
		XLK&	0.514&	1.079&	0.938&	0.958&	NA&	1.591&	0.399&	0.863&	1.792 \\
		XLP&	0.000&	0.000&	0.000&	0.000&	0.000&	NA&	0.000&	0.000&	0.000 \\
		XLU&	1.761&	2.184&	0.441&	0.659&	0.326&	1.228&	NA&	0.840&	0.000 \\
		XLV&	0.444&	0.000&	0.980&	0.385&	0.000&	0.378&	0.000&	NA&	0.000 \\
		XLY&	0.351&	0.000&	0.000&	0.000&	0.317&	0.775&	0.000&	0.000&	NA \\
		\hline\hline
	\end{tabular}
	\begin{minipage}{1\columnwidth}
	$^{\ast}$ \footnotesize Notes: See notes to Table 3a.
	\end{minipage}
	\pagebreak		
	\captionof{subtable}{Results Based on Analysis of 2008 Jump Variation Data*}
	\begin{tabular}{c|ccccccccc}
		\hline\hline
		& \multicolumn{9}{c}{Lagged $QVJ$ from} \\	
		$QVJ$ &XLB	&XLE&	XLF&	XLI&	XLK&	XLP&	XLU&	XLV&	XLY \\ \cmidrule{2-10}
		XLB&	NA&	0.000&	0.000&	0.000&	0.000&	0.000&	0.000&	0.000&	0.000  \\
		XLE&	0.000&	NA&	0.000&	0.000&	0.542&	0.833&	0.000&	0.366&	0.884 \\
		XLF&	0.000&	0.000&	NA&	0.000&	0.000&	0.000&	0.522&	1.086&	0.651 \\
		XLI&	0.000&	0.000&	0.274&	NA&	0.786&	0.000&	0.000&	0.000&	0.000 \\
		XLK&	0.354&	0.332&	0.000&	0.000&	NA&	0.000&	0.000&	0.447&	0.000 \\
		XLP&	0.000&	0.000&	0.392&	0.442&	0.000&	NA&	0.000&	0.000&	0.000 \\
		XLU&	0.000&	0.375&	0.000&	0.000&	0.919&	0.000&	NA&	0.414&	0.000 \\
		XLV&	0.385&	0.000&	0.286&	0.000&	0.000&	0.872&	0.390&	NA&	0.428 \\
		XLY&	0.000&	0.000&	0.000&	0.000&	0.741&	0.674&	0.000&	0.504&	NA \\
		\hline
		& \multicolumn{9}{c}{Lagged $QVJL3$ from} \\
		$QVJL3$&	XLB&	XLE&	XLF&	XLI&	XLK&	XLP&	XLU&	XLV&	XLY  \\ \cmidrule{2-10}
		XLB&	NA&	0.000&	0.000&	0.000&	0.000&	0.470&	0.000&	0.000&	0.000 \\
		XLE&	0.540&	NA&	0.396&	0.433&	0.563&	1.193&	0.390&	0.342&	1.059 \\
		XLF&	0.000&	0.000&	NA&	0.000&	0.000&	0.000&	0.000&	0.000&	0.520 \\
		XLI&	1.028&	0.448&	0.324&	NA&	2.220&	1.131&	1.140&	1.214&	1.910 \\
		XLK&	1.884&	1.740&	1.172&	0.969&	NA&	1.402&	0.361&	0.294&	0.974 \\
		XLP&	0.414&	1.043&	0.000&	1.241&	0.617&	NA&	0.000&	0.000&	0.000 \\
		XLU&	0.374&	0.656&	0.445&	0.000&	1.302&	0.000&	NA&	0.462&	1.139 \\
		XLV&	0.000&	0.000&	0.000&	0.000&	0.000&	0.000&	0.000&	NA&	0.000 \\
		XLY&	0.360&	0.000&	0.324&	0.000&	0.000&	0.000&	0.000&	0.000&	NA \\
		\hline
		& \multicolumn{9}{c}{Lagged $QVJS3$ from} \\
		$QVJS3$ &	XLB&	XLE&	XLF&	XLI&	XLK&	XLP&	XLU&	XLV&	XLY \\ \cmidrule{2-10}
		XLB&	NA&	0.451&	0.875&	0.918&	0.665&	0.797&	0.654&	0.716&	1.377 \\
		XLE&	0.378&	NA&	0.000&	0.468&	0.942&	0.000&	0.000&	0.000&	0.000 \\
		XLF&	0.408&	0.360&	NA&	0.887&	0.334&	0.434&	1.284&	0.000&	0.752 \\
		XLI&	2.109&	0.743&	0.660&	NA&	1.563&	0.825&	1.303&	0.353&	0.302 \\
		XLK&	0.531&	0.000&	0.456&	0.000&	NA&	0.000&	0.000&	0.000&	0.413 \\
		XLP&	0.426&	1.114&	1.363&	0.000&	0.000&	NA&	0.560&	1.791&	0.865 \\
		XLU&	0.494&	0.432&	0.462&	0.489&	0.312&	0.998&	NA&	0.495&	0.399 \\
		XLV&	0.000&	0.469&	0.358&	1.333&	0.000&	2.151&	0.383&	NA&	0.366 \\
		XLY&	0.440&	0.000&	0.000&	0.000&	1.391&	0.000&	0.517&	0.386&	NA \\
		\hline\hline
	\end{tabular}
	\begin{minipage}{1\columnwidth}
	$^{\ast}$ \footnotesize Notes: See notes to Table 3a.
	\end{minipage}
	\pagebreak		
	\captionof{subtable}{Results Based on Analysis of 2009 Jump Variation Data*}
	\begin{tabular}{c|ccccccccc}
		\hline\hline
		& \multicolumn{9}{c}{Lagged $QVJ$ from} \\	
		$QVJ$ &XLB	&XLE&	XLF&	XLI&	XLK&	XLP&	XLU&	XLV&	XLY \\ \cmidrule{2-10}
		XLB&	NA&	1.112&	0.000&	0.000&	0.000&	0.000&	0.000&	0.684&	0.419 \\
		XLE&	0.726&	NA&	0.000&	0.545&	0.000&	0.900&	0.474&	0.830&	0.697 \\
		XLF&	1.000&	0.000&	NA&	0.000&	0.000&	0.471&	0.896&	1.565&	0.521 \\
		XLI&	0.355&	0.557&	0.000&	NA&	0.000&	0.000&	0.920&	0.442&	1.173 \\
		XLK&	0.000&	0.703&	0.801&	0.000&	NA&	0.000&	0.000&	0.357&	0.000 \\
		XLP&	0.819&	0.377&	0.653&	0.291&	0.000&	NA&	1.095&	0.418&	0.000 \\
		XLU&	0.370&	0.400&	1.139&	0.367&	1.035&	0.347&	NA&	0.398&	1.048 \\
		XLV&	0.331&	0.375&	1.125&	0.409&	0.000&	0.000&	0.482&	NA&	0.963 \\
		XLY&	0.762&	0.000&	0.384&	0.705&	0.000&	0.000&	0.000&	0.331&	NA \\
		\hline
		& \multicolumn{9}{c}{Lagged $QVJL3$ from} \\
		$QVJL3$&	XLB&	XLE&	XLF&	XLI&	XLK&	XLP&	XLU&	XLV&	XLY  \\ \cmidrule{2-10}
		XLB&	NA&	2.769&	4.471&	1.648&	2.612&	0.819&	0.418&	1.719&	1.762 \\
		XLE&	0.716&	NA&	0.340&	0.365&	0.501&	0.283&	1.406&	0.642&	1.471 \\
		XLF&	0.000&	0.529&	NA&	0.436&	0.000&	0.000&	0.000&	1.451&	1.459 \\
		XLI&	0.905&	0.431&	0.861&	NA&	1.092&	0.396&	0.000&	0.539&	0.526 \\
		XLK&	0.344&	0.440&	0.389&	0.431&	NA&	0.000&	0.000&	0.387&	0.000 \\
		XLP&	0.000&	0.969&	0.997&	1.038&	1.226&	NA&	0.000&	0.000&	0.484 \\
		XLU&	0.000&	0.000&	1.048&	0.000&	0.000&	0.000&	NA&	0.429&	1.423 \\
		XLV&	0.548&	0.000&	0.000&	0.000&	0.000&	0.000&	0.000&	NA&	0.000 \\
		XLY&	0.000&	0.000&	0.000&	0.000&	0.000&	0.000&	0.000&	0.000&	NA \\
		\hline
		& \multicolumn{9}{c}{Lagged $QVJS3$ from} \\
		$QVJS3$ &	XLB&	XLE&	XLF&	XLI&	XLK&	XLP&	XLU&	XLV&	XLY \\ \cmidrule{2-10}
		XLB&	NA&	0.000&	0.000&	0.463&	0.000&	0.000&	0.000&	0.000&	0.439  \\
		XLE&	0.426&	NA&	0.387&	0.000&	0.371&	0.456&	0.506&	0.881&	0.506 \\
		XLF&	0.587&	0.267&	NA&	0.801&	0.328&	0.000&	0.694&	0.323&	0.000 \\
		XLI&	0.000&	0.000&	0.845&	NA&	0.791&	0.000&	0.000&	0.000&	0.421 \\
		XLK&	0.419&	0.000&	0.000&	0.000&	NA&	0.000&	0.000&	0.000&	0.000 \\
		XLP&	0.000&	0.000&	0.000&	0.000&	0.000&	NA&	0.000&	0.000&	0.000 \\
		XLU&	0.000&	1.039&	0.436&	0.000&	0.000&	1.265&	NA&	0.000&	0.904 \\
		XLV&	3.408&	1.108&	1.112&	2.649&	0.000&	2.390&	0.830&	NA&	3.510 \\
		XLY&	0.967&	0.000&	0.000&	0.332&	0.000&	0.517&	0.313&	0.431&	NA \\
		\hline\hline
	\end{tabular}
	\begin{minipage}{1\columnwidth}
	$^{\ast}$ \footnotesize Notes: See notes to Table 3a.
	\end{minipage}
	\pagebreak		
	\captionof{subtable}{Results Based on Analysis of 2010 Jump Variation Data*}
	\begin{tabular}{c|ccccccccc}
		\hline\hline
		& \multicolumn{9}{c}{Lagged $QVJ$ from} \\	
		$QVJ$ &XLB	&XLE&	XLF&	XLI&	XLK&	XLP&	XLU&	XLV&	XLY \\ \cmidrule{2-10}
		XLB&	NA&	0.000&	0.384&	0.000&	0.000&	0.000&	0.000&	0.000&	0.000 \\
		XLE&	0.000&	NA&	0.000&	0.405&	0.000&	0.000&	0.000&	0.000&	0.000 \\
		XLF&	0.415&	0.000&	NA&	0.397&	0.485&	0.000&	0.000&	0.399&	0.488 \\
		XLI&	0.347&	0.000&	0.667&	NA&	0.338&	1.189&	0.389&	0.000&	0.000 \\
		XLK&	0.000&	0.388&	0.430&	0.733&	NA&	0.895&	1.513&	0.000&	0.450 \\
		XLP&	1.426&	0.000&	0.833&	1.080&	0.000&	NA&	1.531&	1.418&	2.240 \\
		XLU&	0.000&	0.000&	0.000&	0.000&	0.000&	0.000&	NA&	0.000&	0.000 \\
		XLV&	0.000&	0.000&	0.000&	0.000&	0.000&	0.000&	0.000&	NA&	0.000 \\
		XLY&	0.000&	0.000&	0.000&	0.000&	0.841&	0.000&	0.571&	0.469&	NA \\
		\hline
		& \multicolumn{9}{c}{Lagged $QVJL3$ from} \\
		$QVJL3$&	XLB&	XLE&	XLF&	XLI&	XLK&	XLP&	XLU&	XLV&	XLY  \\ \cmidrule{2-10}
		XLB&	NA&	0.353&	0.000&	0.433&	0.854&	0.000&	0.334&	0.000&	0.000  \\
		XLE&	0.000&	NA&	0.000&	0.000&	0.000&	0.000&	0.000&	0.000&	0.000  \\
		XLF&	0.336&	1.118&	NA&	0.513&	0.000&	0.528&	0.000&	0.420&	1.288  \\
		XLI&	0.000&	0.000&	0.368&	NA&	0.000&	0.000&	0.000&	0.000&	0.000  \\
		XLK&	0.000&	0.562&	0.569&	0.640&	NA&	1.688&	0.000&	2.114&	0.600  \\
		XLP&	1.029&	0.848&	0.704&	0.953&	0.904&	NA&	0.549&	2.129&	0.831  \\
		XLU&	0.000&	0.597&	0.000&	0.000&	0.000&	0.000&	NA&	0.000&	0.000  \\
		XLV&	0.822&	0.708&	0.310&	0.482&	0.000&	1.328&	0.000&	NA&	0.486  \\
		XLY&	0.434&	0.945&	1.108&	0.559&	1.554&	0.994&	0.425&	0.351&	NA  \\
		\hline
		& \multicolumn{9}{c}{Lagged $QVJS3$ from} \\
		$QVJS3$ &	XLB&	XLE&	XLF&	XLI&	XLK&	XLP&	XLU&	XLV&	XLY \\ \cmidrule{2-10}
		XLB&	NA&	0.356&	0.000&	0.000&	0.399&	0.377&	0.000&	0.333&	0.000  \\
		XLE&	0.379&	NA&	1.239&	1.103&	1.152&	0.665&	0.722&	0.430&	1.142  \\
		XLF&	0.000&	0.382&	NA&	0.526&	1.342&	0.928&	1.056&	0.335&	1.048  \\
		XLI&	0.485&	0.634&	0.632&	NA&	0.359&	1.503&	1.419&	1.211&	0.981  \\
		XLK&	0.388&	0.896&	0.538&	0.588&	NA&	0.000&	0.409&	0.393&	0.000  \\
		XLP&	0.750&	0.000&	0.000&	0.000&	0.000&	NA&	0.282&	0.293&	0.000  \\
		XLU&	0.440&	0.000&	0.468&	0.442&	0.000&	0.583&	NA&	0.339&	0.402  \\
		XLV&	1.878&	2.021&	0.951&	0.784&	0.666&	1.386&	0.747&	NA&	0.000  \\
		XLY&	0.420&	0.000&	0.451&	0.000&	0.486&	0.000&	0.745&	0.000&	NA  \\
		\hline\hline
	\end{tabular}
	\begin{minipage}{1\columnwidth}
	$^{\ast}$ \footnotesize Notes: See notes to Table 3a.
	\end{minipage}
	\pagebreak
	\captionof{table}{\textbf{Jump Contribution to Excess Returns For Nine SPDR Sector ETFs}}
	\captionof{subtable}{Results Based on Analysis of 2005 Jump Variation Data*}
	\begin{tabular}{c|ccccccccc}
		\hline\hline
		& \multicolumn{9}{c}{Lagged $QVJ$ from} \\		
		Excess Returns of &XLB	&XLE&	XLF&	XLI&	XLK&	XLP&	XLU&	XLV&	XLY \\ \cmidrule{2-10}
		XLB&	0.000&	0.000&	0.000&	1.255&	0.000&	2.858&	0.000&	0.000&	1.360 \\
		XLE&	0.000&	1.949&	0.996&	0.860&	0.883&	1.332&	0.000&	1.502&	0.792 \\
		XLF&	0.000&	0.000&	0.000&	0.000&	0.000&	0.211&	0.000&	0.000&	0.000 \\
		XLI&	0.000&	0.000&	0.000&	0.000&	0.000&	0.371&	0.000&	0.000&	0.139 \\
		XLK&	0.000&	0.000&	0.000&	3.164&	6.182&	2.348&	2.458&	0.000&	0.000 \\
		XLP&	0.000&	0.241&	0.141&	0.000&	0.181&	0.324&	0.000&	0.000&	0.305 \\
		XLU&	0.000&	0.000&	0.000&	0.786&	0.816&	0.748&	0.328&	0.000&	0.000 \\
		XLV&	0.000&	0.000&	0.000&	0.308&	0.000&	0.243&	0.000&	0.000&	0.000 \\
		XLY&	0.000&	0.000&	0.000&	0.000&	0.308&	0.000&	0.000&	0.000&	0.000 \\
		\hline
		& \multicolumn{9}{c}{Lagged $QVJL3$ from} \\
		Excess Returns of &	XLB&	XLE&	XLF&	XLI&	XLK&	XLP&	XLU&	XLV&	XLY  \\ \cmidrule{2-10}
		XLB&	1.062&	1.382&	1.244&	0.000&	0.000&	1.169&	0.000&	1.020&	0.000 \\
		XLE&	1.099&	1.558&	0.000&	0.554&	0.721&	0.000&	0.000&	0.518&	0.000 \\
		XLF&	0.187&	0.000&	0.000&	0.000&	0.000&	0.000&	0.000&	0.000&	0.000 \\
		XLI&	0.126&	0.267&	0.129&	0.081&	0.264&	0.348&	0.000&	0.198&	0.105 \\
		XLK&	0.000&	7.106&	4.254&	2.618&	3.642&	5.234&	2.180&	1.807&	0.000 \\
		XLP&	0.000&	0.000&	0.000&	0.000&	0.000&	0.118&	0.000&	0.000&	0.000 \\
		XLU&	0.643&	0.511&	0.000&	0.000&	0.000&	0.637&	0.000&	0.000&	0.000 \\
		XLV&	0.000&	0.000&	0.000&	0.000&	0.000&	0.000&	0.000&	0.167&	0.000 \\
		XLY&	0.203&	0.235&	0.000&	0.000&	0.000&	0.251&	0.000&	0.000&	0.000 \\
		\hline
		& \multicolumn{9}{c}{Lagged $QVJS3$ from} \\
		Excess Returns of &	XLB&	XLE&	XLF&	XLI&	XLK&	XLP&	XLU&	XLV&	XLY \\ \cmidrule{2-10}
		XLB&	0.000&	0.000&	0.000&	0.000&	1.505&	0.000&	1.790&	2.064&	5.254 \\
		XLE&	0.000&	0.000&	0.000&	0.979&	0.000&	0.000&	0.000&	1.056&	1.364 \\
		XLF&	0.000&	0.000&	0.000&	0.368&	0.000&	0.000&	0.000&	0.000&  0.409 \\
		XLI&	0.000&	0.332&	0.000&	0.192&	0.000&	0.000&	0.000&	0.000&	0.259 \\
		XLK&	0.000&	0.000&	0.000&	0.000&	3.122&	0.000&	3.696&	0.000&	9.986 \\
		XLP&	0.000&	0.567&	0.000&	0.246&	0.153&	0.000&	0.625&	0.214&	0.790 \\
		XLU&	0.000&	0.717&	0.000&	0.000&	0.394&	0.000&	0.579&	0.599&	1.278 \\
		XLV&	0.000&	0.000&	0.000&	0.000&	0.000&	0.000&	0.000&	0.000&	0.410 \\
		XLY&	0.000&	0.000&	0.000&	0.000&	0.000&	0.000&	0.000&	0.000&	0.464 \\
		\hline\hline
	\end{tabular}
	\begin{minipage}{1\columnwidth}
		$^{\ast}$ \footnotesize Notes: Entries are ``aggregate jump effects on excess returns" of lagged jumps from a given sector (see first row of entries) on the excess return for each of the sectors listed in the first column of the table. Aggregate jump effects on excess returns are aggregated absolute coefficient magnitudes, summed for statistically significant (at a 5\% level, based on application of $t$-statistics) coefficients on the lags in the VAR associated with the regression pertaining to each sector listed in the first column of the table, for all lags in the regression pertaining to the sector listed in the first row of entries in the table. For further details refer to Section 3.
	\end{minipage}
	\pagebreak		
	\captionof{subtable}{Results Based on Analysis of 2006 Jump Variation Data*}
	\begin{tabular}{c|ccccccccc}
		\hline\hline
		& \multicolumn{9}{c}{Lagged $QVJ$ from} \\	
		Excess Returns of &XLB	&XLE&	XLF&	XLI&	XLK&	XLP&	XLU&	XLV&	XLY \\ \cmidrule{2-10}
		XLB&	0.000&	1.523&	0.000&	0.000&	0.000&	0.000&	0.955&	1.458&	0.000 \\
		XLE&	0.000&	1.653&	1.611&	1.189&	0.000&	0.000&	1.239&	1.530&	1.063 \\
		XLF&	0.000&	0.575&	0.218&	0.000&	0.000&	0.000&	0.214&	0.308&	0.227 \\
		XLI&	0.000&	0.703&	0.000&	0.621&	0.274&	0.000&	0.249&	0.648&	0.644 \\
		XLK&	0.468&	1.203&	0.728&	0.447&	0.390&	0.000&	0.782&	2.108&	0.385 \\
		XLP&	0.871&	2.522&	0.000&	0.769&	0.000&	0.000&	0.924&	1.287&	0.930 \\
		XLU&	0.000&	0.938&	0.509&	0.000&	0.000&	0.000&	0.328&	0.000&	0.000 \\
		XLV&	1.025&	2.615&	1.225&	2.476&	0.000&	1.209&	0.816&	2.492&	1.836 \\
		XLY&	0.000&	1.308&	0.000&	0.000&	0.000&	0.000&	0.459&	0.847&	0.000 \\
		\hline
		& \multicolumn{9}{c}{Lagged $QVJL3$ from} \\
		Excess Returns of &	XLB&	XLE&	XLF&	XLI&	XLK&	XLP&	XLU&	XLV&	XLY  \\ \cmidrule{2-10}
		XLB&	0.000&	0.000&	0.000&	0.000&	0.000&	0.000&	0.000&	0.763&	0.000  \\
		XLE&	0.000&	0.000&	0.000&	0.000&	0.000&	0.000&	0.000&	0.000&	0.000  \\
		XLF&	0.000&	0.000&	0.226&	0.227&	0.000&	0.192&	0.000&	0.160&	0.000  \\
		XLI&	0.000&	0.000&	0.000&	0.000&	0.000&	0.000&	0.000&	0.000&	0.000  \\
		XLK&	0.000&	0.000&	0.000&	0.369&	0.000&	0.373&	0.000&	0.000&	0.000  \\
		XLP&	0.000&	0.000&	0.000&	0.952&	0.000&	0.847&	0.000&	0.000&	0.000  \\
		XLU&	0.000&	0.000&	0.000&	0.000&	0.000&	0.000&	0.000&	0.268&	0.000  \\
		XLV&	0.000&	0.000&	0.000&	0.972&	0.000&	0.931&	0.000&	0.000&	0.000  \\
		XLY&	0.000&	0.000&	0.000&	0.544&	0.000&	0.515&	0.000&	0.346&	0.000  \\
		\hline
		& \multicolumn{9}{c}{Lagged $QVJS3$ from} \\
		Excess Returns of &	XLB&	XLE&	XLF&	XLI&	XLK&	XLP&	XLU&	XLV&	XLY \\ \cmidrule{2-10}
		XLB&	3.959&	0.000&	1.660&	1.620&	0.000&	0.000&	0.000&	0.000&	0.000 \\
		XLE&	2.108&	0.000&	1.954&	0.000&	0.000&	0.000&	1.588&	0.000&	0.000 \\
		XLF&	0.421&	0.000&	0.402&	0.000&	0.000&	0.000&	0.000&	0.000&	0.000 \\
		XLI&	0.503&	0.000&	0.000&	0.000&	0.000&	0.000&	0.419&	0.000&	0.540 \\
		XLK&	0.000&	0.000&	0.742&	0.000&	0.000&	0.000&	0.000&	0.000&	0.000 \\
		XLP&	0.000&	0.000&	2.825&	1.580&	0.000&	1.526&	1.160&	1.290&	0.000 \\
		XLU&	0.615&	0.000&	0.000&	0.000&	0.000&	0.000&	0.000&	0.000&	0.000 \\
		XLV&	1.973&	0.000&	1.581&	1.263&	0.000&	0.000&	0.000&	0.000&	1.695 \\
		XLY&	1.026&	0.000&	0.981&	0.000&	0.000&	0.000&	0.000&	0.000&	0.000 \\
		\hline\hline
	\end{tabular}
	\begin{minipage}{1\columnwidth}
	$^{\ast}$ \footnotesize Notes: See notes to Table 4a.
	\end{minipage}
	\pagebreak		
	\captionof{subtable}{Results Based on Analysis of 2007 Jump Variation Data*}
	\begin{tabular}{c|ccccccccc}
		\hline\hline
		& \multicolumn{9}{c}{Lagged $QVJ$ from} \\	
		Excess Returns of &XLB	&XLE&	XLF&	XLI&	XLK&	XLP&	XLU&	XLV&	XLY \\ \cmidrule{2-10}
		XLB&	0.000&	0.000&	0.000&	1.107&	1.732&	1.317&	0.000&	0.000&	1.737 \\
		XLE&	2.423&	0.000&	0.000&	1.226&	1.969&	1.218&	0.949&	0.988&	0.000 \\
		XLF&	0.000&	0.000&	0.000&	0.850&	1.432&	1.046&	0.000&	0.000&	0.000 \\
		XLI&	0.630&	0.433&	0.154&	0.184&	1.084&	0.696&	0.312&	0.126&	0.285 \\
		XLK&	0.220&	0.447&	0.405&	0.337&	1.100&	0.747&	0.241&	0.522&	0.588 \\
		XLP&	0.000&	0.000&	0.000&	0.281&	0.528&	0.346&	0.000&	0.000&	0.000 \\
		XLU&	0.000&	0.000&	0.299&	0.207&	0.000&	0.229&	0.000&	0.000&	0.000 \\
		XLV&	0.000&	0.000&	0.000&	0.407&	0.716&	0.474&	0.000&	0.000&	0.000 \\
		XLY&	0.227&	0.000&	0.000&	0.000&	0.195&	0.365&	0.000&	0.229&	0.000 \\
		\hline
		& \multicolumn{9}{c}{Lagged $QVJL3$ from} \\
		Excess Returns of &	XLB&	XLE&	XLF&	XLI&	XLK&	XLP&	XLU&	XLV&	XLY  \\ \cmidrule{2-10}
		XLB&	0.000&	0.000&	1.431&	0.000&	0.000&	0.000&	1.954&	0.000&	0.000  \\
		XLE&	0.000&	0.000&	0.000&	0.000&	0.000&	0.000&	1.575&	0.000&	0.000  \\
		XLF&	0.000&	0.000&	1.228&	1.041&	0.000&	0.000&	1.567&	0.000&	1.063  \\
		XLI&	0.000&	0.830&	0.246&	0.204&	0.462&	0.279&	0.553&	0.203&	0.263  \\
		XLK&	0.000&	0.647&	0.274&	0.454&	0.466&	0.274&	0.688&	1.003&	0.312  \\
		XLP&	0.000&	0.000&	0.412&	0.405&	0.000&	0.000&	0.936&	0.372&	0.000  \\
		XLU&	0.000&	0.338&	0.727&	0.000&	0.353&	0.000&	0.311&	0.000&	0.000  \\
		XLV&	0.000&	0.000&	0.582&	0.000&	0.000&	0.000&	0.670&	0.000&	0.000  \\
		XLY&	0.299&	0.000&	0.000&	0.287&	0.000&	0.215&	0.271&	0.000&	0.000  \\
		\hline
		& \multicolumn{9}{c}{Lagged $QVJS3$ from} \\
		Excess Returns of &	XLB&	XLE&	XLF&	XLI&	XLK&	XLP&	XLU&	XLV&	XLY \\ \cmidrule{2-10}
		XLB&	7.860&	1.975&	0.000&	0.000&	1.544&	0.000&	0.000&	1.717&	1.509 \\
		XLE&	6.079&	1.941&	1.503&	0.000&	2.799&	3.131&	1.891&	1.348&	3.072 \\
		XLF&	10.805&	1.864&	1.592&	0.968&	2.285&	2.943&	2.824&	1.563&	2.512 \\
		XLI&	0.364&	0.245&	0.000&	0.221&	0.252&	0.516&	0.191&	0.456&	1.001 \\
		XLK&	1.757&	0.000&	0.638&	0.554&	0.614&	0.864&	0.257&	0.350&	0.263 \\
		XLP&	1.816&	0.459&	0.000&	0.684&	0.767&	0.488&	0.000&	0.371&	0.387 \\
		XLU&	1.251&	0.920&	0.902&	0.698&	0.000&	0.296&	0.617&	0.297&	1.007 \\
		XLV&	3.968&	0.732&	0.000&	0.992&	1.569&	1.312&	1.214&	0.779&	0.708 \\
		XLY&	1.530&	0.328&	0.916&	0.602&	0.937&	0.463&	1.036&	0.680&	1.175 \\
		\hline\hline
	\end{tabular}
	\begin{minipage}{1\columnwidth}
	$^{\ast}$ \footnotesize Notes: See notes to Table 4a.
	\end{minipage}
	\pagebreak		
	\captionof{subtable}{Results Based on Analysis of 2008 Jump Variation Data*}
	\begin{tabular}{c|ccccccccc}
		\hline\hline
		& \multicolumn{9}{c}{Lagged $QVJ$ from} \\	
		Excess Returns of  &XLB	&XLE&	XLF&	XLI&	XLK&	XLP&	XLU&	XLV&	XLY \\ \cmidrule{2-10}
		XLB&	0.000&	0.000&	5.318&	0.000&	4.341&	3.399&	0.000&	1.608&	1.908 \\
		XLE&	0.180&	0.181&	0.000&	0.000&	0.238&	0.439&	0.506&	0.232&	0.562 \\
		XLF&	0.000&	0.000&	1.027&	1.727&	2.185&	1.332&	0.000&	1.506&	1.655 \\
		XLI&	0.000&	0.000&	2.541&	0.000&	3.953&	2.262&	0.000&	1.251&	1.302 \\
		XLK&	0.000&	0.000&	0.000&	0.000&	2.929&	1.719&	0.883&	0.905&	0.921 \\
		XLP&	0.000&	0.000&	1.543&	0.000&	2.607&	1.926&	0.000&	1.040&	1.178 \\
		XLU&	0.000&	0.000&	2.030&	2.575&	4.329&	2.347&	0.000&	1.414&	1.269 \\
		XLV&	0.475&	0.000&	1.311&	2.587&	0.690&	0.632&	0.000&	0.617&	0.514 \\
		XLY&	0.236&	0.000&	0.213&	0.000&	0.000&	0.000&	0.264&	0.614&	0.242 \\
		\hline
		& \multicolumn{9}{c}{Lagged $QVJL3$ from} \\
		Excess Returns of &	XLB&	XLE&	XLF&	XLI&	XLK&	XLP&	XLU&	XLV&	XLY  \\ \cmidrule{2-10}
		XLB&	5.715&	0.000&	2.583&	3.032&	5.530&	4.535&	1.943&	1.743&	0.000  \\
		XLE&	1.281&	1.313&	0.859&	0.579&	1.000&	1.406&	1.444&	1.091&	2.059  \\
		XLF&	4.947&	0.000&	2.197&	0.000&	5.508&	4.302&	1.937&	4.035&	2.151  \\
		XLI&	3.733&	0.000&	3.287&	1.878&	3.944&	3.163&	1.281&	2.695&	1.432  \\
		XLK&	3.010&	0.000&	2.435&	1.422&	4.264&	2.420&	0.998&	2.035&	0.000  \\
		XLP&	4.862&	0.000&	2.879&	2.745&	5.331&	2.809&	1.325&	3.687&	1.220  \\
		XLU&	11.193&	6.285&	12.577&	6.376&	12.264&	9.282&	3.588&	7.181&	6.544  \\
		XLV&	2.184&	0.000&	1.688&	1.000&	2.170&	0.867&	0.658&	1.655&	0.736  \\
		XLY&	1.857&	0.930&	1.460&	0.525&	1.752&	1.698&	1.173&	1.053&	2.737  \\
		\hline
		& \multicolumn{9}{c}{Lagged $QVJS3$ from} \\
		Excess Returns of &	XLB&	XLE&	XLF&	XLI&	XLK&	XLP&	XLU&	XLV&	XLY \\ \cmidrule{2-10}
		XLB&	3.633&	10.667&	0.000&	0.000&	3.759&	0.000&	2.744&	0.000&	0.000 \\
		XLE&	0.000&	0.000&	0.000&	0.000&	0.708&	0.000&	0.413&	0.000&	0.000 \\
		XLF&	6.898&	12.000&	6.824&	3.493&	8.858&	0.000&	9.562&	0.000&	0.000 \\
		XLI&	4.361&	7.352&	0.000&	0.000&	2.854&	0.000&	0.000&	0.000&	0.000 \\
		XLK&	1.699&	5.255&	0.000&	0.000&	1.884&	0.000&	1.311&	0.000&	0.000 \\
		XLP&	6.263&	7.300&	0.000&	2.283&	2.396&	0.000&	3.993&	0.000&	1.487 \\
		XLU&	4.972&	8.400&	0.000&	0.000&	2.950&	0.000&	2.149&	0.000&	0.000 \\
		XLV&	0.000&	4.595&	0.000&	1.529&	0.000&	0.000&	2.839&	0.000&	2.124 \\
		XLY&	0.000&	0.000&	0.000&	0.000&	0.433&	0.462&	0.000&	0.524&	0.000 \\
		\hline\hline
	\end{tabular}
	\begin{minipage}{1\columnwidth}
	$^{\ast}$ \footnotesize Notes: See notes to Table 4a.
	\end{minipage}
	\pagebreak		
	\captionof{subtable}{Results Based on Analysis of 2009 Jump Variation Data*}
	\begin{tabular}{c|ccccccccc}
		\hline\hline
		& \multicolumn{9}{c}{Lagged $QVJ$ from} \\	
		Excess Returns of &XLB	&XLE&	XLF&	XLI&	XLK&	XLP&	XLU&	XLV&	XLY \\ \cmidrule{2-10}
		XLB&	3.134&	2.922&	4.434&	0.885&	4.933&	0.000&	5.807&	5.157&	3.197 \\
		XLE&	4.841&	8.950&	8.662&	2.693&	4.988&	2.398&	7.035&	8.543&	6.908 \\
		XLF&	6.728&	11.083&	13.736&	2.934&	19.275&	0.000&	22.079&	13.582&	4.253 \\
		XLI&	0.498&	0.596&	1.025&	0.186&	1.079&	0.207&	0.781&	0.521&	0.326 \\
		XLK&	1.820&	2.210&	3.418&	0.607&	3.155&	0.000&	4.279&	2.682&	1.731 \\
		XLP&	3.794&	4.623&	6.767&	1.311&	7.525&	0.000&	8.664&	4.891&	4.562 \\
		XLU&	5.691&	6.468&	8.408&	1.749&	11.409&	0.000&	13.116&	8.053&	0.000 \\
		XLV&	2.246&	3.715&	4.685&	2.274&	5.347&	0.000&	6.222&	2.176&	3.128 \\
		XLY&	2.803&	2.517&	3.111&	0.616&	3.735&	0.000&	4.903&	2.843&	0.924 \\
		\hline
		& \multicolumn{9}{c}{Lagged $QVJL3$ from} \\
		Excess Returns of &	XLB&	XLE&	XLF&	XLI&	XLK&	XLP&	XLU&	XLV&	XLY  \\ \cmidrule{2-10}
		XLB &	0.000&	1.485&	3.585&	1.735&	1.946&	0.000&	1.316&	0.892&	5.074  \\
		XLE &	2.131&	1.387&	5.301&	4.211&	3.337&	0.000&	3.726&	0.000&	3.405  \\
		XLF &	9.185&	18.613&	19.640&	21.273&	21.665&	4.287&	11.449&	0.000&	27.862  \\
		XLI &	0.370&	0.000&	0.839&	0.543&	0.000&	0.398&	0.680&	0.235&	0.000  \\
		XLK &	1.816&	1.208&	2.630&	2.423&	1.658&	0.000&	1.120&	0.000&	1.670  \\
		XLP &	1.784&	2.315&	4.898&	2.714&	3.328&	0.000&	4.365&	0.000&	5.484  \\
		XLU &	5.593&	3.598&	11.883&	11.540&	8.914&	2.704&	9.382&	0.000&	9.509  \\
		XLV &	1.545&	2.170&	6.682&	4.529&	4.627&	0.000&	3.995&	0.000&	7.153  \\
		XLY &	2.169&	1.350&	5.418&	4.540&	1.795&	2.110&	2.493&	0.720&	3.666  \\
		\hline
		& \multicolumn{9}{c}{Lagged $QVJS3$ from} \\
		Excess Returns of  &	XLB&	XLE&	XLF&	XLI&	XLK&	XLP&	XLU&	XLV&	XLY \\ \cmidrule{2-10}
		XLB&	0.000&	2.242&	2.132&	0.000&	1.925&	0.000&	0.000&	4.873&	4.227 \\
		XLE&	0.000&	1.936&	1.638&	0.000&	1.724&	3.211&	0.000&	3.033&	5.869 \\
		XLF&	0.000&	8.467&	0.000&	0.000&	7.602&	0.000&	0.000&	0.000&	7.216 \\
		XLI&	0.665&	0.555&	2.065&	1.788&	0.946&	1.129&	2.226&	2.216&	1.831 \\
		XLK&	0.000&	1.664&	0.000&	0.000&  1.426&	2.825&	0.000&	0.000&	1.575 \\
		XLP&	3.026&	0.000&	0.000&	0.000&	2.583&	0.000&	0.000&	0.000&	5.777 \\
		XLU&	0.000&	8.165&	8.282&	0.000&	0.000&	8.443&	0.000&	14.484&	15.314 \\
		XLV&	0.000&	2.357&	3.009&	0.000&	2.489&	0.000&	0.000&	6.500&	2.714 \\
		XLY&	0.000&	3.119&	0.000&	0.000&	1.368&	3.170&	0.000&	1.556&	5.798 \\
		\hline\hline
	\end{tabular}
	\begin{minipage}{1\columnwidth}
	$^{\ast}$ \footnotesize Notes: See notes to Table 4a.
	\end{minipage}
	\pagebreak		
	\captionof{subtable}{Results Based on Analysis of 2010 Jump Variation Data*}
	\begin{tabular}{c|ccccccccc}
		\hline\hline
		& \multicolumn{9}{c}{Lagged $QVJ$ from} \\	
		Excess Returns of &XLB	&XLE&	XLF&	XLI&	XLK&	XLP&	XLU&	XLV&	XLY \\ \cmidrule{2-10}
		XLB&	0.000&	0.000&	0.000&	2.205&	0.000&	0.000&	0.000&	0.000&	0.000 \\
		XLE&	0.000&	0.000&	3.229&	6.696&	0.000&	0.000&	0.000&	0.000&	0.000 \\
		XLF&	0.000&	0.000&	1.105&	2.479&	0.000&	0.000&	0.000&	0.000&	0.000 \\
		XLI&	0.000&	0.000&	1.067&	1.728&	0.000&	0.000&	0.000&	0.000&	0.000 \\
		XLK&	0.000&	0.000&	1.371&	1.431&	0.000&	0.610&	0.000&	0.000&	0.614 \\
		XLP&	0.000&	0.000&	0.000&	2.384&	0.000&	0.000&	0.000&	0.000&	0.000 \\
		XLU&	0.000&	0.000&	0.000&	0.727&	0.000&	0.000&	0.000&	0.000&	0.929 \\
		XLV&	0.332&	0.000&	0.693&	0.348&	0.000&	0.000&	0.000&	0.000&	0.000 \\
		XLY&	0.000&	0.000&	3.671&	5.368&	0.000&	2.236&	0.000&	0.000&	0.000 \\
		\hline
		& \multicolumn{9}{c}{Lagged $QVJL3$ from} \\
		Excess Returns of &	XLB&	XLE&	XLF&	XLI&	XLK&	XLP&	XLU&	XLV&	XLY  \\ \cmidrule{2-10}
		XLB&	0.000&	0.000&	0.000&	0.000&	0.000&	0.000&	0.000&	0.000&	0.000  \\
		XLE&	0.000&	0.000&	0.000&	4.984&	0.000&	0.000&	0.000&	0.000&	0.000  \\
		XLF&	0.000&	0.000&	0.000&	0.000&	0.000&	0.000&	0.000&	0.000&	0.000  \\
		XLI&	0.000&	0.000&	0.000&	0.000&	0.000&	0.000&	0.000&	0.000&	0.000  \\
		XLK&	0.000&	0.000&	0.000&	0.000&	0.000&	0.000&	0.000&	0.000&	0.000  \\
		XLP&	0.000&	0.000&	0.000&	0.000&	0.000&	0.000&	0.000&	0.000&	0.000  \\
		XLU&	0.000&	0.000&	0.000&	0.000&	0.000&	0.000&	0.000&	0.000&	0.000  \\
		XLV&	0.000&	0.000&	0.000&	0.000&	0.000&	0.000&	0.000&	0.483&	0.000  \\
		XLY&	0.000&	0.000&	0.000&	0.000&	0.000&	0.000&	0.000&	0.000&	0.000  \\
		\hline
		& \multicolumn{9}{c}{Lagged $QVJS3$ from} \\
		Excess Returns of &	XLB&	XLE&	XLF&	XLI&	XLK&	XLP&	XLU&	XLV&	XLY \\ \cmidrule{2-10}
		XLB&	0.000&	0.000&	0.000&	0.000&	0.000&	0.000&	0.721&	0.000&	0.000 \\
		XLE&	0.000&	0.000&	0.000&	4.551&	0.000&	4.809&	0.000&	4.007&	0.000 \\
		XLF&	0.000&	0.000&	0.000&	0.000&	0.000&	0.000&	1.693&	1.505&	0.000 \\
		XLI&	0.000&	0.000&	0.000&	0.000&	0.000&	0.819&	0.948&	0.825&	0.811 \\
		XLK&	0.000&	0.000&	0.000&	0.000&	0.000&	0.000&	0.000&	0.000&	0.000 \\
		XLP&	0.000&	0.000&	0.000&	0.000&	0.000&	0.000&	0.000&	0.000&	0.000 \\
		XLU&	0.000&	0.000&	0.000&	0.000&	0.000&	0.000&	1.174&	0.000&	1.087 \\
		XLV&	0.000&	0.000&	0.000&	0.000&	0.000&	0.000&	0.000&	0.000&	0.000 \\
		XLY&	0.000&	2.272&	12.386&	9.018&	7.471&	8.746&	10.176&	10.696&	3.685 \\
		\hline\hline
	\end{tabular}
	\begin{minipage}{1\columnwidth}
	$^{\ast}$ \footnotesize Notes: See notes to Table 4a.
	\end{minipage}
	
\begin{landscape}
	\begin{figure}[htb]
		\begin{center}
			\includegraphics[scale=1.35]{1yearlargejump}
			\caption{Large Jump Spillover Effects by Year*}
		\end{center}
	\begin{minipage}{1\columnwidth}
	$^{\ast}$ \footnotesize Notes: This figure aggregates the large ($QVJL3$) spillover effects by sector and by year based on the results in Table 3. NA is treated as 0. Each color block in the graph represents the spillover effects of sector $j$ in year $h$, and is calculated as  $\sum_{i}\sum_{k=1}^{k=22}|\beta_{i,j,k,h}^{\ast}|$ ($j \ne i$) as discussed in Section 3.1.
	\end{minipage}
	\end{figure}


	\begin{figure}[htb]
		\begin{center}
			\includegraphics[scale=1.35]{2yearsmalljump}
			\caption{Small Jump Spillover Effects by Year*}
		\end{center}
	\begin{minipage}{1\columnwidth}
		$^{\ast}$ \footnotesize Notes: This figure aggregates the small ($QVJS3$) spillover effects by sector and by year based on the results in Table 3. See notes to Figure 1.
	\end{minipage}
	\end{figure}
	
	\begin{figure}[htb]
		\begin{center}
			\includegraphics[scale=1.35]{3yearjump}
			\caption{Total Jump Spillover Effects by Year*}
		\end{center}
	\begin{minipage}{1\columnwidth}
		$^{\ast}$ \footnotesize Notes: This figure aggregates the total ($QVJ$) spillover effects by sector and by year based on the results in Table 3. See notes to Figure 1.
	\end{minipage}
	\end{figure}
	
	\begin{figure}[htb]
		\begin{center}
			\includegraphics[scale=1.35]{4yearlargejumpex}
			\caption{Large Jump Contribution Level to Excess Returns*}
		\end{center}
	\begin{minipage}{1\columnwidth}
		$^{\ast}$ \footnotesize Notes: This figure aggregates large ($QVJL3$) jump contribution level to excess returns by sector and by year based on the results in Table 4. Each color block in the graph represent the contribution level of jumps in sector $j$ in year $h$, and is calculated as $C\sum_{i}\sum_{k=0}^{k=22}|\beta_{i,j,k,h}^{\ast}|$ as seen in Section 3.2.
	\end{minipage}
	\end{figure}
	
	\begin{figure}[htb]
		\begin{center}
			\includegraphics[scale=1.35]{5yearsmalljumpex}
			\caption{Small Jump Contribution Level to Excess Returns*}
		\end{center}
		\begin{minipage}{1\columnwidth}
		$^{\ast}$ \footnotesize Notes: This figure aggregates the small ($QVJS3$) jump contribution level to excess returns by sector and by year based on the results in Table 4. See notes to Figure 4.
	\end{minipage}
	\end{figure}
	
	\begin{figure}[htb]
		\begin{center}
			\includegraphics[scale=1.35]{6yearjumpex}
			\caption{Total Jump Contribution Level to Excess Returns*}
		\end{center}
		\begin{minipage}{1\columnwidth}
		$^{\ast}$ \footnotesize Notes: This figure aggregates the total ($QVJ$) jump contribution level to excess returns by sector and by year based on the results in Table 4. See notes to Figure 4.
	\end{minipage}
	\end{figure}
\end{landscape}
}

\end{document}