%2multibyte Version: 5.50.0.2960 CodePage: 936

\documentclass[final,notitlepage]{article}
%%%%%%%%%%%%%%%%%%%%%%%%%%%%%%%%%%%%%%%%%%%%%%%%%%%%%%%%%%%%%%%%%%%%%%%%%%%%%%%%%%%%%%%%%%%%%%%%%%%%%%%%%%%%%%%%%%%%%%%%%%%%%%%%%%%%%%%%%%%%%%%%%%%%%%%%%%%%%%%%%%%%%%%%%%%%%%%%%%%%%%%%%%%%%%%%%%%%%%%%%%%%%%%%%%%%%%%%%%%%%%%%%%%%%%%%%%%%%%%%%%%%%%%%%%%%
\usepackage{amsfonts}
\usepackage{amssymb}
\usepackage{graphicx}
\usepackage{amsmath}
\usepackage{latexsym}
\usepackage{caption}

\setcounter{MaxMatrixCols}{10}
%TCIDATA{OutputFilter=LATEX.DLL}
%TCIDATA{Version=5.50.0.2960}
%TCIDATA{Codepage=936}
%TCIDATA{<META NAME="SaveForMode" CONTENT="1">}
%TCIDATA{BibliographyScheme=Manual}
%TCIDATA{Created=Wed May 15 17:28:15 2002}
%TCIDATA{LastRevised=Friday, September 25, 2020 08:43:37}
%TCIDATA{<META NAME="GraphicsSave" CONTENT="32">}
%TCIDATA{<META NAME="DocumentShell" CONTENT="General\Blank Document">}
%TCIDATA{Language=American English}
%TCIDATA{CSTFile=LaTeX article (bright).cst}
%TCIDATA{PageSetup=65,65,72,72,0}
%TCIDATA{AllPages=
%H=36
%F=29,\PARA{038<p type="texpara" tag="Body Text" > \ \ \ \ \ \ \ \ \ \ \ \ \ \ \ \ \ \ \ \ \ \ \ \ \ \ \ \ \ \ \ \ \ \ \ \ \ \ \ \ \ \ \ \ \ \ \ \ \ \ \ \ \ \ \ \ \ \ \ \ \ \ \ \ \ \ \ \ \ \ \ \ \ \ \ \ \ \ \ \ \ \ \ \ \ \ \ \ \ \ \ \ \ \ \ \ \ \ \ \ \ \ \ \ \ \ \ \ \ \ \ \ \ \ \ \ \ \ \ \ \ \ \ \ \ \ \ \ \ \ \ \ \ \ \ \ \ \ \ \ \ \ \ \ \ \ \ \ \ \ \ \ \ \ \ \ \ \ \ \ \ \ \ \ \ \ \ \ \ \ \ \ \ \ \ \ \ \ \ \ \ \ \ \ \ \ \ \ \ \ \ \ \ \ \ \ \ \ \ \ \ \ \ \ \ \ \ \ \ \ \ \ \ \ \ \ \ \ \ \ \ \ \ \ \ \ \ \ \ \ \ \ \ \ \ \ \ \ \ \ \ \ \ \ \ \ \ \ \ \ \ \ \ \ \ \ \ \ \ \ \ \ \ \ \ \ \ \ \ \ \ \ \ \ \ \ \ \ \ \ \ \ \ \ \ \ \ \ \ \ \ \ \ \ \ \  }
%}


\newtheorem{theorem}{Theorem}[section]
\newtheorem{acknowledgement}[theorem]{Acknowledgement}
\newtheorem{algorithm}[theorem]{Algorithm}
\newtheorem{axiom}[theorem]{Axiom}
\newtheorem{case}[theorem]{Case}
\newtheorem{claim}[theorem]{Claim}
\newtheorem{conclusion}[theorem]{Conclusion}
\newtheorem{condition}[theorem]{Condition}
\newtheorem{conjecture}[theorem]{Conjecture}
\newtheorem{corollary}[theorem]{Corollary}
\newtheorem{criterion}[theorem]{Criterion}
\newtheorem{definition}[theorem]{Definition}
\newtheorem{example}[theorem]{Example}
\newtheorem{exercise}[theorem]{Exercise}
\newtheorem{lemma}[theorem]{Lemma}
\newtheorem{notation}[theorem]{Notation}
\newtheorem{problem}[theorem]{Problem}
\newtheorem{proposition}[theorem]{Proposition}
\newtheorem{remark}[theorem]{Remark}
\newtheorem{solution}[theorem]{Solution}
\newtheorem{summary}[theorem]{Summary}
\newenvironment{proof}[1][Proof]{\textbf{#1.} }{\ \rule{0.5em}{0.5em}}
\input{tcilatex}
\textwidth=16.0cm
\oddsidemargin=0cm \evensidemargin=0cm
\topmargin=-20pt
\numberwithin{equation}{section}
\baselineskip=100pt
\textheight=21cm
\def\baselinestretch{1.2}
\begin{document}

\title{}

\begin{center}
{\LARGE \vspace{1in} Robust Forecast Superiority Testing with an Application
to Assessing Pools of Expert Forecasters*} \bigskip

{\Large Valentina Corradi}$^{1}${\Large , Sainan Jin}$^{2},$ {\Large and
Norman R. Swanson}$^{3}\medskip $

$^{1}$University of Surrey, $^{2}$Singapore Management University, and $^{3}$%
Rutgers University

\bigskip

September 2020

\bigskip

Abstract
\end{center}

\noindent {We develop a forecast superiority testing methodology which is
robust to the choice of loss function. Following Jin, Corradi and Swanson
(JCS: 2017), we rely on a mapping between generic loss forecast evaluation
and stochastic dominance principles. However, unlike JCS tests, which are
not uniformly valid, and have correct asymptotic size only under the least
favorable case, our tests are uniformly asymptotically valid and
non-conservative. These properties are derived by first establishing uniform
convergence (over error support) of HAC variance estimators and of their
bootstrap counterparts, and by extending the asymptotic validity of
generalized moment selection tests to the case of non-vanishing recursive
parameter estimation error. Monte Carlo experiments indicate good finite
sample performance of the new tests, and an empirical illustration suggests
that prior forecast accuracy matters in the Survey of Professional
Forecasters. Namely, for our longest forecast horizons (4 quarters ahead),
selecting pools of expert forecasters based on prior accuracy results in
ensemble forecasts that are superior to those based on forming simple
averages and medians from the entire panel of experts. }

{\small \bigskip \bigskip \bigskip }

\noindent

\noindent \textit{Keywords}: Robust Forecast Evaluation, Many Moment
Inequalities, Bootstrap, Estimation Error, Combination Forecasts, Survey of
Professional Forecasters.

\noindent \_\_\_\_\_\_\_\_\_\_\_\_\_\_\_\_\_\_\_\_\_\_\_\_\_

\noindent *{\footnotesize Valentina Corradi, School of Economics, University
of Surrey, Guildford, Surrey, GU2 7XH, UK, v.corradi@surrey.ac.uk; Sainan
Jin, School of \ Economics, Singapore Management University, 90 Stamford
Road, Singapore 178903, snjin@smu.edu.sg; and Norman R. Swanson, Department
of Economics, Rutgers University, 75 Hamilton Street, New Brunswick, NJ
08901, USA, nswanson@econ.rutgers.edu. We are grateful to Kevin Lee, Patrick
Marsh, Luis Martins, Jams Mitchell, Alessia Paccagini, Paulo Parente, Ivan
Petrella, Valerio Poti, Barbara Rossi, Simon Van Norden, Claudio Zoli, and
to the participants at the 2018 NBER-NSF Times Series Conference, the 2016
European Meeting of the Econometric Society, Conference for 50 years of
Keynes College at Kent University, and seminars at Mannheim University, the
University of Nottingham, University College Dublin, Instituto Universit\'{a}%
rio de Lisboa, Universita' di Verona and the Warwick Business School for
useful comments and suggestions. Additionally, many thanks are owed to
Mingmian Cheng for excellent research assistance.}

\setcounter{page}{0} \thispagestyle{empty}

\renewcommand{\baselinestretch}{1.4}%
%TCIMACRO{\TeXButton{nomalsize}{\normalsize{}}}%
%BeginExpansion
\normalsize{}%
%EndExpansion

\newpage

\section{Introduction}

Forecast accuracy is typically measured in terms of a given loss function,
with quadratic and absolute loss being the most common choices. In recent
years, there has been a growing discussion about the choice of the
\textquotedblleft right\textquotedblright\ loss function. Gneiting (2011)
stresses the importance of matching the quantity to be forecasted and the
choice of loss function (or scoring rule). The latter is said to be
consistent for a given statistical functional (e.g. the mean or the median),
if expected loss is minimized when such a functional is used. In a recent
paper, Patton (2019) shows that if forecasts are based on nested information
sets and on correctly specified models, then in the absence of estimation
error, forecast ranking is robust to the choice of loss function within the
class of consistent functions. On the other hand, if any of the above
conditions fail, then model ranking is dependent on the specific loss
function used. This is an important finding, given that it is natural for
researchers to focus on the comparison of multiple misspecified models;
immediately implying that model rankings are loss function dependent.

In summary, given the importance of loss function dependence when comparing
forecast accuracy, an issue of key concern to empirical economists is the
construction of loss function robust forecast accuracy tests. A loss
function free forecast evaluation criterion of interest should be based on
the distribution of raw forecast errors. Heuristically, one can define the
best forecasting model as that producing errors having a step cumulative
distribution function that is equal to zero on the negative real line and
equal to one on the positive real line. Diebold and Shin (2015, 2017) build
on this idea, and suggest choosing the model for which the cumulative
distribution of the forecast errors is closest to a step function. This idea
is also discussed in Corradi and Swanson (2013). Jin, Corradi and Swanson
(JCS: 2017) establish a one to one mapping between generalized loss (GL)
forecast superiority and first order stochastic dominance, as well as a one
to one mapping between convex loss function (CL) and second order stochastic
dominance.\footnote{%
A loss function\ is a GL function if it\ is monotonically non-decreasing as
the error\ moves away from zero. Additionally, CL functions are the subset
of convex GL functions.} In particular, they show that the \textquotedblleft
best\textquotedblright\ model (regardless of loss function) according to a
GL (CL) function is the one which is first (second) order stochastically
dominated on the negative real line and first (second) order stochastically
dominant on the positive real line, when comparing forecast errors. In this
sense, JCS (2017) establish that loss function free tests for forecast
superiority can be framed in terms of tests for stochastic dominance. In
this paper, we note that tests for stochastic dominance can be seen as tests
for infinitely many moment inequalities. This allows us to utilize tools
recently developed by Andrews and Shi (2013, 2017) to derive asymptotically
uniformly valid and non conservative forecast superiority tests.
Importantly, these tests improve over those introduced in JCS (2017), as the
latter were asymptotically non conservative only in the least favorable case
under the null (i.e., when all moment weak inequalities hold with equality).
Needless to say, controlling for slack inequalities is crucial when there
are infinitely many of them.

The implementation of our tests require that sample moments are standardized
by an estimator of the standard deviation. Now, forecast errors are
typically non martingale difference sequences, either because they are based
on dynamically misspecified models or because forecasters do not efficiently
use all the available information, in the case of subjective predictions.
Hence, we require heteroskedasticity and autocorrelation (HAC) robust
variance estimators. In our set-up, each variance estimator depends on a
specific point in the forecasting error support. Thus, in order to introduce
our new tests for forecast superiority, we must establish the consistency of
HAC variance estimators uniformly over the error support. Moreover, in order
to carry out inference, using our tests, we also establish uniform
convergence of the HAC variance estimator bootstrap counterparts. Because of
the presence of the lag truncation parameter, uniform convergence of HAC
estimators and of their bootstrap analogs does not follow straightforwardly
from uniform convergence of (kernel) nonparametric estimators. To the best
of our knowledge this contribution is a novel addition to the vast
literature on HAC covariance matrix estimation. In the sequel, we focus on
the case of judgmental \ forecasts, in which there is no parameter
estimation error. In a supplemental online appendix, we consider the case of
predictions based on estimated models, and extend all of our results to the
case of non vanishing estimation error. This is accomplished under a
recursive estimation scheme, by extending the recursive block bootstrap
introduced in Corradi and Swanson (2007).

Linton, Song and Whang (2010) also develop tests for stochastic dominance
which are correctly asymptotically sized over the boundary of the null, for
the pairwise comparison case. A key role in their asymptotic analysis is
played by the contact set (i.e., the set of $x$ over which the two CDFs are
equal). However, the notion of contact set does not extend straightforwardly
to the multiple comparison case considered in this paper. It should also be
noted that other papers have addressed the problem of forecast evaluation in
the absence of full specification of the loss function. For example, Patton
and Timmermann (2007) have studied forecast optimality under only generic
assumptions on the loss function. However, they do not address the issue of
forecast ranking under (partially) unknown loss. More recently, Barendse and
Patton (2019) introduce forecast multiple comparison under loss functions
which are specified only up to a shape parameter.

We assess the forecast superiority testing methodology discussed in this
paper via a series of Monte Carlo experiments. Simulation results show that
our new tests are in some key cases much more accurately sized and have much
higher power than JCS tests. For example, in size experiments where DGPs
contain some models which are worse than the benchmark model, our new tests
are substantially better sized than the tests of JCS (2017). Additionally,
our new tests exhibit notable power gains, relative to JCS tests, in power
experiments where DGPs contain some alternative models that dominate the
benchmark, while others are strictly dominated. These findings are as
expected, given that JCS tests are undersized, while our new tests are
asymptotically non conservative.

In an empirical illustration, we apply our testing procedure to the Survey
of Professional Forecasters (SPF) dataset. In the SPF, participants are told
which variables to forecast and whether they should provide a point forecast
or instead a probability interval, but they are not given a loss function
(see Crushore (1993) for a detailed description of the SPF). In the context
of analyzing the predictive content of the SPF, many papers find evidence of
the usefulness of forecast combinations constructed using individual SPF
predictions, under quadratic or absolute loss. For example, Zarnowitz and
Braun (1993) find that using the mean or median provides a consensus
forecast with lower average errors than most individual forecasts. Aiolfi,
Capistr\'{a}n, and Timmermann (2011) and Genre, Kenny, Meyler, and
Timmermann (2013) find that equal weighted averages of SPF and ECB \textit{%
(European Central Bank)} SPF forecasts often outperform model based
forecasts. In our illustration, we depart from these papers by noting that
the SPF naturally lends itself to loss function free forecast superiority
testing, since participants are not given loss functions. In light of this,
we apply our new tests, and show that forecast averages (and medians) from
small pools of survey participants ranked according to recent forecast
performance are preferred to forecast averages based on the entire pool of
experts, for our longest forecast horizon (1-year ahead). We thus conclude
that simple average and median forecasts can in some cases be
\textquotedblleft beaten\textquotedblright , regardless of loss function.

The rest of the paper is organized as follows. Section 2 outlines the set-up
and introduces our new tests. Section 3 establishes the asymptotic
properties of the tests in the context of generalized moment selection.
Section 4 contains the results of our Monte Carlo experiments, and Section 5
contains the results of our analysis of GDP growth forecasts from the SPF.
Finally, Section 6\textit{\ }provides a number of concluding remarks. Proofs
are gathered in an appendix. In a supplemental appendix, we establish the
asymptotic properties of our new tests in the context of non-vanishing
parameter estimation error, for the recursive estimation schemes.

\section{Forecast Superiority Tests}

\noindent Assume that we have a time series of forecast errors for each
model/forecaster. Namely, we observe $e_{j,t},$ for $j=1,...,k$ and $%
t=1,...n $, where $k$ denotes the number of models/forecasters, and $n$
denotes the number of observations. As stated earlier, we focus on the case
in which we can ignore estimation error, such as when forecasts are
judgmental or subjective.\textit{\ }Surveys including the SPF are leading
examples of judgmental forecasts. The case of non-vanishing recursive
estimation error is analyzed in the supplemental appendix.\textit{\ }%
Hereafter, the sequence $e_{1,t},$ $t=1,...,n$ is called the
\textquotedblleft benchmark\textquotedblright . In the context of the SPF,
an example of a relevant benchmark against which to compare all other
sequences is the consensus forecast constructed as the simple arithmetic
average of individual forecasts in the survey. Our goal is to test whether
there exists some competing forecast that is superior to the benchmark for
any loss function, $L$, satisfying Assumption A0.

\noindent \textbf{Assumption A0 }(i)\textbf{\ }$L\in \mathcal{L}_{G}$ if $L:%
\mathbb{R\rightarrow R}^{+}$ is continuously differentiable, except for
finitely many points, with derivative $L^{\prime },$ such that $L^{\prime
}(z)\leq 0,$ for all $z\leq 0,$ and $L^{\prime }(z)\geq 0,$ for all $z\geq
0. $ (ii) $L\in \mathcal{L}_{C}$ is a convex function belonging to $\mathcal{%
L}_{G}.$

Note that $\mathcal{L}_{G}$ includes most of the loss functions commonly
used by practitioners, including asymmetric loss, and it basically coincides
with notion of generalized loss in Granger (1999). The only restriction is
that the loss depends solely on the forecast errors. This rules out the
class of loss function considered in e.g. Section 3 of Patton and Timmermann
(2007).

Hereafter, let $F_{j}(x)$ denote the cumulative distribution function (CDF)
of forecast error $e_{j}.$ Also, define $sgn(x)=1$ $if$ $x\geq 0$ $and$ $%
sgn(x)=-1$ $if$ $x<0.$ Propositions 2.2 and 2.3 in JCS (2017) establish the
following results.\medskip

\noindent \textit{1. For any }$L\in L_{G},$\textit{\ }$E(L(e_{1}))\leq
E(L(e_{2})),$\textit{\ if and only if }$(F_{2}(x)-F_{1}(x))sgn(x)\leq 0,$%
\textit{\ for all }$x\in \mathcal{X}.$\textit{\ \medskip }

\noindent \textit{2. For any }$L\in L_{C},$\textit{\ }$E(L(e_{1}))\leq
E(L(e_{2})),$\textit{\ if and only if}

\textit{\noindent }$\left( \int_{-\infty
}^{x}(F_{1}(t)-F_{2}(t))dt1(x<0)+\int_{x}^{\infty
}(F_{2}(t)-F_{1}(t))dt1(x\geq 0)\right) \leq 0,$\textit{\ for all }$x\in 
\mathcal{X}.\medskip $

The first statement establishes a mapping between GL forecast superiority
and first order stochastic dominance (FOSD). In particular, $e_{1}$ is not
GL\ dominated by $e_{2}$ if $F_{1}(x)$ lies below $F_{2}(x)$ on the negative
real line, and lies above $F_{2}(x)$ on the positive real line. Indeed, this
ensures that we choose the forecast whose CDF has larger mass around zero.
Likewise, the second statement establishes a mapping between CL superiority
and second order stochastic dominance.

In this framework, it follows that testing for loss function robust forecast
superiority involves testing: 
\begin{equation}
H_{0}^{G}:\max_{j=2,...,k}\left( E(L(e_{1}))-E(L(e_{k}))\right) \leq 0\text{
for all }L\in L_{G}  \label{Hnull}
\end{equation}%
versus%
\begin{equation}
H_{A}^{G}:\max_{j=2,...,k}\left( E(L(e_{1}))-E(L(e_{k}))\right) >0\text{ for
some }L\in L_{G},  \label{HA}
\end{equation}%
with $H_{0}^{C}$ and $H_{A}^{C},$ defined analogously, replacing $L_{G}$
with $L_{C}.$

\noindent Hereafter, let $\mathcal{X}=\mathcal{X}^{-}\cup \mathcal{X}^{+}$
be the union of the support of $(e_{1},...,e_{k}).$ Given the equivalence
between $GL$ $(CL)$ forecast superiority and first (second) order stochastic
dominance, we can restate $H_{0}^{G},H_{A}^{G},$ $H_{0}^{C}$ and $H_{A}^{C}$
as%
\begin{eqnarray*}
H_{0}^{G} &=&H_{0}^{G-}\cap H_{0}^{G+} \\
&:&\left( F_{1}(x)-F_{j}(x)\leq 0,\text{ for }j=2,...,k,\text{ and for all }%
x\in \mathcal{X}^{-}\right) \\
&&\cap \left( F_{j}(x)-F_{1}(x)\leq 0,\text{ for }j=2,...,k,\text{ and for
all }x\in \mathcal{X}^{+}\right)
\end{eqnarray*}%
versus%
\begin{eqnarray*}
H_{A}^{G} &=&H_{A}^{G-}\cup H_{A}^{G+} \\
&:&\left( F_{1}(x)-F_{j}(x)>0,\text{ for some }j=2,...,k,\text{ and for some 
}x\in \mathcal{X}^{-}\right) \\
&&\cup \left( F_{j}(x)-F_{1}(x)>0,\text{ for some }j=2,...,k,\text{ and for
some }x\in \mathcal{X}^{+}\right) .
\end{eqnarray*}%
Analogously,%
\begin{eqnarray*}
H_{0}^{C} &=&H_{0}^{C-}\cap H_{0}^{C+} \\
&:&\left( \int_{-\infty }^{x}(F_{1}(t)-F_{j}(t))dt\leq 0,\text{ for }%
j=2,...,k\text{, and for all }x\in \mathcal{X}^{-}\right) \\
&&\cap \left( \int_{x}^{\infty }(F_{j}(t)-F_{1}(t))dt\leq 0,\text{ for }%
j=2,...,k,\text{ and for all }x\in \mathcal{X}^{+}\right)
\end{eqnarray*}%
versus%
\begin{eqnarray*}
H_{A}^{C} &=&H_{A}^{C-}\cup H_{A}^{C+} \\
&:&\left( \int_{-\infty }^{x}(F_{1}(t)-F_{j}(t))dt>0,\text{ for some }%
j=2,...,k,\text{ and for some }x\in \mathcal{X}^{-}\right) \\
&&\cup \left( \max_{j=2,...,k}\int_{x}^{\infty }(F_{j}(t)-F_{1}(t))dt>0,%
\text{ for some }j=2,...,k,\text{ and for some }x\in \mathcal{X}^{+}\right) .
\end{eqnarray*}%
It is immediate to see that $H_{0}^{G}$ and $H_{0}^{C}$ can be written as
the intersection of $(k-1)$ moment inequalities, which have to hold
uniformly over $\mathcal{X}.$ This gives rise to an infinite number of
moment conditions. Andrews and Shi (2013) develop tests for conditional
moment inequalities, and as is well known in the literature on consistent
specification testing (e.g., see Bierens (1982, 1990)) a finite number of
conditional moments can be transformed into an infinite number of
unconditional moments. The same is true in the case of weak inequalities.
Andrews and Shi (2017) consider tests for conditional stochastic dominance,
which are then characterized by an infinite number of conditional moment
inequalities and so by a \textquotedblleft twice\textquotedblright\ infinite
number of unconditional inequalities. Recalling that our interest is on
testing GL or CL forecast superiority as in (\ref{Hnull}) and (\ref{HA}), we
confine our attention to unconditional testing of stochastic dominance.

Because of the discontinuity at zero in the tests, $H_{0}^{G+}\left(
H_{0}^{C^{+}}\right) $ and $H_{0}^{G-}(H_{0}^{C-})$ should be tested
separately, and then one can use Holm (1979) bounds to control the two
resulting p-values (see Rules TG and TC in JCS (2017)). In the sequel, for
the sake of brevity, but without loss of generality, we focus our discussion
on testing $H_{0}^{G+}$ versus $H_{A}^{G+}$ and $H_{0}^{C+}$ versus $%
H_{A}^{C+}.$ However, when defining statistics, some discussion of the
statistics associated with the case where $x\in \mathcal{X}^{-}$ is also
given, when needed for clarity of exposition.

We begin by testing GL forecast superiority. Let $G^{+}(x)=\left(
G_{2}^{+}(x),...,G_{k}^{+}(x)\right) ,$ with $G_{j}^{+}(x)=F_{j}(x)-F_{1}(x)$%
, for $x\geq 0.$ Define the empirical analog of $G^{+}(x)$ as $%
G_{n}^{+}(x)=\left( G_{2,n}^{+}(x),...,G_{k,n}^{+}(x)\right) ,$ and for $%
x\geq 0,$ let%
\begin{equation}
G_{j,n}^{+}(x)=\widehat{F}_{j,n}(x)-\widehat{F}_{1,n}(x),  \label{Gn}
\end{equation}%
where $\widehat{F}_{j,n}(x)$ denotes the empirical CDF of $e_{j}.$
Similarly, let $C^{+}(x)=\left( C_{2}^{+}(x),...,C_{k}^{+}(x)\right) ,$ with 
$C_{j}^{+}(x)=\int_{x}^{\infty }\left( F_{j}(t)-F_{1}(t)\right) dt1(x\geq 0)$%
. Define the empirical analog of $C^{+}(x)$ as $C_{n}^{+}(x)=\left(
C_{2,n}^{+}(x),...,C_{k,n}^{+}(x)\right) ,$ and let%
\begin{eqnarray}
C_{j,n}^{+}(x) &=&\int_{x}^{\infty }\left( \widehat{F}_{j,n}(t)-\widehat{F}%
_{1,n}(t)\right) dt1(x\geq 0)  \label{Cn} \\
&=&\frac{1}{n}\sum_{t=1}^{n}\left( \left[ \left( e_{1,t}-x\right) \right]
_{+}-\left[ \left( e_{j,t}-x\right) \right] _{+}\right) ,  \notag
\end{eqnarray}%
\textit{where }$\left[ z\right] _{+}=\max \{0,z\}$\textit{. }Further, define

\begin{equation}
\Sigma ^{G+}\left( x,x^{\prime }\right) =\mathrm{acov}\left( \sqrt{n}%
G^{+}(x),\sqrt{n}G^{+}(x^{\prime })\right)  \label{SIGMA}
\end{equation}%
and%
\begin{equation}
\overline{\Sigma }_{n}^{G+}\left( x,x^{\prime }\right) =\widehat{\Sigma }%
_{n}^{G+}\left( x,x^{\prime }\right) +\varepsilon I_{k-1},  \label{SIGMA-bar}
\end{equation}%
where $\varepsilon \geq 0$, and where $\widehat{\Sigma }_{n}^{G+}\left(
x,x^{\prime }\right) $ is the sample analog of $\Sigma ^{G+}\left(
x,x^{\prime }\right) .$ In (\ref{SIGMA-bar}), the role of the additional $%
\varepsilon I_{k-1}$ term is to correct for the possible singularity of the
covariance estimator, for certain values of $x.$ This is the case when we
compare forecast errors from nested models. Let $\widehat{u}%
_{j,t}(x)=1\left\{ e_{j,t}\leq x\right\} -\frac{1}{n}\sum_{t=1}^{n}1\left\{
e_{j,t}\leq x\right\} $, so that the $jj-$th element of $\widehat{\Sigma }%
_{n}^{G+}\left( x,x\right) $ is given by%
\begin{eqnarray}
\widehat{\sigma }_{jj,n}^{2,G+}(x) &=&\frac{1}{n}\sum_{t=1}^{n}(\widehat{u}%
_{j,t}(x)-\widehat{u}_{1,t}(x))^{2}  \notag \\
&&+2\frac{1}{n}\sum_{\tau =1}^{l_{n}}\sum_{t=\tau +1}^{n}w_{\tau }(\widehat{u%
}_{j,t}(x)-\widehat{u}_{1,t}(x))(\widehat{u}_{j,t-\tau }(x)-\widehat{u}%
_{1,t-\tau }(x)),  \label{HAC}
\end{eqnarray}%
where $w_{\tau }=1-\frac{\tau }{1+l_{n}},$ with $l_{n}\rightarrow \infty $
as $n\rightarrow \infty .$\ Also, let $\overline{\sigma }%
_{jj,n}^{2,G^{+}}(x,x^{\prime })$ be the $jj-$th element of $\overline{%
\Sigma }_{n}^{G+}\left( x,x^{\prime }\right) ,$ and let $\overline{\sigma }%
_{jj,n}^{2,C+}(x,x^{\prime })$ be the $jj-$th element of $\overline{\Sigma }%
_{n}^{C+}\left( x,x^{\prime }\right) .$ Analogously, 
\begin{equation*}
\Sigma ^{C+}\left( x,x^{\prime }\right) =\mathrm{acov}\left( \sqrt{n}%
C^{+}(x),\sqrt{n}C^{+}(x^{\prime })\right)
\end{equation*}%
and%
\begin{equation*}
\overline{\Sigma }_{n}^{C+}\left( x,x^{\prime }\right) =\widehat{\Sigma }%
_{n}^{C+}\left( x,x^{\prime }\right) +\varepsilon I_{k-1},
\end{equation*}%
where$\widehat{\Sigma }_{n}^{C+}\left( x,x^{\prime }\right) $ is the sample
analog of $\Sigma ^{C+}\left( x,x^{\prime }\right) $

\noindent Furthermore, $\widehat{\sigma }_{jj,n}^{2,C+}(x)$ is constructed
by replacing $\widehat{u}_{1,t}(x)$ and $\widehat{u}_{j,t}(x)$ in the above
expression with 
\begin{equation*}
\widehat{\eta }_{1,t}(x)=\left[ \left( e_{1,t}-x\right) \right] _{+}-\frac{1%
}{n}\sum_{t=1}^{n}\left[ \left( e_{1,t}-x\right) \right] _{+}
\end{equation*}%
and 
\begin{equation*}
\widehat{\eta }_{j,t}(x)=\left[ \left( e_{j,t}-x\right) \right] _{+}-\frac{1%
}{n}\sum_{t=1}^{n}\left[ \left( e_{j,t}-x\right) \right] _{+}.
\end{equation*}%
Note that $G_{j,n}^{-}(x),C_{j,n}^{-}(x),$\ $\widehat{\sigma }%
_{jj,n}^{2,G-}(x),$ and $\widehat{\sigma }_{jj,n}^{2,C-}(x)$\ \ can be
defined by utilizing the $sgn$\ function.\textit{\ }Namely, regardless of
whether $x\geq 0$ or $x<0$, one can construct $G_{j,n}(x)=\left( \widehat{F}%
_{j,n}(x)-\widehat{F}_{1,n}(x)\right) sgn(x)$ and 
\begin{eqnarray*}
C_{j,n}(x) &=&\int_{-\infty }^{x}\left( \widehat{F}_{1,n}(t)-\widehat{F}%
_{j,n}(t)\right) dt1(x<0)-\int_{x}^{\infty }\left( \widehat{F}_{j,n}(t)-%
\widehat{F}_{1,n}(t)\right) dt1(x\geq 0) \\
&=&\frac{1}{n}\sum_{t=1}^{n}\left( \left[ \left( e_{1,t}-x\right) sgn(x)%
\right] _{+}-\left[ \left( e_{j,t}-x\right) sgn(x)\right] _{+}\right) ,
\end{eqnarray*}%
\begin{eqnarray*}
\widehat{\sigma }_{jj,n}^{2,G}(x) &=&\frac{1}{n}\sum_{t=1}^{n}(\widehat{u}%
_{j,t}(x)-\widehat{u}_{1,t}(x))^{2} \\
&&+2\frac{1}{n}\sum_{\tau =1}^{l_{n}}\sum_{t=\tau +1}^{n}w_{\tau }(\widehat{u%
}_{j,t}(x)-\widehat{u}_{1,t}(x))sgn(x)(\widehat{u}_{j,t-\tau }(x)-\widehat{u}%
_{1,t-\tau }(x))sgn(x),
\end{eqnarray*}

\noindent and $\widehat{\sigma }_{jj,n}^{2,C}(x)$ by replacing $\widehat{u}%
_{1,t}(x)$ and $\widehat{u}_{j,t}(x)$ in the above expression with 
\begin{equation*}
\widehat{\eta }_{1,t}(x)=\left[ \left( e_{1,t}-x\right) sgn(x)\right] _{+}-%
\frac{1}{n}\sum_{t=1}^{n}\left[ \left( e_{1,t}-x\right) sgn(x)\right] _{+}
\end{equation*}%
and 
\begin{equation*}
\widehat{\eta }_{j,t}(x)=\left[ \left( e_{j,t}-x\right) sgn(x)\right] _{+}-%
\frac{1}{n}\sum_{t=1}^{n}\left[ \left( e_{j,t}-x\right) sgn(x)\right] _{+}.
\end{equation*}%
Given the above framework, our new robust forecast superiority test
statistics are:%
\begin{equation}
S_{n}^{G+}=\underset{x\in \mathcal{X}^{+}}{\int }\underset{j=2}{\sum^{k}}%
\left( \max \left\{ 0,\sqrt{n}\frac{G_{j,n}(x)}{\overline{\sigma }%
_{jj,n}^{G}(x)}\right\} \right) ^{2}\text{ }dQ(x)\text{ and }S_{n}^{G-}=%
\underset{x\in \mathcal{X}^{-}}{\int }\underset{j=2}{\sum^{k}}\left( \max
\left\{ 0,\sqrt{n}\frac{G_{j,n}(x)}{\overline{\sigma }_{jj,n}^{G}(x)}%
\right\} \right) ^{2}dQ(x)  \label{SnG}
\end{equation}%
and%
\begin{equation}
S_{n}^{C+}=\underset{x\in \mathcal{X}^{+}}{\int }\underset{j=2}{\sum^{k}}%
\left( \max \left\{ 0,\sqrt{n}\frac{C_{j,n}(x)}{\overline{\sigma }%
_{jj,n}^{C}(x)}\right\} \right) ^{2}dQ(x)\text{ and }S_{n}^{C-}=\underset{%
x\in \mathcal{X}^{-}}{\int }\underset{j=2}{\sum^{k}}\left( \max \left\{ 0,%
\sqrt{n}\frac{C_{j,n}(x)}{\overline{\sigma }_{jj,n}^{C}(x)}\right\} \right)
^{2}dQ(x),  \label{SnC}
\end{equation}%
where $Q$ is a weighting function defined below; $G_{j,n}^{+}(x)$ and $%
C_{j,n}^{+}(x)$ are the $j-$th components of $G_{n}^{+}(x)$ and $%
C_{n}^{+}(x),$ as defined in (\ref{Gn}) and (\ref{Cn}), respectively$.$
Here, $S_{n}^{G+}$ and $S_{n}^{C+}$ are \textquotedblleft
sum\textquotedblright\ functions, as in equation (3.8) in Andrews and Shi
(2013), and satisfy their Assumptions S1-S4, which are required to guarantee
that convergence is uniform over the null DGPs.\footnote{%
Note that we could have constructed a different \textquotedblleft
sum\textquotedblright\ function, using the statistic in (3.9) of Andrews and
Shi (2013).},\footnote{%
Recall that one main drawback of the $\max_{j=2,...,k}\sup_{x\in \mathcal{X}%
^{+}}\sqrt{n}G_{n}^{+}$ statistic in JCS (2017) is that it diverges to $%
-\infty $ under some sequence of probability measures under the null, thus
ruling out uniformity.} If $k=2$ and $\overline{\sigma }_{jj,n}(x)=1,$ for
all $j$ and $x$ (i.e. no standardization), then $S_{n}^{G+}$ is the
statistic used in Linton, Song and Whang (2010) for testing FOSD.

Of note is that in our context, potential slackness causes a discontinuity
in the pointwise asymptotic distribution of the statistic.\footnote{%
By pointwise asymptotic distribution we mean the limiting distribution under
a fixed probability measure.} This is because the pointwise asymptotic
distribution is discontinuous, unless all moment conditions hold with
equality. On the other hand, the finite sample distribution is not
necessarily discontinuous. Thus, in the presence of slackness, the pointwise
limiting distribution is not a good approximation of the finite sample
distribution, and critical values based on pointwise asymptotics may be
invalid. This is why we construct tests that are uniformly asymptotically
valid (i.e., this is why we study the limiting distribution of our tests
under drifting sequences of probability measures belonging to the null
hypothesis). Moreover, in the infinite dimensional case, there is an
additional source of discontinuity. In particular, the number of moment
inequalities which contributes to the statistic varies across the different
values of $x.$ For example, the key difference between the case of $k=2$ and 
$k>2$ is that in the former case, for each value of $x$ there is only one
moment inequality which can be binding (or not). On the other hand, if $k=3$%
, say, then for each value of $x$ there can be either one or two moment
inequalities which may be binding (or not), and whether or not a particular
inequality is binding (or not) varies over $x$. Under this setup, we require
the following assumptions in order to analyze the asymptotic behavior of our
test statistics.

\noindent \textbf{Assumption A1}: For $j=1,...,k,$ $e_{j,t}$ is strictly
stationary and $\beta -$mixing, with mixing coefficients, $a_{m}=m^{-\beta
}, $ where $\beta >\frac{6\delta }{1-2\delta },$ $0<\delta <1/2$ and $\beta
\delta >1.$

\noindent \textbf{Assumption A2}: The union of the supports of $%
e_{1},..,e_{k}$ is the compact set, $\mathcal{X}=\mathcal{X}^{-}\cup 
\mathcal{X}^{+}$.

\noindent \textbf{Assumption A3:} $F_{j}(x)$ has a continuos bounded density.

\noindent \textbf{Assumption A4}: The weighting function $Q$ has full
support $\mathcal{X}^{+}$ (or $\mathcal{X}^{-}\,).$

We use Assumption A2 in the proof of Lemma 1, where we require $\mathcal{X}%
^{+}$ in (\ref{SnG}) and (\ref{SnC}) to be a compact set. However, for the
case of generalized loss superiority, the union of the supports of $%
e_{1},..,e_{k}$ can be unbounded. This is because $S_{n}^{G+}$ is bounded,
regardless of the boundedness of the support. On the other hand, $%
S_{n}^{C^{+}}$ is bounded only when the union of the support of the
forecasting error is bounded.

\section{Asymptotic Properties}

\subsection{\noindent Uniform Convergence of the HAC Estimator}

We now turn to a discussion of the estimation of the variance in our
forecast superiority test statistics. If $e_{1,t},...,e_{k,t}$ were
martingale difference sequences, then we can still use the sample second
moment as a variance estimator, and uniform consistency will follow by
application of an appropriate uniform law of large numbers. In our set-up we
can assume that $e_{1},...,e_{k}$ are martingale difference sequences if
either: (i) they are judgmental forecasts from professional forecasters,
say, who efficiently use all available information at time $t$ (a strong
assumption, which is tested in the forecast rationality literature)$;$ or
(ii) they are prediction errors from one-step ahead forecasts based on
dynamically correctly specified models. With respect to (i), it is worth
noting that professional forecasters may be rational, ex-post, according to
some loss function (see Elliott, Komunjer and Timmermann (2005,2008),
although it is not as likely that they are rational according to a
generalized loss function. With respect to (ii), it should be noted that at
most one model can be dynamically correctly specified for a given
information set, and thus $e_{j}$ cannot be a martingale difference
sequence, for all $j=1,...,k.$ In light of these facts, we allow for time
dependence in the forecast error sequences used in our statistics, and use a
HAC variance estimator in (\ref{SnG}) and (\ref{SnC}). In order to ensure
that the HAC estimators converge uniformly over $\mathcal{X}^{+},$ it
suffices to establish the counterpart of Lemma A1 of Supplement A of Andrews
and Shi (2013) to the case of mixing sequences. This is done below.

\noindent \textbf{Lemma 1}\textit{: Let Assumptions A1-A3 hold. Then, if }$%
l_{n}\approx n^{\delta }$\textit{\ }$0<\delta <\frac{1}{2},$\textit{\ with }$%
\delta $\textit{\ defined as in Assumption A1:}

\textit{\noindent (i)}%
\begin{equation*}
\sup_{x\in \mathcal{X}^{+}}\left\vert \widehat{\sigma }_{jj,n}^{2,G+}(x)-%
\sigma _{jj}^{2,G+}(x)\right\vert =o_{p}\left( 1\right) ,
\end{equation*}%
\textit{with }$\sigma _{jj}^{2,G+}(x)=avar\left( \sqrt{n}G_{j,n}^{+}(x)%
\right) ;$\textit{\ and}

\textit{\noindent (ii)}%
\begin{equation*}
\sup_{x\in \mathcal{X}^{+}}\left\vert \widehat{\sigma }_{jj,n}^{2,C+}(x)-%
\sigma _{jj}^{2,C+}(x)\right\vert =o_{p}\left( 1\right) ,
\end{equation*}%
\textit{with }$\sigma _{jj}^{2C+}(x)=avar\left( \sqrt{n}C_{j,n}^{+}(x)%
\right) .$

Lemma 1 establishes the uniform convergence over $\mathcal{X}^{+}$ of HAC
estimators. It is the time series counterpart of Lemma A1 in Andrews and Shi
(2013). Of note is that we require $\beta -$mixing. This differs from the
stationary pointwise HAC variance estimator case studied by Andrews (1991),
where $\alpha -$mixing suffices, and where the mixing coefficients decline
to zero slightly slower than in our Assumption A1. This is because there is
a trade-off between the degree of dependence and the rate of growth of the
lag truncation parameter in the HAC estimator. Indeed, in the uniform case,
the covering number (e.g., see Andrews and Pollard (1994)) grows with both $%
l_{n}$ and the degree of dependence, thus leading to a trade-off between the
two. For example, in the case of exponential mixing series, $\delta $ can be
arbitrarily close to $1/2.$

For carrying out inference on our forecast superiority tests, we require a
bootstrap analog of the HAC variance estimator, which can be constructed as
follows. Using the block bootstrap, make $b_{n}$ draws of length $l_{n}$
from $e_{j,1},...,e_{j,n},$ in order to obtain $\left( e_{j,1}^{\ast
},...,e_{j,n}^{\ast }\right) =\left(
e_{j,I_{1+1}},...,e_{j,I_{1+l_{n}}},...,e_{j,I_{b_{n}+l_{n}}}\right) ,$ with 
$b_{n}l_{n}=n,$ where the block size, $l_{n},$ is equal to the lag
truncation parameter in the HAC estimator described above.\footnote{%
We thus use the same notation, $l_{n},$ for both the lag truncation
parameter and the block length.} Now, let $u_{1,t}^{\ast }(x)=1\left\{
e_{1,t}^{\ast }\leq x\right\} -\frac{1}{n}\sum_{t=1}^{n}1\left\{ e_{1,t}\leq
x\right\} ,$ $u_{j,t}^{\ast }(x)=1\left\{ e_{j,t}^{\ast }\leq x\right\} -%
\frac{1}{n}\sum_{t=1}^{n}1\left\{ e_{j,t}\leq x\right\} ,$ and%
\begin{equation}
\widehat{\sigma }_{jj,n}^{2\ast G+}(x)=\frac{1}{b_{n}}\sum_{k=1}^{b_{n}}%
\left( \frac{1}{l_{n}^{1/2}}\sum_{i=1}^{l_{n}}\left(
u_{j,(k-1)l_{n}+i}^{\ast }(x)-u_{1,(k-1)l_{n}+i}^{\ast }(x)\right) \right)
^{2}.  \label{sigmaG*}
\end{equation}%
Define $\widehat{\sigma }_{jj,n}^{\ast 2C^{+}}(x)$ analogously, replacing $%
u_{1,t}^{\ast }(x)$ with $\eta _{1,t}^{\ast }(x)=\left[ e_{1,t}^{\ast }-x%
\right] _{+}-\frac{1}{n}\sum_{t=1}^{n}1\left[ e_{1,t}^{\ast }-x\right] _{+}$
and $u_{j,t}^{\ast }(x)$ with $\eta _{j,t}^{\ast }(x)=\left[ e_{j,t}^{\ast
}-x\right] _{+}-\frac{1}{n}\sum_{t=1}^{n}1\left[ e_{j,t}^{\ast }-x\right]
_{+}.$ Additionally, define

\begin{equation*}
\widehat{\sigma }_{jj^{\prime },n}^{2\ast G+}(x)=\frac{1}{b_{n}}%
\sum_{k=1}^{b_{n}}\left( \frac{1}{l_{n}^{1/2}}\sum_{i=1}^{l_{n}}\left(
u_{j,(k-1)l_{n}+i}^{\ast }(x)-u_{1,(k-1)l_{n}+i}^{\ast }(x)\right) \left(
u_{j^{\prime },(k-1)l_{n}+i}^{\ast }(x)-u_{1,(k-1)l_{n}+i}^{\ast }(x)\right)
\right) .
\end{equation*}

The following result holds.

\noindent \textbf{Lemma 2: }\textit{Let Assumptions A1-A3 hold. Then, if }$%
l_{n}\approx n^{\delta }$\textit{\ }$0<\delta <\frac{1}{2},$\textit{\ with }$%
\delta $\textit{\ defined as in Assumption A1:}

\textit{\noindent (i)}%
\begin{equation*}
\sup_{x\in \mathcal{X}^{+}}\left\vert \widehat{\sigma }_{jj,n}^{\ast G+}(x)-%
\mathrm{E}^{\ast }\left( \widehat{\sigma }_{jj,n}^{\ast G+}(x)\right)
\right\vert =o_{p}^{\ast }\left( 1\right) ,
\end{equation*}%
\textit{and (ii)}%
\begin{equation*}
\sup_{x\in \mathcal{X}^{+}}\left\vert \widehat{\sigma }_{jj,n}^{\ast C+}(x)-%
\mathrm{E}^{\ast }\left( \widehat{\sigma }_{jj,n}^{\ast C+}(x)\right)
\right\vert =o_{p}^{\ast }\left( 1\right) ,
\end{equation*}%
\textit{where }$o_{p}^{\ast }(1)$\textit{\ denotes convergence to zero
according to the bootstrap law, }$P^{\ast },$\textit{\ conditional on the
sample.}

As in our above discussion, when constructing bootstrap counterparts for the
statistics defined in (\ref{SnG}) and (\ref{SnC}) on both the positive and
negatives supports of $X$\ , it suffices to utilize the $sgn$\ function, and
note that $sgn(x)^{2}=1$. For example, replace $\eta _{1,t}^{\ast }(x)$\
with $\eta _{1,t}^{\ast }(x)=\left[ (e_{1,t}^{\ast }-x)sgn(x)\right] _{+}-%
\frac{1}{n}\sum_{t=1}^{n}\left[ (e_{1,t}^{\ast }-x)sgn(x)\right] _{+},$\
replace $\eta _{j,t}^{\ast }(x)$\ with $\eta _{j,t}^{\ast }(x)=\left[
(e_{j,t}^{\ast }-x)sgn(x)\right] _{+}-\frac{1}{n}\sum_{t=1}^{n}\left[
(e_{j,t}^{\ast }-x)sgn(x)\right] _{+},$\ and define

\begin{equation*}
\widehat{\sigma }_{jj^{\prime },n}^{2\ast G}(x)=\frac{1}{b_{n}}%
\sum_{k=1}^{b_{n}}\left( \frac{1}{l_{n}^{1/2}}\sum_{i=1}^{l_{n}}\left(
u_{j,(k-1)l_{n}+i}^{\ast }(x)-u_{1,(k-1)l_{n}+i}^{\ast }(x)\right) \left(
u_{j^{\prime },(k-1)l_{n}+i}^{\ast }(x)-u_{1,(k-1)l_{n}+i}^{\ast }(x)\right)
\right)
\end{equation*}%
and%
\begin{equation*}
\widehat{\sigma }_{jj^{\prime },n}^{2\ast C}(x)=\frac{1}{b_{n}}%
\sum_{k=1}^{b_{n}}\left( \frac{1}{l_{n}^{1/2}}\sum_{i=1}^{l_{n}}\left( \eta
_{j,(k-1)l_{n}+i}^{\ast }(x)-\eta _{1,(k-1)l_{n}+i}^{\ast }(x)\right) \left(
\eta _{j^{\prime },(k-1)l_{n}+i}^{\ast }(x)-\eta _{1,(k-1)l_{n}+i}^{\ast
}(x)\right) \right) .
\end{equation*}

\subsection{Inference Using the Bootstrap and Bounding Limiting Distributions%
}

The statistics $S_{n}^{G+}$ and $S_{n}^{C+}$ are highly discontinuous over $%
x.$ Exactly which moment conditions, and how many of them are binding varies
over $x.$ Hence, $S_{n}^{G+}$ and $S_{n}^{C+}$ do not necessarily have a
well defined limiting distribution; and the continuous mapping theorem
cannot be applied. However, following the generalized moment selection (GMS)
test approach of Andrews and Shi (2013) we can establish lower and upper
bound limiting distributions. Let%
\begin{equation*}
D^{G+}(x)=\mathrm{diag}\Sigma ^{G+}\left( x,x\right) ,
\end{equation*}%
\begin{equation}
h_{A,n}^{G+}(x)=D^{G+}(x)^{-1/2}\left( \sqrt{n}G_{2}^{+}(x),...,\sqrt{n}%
G_{k}^{+}(x)\right) ^{\prime },  \label{ha}
\end{equation}

\begin{equation}
h_{B}^{G+}(x,x^{\prime })=D^{G+}(x)^{-1/2}\left( \Sigma ^{G+}+\varepsilon
I_{k-1}\right) \left( x,x^{\prime }\right) D^{G+}(x^{\prime })^{-1/2},
\label{hB}
\end{equation}%
and%
\begin{equation}
v^{G+}(.)=(v_{2}^{G+}(.),...,v_{k}^{G+}(.))^{\prime },  \label{vu}
\end{equation}%
where $v^{G+}(.)$ is a ($k-1)-$dimensional zero mean Gaussian process with
correlation $h_{B}^{G+}(x,x^{\prime })$. Also, let $%
D^{C+}(x),h_{A,n}^{C+}(x),h_{B}^{C+}(x,x^{\prime }),v^{C+}(.)$ be defined
analogously, by replacing $\Sigma ^{G+}\left( x,x\right)
,G_{2}^{+}(x),...,G_{k}^{+}(x)$ with $\Sigma ^{C+}\left( x,x\right)
,C_{2}^{+}(x),...,C_{k}^{+}(x).$ Finally, define%
\begin{equation}
S_{n}^{\dag G+}=\int_{\mathcal{X}^{+}}\sum_{j=2}^{k}\left( \max \left\{ 0,%
\frac{v_{j}^{G+}(x)+h_{j,A,n}^{G+}(x)}{\sqrt{h_{jj,B}^{G+}(x)}}\right\}
\right) ^{2}\mathrm{d}Q(x)  \label{Sn-dag}
\end{equation}%
where $h_{jj,B}^{G+}(x)$ is the $jj-$th element of $h_{B}^{G+}(x,x)$, and let%
\begin{equation}
S_{\infty }^{\dag G+}=\int_{\mathcal{X}^{+}}\sum_{j=2}^{k}\left( \max
\left\{ 0,\frac{v_{j}^{G+}(x)+h_{j,A,\infty }^{G+}(x)}{\sqrt{h_{jj,B}^{G+}(x)%
}}\right\} \right) ^{2}\mathrm{d}Q(x),  \label{Sn-inf}
\end{equation}%
where $h_{j,A,\infty }^{G+}(x)=0,$ if $G_{j}(x)=0,$ and $h_{j,A,\infty
}^{G+}(x)=-\infty $ , if $G_{j}(x)<0.$ Also, define $S_{n}^{\dag C+}$ and $%
S_{\infty }^{\dag C+}$ analogously, by replacing $%
v_{j}^{G+}(x),h_{j,A,n}^{G+}(x),h_{j,A,\infty }^{G+}(x),$ and $%
h_{jj,B}^{G+}(x)$ with $v_{j}^{C+}(x),h_{j,A,n}^{C+}(x),h_{j,A,\infty
}^{C+}(x),$ and $h_{jj,B}^{C+}(x).$ Hereafter let%
\begin{equation*}
\mathcal{P}_{0}^{G+}=\left\{ P:\text{ }H_{0}^{G+}\text{ holds}\right\}
\end{equation*}%
so that $\mathcal{P}_{0}^{G+}$ is the collection of DGPs under which the
null hypothesis holds. Let $\mathcal{P}_{0}^{C+}$ be defined analogously,
with $H_{0}^{G+}$ replaced by $H_{0}^{C+}.$ The following result holds.

\noindent \textbf{Theorem 1: }\textit{Let Assumptions A1-A4 hold. Then:}

\textit{\noindent (i) under }$H_{0}^{G+},$\textit{\ there exists a }$\delta
>0$\textit{\ such that}%
\begin{equation*}
\lim \sup_{n\rightarrow \infty }\sup_{P\in \mathcal{P}_{0}^{G+}}\left[
P\left( S_{n}^{G+}>a_{h_{A,n}}^{G+}\right) -P\left( S_{n}^{\dag G+}+\delta
>a_{h_{A,n}}^{G+}\right) \right] \leq 0
\end{equation*}%
\textit{and}%
\begin{equation*}
\lim \inf_{n\rightarrow \infty }\inf_{P\in \mathcal{P}_{0}^{G+}}\left[
P\left( S_{n}^{G+}>a_{h_{A,n}}^{G+}\right) -P\left( S_{n}^{\dag G+}-\delta
>a_{h_{A,n}}^{G+}\right) \right] \geq 0;
\end{equation*}%
\textit{and }

\textit{\noindent (ii) under }$H_{0}^{C+},$\textit{\ there exists a }$\delta
>0$\textit{\ such that}%
\begin{equation*}
\lim \sup_{n\rightarrow \infty }\sup_{P\in \mathcal{P}_{0}^{C+}}\left[
P\left( S_{n}^{C+}>a_{h_{A,n}}^{C+}\right) -P\left( S_{n}^{\dag C+}+\delta
>a_{h_{A,n}}^{C+}\right) \right] \leq 0
\end{equation*}%
\textit{and}%
\begin{equation*}
\lim \inf_{n\rightarrow \infty }\inf_{P\in \mathcal{P}_{0}^{C+}}\left[
P\left( S_{n}^{C+}>a_{h_{A,n}}^{C+}\right) -P\left( S_{n}^{\dag C+}-\delta
>a_{h_{A,n}}^{C+}\right) \right] \geq 0.
\end{equation*}%
Theorem 1 provides upper and lower bounds for $P\left(
S_{n}^{G+}>a_{h_{A,n}}^{G+}\right) $ and $P\left(
S_{n}^{C+}>a_{h_{A,n}}^{C+}\right) ,$ uniformly, over the probabilities
under $H_{0}^{G+}$ and $H_{0}^{C+},$ respectively. Note that $%
h_{j,A,n}^{G+}( $\textperiodcentered $)$ and $h_{j,A,n}^{C+}($%
\textperiodcentered $)$ depend on the degree of slackness, and do not need
to converge. Indeed, $S_{n}^{G+}$ and/or $S_{n}^{C+}$ do not have to
converge in distribution for this result to hold.

Following Andrews and Shi (2013), we can construct bootstrap critical values
which properly mimic the critical values of $S_{\infty }^{\dag G+}$ and $%
S_{\infty }^{\dag C+}.$ We rely on the block bootstrap to capture the
dependence in the data when constructing our bootstrap statistics. Consider
the case of $S_{\infty }^{\dag G+}.$ Let $\left( e_{j,1}^{\ast
},...,e_{j,n}^{\ast }\right) ,b_{n},$ and $l_{n}$ be defined as in the
previous subsection, and let:%
\begin{equation}
G_{j,n}^{\ast +}(x)=\frac{1}{n}\sum_{i=1}^{n}\left( 1\left\{ e_{j,i}^{\ast
}\leq x\right\} -1\left\{ e_{1,i}^{\ast }\leq x\right\} \right)  \label{G*}
\end{equation}%
and%
\begin{equation}
v_{n}^{\ast G+}(x)=\sqrt{n}\widehat{D}_{n}^{-1/2,G+}(x)\left( G_{n}^{\ast
+}(x)-G_{n}^{+}(x)\right)  \label{vu*}
\end{equation}

\noindent with $v_{n}^{\ast G+}(x)=\left( v_{2,n}^{\ast
G+}(x),...,v_{k,n}^{\ast G+}(x)\right) $ and $\widehat{D}_{n}^{G+}(x)=%
\mathrm{diag}\widehat{\Sigma }_{n}^{G+}\left( x,x\right) .$ Then, define:%
\begin{equation}
\xi _{j,n}^{G+}(x)=\kappa _{n}^{-1}n^{1/2}\overline{D}%
_{jj,n}^{-1/2,G+}(x)G_{j,n}^{+}(x),  \label{xi}
\end{equation}%
with $\kappa _{n}\rightarrow \infty ,$ as $n\rightarrow \infty .$ Here, $%
\overline{D}_{jj,n}^{G+}(x)$ is the $jj$-th element of $\overline{D}%
_{n}^{G+}(x)=\mathrm{diag}\left( \overline{\Sigma }_{n}^{G+}\left(
x,x\right) \right) ,$ $\xi _{n}^{G+}(x)=\left( \xi _{2,n}^{G+}(x),...,\xi
_{k,n}^{G+}(x)\right) ,$ and%
\begin{equation}
\phi _{j,n}^{G+}(x)=c_{n}1\left\{ \xi _{j,n}^{G+}(x)<-1\right\} ,
\label{phi}
\end{equation}%
with $c_{n}$ a positive sequence, which is bounded away from zero. Thus, $%
\phi _{j,n}^{G+}(x)=c_{n},$ when $G_{j,n}^{+}(x)<-\kappa _{n}n^{-1/2}%
\overline{D}_{jj,n}^{1/2,G+}(x)$ (i.e., when the $j-$th inequality is slack
at $x),$ and is zero otherwise.

It it clear from the selection rule in (\ref{phi}), that we do need an
estimator of the variance of the moment conditions, despite the fact we use
bootstrap critical values. In fact, standardization does not play a crucial
role in the statistics, as all positive sample moment conditions matter. On
the other hand, without the scaling factor in (\ref{xi}), the number of
non-slack moment conditions would depend on the scale, and hence our
bootstrap critical values would no longer be scale invariant. Let%
\begin{equation}
S_{n}^{\ast G+}=\int_{\mathcal{X}^{+}}\sum_{j=2}^{k}\max \left( \left\{ 0,%
\frac{v_{j,n}^{\ast G+}(x)-\phi _{j,n}^{G+}(x)}{\sqrt{\overline{h}%
_{B,jj}^{\ast G+}(x)}}\right\} \right) ^{2}\mathrm{d}Q(x),  \label{Sn*}
\end{equation}%
where $\overline{h}_{B,jj}^{\ast G+}(x)$ is the $jj-th$ element of $%
\overline{D}_{n}^{-1/2,G+}(x)\overline{\Sigma }_{n}^{\ast G+}\left(
x,x\right) \overline{D}_{n}^{-1/2,G+}(x)$ and $\overline{\Sigma }_{n}^{\ast
G+}\left( x,x\right) $ is the bootstrap analog of $\overline{\Sigma }%
_{n}^{G+}\left( x,x\right) .$\footnote{%
Thus, the diagonal elements of $\widehat{\Sigma }_{n}^{\ast G+}\left(
x,x\right) $ are the $\widehat{\sigma }_{jj,n}^{2\ast G+}(x)$ described in
the previous subsection, while the off-diagonal elements of $\widehat{\Sigma 
}_{n}^{\ast G+}\left( x,x\right) $ are defined accordingly, as $\widehat{%
\sigma }_{jj^{\prime },n}^{2\ast G+}(x)$, with $j\neq j^{\prime }.$} Note
that if $c_{n}$ grows with $n,$ then all slack inequalities are discarded,
asymptotically. It is immediate to see that $S_{n}^{\ast G+}$ is the
bootstrap counterpart of $S_{n}^{\dag G+}$ in (\ref{Sn-dag}), with $\phi
_{j,n}^{G+}(x)$ mimicking the contribution of the slackness of inequality $j$
(i.e., of $j-$th element of $h_{A,n}^{G+}(x)).$ However, $\phi
_{j,n}^{G+}(x) $ is not a consistent estimator of $h_{A,n}^{G+}(x),$ since
the latter cannot be consistently estimated.

\noindent Now, consider the case of $S_{\infty }^{\dag C+}.$ Let:%
\begin{equation*}
C_{j,n}^{\ast +}(x)=\frac{1}{n}\sum_{i=1}^{n}\left( \left[ e_{j,t}^{\ast }-x%
\right] _{+}-\left[ e_{1,t}^{\ast }-x\right] _{+}\right) ,
\end{equation*}%
and define $v_{n}^{\ast C+}(x),\widehat{D}_{n}^{C+}(x),\xi _{j,n}^{C+}(x),$
and $\phi _{j,n}^{C+}(x)$ analogously to $v_{n}^{\ast G+}(x),\widehat{D}%
_{n}^{G+}(x),\xi _{j,n}^{G+}(x),$ and $\phi _{j,n}^{G+}(x),$ by replacing $%
G_{n}^{\ast +}(x),G_{n}^{+}(x)$ and $\widehat{\Sigma }_{n}^{G+}\left(
x,x\right) $ with $C_{n}^{\ast +}(x),C_{n}^{+}(x)$ and $\widehat{\Sigma }%
_{n}^{C+}\left( x,x\right) .$ Then, construct:%
\begin{equation}
S_{n}^{\ast C+}=\int_{\mathcal{X}^{+}}\sum_{j=2}^{k}\max \left( \left\{ 0,%
\frac{v_{j,n}^{\ast C+}(x)-\phi _{j,n}^{C+}(x)}{\sqrt{\overline{h}%
_{B,jj}^{\ast C+}(x)}}\right\} \right) ^{2}\mathrm{d}Q(x).  \label{SnC*}
\end{equation}%
By comparing (\ref{SnG}) and (\ref{SnC}) with (\ref{Sn*}) and (\ref{SnC*}),
it is immediate to see that $G_{j,n}^{+}(x)$ $(C_{j,n}^{+}(x))$ does not
contribute to the test statistic when $G_{j,n}^{+}(x)<0,$ $%
(C_{j,n}^{+}(x)<0) $ while it does not contribute to the bootstrap statistic
when $G_{j,n}^{+}(x)<-\kappa _{n}n^{-1/2}\overline{D}_{jj,n}^{1/2,G+}(x)$ $%
(C_{j,n}^{+}(x)<-\kappa _{n}n^{-1/2}\overline{D}_{jj,n}^{1/2,C+}(x))$ with $%
\kappa _{n}n^{-1/2}\rightarrow 0.$ Heuristically, by letting $\kappa _{n}$
grow with the sample size, we control the rejection rates in a uniform
manner.

\noindent It remains to define the GMS bootstrap critical values. Let $%
c_{n,B,1-\alpha }^{\ast G+}\left( \phi _{n}^{G+},\overline{h}_{B,n}^{\ast
G+}\right) $ be the $(1-\alpha )$-th \ critical value of $S_{n}^{\ast G+},$
based on $B$ bootstrap replications, with $\phi _{n}^{G+}$ defined as in (%
\ref{phi}) and $\overline{h}_{B,n}^{\ast G+}(x)=\widehat{D}_{n}^{-1/2,G+}(x)%
\overline{\Sigma }_{n}^{\ast G+}\left( x,x\right) \widehat{D}%
_{n}^{-1/2,G+}(x)$. The ($1-\alpha )$-th GMS bootstrap critical value, $%
c_{0,n,1-\alpha }^{\ast G+}\left( \phi _{n}^{G+},\overline{h}_{B,n}^{\ast
G+}\right) ,$ is defined as:%
\begin{equation*}
c_{0,n,1-\alpha }^{\ast G+}\left( \phi _{n}^{G+},\overline{h}_{B,n}^{\ast
G+}\right) =\lim_{B\rightarrow \infty }c_{n,B,1-\alpha +\eta }^{\ast
G+}\left( \phi _{n}^{G+},\overline{h}_{B,n}^{\ast G+}\right) +\eta ,
\end{equation*}%
for $\eta >0,$ arbitrarily small. Further, $c_{n,B,1-\alpha }^{\ast
C+}\left( \phi _{n}^{C+},\overline{h}_{B,n}^{\ast C+}\right) $ and $%
c_{0,n,1-\alpha }^{\ast C+}\left( \phi _{n}^{C+},\overline{h}%
_{B,n}^{C+}\right) $ are defined analogously.

Here, the constant $\eta $ is used to guarantee uniformity over the infinite
dimensional nuisance parameters, $h_{A,n}^{G+}(.),h_{A,n}^{C+}(.),$
uniformly on $x\in \mathcal{X}^{+},$ and is termed the infinitesimal
uniformity factor by Andrews and Shi (2013). Heuristically, if all moment
conditions are slack, then both the statistic and its bootstrap counterpart
are zero, and by having $\eta >0$ though arbitrarily close to zero we
control the asymptotic rejection rate.

Finally, let%
\begin{equation}
\mathcal{B}^{G+}=\left\{ x\in \mathcal{X}^{+}\text{ s.t. }h_{A,j,\infty
}^{G+}=0,\text{ for some }j=2,...,k\right\}  \label{BG+}
\end{equation}%
and%
\begin{equation}
\mathcal{B}^{C+}=\left\{ x\in \mathcal{X}^{+}\text{ s.t. }h_{A,j,\infty
}^{C+}=0,\text{ for some }j=2,...,k\right\} ,  \label{BC+}
\end{equation}%
where $\mathcal{B}^{G+}$ and $\mathcal{B}^{C+}$ define the sets over which
at least one moment condition holds with strict equality, and these sets
represent the boundaries of $H_{0}^{G+}$ and $H_{0}^{C+},$ respectively.

Although we require that the block length grows at the same rate as the lag
truncation parameter, $l_{n},$ in Lemma 2 (i.e., we require that $%
l_{n}\approx n^{\delta }$ $0<\delta <\frac{1}{2}$ with $\delta $ being the
mixing coefficient in A1), for the asymptotic uniform validity of the
bootstrap critical values, we require that the block length grows at a rate
slower than $n^{1/3}.$ This slower rate is required for the bootstrap
empirical central limit theorem for a mixing process to hold (see Peligrad
(1998)). Needless to say, even in the construction of $\widehat{\sigma }%
_{jj,n}^{2,G+}(x)$, we should thus use $l_{n}=o(n^{1/3})$ $.$ The following
result holds.

\noindent \textbf{Theorem 2: }\textit{Let Assumptions A1-A4 hold, and let }$%
l_{n}\rightarrow \infty $\textit{\ and }$l_{n}n^{\frac{1}{3}-\varepsilon
}\rightarrow 0$\textit{\ as }$n\rightarrow \infty .$ \textit{Under }$%
H_{0}^{G+}:$

\textit{\noindent (i) if as }$n\rightarrow \infty ,$\textit{\ }$\kappa
_{n}\rightarrow \infty $\textit{\ and }$c_{n}/\kappa _{n}\rightarrow 0,$%
\textit{\ then}%
\begin{equation*}
\lim \sup_{n\rightarrow \infty }\sup_{P\in \mathcal{P}_{0}^{G+}}P\left(
S_{n}^{G+}\geq c_{0,n,1-\alpha }^{\ast G+}\left( \phi _{n}^{G+},\overline{h}%
_{B,n}^{\ast G+}\right) \right) \leq \alpha ;
\end{equation*}%
\textit{and }

\textit{\noindent (ii) if as }$n\rightarrow \infty ,$\textit{\ }$\kappa
_{n}\rightarrow \infty ,$\textit{\ }$c_{n}\rightarrow \infty $\textit{, }$%
\sqrt{n}/\kappa _{n}\rightarrow \infty ,$\textit{\ and }$Q\left( \mathcal{B}%
^{G+}\right) >0,$\textit{\ then}%
\begin{equation*}
\lim_{\eta \rightarrow 0}\lim \sup_{n\rightarrow \infty }\sup_{P\in \mathcal{%
P}_{0}^{G+}}P\left( S_{n}^{G+}\geq c_{0,n,1-\alpha }^{\ast G+}\left( \phi
_{n}^{G+},\overline{h}_{B,n}^{\ast G+}\right) \right) =\alpha .
\end{equation*}%
\textit{\noindent Also, under }$H_{0}^{C+}:$

\textit{\noindent (iii) if as }$n\rightarrow \infty ,$\textit{\ }$\kappa
_{n}\rightarrow \infty $\textit{\ and }$c_{n}/\kappa _{n}\rightarrow 0,$%
\textit{\ then}%
\begin{equation*}
\lim \sup_{n\rightarrow \infty }\sup_{P\in \mathcal{P}_{0}^{C+}}P\left(
S_{n}^{C+}\geq c_{0,n,1-\alpha }^{\ast C+}\left( \phi _{n}^{C+},\overline{h}%
_{B,n}^{\ast C+}\right) \right) \leq \alpha ;
\end{equation*}%
\textit{\noindent and (iv) if as }$n\rightarrow \infty ,$\textit{\ }$\kappa
_{n}\rightarrow \infty ,$\textit{\ }$c_{n}\rightarrow \infty $\textit{, }$%
\sqrt{n}/\kappa _{n}\rightarrow \infty ,$\textit{\ and }$Q\left( \mathcal{B}%
^{C+}\right) >0,$\textit{\ then}%
\begin{equation*}
\lim_{\eta \rightarrow 0}\lim \sup_{n\rightarrow \infty }\sup_{P\in \mathcal{%
P}_{0}^{C+}}P\left( S_{n}^{C+}\geq c_{0,n,1-\alpha }^{\ast C+}\left( \phi
_{n}^{C+},\overline{h}_{B,n}^{\ast C+}\right) \right) =\alpha .
\end{equation*}%
Statements (i) and (iii) of Theorem 2 establish that inference based on GMS
bootstrap critical values are uniformly asymptotically valid. Statements
(ii) and (iv) of the theorem establish that inference based on GMS bootstrap
critical values is asymptotically non-conservative, whenever $Q\left( 
\mathcal{B}^{+}\right) >0$ or $Q\left( \mathcal{B}^{C+}\right) >0$ (i.e.,
whenever at least one moment condition holds with equality, over a set $x\in 
\mathcal{X}^{+}$ with non-zero $Q-$measure). Although the GMS based tests
are not similar on the boundary, the degree of non similarity, which is%
\begin{eqnarray*}
&&\lim_{\eta \rightarrow 0}\lim \sup_{n\rightarrow \infty }\sup_{P\in 
\mathcal{P}_{0}^{G+}}P\left( S_{n}^{G+}\geq c_{0,n,1-\alpha }^{\ast
G+}\left( \phi _{n}^{G+},\overline{h}_{B,n}^{\ast G+}\right) \right) \\
&&-\lim_{\eta \rightarrow 0}\lim \inf \inf_{P\in \mathcal{P}%
_{0}^{G+}}P\left( S_{n}^{G+}\geq c_{0,n,1-\alpha }^{\ast G+}\left( \phi
_{n}^{G+},\overline{h}_{B,n}^{\ast G+}\right) \right) ,
\end{eqnarray*}%
is much smaller than that associated with using the \textquotedblleft
usual\textquotedblright\ recentered bootstrap. In the case of pairwise
comparison (i.e., $k=2),$ Theorem 2(ii) of Linton, Song and Whang (2010)
establishes similarity of stochastic dominance tests on a subset of the
boundary.

For implementation of the tests discussed in this paper, it thus follows
that one can use Holm bounds as is done in JCS (2017), with modifications
due to the presence of the constant $\eta $. Estimate bootstrap $p-$values $%
p_{B,n,S_{n}^{G+}}^{G+}=\frac{1}{B}\sum_{s=1}^{B}1\left( (S_{n}^{\ast
G^{+}}+\eta )\geq S_{n}^{G^{+}}\right) $ and $p_{B,n,S_{n}^{G-}}^{G-}=\frac{1%
}{B}\sum_{s=1}^{B}1\left( (S_{n}^{\ast G^{-}}+\eta )\geq
S_{n}^{G^{-}}\right) $. Estimate $p_{B,n,S_{n}^{C+}}^{C+}$ and $%
p_{B,n,S_{n}^{C-}}^{C-}$ in analogous fashion. Then, use the following rules
(Holm (1979)):

\noindent \textbf{Rule }$S_{n}^{G}$\textbf{: }Reject $H_{0}^{G}$ at level $%
\alpha ,$ if $\min \left\{
p_{B,n,S_{n}^{G+}}^{G+},p_{B,n,S_{n}^{G-}}^{G-}\right\} \leq (\alpha -\eta
)/2$.

\noindent \textbf{Rule }$S_{n}^{C}$\textbf{: }Reject $H_{0}^{C}$ at level $%
\alpha ,$ if $\min \left\{
p_{B,n,S_{n}^{C+}}^{C+},p_{B,n,S_{n}^{C-}}^{C-}\right\} \leq (\alpha -\eta
)/2$.

\subsection{Power against Fixed and Local Alternatives}

As our statistics are weighted averages over $\mathcal{X}^{+},$ they have
non-trivial power only if the null is violated over a subset of non zero $Q-$%
measure. This applies to both power against fixed alternative, as well as to
power against $\sqrt{n}-$local alternatives. In particular, for power
against fixed alternatives, we require the following assumption.

\noindent \textbf{Assumption FA: }(i) $Q(B_{FA}^{G+})>0,$ where $%
B_{FA}^{G+}=\left\{ x\in \mathcal{X}^{+}:G_{j}(x)>0\text{ for some }%
j=2,...,k\right\} .$. (ii) $Q(B_{FA}^{C+})>0$ where $B_{FA}^{C+}=\left\{
x\in \mathcal{X}^{+}:C_{j}(x)>0\text{ for some }j=2,...,k\right\} .$

The following result holds.

\noindent \textbf{Theorem 3: }\textit{Let Assumptions A0-A4 hold.}

\textit{\noindent (i) If Assumption FA(i) holds, then under }$H_{A}^{G+}:$%
\begin{equation*}
\lim_{n\rightarrow \infty }P\left( S_{n}^{G+}\geq c_{0,n,1-\alpha }^{\ast
G+}\left( \phi _{n}^{G+},\overline{h}_{B,n}^{\ast G+}\right) \right) =1.
\end{equation*}%
\textit{\noindent (ii) If Assumption FA(ii) holds, then under }$H_{A}^{C+}:$%
\begin{equation*}
\lim_{n\rightarrow \infty }P\left( S_{n}^{C+}\geq c_{0,n,1-\alpha }^{\ast
C+}\left( \phi _{n}^{C+},\overline{h}_{B,n}^{\ast C+}\right) \right) =1.
\end{equation*}%
It is immediate to see that we have unit power against fixed alternatives,
provided that the null hypothesis is violated, for at least one $j=2,...,k,$
over a subset of $\mathcal{X}^{+}$ of non-zero $Q-$measure. Now, if we
instead used a Kolmogorov type statistic (i.e., replace the integral over $%
\mathcal{X}^{+}$ with the supremum over $\mathcal{X}^{+}),$ then we would
not need Assumption FA, and it would suffice to have violation for some $x,$
with possibly zero $Q-$measure, or in general with zero Lebesgue measure.%
\footnote{%
The Kolmorogov versions of $S_{n}^{G+}$ and $S_{n}^{C+}$ are:%
\begin{equation*}
KS_{n}^{G+}=\max_{x\in \mathcal{X}^{+}}\sum_{j=2}^{k}\left( \max \left\{ 0,%
\frac{\sqrt{n}G_{j,n}^{+}(x)}{\overline{\sigma }_{jj,n}^{G+}(x)}\right\}
\right) ^{2}
\end{equation*}%
\begin{equation*}
KS_{n}^{C+}=\max_{x\in \mathcal{X}^{+}}\sum_{j=2}^{k}\left( \max \left\{ 0,%
\frac{\sqrt{n}C_{j,n}^{+}(x)}{\overline{\sigma }_{jj,n}^{G+}(x)}\right\}
\right) ^{2}
\end{equation*}%
} However, as pointed out in Supplement B of Andrews and Shi (2013) the
statements in parts (ii) and (iv) of Theorem 2 do not apply to Kolmogorov
tests, and hence asymptotic non-conservativeness does not necessarily hold.
This is because the proof of those statements use the bounded convergence
theorem, which applies to integrals but not to suprema.

We now consider the following sequences of local alternatives:%
\begin{equation*}
H_{L,n}^{G+}:G_{Lj}^{+}(x)=G_{j}^{+}(x)+\frac{\delta _{1,j}(x)}{\sqrt{n}}%
+o\left( n^{-1/2}\right) ,\text{ for }j=2,...,k,\text{ }x\in \mathcal{X}^{+}
\end{equation*}%
and%
\begin{equation*}
H_{L,n}^{C+}:C_{Lj}^{+}(x)=C_{j}^{+}(x)+\frac{\delta _{2,j}(x)}{\sqrt{n}}%
+o\left( n^{-1/2}\right) ,\text{ for }j=2,...,k,\text{ }x\in \mathcal{X}^{+}.
\end{equation*}%
We have $\lim_{n\rightarrow \infty }\sqrt{n}D^{G+}(x)^{-1/2}G_{Lj}^{+}(x)%
\rightarrow h_{j,A,\infty }^{G+}(x)+\delta _{1,j}(x),$ and $%
\lim_{n\rightarrow \infty }\sqrt{n}D^{C+}(x)^{-1/2}C_{Lj}^{+}(x)\rightarrow
h_{j,A,\infty }^{C^{+}}(x)+\delta _{2,j}(x).$ Define,%
\begin{equation*}
S_{\infty ,\delta _{1},LG}^{\dag ,G+}=\int_{\mathcal{X}^{+}}\sum_{j=2}^{k}%
\left( \max \left\{ 0,\frac{v_{j}^{G+}(x)+h_{j,A,\infty }^{G+}(x)+\delta
_{1,j}(x)}{\sqrt{h_{jj,B}^{G+}(x)}}\right\} \right) ^{2}\mathrm{d}Q(x)
\end{equation*}%
and%
\begin{equation*}
S_{\infty ,\delta _{2},LC}^{\dag ,C+}=\int_{\mathcal{X}^{+}}\sum_{j=2}^{k}%
\left( \max \left\{ 0,\frac{v_{j}^{C+}(x)+h_{j,A,\infty }^{C+}(x)+\delta
_{2,j}(x)}{\sqrt{h_{jj,B}^{C+}(x)}}\right\} \right) ^{2}\mathrm{d}Q(x)
\end{equation*}

\noindent We require the following assumption.

\noindent \textbf{Assumption LA:}(i) $Q(B_{LA}^{G+})>0,$ where

\noindent $B_{LA}^{G+}=\left\{ x:\sqrt{n}D^{G+}(x)^{-1/2}G_{Lj}^{+}(x)%
\rightarrow h_{j,A,\infty }^{G+}(x)+\delta _{1,j}(x),\text{ }0<h_{j,A,\infty
}^{G+}(x)+\delta _{1,j}(x)<\infty ,\text{ for some }j=2,...,k\right\} $.

\noindent (ii) $Q(B_{LA}^{C+})>0,$ where

\noindent $B_{LA}^{C+}=\left\{ x:\sqrt{n}D^{C+}(x)^{-1/2}C_{Lj}^{+}(x)%
\rightarrow h_{j,A,\infty }^{C+}(x)+\delta _{2,j}(x),\text{ }0<h_{j,A,\infty
}^{C+}(x)+\delta _{2,j}(x)<\infty ,\text{ for some }j=2,...,k\right\} .$

The following result holds.

\noindent \textbf{Theorem 4: }\textit{Let Assumptions A1-A4 hold.}

\textit{\noindent (i) If Assumption LA(i) holds, then under }$H_{L,n}^{G+}:$%
\begin{equation*}
\lim_{n\rightarrow \infty }P\left( S_{n}^{G+}\geq c_{0,n,1-\alpha }^{\ast
G+}\left( \phi _{n}^{G+},\overline{h}_{B,n}^{\ast G+}\right) \right)
=P\left( S_{\infty ,\delta _{1},LG}^{\dag G+}\geq c_{LG,1-\alpha }\left(
h_{A,\infty }^{G+},h_{B,\infty }^{G+}\right) \right) ,
\end{equation*}%
\textit{with }$c_{LG,1-\alpha }\left( h_{A,\infty }^{G+},h_{B,\infty
}^{G+}\right) $\textit{\ denoting the (}$1-\alpha )$\textit{-th critical
value of }$S_{\infty ,\delta _{1},LG}^{\dag G+}$\textit{, with }$%
0<h_{j,A,\infty }^{G+}(x)+\delta _{1,j}(x)<\infty ,$\textit{\ for some }$%
j=2,...,k.$

\textit{\noindent (ii) If Assumption LA(ii) holds, then under }$%
H_{L,n}^{C+}: $%
\begin{equation*}
\lim_{n\rightarrow \infty }P\left( S_{n}^{C+}\geq c_{0,n,1-\alpha }^{\ast
C+}\left( \phi _{n}^{C+},\overline{h}_{B,n}^{\ast C+}\right) \right)
=P\left( S_{\infty ,\delta _{2},LC}^{\dag C+}\geq c_{LC,1-\alpha }\left(
h_{A,\infty }^{C+},h_{B,\infty }^{C+}\right) \right) ,
\end{equation*}%
\textit{with }$c_{LC,1-\alpha }\left( h_{A,\infty }^{C+},h_{B,\infty
}^{C+}\right) $\textit{\ denoting the (}$1-\alpha )$\textit{-th critical
value of }$S_{\infty ,\delta _{2},LC}^{\dag C+}$\textit{, with }$%
0<h_{j,A,\infty }^{C+}(x)+\delta _{2,j}(x)<\infty ,$\textit{\ for some }$%
j=2,...,k.$

Theorem 4 establishes that our tests have power against $\sqrt{n}-$%
alternatives, provided that the drifting sequence is bounded away from zero,
over a subset of $\mathcal{X}^{+}$ of non-zero $Q-$measure. Note also that
for given loss function, $L,$ the sequence of local alternatives for the
White reality check can be defined as:%
\begin{equation}
H_{A,n}:\max_{j=2,...,k}\left( E(L(e_{1}))-E(L(e_{j}))\right) =\frac{\lambda 
}{\sqrt{n}}+o\left( n^{-1/2}\right) ,\text{ }\lambda >0.  \label{HAn}
\end{equation}%
For sake of simplicity, suppose that $k=2$ (this is the well known Diebold
and Mariano (1995) test framework). Here,

\begin{eqnarray}
&&0<\lambda =n^{1/2}E(L(e_{1}))-E(L(e_{k}))+o(1)  \notag \\
&=&n^{1/2}\int_{-\infty }^{\infty }L(x)\left( f_{1,n}(x)-f_{2,n}(x)\right) 
\mathrm{d}x  \notag \\
&=&-n^{1/2}\int_{-\infty }^{0}L^{\prime }(x)\left(
F_{1,n}(x)-F_{2,n}(x)\right) \mathrm{d}x  \notag \\
&&-n^{1/2}\int_{0}^{\infty }L^{\prime }(x)\left(
F_{1,n}(x)-F_{2,n}(x)\right) \mathrm{d}x  \notag \\
&=&n^{1/2}\int_{-\infty }^{0}\left( h_{A,\infty }^{G-}(x)+\delta
_{1}(x)\right) Q(x)\mathrm{d}x+n^{1/2}\int_{0}^{\infty }\left( h_{A,\infty
}^{G+}(x)+\delta _{1}(x)\right) Q(x)\mathrm{d}x,  \label{RC}
\end{eqnarray}%
where $F_{j,n}(x)=F_{j}(x)+\frac{\delta _{1,j}(x)}{\sqrt{n}}$, and $\delta
_{1}=\delta _{1,1}-\delta _{1,2}.$ Hence, $H_{A,n}$ in (\ref{HAn}) is
equivalent to $H_{LA}^{G+}\cap H_{LA}^{G-}$, whenever \textit{Assumption A0
holds and} $Q(x)=L^{\prime }(x)sgn(x).$

Analogously, for any convex loss function, $L,$ which satisfies Assumption
A0, $H_{A,n}$ in (\ref{HAn}) is equivalent to $H_{LA}^{C-}\cap H_{LA}^{C+-}$%
, whenever $Q(x)=L^{\prime \prime }(x)sgn(x).$ In fact, it is easy to see
that:

\begin{eqnarray*}
&&0<\delta =n^{1/2}E(L(e_{1}))-E(L(e_{k}))+o(1) \\
&=&n^{1/2}\int_{-\infty }^{\infty }L(x)\left( f_{1,n}(x)-f_{2,n}(x)\right) 
\mathrm{d}x \\
&=&-n^{1/2}\int_{-\infty }^{0}L^{\prime }(x)\left(
F_{1,n}(x)-F_{2,n}(x)\right) \mathrm{d}x-n^{1/2}\int_{0}^{\infty }L^{\prime
}(x)\left( F_{1,n}(x)-F_{2,n}(x)\right) \mathrm{d}x \\
&=&-L^{\prime }(x)n^{1/2}\int_{-\infty }^{x}\left(
F_{1,n}(z)-F_{2,n}(z)\right) \mathrm{d}z\left\vert _{-\infty }^{0}\right.
+n^{1/2}\int_{-\infty }^{0}L^{\prime \prime }(x)\left( \int_{-\infty
}^{x}\left( F_{1,n}(z)-F_{2,n}(z)\right) \mathrm{d}z\right) \mathrm{d}x \\
&&+n^{1/2}L^{\prime }(x)\int_{x}^{\infty }\left(
F_{1,n}(z)-F_{2.n}(z)\right) \mathrm{d}z\left\vert _{0}^{\infty }\right.
-n^{1/2}\int_{0}^{\infty }L^{\prime \prime }(x)\left( \int_{x}^{\infty
}\left( F_{1,n}(z)-F_{2,n}(z)\right) \mathrm{d}z\right) \mathrm{d}x \\
&=&n^{1/2}\int_{-\infty }^{0}L^{\prime \prime }(x)\left( \int_{-\infty
}^{x}\left( F_{1,n}(z)-F_{2,n}(z)\right) \mathrm{d}z\right) \mathrm{d}%
x-n^{1/2}\int_{0}^{\infty }L^{\prime \prime }(x)\left( \int_{x}^{\infty
}\left( F_{1,n}(z)-F_{2,n}(z)\right) \mathrm{d}z\right) \mathrm{d}x \\
&=&n^{1/2}\int_{-\infty }^{0}\left( \int_{-\infty }^{x}\left( h_{A,\infty
}^{C-}(x)+\delta _{2}(x)\right) \mathrm{d}z\right) Q(x)\mathrm{d}%
x-n^{1/2}\int_{0}^{\infty }\left( \int_{x}^{\infty }\left( h_{A,\infty
}^{C+}(x)+\delta _{2}(x)\right) \mathrm{d}z\right) Q(x)\mathrm{d}x.
\end{eqnarray*}

\section{Monte Carlo Experiments}

In this section, we evaluate the finite sample performance of GL and CL\
forecast superiority tests when there are multiple competing sequences of
forecast errors, under stationarity. In addition to analyzing the
performance of our tests based on $S_{n}^{G+}$ and $S_{n}^{G-},$ (GL
forecast superiority) as well as based on $S_{n}^{C+}$ and $S_{n}^{C-}$ (CL
forecast superiority), we also analyze the performance of the related test
statistics from JCS (2017), here called $JCS_{n}^{G+}$, $JCS_{n}^{G-},$ $%
JCS_{n}^{C+}$, and $JCS_{n}^{C-}$. For the sake of brevity, these two
classes of tests are called $S_{n}$ and $JCS_{n}$ tests, respectively.%
\footnote{%
In the construction of the statistics 
\begin{equation*}
S_{n}^{G+}=\underset{x\in \mathcal{X}^{+}}{\int }\underset{j=2}{\sum^{k}}%
\left( \max \left\{ 0,\sqrt{n}\frac{G_{j,n}(x)}{\overline{\sigma }%
_{jj,n}^{G}(x)}\right\} \right) ^{2}\text{ }dQ(x)\text{ and }S_{n}^{G-}=%
\underset{x\in \mathcal{X}^{-}}{\int }\underset{j=2}{\sum^{k}}\left( \max
\left\{ 0,\sqrt{n}\frac{G_{j,n}(x)}{\overline{\sigma }_{jj,n}^{G}(x)}%
\right\} \right) ^{2}dQ(x).
\end{equation*}%
we set $dQ(x)$ =$\frac{1}{1.5n^{0.6}}$. Thus, $Q(\cdot )$ is still uniform.
For inference using our tests, once $B$ is determined, estimate bootstrap $%
p- $values, $p_{B,n,S_{n}}^{G+}=\frac{1}{B}\sum_{s=1}^{B}1\left(
(S_{n}^{\ast G^{+}}+\eta )\geq S_{n}^{G^{+}}\right) .$ and $%
p_{B,n,S_{n}}^{G-}=\frac{1}{B}\sum_{s=1}^{B}1\left( (S_{n}^{\ast G-}+\eta
)\geq S_{n}^{G-}\right) $. Then, use the following rules (Holm, (1979):%
\textbf{\ }Reject $H_{0}^{TG}$ at level $\alpha ,$ if $\min \left\{
p_{B,n,S_{n}}^{G+},p_{B,n,S_{n}}^{G-}\right\} \leq (\alpha -\eta )/2$.%
\textbf{\ }Reject $H_{0}^{TC}$ at level $\alpha ,$ if $\min \left\{
p_{B,n,S_{n}}^{C+},p_{B,n,S_{n}}^{C-}\right\} \leq (\alpha -\eta )/2$.} For
each experiment we carry out 1000 Monte Carlo replications, and the number
of bootstrap samples is $B=500$. Additionally, four different values of the
smoothing parameter, $J_{n}$ are examined for the $JCS_{n}$ tests, including 
$J_{n}=\{0.20,0.35,0.50,0.60\}$; and four different values of the uniformity
constant, $\eta ,$ are examined for the $S_{n}$ tests, including $\eta
=\{0.0015,0.002,0.0025,0.003\}$.\footnote{%
In JCS (2017), the constant that we call $J_{n}$ is called $S_{n}$.}
Additionally, for $S_{n}$ tests, when constructing $\overline{\Sigma }%
_{n}^{G+}$(as well as $\overline{\Sigma }_{n}^{G-}$, etc.), we set $l_{n}=%
\func{integer}[n^{0.2}]$ and $\varepsilon =1e-4.$ Finally, when implementing
the bootstrap counterpart of $S_{n}$, we set $\kappa _{n}=\sqrt{0.3\log (n)}$
and $c_{n}=\sqrt{0.4\log (n)/\log (\log (n))},$ following Andrews and Shi
(2013, 2017). Sample sizes of $n\in \{300,600,900\}$ are generated using
each of the following eight data generating processes (DGPs), with
independent forecast errors.

\noindent DGP1: $e_{1t}$ $\sim $ $i.i.d.N(0,1)$ and $e_{kt}$ $\sim $ $%
i.i.d.N(0,1),$ $k=2,3.$

\noindent DGP2: $e_{1t}$ $\sim $ $i.i.d.N(0,1)$ and $e_{kt}$ $\sim $ $%
i.i.d.N(0,1),$ $k=2,3,4,5.$

\noindent DGP3: $e_{1t}$ $\sim $ $i.i.d.N(0,1)$, $e_{kt}$ $\sim $ $%
i.i.d.N(0,1),$ $k=2,3$ and $e_{kt}$ $\sim i.i.d.N(0,1.4^{2}),$ $k=4,5$

\noindent DGP4: $e_{1t}$ $\sim $ $i.i.d.N(0,1)$, $e_{kt}$ $\sim $ $%
i.i.d.N(0,1),$ $k=2,3$ and $e_{kt}$ $\sim $ $i.i.d.N(0,1.6^{2}),$ $k=4,5$

\noindent DGP5: $e_{1t}$ $\sim $ $i.i.d.N(0,1)$, $e_{kt}$ $\sim $ $%
i.i.d.N(0,0.8^{2}),$ $k=2,3$ and $e_{kt}$ $\sim $ $i.i.d.N(0,1.2^{2}),$ $%
k=4,5.$

\noindent DGP6: $e_{1t}$ $\sim $ $i.i.d.N(0,1),$ $e_{kt}$ $\sim $ $%
i.i.d.N(0,0.8^{2}),$ $k=2,3,4,5$ and $e_{kt}$ $\sim $ $i.i.d.N(0,1.2^{2}),$ $%
k=6,7,8,9.$

\noindent DGP7: $e_{1t}$ $\sim $ $i.i.d.N(0,1)$, $e_{kt}$ $\sim $ $%
i.i.d.N(0,1),$ $k=2,3$ and $e_{kt}$ $\sim $ $i.i.d.N(0,0.8^{2}),$ $k=4,5.$

\noindent DGP8: $e_{1t}$ $\sim $ $i.i.d.N(0,1)$, $e_{kt}$ $\sim $ $%
i.i.d.N(0,1),$ $k=2,3$ and $e_{kt}$ $\sim $ $i.i.d.N(0,0.6^{2}),$ $k=4,5.$

\noindent DGP9: $e_{1t}$ $\sim $ $i.i.d.N(0,1)$ and $e_{kt}$ $\sim $ $%
i.i.d.N(0,0.8^{2}),$ $k=2,3,4,5.$

\noindent DGP10: $e_{1t}$ $\sim $ $i.i.d.N(0,1)$ and $e_{kt}$ $\sim $ $%
i.i.d.N(0,0.6^{2}),$ $k=2,3,4,5.$

Additionally, we conducted experiments using DGPs specified with
autocorrelated errors. For the sake of brevity, these finding are reported
in the supplemental online appendix. Denoting $\widetilde{e}_{i,t}=\varrho 
\widetilde{e}_{i,t-1}+(1-\varrho ^{2})^{1/2}\eta _{i,t},$with $\eta
_{kt}\sim i.i.d.N(0,1)$ $i=1,...,5,$ the DGPs for these experiments are as
follows.

\noindent DGP11: $e_{1t}$ $=\widetilde{e}_{1,t}$ and $e_{kt}$ $=\widetilde{e}%
_{k,t},$ $k=2,3,4,5.$

\noindent DGP12: $e_{1t}$ $=\widetilde{e}_{1,t}$, $e_{kt}$ $=\widetilde{e}%
_{k,t},$ $k=2,3$ and $e_{kt}$ $=1.4\widetilde{e}_{k,t},$ $k=4,5.$

\noindent DGP13: $e_{1t}$ $=\widetilde{e}_{1,t}$, $e_{kt}$ $=0.8\widetilde{e}%
_{k,t},$ $k=2,3$ and $e_{kt}$ $=1.2\widetilde{e}_{k,t},$ $k=4,5.$

\noindent DGP14: $e_{1t}$ $=\widetilde{e}_{1,t}$, $e_{kt}$ $=\widetilde{e}%
_{k,t},$ $k=2,3$ and $e_{kt}$ $=0.6\widetilde{e}_{k,t},$ $k=4,5.$

In the above setup, DGPs 1-4 and DGPs 11-12 are used to conduct size
experiments, while DGPs 4-10 and DGPs 13-14 are used to conduct power
experiments. In all cases, $e_{1t}$ denote the forecast errors from the
benchmark model. Note that DGPs 1-2 correspond to the least favorable
elements in the null, while in DGPs 3-4 and DGPs 11-12, some models
underperform the benchmark. This is the case where we expect significant
improvement when using our new tests instead of JCS tests. In DGPs 5-6 and
DGP13, one half of the competing models outperform the benchmark model and
the other half underperform. In DGPs 7-8, one half of the competing models
outperform, while is DGPs 9-10 and DGP14, the competing models all
outperform the benchmark model. The above DGPs are similar to those examined
in JCS (2017), and are utilized in our experiments because they clearly
illustrate the trade-offs associated with using $JCS_{n}$ and $S_{n}$
forecast superiority tests.

We now discuss the experimental findings gathered in Tables 1 and 2. All
reported results are rejection frequencies based on carrying out the $%
JCS_{n} $ and $S_{n}$ tests using a nominal size equal to 0.1. Turning first
to Table 1, note that results in this table are based on $JCS_{n}$ tests.
Summarizing, $JCS_{n}$ tests have reasonably good size under DGPs 1-2 (the
least favorable case under the null). However, they are often undersized
(and in some cases severely so) in some sample size / $J_{n}$ permutations
when some models are worse than the benchmark (see DGPs 3-4), as should be
expected given that the tests are not asymptotically correctly sized under
these two DGPs. Moreover, in these cases the empirical size is non
monotonic, in paricular for $GL$ forecast superiority. Turning to Table 2,
note that $S_{n}$ tests, which are asymptotically non conservative, often
exhibit better size properties under DGPs 3-4 (compare DGPs 3-4 in Tables 1
and 2) than $JCS_{n}$ tests. For example, for the CL forecast superiority
test the empirical size of the $JCS_{n}$ test is $0.020$ for all values of $%
J_{n}$, when $n=900$ (see Table 1). The analogous value based on
implementation of the $S_{n}$ test is $0.083,$ for all values of $\eta $
(see Table 2)$.$ Again, it is worth stressing that this finding comes as no
surprise, given that the $S_{n}$ test is asymptotically non conservative on
the boundary of the null hypotheses, while the $JCS_{n}$ test is
conservative. Turning to power, note that the power of the $JCS_{n}$ test is
sometimes quite low relative to that of the $S_{n}$ test. For example, under
DGP7, power is $0.445$ for the GL forecast superiority $JCS_{n}$ test and $%
0.845$ for the CL forecast superiority $JCS_{n}$ test, when $n=300$). Note
that analogous rejection frequencies for the $S_{n}$ tests are $0.870$ and $%
0.923$ (see Table 2, DGP7, $n=300$). As expected, thus, $S_{n}$ tests
exhibit improved power relative to $JCS_{n}$ tests, when some models are
worse than the benchmark. All of the above findings pertain to the analysis
of DGPS 1-10, in which forecast errors are serially uncorrelated. Results
for DGPS 11-14, in which errors are serially correlated are gathered in the
supplemental appendix. Examination of the results for these DGPSs (see
Tables Supplemental S1 and S2) are qualitatively the same as those reported
on above. Finally, it should be pointed out that the $S_{n}$ tests are not
overly sensitive to the choice of $\eta $, and the empirical size of $S_{n}$
tests appears \textquotedblleft best\textquotedblright\ when $\eta $ is very
small, as should be expected. In conclusion, there is a clear performance
improvement when comparing our new robust predictive superiority tests with $%
JCS_{n}$ tests.\footnote{%
For a discussion of simulation results based on application of the Diebold
and Mariano (DM: 1995) test (in which specific loss functions are utilized)
in our experimental setup, refer to JCS(2017). Summarizing from that paper,
it is clear that when the loss function is unknown, there is an advantage to
using our approach of testing for forecast superiority. However the DM test
for pairwise comparison or a reality check\ test for multiple comparisons
might yield improved power, for a given loss function. Indeed, under
quadratic loss, JCS (2017) show that when the sample size is small, the DM\
test has better power performance than $JCS_{n}$ type tests.\ When the
sample size increases, the power difference between the two tests becomes
smaller. This is as expected.}

\section{Empirical Illustration:\ Robust Forecast Evaluation of SPF Expert
Pools}

In the real-time forecasting literature, predictions from econometric models
are often compared with surveys of expert forecasters.\footnote{%
See Fair and Shiller (1990), Swanson and White (1997a,b), Aiolfi, Capistr%
\'{a}n and Timmermann (2011), and the references cited therein for further
discussion.} Such comparisons are important when assessing the implications
associated with using econometric models in policy setting contexts, for
example. One key survey dataset collecting expert predictions is the \textit{%
Survey of Professional Forecasters} (SPF), which is maintained by the
Philadelphia Federal Reserve Bank (see Croushore (1993)). This dataset,
formerly known as the \textit{American Statistical Association/National
Bureau of Economic Research Economic Outlook Survey, }collects predictions
on various key economic indicators (including, for example, nominal GDP
growth, real GDP growth, prices, unemployment, and industrial production).
For further discussion of the variables contained in the SPF, refer to
Croushore (1993) and Aiolfi, Capistr\'{a}n, and Timmermann (2011). The SPF
has been examined in numerous papers. For example, Zarnowitz and Braun
(1993) comprehensively study the SPF, and find, among other things, that use
of the mean or median provides a consensus forecast with lower average
errors than most individual forecasts. More recently, Aiolfi, Capistr\'{a}n,
and Timmermann (2011) consider combinations of SPF survey forecasts, and
find that equal weighted averages of survey forecasts outperform model based
forecasts, although in some cases these mean forecasts can be improved upon
by averaging them with mean econometric model-based forecasts. When
utilizing European data from the recently released ECB SPF, Genre, Kenny,
Meyler, and Timmermann (2013) again find that it is very difficult to beat
the simple average. This well known result pervades the macroeconometric
forecasting literature, and reasons for the success of such simple forecast
averaging are discussed in Timmermann (2006). He notes, among other things,
that model misspecification related to instability (non-stationarities) and
estimation error in situations where there are many models and relatively
few observations may account to some degree for the success of simple
forecast and model averaging. Our empirical illustration attempts to shed
further light on the issue of simple model averaging and its importance in
forecasting macroeconomic variables.

Our approach is to address the issue of forecast averaging and combination
(called pooling) by viewing the problem through the lens of forecast
superiority testing. Our use of loss function robust tests is unique to the
SPF literature, to the best of our knowledge. Since we use robust forecast
superiority tests, we do not evaluate pooling by using loss function
specific tests, such as those discussed in Diebold and Mariano (1995),
McCracken (2000), Corradi and Swanson (2003), and Clark and McCracken
(2013). Additionally, our approach differs from that taken by Elliott,
Timmermann, and Komunjer (2005, 2008), where the rationality of sequences of
forecasts is evaluated by determining whether there exists a particular loss
function under which the forecasts are rational. We instead evaluate
predictive accuracy irrespective of the loss function implicitly used by the
forecaster, and determine whether certain forecast combinations are superior
when compared against any loss function, regardless of how the forecasts
were constructed. In our tests, the benchmarks against which we compare our
forecast combinations are simple average and median consensus forecasts. We
aim to assess whether the well documented success of these benchmark
combinations remains intact when they are compared against other
combinations, under generic loss.\footnote{%
For an interesting discussion of machine learning and forecast combination
methods, see Lahiri, Peng, and Zhao (2017); and for a discussion of
probability forecasting and calibrated combining using the SPF, see Lahiri,
Peng, and Zhao (2015). In these papers, various cases where consensus
combinations do not \textquotedblleft win\textquotedblright\ are discussed.}

In all of our experiments, we utilize SPF predictions of nominal GDP growth.
The SPF is a quarterly survey, and the dataset is available at the
Philadelphia Federal Reserve Bank (PFRB) website. The original survey began
in 1968:Q4, and PFRB took control of it in 1990:Q2; but from that date,
there are only around 100 quarterly observations prior to 2018:Q1, where we
end our sample. In our analysis we thus use the entire dataset, which, after
trimming to account for differing forecast horizons in our calculations, is
166 observations.\footnote{%
It should be noted that the \textquotedblleft timing\textquotedblright\ of
the survey was not known with certainty prior to 1990. However, SPF
documentation states that they believe, although are not sure, that the
timing of the survey was similar before and after they took control of it.},%
\footnote{%
Note that the number of experts for which forecasts are recorded for each
calendar date, was approximately 90 experts during each of the 4 quarters of
1968, while there where only approximately 40 experts in each quarter in
2017. For further details on the SPF dataset, refer to the documentation at
https://www.philadelphiafed.org/research-and-data/real-time-center/survey-of-professional-forecasters.%
}

For our analysis, we consider 5 forecast horizons (i.e., $h=0,1,2,3,4).$ The
reason we use $h=0$ for one of the horizons is that the first horizon for
which survey participants predict GDP growth is the quarter in which they
are making their predictions. In light of this, forecasts made at $h=0$ are
called nowcasts. Moreover, it is worth noting that nowcasts are very
important in policy making settings, since first release GDP data are not
available until around the middle of the subsequent quarter. The nominal
GDP\ variable that we examine is called NGDP in the SPF.\textit{\ }All test
statistics are constructed using NGDP growth rate prediction errors. In
particular, assume that one survey participant makes a forecast of NGDP, say 
$y_{t+h}^{f}|\mathcal{F}_{t}$.\footnote{%
Here, $\mathcal{F}_{t}$ denotes the information set available to the expert
forecaster at the time their predictions are made.} The associated forecast
error is: 
\begin{equation*}
e_{t}=\left\{ \ln (y_{t+h})-\ln (y_{t})\right\} -\left\{ \ln (y_{t+h}^{f}|%
\mathcal{F}_{t})-\ln (y_{t})\right\} =\ln (y_{t+h})-\ln (y_{t+h}^{f}|%
\mathcal{F}_{t}),
\end{equation*}%
where the actual NGDP value, $y_{t+h},$ is reported in the SPF, along with
the NGDP predictions of each survey participant. Note that when $h=0$, $%
\mathcal{F}_{t}$ does not include $y_{t}.$ However, for $h>0$, $\mathcal{F}%
_{t}$ includes $y_{t}$. As discussed previously, this is due to the release
dates associated with the availability of NGDP\ data. Figure 1 illustrates
some of the key properties of the NGDP data that we utilize. Namely, note
that the distributions of the expert forecasts vary over time, and exhibit
interesting skewness and kurtosis properties (compare Panels A-D of the
figure, and the skewness and kurtosis statistics reported below the plots in
the figure). Based on examination of the densities in Figure 1, one might
wonder whether \textquotedblleft trimming\textquotedblright\ experts from
the panel, say those experts that provided the forecasts appearing in the
left tails of the distributions, might improve overall predictive accuracy
of the panel. Although this question is not directly addressed in our
analysis, we do construct and analyze the performance of various
\textquotedblleft pools\textquotedblright\ formed by trimming\ experts that
exhibit sub-par predictive accuracy, for example.

In addition to constructing $S_{n}^{G+},S_{n}^{G-},S_{n}^{C-},$ and $%
S_{n}^{C+}$ tests in our empirical investigation, we also test for forecast
superiority using the $JCS_{n}$ tests discussed above, which have correct
size only under the least favorable case under the null. In particular, we
construct $JCS_{n}^{G+},JCS_{n}^{G-},JCS_{n}^{C-},$ and $JCS_{n}^{C+}$ test
statistics (see Section 2 and JCS (2017) for further details). All test
statistics are calculated using the same parameter values (for $B,$ $J_{n}$, 
$\eta ,$ $l_{n},$ and $\varepsilon )$ as used in our Monte Carlo
experiments. However, results are only reported for $J_{n}=0.20$ and $\eta
=0.002,$ since our findings remain unchanged when other values of $J_{n}$
and $\eta $ from our Monte Carlo experiments are used.

Two different benchmark models are considered, including (i) the arithmetic
mean prediction from all participants; and (ii) the median prediction from
all participants. Additionally, a variety of alternative model
\textquotedblleft groups\textquotedblright\ are considered. In all
alternative models, mean and median predictions are again formed, but this
time using subsets of the total available panel of experts, chosen in a
number of ways, as outlined below.

\noindent \textit{Group 1 - Experts Chosen Based on Experience: }Three
expert pools (i.e. three alternative models) consisting of experts with 1,
3, and 5 years of experience.

In all of the remaining groups of combinations, individuals are ranked
according to average absolute forecast errors, as well as according to
average squared forecast errors. Mean (or median) predictions from these
groups are then compared with our benchmark combinations.

\noindent \textit{Group 2 - Experts Chosen Based on Forecast Accuracy I: }%
Three expert pools consisting of most accurate expert over last 1, 3, and 5
years.

\noindent \textit{Group 3 - Experts Chosen Based on Forecast Accuracy II: }%
Three expert pools consisting of most accurate group of 3 experts over last
1, 3, and 5 years.

\noindent \textit{Group 4 - Experts Chosen Based on Forecast Accuracy III: }%
Three expert pools consisting of top 10\% most accurate group of experts
over last 1, 3, and 5 years.

\noindent \textit{Group 5 - Experts Chosen Based on Forecast Accuracy III: }%
Three expert pools consisting of top 25\% most accurate group of experts
over last 1, 3, and 5 years.

Finally, 3 additional groups which combine models from each of Groups 1-5
are analyzed. These include:

\noindent \textit{Group 6: }Five expert pools, including one pool with
experts that have 1 year of experience, and 4 additional pools, one from
each of Groups 2-5, all defined over the last 1 year.

\noindent \textit{Group 7: }Five expert pools, including one pool with
experts that have 3 years of experience, and 4 additional pools, one from
each of Groups 2-5, all defined over the last 3 years.

\noindent \textit{Group 8: }Five expert pools, including one pool with
experts that have 5 years of experience, and 4 additional pools, one from
each of Groups 2-5, all defined over the last 5 years.

As an example of how testing is performed, note that when implementing the $%
S_{n}^{G}$ test using \textit{Group 1, }there are three alternative models.
The same is true when implementing tests using \textit{Groups 2-5}. For 
\textit{Groups 6-8}, tests are implemented using 5 alternative models, where
one alternative is taken from each of \textit{Groups 1-5. }Summarizing, we
consider: (i) two benchmark models, against which each group of alternatives
is compared; (ii) alternative models that are based on either mean or median
pooled forecasts for, \textit{Groups 2-8}; (iii) forecast accuracy pools
used in $Groups$ 1-8 that are based on either average absolute forecast
errors or average squared forecast errors; (iii) 5 forecast horizons.

We now discuss our empirical findings. In Tables 3-4, statistics are
reported for all forecast superiority tests. Entries are $S_{n}^{G}$, $%
S_{n}^{C}$, $JCS_{n}^{G}$, and $JCS_{n}^{C}$ test statistics reported for
forecast horizons $h={0,1,2,3,4}$. More specifically, $S_{n}^{G}=S_{n}^{G+}$
if $p_{B,n,S_{n}^{G+}}^{G+}\leq p_{B,n,S_{n}^{G-}}^{G-}$; otherwise $%
S_{n}^{G}=S_{n}^{G-}$. The other statistics reported in the tables (i.e., $%
S_{n}^{C}$, $JCS_{n}^{G}$, and $JCS_{n}^{C})$ are defined analogously.
Rejections of the null of no forecast superiority at a 10\% level are
denoted by a superscript *. In Table 3, the benchmark model is always the
arithmetic mean prediction from all participants, and expert pool forecasts
are also arithmetic means. Analogously, in Table 4 the benchmark is the
median prediction from all participants, and expert pool forecasts are also
medians. To understand the layout of the tables, turn to Table 3, and note
that for $Group$ $1$, the 4 statistics defined above (i.e., $S_{n}^{G}$, $%
S_{n}^{C}$, $JCS_{n}^{G}$, and $JCS_{n}^{C})$ are given, for each forecast
horizon, $h=0,1,2,3,$ and $4.$ Superscripts denote rejection of the null
hypothesis based on a particular test. For example, note that application of
the $JCS_{n}^{G}$ test in $Group$ $2$ yields a test rejection for horizons $%
h=2$ and $4.$ Turning to the results summarized in the tables, a number of
clear conclusions emerge.

First, the majority of test rejections occur for $h=4$, as can be seen by
inspection of the results in both Tables 3 and 4$.$ In particular, note that
for $h=4$, there are 13 test rejections in Table 3 and 11 test rejections in
Table 4, across $Groups$ 1-8. On the other hand, for all other forecast
horizons combined (i.e., $h=\{0,1,2,3\})$, there are 11 test rejections in
Table 3 and 8 test rejections in Table 4. This suggests that expert pools
which are constructed by \textquotedblleft trimming\textquotedblright\ the
least effective experts are most useful for longer horizon forecasts. These
findings make sense if one assumes that it is easier to make short term
forecasts than long term forecasts. Namely, some experts are simply not
\textquotedblleft up to the task\textquotedblright\ when forecasting at
longer horizons. Summarizing, our main finding indicates that simple average
or median forecasts can be beaten, in cases where forecasts are more
difficult to make (i.e., longer horizons). Second, \textquotedblleft
experience\textquotedblright\ as measured by the length of time an expert
has taken part in the SPF is not a direct indicator of forecast superiority,
since there are no rejections of our tests for $Group$ $1,$ when either mean
(see Table 3) or median (see Table 4) forecasts are used in our tests. This
does not necessarily mean that experience does not matter, at least
indirectly (notice that test rejections sometimes occur for $Groups$ 6-8,
where experience and accuracy traits are combined.\footnote{%
To explore this finding in more detail, we also constructed additional
tables that are closely related to Tables 3 and 4, except that in these
tables, RMSFEs are reported for all of the models used in each test (see
supplemental appendix, Tables S1 and S2). In these tables, we see that
combining experience with prior predictive accuracy can lead to lower
RMSFEs, relative to the case where the entire pool of experts is used.
However, RMSFEs are even lower for various alternative models for which we
only use prior predictive accuracy to select expert pools (compare RMSFEs
for $Groups$ 3-5 with those for $Groups$ 6-8 in the supplemental tables).
\par
\bigskip} Finally, note that Tables S1 and S2 in the supplemental appendix
report root mean square forecast errors (RMSFEs) from the benchmark and
competing models utilized in our empirical analysis. In these tables, we see
that in the majority of cases considered, combination forecasts that utilize
the mean have lower RMSFEs than when the median is used for constructing
combination forecasts. For example, when comparing the benchmark RMSFEs of $%
Group$ 1 that are reported in Tables S1 and S2, RMSFEs associated with mean
combination forecasts (see Table S1) are lower for $h=\{0,2,3,4\}$ than the
RMSFEs associated with median combination forecasts (see Table S2). This is
interesting, given the clear asymmetry and long left tails associated with
the distributions of expert forecasts exhibited in Figure 1, and suggests
that outlier forecasts from \textquotedblleft less
accurate\textquotedblright\ experts are not overly influential when using
measures of central tendency as ensemble forecasts.

Summarizing, we have direct evidence that judicious selection of pools of
experts can lead to loss function robust forecast superiority. However, it
should be stressed that in this illustration of the testing techniques
developed in this paper, we do not consider various combination methods,
including Bayesian model averaging, for example. Additionally, we only look
at nominal GDP, although the SPF has various other variables in it.
Extensions such as these are left to future research.

\section{Concluding Remarks}

We develop uniformly valid forecast superiority tests that are
asymptotically non conservative, and that are robust to the choice of loss
function. Our tests are based on principles of stochastic dominance, which
can be interpreted as tests for infinitely many moment inequalities. In
light of this, we use tools from Andrews and Shi (2013, 2017) when
developing our tests. The tests build on earlier work due to Jin, Corradi,
and Swanson (2017), and are meant to provide a class of predictive accuracy
tests that are not reliant on a choice of loss function, such as the Diebold
and Mariano (1995) test discussed in McCracken (2000). In developing the new
tests, we establish uniform convergence (over error support) of HAC variance
estimators, and of their bootstrap counterparts. In a Supplement, we also
extend the theory of generalized moment selection testing to allow for the
presence of non-vanishing parameter estimation error. In a series of Monte
Carlo experiments, we show that finite sample performance of our tests is
quite good, and that the power of our tests dominates those proposed by JCS
(2017). Additionally, we carry out an empirical analysis of the well known
Survey of Professional Forecasters, and show that utilizing expert pools
based on past forecast quality can lead to loss function robust forecast
superiority, when compared with pools that include all survey participants.
This finding is particularly prevalent for our longest forecast horizon
(i.e., 1-year ahead).

\newpage

\section{Appendix}

$\boldsymbol{\noindent }$\textbf{Proof of Lemma 1: (i)} The proof is the
same for all $j.$ Thus, let $u_{t}(x)=\left( 1\left\{ e_{j,t}\leq x\right\}
-F_{j}(x)\right) -\left( 1\left\{ e_{1,t}\leq x\right\} -F_{1}(x)\right) ,$
and define 
\begin{equation*}
\widehat{\widehat{\sigma }}_{n}^{2,G+}(x)=\frac{1}{n}%
\sum_{t=1}^{n}u_{t}^{2}(x)+2\frac{1}{n}\sum_{\tau =1}^{l_{n}}w_{\tau
}u_{t}(x)u_{t-\tau }(x).
\end{equation*}%
We first show that%
\begin{equation*}
\sup_{x\in \mathcal{X}^{+}}\left\vert \widehat{\widehat{\sigma }}%
_{n}^{2,G+}(x)-\sigma ^{2,G+}(x)\right\vert =o_{p}\left( 1\right) ,
\end{equation*}%
and then we show that 
\begin{equation}
\sup_{x\in \mathcal{X}^{+}}\left\vert \widehat{\widehat{\sigma }}%
_{n}^{2,G+}(x)-\widehat{\sigma }_{n}^{2,G+}(x)\right\vert =o_{p}\left(
1\right) .  \label{hat-hat}
\end{equation}%
Now,%
\begin{eqnarray}
&&\sup_{x\in \mathcal{X}^{+}}\left\vert \widehat{\widehat{\sigma }}%
_{n}^{2,G+}(x)-\sigma ^{2,G+}(x)\right\vert  \notag \\
&\leq &\sup_{x\in \mathcal{X}^{+}}\left\vert \frac{1}{n}\sum_{t=1}^{n}\left(
u_{t}^{2}(x)-\mathrm{E}\left( u_{t}^{2}(x)\right) \right) +2\frac{1}{n}%
\sum_{\tau =1}^{l_{n}}w_{\tau }\sum_{t=1}^{n}\left( u_{t}(x)u_{t-\tau }(x)-%
\mathrm{E}\left( u_{t}(x)u_{t-\tau }(x)\right) \right) \right\vert  \notag \\
&&+\sup_{x\in \mathcal{X}^{+}}\left\vert \left( \sigma ^{2}(x)-\frac{1}{n}%
\sum_{t=1}^{n}\mathrm{E}\left( u_{t}^{2}(x)\right) +2\frac{1}{n}\sum_{\tau
=1}^{l_{n}}w_{\tau }\sum_{t=1}^{n}\mathrm{E}\left( u_{t}(x)u_{t-\tau
}(x)\right) \right) \right\vert .  \label{main}
\end{eqnarray}%
We begin with the first term on the RHS of (\ref{main}). First note that,%
\begin{eqnarray*}
&&\sup_{x\in \mathcal{X}^{+}}\left\vert \frac{1}{n}\sum_{t=1}^{n}\left(
u_{t}^{2}(x)-\mathrm{E}\left( u_{t}^{2}(x)\right) \right) +2\frac{1}{n}%
\sum_{\tau =1}^{l_{n}}w_{\tau }\sum_{t=1}^{n}\left( u_{t}(x)u_{t-\tau }(x)-%
\mathrm{E}\left( u_{t}(x)u_{t-\tau }(x)\right) \right) \right\vert \\
&\leq &\sup_{x\in \mathcal{X}^{+}}2\sum_{\tau =0}^{l_{n}}\left\vert \frac{1}{%
n}\sum_{t=1}^{n}\left( u_{t}(x)u_{t-\tau }(x)-\mathrm{E}\left(
u_{t}(x)u_{t-\tau }(x)\right) \right) \right\vert .
\end{eqnarray*}%
Now,%
\begin{eqnarray*}
&&\Pr \left( \sup_{x\in \mathcal{X}^{+}}2\sum_{\tau =0}^{l_{n}}\left\vert 
\frac{1}{n}\sum_{t=1}^{n}\left( u_{t}(x)u_{t-\tau }(x)-\mathrm{E}\left(
u_{t}(x)u_{t-\tau }(x)\right) \right) \right\vert \geq \varepsilon \right) \\
&\leq &2\sum_{\tau =0}^{l_{n}}\Pr \left( \sup_{x\in \mathcal{X}%
^{+}}\left\vert \frac{1}{n}\sum_{t=1}^{n}\left( u_{t}(x)u_{t-\tau }(x)-%
\mathrm{E}\left( u_{t}(x)u_{t-\tau }(x)\right) \right) \right\vert \geq 
\frac{\varepsilon }{l_{n}}\right) ,
\end{eqnarray*}%
so that we need to show that,%
\begin{equation*}
\Pr \left( \sup_{x\in \mathcal{X}^{+}}\left\vert \frac{1}{n}%
\sum_{t=1}^{n}\left( u_{t}(x)u_{t-\tau }(x)-\mathrm{E}\left(
u_{t}(x)u_{t-\tau }(x)\right) \right) \right\vert \geq \frac{\varepsilon }{%
l_{n}}\right) <\frac{\delta }{l_{n}}.
\end{equation*}%
Given Assumption A2, WLOG, we can set $\mathcal{X}^{+}=[0,\Delta ],$ so that
it can be covered by $a_{n}^{-1}$ balls $S_{j},$ $j=1,...,\Delta a_{n}^{-1},$
centered at $S_{j},$ with radius $a_{n}.$ Then,%
\begin{eqnarray*}
&&\sup_{x\in \mathcal{X}^{+}}\left\vert \frac{1}{n}\sum_{t=1}^{n}\left(
u_{t}(x)u_{t-\tau }(x)-\mathrm{E}\left( u_{t}(x)u_{t-\tau }(x)\right)
\right) \right\vert \\
&\leq &\max_{j=1,..,\Delta a_{n}^{-1}}\left\vert \frac{1}{n}%
\sum_{t=1}^{n}\left( u_{t}(s_{j})u_{t-\tau }(s_{j})-\mathrm{E}\left(
u_{t}(s_{j})u_{t-\tau }(s_{j})\right) \right) \right\vert \\
&&+\max_{j=1,..,\Delta a_{n}^{-1}}\sup_{x\in S_{j}}2\left\vert \left( \frac{1%
}{n}\sum_{t=1}^{n}u_{t-\tau }(s_{j})\left( u_{t}(x)-u_{t}(s_{j})\right)
\right) \right. \\
&&-\left. \left( \frac{1}{n}\sum_{t=1}^{n}\mathrm{E}\left( u_{t-\tau
}(s_{j})\left( u_{t}(x)-u_{t}(s_{j})\right) \right) \right) \right\vert \\
&&+\text{smaller order} \\
&=&I_{n}+II_{n}.
\end{eqnarray*}%
Now,

\begin{eqnarray*}
II_{n} &\leq &\max_{j=1,..,\Delta a_{n}^{-1}}\sup_{x\in S_{j}}\left\vert 
\frac{1}{n}\sum_{t=1}^{n}u_{t-\tau }(s_{j})\left(
u_{t}(x)-u_{t}(s_{j})\right) \right\vert \\
&&+\max_{j=1,..,\Delta a_{n}^{-1}}\sup_{x\in S_{j}}\left\vert \frac{1}{n}%
\sum_{t=1}^{n}\mathrm{E}\left( u_{t-\tau }(s_{j})\left(
u_{t}(x)-u_{t}(s_{j})\right) \right) \right\vert \\
&=&II_{n}^{A}+II_{n}^{B}.
\end{eqnarray*}%
Given Assumption A1, noting that by Cauchy - Schwarz,%
\begin{eqnarray*}
&&\max_{j=1,..,\Delta a_{n}^{-1}}\sup_{x\in S_{j}}\left\vert \frac{1}{n}%
\sum_{t=1}^{n}\mathrm{E}\left( u_{t-\tau }(s_{j})\left(
u_{t}(s_{j})-u_{t-\tau }(s_{j})\right) \right) \right\vert \\
&\leq &\max_{j=1,..,\Delta a_{n}^{-1}}\sup_{x\in S_{j}}\sqrt{\mathrm{E}%
\left( u_{t-\tau }(s_{j})\right) ^{2}}\max_{j=1,..,a_{n}^{-1}}\sup_{x\in
S_{j}}\sqrt{\mathrm{E}\left( u_{t}(s_{j})-u_{t}(x)\right) ^{2}} \\
&=&O\left( a_{n}^{1/2}\right) ,
\end{eqnarray*}%
for some constant $C.$ Recalling given that $u_{t}(x)=\left( 1\left\{
e_{j,t}\leq x\right\} -F_{j}(x)\right) -\left( 1\left\{ e_{1,t}\leq
x\right\} -F_{1}(x)\right) $ and $u_{t}(s_{j})$ stay between $-1$ and $1$%
\begin{eqnarray*}
&&\max_{j=1,..,\Delta a_{n}^{-1}}\sup_{x\in S_{j}}\left\vert \frac{1}{n}%
\sum_{t=1}^{n}u_{t-\tau }(s_{j})\left( u_{t}(s_{j})-u_{t}(x)\right)
\right\vert \\
&\leq &2\max_{j=1,..,\Delta a_{n}^{-1}}\sup_{x\in S_{j}}\frac{1}{n}%
\sum_{t=1}^{n}\left\vert u_{t}(s_{j})-u_{t}(x)\right\vert \\
&\leq &\frac{2}{n}\sum_{t=1}^{n}1\left\{ x-a_{n}\leq e_{1,t}\leq
x+a_{n}\right\} +\frac{2}{n}\sum_{t=1}^{n}1\left\{ x-a_{n}\leq e_{j,t}\leq
x+a_{n}\right\} \\
&&+2\sup_{x\in \mathcal{X}^{+}}\left( f_{1}(x)+f_{j}(x)\right) \\
&=&O_{p}\left( a_{n}\right) =o_{p}(a_{n}^{1/2})
\end{eqnarray*}%
Hence, by Chebyshev inequality%
\begin{equation*}
l_{n}\Pr \left( II_{n}>\frac{\varepsilon }{l_{n}}\right) =O\left(
a_{n}l_{n}^{3}\right) =o(1),
\end{equation*}%
for $a_{n}=o\left( l_{n}^{-3}\right) .$

Now, consider $I_{n}.$ By the Lemma on page 739 of Hansen (2008), setting $%
a_{n}=l_{n}^{-4},$ $m=\frac{\Delta n}{4l_{n}^{2}},$ and $l_{n}=n^{\delta },$
with $\delta <1/2,$ and recalling that given Assumption A1, $\mathrm{var}%
\left( \sum_{t=1}^{m}\left( u_{t}(s_{j})u_{t-\tau }(s_{j})-\mathrm{E}\left(
u_{t}(s_{j})u_{t-\tau }(s_{j})\right) \right) \right) \leq Cm,$ it follows
that for some constant $C,$

\begin{eqnarray*}
&&\Pr \left( \max_{j=1,..,a_{n}^{-1}}\left\vert \frac{1}{n}%
\sum_{t=1}^{n}\left( u_{t}(s_{j})u_{t-\tau }(s_{j})-\mathrm{E}\left(
u_{t}(s_{j})u_{t-\tau }(s_{j})\right) \right) \right\vert \geq \frac{%
\varepsilon }{l_{n}}\right) \\
&\leq &a_{n}^{-1}\Pr \left( \left\vert \sum_{t=1}^{n}\left(
u_{t}(s_{j})u_{t-\tau }(s_{j})-\mathrm{E}\left( u_{t}(s_{j})u_{t-\tau
}(s_{j})\right) \right) \right\vert \geq \frac{n\varepsilon }{l_{n}}\right)
\\
&\leq &4a_{n}^{-1}\left( \exp \left( -\frac{\frac{n^{2}}{l_{n}^{2}}%
\varepsilon ^{2}}{64Cn+\frac{8}{3}\frac{\Delta n^{2}}{4l_{n}^{3}}}\right) +%
\frac{16}{b}l_{n}^{2}\left( \frac{4}{\Delta }\frac{n}{l_{n}^{2}}\right)
^{-\beta }\right) \\
&=&a_{n}^{-1}\exp \left( -\frac{1}{64C\frac{n}{l_{n}^{2}}+\frac{8}{3}\frac{%
\Delta n^{2}}{4l_{n}^{3}}}\right) +\frac{64}{b}a_{n}^{-1}l_{n}^{2}\NEG%
{l}_{n}^{2\beta }n^{-\beta } \\
&=&o(1)+O\left( n^{\delta \left( 6+2\beta \right) }n^{-\beta }\right) \\
&=&o(1)\text{ for }\beta >\frac{6\delta }{1-2\delta }.
\end{eqnarray*}

We now consider the second term on the RHS of (\ref{main}). Note that%
\begin{eqnarray}
&&\sup_{x\in \mathcal{X}^{+}}\left\vert \left( \sigma ^{2,G+}(x)-\frac{1}{n}%
\sum_{t=1}^{n}\mathrm{E}\left( u_{t}^{2}(x)\right) +2\frac{1}{n}\sum_{\tau
=1}^{l_{n}}w_{\tau }\sum_{t=1}^{n}\mathrm{E}\left( u_{t}(x)u_{t-\tau
}(x)\right) \right) \right\vert  \notag \\
&\leq &2\sup_{x\in \mathcal{X}^{+}}\left\vert \frac{1}{n}\sum_{\tau
=1}^{l_{n}}\left( 1-w_{\tau }\right) \sum_{t=1}^{n}\mathrm{E}\left(
u_{t}(x)u_{t-\tau }(x)\right) \right\vert  \label{NW} \\
&&+2\sup_{x\in \mathcal{X}^{+}}\left\vert \frac{1}{n}\sum_{\tau
=l_{n}+1}^{n}w_{\tau }\sum_{t=1}^{n}\mathrm{E}\left( u_{t}(x)u_{t-\tau
}(x)\right) \right\vert .  \notag
\end{eqnarray}%
The first term on the RHS of (\ref{NW}) is $o_{p}(1),$ by the same argument
as that used in Theorem 2 of Newey and West (1987). Also, by Lemma 6.17 in
White (1984), for $q>2,$%
\begin{equation*}
\mathrm{E}\left( u_{t}(x)u_{t-\tau }(x)\right) \leq C\tau ^{-\beta /2-1/q}%
\mathrm{var}\left( u_{t}(x)\right) ^{1/2}\mathrm{E}\left\Vert
u_{t}(x)\right\Vert ^{q}
\end{equation*}%
and 
\begin{eqnarray*}
&&\sup_{x\in \mathcal{X}^{+}}\left\vert \frac{1}{n}\sum_{\tau
=l_{n}+1}^{n}w_{\tau }\sum_{t=1}^{n}\mathrm{E}\left( u_{t}(x)u_{t-\tau
}(x)\right) \right\vert \\
&\leq &C\sup_{x\in \mathcal{X}^{+}}\mathrm{var}\left( u_{t}(x)\right) ^{1/2}%
\mathrm{E}\left\Vert u_{t}(x)\right\Vert ^{q}\sum_{\tau =l_{n}+1}^{n}\tau
^{-\beta /2-1/q}=o(1),
\end{eqnarray*}%
as $\beta \delta >1,$ given Assumption A1, and noting that $q$ can be taken
arbitrarily large because of the boundedness of $u_{t}(x).$

Finally, by the same argument as that used in the proof of (\ref{main}), for
all $j,$%
\begin{equation*}
\sup_{x\in \mathcal{X}^{+}}\frac{1}{n}\sum_{t=1}^{n}\left( 1\left\{
e_{j,t}\leq x\right\} -F_{j}(x)\right) =o_{p}\left( l_{n}^{-1}\right) .
\end{equation*}%
The statement in (\ref{hat-hat}) follows immediately.

\medskip

\noindent \textbf{(ii)} By noting that,%
\begin{eqnarray*}
&&\left[ e_{j,t}-s_{j}\right] _{+}-\left[ e_{j,t}-x\right] _{+} \\
&=&(x-s_{j})1\{e_{t}\geq x\}+(x-s_{j})\left( 1\{e_{t}\geq x\}-1\{e_{t}\geq
s_{j}\}\right) \\
&&+\left( e_{t}-x\right) \left( 1\{e_{t}\geq s_{j}\}-1\{e_{t}\geq x\}\right)
,
\end{eqnarray*}%
the statement follows by the same argument as that used in part (i) of the
proof.

\bigskip

\noindent $\boldsymbol{\noindent }$\textbf{Proof of Lemma 2: }For notational
simplicity, we suppress the $jj$ subscript. Also, we suppress the
superscripts $C^{+}$ and $G^{+},$ as the proof follows by analogous
argument. Note that%
\begin{eqnarray*}
&&\sup_{x\in \mathcal{X}^{+}}\left\vert \widehat{\sigma }_{n}^{\ast 2}(x)-%
\mathrm{E}^{\ast }\left( \widehat{\sigma }_{n}^{\ast }(x)\right) \right\vert
\\
&\leq &\sup_{x\in \mathcal{X}^{+}}\frac{l_{n}}{b}\sum_{k=1}^{b}\left\vert
\left( \frac{1}{l_{n}}\sum_{j=1}^{l_{n}}u_{(k-1)l_{n}+j}^{\ast }(x)\right)
^{2}-\mathrm{E}^{\ast }\left( \left( \frac{1}{l_{n}}%
\sum_{j=1}^{l_{n}}u_{(k-1)l_{n}+j}^{\ast }(x)\right) ^{2}\right) \right\vert
\\
&=&\sup_{x\in \mathcal{X}^{+}}\frac{l_{n}}{b}\sum_{k=1}^{b}\left\vert \frac{1%
}{l_{n}^{2}}\sum_{j=1}^{l_{n}}\sum_{i=1}^{l_{n}}u_{(k-1)l_{n}+j}^{\ast
}(x)u_{(k-1)l_{n}+i}^{\ast }(x)-\mathrm{E}^{\ast }\left(
u_{(k-1)l_{n}+j}^{\ast }(x)u_{(k-1)l_{n}+i}^{\ast }(x)\right) \right\vert
\end{eqnarray*}%
Now,%
\begin{eqnarray*}
&&\Pr \left( \sup_{x\in \mathcal{X}^{+}}\frac{l_{n}}{b}\sum_{k=1}^{b}\left%
\vert \frac{1}{l_{n}^{2}}\sum_{j=1}^{l_{n}}%
\sum_{i=1}^{l_{n}}u_{(k-1)l_{n}+j}^{\ast }(x)u_{(k-1)l_{n}+i}^{\ast }(x)-%
\mathrm{E}^{\ast }\left( u_{(k-1)l_{n}+j}^{\ast }(x)u_{(k-1)l_{n}+i}^{\ast
}(x)\right) \right\vert \geq \varepsilon _{1}a_{n}\right) \\
&\leq &l_{n}\Pr \left( \sup_{x\in \mathcal{X}^{+}}\frac{l_{n}}{b}%
\sum_{k=1}^{b}\left\vert \frac{1}{l_{n}^{2}}\sum_{j=1}^{l_{n}}%
\sum_{i=1}^{l_{n}}u_{(k-1)l_{n}+j}^{\ast }(x)u_{(k-1)l_{n}+i}^{\ast }(x)-%
\mathrm{E}^{\ast }\left( u_{(k-1)l_{n}+j}^{\ast }(x)u_{(k-1)l_{n}+i}^{\ast
}(x)\right) \right\vert \geq \varepsilon _{1}\frac{a_{n}}{l_{n}}\right) .
\end{eqnarray*}%
It suffices to show that, uniformly in $k,$%
\begin{equation*}
\Pr \left( \sup_{x\in \mathcal{X}^{+}}\left\vert \frac{1}{l_{n}^{2}}%
\sum_{j=1}^{l_{n}}\sum_{i=1}^{l_{n}}u_{(k-1)l+j}^{\ast
}(x)u_{(k-1)l+i}^{\ast }(x)-\mathrm{E}^{\ast }\left( u_{(k-1)l+j}^{\ast
}(x)u_{(k-1)l+i}^{\ast }(x)\right) \right\vert \geq \varepsilon _{1}\frac{%
a_{n}}{l_{n}}\right) <\frac{\delta }{l_{n}}.
\end{equation*}%
This follows using the same "covering numbers" argument used in the proof of
Lemma 1.

\medskip

\noindent \textbf{Proof of Theorem 1:} We again suppress the superscripts $%
G^{+}$ and $C^{+},$ as the proof follows by the same argument. We need to
show that the statement in Lemma A1 in the Supplement Appendix of Andrews
and Shi (2013) holds. Then, the proof of the theorem will follow using the
same arguments as those used in the proof of their Theorem 1, as the proof
is the same for independent and dependent observations. In fact, our set-up
differs from Andrews and Shi (2013) only because we have dependent
observations, and because we scale the statistic by a Newey-West variance
estimator. For the rest of the proof, our set-up is simpler as we can fix
their $\theta _{n}$ at a given value, say zero. It suffices to show that:

\noindent (i) $v_{n}\left( .\right) \Rightarrow v(.),$ as a process indexed
by $x\in \mathcal{X}^{+},$ where $v(.)$ is a zero-mean $k-1-$dimensional
Gaussian process, with covariance kernel given $\Sigma (x,x^{\prime }).$

\noindent (ii) $\sup_{x,x^{\prime }\in \mathcal{X}^{+}}\left\Vert \overline{h%
}_{B,n}(x,x^{\prime })-\overline{h}_{B}(x,x^{\prime })\right\Vert =o_{p}(1).$

\noindent Now, statement (ii) follows directly from Lemma 1. It remains to
show that (i) holds. The key difference between the independent and the
dependent cases is that in the former we can rely on the concept of
manageability, while in the latter we cannot. Nevertheless, (i) follows if
we can show that $v_{n}\left( .\right) $ satisfies an empirical process.
Given A1-A3, this follows from Lemma A2 in Jin, Corradi and Swanson (2017).

\medskip

\noindent \textbf{Proof of Theorem 2: (i)} For notational simplicity, we
omit the superscript $G^{+}.$ The proof of this theorem mirrors the proof of
Theorem 2(a) in the Supplement of Andrews and Shi (2013). Let $c_{0}\left(
h_{A,n},\alpha \right) $ be the $\alpha $ critical value of $S_{n}^{\dag },$
as defined in (\ref{Sn-dag}). Given Theorem 1(i), it follows that for all $%
\delta >0,$%
\begin{equation*}
\lim \sup_{n\rightarrow \infty }\sup_{P\in \mathcal{P}_{0}}P\left( S_{n}\geq
c_{0}\left( h_{A,n},\alpha \right) +\delta \right) \leq \alpha .
\end{equation*}%
The statement follows if we can show that 
\begin{equation}
\lim \sup_{n\rightarrow \infty }\sup_{P\in \mathcal{P}_{0}}P\left(
c_{0,n,\alpha }^{\ast }\left( \phi _{n},\overline{h}_{B,n}^{\ast }\right)
\leq c_{0,n,\alpha }\left( h_{A,n},\overline{h}_{B,n}^{\ast }\right) \right)
=0,  \label{c0*}
\end{equation}%
with $c_{0,n,\alpha }\left( h_{A,n},\overline{h}_{B,n}^{\ast }\right) $
defined as $c_{0}\left( h_{A,n},\alpha \right) ;$ but with $\overline{h}%
_{B,n}^{\ast }$ an argument of this function rather than $h_{B}(x);$ and if
we can show that 
\begin{equation}
\lim \sup_{n\rightarrow \infty }\sup_{P\in \mathcal{P}_{0}}P\left(
c_{0,n,\alpha }\left( h_{A,n},\overline{h}_{B,n}^{\ast }\right) \leq
c_{0}\left( h_{A,n},\alpha \right) \right) =0.  \label{c0}
\end{equation}%
For $c_{n}\rightarrow \infty $ and $c_{n}/\kappa _{n}\rightarrow 0,$ $\tau
_{n}\rightarrow \infty $ and $\tau _{n}/\kappa _{n}\rightarrow 0,$

\begin{eqnarray*}
&&\sup_{P\in \mathcal{P}_{0}}P\left( c_{0,n,\alpha }^{\ast }\left( \phi _{n},%
\overline{h}_{B,n}^{\ast }\right) \leq c_{0,n,\alpha }\left( h_{A,n},%
\overline{h}_{B,n}^{\ast }\right) \right) \\
&\leq &\sup_{P\in \mathcal{P}_{0}}P\left( -\phi _{j,n}(x)\leq h_{A,j,n}(x),%
\text{ for some }x\in \mathcal{X}^{+}\text{ and some }j=2,..,k\right) \\
&\leq &\sup_{P\in \mathcal{P}_{0}}P\left( \xi _{j,n}(x)<-1\text{ AND }%
-c_{n}\leq h_{A,j,n}(x),\text{ for some }x\in \mathcal{X}^{+}\text{ and }%
j=2,..,k\right) \\
&\leq &\sup_{P\in \mathcal{P}_{0}}P\left( D(x)^{1/2}\overline{D}%
_{jj,n}^{-1/2}(x)\left( G_{j,n}(x)-G_{j}(x)\right) +D(x)^{1/2}\overline{D}%
_{jj,n}^{-1/2}(x)h_{j,A,n}(x)<-\kappa _{n}\right. \\
&&\left. \text{AND }-c_{n}\leq h_{A,j,n}(x),\text{ for some }x\in \mathcal{X}%
^{+}\text{ and }j=2,..,k\right) \\
&\leq &\sup_{P\in \mathcal{P}_{0}}P\left( -\tau _{n}+D(x)^{-1/2}\overline{D}%
_{jj,n}^{-1/2}(x)h_{j,A,n}(x)<-\kappa _{n}\right. \\
&&\left. \text{AND }-c_{n}\leq h_{A,j,n}(x),\text{ for some }x\in \mathcal{X}%
^{+}\text{ and }j=2,..,k\right) \\
&&+\sup_{P\in \mathcal{P}_{0}}P\left( D(x)^{1/2}\overline{D}%
_{jj,n}^{-1/2}(x)\left( G_{j,n}(x)-G_{j}(x)\right) <-\tau _{n},\text{ for
some }x\in \mathcal{X}^{+}\text{ and }j=2,..,k\right) \\
&\leq &\sup_{P\in \mathcal{P}_{0}}P\left( -D(x)^{-1/2}\overline{D}%
_{jj,n}^{-1/2}(x)h_{j,A,n}(x)<-\kappa _{n}+c_{n}\right. \\
&&\left. \text{AND }-c_{n}\leq h_{A,j,n}(x),\text{ for some }x\in \mathcal{X}%
^{+}\text{ and }j=2,..,k\right) \\
&=&o(1).
\end{eqnarray*}%
This establishes that (\ref{c0*}) holds. Finally, (\ref{c0}) follows from
Lemma 1 and Lemma 2.

\bigskip

\noindent \textbf{(ii)} Recall that $c_{0,n,1-\alpha }^{\ast }\left( \phi
_{n},h_{B,n}\right) $ is the ($1-\alpha )-$ percentile of $S_{n}^{\ast },$
as defined in (\ref{Sn*}); and define $c_{0,n,1-\alpha }^{GMS}\left( \phi
_{n},\overline{h}_{B,n}\right) $ to the ($1-\alpha )-$ percentile of $%
S_{n}^{GMS},$ where%
\begin{equation*}
S_{n}^{GMS}=\max_{x\in \mathcal{X}^{+}}\sum_{j=2}^{k}\max \left( \left\{ 0,%
\frac{\overline{v}_{j,n}(x)-\phi _{j,n}(x)}{\sqrt{\overline{h}_{B,jj}(x)}}%
\right\} \right) ^{2},
\end{equation*}%
with $\overline{v}_{n}=(\overline{v}_{2,n},...,\overline{v}_{k,n})^{\prime }$
is a $k-1$ dimensional Gaussian process, with mean zero and covariance $%
\overline{h}_{B,n}(x,x^{\prime })=\widehat{D}_{n}^{-1/2}(x)\overline{\Sigma }%
(x,x^{\prime })\widehat{D}_{n}^{-1/2}(x^{\prime }).$ Finally, let $%
v=(v_{2},...,v_{k})^{\prime }$ is a $k-1$ dimensional Gaussian process, with
mean zero and covariance $\overline{h}_{B}(x,x^{\prime })=D^{-1/2}(x)%
\overline{\Sigma }(x,x^{\prime })D^{-1/2}(x^{\prime }).$We first need to
show that%
\begin{equation}
c_{0,n,1-\alpha }^{\ast }\left( \phi _{n},h_{B,n}\right) -c_{0,n,1-\alpha
}^{GMS}\left( \phi _{n},\overline{h}_{B,n}\right) =o_{p}(1),  \label{cv}
\end{equation}%
and then to prove that the statement holds when replacing $c_{0,n,1-\alpha
}^{\ast }\left( \phi _{n},h_{B,n}\right) $ with $c_{0,n,1-\alpha
}^{GMS}\left( \phi _{n},\overline{h}_{B,n}\right) .$

From Lemma 2, $\widehat{\Sigma }_{n}^{\ast }\left( x,x^{\prime }\right) -%
\widehat{\Sigma }_{n}\left( x,x^{\prime }\right) =o_{p}^{\ast }(1),$ and so $%
\overline{\Sigma }_{n}^{\ast }\left( x,x^{\prime }\right) -\overline{\Sigma }%
_{n}\left( x,x^{\prime }\right) =o_{p}^{\ast }(1).$ Then, by Theorem 2.3 in
Peligrad (1998),%
\begin{equation*}
v^{\ast }\overset{d^{\ast }}{\Longrightarrow }v\text{ a.s.-}\omega ,
\end{equation*}%
where $v^{\ast }\overset{\ast }{\Longrightarrow }v$ denotes weak
convergence, conditional on sample. As $\overline{v}_{n}\Longrightarrow v,$ (%
\ref{cv}) follows.

Given Assumption A4, by Lemma B3 in the Supplement of Andrews and Shi
(2013), the distribution of $S_{\infty }^{\dag },$ as defined in (\ref%
{Sn-inf}), is continuous. It is also strictly increasing and its $(1-\alpha
)-$quantile is strictly positive, for all $\alpha <1/2.$ The statement then
follows by the same argument as that used in the proof of Theorem 2(b) in
the Supplement of Andrews and Shi (2013).

\textbf{(iii)-(iv)} follow by the same arguments as those used in the proof
of \textbf{(i)} and \textbf{(ii)}, respectively. In the case of $%
S_{n}^{G^{+}},$ we rely on the the stochastic equicontinuity of $\frac{1}{%
\sqrt{n}}\sum_{i=1}^{n}\left( 1\left\{ e_{1,i}\leq x\right\} -1\left\{
e_{1,i}\leq u\right\} \right) ,$ as $|x-u|\rightarrow 0.$ When considering $%
S_{n}^{C^{+}},$ we need to ensure the stochastic equicontinuity of $\frac{1}{%
\sqrt{n}}\sum_{i=1}^{n}\left( \left( e_{1,i}-x\right) _{+}-\left(
e_{1,i}-u\right) _{+}\right) .$ Now,%
\begin{eqnarray*}
&&\frac{1}{\sqrt{n}}\sum_{i=1}^{n}\left( \left( e_{1,i}-x\right) _{+}-\left(
e_{1,i}-u\right) _{+}\right) \\
&=&\frac{1}{\sqrt{n}}\sum_{i=1}^{n}\left( u-x\right) 1\left\{ e_{1,i}\geq
u\right\} +\frac{1}{\sqrt{n}}\sum_{i=1}^{n}\left( e_{1,i}-u\right)
_{+}\left( 1\left\{ e_{1,i}\geq x\right\} -1\left\{ e_{1,i}\geq u\right\}
\right) ,
\end{eqnarray*}%
which, given Assumption 2, is stochastically equicontinous, by the same
argument as those used for $S_{n}^{G^{+}}.$ Hence, Theorem 2.3 in Peligrad
(1998) also holds in this case.

\medskip

\noindent \textbf{Proof of Theorem 3: (i)} Without loss of generality, let $%
B_{FA}^{G+}=\left\{ x\in \mathcal{X}^{+}:G_{2}(x)>0\right\} ,$ and note that
for all $x\in B_{FA}^{G+},$ $\max \left\{ 0,\frac{\sqrt{n}G_{2,n}^{+}(x)}{%
\overline{\sigma }_{22,n}^{G+}(x)}\right\} =\frac{\sqrt{n}G_{2,n}^{+}(x)}{%
\overline{\sigma }_{22,n}^{G+}(x)}.$ Thus,%
\begin{eqnarray*}
S_{n}^{G+} &=&\int_{B_{FA}^{G+}}\sum_{j=2}^{k}\left( \max \left\{ 0,\frac{%
\sqrt{n}G_{j,n}^{+}(x)}{\overline{\sigma }_{jj,n}^{G+}(x)}\right\} \right)
^{2}\mathrm{d}Q(x)+\int_{\mathcal{X}^{+}\backslash
B_{FA}^{G+}}\sum_{j=2}^{k}\left( \max \left\{ 0,\frac{\sqrt{n}G_{j,n}^{+}(x)%
}{\overline{\sigma }_{jj,n}^{G+}(x)}\right\} \right) ^{2}\mathrm{d}Q(x) \\
&=&\int_{B_{FA}^{G+}}\left( \frac{\sqrt{n}G_{2,n}^{+}(x)}{\overline{\sigma }%
_{22,n}^{G+}(x)}\right) ^{2}\mathrm{d}Q(x)+\int_{B_{FA}^{G+}}\sum_{j=3}^{k}%
\left( \max \left\{ 0,\frac{\sqrt{n}G_{j,n}^{+}(x)}{\overline{\sigma }%
_{jj,n}^{G+}(x)}\right\} -\left( \frac{\sqrt{n}G_{2,n}^{+}(x)}{\overline{%
\sigma }_{22,n}^{G+}(x)}\right) \right) ^{2}\mathrm{d}Q(x) \\
&&+\int_{\mathcal{X}^{+}\backslash B_{FA}^{G+}}\sum_{j=2}^{k}\left( \max
\left\{ 0,\frac{\sqrt{n}G_{j,n}^{+}(x)}{\overline{\sigma }_{jj,n}^{G+}(x)}%
\right\} \right) ^{2}\mathrm{d}Q(x) \\
&=&I_{n}+II_{n}+III_{n}.
\end{eqnarray*}%
Now, $I_{n}$ diverges to infinity with probability approaching one, while
Theorem 1 ensures that $II_{n}$ and $III_{n}$ are $O_{p}(1)$. Thus, $%
S_{n}^{G+}$ diverges to infinity$.$ As $S_{n}^{\ast G+}$ is $O_{p^{\ast
}}(1),$ conditional on the sample, the statement follows.

\noindent \textbf{(ii)} Note that $S_{n}^{C^{+}}$ can be treated exactly as $%
S_{n}^{G^{+}}.$

\medskip

\noindent \textbf{Proof of Theorem 4:}

\noindent \textbf{(i)} Define, $S_{\infty ,LA}^{\dag G+}$ as in (\ref{Sn-inf}%
), but with the vector $h_{j,A,\infty }^{G+}(x)$ having at least one
component strictly bounded away above from zero, and finite, for all $x\in
B_{LA}^{G+}.$ Let $\mathcal{P}_{n,LA}^{G+}$ denote the set of probabilities
under the sequence of local alternatives. We have that for all $a>0,$ 
\begin{equation*}
\lim \sup_{n\rightarrow \infty }\sup_{P\in \mathcal{P}_{n,LA}^{G+}}\left[
P\left( S_{n}^{G+}>a\right) -P\left( S_{\infty ,LA}^{\dag G+}>a\right) %
\right] =0,
\end{equation*}%
and the distribution of $S_{\infty ,LA}^{\dag G+}$ is continuous at its $%
(1-\alpha )+\delta $ quintile, for all $0<\alpha <1/2$ and $\delta \geq 0.$
Also, note that for all $x\in B_{LA}^{G+},$ $\phi _{n}^{G+}=0.$ The
statement then follows by the same argument as that used in the proof of
Theorem 2\textbf{(}ii\textbf{)}. (\textbf{ii) }By the same argument as in
part (i).

\newpage

\section{References}

\noindent Aiolfi, M., C. Capistr\'{a}n, and A. Timmermann (2011). Forecast
Combinations. In M.P. Clements and D\medskip .F. Hendry (eds.), \textbf{%
Oxford Handbook of Economic Forecasting}, pp. 355-390, Oxford University
Press, Oxford.\medskip

\noindent Andrews, D.W.K. (1991). Heteroskedasticity and Autocorrelation
Robust Covariance Matrix Estimation. \textit{Econometrica, }59,
817-858.\medskip

\noindent Andrews, D.W.K. and D. Pollard (1994). An Introduction to
Functional Central Limit Theorems for Dependent Stochastic Processes. 
\textit{International Statistical Review, }62, 119-132.\medskip

\noindent Andrews, D.W.K. and X. Shi (2013). Inference Based on Conditional
Moment Inequalities. \textit{Econometrica, }81, 609-666.\medskip

\noindent Andrews, D.W.K. and X. Shi (2017). Inference Based on Many
Conditional Moment Inequalities. \textit{Journal of Econometrics, }196,
275-287.\medskip

\noindent Barendse, S. and A.J. Patton (2019). Comparing Predictive Accuracy
in the Presence of a Loss Function Shape Parameter. Working Paper, Duke
University.\medskip

\noindent Bierens H.J. (1982). Consistent Model Specification Tests. \textit{%
Journal of Econometrics, }20, 105-134.\medskip

\noindent Bierens H.J. (1990). A Consistent Conditional Moment Tests for
Functional Form. \textit{Econometrica, }58, 1443-1458.\medskip

\noindent Clark, T. and M. McCracken (2013). Advances in Forecast
Evaluation. In G. Elliott, C.W.J. Granger and A. Timmermann (eds.), \textbf{%
Handbook of Economic Forecasting Vol. 2}, pp. 1107-1201, Elsevier,
Amsterdam.\medskip

\noindent Corradi, V. and N.R. Swanson (2003). Predictive Density
Evaluation. In G. Elliott, C.W.J. Granger and A. Timmermann (eds.), \textbf{%
Handbook of Economic Forecasting Vol. 1}, pp. 197-284, Elsevier,
Amsterdam.\medskip

\noindent Corradi, V. and N.R. Swanson (2007). Nonparametric Bootstrap
Procedures for Predictive Inference Based on Recursive Estimation Schemes. 
\textit{International Economic Review, }48, 67-109.\medskip

\noindent Corradi, V. and N. R. Swanson (2013). A Survey of Recent Advances
in Forecast Accuracy Comparison Testing, with an Extension to Stochastic
Dominance. In X. Chen and N.R. Swanson (eds.), \textbf{Causality,
Prediction, and Specification Analysis: Recent Advances and Future
Directions, Essays in honor of Halbert L. White, Jr.}, pp. 121-144,
Springer, New York.\medskip

\noindent Croushore, D. (1993). Introducing: The Survey of Professional
Forecasters, The Federal Reserve Bank of Philadelphia Business Review,
November-December, 3-15.\medskip

\noindent Diebold, F. X. and Mariano, R. S. (1995). Comparing Predictive
Accuracy. \textit{Journal of Business and Economic Statistics}, 13,
253-263.\medskip

\noindent Diebold, F.X. and M. Shin (2015). Assessing Point Forecast
Accuracy by Stochastic Loss Distance. \textit{Economics Letters, }130,
37-38.\medskip

\noindent Diebold, F.X. and M. Shin (2017). Assessing Point Forecast
Accuracy by Stochastic Error Distance. \textit{Econometric Reviews, }36,
588-598.\medskip

\noindent Elliott, G., I. Komunjer and A. Timmermann (2005). Estimation and
Testing of Forecast Rationality under Flexible Loss. \textit{Review of
Economic Studies, }72, 1107-1125.\medskip

\noindent Elliott, G., I. Komunjer and A. Timmermann (2008). Biases in
Macroeconomic Forecasts: Irrationality of Asymmetric Loss? \textit{Journal
of the European Economic Association, }6, 122-157.\medskip

\noindent Fair, R.C. and R.J. Shiller (1990). Comparing Information in
Forecasts from Econometric Models. \textit{American Economic Review}, 80,
375-389.\medskip

\noindent Genre, V., G. Kenny, A. Meyler, and A. Timmermann (2013).
Combining the Forecasts in the ECB Survey of Professional Forecasters: Can
Anything Beat the Simple Average. \textit{International Journal of
Forecasting}, 29, 108-121.\medskip

\noindent Gneiting, T. (2011). Making and Evaluating Point Forecast. \textit{%
Journal of the American Statistical Association, }106, 746-762.\medskip

\noindent Granger, C. W. J. (1999). Outline of Forecast Theory using
Generalized Cost Functions. \textit{Spanish Economic Review,} 1,
161-173.\medskip

\noindent Hansen, B.E. (2008). Uniform Convergence Rates for Kernel
Estimators with Dependent Data. \textit{Econometric Theory, }24,
726-748.\medskip

\noindent Holm, S. (1979). A Simple Sequentially Rejective Multiple Test
Procedure. \textit{Scandinavian Journal of Statistics,} 6, 65-70.\medskip

\noindent Jin, S., V. Corradi and N.R. Swanson (2017). Robust Forecast
Comparison. \textit{Econometric Theory, }33, 1306-1351.\medskip

\noindent Lahiri, K., H. Peng, and Y. Zhao (2015). Testing the Value of
Probability Forecasts for Calibrated Combining. \textit{International
Journal of Forecasting}, 31, 113-129.\medskip

\noindent Lahiri, K., H. Peng, and Y. Zhao (2017). Online Learning and
Forecast Combination in Unbalanced Panels. \textit{Econometric Reviews, }36,
257-288.\medskip

\noindent Linton, O., K. Song and Y.J. Whang (2010). An Improved Bootstrap
Test of Stochastic Dominance. \textit{Journal of Econometrics, }154,
186-202.\medskip

\noindent McCracken, M.W. (2000). Robust Out-of-Sample Inference. \textit{%
Journal of Econometrics}, 99, 195-223.\medskip

\noindent Newey, W.K. and West, K.D. (1987). A Simple, Positive Definite,
Heteroskedasticty and Autocorrelation Consistent Covariance Matrix. \textit{%
Econometrica,} 55, 703-708.\medskip

\noindent Patton, A.J. (2019). Comparing Possibly Misspecified Forecasts. 
\textit{Journal of Business \& Economic Statistics, }forthcoming.\medskip

\noindent Patton, A.J. and A. Timermann (2007). \ Testing Forecast
Optimality under Unknown Loss. \textit{Journal of the American Statistical
Association, }v.102, 1172-1184.\medskip

\noindent Peligrad, M. (1998). On the Blockwise Bootstrap for Empirical
Processes for Stationary Sequences. \textit{Annals of Probability, }26,
877-901.\medskip

\noindent Swanson, N.R. and H. White (1997a). A Model Selection Approach to
Real-Time Macroeconomic Forecasting Using Linear Models and Artificial
Neural Networks. \textit{Review of Economics and Statistics}, 79, 1997,
540-550.\medskip

\noindent Swanson, N.R. and H. White (1997b). Forecasting Economic Time
Series Using Adaptive Versus Nonadaptive and Linear Versus Nonlinear
Econometric Models. \textit{International Journal of Forecasting,} 13, 1997,
439-461.\medskip

\noindent Timmermann, A. (2006). Forecast Combinations. In A. Timmermann,
C.W.J. Granger, and G. Elliott (eds.), \textbf{Handbook of Forecasting Vol. 1%
}, pp. 135-196\textbf{. }North Holland, Amsterdam.\medskip

\noindent White, H. (2000). A Reality Check for Data Snooping. \textit{%
Econometrica} 68, 1097-1126.\medskip

\noindent White, H. (1984). \textbf{Asymptotic Theory for Econometricians}.
Academic Press, San Diego.\medskip

\noindent Zarnowitz, V. and P. Braun, (1993). Twenty-Two Years of the
NBER-ASA Quarterly Economic Outlook Surveys: Aspects and Comparisons of
Forecasting Performance. In J.H. Stock and M.W. Watson (eds.), \textbf{%
Business Cycles, Indicators, and Forecasting}, pp. 11-94, University of
Chicago Press, Chicago.

\bigskip

\begin{center}
\linespread{1.1}

\begin{table}[tbp]
\caption{Monte Carlo Results for $JCS_n^{G+}$, $JCS_n^{G-}$, $JCS_n^{C+}$,
and $JCS_n^{C-}$ Forecast Superiority Tests$^*$}{\centering}
\par
\hspace{1cm}
\par
\begin{tabular}{cc|cccc|cccc}
\hline\hline
$DGP$ & $n$ & {$J_n = 0.20$} & {$J_n = 0.35$} & {$J_n = 0.50$} & {$J_n =
0.65 $} & {$J_n = 0.20$} & {$J_n = 0.35$} & {$J_n = 0.50$} & {$J_n = 0.65$}
\\ 
&  & \multicolumn{4}{c}{GL Forecast Superiority} & \multicolumn{4}{c}{CL
Forecast Superiority} \\ \hline
&  & \multicolumn{8}{c}{$Empirical$ $Size$} \\ \hline
DGP1 & 300 & 0.113 & 0.100 & 0.101 & 0.112 & 0.113 & 0.107 & 0.120 & 0.115
\\ 
& 600 & 0.105 & 0.110 & 0.099 & 0.108 & 0.091 & 0.089 & 0.094 & 0.091 \\ 
& 900 & 0.102 & 0.095 & 0.093 & 0.096 & 0.094 & 0.092 & 0.082 & 0.094 \\ 
\hline
DGP2 & 300 & 0.110 & 0.105 & 0.115 & 0.113 & 0.097 & 0.097 & 0.092 & 0.106
\\ 
& 600 & 0.077 & 0.073 & 0.082 & 0.079 & 0.090 & 0.092 & 0.089 & 0.081 \\ 
& 900 & 0.085 & 0.084 & 0.086 & 0.095 & 0.089 & 0.101 & 0.097 & 0.092 \\ 
\hline
DGP3 & 300 & 0.070 & 0.065 & 0.065 & 0.060 & 0.030 & 0.030 & 0.035 & 0.035
\\ 
& 600 & 0.050 & 0.040 & 0.050 & 0.045 & 0.030 & 0.020 & 0.030 & 0.025 \\ 
& 900 & 0.070 & 0.070 & 0.080 & 0.075 & 0.020 & 0.020 & 0.020 & 0.020 \\ 
\hline
DGP4 & 300 & 0.065 & 0.070 & 0.060 & 0.065 & 0.010 & 0.015 & 0.020 & 0.015
\\ 
& 600 & 0.040 & 0.040 & 0.040 & 0.055 & 0.015 & 0.015 & 0.015 & 0.020 \\ 
& 900 & 0.070 & 0.065 & 0.065 & 0.065 & 0.015 & 0.015 & 0.015 & 0.015 \\ 
\hline
&  & \multicolumn{8}{c}{$Empirical$ $Power$} \\ \hline\hline
DGP5 & 300 & 0.496 & 0.485 & 0.490 & 0.477 & 0.753 & 0.759 & 0.745 & 0.759
\\ 
& 600 & 0.775 & 0.771 & 0.773 & 0.773 & 0.991 & 0.986 & 0.989 & 0.981 \\ 
& 900 & 0.943 & 0.951 & 0.947 & 0.938 & 1.000 & 1.000 & 1.000 & 1.000 \\ 
\hline
DGP6 & 300 & 0.483 & 0.480 & 0.476 & 0.474 & 0.758 & 0.741 & 0.745 & 0.736
\\ 
& 600 & 0.768 & 0.778 & 0.772 & 0.774 & 0.984 & 0.975 & 0.981 & 0.980 \\ 
& 900 & 0.954 & 0.949 & 0.954 & 0.947 & 1.000 & 1.000 & 1.000 & 1.000 \\ 
\hline
DGP7 & 300 & 0.490 & 0.475 & 0.475 & 0.445 & 0.875 & 0.865 & 0.845 & 0.855
\\ 
& 600 & 0.835 & 0.820 & 0.820 & 0.800 & 0.995 & 0.995 & 0.995 & 0.995 \\ 
& 900 & 0.975 & 0.970 & 0.970 & 0.965 & 1.000 & 1.000 & 1.000 & 1.000 \\ 
\hline
DGP8 & 300 & 1.000 & 1.000 & 1.000 & 1.000 & 1.000 & 1.000 & 1.000 & 1.000
\\ 
& 600 & 1.000 & 1.000 & 1.000 & 1.000 & 1.000 & 1.000 & 1.000 & 1.000 \\ 
& 900 & 1.000 & 1.000 & 1.000 & 1.000 & 1.000 & 1.000 & 1.000 & 1.000 \\ 
\hline
DGP9 & 300 & 0.643 & 0.660 & 0.650 & 0.629 & 0.949 & 0.944 & 0.937 & 0.948
\\ 
& 600 & 0.913 & 0.885 & 0.890 & 0.896 & 1.000 & 1.000 & 1.000 & 1.000 \\ 
& 900 & 0.990 & 0.986 & 0.984 & 0.983 & 1.000 & 1.000 & 1.000 & 1.000 \\ 
\hline
DGP10 & 300 & 1.000 & 1.000 & 1.000 & 1.000 & 1.000 & 1.000 & 1.000 & 1.000
\\ 
& 600 & 1.000 & 1.000 & 1.000 & 1.000 & 1.000 & 1.000 & 1.000 & 1.000 \\ 
& 900 & 1.000 & 1.000 & 1.000 & 1.000 & 1.000 & 1.000 & 1.000 & 1.000 \\ 
\hline\hline
\end{tabular}%
\par
\begin{minipage}{1\columnwidth}
\vspace{0.1in}
{\noindent $^{*}$ Notes: Entries denote rejection frequencies of ($JCS_n^{G+}$,$JCS_n^{G-}$) tests (i.e., GL forecast superiority) and ($JCS_n^{C+}$,$JCS_n^{C-}$) tests 
(i.e., CL forecast superiority) under a variety of data generating processes 
denoted by DGP1-DGP10. In DGP1-DGP4, no alternative outperforms the benchmark model. In DGP5-DGP10, at least one alternative model outperfroms the benchmark model.
Sample sizes include $n$=300, 600, and 900 observations, as indicated in the second column of entries in the table. Nominal test size is 10\%, and tests 
are carried out using critical values 
constructed for values of $J_n$ including 0.20, 0.35, 0.50, and 0.65. See Section 4 for complete details.}
\end{minipage}
\end{table}

\begin{table}[tbp]
\caption{Monte Carlo Results for $S_n^{G+}$, $S_n^{G-}$, $S_n^{C+}$, and $%
S_n^{C-}$ Forecast Superiority Tests$^*$}{\centering}
\par
\hspace{1cm}
\par
\begin{tabular}{cc|cccc|cccc}
\hline\hline
$DGP$ & $n$ & {$\eta = 0.0015$} & {$\eta = 0.002$} & {$\eta = 0.0025$} & {$%
\eta = 0.003$} & {$\eta = 0.0015$} & {$\eta = 0.002$} & {$\eta = 0.0025$} & {%
$\eta = 0.003$} \\ 
&  & \multicolumn{4}{c}{GL Forecast Superiority} & \multicolumn{4}{c}{CL
Forecast Superiority} \\ \hline
&  & \multicolumn{8}{c}{$Empirical$ $Size$} \\ \hline
DGP1 & 300 & 0.078 & 0.078 & 0.076 & 0.076 & 0.095 & 0.094 & 0.094 & 0.091
\\ 
& 600 & 0.096 & 0.096 & 0.095 & 0.095 & 0.116 & 0.116 & 0.115 & 0.114 \\ 
& 900 & 0.120 & 0.119 & 0.118 & 0.117 & 0.096 & 0.096 & 0.095 & 0.095 \\ 
\hline
DGP2 & 300 & 0.068 & 0.067 & 0.067 & 0.066 & 0.095 & 0.095 & 0.095 & 0.094
\\ 
& 600 & 0.097 & 0.096 & 0.095 & 0.095 & 0.096 & 0.096 & 0.095 & 0.094 \\ 
& 900 & 0.111 & 0.108 & 0.108 & 0.105 & 0.106 & 0.105 & 0.105 & 0.105 \\ 
\hline
DGP3 & 300 & 0.021 & 0.021 & 0.021 & 0.021 & 0.038 & 0.038 & 0.038 & 0.038
\\ 
& 600 & 0.070 & 0.070 & 0.069 & 0.068 & 0.086 & 0.086 & 0.085 & 0.085 \\ 
& 900 & 0.071 & 0.070 & 0.069 & 0.069 & 0.083 & 0.083 & 0.083 & 0.082 \\ 
\hline
DGP4 & 300 & 0.010 & 0.010 & 0.01 & 0.010 & 0.041 & 0.041 & 0.041 & 0.040 \\ 
& 600 & 0.064 & 0.064 & 0.064 & 0.063 & 0.077 & 0.077 & 0.076 & 0.076 \\ 
& 900 & 0.069 & 0.067 & 0.064 & 0.063 & 0.083 & 0.083 & 0.083 & 0.083 \\ 
\hline
&  & \multicolumn{8}{c}{$Empirical$ $Power$} \\ \hline
DGP5 & 300 & 0.911 & 0.911 & 0.910 & 0.909 & 0.972 & 0.972 & 0.971 & 0.971
\\ 
& 600 & 1.000 & 1.000 & 1.000 & 1.000 & 1.000 & 1.000 & 1.000 & 1.000 \\ 
& 900 & 1.000 & 1.000 & 1.000 & 1.000 & 1.000 & 1.000 & 1.000 & 1.000 \\ 
\hline
DGP6 & 300 & 0.957 & 0.956 & 0.956 & 0.956 & 0.989 & 0.989 & 0.989 & 0.989
\\ 
& 600 & 1.000 & 1.000 & 1.000 & 1.000 & 1.000 & 1.000 & 1.000 & 1.000 \\ 
& 900 & 1.000 & 1.000 & 1.000 & 1.000 & 1.000 & 1.000 & 1.000 & 1.000 \\ 
\hline
DGP7 & 300 & 0.874 & 0.872 & 0.871 & 0.870 & 0.925 & 0.924 & 0.923 & 0.923
\\ 
& 600 & 1.000 & 1.000 & 1.000 & 1.000 & 1.000 & 1.000 & 1.000 & 1.000 \\ 
& 900 & 1.000 & 1.000 & 1.000 & 1.000 & 1.000 & 1.000 & 1.000 & 1.000 \\ 
\hline
DGP8 & 300 & 1.000 & 1.000 & 1.000 & 1.000 & 1.000 & 1.000 & 1.000 & 1.000
\\ 
& 600 & 1.000 & 1.000 & 1.000 & 1.000 & 1.000 & 1.000 & 1.000 & 1.000 \\ 
& 900 & 1.000 & 1.000 & 1.000 & 1.000 & 1.000 & 1.000 & 1.000 & 1.000 \\ 
\hline
DGP9 & 300 & 0.995 & 0.995 & 0.995 & 0.995 & 0.995 & 0.995 & 0.995 & 0.995
\\ 
& 600 & 1.000 & 1.000 & 1.000 & 1.000 & 1.000 & 1.000 & 1.000 & 1.000 \\ 
& 900 & 1.000 & 1.000 & 1.000 & 1.000 & 1.000 & 1.000 & 1.000 & 1.000 \\ 
\hline
DGP10 & 300 & 1.000 & 1.000 & 1.000 & 1.000 & 1.000 & 1.000 & 1.000 & 1.000
\\ 
& 600 & 1.000 & 1.000 & 1.000 & 1.000 & 1.000 & 1.000 & 1.000 & 1.000 \\ 
& 900 & 1.000 & 1.000 & 1.000 & 1.000 & 1.000 & 1.000 & 1.000 & 1.000 \\ 
\hline\hline
\end{tabular}%
\par
\begin{minipage}{1\columnwidth}
\vspace{0.1in}
{\noindent $^{*}$ Notes: Entries denote rejection frequencies of ($S_n^{G+}$,$S_n^{G-}$) tests (i.e., GL forecast superiority) and ($S_n^{C+}$,$S_n^{C-}$) tests (i.e., CL forecast superiority) under a variety of data generating processes 
denoted by DGP1-DGP10. In DGP1-DGP4, no alternative outperforms the benchmark model. In DGP5-DGP10, at least one alternative model outperfroms the benchmark model.
Sample sizes include $n$=300, 600, and 900 observations, as indicated in the second column of entries in the table. Nominal test size is 10\%, and tests are carried out using critical values 
constructed for values of $\eta$ including 0.0015, 0.002, 0.0025, and 0.0030. See Section 4 for complete details.}
\end{minipage}
\end{table}

\bigskip

\linespread{1.05}

\begin{table}[tbp]
\caption{SPF Forecast Pooling Analysis of Quarterly Nominal GDP Using Mean
Benchmark Model and Mean Expert Pool Predictions$^*$}{\centering}
\par
\hspace{0.3cm}
\par
\begin{center}
\begin{tabular}{cl|lllll}
\hline\hline
$Group$ & $Statistic$ & \multicolumn{5}{c}{$Forecast$ $Horizon$} \\ \hline
&  & $h=0$ & $h=1$ & $h=2$ & $h=3$ & $h=4$ \\ \hline\hline
Group 1 & $S_n^G$ & 0.000718 & 0.000781 & 0.001284 & 0.001525 & 0.024494 \\ 
& $S_n^C$ & 0.000008 & 0.000020 & 0.000203 & 0.000172 & 0.024470 \\ 
& $JCS_n^G$ & 0.232845 & 0.232845 & 0.232845 & 0.310460 & 0.024611 \\ 
& $JCS_n^C$ & 0.000357 & 0.000333 & 0.000729 & 0.000745 & 0.025276 \\ \hline
Group 2 & $S_n^G$ & 0.000820 & 0.002681 & 0.002334 & 0.002451 & 0.008070 \\ 
& $S_n^C$ & 0.000037 & 0.001516 & 0.001983 & 0.002675 & 0.003287 \\ 
& $JCS_n^G$ & 0.543305 & 0.698535 & 1.164226$^*$ & 0.698535 & 1.397071$^*$
\\ 
& $JCS_n^C$ & 0.000115 & 0.006752 & 0.009945 & 0.011544 & 0.021668 \\ \hline
Group 3 & $S_n^G$ & 0.001975 & 0.004097 & 0.005302 & 0.004784 & 0.009391 \\ 
& $S_n^C$ & 0.000878 & 0.001306 & 0.008277 & 0.004486 & 0.014821$^*$ \\ 
& $JCS_n^G$ & 0.465690 & 0.698535 & 0.776151 & 0.620920 & 1.164226$^*$ \\ 
& $JCS_n^C$ & 0.002955 & 0.005766 & 0.011527$^*$ & 0.010482 & 0.020510$^*$
\\ \hline
Group 4 & $S_n^G$ & 0.001779 & 0.005075 & 0.004359 & 0.003820 & 0.008772 \\ 
& $S_n^C$ & 0.000512 & 0.001760 & 0.004896 & 0.004161 & 0.009259 \\ 
& $JCS_n^G$ & 0.543305 & 0.853766 & 0.776151 & 0.620920 & 1.241841$^*$ \\ 
& $JCS_n^C$ & 0.002621 & 0.007716$^*$ & 0.008856 & 0.010377 & 0.022399$^*$
\\ \hline
Group 5 & $S_n^G$ & 0.001540 & 0.003063 & 0.005878 & 0.007226 & 0.012886$^*$
\\ 
& $S_n^C$ & 0.000507 & 0.000842 & 0.008293 & 0.012770$^*$ & 0.020682$^*$ \\ 
& $JCS_n^G$ & 0.465690 & 0.776151 & 0.620920 & 0.853766 & 1.008996$^*$ \\ 
& $JCS_n^C$ & 0.002235 & 0.004643 & 0.010388$^*$ & 0.015376$^*$ & 0.018873$%
^* $ \\ \hline
Group 6 & $S_n^G$ & 0.002384 & 0.007759 & 0.005390 & 0.004585 & 0.012976 \\ 
& $S_n^C$ & 0.000369 & 0.002493 & 0.008548 & 0.007844 & 0.022130$^*$ \\ 
& $JCS_n^G$ & 0.776151 & 1.164226$^*$ & 0.776151 & 0.698535 & 1.008996 \\ 
& $JCS_n^C$ & 0.002619 & 0.008085$^*$ & 0.009990 & 0.014823 & 0.020172 \\ 
\hline
Group 7 & $S_n^G$ & 0.002284 & 0.006819 & 0.008704 & 0.009044 & 0.008960 \\ 
& $S_n^C$ & 0.000986 & 0.002106 & 0.011060 & 0.008702 & 0.007617 \\ 
& $JCS_n^G$ & 0.465690 & 0.931381$^*$ & 1.086611$^*$ & 0.698535 & 0.931381
\\ 
& $JCS_n^C$ & 0.002803 & 0.005994$^*$ & 0.011401$^*$ & 0.011645 & 0.012845
\\ \hline
Group 8 & $S_n^G$ & 0.001857 & 0.004237 & 0.003911 & 0.006975 & 0.016996 \\ 
& $S_n^C$ & 0.000452 & 0.001066 & 0.004071 & 0.007224 & 0.015145 \\ 
& $JCS_n^G$ & 0.931381 & 0.776151 & 0.698535 & 0.853766 & 1.397071$^*$ \\ 
& $JCS_n^C$ & 0.002526 & 0.004420 & 0.007946 & 0.008453 & 0.020293$^*$ \\ 
\hline\hline
\end{tabular}%
\end{center}
\par
\begin{minipage}{1\columnwidth}
\vspace{0.05in}
{\noindent $^{*}$ Notes: Entries are $S_n^G$, $S_n^C$, $JCS_n^G$, and $JCS_n^C$ test statistics reported for forecast horizons $h={0,1,2,3,4}$. More specifically, 
$S_n^G = S_n^{G+}$ $if$ $p_{B,n,S_n^{G+}}^{G+} \le p_{B,n,S_n^{G-}}^{G-}$; $otherwise$ $S_n^G = S_n^{G-}$. $S_n^C$, $JCS_n^G$, and $JCS_n^C$ are defined analogously. 
Rejections of the null of no forecast superiority at a 10\% level are denoted by a superscipt *. See Section 5 for complete details.}
\end{minipage}
\end{table}

\bigskip

\begin{table}[tbp]
\caption{SPF Forecast Pooling Analysis of Quarterly Nominal GDP Using Median
Benchmark Model and Median Expert Pool Predictions$^*$}{\centering}
\par
\hspace{0.3cm}
\par
\begin{center}
\begin{tabular}{cl|lllll}
\hline\hline
$Group$ & $Statistic$ & \multicolumn{5}{c}{$Forecast$ $Horizon$} \\ \hline
&  & $h=0$ & $h=1$ & $h=2$ & $h=3$ & $h=4$ \\ \hline\hline
Group 1 & $S_n^G$ & 0.000563 & 0.001063 & 0.001381 & 0.002021 & 0.001943 \\ 
& $S_n^C$ & 0.000006 & 0.000152 & 0.000556 & 0.001059 & 0.000194 \\ 
& $JCS_n^G$ & 0.310460 & 0.232845 & 0.232845 & 0.232845 & 0.388075 \\ 
& $JCS_n^C$ & 0.000178 & 0.000907 & 0.000989 & 0.002319 & 0.001603 \\ \hline
Group 2 & $S_n^G$ & 0.000715 & 0.002496 & 0.002127 & 0.002244 & 0.004061 \\ 
& $S_n^C$ & 0.000055 & 0.001409 & 0.001548 & 0.001152 & 0.001612 \\ 
& $JCS_n^G$ & 0.465690 & 0.698535 & 1.086611$^*$ & 0.931381 & 1.008996 \\ 
& $JCS_n^C$ & 0.000858 & 0.007005 & 0.009498 & 0.008086 & 0.012498 \\ \hline
Group 3 & $S_n^G$ & 0.001448 & 0.003776 & 0.004268 & 0.002254 & 0.010113$^*$
\\ 
& $S_n^C$ & 0.000318 & 0.001068 & 0.003558 & 0.001985 & 0.014572$^*$ \\ 
& $JCS_n^G$ & 0.620920 & 0.853766 & 0.853766 & 0.465690 & 1.008996 \\ 
& $JCS_n^C$ & 0.001480 & 0.005813 & 0.010052 & 0.007434 & 0.017275$^*$ \\ 
\hline
Group 4 & $S_n^G$ & 0.001730 & 0.004888 & 0.002593 & 0.002697 & 0.006291 \\ 
& $S_n^C$ & 0.000587 & 0.002044 & 0.002279 & 0.002268 & 0.006855 \\ 
& $JCS_n^G$ & 0.698535 & 0.853766 & 0.776151 & 0.776151 & 1.164226$^*$ \\ 
& $JCS_n^C$ & 0.002062 & 0.007503$^*$ & 0.007827 & 0.012697 & 0.019182$^*$
\\ \hline
Group 5 & $S_n^G$ & 0.001379 & 0.003470 & 0.004909 & 0.005554 & 0.009491 \\ 
& $S_n^C$ & 0.000534 & 0.001044 & 0.007032 & 0.009277$^*$ & 0.017460$^*$ \\ 
& $JCS_n^G$ & 0.388075 & 0.620920 & 0.776151 & 0.776151 & 0.776151 \\ 
& $JCS_n^C$ & 0.001413 & 0.005926 & 0.009500$^*$ & 0.013223$^*$ & 0.016788$%
^* $ \\ \hline
Group 6 & $S_n^G$ & 0.001767 & 0.005915 & 0.004387 & 0.005942 & 0.008262 \\ 
& $S_n^C$ & 0.000177 & 0.002169 & 0.006224 & 0.009615 & 0.020157$^*$ \\ 
& $JCS_n^G$ & 0.931381$^*$ & 0.853766 & 0.698535 & 0.853766 & 0.698535 \\ 
& $JCS_n^C$ & 0.001680 & 0.007416$^*$ & 0.009586 & 0.012696 & 0.018839 \\ 
\hline
Group 7 & $S_n^G$ & 0.002128 & 0.007057 & 0.008938 & 0.004600 & 0.009174 \\ 
& $S_n^C$ & 0.001056 & 0.002847 & 0.008537 & 0.005034 & 0.008895 \\ 
& $JCS_n^G$ & 0.465690 & 0.931381 & 1.008996 & 0.620920 & 1.086611$^*$ \\ 
& $JCS_n^C$ & 0.002062 & 0.007216$^*$ & 0.010561$^*$ & 0.007966 & 0.015189
\\ \hline
Group 8 & $S_n^G$ & 0.001997 & 0.003098 & 0.002059 & 0.004183 & 0.012612 \\ 
& $S_n^C$ & 0.000379 & 0.000686 & 0.001185 & 0.001744 & 0.010617 \\ 
& $JCS_n^G$ & 0.620920 & 0.620920 & 0.388075 & 0.698535 & 1.086611$^*$ \\ 
& $JCS_n^C$ & 0.001764 & 0.003088 & 0.003562 & 0.004157 & 0.016000$^*$ \\ 
\hline\hline
\end{tabular}%
\end{center}
\par
\begin{minipage}{1\columnwidth}

{\noindent $^{*}$ Notes: See notes to Table 3.}
\end{minipage}
\end{table}
\end{center}

\newpage

\begin{center}
\FRAME{}{5.3852in}{3.9496in}{0pt}{\Qcb{{}}}{}{Figure}{\special{language
"Scientific Word";type "GRAPHIC";display "USEDEF";valid_file "T";width
5.3852in;height 3.9496in;depth 0pt;original-width 20.01in;original-height
9.5622in;cropleft "0";croptop "1";cropright "1";cropbottom "0";tempfilename
'QH7SLI00.wmf';tempfile-properties "XPR";}}

Summary Statistics

\FRAME{}{3.9998in}{2.2321in}{0pt}{}{}{Figure}{\special{language "Scientific
Word";type "GRAPHIC";maintain-aspect-ratio TRUE;display "USEDEF";valid_file
"T";width 3.9998in;height 2.2321in;depth 0pt;original-width
4.3552in;original-height 2.4171in;cropleft "0";croptop "1";cropright
"1";cropbottom "0";tempfilename 'QH7SLI01.wmf';tempfile-properties "XPR";}}
\end{center}

* Notes: Figures are kernel density estimates (Epanechnikov kernel) for
various 4 quarter ahead predictions made during the fourth quarters of 1968,
1983, 1998, and 2013. The number of experts in each sample ranges from 31 in
1998 to 87 in 1968, as noted in table of summary statistics below the plots
in the figure. See Section 5 for further details.

\end{document}
