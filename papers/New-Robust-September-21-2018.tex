%2multibyte Version: 5.50.0.2960 CodePage: 936

\documentclass[final,notitlepage]{article}
%%%%%%%%%%%%%%%%%%%%%%%%%%%%%%%%%%%%%%%%%%%%%%%%%%%%%%%%%%%%%%%%%%%%%%%%%%%%%%%%%%%%%%%%%%%%%%%%%%%%%%%%%%%%%%%%%%%%%%%%%%%%%%%%%%%%%%%%%%%%%%%%%%%%%%%%%%%%%%%%%%%%%%%%%%%%%%%%%%%%%%%%%%%%%%%%%%%%%%%%%%%%%%%%%%%%%%%%%%%%%%%%%%%%%%%%%%%%%%%%%%%%%%%%%%%%
\usepackage{amsfonts}
\usepackage{amssymb}
\usepackage{graphicx}
\usepackage{amsmath}
\usepackage{latexsym}
\usepackage{caption}

\setcounter{MaxMatrixCols}{10}
%TCIDATA{OutputFilter=LATEX.DLL}
%TCIDATA{Version=5.50.0.2960}
%TCIDATA{Codepage=936}
%TCIDATA{<META NAME="SaveForMode" CONTENT="1">}
%TCIDATA{BibliographyScheme=Manual}
%TCIDATA{Created=Wed May 15 17:28:15 2002}
%TCIDATA{LastRevised=Friday, September 21, 2018 15:54:39}
%TCIDATA{<META NAME="GraphicsSave" CONTENT="32">}
%TCIDATA{<META NAME="DocumentShell" CONTENT="General\Blank Document">}
%TCIDATA{Language=American English}
%TCIDATA{CSTFile=LaTeX article (bright).cst}
%TCIDATA{PageSetup=65,65,72,72,0}
%TCIDATA{AllPages=
%H=36
%F=29,\PARA{038<p type="texpara" tag="Body Text" > \ \ \ \ \ \ \ \ \ \ \ \ \ \ \ \ \ \ \ \ \ \ \ \ \ \ \ \ \ \ \ \ \ \ \ \ \ \ \ \ \ \ \ \ \ \ \ \ \ \ \ \ \ \ \ \ \ \ \ \ \ \ \ \ \ \ \ \ \ \ \ \ \ \ \ \ \ \ \ \ \ \ \ \ \ \ \ \ \ \ \ \ \ \ \ \ \ \ \ \ \ \ \ \ \ \ \ \ \ \ \ \ \ \ \ \ \ \ \ \ \ \ \ \ \ \ \ \ \ \ \ \ \ \ \ \ \ \ \ \ \ \ \ \ \ \ \ \ \ \ \ \ \ \ \ \ \ \ \ \ \ \ \ \ \ \ \ \ \ \ \ \ \ \ \ \ \ \ \ \ \ \ \ \ \ \ \ \ \ \ \ \ \ \ \ \ \ \ \ \ \ \ \ \ \ \ \ \ \ \ \ \ \ \ \ \ \ \ \ \ \ \ \ \ \ \ \ \ \ \ \ \ \ \ \ \ \ \ \ \ \ \ \ \ \ \ \ \ \ \ \ \ \ \ \ \ \ \ \ \ \ \ \ \ \ \ \ \ \ \ \ \ \ \ \ \ \ \ \ \ \ \ \ \ \ \ \ \ \ \ \ \ \ \ \ \  }
%}


\newtheorem{theorem}{Theorem}[section]
\newtheorem{acknowledgement}[theorem]{Acknowledgement}
\newtheorem{algorithm}[theorem]{Algorithm}
\newtheorem{axiom}[theorem]{Axiom}
\newtheorem{case}[theorem]{Case}
\newtheorem{claim}[theorem]{Claim}
\newtheorem{conclusion}[theorem]{Conclusion}
\newtheorem{condition}[theorem]{Condition}
\newtheorem{conjecture}[theorem]{Conjecture}
\newtheorem{corollary}[theorem]{Corollary}
\newtheorem{criterion}[theorem]{Criterion}
\newtheorem{definition}[theorem]{Definition}
\newtheorem{example}[theorem]{Example}
\newtheorem{exercise}[theorem]{Exercise}
\newtheorem{lemma}[theorem]{Lemma}
\newtheorem{notation}[theorem]{Notation}
\newtheorem{problem}[theorem]{Problem}
\newtheorem{proposition}[theorem]{Proposition}
\newtheorem{remark}[theorem]{Remark}
\newtheorem{solution}[theorem]{Solution}
\newtheorem{summary}[theorem]{Summary}
\newenvironment{proof}[1][Proof]{\textbf{#1.} }{\ \rule{0.5em}{0.5em}}
\input{tcilatex}
\textwidth=16.0cm
\oddsidemargin=0cm \evensidemargin=0cm
\topmargin=-20pt
\numberwithin{equation}{section}
\baselineskip=100pt
\textheight=21cm
\def\baselinestretch{1.2}
\begin{document}

\title{}

\begin{center}
\bigskip

{\LARGE Robust Forecast Superiority Testing with an Application to Assessing
Pools of Expert Forecasters}

\bigskip \bigskip \bigskip

{\Large Valentina Corradi}$^{1}${\Large , Sainan Jin}$^{2},$ {\Large and
Norman R. Swanson}$^{3}\medskip $

$^{1}$University of Surrey, $^{2}$Singapore Management University, and $^{3}$%
Rutgers University

\bigskip

September 2018

\bigskip

Abstract
\end{center}

\noindent {\small Jin, Corradi and Swanson (JCS: 2017) develop a forecast
superiority testing methodology which is robust to the choice of loss
function. They do this by establishing a mapping between generic loss
evaluation and stochastic dominance principles. However, the tests that they
develop are not uniformly valid, and have correct asymptotic size only under
the least favorable case. Since tests for stochastic dominance can be seen
as tests for infinitely many moment inequalities, we use tools from Andrews
and Shi (2013,2017) to develop new tests for robust forecast comparison
which are uniformly asymptotically valid and asymptotically
non-conservative. The extant (many) moment inequality results that we
utilize to this end are valid for }$iid${\small \ observations. However,
forecast errors in our set-up may be non martingale difference sequences,
because of dynamic misspecification. We thus establish uniform convergence
(over error support) of HAC variance estimators and of their bootstrap
counterparts. Furthermore, we extend the asymptotic validity of generalized
moment selection tests to the case of non-vanishing recursive parameter
estimation error. The suggested testing methodology is evaluated in a series
of Monte Carlo experiments, and is used to analyze the Survey of
Professional Forecasters (SPF). Our Monte Carlo experiments indicate
improvements in finite sample performance, relative to the tests described
in JCS (2017). Our empirical findings indicate that experience and forecast
quality matters in the SPF. Namely, combining predictions from pools of
expert forecasters chosen according to recent forecast \textquotedblleft
quality\textquotedblright\ yields improvement relative to mean or median
consensus forecast. On the other hand, forecast combinations from expert
pools chosen solely according to experience do not outperform combinations
that utilize predictions from the entire expert pool. }

{\small \bigskip \bigskip \bigskip }

\noindent \textit{JEL Classification}: C12, C22, C53.

\noindent \textit{Keywords}: Robust Forecast Evaluation, Many Moment
Inequalities, Bootstrap, Estimation Error, Combination Forecasts, Survey of
Professional Forecasters.

\noindent \_\_\_\_\_\_\_\_\_\_\_\_\_\_\_\_\_\_\_\_\_\_\_\_\_

\noindent {\tiny Valentina Corradi, School of Economics, University of
Surrey, Guildford, Surrey, GU2 7XH, UK, v.corradi@surrey.ac.uk; Jin Sainan,
School of \ Economics, Singapore Management University, 90 Stamford Road,
\#05-30, Singapore 178903, snjin@smu.edu.sg; and Norman R. Swanson,
Department of Economics, Rutgers University, 75 Hamilton Street, New
Brunswick, NJ 08901, USA, nswanson@econ.rutgers.edu. We are grateful to
Kevin Lee, Patrick Marsh, Luis Martins, Jams Mitchell, Alessia Paccagini,
Paulo Parente, Ivan Petrella, Valerio Poti, Barbara Rossi, Simon Van Norden,
Claudio Zoli, and to the participants at the 2018 NBER-NSF Times Series
Conference, the 2016 European Meeting of the Econometric Society, Conference
for 50 years of Keynes College at Kent University, and seminars at Mannheim
University, the University of Nottingham, University College Dublin,
Instituto Universit\'{a}rio de Lisboa, Universita' di Verona and the Warwick
Business School for useful comments and suggestions. Additionally, many
thanks are owed to Mingmian Cheng for excellent research assistance.}

{\small \newpage }

\section{Introduction}

Forecast accuracy is typically measured in terms of a given loss function.
This approach has three main drawbacks. First, it is well known that the
relative ranking of forecasts from misspecified models is loss function
dependent. Second, sequences of forecasts evaluated with the same synthetic
measure of accuracy, such as the mean square forecast error (MSFE) or mean
absolute forecast error (MAFE), can be characterized by very different error
distributions. Third, expected loss functions are very sensitive to
outliers. This is particularly true when the loss function is convex (e.g.,
quadratic or linear-exponential (Linex) loss functions). Corradi and Swanson
(2013) address the second and third of these limitations by introducing an
alternative criterion for predictive evaluation which measures accuracy via
examination of the quantiles of expected loss distributions. However, their
criterion is still loss function specific. In order to develop a loss
function free criterion, one must evaluate the distribution of raw errors.
Diebold and Shin (2015, 2017) make important contributions in this direction
by introducing so-called stochastic error distance (SED) measures, in which
Cram\'{e}r-von Mises type statistics are constructed using raw errors. In
their framework, in which they equate SED with mean absolute error loss, one
chooses the model for which the cumulative error distribution is closest to
a step function which is equal to zero over the negative real line and is
equal to one over the positive real line. Jin, Corradi and Swanson (JSC:
2017) make further advances in this area by developing a forecast
superiority testing methodology which is robust to the choice of loss
function. They do this by establishing a mapping between generic loss
function forecast evaluation and stochastic dominance principles. Hence,
they test forecast superiority via stochastic dominance tests. However, the
suggested tests are not uniformly valid and have correct size only under the
least favorable case. In the current paper, we improve on JSC (2017) by
developing loss function \textquotedblleft free\textquotedblright\ forecast
superiority tests which are uniformly asymptotically valid and
asymptotically non conservative. This is done in part by noting that tests
for stochastic dominance can be seen as tests for infinitely many moment
inequalities, hence allowing us to utilize tools recently developed by
Andrews and Shi (2013, 2017) to derive asymptotically uniformly valid and
non conservative forecast superiority tests.

The implementation of such tests require that sample moments are
standardized by an estimator of the standard deviation. Now, forecast errors
are typically non martingale difference sequences, either because they are
based on dynamically misspecified models or because forecasters do not
efficiently use all the available information, in the case of subjective
predictions. Hence, we require heteroskedasticity and autocorrelation (HAC)
robust variance estimators. In our set-up, each such variance estimator
depends on a specific point in the forecasting error support. Our first
methodological contribution is to establish the consistency of HAC variance
estimators uniformly over the error support. We also establish uniform
convergence of their bootstrap counterparts. This is important because of
the presence of the lag truncation parameter, in which case uniform
convergence of HAC estimators and of their bootstrap analogs does not follow
straightforwardly from uniform convergence of (kernel) nonparametric
estimators. To the best of our knowledge this contribution is a novel
addition to the vast literature on HAC covariance matrix estimation. When
forecasts are based on estimated models, one also has to take into account
the contribution of parameter estimation error to the limiting distribution.
As a second methodological contribution, we develop uniformly valid and
asymptotically non conservative forecast superiority tests for the case of
non vanishing parameter estimation error, under recursive estimation scheme.
This is accomplished by extending the recursive block bootstrap introduced
in Corradi and Swanson (2007).

The forecast superiority testing methodology discussed in this paper is
evaluated via a series of Monte Carlo experiments, and via a detailed
empirical analysis of the Survey of Professional Forecasters (SPF). Our
Monte Carlo experiments indicate clear improvement in the finite sample
power associated with using the tests that we introduce, when compared with
those introduced in JCS(2017). In particular, the new tests have much higher
power against alternatives in which some model beats the benchmark but other
are strictly dominated, even in sample of only $250$ observations. Our
empirical analysis builds on a very large literature studying the SPF, in
which many papers find (loss function specific) evidence of the usefulness
of forecast combination. For example, Zarnowitz and Braun (1992) find that
using the mean or median provides a consensus forecast with lower average
errors than most individual forecasts. More recently, Aiolfi, Capistr\'{a}n,
and Timmermann (2011) and Genre, Kenny, Meyler, and Timmermann (2013) find
that equal weighted averages of SPF and ECB SPF forecasts outperform model
based forecasts, although in some cases there is an improvement by averaging
them with (mean of) model-based forecasts. Using our new loss function free
methodology, we are able to uncover distribution based evidence that
forecast averages from small pools of survey participants ranked according
to recent absolute and mean square forecast error performance are preferred
to forecast averages based on the entire pool of experts. For example, for
forecasting U.S. GDP\ growth, expert pools consisting of the
\textquotedblleft top 25\%\textquotedblright\ of forecasters yield loss
function free dominating combination predictions, when forecasters are
required to have at least 1 year of experience. In our analysis, however,
solely organizing expert pools based on experience is not enough. Instead,
it is crucial that pools of experts also be chosen based on the quality of
their predictions. Indeed, only requiring either 1, 3, or 5 years of
experience does not yield a pool of experts whose forecast combinations are
superior to equal weighted averages from the entire pool of experts, when
using our loss function robust methods.

The rest of the paper is organized as follows. Section 2 outlines the set-up
and introduces our new tests. Section 3 establishes the asymptotic
properties of the tests in the context of generalized moment selection.
Section 4 establishes the asymptotic properties of the tests in the context
of non-vanishing parameter estimation error, for the recursive estimation
schemes. Section 5 contains the results of our Monte Carlo experiments, and
Section 6 contains the results of our analysis of GDP growth forecasts from
the SPF. Finally, Section 7 provides a number of concluding remarks. Proofs
are gathered in an appendix.

\section{Set-Up}

\subsection{GL and CL Forecast Superiority Tests}

\noindent Let $e_{j,t}$ be a forecast error, and let there be $j=1,...,k$
such errors at each point in time, $t=1,...,n$, corresponding to $k$
different forecast models or judgmental forecasts, for example. We begin by
ignoring estimation error, such as when forecasts are judgmental or
subjective. Surveys including the SPF are leading examples of judgmental
forecasts. The case where predictions are based on estimated models is
considered in Section 4. Hereafter, sequence $e_{1,t},t=1,...,n$ is called
the \textquotedblleft benchmark\textquotedblright . In the context of the
SPF, an example of a relevant benchmark against which to compare all other
sequences is the consensus forecast constructed as the simple arithmetic
average of individual forecasts in the survey. Our goal is to test whether
there exists some competing forecast that is superior to the benchmark for
any loss function, $L$, satisfying Assumption A0.

\noindent \textbf{Assumption A0 }(i)\textbf{\ }$L\in \mathcal{L}_{G}$ if $L:%
\mathbb{R\rightarrow R}^{+}$ is continuously differentiable, except for
finitely many points, with derivative $L^{\prime },$ such that $L^{\prime
}(z)\leq 0,$ for all $z\leq 0,$ and $L^{\prime }(z)\geq 0,$ for all $z\geq
0. $ (ii) $L\in \mathcal{L}_{C}$ is a convex function belonging to $\mathcal{%
L}_{G}.\medskip $

Hereafter, let $F_{j}(x)$ denote the cumulative distribution function (CDF)
of forecast error $e_{j}.$ JCS (2017) establish the following two
results.\medskip

\noindent \textit{1. For any }$L\in L_{G},$\textit{\ }$E(L(e_{1}))\leq
E(L(e_{2})),$\textit{\ if and only if }$(F_{2}(x)-F_{1}(x))sgn(x)\leq 0,$%
\textit{\ for all }$x\in \mathcal{X}.$\textit{\ \medskip }

\noindent \textit{2. For any }$L\in L_{C},$\textit{\ }$E(L(e_{1}))\leq
E(L(e_{2})),$\textit{\ if and only if}

\textit{\noindent }$\left( \int_{-\infty
}^{x}(F_{1}(t)-F_{2}(t))dt1(x<0)+\int_{x}^{\infty
}(F_{2}(t)-F_{1}(t))dt1(x\geq 0)\right) \leq 0,$\textit{\ for all }$x\in 
\mathcal{X}.$\textit{\ \medskip }

The first statement establishes a mapping between GL forecast superiority
and first order stochastic dominance (FOSD). In particular, $e_{1}$ is not
GL\ dominated by $e_{2}$ if $F_{1}(x)$ lies below $F_{2}(x)$ on the negative
real line, and lies above $F_{2}(x)$ on the positive real line. Indeed, this
ensure we choose the forecast whose CDF has larger mass around zero.
Likewise, the second statement establishes a mapping between CL superiority
and second order stochastic dominance.

In this framework, it follows that testing for loss function robust forecast
superiority involves testing: 
\begin{equation}
H_{0}:\max_{j=2,...,k}\left( E(L(e_{1}))-E(L(e_{k}))\right) \leq 0
\label{H0}
\end{equation}%
versus%
\begin{equation}
H_{A}:\max_{j=2,...,k}\left( E(L(e_{1}))-E(L(e_{k}))\right) >0.  \label{HA}
\end{equation}%
Consider $L\in \mathcal{L}_{G}.$ Then, these hypotheses can be restated as
follows:%
\begin{eqnarray*}
H_{0}^{G} &=&H_{0}^{G-}\cap H_{0}^{G+} \\
&:&\left( \max_{j=2,...,k}\left( F_{1}(x)-F_{j}(x))\right) \leq 0,\text{ for 
}x\leq 0\right) \\
&&\cap \left( \max_{j=2,...,k}\left( F_{j}(x)-F_{1}(x))\right) \leq 0,\text{
for }x>0\right)
\end{eqnarray*}%
versus%
\begin{eqnarray*}
H_{A}^{G} &=&H_{A}^{G-}\cup H_{A}^{G+} \\
&:&\left( \max_{j=2,...,k}\left( F_{1}(x)-F_{j}(x))\right) >0,\text{ for
some }x\leq 0\right) \\
&&\cup \left( \max_{j=2,...,k}\left( F_{j}(x)-F_{1}(x))\right) >0,\text{ for
some }x>0\right) .
\end{eqnarray*}%
Note that the null hypothesis is the intersection of two different null
hypotheses because of a discontinuity at zero. Similarly, for the case of $%
L\in \mathcal{L}_{C,}$ $H_{0}$ and $H_{A}$ can be restated as:%
\begin{eqnarray*}
H_{0}^{C} &=&H_{0}^{C-}\cap H_{0}^{C+} \\
&:&\left( \max_{j=2,...,k}\int_{-\infty }^{x}(F_{1}(t)-F_{j}(t))dt\leq 0,%
\text{ for }x\leq 0\right) \\
&&\cap \left( \max_{j=2,...,k}\int_{x}^{\infty }(F_{j}(t)-F_{1}(t))dt\leq 0,%
\text{ for }x>0\right)
\end{eqnarray*}%
versus%
\begin{eqnarray*}
H_{A}^{C} &=&H_{A}^{C-}\cup H_{A}^{C+} \\
&:&\left( \max_{j=2,...,k}\int_{-\infty }^{x}(F_{1}(t)-F_{k}(t))dt>0,\text{
for some }x\leq 0\right) \\
&&\cup \left( \max_{j=2,...,k}\int_{x}^{\infty }(F_{j}(t)-F_{1}(t))dt>0,%
\text{ for some }x>0\right) .
\end{eqnarray*}%
In order to test $H_{0}^{G^{+}}$ against $H_{A}^{G^{+}},$ and $H_{0}^{C+}$
against $H_{A}^{C^{+}}$JCS (2017) utilize the following statistics:%
\begin{equation}
\sqrt{n}G_{j,n}^{+}(x)=\sqrt{n}\left( \widehat{F}_{j,n}(x)-\widehat{F}%
_{1,n}(x)\right) ,  \label{Gn}
\end{equation}%
where $\widehat{F}_{k,n}(x)$ denotes the empirical CDF of $e_{k};$ and%
\begin{eqnarray}
\sqrt{n}C_{j,n}^{+}(x) &=&\sqrt{n}\int_{x}^{\infty }\left( \widehat{F}%
_{j,n}(t)-\widehat{F}_{1,n}(t)\right) dt  \notag \\
&=&\frac{1}{\sqrt{n}}\sum_{t=1}^{n}\left( \left[ \left( e_{1,t}-x\right) %
\right] _{+}-\left[ \left( e_{j,t}-x\right) \right] _{+}\right) ,  \label{Cn}
\end{eqnarray}%
where $[y]_{+}=\max \{0,y\},$ and where the equality in the preceding
expression follows from integration by parts. Now, $G_{j,n}^{+}$ and $%
C_{j,n}^{+}$ are similar to the test statistics discussed in LMW (2005).
Also, as stated in Theorem 3 of JCS (2017),%
\begin{eqnarray}
\max_{j=2,...,k}\sup_{x\in \mathcal{X}^{+}}\sqrt{n}G_{j,n}^{+} &\Rightarrow &%
\underset{j=2,..,k}{\max }\underset{\text{ }x\in \mathcal{B}_{j}^{g+}}{\sup }%
g_{j}(x),\text{ if }\max_{j=2,...,k}F_{j}(x)-F_{1}(x))=0  \notag \\
&\Rightarrow &-\infty \text{ }\ \ \ \ \ \ \ \ \text{if }%
\max_{j=2,...,k}F_{j}(x)-F_{1}(x))<0,  \label{thm3}
\end{eqnarray}%
where $g=(g_{2},...,g_{k})$ denotes a $k-1$ dimensional zero mean Gaussian
process, and $\mathcal{B}_{j}^{g+}=\{x\in \mathcal{X}^{+}$ : $%
F_{1}(x)=F_{j}(x)\}.$ JCS (2017) also suggest to evaluate $H_{0}^{G^{+}}$
using the percentiles of the empirical distribution of: \footnote{%
i.e., $H_{0}^{G}$ is (not) rejected at level $\alpha ,$ if the smallest
bootstrap p-values associated to $H_{0}^{G-}$ and $H_{0}^{G+}$ is smaller
(larger) than $\alpha /2.$}%
\begin{equation*}
\max_{j=2,...,k}\sup_{x\in \mathcal{X}^{+}}\sqrt{n}\left( \left( \widehat{F}%
_{j,n}^{\ast }(x)-\widehat{F}_{j,n}(x)\right) -\left( \widehat{F}%
_{1,n}^{\ast }(x)-\widehat{F}_{1,n}(x)\right) \right) .
\end{equation*}%
Hence, their tests have asymptotically correct size only under the least
favorable case under the null. Namely, when $F_{1}(x)=F_{j}(x),$ for all $j,$
and for all $x\in \mathcal{X}^{+}.$ In this sense, their test is not
asymptotically similar on the boundary, $\mathcal{B}_{0}^{+}=\left\{
\max_{k=2,...,j}\left( F_{j}(x)-F_{1}(x)\right) =0,\text{ for some }x\in 
\mathcal{X}^{+}\right\} ;$ and furthermore is asymptotically biased towards
certain local alternatives.\footnote{%
Hansen (2005) shows that $p$-values associated with the use of stationary
bootstrap tests are actually upper bounds for an asymptotically unbiased
test.} On the other hand, as pointed out by LMW (2005), a test based on
subsampled critical values is asymptotically similar on the boundary, $%
\mathcal{B}_{0}^{+}.$ This is because the subsampling distribution mimics
the sampling distribution. A drawback is that tests which are similar on the
boundary may have very little power against certain sequence of
alternatives, where $F_{k}(x)-F_{1}(x)>0,$ for $x\in \mathcal{X}_{A}\subset 
\mathcal{X}^{+};$ and where $F_{k}(x)-F_{1}(x)<0,$ for $x\in \mathcal{X}%
^{+}\backslash \mathcal{X}_{A}.$ Donald and Hsu (2016) provide an
interesting example of this issue for the case of $k=2$. In addition, for
the case of a finite number of moment inequalities, Andrews (2012) shows
that tests which are similar on the boundary have trivial power against
certain alternatives. Moreover, the weak convergence of the JCS (2017)
statistic, and of its bootstrap counterpart, is \textquotedblleft
only\textquotedblright\ pointwise. This is also true under subsampling.
Hence, inference based on subsampling or on centered bootstrap may be not
asymptotically valid, uniformly, over all probabilities under the null
hypothesis. Lack of uniformity is typical of tests based on weak
inequalities (see Mikusheva (2007) and Andrews and Guggenberger (2010)). In
light of this, and for the case of finitely many moment inequalities,
Andrews and Soares (2010) and Andrews and Barwick (2013) introduce bootstrap
tests which ensure that the asymptotic size (coverage) is at most (at least) 
$\alpha $\ $(1-\alpha ),$\ uniformly, over all probabilities under the null
hypothesis.

\subsection{Improved Forecast Superiority Tests}

It is immediate to see that we can restate $H_{0}^{G}$ and $H_{0}^{C}$ in
terms of infinitely many moment inequalities. Hereafter, let $\mathcal{X}=%
\mathcal{X}^{-}\cup \mathcal{X}^{+}$ be the union of the support of $%
(e_{1},...,e_{k}).$Then,%
\begin{eqnarray*}
H_{0}^{G} &=&H_{0}^{G-}\cap H_{0}^{G+} \\
&:&\left( F_{1}(x)-F_{j}(x)\leq 0,\text{ for }j=2,...,k,\text{ and for all }%
x\in \mathcal{X}^{-}\right) \\
&&\cap \left( F_{j}(x)-F_{1}(x)\leq 0,\text{ for }j=2,...,k,\text{ and for
all }x\in \mathcal{X}^{+}\right)
\end{eqnarray*}%
versus%
\begin{eqnarray*}
H_{A}^{G} &=&H_{A}^{G-}\cup H_{A}^{G+} \\
&:&\left( F_{1}(x)-F_{j}(x)>0,\text{ for some }j=2,...,k,\text{ and for some 
}x\in \mathcal{X}^{-}\right) \\
&&\cup \left( F_{j}(x)-F_{1}(x)>0,\text{ for some }j=2,...,k,\text{ and for
some }x\in \mathcal{X}^{+}\right) .
\end{eqnarray*}%
Analogously,%
\begin{eqnarray*}
H_{0}^{C} &=&H_{0}^{C-}\cap H_{0}^{C+} \\
&:&\left( \int_{-\infty }^{x}(F_{1}(t)-F_{j}(t))dt\leq 0,\text{ for }%
j=2,...,k\text{, and for all }x\in \mathcal{X}^{-}\right) \\
&&\cap \left( \int_{x}^{\infty }(F_{j}(t)-F_{1}(t))dt\leq 0,\text{ for }%
j=2,...,k,\text{ and for all }x\in \mathcal{X}^{+}\right)
\end{eqnarray*}%
versus%
\begin{eqnarray*}
H_{A}^{C} &=&H_{A}^{C-}\cup H_{A}^{C+} \\
&:&\left( \int_{-\infty }^{x}(F_{1}(t)-F_{k}(t))dt>0,\text{ for some }%
j=2,...,k,\text{ and for some }x\in \mathcal{X}^{-}\right) \\
&&\cup \left( \max_{j=2,...,k}\int_{x}^{\infty }(F_{j}(t)-F_{1}(t))dt>0,%
\text{ for some }j=2,...,k,\text{ and for some }x\in \mathcal{X}^{+}\right) .
\end{eqnarray*}%
Evidently, $H_{0}^{G}$ and $H_{0}^{C}$ can be written as the intersection of 
$(k-1)$ moment inequalities, which have to hold uniformly over $\mathcal{X}.$
This gives rise to an infinite number of moment conditions. Andrews and Shi
(2013) develop tests for conditional moment inequalities, and as is well
known in the literature on consistent specification testing (e.g., see
Bierens (1982, 1990)) a finite number of conditional moments can be
transformed into an infinite number of unconditional moments. The same is
true in the case of weak inequalities. Andrews and Shi (2017) consider tests
for conditional stochastic dominance, which are then characterized by an
infinite number of conditional moment inequalities and so by a
\textquotedblleft twice\textquotedblright\ infinite number of unconditional
inequalities. Recalling that our interest is on testing GL or CL forecast
superiority as in (\ref{H0}) and (\ref{HA}), we confine our attention to
unconditional testing of stochastic dominance.

Because of the discontinuity around zero in the tests discussed in JCS\
(2017), \ they separately test $H_{0}^{G+}\left( H_{0}^{C^{+}}\right) $ and $%
H_{0}^{G-}(H_{0}^{C-}),$ and then use the methods of Holm (1979) to control
the two resulting p-values (see Rules TG and TC in JCS (2017)). The same
approach is taken in this paper. Hence, without loss of generality, we focus
our discussion in the sequel on testing $H_{0}^{G+}$ versus $H_{A}^{G+}$ and 
$H_{0}^{C+}$ versus $H_{A}^{C+}.$ Testing $H_{0}^{G-}$ versus $H_{A}^{G-}$
and $H_{0}^{C-}$ versus $H_{A}^{C-}$ follows immediately.

We begin by testing GL forecast superiority. Let $G_{n}^{+}=\left(
G_{2,n}^{+},...,G_{k,n}^{+}\right) ,$

\begin{equation}
\Sigma ^{G+}\left( x,x^{\prime }\right) =\mathrm{acov}\left( \sqrt{n}%
G_{n}^{+}(x),\sqrt{n}G_{n}^{+}(x^{\prime })\right)  \label{SIGMA}
\end{equation}%
and%
\begin{equation}
\overline{\Sigma }_{n}^{G+}\left( x,x^{\prime }\right) =\widehat{\Sigma }%
_{n}^{G+}\left( x,x^{\prime }\right) +\varepsilon I_{k-1},  \label{SIGMA-bar}
\end{equation}%
where $\varepsilon \geq 0$, and where $\widehat{\Sigma }_{n}^{G+}\left(
x,x^{\prime }\right) $ is the sample analog of $\Sigma ^{G+}\left(
x,x^{\prime }\right) .$ Let $\widehat{u}_{j,t}(x)=1\left\{ e_{j,t}\leq
x\right\} -\frac{1}{n}\sum_{t=1}^{n}1\left\{ e_{j,t}\leq x\right\} $, so
that the $jj-$th element of $\widehat{\Sigma }_{n}^{G+}\left( x,x\right) $
is given by%
\begin{eqnarray}
\widehat{\sigma }_{jj,n}^{2,G+}(x) &=&\frac{1}{n}\sum_{t=1}^{n}(\widehat{u}%
_{j,t}(x)-\widehat{u}_{1,t}(x))^{2}  \notag \\
&&+2\frac{1}{n}\sum_{\tau =1}^{l_{n}}\sum_{t=\tau +1}^{n}w_{\tau }(\widehat{u%
}_{j,t}(x)-\widehat{u}_{1,t}(x))(\widehat{u}_{j,t-\tau }(x)-\widehat{u}%
_{1,t-\tau }(x)),  \label{HAC}
\end{eqnarray}%
where $w_{\tau }=1-\frac{\tau }{1+l_{n}},$ with $l_{n}\rightarrow \infty $
as $n\rightarrow \infty .$\ In (\ref{SIGMA-bar}), the role of the additional 
$\varepsilon I_{k-1}$ term is to correct for the possible singularity of the
covariance estimator, for certain values of $x.$ This is the case when we
compare forecast errors from nested models. Also, let $\overline{\sigma }%
_{jj,n}^{2,G^{+}}(x,x^{\prime })$ be the $jj-$th element of $\overline{%
\Sigma }_{n}^{G+}\left( x,x^{\prime }\right) ,$ and let $\overline{\sigma }%
_{jj,n}^{2,C+}(x,x^{\prime })$ be the $jj-$th element of $\overline{\Sigma }%
_{n}^{C+}\left( x,x^{\prime }\right) .$ Construct the following test
statistics:%
\begin{equation}
S_{n}^{G+}=\int_{\mathcal{X}^{+}}\sum_{j=2}^{k}\left( \max \left\{ 0,\frac{%
\sqrt{n}G_{j,n}^{+}(x)}{\overline{\sigma }_{jj,n}^{G+}(x)}\right\} \right)
^{2}\mathrm{d}Q(x).  \label{SnG}
\end{equation}%
and%
\begin{equation}
S_{n}^{C+}=\int_{\mathcal{X}^{+}}\sum_{j=2}^{k}\left( \max \left\{ 0,\frac{%
\sqrt{n}C_{j,n}^{+}(x)}{\overline{\sigma }_{jj,n}^{C+}(x)}\right\} \right)
^{2}\mathrm{d}Q(x),  \label{SnC}
\end{equation}%
where $Q$ is a weighting function defined below; $G_{j,n}^{+}(x)$ and $%
C_{j,n}^{+}(x)$ are the $j-$th components of $G_{n}^{+}(x)$ and $%
C_{n}^{+}(x),$ as defined in (\ref{Gn}) and (\ref{Cn}), respectively; and $%
\overline{\sigma }_{jj,n}^{C+}(x)$ is constructed in analogous manner to $%
\overline{\sigma }_{jj,n}^{G+}(x),$ by replacing $\widehat{u}_{j,t}(x)$ in (%
\ref{HAC}) with $\widehat{\eta }_{j,t}(x)=\left[ e_{j,t}-x\right] _{+}-\frac{%
1}{n}\sum_{t=1}^{n}\left[ e_{j,t}-x\right] _{+}.$

$S_{n}^{G+}$ and $S_{n}^{C+}$ are \textquotedblleft sum\textquotedblright\
functions, as in equation (3.8) in Andrews and Shi (2013), and satisfy their
Assumptions S1-S4, which are required to guarantee that convergence is
uniform over the null DGPs.\footnote{%
Note that we could have constructed a different "sum" function, using e.g.
the statistic in (3.9) in Andrews and Shi (2013).}\footnote{%
Recall that one main drawback of the $\max_{j=2,...,k}\sup_{x\in \mathcal{X}%
^{+}}\sqrt{n}G_{n}^{+}$ statistic in JCS (2017) is that it diverges to $%
-\infty $ under some sequence of probability measures under the null, thus
ruling out uniformity.} If $k=2$ and $\overline{\sigma }_{jj,n}(x)=1,$ for
all $j$ and $x$ (i.e. no standardization), then $S_{n}^{G+}$ is the
statistic used in Linton, Song and Whang (2010) for testing FOSD.

Of note is that in our context, potential slackness causes a discontinuity
in the pointwise asymptotic distribution of the statistic.\footnote{%
By pointwise asymptotic distribution we mean the limiting distribution under
a fixed probability measure.} This is because the pointwise asymptotic
distribution is discontinuous, unless all moment conditions hold with
equality. On the other hand, the finite sample distribution is not
necessarily discontinuous. Thus, in the presence of slackness, the pointwise
limiting distribution is not a good approximation of the finite sample
distribution, and critical values based on pointwise asymptotics may be
invalid. This is why we construct tests that are uniformly asymptotically
valid (i.e., this is why we study the limiting distribution of our tests
under drifting sequences of probability measures belonging to the null
hypothesis). Moreover, in the infinite dimensional case, there is an
additional source of discontinuity. In particular, the number of moment
inequalities which contributes to the statistic varies across the different
values of $x.$ For example, the key difference between the case of $k=2$ and 
$k>2$ is that in the former case, for each value of $x$ there is only one
moment inequality which can be binding (or not). On the other hand, if $k=3$%
, say, then for each value of $x$ there can be either one or two moment
inequalities which may be binding (or not), and whether or not a particular
inequality is binding (or not) varies over $x$. Under this setup, we require
the following assumptions in order to analyze the asymptotic behavior of our
test statistics.

\noindent \textbf{Assumption A1}: For $j=1,...,k,$ $e_{t,j}$ is strictly
stationary and $\beta -$mixing, with mixing coefficients, $a_{m}=m^{-\beta
}, $ where $\beta >\frac{6\delta }{1-2\delta },$ $0<\delta <1/2$ and $\beta
\delta >1.$

\noindent \textbf{Assumption A2}: The union of the supports of $%
e_{1},..,e_{k}$ is the compact set, $\mathcal{X}=\mathcal{X}^{-}\cup 
\mathcal{X}^{+}$.

\noindent \textbf{Assumption A3:} $F_{j}(x)$ has a continuos bounded density.

\noindent \textbf{Assumption A4}: The weighting function $Q$ has full
support $\mathcal{X}^{+}.$

\section{Asymptotic Properties}

\subsection{\noindent Uniform Convergence of the HAC Estimator}

We now turn to a discussion of the estimation of the variance in our
forecast superiority test statistics. If $e_{1,t},...,e_{k,t}$ were
martingale difference sequences, then we can still use the sample second
moment as a variance estimator, and uniform consistency will follow by
application of an appropriate uniform law of large numbers. In our set-up we
can assume that $e_{1},...,e_{k}$ are martingale difference sequences if
either: (i) they are judgmental forecasts from professional forecasters,
say, who efficiently use all available information at time $t$ (a strong
assumption, which is tested in the forecast rationality literature)$;$ or
(ii) they are prediction errors from one-step ahead forecasts based on
dynamically correctly specified models. With respect to (i), it is worth
noting that professional forecasters may be rational, ex-post, according to
some loss function (see Elliott, Komunjer and Timmermann (2005,2008),
although it is not as likely that they are rational according to a
generalized loss function. With respect to (ii), it should be noted that at
most one model can be dynamically correctly specified for a given
information set, and thus $e_{j}$ cannot be a martingale difference
sequence, for all $j=1,...,k.$ In light of these facts, we allow for time
dependence in the forecast error sequences used in our statistics, and use a
HAC variance estimator in (\ref{SnG}) and (\ref{SnC}). In order to ensure
that the HAC estimators converge uniformly over $\mathcal{X}^{+},$ it
suffices to establish the counterpart of Lemma A1 of Supplement A of Andrews
and Shi (2013) to the case of mixing sequences. This is done below.

\noindent \textbf{Lemma 1}\textit{: Let Assumptions A1-A3 hold. Then, if }$%
l_{n}\approx n^{\delta }$\textit{\ }$0<\delta <\frac{1}{2},$\textit{\ with }$%
\delta $\textit{\ defined as in Assumption A1:}

\textit{\noindent (i)}%
\begin{equation*}
\sup_{x\in \mathcal{X}^{+}}\left\vert \widehat{\sigma }_{jj,n}^{2,G+}(x)-%
\sigma _{jj}^{2,G+}(x)\right\vert =o_{p}\left( 1\right) ,
\end{equation*}%
\textit{with }$\sigma _{jj}^{2,G+}(x)=avar\left( \sqrt{n}G_{j,n}^{+}(x)%
\right) ;$\textit{\ and}

\textit{\noindent (ii)}%
\begin{equation*}
\sup_{x\in \mathcal{X}^{+}}\left\vert \widehat{\sigma }_{jj,n}^{2,C+}(x)-%
\sigma _{jj}^{2,C+}(x)\right\vert =o_{p}\left( 1\right) ,
\end{equation*}%
\textit{with }$\sigma _{jj}^{2C+}(x)=avar\left( \sqrt{n}C_{j,n}^{+}(x)%
\right) .$

Lemma 1 establishes the uniform convergence over $\mathcal{X}^{+}$ of HAC
estimators. It is the time series counterpart of Lemma A1 in Andrews and Shi
(2013). Of note is that we require $\beta -$mixing. This differs from the
stationary pointwise HAC variance estimator case studied by Andrews (1991),
where $\alpha -$mixing is required, and where the mixing coefficients
declines to zero slightly slower than in our Assumption 1. This is because
there is a trade-off between the degree of dependence and the rate of growth
of the lag truncation parameter in the HAC estimator. Indeed, in the uniform
case, the covering number (e.g., see Andrews and Pollard (1994)) grows with
both $l_{n}$ and the degree of dependence, thus leading to a trade-off
between the two. For example, in the case of exponential mixing series, $%
\delta $ can be arbitrarily close to $1/2.$

In the proof of Lemma 1, we require $\mathcal{X}^{+}$ in (\ref{SnG}) and (%
\ref{SnC}) to be a compact set. However, for the case of generalized loss
superiority, the union of the supports of $e_{1},..,e_{k}$ can be unbounded.
This is because $S_{n}^{G+}$ is bounded, regardless of the boundedness of
the support. On the other hand, $S_{n}^{C^{+}}$ is bounded only when the
union of the support of the forecasting error is bounded.

For carrying out inference on our forecast superiority tests, we require a
bootstrap analog of the HAC variance estimator, which can be constructed as
follows. Using the block bootstrap, make $b_{n}$ draws of length $l_{n}$
from $e_{1,t},...,e_{k,t},$ in order to obtain $\left( e_{j,1}^{\ast
},...,e_{j,n}^{\ast }\right) =\left( e_{j,I_{1+1}},...,e_{j,I_{1+l}}^{\ast
},...,e_{j,I_{b}+l}^{\ast }\right) ,$ with $b_{n}l_{n}=n,$ where the block
size, $l_{n},$ is equal to the lag truncation parameter in the HAC estimator
described above.\footnote{%
We thus use the same notation, $l_{n},$ for both the lag truncation
parameter and the block length.} Now, let $u_{j,t}^{\ast }(x)=1\left\{
e_{j,t}^{\ast }\leq x\right\} -1\left\{ e_{j,t}\leq x\right\} ,$ and%
\begin{equation}
\widehat{\sigma }_{jj,n}^{2\ast G+}(x)=\frac{1}{b_{n}}\sum_{k=1}^{b_{n}}%
\left( \frac{1}{l_{n}^{1/2}}\sum_{i=1}^{l_{n}}\left(
u_{j,(k-1)l_{n}+i}^{\ast }(x)-u_{1,(k-1)l_{n}+i}^{\ast }(x)\right) \right)
^{2}.  \label{sigmaG*}
\end{equation}%
Define $\widehat{\sigma }_{jj,n}^{\ast 2C^{+}}(x)$ analogously, replacing $%
u_{j,t}^{\ast }(x)$ with $\eta _{j,t}^{\ast }(x)=\left[ e_{j,t}^{\ast }-x%
\right] _{+}-\left[ e_{j,t}-x\right] _{+}.$ The following result holds.

\noindent \textbf{Lemma 2: }\textit{Let Assumptions A1-A3 hold. Then, if }$%
l_{n}\approx n^{\delta }$\textit{\ }$0<\delta <\frac{1}{2},$\textit{\ with }$%
\delta $\textit{\ defined as in Assumption A1:}

\textit{\noindent (i)}%
\begin{equation*}
\sup_{x\in \mathcal{X}^{+}}\left\vert \widehat{\sigma }_{jj,n}^{\ast G+}(x)-%
\mathrm{E}^{\ast }\left( \widehat{\sigma }_{jj,n}^{\ast G+}(x)\right)
\right\vert =o_{p}^{\ast }\left( 1\right) ,
\end{equation*}%
\textit{and (ii)}%
\begin{equation*}
\sup_{x\in \mathcal{X}^{+}}\left\vert \widehat{\sigma }_{jj,n}^{\ast C+}(x)-%
\mathrm{E}^{\ast }\left( \widehat{\sigma }_{jj,n}^{\ast C+}(x)\right)
\right\vert =o_{p}^{\ast }\left( 1\right) ,
\end{equation*}%
\textit{where }$o_{p}^{\ast }(1)$\textit{\ denotes convergence to zero
according to the bootstrap law, }$P^{\ast },$\textit{\ conditional on the
sample.}

\subsection{Inference Using the Bootstrap and Bounding Limiting Distributions%
}

The statistics $S_{n}^{G+}$ and $S_{n}^{C+}$ are highly discontinuous over $%
x.$ Exactly which moment conditions, and how many of them are binding varies
over $x.$ Hence, $S_{n}^{G+}$ and $S_{n}^{C+}$ do not necessarily have a
well defined limiting distribution; and the continuous mapping theorem
cannot be applied. However, following the generalized moment selection (GMS)
test approach of Andrews and Shi (2013) we can establish lower and upper
bound limiting distributions. Let%
\begin{equation*}
D^{G+}(x)=\mathrm{diag}\Sigma _{n}^{G+}\left( x,x\right) ,
\end{equation*}%
\begin{equation}
h_{A,n}^{G+}(x)=D^{G+}(x)^{-1/2}\left( \sqrt{n}G_{2}^{+}(x),...,\sqrt{n}%
G_{k}^{+}(x)\right) ^{\prime },  \label{ha}
\end{equation}

\begin{equation}
h_{B}^{G+}(x,x^{\prime })=D^{G+}(x)^{-1/2}\left( \Sigma
_{n}^{G+}+\varepsilon I_{k-1}\right) \left( x,x^{\prime }\right)
D^{G+}(x^{\prime })^{-1/2},  \label{hB}
\end{equation}%
and%
\begin{equation}
v^{G+}(.)=(v_{2}^{G+}(.),...,v_{k}^{G+}(.))^{\prime },  \label{vu}
\end{equation}%
where $v^{G+}(.)$ is a ($k-1)-$dimensional zero mean Gaussian process with
correlation $h_{B}^{G+}(x,x^{\prime })$. Also, let $%
D^{C+}(x),h_{A,n}^{C+}(x),h_{B}^{C+}(x,x^{\prime }),v^{C+}(.)$ be defined
analogously, by replacing $\Sigma ^{G+}\left( x,x\right)
,G_{2}^{+}(x),...,G_{k}^{+}(x)$ with $\Sigma ^{C+}\left( x,x\right)
,C_{2}^{+}(x),...,C_{k}^{+}(x).$ Finally, define%
\begin{equation}
S_{n}^{\dag G+}=\int_{\mathcal{X}^{+}}\sum_{j=2}^{k}\left( \max \left\{ 0,%
\frac{v_{j}^{G+}(x)+h_{j,A,n}^{G+}(x)}{\sqrt{h_{jj,B}^{G+}(x)}}\right\}
\right) ^{2}\mathrm{d}Q(x)  \label{Sn-dag}
\end{equation}%
where $h_{jj,B}^{G+}(x)$ is the $jj-$th element of $h_{B}^{G+}(x,x)$, and let%
\begin{equation}
S_{\infty }^{\dag G+}=\int_{\mathcal{X}^{+}}\sum_{j=2}^{k}\left( \max
\left\{ 0,\frac{v_{j}^{G+}(x)+h_{j,A,\infty }^{G+}(x)}{\sqrt{h_{jj,B}^{G+}(x)%
}}\right\} \right) ^{2}\mathrm{d}Q(x),  \label{Sn-inf}
\end{equation}%
where $h_{j,A,\infty }^{G+}(x)=0,$ if $G_{j}(x)=0,$ and $h_{j,A,\infty
}^{G+}(x)=-\infty $ , if $G_{j}(x)<0.$ Also, define $S_{n}^{\dag C+}$ and $%
S_{\infty }^{\dag C+}$ analogously, by replacing $%
v_{j}^{G+}(x),h_{j,A,n}^{G+}(x),h_{j,A,\infty }^{G+}(x),$ and $%
h_{jj,B}^{G+}(x)$ with $v_{j}^{C+}(x),h_{j,A,n}^{C+}(x),h_{j,A,\infty
}^{C+}(x),$ and $h_{jj,B}^{C+}(x).$ Hereafter let%
\begin{equation*}
\mathcal{P}_{0}^{G+}=\left\{ P:\text{ }H_{0}^{G+}\text{ holds}\right\}
\end{equation*}%
so that $\mathcal{P}_{0}^{G+}$ is the collection of DGPs under which the
null hypothesis holds. Needless to say, if Assumption A0 also hold, then $%
H_{0}^{G}=H_{0}^{G+}\cap H_{0}^{G-}$ is equivalent to $H_{0},$ as defined in
(\ref{H0}). Let $\mathcal{P}_{0}^{C+}$ be defined analogously, with $%
H_{0}^{G+}$ replaced by $H_{0}^{C+}.$ The following result holds.

\noindent \textbf{Theorem 1: }\textit{Let Assumptions A1-A4 hold. Then:}

\textit{\noindent (i) under }$H_{0}^{G+},$\textit{\ there exists an }$\delta
>0$\textit{\ such that}%
\begin{equation*}
\lim \sup_{n\rightarrow \infty }\sup_{P\in \mathcal{P}_{0}^{G+}}\left[
P\left( S_{n}^{G+}>a_{h_{A,n}}^{G+}\right) -P\left( S_{n}^{\dag G+}+\delta
>a_{h_{A,n}}^{G+}\right) \right] \leq 0
\end{equation*}%
\textit{and}%
\begin{equation*}
\lim \inf_{n\rightarrow \infty }\inf_{P\in \mathcal{P}_{0}^{G+}}\left[
P\left( S_{n}^{G+}>a_{h_{A,n}}^{G+}\right) -P\left( S_{n}^{\dag G+}-\delta
>a_{h_{A,n}}^{G+}\right) \right] \geq 0;
\end{equation*}%
\textit{and \noindent (ii) under }$H_{0}^{C+},$\textit{\ there exists an }$%
\delta >0$\textit{\ such that}%
\begin{equation*}
\lim \sup_{n\rightarrow \infty }\sup_{P\in \mathcal{P}_{0}^{C+}}\left[
P\left( S_{n}^{C+}>a_{h_{A,n}}^{C+}\right) -P\left( S_{n}^{\dag C+}+\delta
>a_{h_{A,n}}^{C+}\right) \right] \leq 0
\end{equation*}%
\textit{and}%
\begin{equation*}
\lim \inf_{n\rightarrow \infty }\inf_{P\in \mathcal{P}_{0}^{C+}}\left[
P\left( S_{n}^{C+}>a_{h_{A,n}}^{C+}\right) -P\left( S_{n}^{\dag C+}-\delta
>a_{h_{A,n}}^{C+}\right) \right] \geq 0.
\end{equation*}%
Theorem 1 provides upper and lower bounds for $P\left(
S_{n}^{G+}>a_{h_{A,n}}^{G+}\right) $ and $P\left(
S_{n}^{C+}>a_{h_{A,n}}^{C+}\right) ,$ uniformly, over the probabilities
under $H_{0}^{G+}$ and $H_{0}^{C+},$ respectively. Note that $%
h_{j,A,n}^{G+}(.)$ and $h_{j,A,n}^{C+}(.)$ depend on the degree of
slackness, and do not need to converge. Indeed, $S_{n}^{G+}$ and/or $%
S_{n}^{C+}$ do not have to converge in distribution for this result to hold.

Following Andrews and Shi (2013), we can construct bootstrap critical values
which properly mimic the critical values of $S_{\infty }^{\dag G+}$ and $%
S_{\infty }^{\dag C+}.$ We rely on the block bootstrap to capture the
dependence in the data when constructing our bootstrap statistics. Consider
the case of $S_{\infty }^{\dag G+}.$ Let $\left( e_{j,1}^{\ast
},...,e_{j,n}^{\ast }\right) ,b_{n},$ and $l_{n}$ be defined as in the
previous subsection, and let:%
\begin{equation}
G_{j,n}^{\ast +}(x)=\frac{1}{\sqrt{n}}\sum_{i=1}^{n}\left( 1\left\{
e_{j,i}^{\ast }\leq x\right\} -1\left\{ e_{1,i}^{\ast }\leq x\right\} \right)
\label{G*}
\end{equation}%
and%
\begin{equation}
v_{n}^{\ast G+}(x)=\sqrt{n}\widehat{D}_{n}^{-1/2,G+}(x)\left( G_{n}^{\ast
+}(x)-G_{n}^{+}(x)\right)  \label{vu*}
\end{equation}

\noindent with $v_{n}^{\ast G+}(x)=\left( v_{2,n}^{\ast
G+}(x),...,v_{k,n}^{\ast G+}(x)\right) $ and $\widehat{D}_{n}^{G+}(x)=%
\mathrm{diag}\widehat{\Sigma }_{n}^{G+}\left( x,x\right) .$ Then, define:%
\begin{equation}
\xi _{j,n}^{G+}(x)=\kappa _{n}^{-1}n^{1/2}\overline{D}%
_{jj,n}^{-1/2,G+}(x)G_{j,n}^{+}(x),  \label{xi}
\end{equation}%
with $\kappa _{n}\rightarrow \infty ,$ as $n\rightarrow \infty .$ Here, $%
\overline{D}_{jj,n}^{G+}(x)$ is the $jj$-th element of $\overline{D}%
_{n}^{G+}(x)=\mathrm{diag}\left( \overline{\Sigma }_{n}^{G+}\left(
x,x\right) \right) ,$ $\xi _{n}^{G+}(x)=\left( \xi _{2,n}^{G+}(x),...,\xi
_{k,n}^{G+}(x)\right) ,$ and%
\begin{equation}
\phi _{j,n}^{G+}(x)=c_{n}1\left\{ \xi _{j,n}^{G+}(x)<-1\right\} ,
\label{phi}
\end{equation}%
with $c_{n}$ a positive sequence, which is bounded above from zero. Thus, $%
\phi _{j,n}^{G+}(x)=c_{n},$ when $G_{j,n}^{+}(x)<-\kappa _{n}n^{-1/2}%
\overline{D}_{jj,n}^{-1/2,G+}(x)$ (i.e., when the $j-$th inequality is slack
at $x),$ and is zero otherwise.

It it clear from the selection rule in (\ref{phi}), that we do need an
estimator of the variance of the moment conditions, despite the fact we use
bootstrap critical values. In fact, standardization does not play a crucial
role in the statistics, as all positive sample moment conditions matter. On
the other hand, without the scaling factor in (\ref{xi}), the number of
non-slack moment conditions would depend on the scale, and hence our
bootstrap critical values would no longer be scale invariant. Let%
\begin{equation}
S_{n}^{\ast G+}=\int_{\mathcal{X}^{+}}\sum_{j=2}^{k}\max \left( \left\{ 0,%
\frac{v_{j,n}^{\ast G+}(x)-\phi _{j,n}^{G+}(x)}{\sqrt{\overline{h}%
_{B,jj}^{\ast G+}(x)}}\right\} \right) ^{2}\mathrm{d}Q(x),  \label{Sn*}
\end{equation}%
where $\overline{h}_{B,jj}^{\ast G+}(x)$ is the $jj$ element of $\widehat{D}%
_{n}^{-1/2,G+}(x)\overline{\Sigma }_{n}^{\ast G+}\left( x,x\right) \widehat{D%
}_{n}^{-1/2,G+}(x)$ and $\overline{\Sigma }_{n}^{\ast G+}\left( x,x\right) $
is the bootstrap analog of $\overline{\Sigma }_{n}^{G+}\left( x,x\right) .$%
\footnote{%
Thus, the diagonal elements of $\widehat{\Sigma }_{n}^{\ast G+}\left(
x,x\right) $ are the $\widehat{\sigma }_{jj,n}^{2\ast G+}(x)$ described in
the previous subsection, while the off-diagonal elements of $\widehat{\Sigma 
}_{n}^{\ast G+}\left( x,x\right) $ are defined accordingly, as $\widehat{%
\sigma }_{jj^{\prime },n}^{2\ast G+}(x)$, with $j\neq j^{\prime }.$} Note
that if $c_{n}$ grows with $n,$ then all slack inequalities are discarded,
asymptotically. It is immediate to see that $S_{n}^{\ast G+}$ is the
bootstrap counterpart of $S_{n}^{\dag G+}$ in (\ref{Sn-dag}), with $\phi
_{j,n}^{G+}(x)$ mimicking the contribution of the slackness of inequality $j$
(i.e., of $j-$th element of $h_{A,n}^{G+}(x)).$ However, $\phi
_{j,n}^{G+}(x) $ is not a consistent estimator of $h_{A,n}^{G+}(x),$ since
the latter cannot be consistently estimated.

\noindent

Now, consider the case of $S_{\infty }^{\dag C+}.$ Let:%
\begin{equation*}
C_{j,n}^{\ast +}(x)=\frac{1}{\sqrt{n}}\sum_{i=1}^{n}\left( \left[
e_{j,t}^{\ast }-x\right] _{+}-\left[ e_{1,t}^{\ast }-x\right] _{+}\right) ,
\end{equation*}%
and define $v_{n}^{\ast C+}(x),\widehat{D}_{n}^{C+}(x),\xi _{j,n}^{C+}(x),$
and $\phi _{j,n}^{C+}(x)$ analogously to $v_{n}^{\ast G+}(x),\widehat{D}%
_{n}^{G+}(x),\xi _{j,n}^{G+}(x),$ and $\phi _{j,n}^{G+}(x),$ by replacing $%
G_{n}^{\ast +}(x),G_{n}^{+}(x)$ and $\widehat{\Sigma }_{n}^{G+}\left(
x,x\right) $ with $C_{n}^{\ast +}(x),C_{n}^{+}(x)$ and $\widehat{\Sigma }%
_{n}^{C+}\left( x,x\right) .$ Then, construct:%
\begin{equation}
S_{n}^{\ast C+}=\int_{\mathcal{X}^{+}}\sum_{j=2}^{k}\max \left( \left\{ 0,%
\frac{v_{j,n}^{\ast C+}(x)-\phi _{j,n}^{C+}(x)}{\sqrt{\overline{h}%
_{B,jj}^{\ast C+}(x)}}\right\} \right) ^{2}\mathrm{d}Q(x).  \label{SnC*}
\end{equation}%
By comparing (\ref{SnG}) and (\ref{SnC}) with (\ref{Sn*}) and (\ref{SnC*}),
it is immediate to see that $G_{j,n}^{+}(x)$ does not contribute to the test
statistic when $G_{j,n}^{+}(x)<0,$ while it does not contribute to the
bootstrap statistic when $G_{j,n}^{+}(x)<-\kappa _{n}n^{-1/2}\overline{D}%
_{jj,n}^{-1/2,G+}(x)$, with $\kappa _{n}n^{-1/2}\rightarrow 0.$
Heuristically, by letting $\kappa _{n}$ grow with the sample size, we
control the rejection rates in a uniform manner.

\noindent

It remains to define the GMS bootstrap critical values. Let $c_{n,B,1-\alpha
}^{\ast G+}\left( \phi _{n}^{G+},\overline{h}_{B,n}^{\ast G+}\right) $ be
the $(1-\alpha )$-th \ critical value of $S_{n}^{\ast G+},$ based on $B$
bootstrap replications, with $\phi _{n}^{G+}$ defined as in (\ref{phi}) and $%
\overline{h}_{B,n}^{\ast G+}(x)=\widehat{D}_{n}^{-1/2,G+}(x)\overline{\Sigma 
}_{n}^{\ast G+}\left( x,x\right) \widehat{D}_{n}^{-1/2,G+}(x)$. The ($%
1-\alpha )$-th GMS bootstrap critical value, $c_{0,n,1-\alpha }^{\ast
G+}\left( \phi _{n}^{G+},\overline{h}_{B,n}^{\ast G+}\right) ,$ is defined
as:%
\begin{equation*}
c_{0,n,1-\alpha }^{\ast G+}\left( \phi _{n}^{G+},\overline{h}_{B,n}^{\ast
G+}\right) =\lim_{B\rightarrow \infty }c_{n,B,1-\alpha +\eta }^{\ast
G+}\left( \phi _{n}^{G+},\overline{h}_{B,n}^{\ast G+}\right) +\eta ,
\end{equation*}%
for $\eta >0,$ arbitrarily small. Further, $c_{n,B,1-\alpha }^{\ast
C+}\left( \phi _{n}^{C+},\overline{h}_{B,n}^{\ast C+}\right) $ and $%
c_{0,n,1-\alpha }^{\ast C+}\left( \phi _{n}^{C+},\overline{h}%
_{B,n}^{C+}\right) $ are defined analogously.

Here, the constant $\eta $ is used to guarantee uniformity over the infinite
dimensional nuisance parameters, $h_{A,n}^{G+}(.),h_{A,n}^{C+}(.),$
uniformly on $x\in \mathcal{X}^{+},$ and is termed the infinitesimal
uniformity factor by Andrews and Shi (2013). Heuristically, if all moment
conditions are slack, then both the statistic and its bootstrap counterpart
are zero, and by having $\eta >0$ though arbitrarily close to zero we
control the asymptotic rejection rate.

Finally, let%
\begin{equation}
\mathcal{B}^{G+}=\left\{ x\in \mathcal{X}^{+}\text{ s.t. }h_{A,j,\infty
}^{G+}=0,\text{ for some }j=2,...,k\right\}  \label{BG+}
\end{equation}%
and%
\begin{equation}
\mathcal{B}^{C+}=\left\{ x\in \mathcal{X}^{+}\text{ s.t. }h_{A,j,\infty
}^{C+}=0,\text{ for some }j=2,...,k\right\} ,  \label{BC+}
\end{equation}%
where $\mathcal{B}^{G+}$ and $\mathcal{B}^{C+}$ define the sets over which
at least one moment condition holds with strict equality, and these sets
represent the boundaries of $H_{0}^{G+}$ and $H_{0}^{C+},$ respectively.

Although we require that the block length grows at the same rate as the lag
truncation parameter, $l_{n},$ in Lemma 2 (i.e., we require that $%
l_{n}\approx n^{\delta }$ $0<\delta <\frac{1}{2}$ with $\delta $ being the
mixing coefficient in A1), for the asymptotic uniform validity of the
bootstrap critical values, we require that the block length grows at a rate
slower than $n^{1/3}.$ This slower rate is required for the bootstrap
empirical central limit theorem for mixing process to hold (see Peligrad
(1998)). Needless to say, even in the construction of $\widehat{\sigma }%
_{jj,n}^{2,G+}(x)$, we should thus use $l_{n}=o(n^{1/3})$ $.$ The following
result holds.

\noindent \textbf{Theorem 2: }\textit{Let Assumptions A1-A4 hold, and let }$%
l_{n}\rightarrow \infty $\textit{\ and }$l_{n}n^{\frac{1}{3}-\varepsilon
}\rightarrow 0$\textit{\ as }$n\rightarrow \infty .$ \textit{Under }$%
H_{0}^{G+}:$

\textit{\noindent (i) if as }$n\rightarrow \infty ,$\textit{\ }$\kappa
_{n}\rightarrow \infty $\textit{\ and }$c_{n}/\kappa _{n}\rightarrow 0,$%
\textit{\ then}%
\begin{equation*}
\lim \sup_{n\rightarrow \infty }\sup_{P\in \mathcal{P}_{0}^{G+}}P\left(
S_{n}^{G+}\geq c_{0,n,1-\alpha }^{\ast G+}\left( \phi _{n}^{G+},\overline{h}%
_{B,n}^{\ast G+}\right) \right) \leq \alpha ;
\end{equation*}%
\textit{and }

\textit{\noindent (ii) if as }$n\rightarrow \infty ,$\textit{\ }$\kappa
_{n}\rightarrow \infty ,$\textit{\ }$c_{n}\rightarrow \infty $\textit{, }$%
\sqrt{n}/\kappa _{n}\rightarrow \infty ,$\textit{\ and }$Q\left( \mathcal{B}%
^{G+}\right) >0,$\textit{\ then}%
\begin{equation*}
\lim_{\eta \rightarrow 0}\lim \sup_{n\rightarrow \infty }\sup_{P\in \mathcal{%
P}_{0}^{G+}}P\left( S_{n}^{G+}\geq c_{0,n,1-\alpha }^{\ast G+}\left( \phi
_{n}^{G+},\overline{h}_{B,n}^{\ast G+}\right) \right) =\alpha .
\end{equation*}%
\textit{\noindent Also, under }$H_{0}^{C+}:$

\textit{\noindent (iii) if as }$n\rightarrow \infty ,$\textit{\ }$\kappa
_{n}\rightarrow \infty $\textit{\ and }$c_{n}/\kappa _{n}\rightarrow 0,$%
\textit{\ then}%
\begin{equation*}
\lim \sup_{n\rightarrow \infty }\sup_{P\in \mathcal{P}_{0}^{C+}}P\left(
S_{n}^{C+}\geq c_{0,n,1-\alpha }^{\ast C+}\left( \phi _{n}^{C+},\overline{h}%
_{B,n}^{\ast C+}\right) \right) \leq \alpha ;
\end{equation*}%
\textit{\noindent and (iv) if as }$n\rightarrow \infty ,$\textit{\ }$\kappa
_{n}\rightarrow \infty ,$\textit{\ }$c_{n}\rightarrow \infty $\textit{, }$%
\sqrt{n}/\kappa _{n}\rightarrow \infty ,$\textit{\ and }$Q\left( \mathcal{B}%
^{C+}\right) >0,$\textit{\ then}%
\begin{equation*}
\lim_{\eta \rightarrow 0}\lim \sup_{n\rightarrow \infty }\sup_{P\in \mathcal{%
P}_{0}^{C+}}P\left( S_{n}^{C+}\geq c_{0,n,1-\alpha }^{\ast C+}\left( \phi
_{n}^{C+},\overline{h}_{B,n}^{\ast C+}\right) \right) =\alpha .
\end{equation*}%
Statements (i) and (iii) of Theorem 2 establish that inference based on GMS
bootstrap critical values has uniform correct size. Statements (ii) and (iv)
of the theorem establish that inference based on GMS bootstrap critical
values is asymptotically non-conservative, whenever $Q\left( \mathcal{B}%
^{+}\right) >0$ or $Q\left( \mathcal{B}^{C+}\right) >0$ (i.e., whenever at
least one moment condition holds with equality, over a set $x\in \mathcal{X}%
^{+}$ with non-zero $Q-$measure). Although the GMS based tests are not
similar on the boundary, the degree of non similarity, which is%
\begin{eqnarray*}
&&\lim_{\eta \rightarrow 0}\lim \sup_{n\rightarrow \infty }\sup_{P\in 
\mathcal{P}_{0}^{G+}}P\left( S_{n}^{G+}\geq c_{0,n,1-\alpha }^{\ast
G+}\left( \phi _{n}^{G+},\overline{h}_{B,n}^{\ast G+}\right) \right) \\
&&-\lim_{\eta \rightarrow 0}\lim \inf \inf_{P\in \mathcal{P}%
_{0}^{G+}}P\left( S_{n}^{G+}\geq c_{0,n,1-\alpha }^{\ast G+}\left( \phi
_{n}^{G+},\overline{h}_{B,n}^{\ast G+}\right) \right) ,
\end{eqnarray*}%
is much smaller than that associated with using the \textquotedblleft
usual\textquotedblright\ recentered bootstrap.

In the case of pairwise comparison (i.e., $k=2),$ Theorem 2(ii) of Linton,
Song and Whang (2010) establishes similarity of stochastic dominance tests
on a subset of the boundary.

\subsection{Power against Fixed and Local Alternatives}

As our statistics are weighted averages over $\mathcal{X}^{+},$ they have
non-trivial power only if the null is violated over a subset of non zero $Q-$%
measure. This applies to both power against fixed alternative, as well as to
power against $\sqrt{n}-$local alternatives. In particular, for power
against fixed alternatives, we require the following assumption.

\noindent \textbf{Assumption FA: }(i) $Q(B_{FA}^{G+})>0,$ where $%
B_{FA}^{G+}=\left\{ x\in \mathcal{X}^{+}:G_{j}(x)>0\text{ for some }%
j=2,...,k\right\} .$. (ii) $Q(B_{FA}^{C+})>0$ where $B_{FA}^{C+}=\left\{
x\in \mathcal{X}^{+}:C_{j}(x)>0\text{ for some }j=2,...,k\right\} .$

The following result holds.

\noindent \textbf{Theorem 3: }\textit{Let Assumptions A1-A4 hold. }

\textit{\noindent (i) If Assumption FA(i) holds, then under }$H_{A}^{G+}:$%
\begin{equation*}
\lim_{n\rightarrow \infty }P\left( S_{n}^{G+}\geq c_{0,n,1-\alpha }^{\ast
G+}\left( \phi _{n}^{G+},\overline{h}_{B,n}^{\ast G+}\right) \right) =1.
\end{equation*}%
\textit{\noindent (ii) If Assumption FA(ii) holds, then under }$H_{A}^{C+}:$%
\begin{equation*}
\lim_{n\rightarrow \infty }P\left( S_{n}^{C+}\geq c_{0,n,1-\alpha }^{\ast
C+}\left( \phi _{n}^{C+},\overline{h}_{B,n}^{\ast C+}\right) \right) =1.
\end{equation*}%
It is immediate to see that we have unit power against fixed alternatives,
provided that the null hypothesis is violated, for at least one $j=2,...,k,$
over a subset of $\mathcal{X}^{+}$ of non-zero $Q-$measure. Now, if we
instead used a Kolmogorov type statistic (i.e., replace the integral over $%
\mathcal{X}^{+}$ with the supremum over $\mathcal{X}^{+}),$ then we would
not need Assumption FA, and it would suffice to have violation for some $x,$
with possibly zero $Q-$measure, or in general with zero Lebesgue measure.%
\footnote{%
The Kolmorogov versions of $S_{n}^{G+}$ and $S_{n}^{C+}$ are:%
\begin{equation*}
KS_{n}^{G+}=\max_{x\in \mathcal{X}^{+}}\sum_{j=2}^{k}\left( \max \left\{ 0,%
\frac{\sqrt{n}G_{j,n}^{+}(x)}{\overline{\sigma }_{jj,n}^{G+}(x)}\right\}
\right) ^{2}
\end{equation*}%
\begin{equation*}
KS_{n}^{C+}=\max_{x\in \mathcal{X}^{+}}\sum_{j=2}^{k}\left( \max \left\{ 0,%
\frac{\sqrt{n}C_{j,n}^{+}(x)}{\overline{\sigma }_{jj,n}^{G+}(x)}\right\}
\right) ^{2}
\end{equation*}%
} However, as pointed out in Supplement B of Andrews and Shi (2013) the
statements in parts (ii) and (iv) of Theorem 2 do not apply to Kolmogorov
tests, and hence asymptotic non-conservativeness does not necessarily hold.
This is because the proof of those statements use the bounded convergence
theorem, which applies to integrals but not to suprema.

We now consider the following sequences of local alternatives:%
\begin{equation*}
H_{L,n}^{G+}:G_{Lj}^{+}(x)=G_{j}^{+}(x)+\frac{\delta _{1,j}(x)}{\sqrt{n}}%
+o\left( n^{-1/2}\right) ,\text{ for }j=2,...,k,\text{ }x\in \mathcal{X}^{+}
\end{equation*}%
and%
\begin{equation*}
H_{L,n}^{C+}:C_{Lj}^{+}(x)=C_{j}^{+}(x)+\frac{\delta _{2,j}(x)}{\sqrt{n}}%
+o\left( n^{-1/2}\right) ,\text{ for }j=2,...,k,\text{ }x\in \mathcal{X}^{+}.
\end{equation*}%
We have $\lim_{n\rightarrow \infty }\sqrt{n}D^{G+}(x)^{-1/2}G_{Lj}^{+}(x)%
\rightarrow h_{j,A,\infty }^{G+}(x)+\delta _{1,j}(x),$ and $%
\lim_{n\rightarrow \infty }\sqrt{n}D^{C+}(x)^{-1/2}C_{Lj}^{+}(x)\rightarrow
h_{j,A,\infty }^{C^{+}}(x)+\delta _{2,j}(x).$ Define,%
\begin{equation*}
S_{\infty ,\delta _{1},LG}^{\dag ,G+}=\int_{\mathcal{X}^{+}}\sum_{j=2}^{k}%
\left( \max \left\{ 0,\frac{v_{j}^{G+}(x)+h_{j,A,\infty }^{G+}(x)+\delta
_{1,j}(x)}{\sqrt{h_{jj,B}^{G+}(x)}}\right\} \right) ^{2}\mathrm{d}Q(x)
\end{equation*}%
and%
\begin{equation*}
S_{\infty ,\delta _{2},LC}^{\dag ,C+}=\int_{\mathcal{X}^{+}}\sum_{j=2}^{k}%
\left( \max \left\{ 0,\frac{v_{j}^{C+}(x)+h_{j,A,\infty }^{C+}(x)+\delta
_{2,j}(x)}{\sqrt{h_{jj,B}^{C+}(x)}}\right\} \right) ^{2}\mathrm{d}Q(x)
\end{equation*}

\noindent We require the following assumption.

\noindent \textbf{Assumption LA:}(i) $Q(B_{LA}^{G+})>0,$ where

\noindent $B_{LA}^{G+}=\left\{ x:\sqrt{n}D^{G+}(x)^{-1/2}G_{j}^{+}(x)%
\rightarrow h_{j,A,\infty }^{G+}(x),\text{ }0<h_{j,A,\infty }^{G+}(x)<\infty
,\text{ for some }j=2,...,k\right\} $.

\noindent (ii) $Q(B_{LA}^{C+})>0,$ where

\noindent $B_{LA}^{C+}=\left\{ x:\sqrt{n}D^{C+}(x)^{-1/2}C_{j}^{+}(x)%
\rightarrow h_{j,A,\infty }^{C+}(x),\text{ }0<h_{j,A,\infty }^{C+}(x)<\infty
,\text{ for some }j=2,...,k\right\} .$

The following result holds.

\noindent \textbf{Theorem 4: }\textit{Let Assumptions A1-A4 hold.}

\textit{\noindent (i) If Assumption AL(i) holds, then under }$H_{L,n}^{G+}:$%
\begin{equation*}
\lim_{n\rightarrow \infty }P\left( S_{n}^{G+}\geq c_{0,n,1-\alpha }^{\ast
G+}\left( \phi _{n}^{G+},\overline{h}_{B,n}^{\ast G+}\right) \right)
=P\left( S_{\infty ,\delta _{1},LG}^{\dag G+}\geq c_{LG,1-\alpha }\left(
h_{A,\infty }^{G+},h_{B,\infty }^{G+}\right) \right) ,
\end{equation*}%
\textit{with }$c_{LG,1-\alpha }\left( h_{A,\infty }^{G+},h_{B,\infty
}^{G+}\right) $\textit{\ denoting the (}$1-\alpha )$\textit{-th critical
value of }$S_{\infty ,\delta _{1},LG}^{\dag G+}$\textit{, with }$%
0<h_{j,A,\infty }^{G+}(x)<\infty ,$\textit{\ for some }$j=2,...,k.$

\textit{\noindent (ii) If Assumption AL(ii) holds, then under }$%
H_{L,n}^{C+}: $%
\begin{equation*}
\lim_{n\rightarrow \infty }P\left( S_{n}^{C+}\geq c_{0,n,1-\alpha }^{\ast
C+}\left( \phi _{n}^{C+},\overline{h}_{B,n}^{\ast C+}\right) \right)
=P\left( S_{\infty ,\delta _{2},LC}^{\dag C+}\geq c_{LC,1-\alpha }\left(
h_{A,\infty }^{C+},h_{B,\infty }^{C+}\right) \right) ,
\end{equation*}%
\textit{with }$c_{LC,1-\alpha }\left( h_{A,\infty }^{C+},h_{B,\infty
}^{C+}\right) $\textit{\ denoting the (}$1-\alpha )$\textit{-th critical
value of }$S_{\infty ,\delta _{2},LC}^{\dag C+}$\textit{, with }$%
0<h_{j,A,\infty }^{C+}(x)<\infty ,$\textit{\ for some }$j=2,...,k.$

Theorem 4 establishes that our tests have power against $\sqrt{n}-$%
alternatives, provided that the drifting sequence is bounded away from zero,
over a subset of $\mathcal{X}^{+}$ of non-zero $Q-$measure. Note also that
for given loss function, $L,$ the sequence of local alternatives for the
White reality check can be defined as:%
\begin{equation}
H_{A,n}:\max_{j=2,...,k}\left( E(L(e_{1}))-E(L(e_{j}))\right) =\frac{\lambda 
}{\sqrt{n}}+o\left( n^{-1/2}\right) ,\text{ }\lambda >0.  \label{HAn}
\end{equation}%
For sake of simplicity, suppose that $k=2$ (this is the well known
Diebold-Mariano test framework). Here,

\begin{eqnarray}
&&0<\lambda =n^{1/2}E(L(e_{1}))-E(L(e_{k}))  \notag \\
&=&n^{1/2}\int_{-\infty }^{\infty }L(x)\left( f_{1,n}(x)-f_{2,n}(x)\right) 
\mathrm{d}z  \notag \\
&=&-n^{1/2}\int_{-\infty }^{0}L^{\prime }(x)\left(
F_{1,n}(x)-F_{2,n}(x)\right) \mathrm{d}x  \notag \\
&&-n^{1/2}\int_{0}^{\infty }L^{\prime }(x)\left(
F_{1,n}(x)-F_{2,n}(x)\right) \mathrm{d}x  \notag \\
&=&n^{1/2}\int_{-\infty }^{0}\left( h_{A,\infty }^{G-}(x)+\delta
_{1}(x)\right) Q(x)\mathrm{d}x+n^{1/2}\int_{0}^{\infty }\left( h_{A,\infty
}^{G+}(x)+\delta _{1}(x)\right) Q(x)\mathrm{d}x,  \label{RC}
\end{eqnarray}%
where $F_{j,n}(x)=F_{j}(x)+\frac{\delta _{1,j}(x)}{\sqrt{n}}$ as defined in
JCS (2017) and above, and and $\delta _{1}=\delta _{1,1}-\delta _{1,2}.$
Hence, $H_{A,n}$ in (\ref{HAn}) is equivalent to $H_{LA}^{G+}\cap
H_{LA}^{G-} $, whenever $Q(x)=L^{\prime }(x)sign(x).$

Analogously, for any convex loss function, $L,$ which satisfies Assumption
A0, $H_{A,n}$ in (\ref{HAn}) is equivalent to $H_{LA}^{C-}\cap H_{LA}^{C+-}$%
, whenever $Q(x)=L^{\prime \prime }(x)sign(x).$ In fact, it is easy to see
that:

\begin{eqnarray*}
&&0<\delta =n^{1/2}E(L(e_{1}))-E(L(e_{k})) \\
&=&n^{1/2}\int_{-\infty }^{\infty }L(x)\left( f_{1,n}(x)-f_{2,n}(x)\right) 
\mathrm{d}x \\
&=&-n^{1/2}\int_{-\infty }^{0}L^{\prime }(x)\left(
F_{1,n}(x)-F_{2,n}(x)\right) \mathrm{d}z-n^{1/2}\int_{0}^{\infty }L^{\prime
}(x)\left( F_{1,n}(x)-F_{2,n}(x)\right) \mathrm{d}z \\
&=&-L^{\prime }(x)n^{1/2}\int_{-\infty }^{x}\left(
F_{1,n}(z)-F_{2,n}(z)\right) \mathrm{d}z\left\vert _{-\infty }^{0}\right.
+n^{1/2}\int_{-\infty }^{0}L^{\prime \prime }(x)\left( \int_{-\infty
}^{x}\left( F_{1,n}(z)-F_{2,n}(z)\right) \mathrm{d}z\right) \mathrm{d}x \\
&&+n^{1/2}L^{\prime }(x)\int_{x}^{\infty }\left(
F_{1,n}(z)-F_{2.n}(z)\right) \mathrm{d}z\left\vert _{0}^{\infty }\right.
-n^{1/2}\int_{0}^{\infty }L^{\prime \prime }(x)\left( \int_{x}^{\infty
}\left( F_{1,n}(z)-F_{2,n}(z)\right) \mathrm{d}z\right) \mathrm{d}x \\
&=&n^{1/2}\int_{-\infty }^{0}L^{\prime \prime }(x)\left( \int_{-\infty
}^{x}\left( F_{1,n}(z)-F_{2,n}(z)\right) \mathrm{d}z\right) \mathrm{d}%
x-n^{1/2}\int_{0}^{\infty }L^{\prime \prime }(x)\left( \int_{x}^{\infty
}\left( F_{1,n}(z)-F_{2,n}(z)\right) \mathrm{d}z\right) \mathrm{d}x \\
&=&n^{1/2}\int_{-\infty }^{0}\left( \int_{-\infty }^{x}\left( h_{A,\infty
}^{C-}(x)+\delta _{2}(x)\right) \mathrm{d}z\right) Q(x)\mathrm{d}%
x-n^{1/2}\int_{0}^{\infty }\left( \int_{x}^{\infty }\left( h_{A,\infty
}^{C+}(x)+\delta _{2}(x)\right) \mathrm{d}z\right) Q(x)\mathrm{d}x.
\end{eqnarray*}

\section{Forecast Superiority Tests in the Presence of Recursive Estimation
Error}

\subsection{\noindent The Statistics}

Thus far, we have considered the case of subjective or judgmental forecasts,
in which the econometrician is provided with sequences of forecasts and
forecast errors. Our analysis has thus far been model free. When the
objective is the evaluation of forecasts generated by estimated models,
parameter estimation error must be accounted for. Consider the following
standard setup. Let $T=R+n.$ We use the first $R$ observations for
estimation and the last $n$ for out of sample predictive evaluation. For
brevity, we outline the case of recursive estimation. Namely, at each point
in time, $t>R,$ update model parameter estimates prior to the construction
of each new forecast, using all the available information.\footnote{%
In the rolling estimation case, we use only the most recent $R$ observations
to re-estiamte the forecasdting model, for each $t>R.$ The rolling case can
be treated analogously, and it is omitted only for brevity.}

Formalizing the idea of recursive estimation, for $j=1,...,k,$ use the first 
$R$ observations to compute $\widehat{\theta }_{j,R},$ and construct the
first prediction error: 
\begin{equation*}
\widehat{e}_{j,R+1}=X_{R+1}-\phi _{j}\left( Z_{j,R},\widehat{\theta }%
_{j,R}\right) ,
\end{equation*}%
where $Z_{j,R}$ contains lags of $X$ as well as other regressors. Then, use
the first $R+1$ observations to construct%
\begin{equation*}
\widehat{e}_{j,R+2}=X_{R+2}-\phi _{j}\left( Z_{j,R+1},\widehat{\theta }%
_{j,R+1}\right) .
\end{equation*}%
Proceed in the same manner until a sequence of $n$ prediction errors has
been constructed, defined as: 
\begin{equation}
\widehat{e}_{j,t+1}=X_{t+1}-\phi _{j}\left( Z_{j,t},\widehat{\theta }%
_{j,t}\right) ,  \label{e-hat}
\end{equation}%
for $t=R,...,R+n-1,$ where $\widehat{\theta }_{j,t}$ is the estimator
computed using observations up to time $t.$ In the sequel, assume that $%
\widehat{\theta }_{j,t}$ is a recursive $m-$estimator, so that:%
\begin{equation}
\widehat{\theta }_{j,t}=\arg \min_{\theta _{j}\in \Theta _{j}}\frac{1}{t}%
\sum_{i=2}^{t}m_{j}(X_{i},Z_{j,i-1},\theta _{j}),\text{ \ \ }R\leq t\leq n-1,%
\text{ }j=1,...,k,  \label{theta-i}
\end{equation}%
and%
\begin{equation*}
\theta _{j}^{\dagger }=\arg \min_{\theta _{j}\in \Theta
_{j}}E(m_{j}(X_{i},Z_{j,i-1},\theta _{j})).
\end{equation*}%
For $x\geq 0,$ define:%
\begin{equation}
\widetilde{G}_{j,n}^{+}(x)=\frac{1}{n}\sum_{t=R}^{T-1}\left( 1\left\{ 
\widehat{e}_{j,t+1}\leq x\right\} -1\left\{ \widehat{e}_{1,t+1}\leq
x\right\} \right) =\left( \widetilde{F}_{j,n}(x)-\widetilde{F}%
_{1,n}(x)\right)  \label{Gtilde}
\end{equation}%
and%
\begin{eqnarray}
\widetilde{C}_{j,n}^{+}(x) &=&\int_{x}^{\infty }\left( \widetilde{F}%
_{j,n}(t)-\widetilde{F}_{1,n}(t)\right) dt  \notag \\
&=&\frac{1}{n}\sum_{t=R}^{T-1}\left\{ \left[ \left( \widehat{e}%
_{1,t+1}-x\right) \right] _{+}-\left[ \left( \widehat{e}_{j,t+1}-x\right) %
\right] _{+}\right\} .  \label{Ctilde}
\end{eqnarray}%
As shown in the proof of Lemma 3(i) in the Appendix, 
\begin{eqnarray*}
&&\frac{1}{\sqrt{n}}\sum_{t=R}^{T-1}\left( 1\left\{ \widehat{e}_{j,t+1}\leq
x\right\} -F_{j}(x)\right) \\
&=&\frac{1}{\sqrt{n}}\sum_{t=R}^{T-1}\left( 1\left\{ e_{j,t+1}\leq x\right\}
-F_{j}(x)\right) +f_{j}(x)\frac{1}{\sqrt{n}}%
\sum_{t=R}^{T-1}PEE_{j,t}+o_{p}(1),
\end{eqnarray*}%
where, for $t\geq R,$%
\begin{eqnarray}
&&PEE_{j,t}  \notag \\
&=&\mathrm{E}\left( \nabla _{\theta _{j}}\phi _{j}\left( Z_{j,t+1},\theta
_{j}^{\dagger }\right) \right)  \notag \\
&&\times \left( \mathrm{E}\left( \nabla _{\theta
}^{2}m_{j}(X_{i},Z_{j,i-1},\theta _{j}^{\dagger })\right) \right) ^{-1}\frac{%
1}{t}\sum_{i=1}^{t}\nabla _{\theta }m_{j}(X_{i},Z_{j,i-1},\theta
_{j}^{\dagger })+o_{p}(1).  \label{PEEj}
\end{eqnarray}

It is immediate to see that if $n=o(R),$ then $\frac{1}{\sqrt{n}}%
\sum_{t=R}^{T-1}PEE_{j,t}=o_{p}(1),$ and thus does not contribute to the
asymptotic covariance of the above statistics.

Define the following forecast superiority test statistics:%
\begin{equation*}
\widetilde{S}_{n}^{G+}=\int_{\mathcal{X}^{+}}\sum_{j=2}^{k}\left( \max
\left\{ 0,\frac{\sqrt{n}\widetilde{G}_{j,n}^{+}(x)}{\widetilde{\sigma }%
_{jj,n}^{G+}(x)+\varepsilon }\right\} \right) ^{2}\mathrm{d}Q(x)
\end{equation*}%
and%
\begin{equation*}
\widetilde{S}_{n}^{C+}=\int_{\mathcal{X}^{+}}\sum_{j=2}^{k}\left( \max
\left\{ 0,\frac{\sqrt{n}\widetilde{C}_{j,n}^{+}(x)}{\widetilde{\sigma }%
_{jj,n}^{C+}(x)+\varepsilon }\right\} \right) ^{2}\mathrm{d}Q(x),
\end{equation*}%
where $\widetilde{\sigma }_{jj,n}^{G+}(x)$ and $\widetilde{\sigma }%
_{jj,n}^{C+}(x)$ include terms accounting for the contribution of parameter
estimation error to asymptotic covariance. Here, $\widetilde{\sigma }%
_{jj,n}^{2,G+}(x)$ is defined as:%
\begin{eqnarray*}
\widetilde{\sigma }_{jj,n}^{2,G+}(x) &=&\widehat{\sigma }_{jj,n}^{2,G+}(x)+2%
\widehat{\Pi }\widehat{f}_{1,n,h}^{2}(x)\widehat{A}_{1}\widehat{\Sigma }_{11}%
\widehat{A}_{1}^{\prime }+2\widehat{\Pi }\widehat{f}_{j,n,h}^{2}(x)\widehat{A%
}_{j}\widehat{\Sigma }_{jj}\widehat{A}_{j}^{\prime } \\
&&-4\widehat{\Pi }\widehat{f}_{1,n,h}(x)\widehat{A}_{1}\widehat{\Sigma }_{1j}%
\widehat{A}_{j}^{\prime }\widehat{f}_{j,n,h}(x)+2\widehat{\Pi }\widehat{f}%
_{1,n,h}(x)\widehat{A}_{1}\widehat{\Sigma }_{u1}(x)-2\widehat{\Pi }\widehat{f%
}_{j,n,h}(x)\widehat{A}_{j}\widehat{\Sigma }_{uj}(x),
\end{eqnarray*}%
where $\widehat{\sigma }_{jj,n}^{2,G+}(x)$ is defined as in (\ref{HAC})%
\footnote{%
However, $e_{t,j}$ and $e_{t,1}$ are replaced by $\widehat{e}_{t,j}$ and $%
\widehat{e}_{t,1},$ and only the last $n$ observations, from $R+1$ to $T,$
are used in test statistic construction.}, $\widehat{\Pi }=1-\frac{R}{n}\ln
\left( 1+\frac{n}{R}\right) ,$%
\begin{equation*}
\widehat{f}_{j,n,h}(x)=\frac{1}{nh}\sum_{t=R+1}^{n}K\left( \frac{\widehat{e}%
_{j,t}-x}{h}\right) ,
\end{equation*}%
\begin{equation*}
\widehat{A}_{j}=\frac{1}{n}\sum_{t=R+1}^{T}\nabla _{\theta _{j}}\phi
_{j}\left( Z_{j,t+1},\widehat{\theta }_{j,R}\right) ^{\prime }\left( \frac{1%
}{R}\sum_{t=1}^{R}\nabla _{\theta _{j}}^{2}m_{j}(X_{t},Z_{j,t-1},\widehat{%
\theta }_{j,R})\right) ^{-1},
\end{equation*}%
\begin{eqnarray*}
\widehat{\Sigma }_{jj} &=&\frac{1}{n}\sum_{t=R+1}^{T}\nabla _{\theta
_{j}}m_{j}(X_{t},Z_{t,i-1},\widehat{\theta }_{j,R})\nabla _{\theta
_{j}}m_{j}(X_{t},Z_{t,i-1},\widehat{\theta }_{j,R})^{\prime } \\
&&+2\frac{1}{n}\sum_{\tau =1}^{l_{n}}\sum_{t=R+\tau +1}^{T}w_{\tau }\nabla
_{\theta _{j}}m_{j}(X_{t},Z_{t,i-1},\widehat{\theta }_{j,R})\nabla _{\theta
_{j}}m_{j}(X_{t-\tau },Z_{t-\tau ,i-1},\widehat{\theta }_{j,R})^{\prime },
\end{eqnarray*}%
and%
\begin{eqnarray*}
\widehat{\Sigma }_{uj}(x) &=&\frac{1}{n}\sum_{t=R+1}^{T}\nabla _{\theta
_{j}}m_{j}(X_{t},Z_{t,i-1},\widehat{\theta }_{j,R})\left( \left( 1\left\{ 
\widehat{e}_{j,t}\leq x\right\} -\frac{1}{n}\sum_{t=1}^{n}1\left\{ \widehat{e%
}_{j,t}\leq x\right\} \right) \right. \\
&&\left. -\left( 1\left\{ \widehat{e}_{1,t}\leq x\right\} -\frac{1}{n}%
\sum_{t=1}^{n}1\left\{ \widehat{e}_{1,t}\leq x\right\} \right) \right) \\
&&+2\frac{1}{n}\sum_{\tau =1}^{l_{n}}\sum_{t=R+\tau +1}^{T}w_{\tau }\nabla
_{\theta _{j}}m_{j}(X_{t},Z_{t,i-1},\widehat{\theta }_{j,R})\left( \left(
1\left\{ \widehat{e}_{j,t-\tau }\leq x\right\} -\frac{1}{n}%
\sum_{t=1}^{n}1\left\{ \widehat{e}_{j,t-\tau }\leq x\right\} \right) \right.
\\
&&\left. -\left( 1\left\{ \widehat{e}_{1,t-\tau }\leq x\right\} -\frac{1}{n}%
\sum_{t=1}^{n}1\left\{ \widehat{e}_{1,t-\tau }\leq x\right\} \right) \right)
,
\end{eqnarray*}%
By noting that%
\begin{eqnarray}
&&\widetilde{C}_{j,n}^{+}(x)  \notag \\
&=&\frac{1}{n}\sum_{t=R}^{T-1}\left( \left[ \left( e_{1,t+1}-x\right) \right]
_{+}-\left[ \left( e_{j,t+1}-x\right) \right] _{+}\right)  \notag \\
&&+\frac{1}{n}\sum_{t=R}^{T-1}\left( \left( \widehat{e}_{1,t}-e_{1,t}\right)
1\left\{ e_{1,t}\geq x\right\} -\left( \widehat{e}_{j,t}-e_{j,t}\right)
1\left\{ e_{j,t}\geq x\right\} \right)  \notag \\
&&+\frac{1}{n}\sum_{t=R}^{T-1}\left( \left( e_{1,t}-x\right) \left( 1\left\{ 
\widehat{e}_{1,t}\geq x\right\} -1\left\{ \widehat{e}_{1,t}\geq x\right\}
\right) -\left( e_{j,t}-x\right) \left( 1\left\{ \widehat{e}_{j,t}\geq
x\right\} -1\left\{ e_{j,t}\geq x\right\} \right) \right)  \label{C-PEE} \\
&&+\frac{1}{n}\sum_{t=R}^{T-1}\left( \left( \widehat{e}_{1,t}-e_{1,t}\right)
\left( 1\left\{ \widehat{e}_{1,t}\geq x\right\} -1\left\{ e_{1,t}\geq
x\right\} \right) -\left( \widehat{e}_{j,t}-e_{j,t}\right) \left( 1\left\{ 
\widehat{e}_{j,t}\geq x\right\} -1\left\{ e_{j,t}\geq x\right\} \right)
\right)  \notag
\end{eqnarray}%
we see that $\widetilde{\sigma }_{jj,n}^{2,C+}(x)$ is defined as:%
\begin{eqnarray*}
&&\widetilde{\sigma }_{jj,n}^{2,G+}(x) \\
&=&\widehat{\sigma }_{jj,n}^{2,C+}(x)+2\widehat{\Pi }\widehat{f}%
_{1,n,h}^{2}(x)\widetilde{A}_{1}(x)\widehat{\Sigma }_{11}\widetilde{A}%
_{1}^{\prime }(x)+2\widehat{\Pi }\widehat{f}_{j,n,h}^{2}(x)\widetilde{A}%
_{j}(x)\widehat{\Sigma }_{jj}\widetilde{A}_{j}^{\prime }(x) \\
&&-4\widehat{\Pi }\widehat{f}_{1,n,h}(x)\widetilde{A}_{1}(x)\widehat{\Sigma }%
_{1j}\widetilde{A}_{j}^{\prime }(x)\widehat{f}_{j,n,h}(x)+2\widehat{\Pi }%
\widehat{f}_{1,n,h}(x)\widetilde{A}_{1}(x)\widehat{\Sigma }_{u1}(x)-2%
\widehat{\Pi }\widehat{f}_{j,n,h}(x)\widetilde{A}_{j}(x)\widehat{\Sigma }%
_{uj}(x) \\
&&+2\widehat{\Pi }\widetilde{B}_{1}(x)\widehat{\Sigma }_{11}\widetilde{B}%
_{1}^{\prime }(x)+2\widehat{\Pi }\widetilde{B}_{j}^{\prime }(x)\widehat{%
\Sigma }_{jj}\widetilde{B}_{j}^{\prime }(x)-4\widehat{\Pi }\widetilde{B}%
_{1}(x)\widehat{\Sigma }_{1j}\widetilde{B}_{j}^{\prime }(x) \\
&&+2\widehat{\Pi }\widetilde{B}_{1}(x)\widehat{\Sigma }_{u1}(x)-2\widehat{%
\Pi }\widetilde{B}_{j}(x)^{\prime }\widehat{\Sigma }_{uj}(x) \\
&&+2\widehat{\Pi }\widehat{f}_{1,n,h}(x)\widetilde{A}_{1}(x)\widehat{\Sigma }%
_{11}\widetilde{B}_{1}^{\prime }(x)+2\widehat{\Pi }\widehat{f}_{j,n,h}(x)%
\widetilde{A}_{j}\widehat{\Sigma }_{jj}\widetilde{B}_{j}^{\prime }(x)-2%
\widehat{\Pi }\widetilde{B}_{1}(x)\widehat{\Sigma }_{1j}\widetilde{A}%
_{j}^{\prime }(x)\widehat{f}_{j,n,h}(x) \\
&&-2\widehat{\Pi }\widetilde{B}_{j}(x)\widehat{\Sigma }_{1j}\widetilde{A}%
_{1}^{\prime }(x)\widehat{f}_{1,n,h}(x),
\end{eqnarray*}%
where $\widehat{\sigma }_{jj,n}^{2,C+}(x)$ is defined as in the statement of
Lemma 1, but computed using only the last $n$ observations, with prediction
errors replaced by estimated prediction errors. Also, 
\begin{equation*}
\widetilde{A}_{j}(x)=\frac{1}{n}\sum_{t=R+1}^{T}\left( \widehat{e}%
_{t+1,j}-x\right) \nabla _{\theta _{j}}\phi _{j}\left( Z_{j,t+1},\widehat{%
\theta }_{j,R}\right) ^{\prime }\left( \frac{1}{R}\sum_{t=1}^{R}\nabla
_{\theta _{j}}^{2}m_{j}(X_{t},Z_{j,t-1},\widehat{\theta }_{j,R})\right) ^{-1}
\end{equation*}%
and%
\begin{equation*}
\widetilde{B}_{j}(x)=\frac{1}{n}\sum_{t=R+1}^{T}1\left\{ \widehat{e}%
_{t+1,j}>x\right\} \nabla _{\theta _{j}}\phi _{j}\left( Z_{j,t+1},\widehat{%
\theta }_{j,R}\right) ^{\prime }\left( \frac{1}{R}\sum_{t=1}^{R}\nabla
_{\theta _{j}}^{2}m_{j}(X_{t},Z_{j,t-1},\widehat{\theta }_{j,R})\right)
^{-1}.
\end{equation*}%
In order to formalize the case of asymptotically non-vanishing parameter
estimation error, we require the following assumptions.

\noindent \textbf{Assumption A5: }$\phi _{j}$ is twice continuously
differentiable on the interior of $\Theta _{j}$ and the elements of $\nabla
_{\theta _{j}}\phi _{j}(Z_{j,i-1},\theta _{i})$ and $\nabla _{\theta
_{j}}^{2}\phi _{j}(Z_{j,i-1},\theta _{i})$ are $p-$dominated on $\Theta
_{i}, $ for $j=1,...,k,$ with $p>4.$

\noindent \textbf{Assumption A6: }For $j=1,...,k:$ (i) $\theta _{j}^{\dagger
}$ is uniquely identified (i.e. $E(m_{j}(X_{t},Z_{j,t-1},\theta
_{j}))>E(m_{j}(X_{t},Z_{j,t-1},\theta _{j}^{\dagger })),$ for any $\theta
_{j}\neq \theta _{j}^{\dagger });$ (ii) $m_{j}$ is twice continuously
differentiable on the interior of $\Theta _{j};$ (iii) the elements of $%
\nabla _{\theta _{j}}m_{j}$ and $\nabla _{\theta _{j}}^{2}m_{j}$ are $p-$%
dominated on $\Theta _{j},$ with $p>4$; and (iii) $E\left( -\nabla _{\theta
_{j}}^{2}m_{j}(\theta _{j})\right) $ is positive definite, uniformly on $%
\Theta _{j}.$\footnote{%
We say that $\nabla _{\theta _{j}}\ln f_{j}(y_{t},Z^{t-1},\theta _{j})$ is $%
2r-$dominated on $\Theta _{j}$ if its $v-th$ element, $v=1,...,\varrho (j),$
is such that $\left\vert \nabla _{\theta _{j}}\ln f_{j}(y_{t},Z^{t-1},\theta
_{j})\right\vert _{v}\leq D_{t},$ and $E(\left\vert D_{t}\right\vert
^{2r})<\infty .$ For more details on domination conditions, see Gallant and
White (1988, pp. 33).}

\noindent \textbf{Assumption A7: }$T=R+n,$ and as $T\rightarrow \infty ,$ $%
n/R\rightarrow \pi ,$ with $0\leq \pi <\infty .$

As explained earlier, it is crucial to have a consistent estimator of the
variance of the moment conditions. Otherwise, bootstrap critical values are
not scale invariant. Hence, we need to construct estimators which properly
capture parameters estimation error, regardless the fact that we rely on
bootstrap critical values. GMS tests in the presence of non-vanishing
estimation error have been considered in Coroneo, Corradi and
Santos-Monteiro (2017). We have the following result.

\noindent \textbf{Lemma 3: }\textit{Let Assumptions A1-A3, and A5-A7 hold.
If }$l_{n}\approx n^{\delta }$\textit{\ }$\delta <\frac{1}{2},$\textit{\ as
defined in Assumption A1}$,$\textit{\ then:}

\textit{\noindent (i) }$\sup_{x\in \mathcal{X}^{+}}\left\vert \widetilde{%
\sigma }_{jj,n}^{2,G+}(x)-\omega _{jj}^{2,G+}(x)\right\vert =o_{p}\left(
1\right) ,$\textit{\ with }$\omega _{jj}^{2G+}(x)=avar\left( \sqrt{n}%
\widetilde{G}_{j,n}^{+}(x)\right) ;$\textit{\ and}

\textit{\noindent (ii) }$\sup_{x\in \mathcal{X}^{+}}\left\vert \widetilde{%
\sigma }_{jj,n}^{2,C+}(x)-\omega _{jj}^{2,C+}(x)\right\vert =o_{p}\left(
1\right) $\textit{, with }$\omega _{jj}^{2C+}(x)=avar\left( \sqrt{n}%
\widetilde{C}_{j,n}^{+}(x)\right) .$

\noindent Lemma 3 mirrors Lemma 1 for the case of non-vanishing estimation
error. In order to provide the analog of Theorem 1 for the case of non
vanishing estimation error, we need define the counterparts of $S_{n}^{\dag
G+}$ and $S_{n}^{\dag C+}$ which take into account of parameter estimation
error. Let $\overline{\Omega }^{G+}\left( x,x\right) =\Omega ^{G+}\left(
x,x\right) +\varepsilon I_{k-1},$ where $\Omega ^{G+}\left( x,x\right)
=[\omega _{ij,n}^{2,G+}(x)].$ Also,

\begin{equation*}
\mathcal{D}^{G+}(x)=\mathrm{diag}\Omega ^{G+}\left( x,x\right) ,
\end{equation*}%
\begin{equation*}
h_{1,n}^{G+}(x)=\mathcal{D}^{G+}(x)^{-1/2}\left( \sqrt{n}G_{2}^{+}(x),...,%
\sqrt{n}G_{k}^{+}(x)\right) ^{\prime },
\end{equation*}

\begin{equation*}
h_{2}^{G+}(x,x^{\prime })=\mathcal{D}^{G+}(x)^{-1/2}\overline{\Omega }%
^{G+}\left( x,x^{\prime }\right) \mathcal{D}^{G+}(x^{\prime })^{-1/2},
\end{equation*}%
and%
\begin{equation*}
v^{G+}(.)=(v_{2}^{G+}(.),...,v_{k}^{G+}(.))^{\prime }.
\end{equation*}%
Here, $v^{G+}(.)$ is a ($k-1)-$dimensional zero mean Gaussian process with
correlation $h_{2}^{G+}(x,x^{\prime })$. Also, let $\mathcal{D}%
^{C+}(x),h_{1,n}^{C+}(x),h_{2}^{C+}(x,x^{\prime }),$ and $v^{C+}(.)$ be
defined analogously by replacing $\Omega ^{G+}\left( x,x\right)
,G_{2}^{+}(x),...,G_{k}^{+}(x)$ with $\Omega ^{C+}\left( x,x\right)
,C_{2}^{+}(x),...,C_{k}^{+}(x).$

Finally, define%
\begin{equation*}
S_{n}^{\ddag G+}=\int_{\mathcal{X}^{+}}\sum_{j=2}^{k}\left( \max \left\{ 0,%
\frac{v_{j}^{G+}(x)+h_{j,1,n}^{G+}(x)}{\sqrt{h_{jj,2}^{G+}(x)}}\right\}
\right) ^{2}\mathrm{d}Q(x),
\end{equation*}%
where $h_{jj,2}^{G+}(x)$ is the $jj-$th element of $h_{2}^{G+}(x,x)$, and
let 
\begin{equation*}
S_{n}^{\ddag C+}=\int_{\mathcal{X}^{+}}\sum_{j=2}^{k}\left( \max \left\{ 0,%
\frac{v_{j}^{C+}(x)+h_{j,1,n}^{C+}(x)}{\sqrt{h_{jj,2}^{C+}(x)}}\right\}
\right) ^{2}\mathrm{d}Q(x),
\end{equation*}%
which is defined analogously, by replacing $v_{j}^{G+}(x),h_{j,1,n}^{G+}(x),$
and $h_{jj,2}^{G+}(x)$ with $v_{j}^{C+}(x),h_{j,1,n}^{C+}(x),$ and $%
h_{jj,2}^{C+}(x).$ The following result holds.

\noindent \textbf{Theorem 5: }\textit{Let Assumptions A1-A7 hold. }

\textit{(i) Under }$H_{0}^{G+},$\textit{\ there exist }$\delta >0$\textit{\
such that:}%
\begin{equation*}
\lim \sup_{n\rightarrow \infty }\sup_{P\in \mathcal{P}_{0}^{G+}}\left[
P\left( \widetilde{S}_{n}^{G+}>a_{h_{A,n}}^{G+}\right) -P\left( S_{n}^{\ddag
G+}+\delta >a_{h_{A,n}}^{G+}\right) \right] \leq 0
\end{equation*}%
\textit{and}%
\begin{equation*}
\lim \inf_{n\rightarrow \infty }\inf_{P\in \mathcal{P}_{0}^{G+}}\left[
P\left( \widetilde{S}_{n}^{G+}>a_{h_{A,n}}^{G+}\right) -P\left( S_{n}^{\ddag
G+}-\delta >a_{h_{A,n}}^{G+}\right) \right] \geq 0.
\end{equation*}%
\textit{\noindent \qquad }

\textit{(ii) Under }$H_{0}^{C+},$\textit{\ there exist }$\delta >0$\textit{\
such that:}%
\begin{equation*}
\lim \sup_{n\rightarrow \infty }\sup_{P\in \mathcal{P}_{0}^{C+}}\left[
P\left( \widetilde{S}_{n}^{C+}>a_{h_{A,n}}^{C+}\right) -P\left( S_{n}^{\ddag
C+}+\delta >a_{h_{A,n}}^{C+}\right) \right] \leq 0
\end{equation*}%
\textit{and}%
\begin{equation*}
\lim \inf_{n\rightarrow \infty }\inf_{P\in \mathcal{P}_{0}^{C+}}\left[
P\left( \widetilde{S}_{n}^{C+}>a_{h_{A,n}}^{C+}\right) -P\left( S_{n}^{\ddag
C+}-\delta >a_{h_{A,n}}^{C+}\right) \right] \geq 0.
\end{equation*}%
Theorem 5 provides upper and lower bounds for $P\left( \widetilde{S}%
_{n}^{G+}>a_{h_{A,n}}^{G+}\right) $ and $P\left( \widetilde{S}%
_{n}^{C+}>a_{h_{A,n}}^{C+}\right) ,$ uniformly, over the probabilities under
the null $H_{0}^{G+}$ and $H_{0}^{C+},$ respectively. Note that $%
h_{j,A,n}^{G+}(.)$ and $h_{j,A,n}^{C+}(.)$ depend on the degree of slackness
and do not need to converge. Indeed, $\widetilde{S}_{n}^{G+}$ and $%
\widetilde{S}_{n}^{C+}$ do not have to converge in distribution.

\subsection{Bootstrap Estimators}

When computing recursive $m-$estimators, it is important to note that
earlier observations are used more frequently than temporally subsequent
observations. On the other hand, in the standard block bootstrap, all blocks
from the original sample have the same probability of being selected,
regardless of the dates of the observations in the blocks. Thus, the
bootstrap estimator, say $\widehat{\theta }_{j,t}^{\ast },$ which is
constructed as a direct analog of $\widehat{\theta }_{j,t}$ in (\ref{theta-i}%
)$,$ is characterized by a location bias that can be either positive or
negative, depending on the sample that we observe. In order to circumvent
this problem, Corradi and Swanson (2007) suggest a re-centering of the
bootstrap score which ensures that the new bootstrap estimator, which is no
longer the direct analog of $\widehat{\theta }_{j,t},$ is asymptotically
unbiased. It should be noted that the idea of re-centering is not new in the
bootstrap literature for the case of full sample estimation. In the context
of $m-$estimators using the full sample, re-centering is needed only for
higher order asymptotics, but not for first order validity, in the sense
that the bias term is of smaller order than $n^{-1/2}$ (see e.g. Andrews
(2002)). In the case of recursive $m-$estimators, on the other hand, the
bias term is of order $n^{-1/2},$ so that it does contribute to the limiting
distribution. In the sequel, we assume that the block length grows with the
sample$.$ Also, assume that $T=R+n=b_{T}l_{T},$ with $b_{T}=b_{n}\frac{T}{n}$
and $l_{T}=l_{n}\frac{T}{n},$ and define:%
\begin{equation*}
\widetilde{\theta }_{j,t}^{\ast }=\arg \min_{\theta _{j}\in \Theta _{j}}%
\frac{1}{t}\sum_{i=1}^{t}\left( m_{j}(X_{i}^{\ast },Z_{j,i-1}^{\ast },\theta
_{j})-\theta _{j}^{\prime }\left( \frac{1}{T}\sum_{k=1}^{T-1}\nabla _{\theta
_{j}}m_{j}(X_{k},Z_{j,k-1},\widehat{\theta }_{j,t})\right) \right) ,
\end{equation*}%
where $X_{i}^{\ast },Z_{j,i-1}^{\ast }$ are resampled via the
\textquotedblleft standard\textquotedblright\ block bootstrap outlined in
the previous section, but with block length $b_{T}.$ Theorem 1 in Corradi
and Swanson (2007) establish that $\frac{1}{\sqrt{n}}\sum_{t=R}^{n-1}\left( 
\widetilde{\theta }_{j,t}^{\ast }-\widehat{\theta }_{j,t}\right) $ has the
same limiting distribution as $\frac{1}{\sqrt{n}}\sum_{t=R}^{n-1}\left( 
\widehat{\theta }_{j,t}-\theta _{j}^{\dagger }\right) ,$ conditional of the
sample.

With a slight abuse of notation, let $u_{j,t}^{\ast }(x)=1\{e_{j,t}^{\ast
}\leq x\}-\frac{1}{T}\sum_{t=1}^{T}1\{\widehat{e}_{j,t}\leq x\}$ and $\eta
_{j,t}^{\ast }(x)=\left[ e_{j,t}^{\ast }-x\right] _{+}-\frac{1}{T}%
\sum_{t=1}^{n}\left[ \widehat{e}_{j,t}-x\right] _{+},$ with $e_{j,t+1}^{\ast
}=X_{t+1}^{\ast }-\phi _{j}\left( Z_{j,t}^{\ast },\widehat{\theta }%
_{j,t}\right) $, and let $\widehat{u}_{j,t}^{\ast }(x)=1\{\widehat{e}%
_{j,t}^{\ast }\leq x\}-\frac{1}{T}\sum_{t=1}^{T}1\{\widehat{e}_{j,t}\leq x\}$
and $\widehat{\eta }_{j,t}^{\ast }(x)=\left[ \widehat{e}_{j,t}^{\ast }-x%
\right] _{+}-\frac{1}{T}\sum_{t=1}^{n}\left[ \widehat{e}_{j,t}-x\right]
_{+}, $ with $\widehat{e}_{j,t+1}^{\ast }=X_{t+1}^{\ast }-\phi _{j}\left(
Z_{j,t}^{\ast },\widetilde{\theta }_{j,t}^{\ast }\right) .$ Our first goal
is to construct the bootstrap counterparts of $\widetilde{\sigma }%
_{jj,n}^{2,G+}(x)$ and $\widetilde{\sigma }_{jj,n}^{2,C+}(x)$, called $%
\widetilde{\sigma }_{jj,n}^{\ast 2,G+}(x)$ and $\widetilde{\sigma }%
_{jj,n}^{\ast 2,C+}(x).$ Define:%
\begin{eqnarray*}
&&\widetilde{\sigma }_{jj,n}^{\ast 2,G+}(x) \\
&=&\widehat{\mathrm{avar}^{\ast }}\left( \frac{1}{\sqrt{n}}%
\sum_{t=R}^{n-1}\left( \widehat{u}_{j,t}^{\ast }(x)-\widehat{u}_{1,t}^{\ast
}(x)\right) \right) \\
&=&\widehat{\mathrm{avar}^{\ast }}\left( \frac{1}{\sqrt{n}}%
\sum_{t=R}^{n-1}\left( u_{j,t}^{\ast }(x)-u_{1,t}^{\ast }(x)\right) \right) +%
\widehat{\mathrm{avar}^{\ast }}\left( \frac{1}{\sqrt{n}}\sum_{t=R}^{n-1}%
\left( \widehat{f}_{j,n,h}^{\ast }(x)\widehat{PEE^{\ast }}_{j,t}-\widehat{f}%
_{1,n,h}^{\ast }(x)\widehat{PEE^{\ast }}_{1,t}\right) \right) \\
&&-2\widehat{\mathrm{acov}^{\ast }}\left( \frac{1}{\sqrt{n}}%
\sum_{t=R}^{n-1}\left( u_{j,t}^{\ast }(x)-u_{1,t}^{\ast }(x)\right) ,\frac{1%
}{\sqrt{n}}\sum_{t=R}^{n-1}\left( \widehat{f}_{j,n,h}^{\ast }(x)\widehat{%
PEE^{\ast }}_{j,t}-\widehat{f}_{1,n,h}^{\ast }(x)\widehat{PEE^{\ast }}%
_{1,t}\right) \right) ,
\end{eqnarray*}%
and%
\begin{eqnarray*}
&&\widetilde{\sigma }_{jj,n}^{\ast 2,C+}(x) \\
&=&\widehat{\mathrm{avar}^{\ast }}\left( \frac{1}{\sqrt{n}}%
\sum_{t=R}^{n-1}\left( \widehat{\eta }_{j,t}^{\ast }(x)-\widehat{\eta }%
_{1,t}^{\ast }(x)\right) \right) \\
&=&\widehat{\mathrm{avar}^{\ast }}\left( \frac{1}{\sqrt{n}}%
\sum_{t=R}^{n-1}\left( \eta _{j,t}^{\ast }(x)-\eta _{1,t}^{\ast }(x)\right)
\right) \\
&&+\widehat{\mathrm{avar}^{\ast }}\left( \frac{1}{\sqrt{n}}%
\sum_{t=R}^{n-1}\left( \left[ \widehat{f}_{j,n,h}^{\ast }\widehat{PEE}%
_{j,t}^{\ast }-x\right] _{+}-\left[ \widehat{f}_{1,n,h}^{\ast }\widehat{PEE}%
_{1,t}^{\ast }-x\right] _{+}\right) \right) + \\
&&-2\widehat{\mathrm{acov}^{\ast }}\left( \frac{1}{\sqrt{n}}%
\sum_{t=R}^{n-1}\left( \eta _{j,t}^{\ast }(x)-\eta _{1,t}^{\ast }(x)\right) ,%
\frac{1}{\sqrt{n}}\sum_{t=R}^{n-1}\left( \left[ \widehat{f}_{j,n,h}^{\ast }%
\widehat{PEE}_{j,t}^{\ast }-x\right] _{+}-\left[ \widehat{f}_{1,n,h}^{\ast }%
\widehat{PEE}_{1,t}^{\ast }-x\right] _{+}\right) \right) ,
\end{eqnarray*}%
where \textrm{avar}$^{\ast }$ and $\mathrm{acov}^{\ast }$ denote asymptotic
variances and covariances, with respect to the bootstrap probability
measure, $\widehat{f}_{j,n,h}^{\ast }$ is an estimator of the density of $%
e_{j}$ based on the resampled observations, and $\widehat{PEE^{\ast }}_{j,t}$
is an estimator of:%
\begin{eqnarray}
PEE_{j,t}^{\ast } &=&\mathrm{E}^{\ast }\left( \nabla _{\theta _{j}}\phi
_{j}\left( Z_{j,t}^{\ast },\widetilde{\theta }_{j,t}^{\ast }\right) \right) 
\mathrm{E}^{\ast }\left( \nabla _{\theta }^{2}m_{j}\left( X_{i}^{\ast
},Z_{j,i-1}^{\ast },\widetilde{\theta }_{j,t}^{\ast }\right) \right)  \notag
\\
&&\frac{1}{t}\sum_{i=1}^{t}\left( \nabla _{\theta }m_{j}\left( X_{i}^{\ast
},Z_{j,i-1}^{\ast },\widehat{\theta }_{j,t}\right) -\frac{1}{T}%
\sum_{i=1}^{T}\nabla _{\theta _{j}}m_{j}(X_{k},Z_{j,k-1},\widehat{\theta }%
_{j,t}\right) .  \label{PEE*}
\end{eqnarray}%
Closed form expressions for $\widehat{PEE^{\ast }}_{j,t}$, $\widehat{\mathrm{%
avar}^{\ast }}$, and $\widehat{\mathrm{acov}^{\ast }}$ are given in the
proof of Lemma 4. We have the following result.

\noindent \textbf{Lemma 4: }\textit{Let Assumptions A1-A3 and A5-A7 hold.
Then, if }$l_{n}\approx n^{\delta }$\textit{\ }$\delta <\frac{1}{2},$\textit{%
\ and }$\beta $\textit{\ the mixing coefficient in Assumption A1 is such
that }$\beta >\frac{6\delta }{1-2\delta }:$

\textit{\noindent (i) }$\sup_{x\in \mathcal{X}^{+}}\left\vert \widetilde{%
\sigma }_{jj,n}^{\ast 2,G+}(x)-\mathrm{E}^{\ast }\left( \widetilde{\sigma }%
_{jj,n}^{\ast 2,G+}(x)\right) \right\vert =o_{p}\left( 1\right) $\textit{\
and}

\textit{\noindent (ii) }$\sup_{x\in \mathcal{X}^{+}}\left\vert \widetilde{%
\sigma }_{jj,n}^{\ast 2,C+}(x)-\mathrm{E}^{\ast }\left( \widetilde{\sigma }%
_{jj,n}^{\ast 2,C+}(x)\right) \right\vert =o_{p}\left( 1\right) .$\textit{\ }

\subsection{Bootstrap Critical Values}

The bootstrap statistics in the non-vanishing recursive parameter estimation
error case are:%
\begin{equation}
\widetilde{S}_{n}^{\ast G+}=\int_{\mathcal{X}^{+}}\sum_{j=2}^{k}\max \left(
\left\{ 0,\frac{\widetilde{v}_{j,n}^{\ast G+}(x)-\widetilde{\phi }%
_{j,n}^{G+}(x)}{\sqrt{\widetilde{\overline{h}}_{2,jj}^{\ast G+}(x)}}\right\}
\right) ^{2}\mathrm{d}Q(x),  \label{SG*PEE}
\end{equation}%
where$\widetilde{\text{ }\overline{h}}_{2,jj}^{\ast G+}(x)$ is the $jj$
element of $\widetilde{D}_{n}^{-1/2,G+}(x)\overline{\widetilde{\Sigma }}%
_{n}^{\ast G+}\left( x,x\right) \widetilde{D}_{n}^{-1/2,G+}(x),$ with $%
\widetilde{D}_{n}^{G+}(x)=\mathrm{diag}\widetilde{\Sigma }_{n}^{\ast
G+}\left( x,x\right) ,$ $\widetilde{\Sigma }_{n}^{\ast G+}\left( x,x\right)
=[\widetilde{\sigma }_{ij,n}^{\ast 2,G+}(x)]$ $i,j=1,...,k,$ and $\overline{%
\widetilde{\Sigma }}_{n}^{\ast G+}=\widetilde{\Sigma }_{n}^{\ast
G+}+\varepsilon I_{k-1}.$ Also,%
\begin{eqnarray*}
\widetilde{v}_{n}^{\ast G+}(x) &=&\sqrt{n}\widetilde{D}_{n}^{-1/2,G+}(x)%
\frac{1}{\sqrt{n}}\sum_{i=R+1}^{n}\left( \left( 1\left\{ \widehat{e}%
_{j,i}^{\ast }\leq x\right\} -1\left\{ \widehat{e}_{1,i}^{\ast }\leq
x\right\} \right) \right. \\
&&\left. \frac{1}{T}\sum_{t=1}^{T}\left( 1\left\{ \widehat{e}_{j,t}\leq
x\right\} -1\left\{ \widehat{e}_{1,t}\leq x\right\} \right) \right)
\end{eqnarray*}%
and for $\widetilde{\xi }_{j,n}^{G+}(x)=\kappa _{n}^{-1}n^{1/2}\overline{%
\widetilde{D}}_{jj,n}^{-1/2,G+}(x)\widetilde{G}_{j,n}^{+}(x),$%
\begin{equation}
\widetilde{\phi }_{j,n}^{G+}(x)=c_{n}1\left\{ \widetilde{\xi }%
_{j,n}^{G+}(x)<-1\right\} .  \label{phi-pee}
\end{equation}%
Finally, also define 
\begin{equation*}
\widetilde{S}_{n}^{\ast C+}=\int_{\mathcal{X}^{+}}\sum_{j=2}^{k}\max \left(
\left\{ 0,\frac{\widetilde{v}_{j,n}^{\ast C+}(x)-\widetilde{\phi }%
_{j,n}^{C+}(x)}{\sqrt{\widetilde{\overline{h}}_{2,jj}^{\ast C+}(x)}}\right\}
\right) ^{2}\mathrm{d}Q(x),
\end{equation*}%
where $\widetilde{v}_{n}^{\ast C+}(x),\widetilde{D}_{n}^{C^{+}}(x),%
\widetilde{\xi }_{j,n}^{C+}(x),$ and $\widetilde{\phi }_{j,n}^{C+}(x)$ are
defined analogously to $\widetilde{v}_{n}^{\ast G+}(x),\widetilde{D}%
_{n}^{G+}(x),\widetilde{\xi }_{j,n}^{G+}(x),$ and $\widetilde{\phi }%
_{j,n}^{G+}(x)$. It is immediate to see that estimation error contributes to
the bootstrap statistics not only as a scaling factor, but also in
determining which moment conditions are binding. This is why we need an
estimator of the variance, even if inference is based on bootstrap critical
values.

We now define the GMS bootstrap critical values for the case of
non-vanishing recursive estimation error. Let $\widetilde{c}_{n,B,1-\alpha
}^{\ast G+}\left( \widetilde{\phi }_{n}^{G+},\overline{h}_{2,n}^{\ast
G+}\right) $ be the $(1-\alpha )$-th \ critical value of $\widetilde{S}%
_{n}^{\ast G+},$ based on $B$ bootstrap replications, with $\widetilde{\phi }%
_{n}^{G+}$ as in (\ref{phi-pee}) and $\widetilde{\overline{h}}_{2,jj}^{\ast
G+}(x)$ as in (\ref{SG*PEE}). The ($1-\alpha )$-th GMS bootstrap critical
value, $\widetilde{c}_{0,n,1-\alpha }^{\ast G+}\left( \widetilde{\phi }%
_{n}^{G+},\overline{h}_{2,n}^{\ast G+}\right) $ is defined as:%
\begin{equation*}
\widetilde{c}_{0,n,1-\alpha }^{\ast G+}\left( \widetilde{\phi }_{n}^{G+},%
\overline{h}_{B,n}^{\ast G+}\right) =\lim_{B\rightarrow \infty }\widetilde{c}%
_{n,B,1-\alpha +\eta }^{\ast G+}\left( \widetilde{\phi }_{n}^{G+},\overline{h%
}_{2,n}^{\ast G+}\right) +\eta ,
\end{equation*}%
for arbitrarily small $\eta >0$. Also, $\widetilde{c}_{n,B,1-\alpha +\eta
}^{\ast C+}\left( \widetilde{\phi }_{n}^{C+},\overline{h}_{2,n}^{\ast
C+}\right) $ and $\widetilde{c}_{0,n,1-\alpha }^{\ast C+}\left( \widetilde{%
\phi }_{n}^{C+},\overline{h}_{B,n}^{\ast C+}\right) $ are defined
analogously. The following result then holds.

\noindent \textbf{Theorem 6: }Let Assumptions A1-A7 hold, and let $%
l_{n}\rightarrow \infty $ and $l_{n}n^{\frac{1}{3}-\varepsilon }\rightarrow
0,$ as $n\rightarrow \infty .$ Under $H_{0}^{G+}:$

\noindent (i) if as $n\rightarrow \infty ,$ $\kappa _{n}\rightarrow \infty $
and $c_{n}/\kappa _{n}\rightarrow 0,$ then%
\begin{equation*}
\lim \sup_{n\rightarrow \infty }\sup_{P\in \mathcal{P}_{0}^{G+}}P\left( 
\widetilde{S}_{n}^{G+}\geq \widetilde{c}_{n,B,1-\alpha +\eta }^{\ast
C+}\left( \widetilde{\phi }_{n}^{C+},\overline{h}_{2,n}^{\ast C+}\right)
\right) \leq \alpha ;
\end{equation*}%
\noindent and (ii) if as $n\rightarrow \infty ,$ $\kappa _{n}\rightarrow
\infty ,$ $c_{n}\rightarrow \infty $, $\sqrt{n}/\kappa _{n}\rightarrow
\infty $ and $Q\left( \mathcal{B}^{G+}\right) >0,$ $\mathcal{B}^{G+}$ as in (%
\ref{BG+}), then%
\begin{equation*}
\lim_{\eta \rightarrow 0}\lim \sup_{n\rightarrow \infty }\sup_{P\in \mathcal{%
P}_{0}^{G+}}P\left( \widetilde{S}_{n}^{G+}\geq \widetilde{c}_{n,B,1-\alpha
+\eta }^{\ast C+}\left( \widetilde{\phi }_{n}^{C+},\overline{h}_{2,n}^{\ast
C+}\right) \right) =\alpha .
\end{equation*}%
\noindent Also, under $H_{0}^{C+},$

\noindent (iii) if as $n\rightarrow \infty ,$ $\kappa _{n}\rightarrow \infty 
$ and $c_{n}/\kappa _{n}\rightarrow 0,$ then%
\begin{equation*}
\lim \sup_{n\rightarrow \infty }\sup_{P\in \mathcal{P}_{0}^{C+}}P\left( 
\widetilde{S}_{n}^{C+}\geq \widetilde{c}_{0,n,1-\alpha }^{\ast C+}\left( 
\widetilde{\phi }_{n}^{C+},\overline{h}_{B,n}^{\ast C+}\right) \right) \leq
\alpha ;
\end{equation*}%
\noindent and (iv) if as $n\rightarrow \infty ,$ $\kappa _{n}\rightarrow
\infty ,$ $c_{n}\rightarrow \infty $, $\sqrt{n}/\kappa _{n}\rightarrow
\infty $ and $Q\left( \mathcal{B}^{C+}\right) >0,$ $\mathcal{B}^{C+}$ as in (%
\ref{BC+}), then%
\begin{equation*}
\lim_{\eta \rightarrow 0}\lim \sup_{n\rightarrow \infty }\sup_{P\in \mathcal{%
P}_{0}^{C+}}P\left( \widetilde{S}_{n}^{C+}\geq \widetilde{c}_{0,n,1-\alpha
}^{\ast C+}\left( \widetilde{\phi }_{n}^{C+},\overline{h}_{B,n}^{\ast
C+}\right) \right) =\alpha .
\end{equation*}

Statements (i) and (iii) of Theorem 6 establish that inference based on GMS
bootstrap critical values has uniform correct size, in the parameter
estimation error case. Statements (ii) and (iv) of the theorem establish
that inference based on the GMS bootstrap critical values is asymptotically
non-conservative, whenever $Q\left( \mathcal{B}^{+}\right) >0$ or $Q\left( 
\mathcal{B}^{C+}\right) >0.$

\section{Monte Carlo Experiments}

In this section, we evaluate the finite sample performance of GL and CL\
forecast superiority tests when there are multiple competing sequences of
forecast errors, under stationarity. In addition to analyzing the
performance of our tests based on $S_{n}^{G+}$ and $S_{n}^{G-},$ (GL
forecast superiority) as well as based on $S_{n}^{C+}$ and $S_{n}^{C-}$ (CL
forecast superiority), we also analyze the performance of the related test
statistics from JCS (2017), here called $JCS_{n}^{G+}$, $JCS_{n}^{G-},$ $%
JCS_{n}^{C+}$, and $JCS_{n}^{C-}$. For the sake of brevity, these two
classes of tests are called $S_{n}$ and $JCS_{n}$ type tests, respectively.
When computing the suprema used in all of these statistics, we take a
maximum over an equally spaced grid of size $\lceil 1.5n^{0.6}\rfloor ,$
over a 98\% range of the pooled empirical distribution; that is, we take the
1\% and 99\% percentiles of this empirical distribution and then form an
equally spaced grid between these two extremes.\footnote{%
Consider $S_{n}^{G+}$ and $S_{n}^{G-}$ in order to illustrate how our
statistics are constructed. Namely, 
\begin{equation*}
S_{n}^{G+}=\underset{x\in \mathcal{X}^{+}}{\int }\underset{j=2}{\sum^{k}}%
\left( \max \left\{ 0,\sqrt{n}\frac{G_{j,n}(x)}{\overline{\sigma }%
_{jj,n}^{G}(x)}\right\} \right) ^{2}\text{ }dQ(x)\text{ and }S_{n}^{G-}=%
\underset{x\in \mathcal{X}^{-}}{\int }\underset{j=2}{\sum^{k}}\left( \max
\left\{ 0,\sqrt{n}\frac{G_{j,n}(x)}{\overline{\sigma }_{jj,n}^{G}(x)}%
\right\} \right) ^{2}dQ(x).
\end{equation*}%
Or, using simpler notation, 
\begin{equation*}
S_{n}^{G+}=\underset{x\in \mathcal{X}^{+}}{\int }H_{n}(x)\text{ }dQ(x)\text{
and }S_{n}^{G-}=\underset{x\in \mathcal{X}^{-}}{\int }H_{n}(x)\text{ }dQ(x),
\end{equation*}%
say. As an example, call the 1\% and 99\% percentiles of the pooled
empirical distribution are $c1$ and $c2$, and set these values equal to
-0.0026 and 0.0039, respectively. Then the actual 98\% support is [-0.0026,
0.0039]. We form an equally spaced grid, with size [$1.5n^{0.6}$], over this
range. To approximate the integral, instead of directly using $dx$ (in our
example, $dx$ = 0.0002 if $n=200$), we use $dQ(x)$ = $\frac{dx}{c2-c1}=\frac{%
1}{1.5n^{0.6}}$. Thus, $Q(\cdot )$ is still uniform. Also, $H_{n}(x)$ is
still evaluated at the original grid of points. For inference using our
tests, once $B$ is determined, estimate bootstrap $p-$values, $%
p_{B,n,S_{n}}^{G+}=\frac{1}{B}\sum_{s=1}^{B}1\left( (S_{n}^{\ast G^{+}}+\eta
)\geq S_{n}^{G^{+}}\right) .$ and $p_{B,n,S_{n}}^{G-}=\frac{1}{B}%
\sum_{s=1}^{B}1\left( (S_{n}^{\ast G-}+\eta )\geq S_{n}^{G-}\right) $. Then,
use the following rules (Holm, (1979):\textbf{\ }Reject $H_{0}^{TG}$ at
level $\alpha ,$ if $\min \left\{
p_{B,n,S_{n}}^{G+},p_{B,n,S_{n}}^{G-}\right\} \leq (\alpha -\eta )/2$.%
\textbf{\ }Reject $H_{0}^{TC}$ at level $\alpha ,$ if $\min \left\{
p_{B,n,S_{n}}^{C+},p_{B,n,S_{n}}^{C-}\right\} \leq (\alpha -\eta )/2$.} For
each experiment we carry out 1000 Monte Carlo replications, and the number
of bootstrap resamples is $B=300$. Additionally, four different values of
the smoothing parameter, $J_{n}$ are examined for the $JCS_{n}$ type tests,
including $J_{n}=\{0.20,,0.35,0.50,0.60\}$; and four different values of the
uniformity constant, $\eta $ are examined for the $S_{n}$ type tests,
including $\eta =\{0.045,0.060,0.075,0.090\}$.\footnote{%
In JCS (2017), the constant that we call $J_{n}$ is called $S_{n}$.} For the 
$S_{n}$ type tests, when constructing $\overline{\Sigma }_{n}^{G+}$(as well
as $\overline{\Sigma }_{n}^{G-}$, etc.) we set $l_{n}=\func{integer}%
[n^{0.2}] $ and $\varepsilon =1e-8.$ Finally, when constructing bootstrap
statistics for $S_{n}$ type tests, we set $\kappa _{n}=\log (n)$ and $%
c_{n}=\log (\log (n))$. Sample sizes of $n\in \{250,500,1000\}$ are
generated using each of the following eight data generating processes
(DGPs), with independent forecast errors and $i.i.d.$ observations. We do
not introduce forecast error dependence or parameter estimation error into
our setup, as the effects of these departures from our setup are discussed
in JCS (2017). In particular, for the following eight data generating
processes (DGPs), we fix $e_{1t}$ $_{\symbol{126}}$ $i.i.d.N(0,1),$ and let
the number of competing forecasting models vary.

\noindent DGP1: $e_{1t}$ $_{\symbol{126}}$ $i.i.d.N(0,1)$ and $e_{kt}$ $_{%
\symbol{126}}$ $i.i.d.N(0,1),$ $k=2,3.$

\noindent DGP2: $e_{1t}$ $_{\symbol{126}}$ $i.i.d.N(0,1)$ and $e_{kt}$ $_{%
\symbol{126}}$ $i.i.d.N(0,1),$ $k=2,3,4,5.$

\noindent DGP3: $e_{1t}$ $_{\symbol{126}}$ $i.i.d.N(0,1),$ $e_{kt}$ $_{%
\symbol{126}}$ $i.i.d.N(0,1),$ $k=2,3,4,5$ and $e_{kt}$ $_{\symbol{126}}$ $%
i.i.d.N(0,1.2^{2}),$ $k=6,7,8,9.$

\noindent DGP4: $e_{1t}$ $_{\symbol{126}}$ $i.i.d.N(0,1),$ $e_{kt}$ $_{%
\symbol{126}}$ $i.i.d.N(0,0.8^{2}),$ $k=2,3,4,5$ and $e_{kt}$ $_{\symbol{126}%
}$ $i.i.d.N(0,1.2^{2}),$ $k=6,7,8,9.$

\noindent DGP5: $e_{1t}$ $_{\symbol{126}}$ $i.i.d.N(0,1)$ and $e_{kt}$ $_{%
\symbol{126}}$ $i.i.d.N(0,0.8^{2}),$ $k=2,3.$

\noindent DGP6: $e_{1t}$ $_{\symbol{126}}$ $i.i.d.N(0,1)$ and $e_{kt}$ $_{%
\symbol{126}}$ $i.i.d.N(0,0.6^{2}),$ $k=2,3.$

\noindent DGP7: $e_{1t}$ $_{\symbol{126}}$ $i.i.d.N(0,1)$ and $e_{kt}$ $_{%
\symbol{126}}$ $i.i.d.N(0,0.8^{2}),$ $k=2,3,4,5.$

\noindent DGP8: $e_{1t}$ $_{\symbol{126}}$ $i.i.d.N(0,1)$ and $e_{kt}$ $_{%
\symbol{126}}$ $i.i.d.N(0,0.6^{2}),$ $k=2,3,4,5.$

Here, DGPs 1-3 are our \textquotedblleft null\textquotedblright\ models,
while DGPs 4-8 are our \textquotedblleft alternative\textquotedblright\
models. DGPs 1 and 2 correspond to the least favorable elements in the null.
In DGP3, the benchmark model outperforms some of the competing models, while
in DGP4, one half of the competing models outperform the benchmark model and
the other half underperform. In the rest of the DGPs, the competing models
all outperform the benchmark model. These particular DGPs are also examined
in JCS (2017), and are utilized in our experiments because they illustrate
the trade-offs associated with using $JCS_{n}$ and $S_{n}$ forecast
superiority tests.

We now discuss the experimental findings gathered in Tables 1 and 2. All
reported results are rejection frequencies based on carrying out the $%
JCS_{n} $ and $S_{n}$ tests using a nominal size equal to 0.1. Turning first
to Table 1, note that results in this table mirror those found in JCS
(2017), as the table contains findings based on $JCS_{n}$ type tests. The
test is well sized, under DGPs 1 and 2, is somewhat undersized under DGP3
when used to test GL forecast superiority, and is quite undersized under
DGP3 when used to test CL\ forecast superiority. Just as importantly, the
power of the GL forecast superiority version of the $JCS_{n}$ test is as low
as 0.382 (under DGP4, when $n=250$). The analogous power of the CL forecast
superiority test is 0.639. Note that analogous rejection frequencies for the 
$S_{n}$ type test are 0.958 and 0.961 (see Table 2). On the other hand, for
our largest sample size of $n=1000$, both types of tests have very good
empirical power (compare Tables 1 and 2). Thus, at least for relatively
small samples, there appears to be a marked improvement in finite sample
empirical power when using $S_{n}$ type tests for checking forecast
superiority, relative to $JCS_{n}$ type tests. That said, it should also be
pointed out that the $S_{n}$ type tests are clearly quite sensitive to the
choice of $\eta $. When $\eta $ is \textquotedblleft too
small\textquotedblright , the tests are quite oversized (see Table 2). The
empirical size of $S_{n}$ type tests appears \textquotedblleft
best\textquotedblright\ when $\eta $ is approximately in the range $%
[0.075,0.090].$ For this reason, in our empirical analysis, we only report
findings for $\eta =0.075$ and $0.090.$\footnote{%
For a discussion of simulation results based on application of the Diebold
and Mariano (DM: 1995) test (in which specific loss functions are utilized)
in our experimental setup, when applied using $JCS_{n}$ type tests, refer to
JCS(2017). Summarizing from that paper, it is clear that when the loss
function is unknown, there is an advantage to using our approach of testing
for forecast superiority. However the DM test for pairwise comparison or a
reality check\ test for multiple comparisons might yield improved power, for
a given loss function. Indeed, under quadratic loss, JCS (2017) show that
when the sample size is small, the DM\ test has better power performance
than $JCS_{n}$ type tests.\ When the sample size increases, the power
difference between the two tests becomes smaller. This is as expected.}

\section{Robust Forecast Evaluation of SPF Expert Pools}

In the real-time forecasting literature, predictions from econometric models
are often compared with surveys of expert forecasters.\footnote{%
See Fair and Shiller (1990), Swanson and White (1997a,b), Aiolfi, Capistr%
\'{a}n and Timmermann (2011), and the references cited therein for further
discussion.} Such comparisons are important when assessing the implications
associated with using econometric models in policy setting contexts, for
example. One key survey dataset collecting expert predictions is the \textit{%
Survey of Professional Forecasters} (SPF), which is maintained by the
Philadelphia Federal Reserve Bank (see Croushore (1993)). This dataset,
formerly known as the \textit{American Statistical Association/National
Bureau of Economic Research Economic Outlook Survey, }collects predictions
on various key economic indicators (including, for example, nominal GDP
growth, real GDP growth, prices, unemployment, and industrial production).
For further discussion of the variables contained in the SPF, refer to
Croushore (1993) and Aiolfi, Capistr\'{a}n, and Timmermann (2011). The SPF
has been examined in numerous papers in recent decades. For example,
Zarnowitz and Braun (1992) comprehensively study the SPF, and find, among
other things, that use of the mean or median provides a consensus forecast
with lower average errors than most individual forecasts. More recently,
Aiolfi, Capistr\'{a}n, and Timmermann (2011) consider combinations of the
subjective SPF survey forecasts, and find that equal weighted averages of
survey forecasts outperform model based forecasts, although in some cases
these mean forecasts can be improved upon by averaging them with mean
econometric model-based forecasts. When utilizing European data from the
recently released ECB SPF, Genre, Kenny, Meyler, and Timmermann (2013) again
find that it is very difficult to beat the simple average. This well known
result pervades the macroeconometric forecasting literature, and reasons for
the success of such simple forecast averaging are discussed in Timmermann
(2006). For example, Timmermann notes that model misspecification related to
instability (non-stationarities) and estimation error in situations where
there are many models and relatively few observations may account to some
degree for the success of simple forecast and model averaging.

In this section, we address the issue of forecast averaging and combination
by viewing the problem through the lens of forecast superiority testing. In
particular, we argue that our loss function free superiority tests can be
used to shed additional light on the reasons for the success of simple
averaging methods. The primary motivation for our analysis is that the
majority of papers in this field focus on specific loss functions. We
instead use the forecast superiority tests discussed above. Our approach of
using loss function robust tests differs from the approach taken by Elliott,
Timmermann, and Komunjer (2005,2008), where the rationality of sequences of
forecasts are evaluated by determining whether there exists a particular
loss function under which the forecasts are rational. We instead evaluate
predictive accuracy irrespective of the loss function implicitly used by the
forecaster, and determine whether certain forecast combinations are superior
when compared against any loss function, regardless of how the forecasts
were constructed. In our tests, the benchmarks against which we compare
various forecast combinations are simple average and median consensus
forecasts. We aim to assess whether the well documented success of these
benchmark combinations remains intact when they are compared against other
combinations, under generic loss.\footnote{%
For an interesting discussion of machine learning and forecast combination
methods, see Lahiri, Peng, and Zhao (2017); and for a discussion of
probability forecasting and calibrated combining using the SPF, see Lahiri,
Peng, and Zhao (2015). In these papers, various cases where consensus
combinations do not \textquotedblleft win\textquotedblright\ are discussed.}
To this end, we examine expert SPF predictions of nominal GDP growth$.$

The remainder of this section is divided into three subsections, including:
(i) data, (ii) empirical setup and combination methods, and (iii) empirical
findings.

\subsection{SPF Dataset}

The SPF is a quarterly survey, and the dataset is available at the
Philadelphia Federal Reserve Bank (PFRB) website. The original survey began
in 1968:Q4, and PFRB took control of it in 1990:Q2; but from that date,
there are only around 100 quarterly observations. In our analysis we use the
entire dataset of over 160 observations. However, it should be noted that
the timing of the survey was not known with certainty prior to 1990. Still,
PFRB documentation states that they believe, although are not sure, that the
timing of the survey was similar before and after they took control of it.

For our analysis, we consider 5 forecast horizons (i.e., $h=0,1,2,3,4).$ The
reason we use $h=0$ for one of the horizons is that the first horizon for
which survey participants predict GDP growth is the quarter in which they
are making their predictions. In light of this, forecasts made at $h=0$ are
called nowcasts. Nowcasts are very important in policy making settings,
since first release GDP data are not available until around the middle of
the subsequent quarter. The nominal GDP\ variable that we examine is called
NGDP in the SPF.\textit{\ }All test statistics constructed in this section
are based on NGDP growth rate prediction errors associated. In particular,
assume that one survey participant makes a forecast of NGDP, say $%
y_{t+h}^{f}|\mathcal{F}_{t}$.\footnote{%
Here, $\mathcal{F}_{t}$ denotes the information set available to the expert
forecaster at the time their predictions are made.} The associated forecast
error is: 
\begin{equation*}
e_{t}=\left\{ \ln (y_{t+h})-\ln (y_{t})\right\} -\left\{ \ln (y_{t+h}^{f}|%
\mathcal{F}_{t})-\ln (y_{t})\right\} =\ln (y_{t+h})-\ln (y_{t+h}^{f}|%
\mathcal{F}_{t}),
\end{equation*}%
where the actual NGDP value, $y_{t+h},$ is reported in the SPF, along with
the NGDP predictions of each survey participant. Note that when $h=0$, $%
\mathcal{F}_{t}$ does not include $y_{t}.$ However, for $h>0$, $\mathcal{F}%
_{t}$ includes $y_{t}$. As discussed previously, this is due to the release
dates associated with the availability of NGDP\ data.

\subsection{Empirical Setup and Combination Methods}

When construct all $S_{n}$ type test statistics, including $%
S_{n}^{G+},S_{n}^{G-},S_{n}^{C-},$ and $S_{n}^{C+}$. In particular, GL\
forecast superiority is tested using $S_{n}^{G+}$ and $S_{n}^{G-}$
statistics; while CL\ forecast superiority is tested using $S_{n}^{C+}$ and $%
S_{n}^{C-}$ statistics$.$ We also test for forecast superiority using the $%
JCS_{n}$ type tests discussed above, which are not uniformly valid and have
correct size only under the least favorable case under the null. In
particular, we construct $JCS_{n}^{G+},JCS_{n}^{G-},JCS_{n}^{C-},$ and $%
JCS_{n}^{C+}$ test statistics (see Section 2 for further details). All test
statistics are calculated using the same parameter values (for $B,$ $J_{n}$, 
$\eta ,$ $l_{n},$ and $\varepsilon )$ as used in our Monte Carlo
experiments, although results are only reported for $J_{n}=\{0.20,0.35\}$
and $\eta =\{0.075,0.090\}$, as these are the values that yielded the best
results in our Monte Carlo experiments.

Two different benchmark models are considered, including (i) the arithmetic
mean prediction from all participants; and (ii) the median prediction from
all participants. Additionally, a variety of alternative combinations are
considered.

The first group of alternative combinations allows us to answer the
question: \textit{Does experience matter?} Combinations in this group
include:

\noindent $\circ $ Combination 1a: Mean (or Median) of all participants with
at least 1 year of experience.

\noindent $\circ $ Combination 1b: Mean (or Median) of all participants with
at least 3 years of experience.

\noindent $\circ $ Combination 1c: Mean (or Median) of all participants with
at least 5 years of experience.

In all of the remaining groups of combinations, individuals are ranked
according to average absolute forecast errors, as well as according to
average squared forecast errors. Mean (or median) predictions from these
groups are then compared with our benchmark combinations.

In the first such set of combinations, we ask the following question: 
\textit{Does the most highly ranked expert exhibit forecast superiority,
when compared with our benchmark combinations, under various experience
levels?} This leads to the following combinations:

\noindent $\circ $ Combination 2a: Highest ranked participant in last 1 year.

\noindent $\circ $ Combination 2b: Highest ranked participant in last 3
years.

\noindent $\circ $ Combination 2c: Highest ranked participant in last 5
years.

We then ask the following question: \textit{Does the most highly ranked
group of 3 experts exhibit forecast superiority, when compared with our
benchmark combinations, under various experience levels?} This leads to the
following combinations:

\noindent $\circ $ Combination 3a: Highest ranked group of 3 participants in
last 1 year.

\noindent $\circ $ Combination 3b: Highest ranked group of 3 participants in
last 3 years.

\noindent $\circ $ Combination 3c: Highest ranked group of 3 participants in
last 5 years.

We next ask the following question: \textit{Do groups including the top 10\%
of participants, exhibit forecast superiority, when compared with our
benchmark combinations, under various experience levels?} This leads to the
following combinations:

\noindent $\circ $ Combination 4a: Group of top 10\% of participants in last
1 year.

\noindent $\circ $ Combination 4b: Group of top 10\% of participants in last
3 years.

\noindent $\circ $ Combination 4c: Group of top 10\% of participants in last
5 years.

Finally, we ask the following question: \textit{Do groups including the top
25\% of participants, exhibit forecast superiority, when compared with our
benchmark combinations, under various experience levels?} This leads to the
following combinations:

\noindent $\circ $ Combination 5a: Group of top 25\% of participants in last
1 year.

\noindent $\circ $ Combination 5b: Group of top 25\% of participants in last
3 years.

\noindent $\circ $ Combination 5c: Group of top 25\% of participants in last
5 years.

\subsection{Empirical Findings}

In Panels A-E of Tables 3-8, test statistics are reported under the heading
\textquotedblleft Statistic Values\textquotedblright . Rejection (or not) of
the null hypothesis of equal forecast accuracy is reported by
\textquotedblleft no\textquotedblright\ (indicating rejection) or
\textquotedblleft yes\textquotedblright\ (indicating the converse), under
the headers \textquotedblleft Rejection with $p$-Value Type
I\textquotedblright\ (for $\eta $ = 0.075 and $J_{n}$ = 0.20) or
\textquotedblleft Rejection with $p$-Value Type II\textquotedblright\ (for $%
\eta $ = 0.090 and $J_{n}$ = 0.35). Nominal test size is 10\%. In Panel F of
the tables, root mean square forecast errors (RMSFEs) are reported for all
of the combinations evaluated.

In the remainder of this subsection, we discuss our findings by answering
the questions posed above. Due to the large volume of output associated with
carrying out our analysis, most results are contained only in an online
appendix. Here, we discuss selected empirical results that are presented in
Tables 2-8. However, the conclusions that we draw are the qualitatively the
same as those associated with inspection of our entire set of empirical
results. \medskip

\noindent $\circ $ \textit{Does experience matter?\medskip }

Inspection of the results reported in Table 3 indicates that the answer to
this question is no. The forecast superiority null hypothesis fails to
reject in all but one case (i.e., $h=1,$ for $S_{n}$ type GL forecast
superiority). Namely, for nowcasting ($h=0$) as well as forecasting ($%
h=1,2,3,4)$, a benchmark that utilizes all predictions from all survey
participants is not dominated by pools of experts chosen based on whether
they have at least 1, 3, or 5 years of experience. Thus, when compared only
with experienced pools of experts, the \textquotedblleft
best\textquotedblright\ pool contains all experts, regardless of experience.
As stressed elsewhere in this paper, this result is robust to the choice of
loss function. For the time being, then, our analysis tends to support
utilizing predictions that take advantage of the largest pool of experts
possible, in agreement with the findings of Aiolfi, Capistr\'{a}n, and
Timmermann (2011) and Genre, Kenny, Meyler, and Timmermann (2013), for
example.

Additionally, it is worth noting that loss functions do matter, as should be
expected. This can be seen by noting that some alternative models are
slightly better the entire pool of experts, when comparing quadratic loss
(see the RMSFEs in Panel F of the table). For example, \textquotedblleft Alt
Model 1\textquotedblright ,\ which is Combination 1a above, and which pools
all experts with at least 1 year of experience, yields slightly lower RMSFEs
than the benchmark, for all values of $h.$ On the other hand, results for
\textquotedblleft Alt Model 2\textquotedblright\ (i.e., Combination 1b) are
mixed, with the benchmark winning for some values of $h$, while pooling
experts with at least 5 years of experience (i.e., \textquotedblleft Alt
Model 3\textquotedblright , which is Combination 1c) never results in a
lower RMSFE than the arithmetic mean benchmark, regardless of $h$.\footnote{%
In Tables 3-4, \textquotedblleft Alt Models 1-3\textquotedblright\
correspond to Combinations 1a-c, respectively. Also, in Table $x$ ,
\textquotedblleft Alt Model\textquotedblright\ 1-3 corresponds to
Combinations $x$a-c, for $x$=5,6,7,8.}

Finally, it is worth noting that when the benchmark is the median rather
than mean forecast, and when the alternative combinations are selectively
pooled median forecasts, results are the same as those reported above (see
Table 4). Namely, the forecast superiority null hypothesis fails to reject
in all cases (i.e., for all values of $h,$ and for both $S_{n}$ and $JCS_{n}$
type GL and CL tests).\medskip

\noindent $\circ $ \textit{Does the most highly ranked expert exhibit
forecast superiority, when compared with our benchmark combinations, under
various experience levels?\medskip }

Turning to Table 5, results when the alternative models are Combinations
2a-c are largely the same as those reported above. Namely, the forecast
superiority null hypothesis fails to reject in 37 of 40 cases. Thus, the
entire pool of participants is preferred to the single \textquotedblleft
best\textquotedblright\ participant, regardless of experience level. Recall
that in this context, \textquotedblleft best\textquotedblright\ refers to
the lowest average least absolute deviation forecast error participant based
on either 1, 3, or 5 years of experience. However, results based on ranking
according to square forecast error performance rather than absolute forecast
error performance are the same as those reported in Table 5. The reason why
we report results from experiments where participant rankings are done using
absolute forecast error performance is that RMSFEs of all of our alternative
combinations are lower when ranking is done according to absolute forecast
error performance than when ranking is done according to square forecast
error performance. For this reason, and because our qualitative findings do
not change based on ranking method, we report subsequent results only for
absolute forecast error rankings.\medskip

\noindent $\circ $ \textit{Does the most highly ranked group of 3 experts
exhibit forecast superiority, when compared with our benchmark combinations,
under various experience levels? \medskip }

The story changes when the alternative models are Combinations 3a-c. In
particular, in Table 6 note that the null of equal forecast accuracy is
rejected in many cases, when 3 experts comprise the pool. For example, the
null hypothesis is rejected in 14 of 20 cases, across all values of $h,$
when $S_{n\text{ }}$statistics are used to test forecast superiority. The
same cannot be said for $JCS_{n}$ type GL tests (rejection occurs in only 8
of 20 cases), although this might be expected given the lower power of $%
JCS_{n}$ type tests when $n$ is small. Evidence based on analysis of the
RMSFEs in Panel F if the table suggests that for $h=0$, more than 1 year of
experience is preferred, while for all other values of $h$, 1 year of
experience is enough, and delivers a pool of 3 experts that outperforms the
entire pool of experts. Thus, we have direct evidence that experience
coupled with selecting a small group of the very best experts leads to loss
function robust forecast superiority, in many cases. This finding is in
contrast to findings discussed above where the entire pool yield
combinations that are preferred. However, our finding is in broad agreement
with Lahiri, Peng, and Zhao (2015), who find that using \textquotedblleft
valuable\textquotedblright\ individual forecasts leads to predictions that
outperform simple average predictions.\footnote{%
Interestingly, Lahiri, Peng, and Zhao (2015) find that the numbers of
forecasters making valuable predictions dimishes as $h$ increases. Our
results, which are loss function robust, do not make this distinction.}%
\medskip

\noindent $\circ $ \textit{Do groups including the top 10\% of participants,
exhibit forecast superiority, when compared with our benchmark combinations,
under various experience levels?\medskip\ }

Results based on Combinations 4a-c are reported in Table 7, and again
indicate various rejections of the null of equal forecast accuracy. Namely,
we observe rejections in 12 of 20 cases, across all values of $h,$ when $%
S_{n}$ statistics are used to test forecast superiority. Here, the incidence
of rejection of the null is not quite as pervasive as under Combinations
3a-c. Apparently, a 10\% cut-off for the number of experts is not quite
enough to see the benefits associated with utilizing experienced expert
pools.\medskip

\noindent $\circ $ \textit{Do groups including the top 25\% of participants,
exhibit forecast superiority, when compared with our benchmark combinations,
under various experience levels?}\medskip

However, Combinations 5a-c, which utilize the top 25\% of experts again lead
to rejection of the null hypothesis in 14 of 20 cases, across all values of $%
h,$ when $S_{n\text{ }}$statistics are used to test forecast superiority. Of
course, this finding is clearly dependent upon the number of participants in
the survey. Nevertheless, it is interesting to further note that the 25\%
cut-off leads to rejection in 7 of 10 cases when $S_{n\text{ }}$type GL
forecast superiority is tested for, as well as in 7 of 10 cases when $S_{n%
\text{ }}$type CL forecast superiority is tested for. Evidence based on
analysis of the RMSFEs in Panel F of the table suggest that for $h=0$, all
levels of experience (i.e., 1, 3, or 5 years of experience) yield improved
forecast performance, while for all other values of $h$, requiring at least
3 year of experience is needed in order to deliver a pool of
\textquotedblleft top 25\%\textquotedblright\ experts that outperforms the
entire pool of experts.

Summarizing, we have direct evidence that judicious selection of a pool of
the very best experienced experts can lead to loss function robust forecast
superiority.

\section{Concluding Remarks}

We develop uniformly valid forecast superiority tests that are
asymptotically correctly sized, and that are robust to the choice of loss
function. Our tests are based on principles of stochastic dominance, which
can be interpreted as tests for infinitely many moment inequalities. In
light of this, we use tools from Andrews and Shi (2013, 2017) when
developing our tests. The tests build on earlier work due to Jin, Corradi,
and Swanson (2017), and are meant to provide a class of predictive accuracy
tests that are not reliant on a choice of loss function, such as the Diebold
and Mariano (1995) test discussed in McCracken (2000). In developing the new
tests, we establish uniform convergence (over error support) of HAC variance
estimators, and of their bootstrap counterparts. We also extend the theory
of generalized moment selection testing to allow for the presence of
non-vanishing parameter estimation error. In a series of Monte Carlo
experiments, we show that finite sample performance of our tests is quite
good, and that the power of our tests dominates those proposed by JCS
(2017). Additionally, we carry out an extensive empirical analysis of the
well known Survey of Professional Forecasters, and show that utilizing
expert pools based on past forecast quality as well as years of experience
can lead to loss function robust forecast superiority, when compared with
pools that include all survey participants. \pagebreak

\section{Appendix}

$\boldsymbol{\noindent }$\textbf{Proof of Lemma 1: (i)} The proof is the
same for all $j.$ Thus, let $u_{t}(x)=\left( 1\left\{ e_{j,t}\leq x\right\}
-F_{j}(x)\right) -\left( 1\left\{ e_{1,t}\leq x\right\} -F_{1}(x)\right) ,$
and define 
\begin{equation*}
\widehat{\widehat{\sigma }}_{n}^{2,G+}(x)=\frac{1}{n}%
\sum_{t=1}^{n}u_{t}^{2}(x)+2\frac{1}{n}\sum_{\tau =1}^{l_{n}}w_{\tau
}u_{t}(x)u_{t-\tau }(x).
\end{equation*}%
We first show that%
\begin{equation*}
\sup_{x\in \mathcal{X}^{+}}\left\vert \widehat{\widehat{\sigma }}%
_{n}^{2,G+}(x)-\sigma ^{2,G+}(x)\right\vert =o_{p}\left( 1\right) ,
\end{equation*}%
and then we show that 
\begin{equation}
\sup_{x\in \mathcal{X}^{+}}\left\vert \widehat{\widehat{\sigma }}%
_{n}^{2,G+}(x)-\widehat{\sigma }_{n}^{2,G+}(x)\right\vert =o_{p}\left(
1\right) .  \label{hat-hat}
\end{equation}%
Now,%
\begin{eqnarray}
&&\sup_{x\in \mathcal{X}^{+}}\left\vert \widehat{\widehat{\sigma }}%
_{n}^{2,G+}(x)-\sigma ^{2,G+}(x)\right\vert  \notag \\
&\leq &\sup_{x\in \mathcal{X}^{+}}\left\vert \frac{1}{n}\sum_{t=1}^{n}\left(
u_{t}^{2}(x)-\mathrm{E}\left( u_{t}^{2}(x)\right) \right) +2\frac{1}{n}%
\sum_{\tau =1}^{l_{n}}w_{\tau }\sum_{t=1}^{n}\left( u_{t}(x)u_{t-\tau }(x)-%
\mathrm{E}\left( u_{t}(x)u_{t-\tau }(x)\right) \right) \right\vert  \notag \\
&&+\sup_{x\in \mathcal{X}^{+}}\left\vert \left( \sigma ^{2}(x)-\frac{1}{n}%
\sum_{t=1}^{n}\mathrm{E}\left( u_{t}^{2}(x)\right) +2\frac{1}{n}\sum_{\tau
=1}^{l_{n}}w_{\tau }\sum_{t=1}^{n}\mathrm{E}\left( u_{t}(x)u_{t-\tau
}(x)\right) \right) \right\vert .  \label{main}
\end{eqnarray}%
We begin with the first term on the RHS of (\ref{main}). First note that,%
\begin{eqnarray*}
&&\sup_{x\in \mathcal{X}^{+}}\left\vert \frac{1}{n}\sum_{t=1}^{n}\left(
u_{t}^{2}(x)-\mathrm{E}\left( u_{t}^{2}(x)\right) \right) +2\frac{1}{n}%
\sum_{\tau =1}^{l_{n}}w_{\tau }\sum_{t=1}^{n}\left( u_{t}(x)u_{t-\tau }(x)-%
\mathrm{E}\left( u_{t}(x)u_{t-\tau }(x)\right) \right) \right\vert \\
&\leq &\sup_{x\in \mathcal{X}^{+}}2\sum_{\tau =0}^{l_{n}}\left\vert \frac{1}{%
n}\sum_{t=1}^{n}\left( u_{t}(x)u_{t-\tau }(x)-\mathrm{E}\left(
u_{t}(x)u_{t-\tau }(x)\right) \right) \right\vert .
\end{eqnarray*}%
Now,%
\begin{eqnarray*}
&&\Pr \left( \sup_{x\in \mathcal{X}^{+}}2\sum_{\tau =0}^{l_{n}}\left\vert 
\frac{1}{n}\sum_{t=1}^{n}\left( u_{t}(x)u_{t-\tau }(x)-\mathrm{E}\left(
u_{t}(x)u_{t-\tau }(x)\right) \right) \right\vert \geq \varepsilon \right) \\
&\leq &2\sum_{\tau =0}^{l_{n}}\Pr \left( \sup_{x\in \mathcal{X}%
^{+}}\left\vert \frac{1}{n}\sum_{t=1}^{n}\left( u_{t}(x)u_{t-\tau }(x)-%
\mathrm{E}\left( u_{t}(x)u_{t-\tau }(x)\right) \right) \right\vert \geq 
\frac{\varepsilon }{l_{n}}\right) ,
\end{eqnarray*}%
so that we need to show that,%
\begin{equation*}
\Pr \left( \sup_{x\in \mathcal{X}^{+}}\left\vert \frac{1}{n}%
\sum_{t=1}^{n}\left( u_{t}(x)u_{t-\tau }(x)-\mathrm{E}\left(
u_{t}(x)u_{t-\tau }(x)\right) \right) \right\vert \geq \frac{\varepsilon }{%
l_{n}}\right) <\frac{\delta }{l_{n}}.
\end{equation*}%
Given Assumption A2, WLOG, we can set $\mathcal{X}^{+}=[0,\Delta ],$ so that
it can be covered by $a_{n}^{-1}$ balls $S_{j},$ $j=1,...,\Delta a_{n}^{-1},$
centered at $S_{j},$ with radius $a_{n}.$ Then,%
\begin{eqnarray*}
&&\sup_{x\in \mathcal{X}^{+}}\left\vert \frac{1}{n}\sum_{t=1}^{n}\left(
u_{t}(x)u_{t-\tau }(x)-\mathrm{E}\left( u_{t}(x)u_{t-\tau }(x)\right)
\right) \right\vert \\
&\leq &\max_{j=1,..,\Delta a_{n}^{-1}}\left\vert \frac{1}{n}%
\sum_{t=1}^{n}\left( u_{t}(s_{j})u_{t-\tau }(s_{j})-\mathrm{E}\left(
u_{t}(s_{j})u_{t-\tau }(s_{j})\right) \right) \right\vert \\
&&+\max_{j=1,..,\Delta a_{n}^{-1}}\sup_{x\in S_{j}}2\left\vert \left( \frac{1%
}{n}\sum_{t=1}^{n}u_{t-\tau }(s_{j})\left( u_{t}(x)-u_{t}(s_{j})\right)
\right) \right. \\
&&-\left. \left( \frac{1}{n}\sum_{t=1}^{n}\mathrm{E}\left( u_{t-\tau
}(s_{j})\left( u_{t}(x)-u_{t}(s_{j})\right) \right) \right) \right\vert \\
&&+\text{smaller order} \\
&=&I_{n}+II_{n}.
\end{eqnarray*}%
Now,

\begin{eqnarray*}
II_{n} &\leq &\max_{j=1,..,\Delta a_{n}^{-1}}\sup_{x\in S_{j}}\left\vert 
\frac{1}{n}\sum_{t=1}^{n}u_{t-\tau }(s_{j})\left(
u_{t}(x)-u_{t}(s_{j})\right) \right\vert  \\
&&+\max_{j=1,..,\Delta a_{n}^{-1}}\sup_{x\in S_{j}}\left\vert \frac{1}{n}%
\sum_{t=1}^{n}\mathrm{E}\left( u_{t-\tau }(s_{j})\left(
u_{t}(x)-u_{t}(s_{j})\right) \right) \right\vert  \\
&=&II_{n}^{A}+II_{n}^{B}.
\end{eqnarray*}%
Given Assumption A1, noting that by Cauchy  - Schwarz,%
\begin{eqnarray*}
&&\max_{j=1,..,\Delta a_{n}^{-1}}\sup_{x\in S_{j}}\left\vert \frac{1}{n}%
\sum_{t=1}^{n}\mathrm{E}\left( u_{t-\tau }(s_{j})\left(
u_{t}(s_{j})-u_{t-\tau }(s_{j})\right) \right) \right\vert  \\
&\leq &\max_{j=1,..,\Delta a_{n}^{-1}}\sup_{x\in S_{j}}\sqrt{\mathrm{E}%
\left( u_{t-\tau }(s_{j})\right) ^{2}}\max_{j=1,..,a_{n}^{-1}}\sup_{x\in
S_{j}}\sqrt{\mathrm{E}\left( u_{t}(s_{j})-u_{t}(x)\right) ^{2}} \\
&=&O\left( a_{n}^{1/2}\right) ,
\end{eqnarray*}%
for some constant $C.$ Recalling given that $u_{t}(x)=\left( 1\left\{
e_{j,t}\leq x\right\} -F_{j}(x)\right) -\left( 1\left\{ e_{1,t}\leq
x\right\} -F_{1}(x)\right) $ and $u_{t}(s_{j})$ stay between $-\NEG{1}$ and $%
1$%
\begin{eqnarray*}
&&\max_{j=1,..,\Delta a_{n}^{-1}}\sup_{x\in S_{j}}\left\vert \frac{1}{n}%
\sum_{t=1}^{n}u_{t-\tau }(s_{j})\left( u_{t}(s_{j})-u_{t}(x)\right)
\right\vert  \\
&\leq &2\max_{j=1,..,\Delta a_{n}^{-1}}\sup_{x\in S_{j}}\frac{1}{n}%
\sum_{t=1}^{n}\left\vert u_{t}(s_{j})-u_{t}(x)\right\vert  \\
&\leq &\frac{2}{n}\sum_{t=1}^{n}1\left\{ x-a_{n}\leq e_{1,t}\leq
x+a_{n}\right\} +\frac{2}{n}\sum_{t=1}^{n}1\left\{ x-a_{n}\leq e_{j,t}\leq
x+a_{n}\right\}  \\
&&+2\sup_{x\in \mathcal{X}^{+}}\left( f_{1}(x)+f_{j}(x)\right)  \\
&=&O_{p}\left( a_{n}\right) =o_{p}(a_{n}^{1/2})
\end{eqnarray*}%
Hence, by Chebyshev inequality%
\begin{equation*}
l_{n}\Pr \left( II_{n}>\frac{\varepsilon }{l_{n}}\right) =O\left(
a_{n}l_{n}^{3}\right) =o(1),
\end{equation*}%
for $a_{n}=o\left( l_{n}^{-3}\right) .$

Now, consider $I_{n}.$ By the Lemma on page 739 of Hansen (2008), setting $%
a_{n}=l_{n}^{-4},$ $m=\frac{\Delta n}{4l_{n}^{2}},$ and $l_{n}=n^{\delta },$
with $\delta <1/2,$ and recalling that given Assumption A1, $\mathrm{var}%
\left( \sum_{t=1}^{m}\left( u_{t}(s_{j})u_{t-\tau }(s_{j})-\mathrm{E}\left(
u_{t}(s_{j})u_{t-\tau }(s_{j})\right) \right) \right) \leq Cm,$ it follows
that for some constant $C,$

\begin{eqnarray*}
&&\Pr \left( \max_{j=1,..,a_{n}^{-1}}\left\vert \frac{1}{n}%
\sum_{t=1}^{n}\left( u_{t}(s_{j})u_{t-\tau }(s_{j})-\mathrm{E}\left(
u_{t}(s_{j})u_{t-\tau }(s_{j})\right) \right) \right\vert \geq \frac{%
\varepsilon }{l_{n}}\right) \\
&\leq &a_{n}^{-1}\Pr \left( \left\vert \sum_{t=1}^{n}\left(
u_{t}(s_{j})u_{t-\tau }(s_{j})-\mathrm{E}\left( u_{t}(s_{j})u_{t-\tau
}(s_{j})\right) \right) \right\vert \geq \frac{n\varepsilon }{l_{n}}\right)
\\
&\leq &4a_{n}^{-1}\left( \exp \left( -\frac{\frac{n^{2}}{l_{n}^{2}}%
\varepsilon ^{2}}{64Cn+\frac{8}{3}\frac{\Delta n^{2}}{4l_{n}^{3}}}\right) +%
\frac{16}{b}l_{n}^{2}\left( \frac{4}{\Delta }\frac{n}{l_{n}^{2}}\right)
^{-\beta }\right) \\
&=&a_{n}^{-1}\exp \left( -\frac{1}{64C\frac{n}{l_{n}^{2}}+\frac{8}{3}\frac{%
\Delta n^{2}}{4l_{n}^{3}}}\right) +\frac{64}{b}a_{n}^{-1}l_{n}^{2}\NEG%
{l}_{n}^{2\beta }n^{-\beta } \\
&=&o(1)+O\left( n^{\delta \left( 6+2\beta \right) }n^{-\beta }\right) \\
&=&o(1)\text{ for }\beta >\frac{6\delta }{1-2\delta }.
\end{eqnarray*}

We now consider the second term on the RHS of (\ref{main}). Note that%
\begin{eqnarray}
&&\sup_{x\in \mathcal{X}^{+}}\left\vert \left( \sigma ^{2,G+}(x)-\frac{1}{n}%
\sum_{t=1}^{n}\mathrm{E}\left( u_{t}^{2}(x)\right) +2\frac{1}{n}\sum_{\tau
=1}^{l_{n}}w_{\tau }\sum_{t=1}^{n}\mathrm{E}\left( u_{t}(x)u_{t-\tau
}(x)\right) \right) \right\vert  \notag \\
&\leq &2\sup_{x\in \mathcal{X}^{+}}\left\vert \frac{1}{n}\sum_{\tau
=1}^{l_{n}}\left( 1-w_{\tau }\right) \sum_{t=1}^{n}\mathrm{E}\left(
u_{t}(x)u_{t-\tau }(x)\right) \right\vert  \label{NW} \\
&&+2\sup_{x\in \mathcal{X}^{+}}\left\vert \frac{1}{n}\sum_{\tau
=l_{n}+1}^{n}w_{\tau }\sum_{t=1}^{n}\mathrm{E}\left( u_{t}(x)u_{t-\tau
}(x)\right) \right\vert .  \notag
\end{eqnarray}%
The first term on the RHS of (\ref{NW}) is $o_{p}(1),$ by the same argument
as that used in Theorem 2 of Newey and West (1987). Also, by Lemma 6.17 in
White (1984), for $q>2,$%
\begin{equation*}
\mathrm{E}\left( u_{t}(x)u_{t-\tau }(x)\right) \leq C\tau ^{-\beta /2-1/q}%
\mathrm{var}\left( u_{t}(x)\right) ^{1/2}\mathrm{E}\left\Vert
u_{t}(x)\right\Vert ^{q}
\end{equation*}%
and 
\begin{eqnarray*}
&&\sup_{x\in \mathcal{X}^{+}}\left\vert \frac{1}{n}\sum_{\tau
=l_{n}+1}^{n}w_{\tau }\sum_{t=1}^{n}\mathrm{E}\left( u_{t}(x)u_{t-\tau
}(x)\right) \right\vert \\
&\leq &C\sup_{x\in \mathcal{X}^{+}}\mathrm{var}\left( u_{t}(x)\right) ^{1/2}%
\mathrm{E}\left\Vert u_{t}(x)\right\Vert ^{q}\sum_{\tau =l_{n}+1}^{n}\tau
^{-\beta /2-1/q}=o(1),
\end{eqnarray*}%
as $\beta \delta >1,$ given Assumption A1, and noting that $q$ can be taken
arbitrarily large because of the boundedness of $u_{t}(x).$

Finally, by the same argument as that used in the proof of (\ref{main}), for
all $j,$%
\begin{equation*}
\sup_{x\in \mathcal{X}^{+}}\frac{1}{n}\sum_{t=1}^{n}\left( 1\left\{
e_{j,t}\leq x\right\} -F_{j}(x)\right) =o_{p}\left( l_{n}^{-1}\right) .
\end{equation*}%
The statement in (\ref{hat-hat}) follows immediately.

\medskip

\noindent \textbf{(ii)} By noting that,%
\begin{eqnarray*}
&&\left[ e_{j,t}-s_{j}\right] _{+}-\left[ e_{j,t}-x\right] _{+} \\
&=&(x-s_{j})1\{e_{t}\geq x\}+(x-s_{j})\left( 1\{e_{t}\geq x\}-1\{e_{t}\geq
s_{j}\}\right) \\
&&+\left( e_{t}-x\right) \left( 1\{e_{t}\geq s_{j}\}-1\{e_{t}\geq x\}\right)
,
\end{eqnarray*}%
the statement follows by the same argument as that used in part (i) of the
proof.

\bigskip

\noindent $\boldsymbol{\noindent }$\textbf{Proof of Lemma 2: }For notational
simplicity, we suppress the $jj$ subscript. Also, we suppress the
superscripts $C^{+}$ and $G^{+},$ as the proof follows by analogous
argument. Note that%
\begin{eqnarray*}
&&\sup_{x\in \mathcal{X}^{+}}\left\vert \widehat{\sigma }_{n}^{\ast 2}(x)-%
\mathrm{E}^{\ast }\left( \widehat{\sigma }_{n}^{\ast }(x)\right) \right\vert
\\
&\leq &\sup_{x\in \mathcal{X}^{+}}\frac{l_{n}}{b}\sum_{k=1}^{b}\left\vert
\left( \frac{1}{l_{n}}\sum_{j=1}^{l_{n}}u_{(k-1)l_{n}+j}^{\ast }(x)\right)
^{2}-\mathrm{E}^{\ast }\left( \left( \frac{1}{l_{n}}%
\sum_{j=1}^{l_{n}}u_{(k-1)l_{n}+j}^{\ast }(x)\right) ^{2}\right) \right\vert
\\
&=&\sup_{x\in \mathcal{X}^{+}}\frac{l_{n}}{b}\sum_{k=1}^{b}\left\vert \frac{1%
}{l_{n}^{2}}\sum_{j=1}^{l_{n}}\sum_{i=1}^{l_{n}}u_{(k-1)l_{n}+j}^{\ast
}(x)u_{(k-1)l_{n}+i}^{\ast }(x)-\mathrm{E}^{\ast }\left(
u_{(k-1)l_{n}+j}^{\ast }(x)u_{(k-1)l_{n}+i}^{\ast }(x)\right) \right\vert
\end{eqnarray*}%
Now,%
\begin{eqnarray*}
&&\Pr \left( \sup_{x\in \mathcal{X}^{+}}\frac{l_{n}}{b}\sum_{k=1}^{b}\left%
\vert \frac{1}{l_{n}^{2}}\sum_{j=1}^{l_{n}}%
\sum_{i=1}^{l_{n}}u_{(k-1)l_{n}+j}^{\ast }(x)u_{(k-1)l_{n}+i}^{\ast }(x)-%
\mathrm{E}^{\ast }\left( u_{(k-1)l_{n}+j}^{\ast }(x)u_{(k-1)l_{n}+i}^{\ast
}(x)\right) \right\vert \geq \varepsilon _{1}a_{n}\right) \\
&\leq &l_{n}\Pr \left( \sup_{x\in \mathcal{X}^{+}}\frac{l_{n}}{b}%
\sum_{k=1}^{b}\left\vert \frac{1}{l_{n}^{2}}\sum_{j=1}^{l_{n}}%
\sum_{i=1}^{l_{n}}u_{(k-1)l_{n}+j}^{\ast }(x)u_{(k-1)l_{n}+i}^{\ast }(x)-%
\mathrm{E}^{\ast }\left( u_{(k-1)l_{n}+j}^{\ast }(x)u_{(k-1)l_{n}+i}^{\ast
}(x)\right) \right\vert \geq \varepsilon _{1}\frac{a_{n}}{l_{n}}\right) .
\end{eqnarray*}%
It suffices to show that, uniformly in $k,$%
\begin{equation*}
\Pr \left( \sup_{x\in \mathcal{X}^{+}}\left\vert \frac{1}{l_{n}^{2}}%
\sum_{j=1}^{l_{n}}\sum_{i=1}^{l_{n}}u_{(k-1)l+j}^{\ast
}(x)u_{(k-1)l+i}^{\ast }(x)-\mathrm{E}^{\ast }\left( u_{(k-1)l+j}^{\ast
}(x)u_{(k-1)l+i}^{\ast }(x)\right) \right\vert \geq \varepsilon _{1}\frac{%
a_{n}}{l_{n}}\right) <\frac{\delta }{l_{n}}.
\end{equation*}%
This follows using the same "covering numbers" argument used in the proof of
Lemma 1.

\medskip

\noindent \textbf{Proof of Theorem 1:} We again suppress the superscripts $%
G^{+}$ and $C^{+},$ as the proof follows by the same argument. We need to
show that the statement in Lemma A1 in the Supplement Appendix of Andrews
and Shi (2013) holds. Then, the proof of the theorem will follow using the
same arguments as those used in the proof of their Theorem 1, as the proof
is the same for independent and dependent observations. In fact, our set-up
differs from Andrews and Shi (2013) only because we have dependent
observations, and because we scale the statistic by a Newey-West variance
estimator. For the rest of the proof, our set-up is simpler as we can fix
their $\theta _{n}$ at a given value, say zero. It suffices to show that:

\noindent (i) $v_{n}\left( .\right) \Rightarrow v(.),$ as a process indexed
by $x\in \mathcal{X}^{+},$ where $v(.)$ is a zero-mean $k-1-$dimensional
Gaussian process, with covariance kernel given $\Sigma (x,x^{\prime }).$

\noindent (ii) $\sup_{x,x^{\prime }\in \mathcal{X}^{+}}\left\Vert \overline{h%
}_{B,n}(x,x^{\prime })-\overline{h}_{B}(x,x^{\prime })\right\Vert =o_{p}(1).$

\noindent Now, statement (ii) follows directly from Lemma 1. It remains to
show that (i) holds. The key difference between the independent and the
dependent cases is that in the former we can rely on the concept of
manageability, while in the latter we cannot. Nevertheless, (i) follows if
we can show that $v_{n}\left( .\right) $ satisfies an empirical process.
Given A1-A3, this follows from Lemma A2 in Jin, Corradi and Swanson (2017).

\medskip

\noindent \textbf{Proof of Theorem 2: (i)} For notational simplicity, we
omit the superscript $G^{+}.$ The proof of this theorem mirrors the proof of
Theorem 2(a) in the Supplement of Andrews and Shi (2013). Let $c_{0}\left(
h_{A,n},\alpha \right) $ be the $\alpha $ critical value of $S_{n}^{\dag },$
as defined in (\ref{Sn-dag}). Given Theorem 1(i), it follows that for all $%
\delta >0,$%
\begin{equation*}
\lim \sup_{n\rightarrow \infty }\sup_{P\in \mathcal{P}_{0}}P\left( S_{n}\geq
c_{0}\left( h_{A,n},\alpha \right) +\delta \right) \leq \alpha .
\end{equation*}%
The statement follows if we can show that 
\begin{equation}
\lim \sup_{n\rightarrow \infty }\sup_{P\in \mathcal{P}_{0}}P\left(
c_{0,n,\alpha }^{\ast }\left( \phi _{n},\overline{h}_{B,n}^{\ast }\right)
\leq c_{0,n,\alpha }\left( h_{A,n},\overline{h}_{B,n}^{\ast }\right) \right)
=0,  \label{c0*}
\end{equation}%
with $c_{0,n,\alpha }\left( h_{A,n},\overline{h}_{B,n}^{\ast }\right) $
defined as $c_{0}\left( h_{A,n},\alpha \right) ;$ but with $\overline{h}%
_{B,n}^{\ast }$ an argument of this function rather than $h_{B}(x);$ and if
we can show that 
\begin{equation}
\lim \sup_{n\rightarrow \infty }\sup_{P\in \mathcal{P}_{0}}P\left(
c_{0,n,\alpha }\left( h_{A,n},\overline{h}_{B,n}^{\ast }\right) \leq
c_{0}\left( h_{A,n},\alpha \right) \right) =0.  \label{c0}
\end{equation}%
For $c_{n}\rightarrow \infty $ and $c_{n}/\kappa _{n}\rightarrow 0,$ $\tau
_{n}\rightarrow \infty $ and $\tau _{n}/\kappa _{n}\rightarrow 0,$%
\begin{eqnarray*}
&&\sup_{P\in \mathcal{P}_{0}}P\left( c_{0,n,\alpha }^{\ast }\left( \phi _{n},%
\overline{h}_{B,n}^{\ast }\right) \leq c_{0,n,\alpha }\left( h_{A,n},%
\overline{h}_{B,n}^{\ast }\right) \right) \\
&\leq &\sup_{P\in \mathcal{P}_{0}}P\left( -\phi _{j,n}(x)\leq h_{A,j,n}(x),%
\text{ for some }x\in \mathcal{X}^{+}\text{ and some }j=2,..,k\right) \\
&\leq &\sup_{P\in \mathcal{P}_{0}}P\left( \xi _{j,n}(x)<-1\text{ AND }%
-c_{n}\leq h_{A,j,n}(x),\text{ for some }x\in \mathcal{X}^{+}\text{ and }%
j=2,..,k\right) \\
&\leq &\sup_{P\in \mathcal{P}_{0}}P\left( D(x)^{1/2}\overline{D}%
_{jj,n}^{-1/2}(x)v_{j,n}(x)+D(x)^{1/2}\overline{D}%
_{jj,n}^{-1/2}(x)h_{j,A,n}(x)<-\kappa _{n}\right. \\
&&\left. \text{AND }-c_{n}\leq h_{A,j,n}(x),\text{ for some }x\in \mathcal{X}%
^{+}\text{ and }j=2,..,k\right) \\
&\leq &\sup_{P\in \mathcal{P}_{0}}P\left( -\tau _{n}+D(x)^{-1/2}\overline{D}%
_{jj,n}^{-1/2}(x)h_{j,A,n}(x)<-\kappa _{n}\right. \\
&&\left. \text{AND }-c_{n}\leq h_{A,j,n}(x),\text{ for some }x\in \mathcal{X}%
^{+}\text{ and }j=2,..,k\right) \\
&&+\sup_{P\in \mathcal{P}_{0}}P\left( D(x)^{1/2}\overline{D}%
_{jj,n}^{-1/2}(x)v_{j,n}(x)<-\tau _{n},\text{ for some }x\in \mathcal{X}^{+}%
\text{ and }j=2,..,k\right) \\
&\leq &\sup_{P\in \mathcal{P}_{0}}P\left( -D(x)^{-1/2}\overline{D}%
_{jj,n}^{-1/2}(x)h_{j,A,n}(x)<-\kappa _{n}+c_{n}\right. \\
&&\left. \text{AND }-c_{n}\leq h_{A,j,n}(x),\text{ for some }x\in \mathcal{X}%
^{+}\text{ and }j=2,..,k\right) \\
&=&o(1).
\end{eqnarray*}%
This establishes that (\ref{c0*}) holds. Finally, (\ref{c0}) follows from
Lemma 1 and Lemma 2.

\bigskip

\noindent \textbf{(ii)} Recall that $c_{0,n,1-\alpha }^{\ast }\left( \phi
_{n},h_{B,n}\right) $ is the ($1-\alpha )-$ percentile of $S_{n}^{\ast },$
as defined in (\ref{Sn*}); and define $c_{0,n,1-\alpha }^{GMS}\left( \phi
_{n},\overline{h}_{B,n}\right) $ to the ($1-\alpha )-$ percentile of $%
S_{n}^{GMS},$ where%
\begin{equation*}
S_{n}^{GMS}=\max_{x\in \mathcal{X}^{+}}\sum_{j=2}^{k}\max \left( \left\{ 0,%
\frac{\overline{v}_{j,n}(x)-\phi _{j,n}(x)}{\sqrt{\overline{h}_{B,jj}(x)}}%
\right\} \right) ^{2},
\end{equation*}%
with $\overline{v}_{n}=(\overline{v}_{2,n},...,\overline{v}_{k,n})^{\prime }$
is a $k-1$ dimensional Gaussian process, with mean zero and covariance $%
\overline{h}_{B}(x,x^{\prime })=\widehat{D}_{n}^{-1/2}(x)\overline{\Sigma }%
(x,x^{\prime })\widehat{D}_{n}^{-1/2}(x^{\prime }).$ We first need to show
that%
\begin{equation}
c_{0,n,1-\alpha }^{\ast }\left( \phi _{n},h_{B,n}\right) -c_{0,n,1-\alpha
}^{GMS}\left( \phi _{n},\overline{h}_{B,n}\right) =o_{p}(1),  \label{cv}
\end{equation}%
and then to prove that the statement holds when replacing $c_{0,n,1-\alpha
}^{\ast }\left( \phi _{n},h_{B,n}\right) $ with $c_{0,n,1-\alpha
}^{GMS}\left( \phi _{n},\overline{h}_{B,n}\right) .$

From Lemma 2, $\widehat{\Sigma }_{n}^{\ast }\left( x,x^{\prime }\right) -%
\widehat{\Sigma }_{n}\left( x,x^{\prime }\right) =o_{p}^{\ast }(1),$ and so $%
\overline{\Sigma }_{n}^{\ast }\left( x,x^{\prime }\right) -\overline{\Sigma }%
_{n}\left( x,x^{\prime }\right) =o_{p}^{\ast }(1).$ Then, by Theorem 2.3 in
Peligrad (1998),%
\begin{equation*}
v^{\ast }\overset{d^{\ast }}{\Longrightarrow }v\text{ a.s.-}\omega ,
\end{equation*}%
where $v^{\ast }\overset{\ast }{\Longrightarrow }v$ denotes weak
convergence, conditional on sample. As $\overline{v}_{n}\Longrightarrow v,$ (%
\ref{cv}) follows.

Given Assumption A4, by Lemma B3 in the Supplement of Andrews and Shi
(2013), the distribution of $S_{\infty }^{\dag },$ as defined in (\ref%
{Sn-inf}), is continuous. It is also strictly increasing and its $(1-\alpha
)-$quantile is strictly positive, for all $\alpha <1/2.$ The statement then
follows by the same argument as that used in the proof of Theorem 2(b) in
the Supplement of Andrews and Shi (2013).

\textbf{(iii)-(iv)} follow by the same arguments as those used in the proof
of \textbf{(i)} and \textbf{(ii)}, respectively. In the case of $%
S_{n}^{G^{+}},$ we rely on the the stochastic equicontinuity of $\frac{1}{%
\sqrt{n}}\sum_{i=1}^{n}\left( 1\left\{ e_{1,i}\leq x\right\} -1\left\{
e_{1,i}\leq u\right\} \right) ,$ as $|x-u|\rightarrow 0.$ When considering $%
S_{n}^{C^{+}},$ we need to ensure the stochastic equicontinuity of $\frac{1}{%
\sqrt{n}}\sum_{i=1}^{n}\left( \left( e_{1,i}-x\right) _{+}-\left(
e_{1,i}-u\right) _{+}\right) .$ Now,%
\begin{eqnarray*}
&&\frac{1}{\sqrt{n}}\sum_{i=1}^{n}\left( \left( e_{1,i}-x\right) _{+}-\left(
e_{1,i}-u\right) _{+}\right) \\
&=&\frac{1}{\sqrt{n}}\sum_{i=1}^{n}\left( u-x\right) 1\left\{ e_{1,i}\geq
u\right\} +\frac{1}{\sqrt{n}}\sum_{i=1}^{n}\left( e_{1,i}-u\right)
_{+}\left( 1\left\{ e_{1,i}\geq x\right\} -1\left\{ e_{1,i}\geq u\right\}
\right) ,
\end{eqnarray*}%
which, given Assumption 2, is stochastically equicontinous, by the same
argument as those used for $S_{n}^{G^{+}}.$ Hence, Theorem 2.3 in Peligrad
(1998) also holds in this case.

\medskip

\noindent \textbf{Proof of Theorem 3: (i)} Without loss of generality, let $%
B_{FA}^{G+}=\left\{ x\in \mathcal{X}^{+}:G_{2}(x)>0\right\} ,$ and note that
for all $x\in B_{FA}^{G+},$ $\max \left\{ 0,\frac{\sqrt{n}G_{2,n}^{+}(x)}{%
\overline{\sigma }_{22,n}^{G+}(x)}\right\} =\frac{\sqrt{n}G_{2,n}^{+}(x)}{%
\overline{\sigma }_{22,n}^{G+}(x)}.$ Thus,%
\begin{eqnarray*}
S_{n}^{G+} &=&\int_{B_{FA}^{G+}}\sum_{j=2}^{k}\left( \max \left\{ 0,\frac{%
\sqrt{n}G_{j,n}^{+}(x)}{\overline{\sigma }_{jj,n}^{G+}(x)}\right\} \right)
^{2}\mathrm{d}Q(x)+\int_{\mathcal{X}^{+}\backslash
B_{FA}^{G+}}\sum_{j=2}^{k}\left( \max \left\{ 0,\frac{\sqrt{n}G_{j,n}^{+}(x)%
}{\overline{\sigma }_{jj,n}^{G+}(x)}\right\} \right) ^{2}\mathrm{d}Q(x) \\
&=&\int_{B_{FA}^{G+}}\left( \frac{\sqrt{n}G_{2,n}^{+}(x)}{\overline{\sigma }%
_{22,n}^{G+}(x)}\right) ^{2}\mathrm{d}Q(x)+\int_{B_{FA}^{G+}}\sum_{j=3}^{k}%
\left( \max \left\{ 0,\frac{\sqrt{n}G_{j,n}^{+}(x)}{\overline{\sigma }%
_{jj,n}^{G+}(x)}\right\} -\left( \frac{\sqrt{n}G_{2,n}^{+}(x)}{\overline{%
\sigma }_{22,n}^{G+}(x)}\right) \right) ^{2}\mathrm{d}Q(x) \\
&&+\int_{\mathcal{X}^{+}\backslash B_{FA}^{G+}}\sum_{j=2}^{k}\left( \max
\left\{ 0,\frac{\sqrt{n}G_{j,n}^{+}(x)}{\overline{\sigma }_{jj,n}^{G+}(x)}%
\right\} \right) ^{2}\mathrm{d}Q(x) \\
&=&I_{n}+II_{n}+III_{n}.
\end{eqnarray*}%
Now, $I_{n}$ diverges to infinity with probability approaching one, while
Theorem 1 ensures that $II_{n}$ and $III_{n}$ are $O_{p}(1)$. Thus, $%
S_{n}^{G+}$ diverges to infinity$.$ As $S_{n}^{\ast G+}$ is $O_{p^{\ast
}}(1),$ conditional on the sample, the statement follows.

\noindent \textbf{(ii)} Note that $S_{n}^{C^{+}}$ can be treated exactly as $%
S_{n}^{G^{+}}.$

\medskip

\noindent \textbf{Proof of Theorem 4:}

\noindent \textbf{(i)} Define, $S_{\infty ,LA}^{\dag G+}$ as in (\ref{Sn-inf}%
), but with the vector $h_{j,A,\infty }^{G+}(x)$ having at least one
component strictly bounded away above from zero, and finite, for all $x\in
B_{LA}^{G+}.$ Let $\mathcal{P}_{n,LA}^{G+}$ denote the set of probabilities
under the sequence of local alternatives. We have that for all $a>0,$ 
\begin{equation*}
\lim \sup_{n\rightarrow \infty }\sup_{P\in \mathcal{P}_{n,LA}^{G+}}\left[
P\left( S_{n}^{G+}>a\right) -P\left( S_{\infty ,LA}^{\dag G+}>a\right) %
\right] =0,
\end{equation*}%
and the distribution of $S_{\infty ,LA}^{\dag G+}$ is continuous at its $%
(1-\alpha )+\delta $ quintile, for all $0<\alpha <1/2$ and $\delta \geq 0.$
Also, note that for all $x\in B_{LA}^{G+},$ $\phi _{n}^{G+}=0.$ The
statement then follows by the same argument as that used in the proof of
Theorem 2\textbf{(}ii\textbf{)}. (\textbf{ii) }By the same argument as in
part (i).

\medskip

\noindent \textbf{Proof of Lemma 3: (i)} Letting $\overline{F}_{j}(x)=\frac{1%
}{n}\sum_{t=R}^{T-1}1\left\{ \widehat{e}_{j,t+1}\leq x\right\} ,$ by an
intermediate value expansion, in the case of a recursive estimation scheme,
we have that%
\begin{eqnarray}
&&\frac{1}{\sqrt{n}}\sum_{t=R}^{T-1}\left( 1\left\{ \widehat{e}_{j,t+1}\leq
x\right\} -F_{j}(x)\right)  \notag \\
&=&\frac{1}{\sqrt{n}}\sum_{t=R}^{T-1}\left( 1\left\{ e_{j,t+1}\leq x\right\}
-F_{j}(x)\right) +\frac{1}{\sqrt{n}}\sum_{t=R}^{T-1}\left( 1\left\{ \widehat{%
e}_{j,t+1}\leq x\right\} -1\left\{ e_{j,t+1}\leq x\right\} \right)  \notag \\
&=&\frac{1}{\sqrt{n}}\sum_{t=R}^{T-1}\left( 1\left\{ e_{j,t+1}\leq x\right\}
-F_{j}(x)\right) +\left( \frac{1}{\sqrt{n}}\sum_{t=R}^{T-1}\left( 1\left\{
e_{j,t+1}\leq x-\nabla _{\theta _{j}}\phi _{j}\left( Z_{j,t+1},\overline{%
\theta }_{j,t}\right) \left( \widehat{\theta }_{j,t}-\theta _{j}^{\dag
}\right) \right\} \right. \right.  \notag \\
&&\left. \left. -F_{j}\left( x-\nabla _{\theta _{j}}\phi _{j}\left(
Z_{j,t+1},\overline{\theta }_{j,t}\right) \left( \widehat{\theta }%
_{j,t}-\theta _{j}^{\dag }\right) \right) \right) -\frac{1}{\sqrt{n}}%
\sum_{t=R}^{T-1}\left( 1\left\{ e_{j,t+1}\leq x\right\} -F_{j}(x)\right)
\right)  \notag \\
&&+\frac{1}{\sqrt{n}}\sum_{t=R}^{T-1}\left( F_{j}\left( x-\nabla _{\theta
_{j}}\phi _{j}\left( Z_{j,t+1},\overline{\theta }_{j,t}\right) \left( 
\widehat{\theta }_{j,t}-\theta _{j}^{\dag }\right) \right) -F_{j}(x)\right) 
\notag \\
&=&\frac{1}{\sqrt{n}}\sum_{t=R}^{T-1}\left( 1\left\{ e_{j,t+1}\leq x\right\}
-F_{j}(x)\right) +\frac{1}{\sqrt{n}}\sum_{t=R}^{T-1}\left( F_{j}\left(
x-\nabla _{\theta _{j}}\phi _{j}\left( Z_{j,t+1},\overline{\theta }%
_{j,t}\right) \left( \widehat{\theta }_{j,t}-\theta _{j}^{\dag }\right)
\right) -F_{j}(x)\right)  \notag \\
&&+o_{p}(1)  \notag \\
&=&\frac{1}{\sqrt{n}}\sum_{t=R}^{T-1}\left( 1\left\{ e_{j,t+1}\leq x\right\}
-F_{j}(x)\right) -f_{j}(x)\frac{1}{\sqrt{n}}\sum_{t=R}^{T-1}\nabla _{\theta
_{j}}\phi _{j}\left( Z_{j,t+1},\overline{\theta }_{j,t}\right) \left( 
\widehat{\theta }_{j,t}-\theta _{j}^{\dag }\right) +o_{p}(1)  \notag \\
&=&\frac{1}{\sqrt{n}}\sum_{t=R}^{T-1}\left( 1\left\{ e_{j,t+1}\leq x\right\}
-F_{j}(x)\right)  \label{der} \\
&&-f_{j}(x)\frac{1}{\sqrt{n}}\sum_{t=R}^{T-1}\nabla _{\theta _{j}}\phi
_{j}\left( Z_{j,t+1},\overline{\theta }_{j,t}\right) ^{\prime }\frac{1}{t}%
\sum_{i=1}^{t}\left( \nabla _{\theta _{j}}^{2}m_{j}(X_{i},Z_{j,i-1},\theta
_{j}^{\dagger })\right) ^{-1}\left( \nabla _{\theta
_{j}}m_{j}(X_{i},Z_{j,i-1},\theta _{j}^{\dagger })\right) +o_{p}(1)  \notag
\\
&=&\frac{1}{\sqrt{n}}\sum_{t=R}^{T-1}\left( 1\left\{ e_{j,t+1}\leq x\right\}
-F_{j}(x)\right) -f_{j}(x)\widehat{A}_{j}\frac{1}{\sqrt{n}}\sum_{t=R}^{T-1}%
\frac{1}{t}\sum_{i=1}^{t}\left( \nabla _{\theta
_{j}}m_{j}(X_{i},Z_{j,i-1},\theta _{j}^{\dagger })\right) +o_{p}(1)  \notag
\end{eqnarray}%
where the $o_{p}(1)$ term on the RHS of the third equality in (\ref{der})
comes from the fact that%
\begin{eqnarray*}
&&\frac{1}{\sqrt{n}}\sum_{t=R}^{T-1}\left( 1\left\{ e_{j,t+1}\leq x-\nabla
_{\theta _{j}}\phi _{j}\left( Z_{j,t+1},\overline{\theta }_{j,t}\right)
\left( \widehat{\theta }_{j,t}-\theta _{j}^{\dag }\right) \right\} \right. \\
&&\left. -F_{j}\left( x-\nabla _{\theta _{j}}\phi _{j}\left( Z_{j,t+1},%
\overline{\theta }_{j,t}\right) \left( \widehat{\theta }_{j,t}-\theta
_{j}^{\dag }\right) \right) \right) -\frac{1}{\sqrt{n}}\sum_{t=R}^{T-1}%
\left( 1\left\{ e_{j,t+1}\leq x\right\} -F_{j}(x)\right) =o_{p}(1),
\end{eqnarray*}%
because of stochastic equicontinuity.

Hence,%
\begin{eqnarray*}
&&\mathrm{var}\left( \frac{1}{\sqrt{n}}\sum_{t=R}^{T-1}\left( \left(
1\left\{ \widehat{e}_{1,t+1}\leq x\right\} -F_{1}(x)\right) -\left( 1\left\{ 
\widehat{e}_{j,t+1}\leq x\right\} -F_{j}(x)\right) \right) \right) \\
&=&\mathrm{var}\left( \frac{1}{\sqrt{n}}\sum_{t=R}^{T-1}\left( \left(
1\left\{ e_{1,t+1}\leq x\right\} -F_{1}(x)\right) -\left( 1\left\{
e_{j,t+1}\leq x\right\} -F_{j}(x)\right) \right) \right) \\
&&+f_{1}(x)^{2}\mathrm{E}\left( \nabla _{\theta _{1}}\phi _{1}\left(
Z_{1,t+1},\theta _{1}^{\dagger }\right) \right) ^{\prime }\left( \mathrm{E}%
\left( \nabla _{\theta _{1}}^{2}m_{1}(X_{i},Z_{1,i-1},\theta _{1}^{\dagger
})\right) \right) ^{-1} \\
&&\mathrm{var}\left( \frac{1}{\sqrt{n}}\sum_{t=R}^{T-1}\frac{1}{t}%
\sum_{i=1}^{t}\left( \nabla _{\theta _{1}}m_{1}(X_{i},Z_{1,i-1},\theta
_{1}^{\dagger })\right) \right) \\
&&\left( \mathrm{E}\left( \nabla _{\theta
_{1}}^{2}m_{1}(X_{i},Z_{1,i-1},\theta _{1}^{\dagger })\right) \right) ^{-1}%
\mathrm{E}\left( \nabla _{\theta _{1}}\phi _{1}\left( Z_{1,t+1},\theta
_{1}^{\dagger }\right) \right) \\
&&+f_{j}(x)^{2}\mathrm{E}\left( \nabla _{\theta _{j}}\phi _{j}\left(
Z_{j,t+1},\theta _{j}^{\dagger }\right) \right) ^{\prime }\left( \mathrm{E}%
\left( \nabla _{\theta _{j}}^{2}m_{j}(X_{i},Z_{j,i-1},\theta _{j}^{\dagger
})\right) \right) ^{-1} \\
&&\mathrm{var}\left( \frac{1}{\sqrt{n}}\sum_{t=R}^{T-1}\frac{1}{t}%
\sum_{i=1}^{t}\left( \nabla _{\theta _{j}}m_{j}(X_{i},Z_{j,i-1},\theta
_{j}^{\dagger })\right) \right) \\
&&\left( \mathrm{E}\left( \nabla _{\theta
_{j}}^{2}m_{j}(X_{i},Z_{j,i-1},\theta _{j}^{\dagger })\right) \right) ^{-1}%
\mathrm{E}\left( \nabla _{\theta _{j}}\phi _{j}\left( Z_{j,t+1},\theta
_{j}^{\dagger }\right) \right) \\
&&-2f_{1}(x)f_{j}(x)\mathrm{E}\left( \nabla _{\theta _{1}}\phi _{1}\left(
Z_{1,t+1},\theta _{1}^{\dagger }\right) \right) ^{\prime }\left( \mathrm{E}%
\left( \nabla _{\theta _{1}}^{2}m_{j}(X_{i},Z_{1,i-1},\theta _{1}^{\dagger
})\right) \right) ^{-1} \\
&&\mathrm{cov}\left( \frac{1}{\sqrt{n}}\sum_{t=R}^{T-1}\frac{1}{t}%
\sum_{i=1}^{t}\left( \nabla _{\theta _{1}}m_{1}(X_{i},Z_{1,i-1},\theta
_{1}^{\dagger })\right) \frac{1}{\sqrt{n}}\sum_{t=R}^{T-1}\frac{1}{t}%
\sum_{i=1}^{t}\left( \nabla _{\theta _{j}}m_{j}(X_{i},Z_{j,i-1},\theta
_{j}^{\dagger })\right) \right) \\
&&\left( \mathrm{E}\left( \nabla _{\theta
_{j}}^{2}m_{j}(X_{i},Z_{j,i-1},\theta _{j}^{\dagger })\right) \right) ^{-1}%
\mathrm{E}\left( \nabla _{\theta _{j}}\phi _{j}\left( Z_{j,t+1},\theta
_{j}^{\dagger }\right) \right) \\
&&+2f_{1}(x)\mathrm{E}\left( \nabla _{\theta _{1}}\phi _{1}\left(
Z_{1,t+1},\theta _{1}^{\dagger }\right) \right) ^{\prime }\left( \mathrm{E}%
\left( \nabla _{\theta _{1}}^{2}m_{1}(X_{i},Z_{1,i-1},\theta _{1}^{\dagger
})\right) \right) ^{-1} \\
&&\mathrm{cov}\left( \frac{1}{\sqrt{n}}\sum_{t=R}^{T-1}\left( \left(
1\left\{ e_{1,t+1}\leq x\right\} -F_{1}(x)\right) -\left( 1\left\{
e_{j,t+1}\leq x\right\} -F_{j}(x)\right) \right) \frac{1}{\sqrt{n}}%
\sum_{t=R}^{T-1}\frac{1}{t}\sum_{i=1}^{t}\left( \nabla _{\theta
_{1}}m_{1}(X_{i},Z_{1,i-1},\theta _{1}^{\dagger })\right) \right) \\
&&-2f_{j}(x)^{2}\mathrm{E}\left( \nabla _{\theta _{j}}\phi _{j}\left(
Z_{j,t+1},\theta _{j}^{\dagger }\right) \right) ^{\prime }\left( \mathrm{E}%
\left( \nabla _{\theta _{j}}^{2}m_{j}(X_{i},Z_{j,i-1},\theta _{j}^{\dagger
})\right) \right) ^{-1} \\
&&\mathrm{cov}\left( \frac{1}{\sqrt{n}}\sum_{t=R}^{T-1}\left( \left(
1\left\{ e_{1,t+1}\leq x\right\} -F_{1}(x)\right) -\left( 1\left\{
e_{j,t+1}\leq x\right\} -F_{j}(x)\right) \right) \frac{1}{\sqrt{n}}%
\sum_{t=R}^{T-1}\frac{1}{t}\sum_{i=1}^{t}\left( \nabla _{\theta
_{j}}m_{j}(X_{i},Z_{j,i-1},\theta _{j}^{\dagger })\right) \right)
\end{eqnarray*}%
(ii) Recalling (\ref{C-PEE}) by a similar argument as in part (i).

\medskip

\noindent \textbf{Proof of Theorem 5:}

\noindent Given Lemma 3, the statement follows by the same argument as in
Theorem 1.

\medskip

\noindent \textbf{Proof of Lemma 4:}

\noindent (i) Note that $\widehat{\mathrm{avar}^{\ast }}\left( \frac{1}{%
\sqrt{n}}\sum_{t=R}^{n-1}\left( \eta _{j,t}^{\ast }(x)-\eta _{1,t}^{\ast
}(x)\right) \right) =\widehat{\sigma }_{jj,n}^{2\ast G+}(x)$ as defined in (%
\ref{sigmaG*}), $\widehat{PEE^{\ast }}_{j,t}$ is defined as $PEE_{j,t}^{\ast
}$ with \textrm{E}$^{\ast }$ replaced by an average, also%
\begin{eqnarray*}
&&\widehat{\mathrm{avar}^{\ast }}\left( \frac{1}{\sqrt{n}}%
\sum_{t=R}^{n-1}\left( \widehat{PEE^{\ast }}_{j,t}-\widehat{PEE^{\ast }}%
_{1,t}\right) \right) \\
&=&\frac{1}{b_{n}}\sum_{k=1}^{b_{n}}\left( \frac{1}{l_{n}^{1/2}}%
\sum_{i=1}^{l_{n}}\left( \widehat{PEE^{\ast }}_{j,(k-1)\NEG{l}_{n}+i}-%
\widehat{PEE^{\ast }}_{j,(k-1)\NEG{l}_{n}+i}\right) \right) ^{2}
\end{eqnarray*}%
and by Theorem 1 in Corradi and Swanson (2007),%
\begin{eqnarray*}
&&\frac{1}{\sqrt{n}}\sum_{t=R}^{n-1}\left( \widehat{PEE^{\ast }}_{j,t}-%
\widehat{PEE^{\ast }}_{1,t}\right) \\
&=&\frac{1}{\sqrt{n}}\sum_{t=R}^{n-1}\left( \widehat{PEE}_{j,t}-\widehat{PEE}%
_{1,t}\right) +o_{p}(1)^{\ast }.
\end{eqnarray*}%
\begin{eqnarray*}
&&\widehat{\mathrm{acov}^{\ast }}\left( \frac{1}{\sqrt{n}}%
\sum_{t=R}^{n-1}\left( u_{j,t}^{\ast }(x)-u_{1,t}^{\ast }(x)\right) ,\frac{1%
}{\sqrt{n}}\sum_{t=R}^{n-1}\left( \widehat{PEE^{\ast }}_{j,t}-\widehat{%
PEE^{\ast }}_{1,t}\right) \right) \\
&=&\frac{1}{b_{n}}\sum_{k=1}^{b_{n}}\left( \frac{1}{l_{n}^{1/2}}%
\sum_{i=1}^{l_{n}}\left( \widehat{PEE^{\ast }}_{j,(k-1)\NEG{l}_{n}+i}-%
\widehat{PEE^{\ast }}_{j,(k-1)\NEG{l}_{n}+i}\right) \right. \\
&&\left. \frac{1}{l_{n}^{1/2}}\sum_{i=1}^{l_{n}}\left( u_{j,t}^{\ast
}(x)-u_{1,t}^{\ast }(x)\right) \right)
\end{eqnarray*}%
and for $h\rightarrow 0,$ $nh\rightarrow \infty ,$ $\widehat{f}%
_{j,n,h}^{\ast }(x)=\widehat{f}_{j,n,h}(x)+o_{p^{\ast
}}(1)=f(x)+o_{p}(1)+o_{p^{\ast }}(1).$ The statement then follow by the same
argument as in Lemma 2 and Lemma 3.

\noindent (ii) By a similar argument as in Part (i).

\medskip

\noindent \textbf{Proof of Theorem 6:}

\noindent (i) By a similar argument as in the proof of Theorem 2 in Corradi
and Swanson (2007),%
\begin{eqnarray*}
\widetilde{S}_{n}^{\ast G+} &=&\int_{\mathcal{X}^{+}}\sum_{j=2}^{k}\max
\left( \left\{ 0,\frac{\widetilde{v}_{j,n}^{\ast G+}(x)-\widetilde{\phi }%
_{j,n}^{G+}(x)}{\sqrt{\widetilde{\overline{h}}_{2,jj}^{\ast G+}(x)}}\right\}
\right) ^{2}\mathrm{d}Q(x) \\
&=&\int_{\mathcal{X}^{+}}\sum_{j=2}^{k}\max \left( \left\{ 0,\frac{%
\widetilde{v}_{j,n}^{G+}(x)-\widetilde{\phi }_{j,n}^{G+}(x)}{\sqrt{%
\widetilde{\overline{h}}_{2,jj}^{G+}(x)}}\right\} \right) ^{2}\mathrm{d}%
Q(x)+o_{p^{\ast }}(1)
\end{eqnarray*}%
The statement then follows from Lemma 4 and Theorem 2.

\noindent (ii) By a similar argument as in Part (i).

\pagebreak

\section{References}

\noindent Aiolfi, M., C. Capistr\'{a}n, and A. Timmermann (2011). Forecast
Combinations. In M.P. Clements and D\smallskip .F. Hendry (eds.), \textbf{%
Oxford Handbook of Economic Forecasting}, Oxford University Press, Oxford.

\noindent Andrews, D.W.K. (1991). Heteroskedasticity and Autocorrelation
Robust Covariance Matrix Estimation. \textit{Econometrica, }59,
817-858.\smallskip

\noindent Andrews, D.W.K. (2011). Similar-on-the-Boundary Tests for Moment
Inequalities Exist, but Have Very Poor Power. Cowles Foundation Working
Paper 1815R.\smallskip

\noindent Andrews, D.W.K. and D. Pollard (1994). An Introduction to
Functional Central Limit Theorems for Dependent Stochastic Processes. 
\textit{International Statistical Review, }61, 119-132.\smallskip

\noindent Andrews, D.W.K. and P. Guggenberger (2010). Asymptotic Size and a
Problem with Subsampling and with the m out of n Bootstrap. \textit{%
Econometric Theory,} 26, 426-468.\smallskip

\noindent Andrews, D.W.K. and P. Guggenberger (2010). Inference for
Parameters Defined by Moment Inequalities Using Generalized Moment
Selection. \textit{Econometrica, }78, 119-157.\smallskip

\noindent Andrews, D.W.K. and P.J. Barwick (2012). Inference for Parameters
Defined by Moment Inequalities: A Recommended Moment Selection Procedure. 
\textit{Econometrica, }80, 2805-2826.\smallskip

\noindent Andrews, D.W.K. and X. Shi (2013). Inference Based on Conditional
Moment Inequalities. \textit{Econometrica, }81, 609-666.\smallskip

\noindent Andrews, D.W.K. and X. Shi (2017). Inference Based on Many
Conditional Moment Inequalities. \textit{Journal of Econometrics, }196,
275-287.\smallskip

\noindent Bierens H.J. (1982). Consistent Model Specification Tests. \textit{%
Journal of Econometrics, }20, 105-134.\smallskip

\noindent Bierens H.J. (1990). A Consistent Conditional Moment Tests for
Functional Form. \textit{Econometrica, }58, 1443-1458.\smallskip

\noindent Coroneo, L., V. Corradi and P. Santos-Monteiro (2017), Testing for
Optimal Monetary Policy via Moment Inequalities. \textit{Journal of Applied
Econometrics, }forthcoming.

\noindent Corradi, V. (1999). Deciding Between $I(0)$ and $I(1)$ via
FLIL-based Bounds. \textit{Econometric Theory}, 15, 643-663.\smallskip

\noindent Corradi, V. and N.R. Swanson (2007). Nonparametric Bootstrap
Procedures for Predictive Inference Based on Recursive Estimation Schemes. 
\textit{International Economic Review, }48, 67-109.\smallskip

\noindent Corradi, V. and N. R. Swanson (2013). A Survey of Recent Advances
in Forecast Accuracy Comparison Testing, with an Extension to Stochastic
Dominance. In X. Chen and N.R. Swanson (eds.), \textbf{Causality,
Prediction, and Specification Analysis: Recent Advances and Future
Directions, Essays in honor of Halbert L. White, Jr.}, Springer, New
York.\smallskip

\noindent Croushore, D. (1993). Introducing: The Survey of Professional
Forecasters, The Federal Reserve Bank of Philadelphia Business Review,
November-December, 3-15.

\noindent Diebold, F. X. and Mariano, R. S. (1995). Comparing Predictive
Accuracy. \textit{Journal of Business and Economic Statistics}, 13, 253-263.

\noindent Diebold, F.X. and M. Shin (2015). Assessing Point Forecast
Accuracy by Stochastic Loss Distance. \textit{Economics Letters, }130,
37-38.\smallskip

\noindent Diebold, F.X. and M. Shin (2017). Assessing Point Forecast
Accuracy by Stochastic Error Distance. \textit{Econometric Reviews, }36,
588-598.\smallskip

\noindent Donald, S.G. and Y.C. Hsu (2016). Improving the Power of Tests for
Stochastic Dominance. \textit{Econometric Reviews, }35, 553-585.\smallskip

\noindent Elliott, G., I. Komunjer and A. Timmermann (2005). Estimation and
Testing of Forecast Rationality under Flexible Loss. \textit{Review of
Economic Studies, }72, 1107-1125.\smallskip

\noindent Elliott, G., I. Komunjer and A. Timmermann (2008). Biases in
Macroeconomic Forecasts: Irrationality of Asymmetric Loss? \textit{Journal
of the European Economic Association, }6, 122-157.\smallskip

\noindent Fair, R.C. and R.J. Shiller (1990). Comparing Information in
Forecasts from Econometric Models. \textit{American Economic Review}, 80,
375-389.

\noindent Genre, V., G. Kenny, A. Meyler, and A. Timmermann (2013).
Combining the Forecasts in the ECB Survey of Professional Forecasters: Can
Anything Beat the Simple Average. \textit{International Journal of
Forecasting}, 29, 108-121.

\noindent Granger, C. W. J. (1999). Outline of Forecast Theory using
Generalized Cost Functions. \textit{Spanish Economic Review,} 1,
161-173\smallskip .

\noindent Hansen, B.E. (2008). Uniform Convergence Rates for Kernel
Estimators with Dependent Data. \textit{Econometric Theory, }24,
726-748.\smallskip

\noindent Hansen, P. R. (2005). A Test for Superior Predictive Ability. 
\textit{Journal of Business and Economic Statistics,} 23, 365--380.\smallskip

\noindent Holm, S. (1979). A Simple Sequentially Rejective Multiple Test
Procedure. \textit{Scandinavian Journal of Statistics,} 6, 65--70.\smallskip

\noindent Jin, S., V. Corradi and N.R. Swanson (2017). Robust Forecast
Comparison. \textit{Econometric Theory, }33, 1306-1351.\smallskip

\noindent Lahiri, K., H. Peng, and Y. Zhao (2015). Testing the Value of
Probability Forecasts for Calibrated Combining. \textit{International
Journal of Forecasting}, 31, 113-129.

\noindent Lahiri, K., H. Peng, and Y. Zhao (2017). Online Learning and
Forecast Combination in Unbalanced Panels. \textit{Econometric Reviews, }36,
257-288.

\noindent Linton, O., E. Maasoumi, and Y. J. Whang (2005). Consistent
Testing for Stochastic Dominance: A Subsampling Approach. \textit{Review of
Economic Studies,} 72, 735-765.\smallskip

\noindent Linton, O., K. Song and Y.J. Whang (2010). An Improved Bootstrap
Test of Stochastic Dominance. \textit{Journal of Econometrics, }154,
186-202.\smallskip

\noindent McCracken, M.W. (2000). Robust Out-of-Sample Inference. \textit{%
Journal of Econometrics}, 99, 195-223.

\noindent Mikusheva, A. (2007). Uniform Inference in Autoregressive
Processes. \textit{Econometrica, }75, 1411-1452.\smallskip

\noindent Peligrad, M. (1998). On the Blockwise Bootstrap for Empirical
Processes for Stationary Sequences. \textit{Annals of Probability, }26,
877-901.\smallskip

\noindent Pollard, D. (1990). Empirical Processes: Theory and Applications.
In \textbf{CBMS Conference Series in Probability and Statistic, Vol.2},
Institute of Mathematical Statistics, Hayward.\smallskip

\noindent Swanson, N.R. and H. White (1997a). A Model Selection Approach to
Real-Time Macroeconomic Forecasting Using Linear Models and Artificial
Neural Networks. \textit{Review of Economics and Statistics}, 79, 1997,
540-550.

\noindent Swanson, N.R. and H. White (1997b). Forecasting Economic Time
Series Using Adaptive Versus Nonadaptive and Linear Versus Nonlinear
Econometric Models. \textit{International Journal of Forecasting,} 13, 1997,
439-461.

\noindent Timmermann, A. (2006). Forecast Combinations. In A. Timmermann,
C.W.J. Granger, and G. Elliott (eds.), \textbf{Handbook of Forecasting Vol.
1. }North Holland, Amsterdam.

\noindent White, H. (2000). A Reality Check for Data Snooping. \textit{%
Econometrica} 68, 1097-1126.

\noindent Zarnowitz, V. and P. Braun, (1992). Twenty-Two Years of the
NBER-ASA Quarterly Economic Outlook Surveys: Aspects and Comparisons of
Forecasting Performance. In J.H. Stock and M.W. Watson (eds.), \textbf{%
Business Cycles, Indicators, and Forecasting: Studies in Business Cycles,
Vol. 28}, University of Chicago Press, Chicago.

\bigskip

\begin{center}
\linespread{1.1}

\begin{table}[tbp]
\caption{Monte Carlo Results for $JCS_n^{G+}$, $JCS_n^{G-}$, $JCS_n^{C+}$,
and $JCS_n^{C-}$ Forecast Superiority Tests$^*$}{\centering}
\par
\hspace{1cm}
\par
\begin{tabular}{cc|cccc|cccc}
\hline\hline
& $n$ & {$J_n = 0.20$} & {$J_n = 0.35$} & {$J_n = 0.50$} & {$J_n = 0.65$} & {%
$J_n = 0.20$} & {$J_n = 0.35$} & {$J_n = 0.50$} & {$J_n = 0.65$} \\ 
&  & \multicolumn{4}{c}{GL forecast superiority} & \multicolumn{4}{c}{CL
Forecast Superiority} \\ \hline
&  & \multicolumn{8}{c}{$Empirical$ $Size$} \\ \hline
DGP1 & 250 & 0.107 & 0.101 & 0.077 & 0.088 & 0.096 & 0.112 & 0.087 & 0.097
\\ 
& 500 & 0.102 & 0.101 & 0.086 & 0.099 & 0.103 & 0.121 & 0.091 & 0.108 \\ 
& 1000 & 0.099 & 0.104 & 0.101 & 0.110 & 0.108 & 0.100 & 0.094 & 0.108 \\ 
\hline
DGP2 & 250 & 0.103 & 0.095 & 0.099 & 0.113 & 0.107 & 0.112 & 0.094 & 0.119
\\ 
& 500 & 0.104 & 0.102 & 0.107 & 0.104 & 0.111 & 0.089 & 0.100 & 0.103 \\ 
& 1000 & 0.112 & 0.093 & 0.109 & 0.104 & 0.097 & 0.108 & 0.106 & 0.101 \\ 
\hline
DGP3 & 250 & 0.080 & 0.076 & 0.065 & 0.072 & 0.036 & 0.035 & 0.032 & 0.035
\\ 
& 500 & 0.076 & 0.071 & 0.081 & 0.082 & 0.030 & 0.047 & 0.032 & 0.041 \\ 
& 1000 & 0.076 & 0.065 & 0.070 & 0.083 & 0.043 & 0.031 & 0.039 & 0.051 \\ 
\hline
&  & \multicolumn{8}{c}{$Empirical$ $Power$} \\ \hline
DGP4 & 250 & 0.382 & 0.392 & 0.402 & 0.418 & 0.639 & 0.671 & 0.646 & 0.633
\\ 
& 500 & 0.709 & 0.717 & 0.719 & 0.744 & 0.938 & 0.944 & 0.943 & 0.942 \\ 
& 1000 & 0.976 & 0.973 & 0.973 & 0.976 & 0.999 & 1.000 & 1.000 & 1.000 \\ 
\hline
DGP5 & 250 & 0.542 & 0.562 & 0.521 & 0.549 & 0.880 & 0.908 & 0.883 & 0.913
\\ 
& 500 & 0.838 & 0.846 & 0.836 & 0.848 & 0.991 & 0.992 & 0.995 & 0.993 \\ 
& 1000 & 0.996 & 0.989 & 0.991 & 0.994 & 1.000 & 1.000 & 1.000 & 1.000 \\ 
\hline
DGP6 & 250 & 0.999 & 0.999 & 0.999 & 1.000 & 1.000 & 1.000 & 1.000 & 1.000
\\ 
& 500 & 1.000 & 1.000 & 1.000 & 1.000 & 1.000 & 1.000 & 1.000 & 1.000 \\ 
& 1000 & 1.000 & 1.000 & 1.000 & 1.000 & 1.000 & 1.000 & 1.000 & 1.000 \\ 
\hline
DGP7 & 250 & 0.600 & 0.551 & 0.562 & 0.541 & 0.920 & 0.920 & 0.905 & 0.927
\\ 
& 500 & 0.845 & 0.872 & 0.846 & 0.840 & 0.991 & 0.997 & 0.992 & 0.992 \\ 
& 1000 & 0.995 & 0.996 & 0.993 & 0.994 & 1.000 & 1.000 & 1.000 & 1.000 \\ 
\hline
DGP8 & 250 & 1.000 & 0.998 & 1.000 & 0.999 & 1.000 & 1.000 & 1.000 & 1.000
\\ 
& 500 & 1.000 & 1.000 & 1.000 & 1.000 & 1.000 & 1.000 & 1.000 & 1.000 \\ 
& 1000 & 1.000 & 1.000 & 1.000 & 1.000 & 1.000 & 1.000 & 1.000 & 1.000 \\ 
\hline\hline
\end{tabular}%
\par
\begin{minipage}{1\columnwidth}
\vspace{0.3in}
{\noindent $^{*}$ Notes: Entries denote rejection frequencies of ($JCS_n^{G+}$,$JCS_n^{G-}$) tests (i.e., GL forecast superiority) and ($JCS_n^{C+}$,$JCS_n^{C-}$) tests (i.e., CL forecast superiority) under a variety of data generating processes 
denoted by DGP1-DGP8. In DGP1-DGP3, no alternative outperforms the benchmark model. In DGP4-DGP8, at least one alternative model outperfroms the benchmark model.
Sample sizes include $n$=250, 500, and 1000 observations, as indicated in the second column of entries in the table. Nominal test size is 10\%, and tests 
are carried out using critical values 
constructed for values of $J_n$ including 0.20, 0.35, 0.50, and 0.65. See Section 6 for complete details.}
\end{minipage}
\end{table}

\begin{table}[tbp]
\caption{Monte Carlo Results for $S_n^{G+}$, $S_n^{G-}$, $S_n^{C+}$, and $%
S_n^{C-}$ Forecast Superiority Tests$^*$}{\centering}
\par
\hspace{1cm}
\par
\begin{tabular}{cc|cccc|cccc}
\hline\hline
& $n$ & {$\eta = 0.045$} & {$\eta = 0.060$} & {$\eta = 0.075$} & {$\eta =
0.090$} & {$\eta = 0.045$} & {$\eta = 0.060$} & {$\eta = 0.075$} & {$\eta =
0.090$} \\ 
&  & \multicolumn{4}{c}{GL forecast superiority} & \multicolumn{4}{c}{CL
Forecast Superiority} \\ \hline
&  & \multicolumn{8}{c}{$Empirical$ $Size$} \\ \hline
DGP1 & 250 & 0.272 & 0.213 & 0.145 & 0.116 & 0.232 & 0.185 & 0.121 & 0.096
\\ 
& 500 & 0.251 & 0.207 & 0.142 & 0.101 & 0.204 & 0.161 & 0.114 & 0.079 \\ 
& 1000 & 0.263 & 0.211 & 0.159 & 0.096 & 0.207 & 0.177 & 0.130 & 0.080 \\ 
\hline
DGP2 & 250 & 0.277 & 0.261 & 0.174 & 0.115 & 0.219 & 0.193 & 0.131 & 0.081
\\ 
& 500 & 0.276 & 0.238 & 0.167 & 0.111 & 0.230 & 0.191 & 0.130 & 0.091 \\ 
& 1000 & 0.266 & 0.245 & 0.187 & 0.114 & 0.203 & 0.199 & 0.142 & 0.094 \\ 
\hline
DGP3 & 250 & 0.105 & 0.083 & 0.049 & 0.039 & 0.065 & 0.045 & 0.030 & 0.014
\\ 
& 500 & 0.084 & 0.066 & 0.029 & 0.022 & 0.072 & 0.045 & 0.026 & 0.015 \\ 
& 1000 & 0.080 & 0.053 & 0.026 & 0.023 & 0.066 & 0.038 & 0.025 & 0.010 \\ 
\hline
&  & \multicolumn{8}{c}{$Empirical$ $Power$} \\ \hline
DGP4 & 250 & 0.958 & 0.955 & 0.908 & 0.820 & 0.961 & 0.968 & 0.936 & 0.855
\\ 
& 500 & 1.000 & 1.000 & 0.997 & 0.992 & 0.999 & 1.000 & 0.998 & 0.997 \\ 
& 1000 & 1.000 & 1.000 & 1.000 & 1.000 & 1.000 & 1.000 & 1.000 & 1.000 \\ 
\hline
DGP5 & 250 & 0.996 & 0.991 & 0.975 & 0.948 & 0.997 & 0.990 & 0.980 & 0.964
\\ 
& 500 & 1.000 & 1.000 & 1.000 & 0.999 & 1.000 & 1.000 & 1.000 & 0.999 \\ 
& 1000 & 1.000 & 1.000 & 1.000 & 1.000 & 1.000 & 1.000 & 1.000 & 1.000 \\ 
\hline
DGP6 & 250 & 1.000 & 1.000 & 1.000 & 1.000 & 1.000 & 1.000 & 1.000 & 1.000
\\ 
& 500 & 1.000 & 1.000 & 1.000 & 1.000 & 1.000 & 1.000 & 1.000 & 1.000 \\ 
& 1000 & 1.000 & 1.000 & 1.000 & 1.000 & 1.000 & 1.000 & 1.000 & 1.000 \\ 
\hline
DGP7 & 250 & 0.995 & 0.995 & 0.991 & 0.979 & 0.994 & 0.999 & 0.993 & 0.976
\\ 
& 500 & 1.000 & 1.000 & 1.000 & 1.000 & 1.000 & 1.000 & 1.000 & 1.000 \\ 
& 1000 & 1.000 & 1.000 & 1.000 & 1.000 & 1.000 & 1.000 & 1.000 & 1.000 \\ 
\hline
DGP8 & 250 & 1.000 & 1.000 & 1.000 & 1.000 & 1.000 & 1.000 & 1.000 & 1.000
\\ 
& 500 & 1.000 & 1.000 & 1.000 & 1.000 & 1.000 & 1.000 & 1.000 & 1.000 \\ 
& 1000 & 1.000 & 1.000 & 1.000 & 1.000 & 1.000 & 1.000 & 1.000 & 1.000 \\ 
\hline\hline
\end{tabular}%
\par
\begin{minipage}{1\columnwidth}
\vspace{0.3in}
{\noindent $^{*}$ Notes: Entries denote rejection frequencies of ($S_n^{G+}$,$S_n^{G-}$) tests (i.e., GL forecast superiority) and ($S_n^{C+}$,$S_n^{C-}$) tests (i.e., CL forecast superiority) under a variety of data generating processes 
denoted by DGP1-DGP8. In DGP1-DGP3, no alternative outperforms the benchmark model. In DGP4-DGP8, at least one alternative model outperfroms the benchmark model.
Sample sizes include $n$=250, 500, and 1000 observations, as indicated in the second column of entries in the table. Nominal test size is 10\%, and tests are carried out using critical values 
constructed for values of $\eta$ including 0.045, 0.060, 0.075, and 0.090. See Section 5 for complete details.}
\end{minipage}
\end{table}

\begin{table}[tbp]
\caption{Forecast Superiority Test Results Part I: Do Years of Experience
Matter for Mean SPF Forecasting of GDP Growth Rates?$^*$ }{\centering}
\par
\hspace{1cm}
\par
\begin{tabular}{ccccccccc}
\hline\hline
Test & \multicolumn{2}{c}{$S_n^G$} & \multicolumn{2}{c}{$S_n^C$} & 
\multicolumn{2}{c}{$JCS_n^G$} & \multicolumn{2}{c}{$JCN_n^C$} \\ 
& $S_n^{G+}$ & $S_n^{G-}$ & $S_n^{C+}$ & $S_n^{C-}$ & $JCS_n^{G+}$ & $%
JCS_n^{G-}$ & $JCS_n^{C+}$ & $JCS_n^{C-}$ \\ \hline\hline
\multicolumn{1}{l}{$Panel$ $A:$ $Forecast$ $Horizon:$ $h$=0} &  &  &  &  & 
&  &  &  \\ \hline
Statistic Values & 0.3186 & 0.1979 & 0.3972 & 0.1735 & 0.2268 & 0.0756 & 
0.0003 & 0.0002 \\ 
Rejection with $p$-Value Type 1 & \multicolumn{2}{c}{no} & 
\multicolumn{2}{c}{no} & \multicolumn{2}{c}{no} & \multicolumn{2}{c}{no} \\ 
Rejection with $p$-Value Type 2 & \multicolumn{2}{c}{no} & 
\multicolumn{2}{c}{no} & \multicolumn{2}{c}{no} & \multicolumn{2}{c}{no} \\ 
\hline
\multicolumn{1}{l}{$Panel$ $B:$ $Forecast$ $Horizon:$ $h$=1} &  &  &  &  & 
&  &  &  \\ \hline
Statistics Values & 0.8773 & 0.4182 & 0.8726 & 0.1594 & 0.2274 & 0.3032 & 
0.0011 & 0.0002 \\ 
Rejection with $p$-Value Type 1 & \multicolumn{2}{c}{yes} & 
\multicolumn{2}{c}{yes} & \multicolumn{2}{c}{no} & \multicolumn{2}{c}{no} \\ 
Rejection with $p$-Value Type 2 & \multicolumn{2}{c}{no} & 
\multicolumn{2}{c}{no} & \multicolumn{2}{c}{no} & \multicolumn{2}{c}{no} \\ 
\hline
\multicolumn{1}{l}{$Panel$ $C:$ $Forecast$ $Horizon:$ $h$=2} &  &  &  &  & 
&  &  &  \\ \hline
Statistics Values & 0.1933 & 0.9600 & 0.0333 & 1.0000 & 0.8767 & 0.9833 & 
0.2900 & 0.9567 \\ 
Rejection with $p$-Value Type 1 & \multicolumn{2}{c}{no} & 
\multicolumn{2}{c}{no} & \multicolumn{2}{c}{no} & \multicolumn{2}{c}{no} \\ 
Rejection with $p$-Value Type 2 & \multicolumn{2}{c}{no} & 
\multicolumn{2}{c}{no} & \multicolumn{2}{c}{no} & \multicolumn{2}{c}{no} \\ 
\hline
\multicolumn{1}{l}{$Panel$ $D:$ $Forecast$ $Horizon:$ $h$=3} &  &  &  &  & 
&  &  &  \\ \hline
Statistics Values & 0.2733 & 0.2133 & 0.2100 & 0.7133 & 0.6700 & 0.8200 & 
0.5600 & 0.8233 \\ 
Rejection with $p$-Value Type 1 & \multicolumn{2}{c}{no} & 
\multicolumn{2}{c}{no} & \multicolumn{2}{c}{no} & \multicolumn{2}{c}{no} \\ 
Rejection with $p$-Value Type 2 & \multicolumn{2}{c}{no} & 
\multicolumn{2}{c}{no} & \multicolumn{2}{c}{no} & \multicolumn{2}{c}{no} \\ 
\hline
\multicolumn{1}{l}{$Panel$ $E:$ $Forecast$ $Horizon:$ $h$=4} &  &  &  &  & 
&  &  &  \\ \hline
Statistics Values & 0.4092 & 0.2271 & 0.5519 & 0.0784 & 0.3881 & 0.0776 & 
0.0007 & 0.0002 \\ 
Rejection with $p$-Value Type 1 & \multicolumn{2}{c}{no} & 
\multicolumn{2}{c}{no} & \multicolumn{2}{c}{no} & \multicolumn{2}{c}{no} \\ 
Rejection with $p$-Value Type 2 & \multicolumn{2}{c}{no} & 
\multicolumn{2}{c}{no} & \multicolumn{2}{c}{no} & \multicolumn{2}{c}{no} \\ 
\hline\hline
\multicolumn{2}{l}{$Panel$ $F:$ $Root$ $Mean$ $Square$ $Forecast$ $Errors$}
&  &  &  &  &  &  &  \\ \hline
& \multicolumn{2}{c}{Benchmark} & \multicolumn{2}{c}{Alt Model 1} & 
\multicolumn{2}{c}{Alt Model 2} & \multicolumn{2}{c}{Alt Model 3} \\ 
$h=0$ & \multicolumn{2}{c}{0.00741} & \multicolumn{2}{c}{0.00740} & 
\multicolumn{2}{c}{0.00738} & \multicolumn{2}{c}{0.00742} \\ 
$h=1$ & \multicolumn{2}{c}{0.01248} & \multicolumn{2}{c}{0.01246} & 
\multicolumn{2}{c}{0.01243} & \multicolumn{2}{c}{0.01250} \\ 
$h=2$ & \multicolumn{2}{c}{0.01682} & \multicolumn{2}{c}{0.01681} & 
\multicolumn{2}{c}{0.01689} & \multicolumn{2}{c}{0.01694} \\ 
$h=3$ & \multicolumn{2}{c}{0.02088} & \multicolumn{2}{c}{0.02087} & 
\multicolumn{2}{c}{0.02103} & \multicolumn{2}{c}{0.02112} \\ 
$h=4$ & \multicolumn{2}{c}{0.02449} & \multicolumn{2}{c}{0.02447} & 
\multicolumn{2}{c}{0.02461} & \multicolumn{2}{c}{0.02528} \\ \hline\hline
\end{tabular}%
\par
\begin{minipage}{1\columnwidth}
\vspace{0.3in}
\noindent $^{*}$ Notes: Numerical entries in Panels A-E of this table are forecast superiority test statistics, for 
$S_n^{G+}$, $S_n^{G-}$, $S_n^{C+}$, $S_n^{C-}$, $JCS_n^{G+}$, $JCS_n^{G-}$, $JCS_n^{C+}$, and $JCS_n^{C-}$ tests. 
The benchmark model is the arithmetic mean of all survey participants. Three alternative models are considered, including the mean of all participants with each of 1
year, 2 years, and 3 years of experience. These are called Alt Models 1, 2, and 3, respectively, in Panel F of the table.
Test critical values are calculated using $\eta$ = 0.075 and 0.090 
(for $S_n^{G+}$, $S_n^{G-}$, $S_n^{C+}$, $S_n^{C-}$ tests), or using $J_n$ = 0.20 and 0.35
(for $JCS_n^{G+}$, $JCS_n^{G-}$, $JCS_n^{C+}$, and $JCS_n^{C-}$ tests). Nominal test size is 10\%. Test outcomes are reported under the 
heading ``$p$-Value Type 1'' (for $\eta$ = 0.075 and $J_n$ = 0.20), and 
``$p$-Value Type 2'' (for $\eta$ = 0.090 and $J_n$ = 0.35). ``No'' denotes failure to reject the null hypothesis, while ``yes'' denotes rejection of the null. 
Results are reported for forecast horizons $h$=0 to $h$=4 ($h$=0 denotes ``nowcasts'').
Additionally, in Panel F, root mean square forecast errors for all models (i.e., the benchmark and each of the three alternative models) are reported,
for each forecast horizon. See Section 6 for complete details.
\end{minipage}
\end{table}

\begin{table}[tbp]
\caption{Forecast Superiority Test Results Part II: Do Years of Experience
Matter for Median SPF Forecasting of GDP Growth Rates?$^*$ }{\centering}
\par
\hspace{1cm}
\par
\begin{tabular}{ccccccccc}
\hline\hline
Test & \multicolumn{2}{c}{$S_n^G$} & \multicolumn{2}{c}{$S_n^C$} & 
\multicolumn{2}{c}{$JCS_n^G$} & \multicolumn{2}{c}{$JCN_n^C$} \\ 
& $S_n^{G+}$ & $S_n^{G-}$ & $S_n^{C+}$ & $S_n^{C-}$ & $JCS_n^{G+}$ & $%
JCS_n^{G-}$ & $JCS_n^{C+}$ & $JCS_n^{C-}$ \\ \hline\hline
\multicolumn{1}{l}{$Panel$ $A:$ $Forecast$ $Horizon:$ $h$=0} &  &  &  &  & 
&  &  &  \\ \hline
Statistic Values & 0.3615 & 0.1597 & 0.2546 & 0.0117 & 0.1512 & 0.1512 & 
0.0003 & 0.0000 \\ 
Rejection with $p$-Value Type 1 & \multicolumn{2}{c}{no} & 
\multicolumn{2}{c}{no} & \multicolumn{2}{c}{no} & \multicolumn{2}{c}{no} \\ 
Rejection with $p$-Value Type 2 & \multicolumn{2}{c}{no} & 
\multicolumn{2}{c}{no} & \multicolumn{2}{c}{no} & \multicolumn{2}{c}{no} \\ 
\hline
\multicolumn{1}{l}{$Panel$ $B:$ $Forecast$ $Horizon:$ $h$=1} &  &  &  &  & 
&  &  &  \\ \hline
Statistics Values & 0.4769 & 0.1954 & 1.4752 & 0.1506 & 0.3032 & 0.0758 & 
0.0017 & 0.0002 \\ 
Rejection with $p$-Value Type 1 & \multicolumn{2}{c}{no} & 
\multicolumn{2}{c}{no} & \multicolumn{2}{c}{no} & \multicolumn{2}{c}{no} \\ 
Rejection with $p$-Value Type 2 & \multicolumn{2}{c}{no} & 
\multicolumn{2}{c}{no} & \multicolumn{2}{c}{no} & \multicolumn{2}{c}{no} \\ 
\hline
\multicolumn{1}{l}{$Panel$ $C:$ $Forecast$ $Horizon:$ $h$=2} &  &  &  &  & 
&  &  &  \\ \hline
Statistics Values & 0.5554 & 0.0000 & 0.7708 & 0.0000 & 0.3041 & 0.0000 & 
0.0012 & -0.0001 \\ 
Rejection with $p$-Value Type 1 & \multicolumn{2}{c}{no} & 
\multicolumn{2}{c}{no} & \multicolumn{2}{c}{no} & \multicolumn{2}{c}{no} \\ 
Rejection with $p$-Value Type 2 & \multicolumn{2}{c}{no} & 
\multicolumn{2}{c}{no} & \multicolumn{2}{c}{no} & \multicolumn{2}{c}{no} \\ 
\hline
\multicolumn{1}{l}{$Panel$ $D:$ $Forecast$ $Horizon:$ $h$=3} &  &  &  &  & 
&  &  &  \\ \hline
Statistics Values & 0.5121 & 0.2350 & 0.0000 & 0.3050 & 0.1525 & 0.0018 & 
0.0000 &  \\ 
Rejection with $p$-Value Type 1 & \multicolumn{2}{c}{no} & 
\multicolumn{2}{c}{no} & \multicolumn{2}{c}{no} & \multicolumn{2}{c}{no} \\ 
Rejection with $p$-Value Type 2 & \multicolumn{2}{c}{no} & 
\multicolumn{2}{c}{no} & \multicolumn{2}{c}{no} & \multicolumn{2}{c}{no} \\ 
\hline
\multicolumn{1}{l}{$Panel$ $E:$ $Forecast$ $Horizon:$ $h$=4} &  &  &  &  & 
&  &  &  \\ \hline
Statistics Values & 0.4827 & 0.0757 & 0.1712 & 0.1222 & 0.3881 & 0.1552 & 
0.0016 & 0.0002 \\ 
Rejection with $p$-Value Type 1 & \multicolumn{2}{c}{no} & 
\multicolumn{2}{c}{no} & \multicolumn{2}{c}{no} & \multicolumn{2}{c}{no} \\ 
Rejection with $p$-Value Type 2 & \multicolumn{2}{c}{no} & 
\multicolumn{2}{c}{no} & \multicolumn{2}{c}{no} & \multicolumn{2}{c}{no} \\ 
\hline\hline
\multicolumn{2}{l}{$Panel$ $F:$ $Root$ $Mean$ $Square$ $Forecast$ $Errors$}
&  &  &  &  &  &  &  \\ \hline
& \multicolumn{2}{c}{Benchmark} & \multicolumn{2}{c}{Alt Model 1} & 
\multicolumn{2}{c}{Alt Model 2} & \multicolumn{2}{c}{Alt Model 3} \\ 
$h=0$ & \multicolumn{2}{c}{0.00745} & \multicolumn{2}{c}{0.00745} & 
\multicolumn{2}{c}{0.00746} & \multicolumn{2}{c}{0.00749} \\ 
$h=1$ & \multicolumn{2}{c}{0.01249} & \multicolumn{2}{c}{0.01249} & 
\multicolumn{2}{c}{0.01249} & \multicolumn{2}{c}{0.01241} \\ 
$h=2$ & \multicolumn{2}{c}{0.01698} & \multicolumn{2}{c}{0.01706} & 
\multicolumn{2}{c}{0.01699} & \multicolumn{2}{c}{0.01712} \\ 
$h=3$ & \multicolumn{2}{c}{0.02096} & \multicolumn{2}{c}{0.02102} & 
\multicolumn{2}{c}{0.02106} & \multicolumn{2}{c}{0.02117} \\ 
$h=4$ & \multicolumn{2}{c}{0.02470} & \multicolumn{2}{c}{0.02478} & 
\multicolumn{2}{c}{0.02473} & \multicolumn{2}{c}{0.02551} \\ \hline\hline
\end{tabular}%
\par
\begin{minipage}{1\columnwidth}
\vspace{0.3in}
\noindent $^{*}$ Notes: See notes to Table 3. The benchmark model is the median of all survey participants.
Three alternative models are considered, including the median of all participants with each of 1
year, 2 years, and 3 years of experience.
\end{minipage}
\end{table}

\begin{table}[tbp]
\caption{Forecast Superiority Test Results Part III: Should We Use Only the
Top Performer for Mean SPF Forecasting of GDP Growth Rates?$^*$ }{\centering}
\par
\hspace{1cm}
\par
\begin{tabular}{ccccccccc}
\hline\hline
Test & \multicolumn{2}{c}{$S_n^G$} & \multicolumn{2}{c}{$S_n^C$} & 
\multicolumn{2}{c}{$JCS_n^G$} & \multicolumn{2}{c}{$JCN_n^C$} \\ 
& $S_n^{G+}$ & $S_n^{G-}$ & $S_n^{C+}$ & $S_n^{C-}$ & $JCS_n^{G+}$ & $%
JCS_n^{G-}$ & $JCS_n^{C+}$ & $JCS_n^{C-}$ \\ \hline\hline
\multicolumn{1}{l}{$Panel$ $A:$ $Forecast$ $Horizon:$ $h$=0} &  &  &  &  & 
&  &  &  \\ \hline
Statistic Values & 0.6206 & 0.0460 & 0.1788 & 0.0059 & 0.6047 & 0.1512 & 
0.0014 & 0.0001 \\ 
Rejection with $p$-Value Type 1 & \multicolumn{2}{c}{no} & 
\multicolumn{2}{c}{no} & \multicolumn{2}{c}{no} & \multicolumn{2}{c}{no} \\ 
Rejection with $p$-Value Type 2 & \multicolumn{2}{c}{no} & 
\multicolumn{2}{c}{no} & \multicolumn{2}{c}{no} & \multicolumn{2}{c}{no} \\ 
\hline
\multicolumn{1}{l}{$Panel$ $B:$ $Forecast$ $Horizon:$ $h$=1} &  &  &  &  & 
&  &  &  \\ \hline
Statistics Values & 1.3935 & 0.0000 & 1.6737 & 0.0000 & 0.7581 & 0.0000 & 
0.0068 & -0.0002 \\ 
Rejection with $p$-Value Type 1 & \multicolumn{2}{c}{yes} & 
\multicolumn{2}{c}{no} & \multicolumn{2}{c}{no} & \multicolumn{2}{c}{no} \\ 
Rejection with $p$-Value Type 2 & \multicolumn{2}{c}{no} & 
\multicolumn{2}{c}{no} & \multicolumn{2}{c}{no} & \multicolumn{2}{c}{no} \\ 
\hline
\multicolumn{1}{l}{$Panel$ $C:$ $Forecast$ $Horizon:$ $h$=2} &  &  &  &  & 
&  &  &  \\ \hline
Statistics Values & 0.7014 & 0.0000 & 0.1704 & 0.0000 & 0.9123 & 0.0000 & 
0.0051 & -0.0004 \\ 
Rejection with $p$-Value Type 1 & \multicolumn{2}{c}{no} & 
\multicolumn{2}{c}{no} & \multicolumn{2}{c}{no} & \multicolumn{2}{c}{no} \\ 
Rejection with $p$-Value Type 2 & \multicolumn{2}{c}{no} & 
\multicolumn{2}{c}{no} & \multicolumn{2}{c}{no} & \multicolumn{2}{c}{no} \\ 
\hline
\multicolumn{1}{l}{$Panel$ $D:$ $Forecast$ $Horizon:$ $h$=3} &  &  &  &  & 
&  &  &  \\ \hline
Statistics Values & 0.5410 & 0.1039 & 0.3500 & 0.0026 & 0.6862 & 0.5337 & 
0.0080 & 0.0018 \\ 
Rejection with $p$-Value Type 1 & \multicolumn{2}{c}{no} & 
\multicolumn{2}{c}{no} & \multicolumn{2}{c}{no} & \multicolumn{2}{c}{no} \\ 
Rejection with $p$-Value Type 2 & \multicolumn{2}{c}{no} & 
\multicolumn{2}{c}{no} & \multicolumn{2}{c}{no} & \multicolumn{2}{c}{no} \\ 
\hline
\multicolumn{1}{l}{$Panel$ $E:$ $Forecast$ $Horizon:$ $h$=4} &  &  &  &  & 
&  &  &  \\ \hline
Statistics Values & 1.4847 & 0.1084 & 0.6600 & 0.0000 & 1.3038 & 0.3835 & 
0.0150 & -0.0004 \\ 
Rejection with $p$-Value Type 1 & \multicolumn{2}{c}{no} & 
\multicolumn{2}{c}{no} & \multicolumn{2}{c}{no} & \multicolumn{2}{c}{no} \\ 
Rejection with $p$-Value Type 2 & \multicolumn{2}{c}{yes} & 
\multicolumn{2}{c}{yes} & \multicolumn{2}{c}{no} & \multicolumn{2}{c}{no} \\ 
\hline\hline
\multicolumn{2}{l}{$Panel$ $F:$ $Root$ $Mean$ $Square$ $Forecast$ $Errors$}
&  &  &  &  &  &  &  \\ \hline
& \multicolumn{2}{c}{Benchmark} & \multicolumn{2}{c}{Alt Model 1} & 
\multicolumn{2}{c}{Alt Model 2} & \multicolumn{2}{c}{Alt Model 3} \\ 
$h=0$ & \multicolumn{2}{c}{0.00741} & \multicolumn{2}{c}{0.00882} & 
\multicolumn{2}{c}{0.00759} & \multicolumn{2}{c}{0.00764} \\ 
$h=1$ & \multicolumn{2}{c}{0.01248} & \multicolumn{2}{c}{0.01326} & 
\multicolumn{2}{c}{0.01300} & \multicolumn{2}{c}{0.01312} \\ 
$h=2$ & \multicolumn{2}{c}{0.01682} & \multicolumn{2}{c}{0.01776} & 
\multicolumn{2}{c}{0.01889} & \multicolumn{2}{c}{0.01858} \\ 
$h=3$ & \multicolumn{2}{c}{0.02088} & \multicolumn{2}{c}{0.02166} & 
\multicolumn{2}{c}{0.02192} & \multicolumn{2}{c}{0.02317} \\ 
$h=4$ & \multicolumn{2}{c}{0.02461} & \multicolumn{2}{c}{0.02618} & 
\multicolumn{2}{c}{0.02808} & \multicolumn{2}{c}{0.02620} \\ \hline\hline
\end{tabular}%
\par
\begin{minipage}{1\columnwidth}
\vspace{0.3in}
\noindent $^{*}$ Notes: See notes to Table 3. The benchmark model is the arithmetic mean of all survey participants.
Three alternative models are considered, including the top performer based on the comparison of
the mean of absolute forecast errors for all participants with each of 1 year, 2
years, and 3 years of experience.
\end{minipage}
\end{table}

\begin{table}[tbp]
\caption{Forecast Superiority Test Results Part IV: Should We Use Only the
Top Three Performers for Mean SPF Forecasting of GDP Growth Rates?$^*$ }{%
\centering}
\par
\hspace{1cm}
\par
\begin{tabular}{ccccccccc}
\hline\hline
Test & \multicolumn{2}{c}{$S_n^G$} & \multicolumn{2}{c}{$S_n^C$} & 
\multicolumn{2}{c}{$JCS_n^G$} & \multicolumn{2}{c}{$JCN_n^C$} \\ 
& $S_n^{G+}$ & $S_n^{G-}$ & $S_n^{C+}$ & $S_n^{C-}$ & $JCS_n^{G+}$ & $%
JCS_n^{G-}$ & $JCS_n^{C+}$ & $JCS_n^{C-}$ \\ \hline\hline
\multicolumn{1}{l}{$Panel$ $A:$ $Forecast$ $Horizon:$ $h$=0} &  &  &  &  & 
&  &  &  \\ \hline
Statistic Values & 1.5834 & 0.0869 & 2.3071 & 0.0211 & 1.1959 & 0.0104 & 
0.6412 & 0.0009 \\ 
Rejection with $p$-Value Type 1 & \multicolumn{2}{c}{yes} & 
\multicolumn{2}{c}{yes} & \multicolumn{2}{c}{no} & \multicolumn{2}{c}{no} \\ 
Rejection with $p$-Value Type 2 & \multicolumn{2}{c}{no} & 
\multicolumn{2}{c}{yes} & \multicolumn{2}{c}{no} & \multicolumn{2}{c}{no} \\ 
\hline
\multicolumn{1}{l}{$Panel$ $B:$ $Forecast$ $Horizon:$ $h$=1} &  &  &  &  & 
&  &  &  \\ \hline
Statistics Values & 3.1392 & 0.0362 & 1.7618 & 0.0956 & 0.9855 & 0.0758 & 
0.0073 & 0.0002 \\ 
Rejection with $p$-Value Type 1 & \multicolumn{2}{c}{yes} & 
\multicolumn{2}{c}{yes} & \multicolumn{2}{c}{no} & \multicolumn{2}{c}{no} \\ 
Rejection with $p$-Value Type 2 & \multicolumn{2}{c}{yes} & 
\multicolumn{2}{c}{yes} & \multicolumn{2}{c}{yes} & \multicolumn{2}{c}{yes}
\\ \hline
\multicolumn{1}{l}{$Panel$ $C:$ $Forecast$ $Horizon:$ $h$=2} &  &  &  &  & 
&  &  &  \\ \hline
Statistics Values & 1.6563 & 0.1096 & 4.0753 & 0.0171 & 0.7603 & 0.3801 & 
0.0098 & 0.0013 \\ 
Rejection with $p$-Value Type 1 & \multicolumn{2}{c}{yes} & 
\multicolumn{2}{c}{yes} & \multicolumn{2}{c}{yes} & \multicolumn{2}{c}{no}
\\ 
Rejection with $p$-Value Type 2 & \multicolumn{2}{c}{no} & 
\multicolumn{2}{c}{no} & \multicolumn{2}{c}{no} & \multicolumn{2}{c}{yes} \\ 
\hline
\multicolumn{1}{l}{$Panel$ $D:$ $Forecast$ $Horizon:$ $h$=3} &  &  &  &  & 
&  &  &  \\ \hline
Statistics Values & 1.3145 & 0.5176 & 1.1122 & 0.2517 & 0.8387 & 0.6862 & 
0.0106 & 0.0071 \\ 
Rejection with $p$-Value Type 1 & \multicolumn{2}{c}{yes} & 
\multicolumn{2}{c}{yes} & \multicolumn{2}{c}{no} & \multicolumn{2}{c}{no} \\ 
Rejection with $p$-Value Type 2 & \multicolumn{2}{c}{no} & 
\multicolumn{2}{c}{no} & \multicolumn{2}{c}{no} & \multicolumn{2}{c}{no} \\ 
\hline
\multicolumn{1}{l}{$Panel$ $E:$ $Forecast$ $Horizon:$ $h$=4} &  &  &  &  & 
&  &  &  \\ \hline
Statistics Values & 2.3830 & 0.3138 & 4.5114 & 0.0267 & 1.1504 & 0.5369 & 
0.0210 & 0.0045 \\ 
Rejection with $p$-Value Type 1 & \multicolumn{2}{c}{yes} & 
\multicolumn{2}{c}{yes} & \multicolumn{2}{c}{yes} & \multicolumn{2}{c}{yes}
\\ 
Rejection with $p$-Value Type 2 & \multicolumn{2}{c}{yes} & 
\multicolumn{2}{c}{no} & \multicolumn{2}{c}{yes} & \multicolumn{2}{c}{yes}
\\ \hline\hline
\multicolumn{2}{l}{$Panel$ $F:$ $Root$ $Mean$ $Square$ $Forecast$ $Errors$}
&  &  &  &  &  &  &  \\ \hline
& \multicolumn{2}{c}{Benchmark} & \multicolumn{2}{c}{Alt Model 1} & 
\multicolumn{2}{c}{Alt Model 2} & \multicolumn{2}{c}{Alt Model 3} \\ 
$h=0$ & \multicolumn{2}{c}{0.00741} & \multicolumn{2}{c}{0.00751} & 
\multicolumn{2}{c}{0.00719} & \multicolumn{2}{c}{0.00720} \\ 
$h=1$ & \multicolumn{2}{c}{0.01248} & \multicolumn{2}{c}{0.01249} & 
\multicolumn{2}{c}{0.01287} & \multicolumn{2}{c}{0.01260} \\ 
$h=2$ & \multicolumn{2}{c}{0.01682} & \multicolumn{2}{c}{0.01593} & 
\multicolumn{2}{c}{0.01705} & \multicolumn{2}{c}{0.01709} \\ 
$h=3$ & \multicolumn{2}{c}{0.02088} & \multicolumn{2}{c}{0.01994} & 
\multicolumn{2}{c}{0.02199} & \multicolumn{2}{c}{0.02152} \\ 
$h=4$ & \multicolumn{2}{c}{0.02461} & \multicolumn{2}{c}{0.02325} & 
\multicolumn{2}{c}{0.02426} & \multicolumn{2}{c}{0.02465} \\ \hline\hline
\end{tabular}%
\par
\begin{minipage}{1\columnwidth}
\vspace{0.3in}
\noindent $^{*}$ Notes: See notes to Table 3. The benchmark model is the arithmetic mean of all survey participants.
Three alternative models are considered, including the top three performers based on the comparison of the
mean of absolute forecast errors for all participants with each of 1 year, 2
years, and 3 years of experience.
\end{minipage}
\end{table}

\begin{table}[tbp]
\caption{Forecast Superiority Test Results Part V: Should We Use Only the
Top 10\% of Performers for Mean SPF Forecasting of GDP Growth Rates?$^*$ }{%
\centering}
\par
\hspace{1cm}
\par
\begin{tabular}{ccccccccc}
\hline\hline
Test & \multicolumn{2}{c}{$S_n^G$} & \multicolumn{2}{c}{$S_n^C$} & 
\multicolumn{2}{c}{$JCS_n^G$} & \multicolumn{2}{c}{$JCN_n^C$} \\ 
& $S_n^{G+}$ & $S_n^{G-}$ & $S_n^{C+}$ & $S_n^{C-}$ & $JCS_n^{G+}$ & $%
JCS_n^{G-}$ & $JCS_n^{C+}$ & $JCS_n^{C-}$ \\ \hline\hline
\multicolumn{1}{l}{$Panel$ $A:$ $Forecast$ $Horizon:$ $h$=0} &  &  &  &  & 
&  &  &  \\ \hline
Statistic Values & 1.5543 & 0.0804 & 1.1251 & 0.0551 & 0.8315 & 0.1512 & 
0.0029 & 0.0006 \\ 
Rejection with $p$-Value Type 1 & \multicolumn{2}{c}{yes} & 
\multicolumn{2}{c}{yes} & \multicolumn{2}{c}{no} & \multicolumn{2}{c}{no} \\ 
Rejection with $p$-Value Type 2 & \multicolumn{2}{c}{no} & 
\multicolumn{2}{c}{no} & \multicolumn{2}{c}{no} & \multicolumn{2}{c}{no} \\ 
\hline
\multicolumn{1}{l}{$Panel$ $B:$ $Forecast$ $Horizon:$ $h$=1} &  &  &  &  & 
&  &  &  \\ \hline
Statistics Values & 2.9458 & 0.0154 & 1.6952 & 0.0000 & 0.9097 & 0.1516 & 
0.0084 & -0.0001 \\ 
Rejection with $p$-Value Type 1 & \multicolumn{2}{c}{yes} & 
\multicolumn{2}{c}{yes} & \multicolumn{2}{c}{no} & \multicolumn{2}{c}{no} \\ 
Rejection with $p$-Value Type 2 & \multicolumn{2}{c}{yes} & 
\multicolumn{2}{c}{yes} & \multicolumn{2}{c}{yes} & \multicolumn{2}{c}{yes}
\\ \hline
\multicolumn{1}{l}{$Panel$ $C:$ $Forecast$ $Horizon:$ $h$=2} &  &  &  &  & 
&  &  &  \\ \hline
Statistics Values & 1.5726 & 0.0292 & 2.5697 & 0.0000 & 0.6843 & 0.2281 & 
0.0081 & -0.0003 \\ 
Rejection with $p$-Value Type 1 & \multicolumn{2}{c}{yes} & 
\multicolumn{2}{c}{yes} & \multicolumn{2}{c}{yes} & \multicolumn{2}{c}{yes}
\\ 
Rejection with $p$-Value Type 2 & \multicolumn{2}{c}{no} & 
\multicolumn{2}{c}{no} & \multicolumn{2}{c}{no} & \multicolumn{2}{c}{no} \\ 
\hline
\multicolumn{1}{l}{$Panel$ $D:$ $Forecast$ $Horizon:$ $h$=3} &  &  &  &  & 
&  &  &  \\ \hline
Statistics Values & 1.0083 & 0.6593 & 0.8208 & 0.5017 & 0.6100 & 0.8387 & 
0.0080 & 0.0114 \\ 
Rejection with $p$-Value Type 1 & \multicolumn{2}{c}{no} & 
\multicolumn{2}{c}{no} & \multicolumn{2}{c}{no} & \multicolumn{2}{c}{no} \\ 
Rejection with $p$-Value Type 2 & \multicolumn{2}{c}{no} & 
\multicolumn{2}{c}{no} & \multicolumn{2}{c}{no} & \multicolumn{2}{c}{no} \\ 
\hline
\multicolumn{1}{l}{$Panel$ $E:$ $Forecast$ $Horizon:$ $h$=4} &  &  &  &  & 
&  &  &  \\ \hline
Statistics Values & 2.1550 & 0.3916 & 2.7560 & 0.0479 & 1.4572 & 0.5369 & 
0.0229 & 0.0061 \\ 
Rejection with $p$-Value Type 1 & \multicolumn{2}{c}{yes} & 
\multicolumn{2}{c}{yes} & \multicolumn{2}{c}{yes} & \multicolumn{2}{c}{yes}
\\ 
Rejection with $p$-Value Type 2 & \multicolumn{2}{c}{yes} & 
\multicolumn{2}{c}{yes} & \multicolumn{2}{c}{yes} & \multicolumn{2}{c}{yes}
\\ \hline\hline
\multicolumn{2}{l}{$Panel$ $F:$ $Root$ $Mean$ $Square$ $Forecast$ $Errors$}
&  &  &  &  &  &  &  \\ \hline
& \multicolumn{2}{c}{Benchmark} & \multicolumn{2}{c}{Alt Model 1} & 
\multicolumn{2}{c}{Alt Model 2} & \multicolumn{2}{c}{Alt Model 3} \\ 
$h=0$ & \multicolumn{2}{c}{0.00741} & \multicolumn{2}{c}{0.00784} & 
\multicolumn{2}{c}{0.00708} & \multicolumn{2}{c}{0.00738} \\ 
$h=1$ & \multicolumn{2}{c}{0.01248} & \multicolumn{2}{c}{0.01225} & 
\multicolumn{2}{c}{0.01279} & \multicolumn{2}{c}{0.01304} \\ 
$h=2$ & \multicolumn{2}{c}{0.01682} & \multicolumn{2}{c}{0.01623} & 
\multicolumn{2}{c}{0.01758} & \multicolumn{2}{c}{0.01776} \\ 
$h=3$ & \multicolumn{2}{c}{0.02088} & \multicolumn{2}{c}{0.01973} & 
\multicolumn{2}{c}{0.02168} & \multicolumn{2}{c}{0.02203} \\ 
$h=4$ & \multicolumn{2}{c}{0.02461} & \multicolumn{2}{c}{0.02297} & 
\multicolumn{2}{c}{0.02618} & \multicolumn{2}{c}{0.02531} \\ \hline\hline
\end{tabular}%
\par
\begin{minipage}{1\columnwidth}
\vspace{0.3in}
\noindent $^{*}$ Notes: See notes to Table 3. The benchmark model is the arithmetic mean of all survey participants.
Three alternative models are considered, including the top 10\% of performers based on the comparison of the
mean of absolute forecast errors for all participants with each of 1 year, 2
years, and 3 years of experience.
\end{minipage}
\end{table}

\begin{table}[tbp]
\caption{Forecast Superiority Test Results Part VI: Should We Use Only the
Top 25\% of Performers for Mean SPF Forecasting of GDP Growth Rates?$^*$ }{%
\centering}
\par
\hspace{1cm}
\par
\begin{tabular}{ccccccccc}
\hline\hline
Test & \multicolumn{2}{c}{$S_n^G$} & \multicolumn{2}{c}{$S_n^C$} & 
\multicolumn{2}{c}{$JCS_n^G$} & \multicolumn{2}{c}{$JCN_n^C$} \\ 
& $S_n^{G+}$ & $S_n^{G-}$ & $S_n^{C+}$ & $S_n^{C-}$ & $JCS_n^{G+}$ & $%
JCS_n^{G-}$ & $JCS_n^{C+}$ & $JCS_n^{C-}$ \\ \hline\hline
\multicolumn{1}{l}{$Panel$ $A:$ $Forecast$ $Horizon:$ $h$=0} &  &  &  &  & 
&  &  &  \\ \hline
Statistic Values & 1.4563 & 0.1437 & 1.6961 & 0.0000 & 0.8315 & 0.2268 & 
0.0026 & 0.0000 \\ 
Rejection with $p$-Value Type 1 & \multicolumn{2}{c}{yes} & 
\multicolumn{2}{c}{no} & \multicolumn{2}{c}{yes} & \multicolumn{2}{c}{no} \\ 
Rejection with $p$-Value Type 2 & \multicolumn{2}{c}{no} & 
\multicolumn{2}{c}{yes} & \multicolumn{2}{c}{no} & \multicolumn{2}{c}{no} \\ 
\hline
\multicolumn{1}{l}{$Panel$ $B:$ $Forecast$ $Horizon:$ $h$=1} &  &  &  &  & 
&  &  &  \\ \hline
Statistics Values & 1.9542 & 0.0951 & 1.0744 & 0.2393 & 0.5307 & 0.1516 & 
0.0058 & 0.0003 \\ 
Rejection with $p$-Value Type 1 & \multicolumn{2}{c}{yes} & 
\multicolumn{2}{c}{yes} & \multicolumn{2}{c}{no} & \multicolumn{2}{c}{no} \\ 
Rejection with $p$-Value Type 2 & \multicolumn{2}{c}{no} & 
\multicolumn{2}{c}{no} & \multicolumn{2}{c}{yes} & \multicolumn{2}{c}{yes}
\\ \hline
\multicolumn{1}{l}{$Panel$ $C:$ $Forecast$ $Horizon:$ $h$=2} &  &  &  &  & 
&  &  &  \\ \hline
Statistics Values & 2.0036 & 0.2440 & 4.3577 & 0.3473 & 0.7603 & 0.3801 & 
0.0090 & 0.0025 \\ 
Rejection with $p$-Value Type 1 & \multicolumn{2}{c}{yes} & 
\multicolumn{2}{c}{yes} & \multicolumn{2}{c}{yes} & \multicolumn{2}{c}{yes}
\\ 
Rejection with $p$-Value Type 2 & \multicolumn{2}{c}{no} & 
\multicolumn{2}{c}{no} & \multicolumn{2}{c}{yes} & \multicolumn{2}{c}{yes}
\\ \hline
\multicolumn{1}{l}{$Panel$ $D:$ $Forecast$ $Horizon:$ $h$=3} &  &  &  &  & 
&  &  &  \\ \hline
Statistics Values & 2.5623 & 0.2940 & 4.2772 & 0.1036 & 0.9150 & 0.3812 & 
0.0142 & 0.0034 \\ 
Rejection with $p$-Value Type 1 & \multicolumn{2}{c}{yes} & 
\multicolumn{2}{c}{yes} & \multicolumn{2}{c}{yes} & \multicolumn{2}{c}{yes}
\\ 
Rejection with $p$-Value Type 2 & \multicolumn{2}{c}{yes} & 
\multicolumn{2}{c}{yes} & \multicolumn{2}{c}{yes} & \multicolumn{2}{c}{yes}
\\ \hline
\multicolumn{1}{l}{$Panel$ $E:$ $Forecast$ $Horizon:$ $h$=4} &  &  &  &  & 
&  &  &  \\ \hline
Statistics Values & 3.4393 & 0.4863 & 6.6957 & 0.0565 & 1.0738 & 0.4602 & 
0.0193 & 0.0054 \\ 
Rejection with $p$-Value Type 1 & \multicolumn{2}{c}{yes} & 
\multicolumn{2}{c}{yes} & \multicolumn{2}{c}{yes} & \multicolumn{2}{c}{yes}
\\ 
Rejection with $p$-Value Type 2 & \multicolumn{2}{c}{yes} & 
\multicolumn{2}{c}{yes} & \multicolumn{2}{c}{yes} & \multicolumn{2}{c}{yes}
\\ \hline\hline
\multicolumn{2}{l}{$Panel$ $F:$ $Root$ $Mean$ $Square$ $Forecast$ $Errors$}
&  &  &  &  &  &  &  \\ \hline
& \multicolumn{2}{c}{Benchmark} & \multicolumn{2}{c}{Alt Model 1} & 
\multicolumn{2}{c}{Alt Model 2} & \multicolumn{2}{c}{Alt Model 3} \\ 
$h=0$ & \multicolumn{2}{c}{0.00741} & \multicolumn{2}{c}{0.00736} & 
\multicolumn{2}{c}{0.00733} & \multicolumn{2}{c}{0.00731} \\ 
$h=1$ & \multicolumn{2}{c}{0.01248} & \multicolumn{2}{c}{0.01228} & 
\multicolumn{2}{c}{0.01267} & \multicolumn{2}{c}{0.01257} \\ 
$h=2$ & \multicolumn{2}{c}{0.01682} & \multicolumn{2}{c}{0.01592} & 
\multicolumn{2}{c}{0.01694} & \multicolumn{2}{c}{0.01728} \\ 
$h=3$ & \multicolumn{2}{c}{0.02088} & \multicolumn{2}{c}{0.01969} & 
\multicolumn{2}{c}{0.02127} & \multicolumn{2}{c}{0.02109} \\ 
$h=4$ & \multicolumn{2}{c}{0.02461} & \multicolumn{2}{c}{0.02343} & 
\multicolumn{2}{c}{0.02467} & \multicolumn{2}{c}{0.02444} \\ \hline\hline
\end{tabular}%
\par
\begin{minipage}{1\columnwidth}
\vspace{0.3in}
\noindent $^{*}$ Notes: See notes to Table 3. The benchmark model is the arithmetic mean of all survey participants.
Three alternative models are considered, including the top 25\% of performers based on the comparison of the
mean of absolute forecast errors for all participants with each of 1 year, 2
years, and 3 years of experience.
\end{minipage}
\end{table}
\end{center}

\end{document}
