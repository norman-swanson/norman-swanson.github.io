%2multibyte Version: 5.50.0.2953 CodePage: 936

\documentclass[11pt]{article}
%%%%%%%%%%%%%%%%%%%%%%%%%%%%%%%%%%%%%%%%%%%%%%%%%%%%%%%%%%%%%%%%%%%%%%%%%%%%%%%%%%%%%%%%%%%%%%%%%%%%%%%%%%%%%%%%%%%%%%%%%%%%%%%%%%%%%%%%%%%%%%%%%%%%%%%%%%%%%%%%%%%%%%%%%%%%%%%%%%%%%%%%%%%%%%%%%%%%%%%%%%%%%%%%%%%%%%%%%%%%%%%%%%%%%%%%%%%%%%%%%%%%%%%%%%%%
\usepackage{amssymb}
\usepackage{amsmath}
\usepackage{portland}
\usepackage{lscape}
\usepackage{geometry}

\setcounter{MaxMatrixCols}{10}
%TCIDATA{OutputFilter=LATEX.DLL}
%TCIDATA{Version=5.50.0.2953}
%TCIDATA{Codepage=936}
%TCIDATA{<META NAME="SaveForMode" CONTENT="1">}
%TCIDATA{BibliographyScheme=Manual}
%TCIDATA{Created=Tuesday, February 24, 2009 18:42:35}
%TCIDATA{LastRevised=Tuesday, February 28, 2012 19:49:29}
%TCIDATA{<META NAME="GraphicsSave" CONTENT="32">}
%TCIDATA{<META NAME="DocumentShell" CONTENT="Articles\SW\Lili">}
%TCIDATA{Language=American English}
%TCIDATA{CSTFile=article.cst}
%TCIDATA{PageSetup=72,72,72,72,0}
%TCIDATA{ComputeDefs=
%$k$
%}

%TCIDATA{AllPages=
%H=36
%F=36
%}


\newtheorem{theorem}{Theorem}
\newtheorem{acknowledgement}[theorem]{Acknowledgement}
\newtheorem{algorithm}[theorem]{Algorithm}
\newtheorem{axiom}[theorem]{Axiom}
\newtheorem{case}[theorem]{Case}
\newtheorem{claim}[theorem]{Claim}
\newtheorem{conclusion}[theorem]{Conclusion}
\newtheorem{condition}[theorem]{Condition}
\newtheorem{conjecture}[theorem]{Conjecture}
\newtheorem{corollary}[theorem]{Corollary}
\newtheorem{criterion}[theorem]{Criterion}
\newtheorem{definition}[theorem]{Definition}
\newtheorem{example}[theorem]{Example}
\newtheorem{exercise}[theorem]{Exercise}
\newtheorem{lemma}[theorem]{Lemma}
\newtheorem{notation}[theorem]{Notation}
\newtheorem{problem}[theorem]{Problem}
\newtheorem{proposition}[theorem]{Proposition}
\newtheorem{remark}[theorem]{Remark}
\newtheorem{solution}[theorem]{Solution}
\newtheorem{summary}[theorem]{Summary}
\newenvironment{proof}[1][Proof]{\noindent\textbf{#1.} }{\ \rule{0.5em}{0.5em}}
\input{tcilatex}
\geometry{left=1in,right=1in,top=1in,bottom=1in}

\begin{document}


\begin{center}
{\Huge Density and Conditional Distribution Based Specification Analysis}%
{\LARGE $^{\ast }$}

\bigskip \bigskip \bigskip 

{\large Diep Duong and Norman R. Swanson}$^{\ast}${\large \ }

{\large Rutgers University \bigskip }

{\large February} {\large 2012\bigskip \bigskip }
\end{center}

{\footnotesize The technique of using densities and conditional
distributions to carry out consistent specification testing and model
selection amongst multiple diffusion processes have received considerable
attention from\ both financial theoreticians and empirical econometricians
over the last two decades. One reason for this interest is that correct
specification of diffusion models describing dynamics of financial assets is
crucial for many areas in finance including equity and option pricing, term
structure modeling, and risk management, for example. In this paper, we
discuss\ advances to this literature introduced by Corradi and Swanson
(2005), who compare the cumulative distribution (marginal or joint) implied
by a hypothesized null model with corresponding empirical distributions of
observed data. We also outline and expand upon further testing results from
Bhardwaj, Corradi and Swanson (BCS: 2008) and Corradi and Swanson (2011). In
particular, parametric specification tests in the spirit of\ the conditional
Kolmogorov test of Andrews (1997) that rely on block bootstrap resampling
methods in order to construct test critical values are first discussed.
Thereafter, extensions due to BCS (2008) for cases\ where the functional
form of the conditional density is unknown are introduced, and related
continuous time simulation methods are introduced. Finally, we broaden our
discussion from single process specification testing to multiple process
model selection by discussing how to construct predictive densities and how
to compare the accuracy of predictive densities derived from alternative
(possibly misspecified) diffusion models. In particular, we generalize
simulation Steps outlined in Cai and Swanson (2011) to multifactor models
where the number of latent variables is larger than three. These final tests
can be thought of as continuous time generalizations of the discrete time} 
{\footnotesize \textquotedblleft reality check\textquotedblright\ test
statistics of White (2000), which are widely used in empirical finance (see
e.g. Sullivan, Timmermann and White (1999, 2001)).} {\footnotesize We finish
the chapter with an empirical illustration of model selection amongst
alternative short term interest rate models. }

\setcounter{page}{0} \thispagestyle{empty}

\bigskip\ 

\noindent\textit{Keywords}: multi-factor diffusion process, specification
test, out-of-sample forecasts, conditional distribution, model
selection,block bootstrap, jump process.

\_\_\_\_\_\_\_\_\_\_\_\_\_\_\_\_\_\_\_\_\_\_\_\_\_\_\_

{\small $^{\ast }$ }{\footnotesize Diep Duong, Department of Economics,
Rutgers University, 75 Hamilton Street, New Brunswick, NJ 08901, USA,
dduong@econ.rutgers.edu. Norman R. Swanson, Department of Economics, Rutgers
University, 75 Hamilton Street, New Brunswick, NJ 08901, USA,
nswanson@econ.rutgers.edu.\ The authors thanks the editor, Cheng-Few Lee for
many useful suggestions given during the writing of this paper. Duong and
Swanson would like to thank the Research Council at Rutgers University for
research support. }

\newpage

\baselineskip=20pt

\renewcommand{\baselinestretch}{1.5}\newpage

\section{Introduction}

The last three decades have provided a unique opportunity to observe
numerous interesting developments in finance, financial econometrics and
statistics. For example, although starting as a narrow sub-field, financial
econometrics has recently transformed itself into an important discipline,
equipping financial economic researchers and industry practitioners with
immensely helpful tools for estimation, testing and forecasting. One of
these developments has involved the development of \textquotedblleft state
of the art\textquotedblright\ consistent specification tests for continuous
time models, including not only the geometric Brownian motion process used
to describe the dynamics of asset returns (Merton (1973)), but also a myriad
of other diffusion models used in finance, such as the Ornstein-Uhlenbeck
process introduced by Vacisek (1977), the constant elastic volatility
process applied by Beckers (1980), the square root process due to Cox,
Ingersoll and Ross (CIR: 1985), the so called CKLS model by Chan, Karolyi,
Longstaff and Sanders (CKLS: 1992), various three factor models proposed
Chen (1996), stochastic volatility processes such as generalized CIR of
Andersen and Lund (1997), and the generic class of affine jump diffusion
processes discussed in Duffie, Pan and Singleton (2000).\footnote{%
For complete details, see Section 2.2.}

The plethora of available diffusion models allow decision makers to be
flexible when choosing a specification to be subsequently used in contexts
ranging from equity and option pricing, to term structure modeling and risk
management. Moreover, the use of high frequency data when estimating such
models, in continuous time contexts, allows investors to continuously update
their dynamic trading strategies in real-time.\footnote{%
For further discussion, see Duong and Swanson (2010, 2011).} However, for
statisticians and econometricians, the vast number of available models has
important implications for formalizing model selection and specification
testing methods. This has led to several key papers that have recently been
published in the area of parametric and non-parametric specification
testing. Most of the papers focus on the ongoing \textquotedblleft
search\textquotedblright\ for correct Markov and stationary models that
\textquotedblleft fit\textquotedblright\ historical data and associated
dynamics. In this literature, it is important to note that correct
specification of a joint distribution is not the same as that of a
conditional distribution, and hence the recent focus on conditional
distributions, given that most models have an interpretation as conditional
models. In summary, the key issue in the construction of \ model selection
and specification tests of conditional distributions is the fact that
knowledge of the transition density (or conditional distribution) in general
cannot be inferred from knowledge of the drift and variance terms of a
diffusion model. If the functional form of the density is available
parametrically, though, one can test the hypothesis of correct specification
of a diffusion via the probability integral transform approach of Diebold,
Gunther, and Tay (1998); the cross-spectrum approach of Hong (2001), Hong
and Li (2005) and Hong, Li, and Zhao (2007); the martingalization-type
Kolmogorov test of Bai (2003); or via the normality transformation
approaches of Bontemps and Meddahi (2005) and Duan (2003). Furthermore, if
the transition density is unknown, one can construct a non-parametric test
by comparing a kernel density estimator of the actual and simulated data,
for example, as in Altissimo and Mele (2009) and Thompson (2008); or by
comparing the conditional distribution of the simulated and the historical
data, as in Bhardwaj, Corradi, and Swanson (BCS: 2008). One can also use the
methods of A\"{\i}t-Sahalia (2002) and A\"{\i}t-Sahalia, Fan, and Peng
(2009), in which they compare closed form approximations of conditional
densities under the null, using data-driven kernel density estimates.

For clarity and ease of presentation, we categorize the above literature
into two areas. The first area, initiated by the seminal work of A\"{\i}%
t-Sahalia (1996) and later followed by Pritsker (1998) and Jiang (1998),
breaks new ground in the continuous time specification testing literature by
comparing marginal densities implied by hypothesized null models with
nonparametric estimates thereof. These sorts of tests examine one factor
specifications. The second area of testing, as initiated in Corradi and
Swanson (CS: 2005) does not look at densities. Instead, they compare
cumulative distributions (marginal, joint, or conditional) implied by a
hypothesized null model with corresponding empirical distributions. A
natural extension of these sorts of tests involves model selection amongst
alternative predictive densities associated with competing models. While CS
(2005) focus on cases where the functional form of the conditional density
is known, BCS (2008) use simulation methods to examine testing in cases
where the functional form of the conditional density is unknown. Corradi and
Swanson (CS: 2011) and Cai and Swanson (2011) take the analysis of BCS
(2008) on Step further, and focus on the comparison of out of sample
predictive accuracy of possibly misspecified diffusion models, when the
conditional distribution is not known in closed form (i.e., they
\textquotedblleft choose\textquotedblright\ amongst competing models based
on predictive density model performance). The \textquotedblleft
best\textquotedblright\ model is selected by constructing tests that compare
both predictive densities and/or predictive conditional confidence intervals
associated with alternative models

In this paper, we primarily focus our attention on the second area of the
model selection and testing literature.\footnote{%
For a recent survey on results in the first area of this literature, see A%
\"{\i}t-Sahalia (2007).} One feature of all of the tests that we shall
discuss is that, given that they are based on the comparison of CDFs, they
obtain parametric rates. Moreover, the tests can be used to evaluate single
and multiple factor and dimensional models, regardless of whether or not the
functional form of the conditional distribution is known. 

In addition to discussing simple diffusion process specification tests of CS
(2005), we discuss tests discussed in BCS (2008) and CS (2011), and provide
some generalizations and additional results. In particular, parametric
specification tests in the spirit of\ the conditional Kolmogorov test of
Andrews (1997) that rely on block bootstrap resampling methods in order to
construct test critical values are first discussed. Thereafter, extensions
due to BCS (2008) for cases\ where the functional form of the conditional
density is unknown are introduced, and related continuous time simulation
methods are introduced. Finally, we broaden our discussion from single
dimensional specification testing to multiple dimensional selection by
discussing how to construct predictive densities and how to compare the
accuracy of predictive densities derived from alternative (possibly
misspecified) diffusion models as in CS (2011). In addition, we generalize
simulation and testing procedures introduced in Cai and Swanson (2011) to
more complicated multi-factor and multi-dimensional models where the number
of latent variables larger than three. These final tests can be thought of
as continuous time generalizations of the discrete time \textquotedblleft
reality check\textquotedblright\ test statistics of White (2000), which are
widely used in empirical finance (see e.g. Sullivan, Timmermann and White
(1999, 2001)). We finish the chapter with an empirical illustration of model
selection amongst alternative short term interest rate models, drawing on
BCS (2008), CS (2011) and Cai and Swanson (2011).

Of the final note is that the test statistics discussed here are implemented
via use of simple bootstrap methods for critical value simulation. We use
the bootstrap because the covariance kernels of the (Gaussian) asymptotic
limiting distributions of the test statistics are shown to contain terms
deriving from both the contribution of recursive parameter estimation error
(PEE) and the time dependence of data. Asymptotic critical value thus cannot
be tabulated in a usual way. Several methods can easily be implemented in
this context. First one can use block bootstrapping procedures, as discussed
below. Second one can use the conditional p-value approach of Corradi and
Swanson (2002) which extends the work of Hansen (1996) and Inoue (2001) to
the case of non vanishing parameter estimation error. Third is the
subsampling method of Politis, Romano and Wolf (1999), which has clear
efficiency \textquotedblleft costs\textquotedblright , but is easy
implement. Use of the latter two methods yields simulated (or subsample
based) critical values that diverge at rate equivalent to the blocksize
length under the alternative. This is the main drawback to their use in our
context. We therefore focus on use of a block bootstrap that mimics the
contribution of parameter estimation error in a recursive setting and in the
context of time series data. In general, use of the block bootstrap approach
is made feasible by establishing consistency and asymptotic normality of
both simulated generalized method of moments (SGMM) and nonparametric
simulated quasi maximum likelihood (NPSQML) estimators of (possibly
misspecified) diffusion models, in a recursive setting, and by establishing
the first-order validity of their bootstrap analogs.

The rest of the paper is organized as follows. In Section 2, we present our
setup, and discuss various diffusion models used in finance and financial
econometrics. Section 3 outlines the specification testing hypotheses,
presents the cumulative distribution based test statistics for one factor
and multiple factor models, discusses relevant procedures for simulation and
estimation, and outlines bootstrap techniques that can be used for critical
value tabulation. In Section 4, we present a small empirical illustration.
Section 5 summarizes and concludes.

\section{Setup}

\subsection{Diffusion Models in Finance and Financial Econometrics}

For the past two decades, continuous time models have taken center stage in
the field of financial econometrics, particularly in the context of
structural modeling, option pricing, risk management, and volatility
forecasting. One key advantage of continuous time models is that they allow
financial econometricians to use the full information set that is available.
With the availability of high frequency data and current computation
capability, one can update information, model estimates, and predictions in
milliseconds. In this Section we will summarize some of the standard models
that have been used in asset pricing as well as term structure modelling.
Generally, assume that financial asset returns follow Ito-semimartingale
processes with jumps, which are the solution to the following stochastic
differential equation system.%
\begin{equation}
X(t_{-})=\int_{0}^{t}b(X(s_{-}),\theta _{0})ds-\lambda _{0}t\int_{Y}y\phi
(y)dy+\int_{0}^{t}\sigma (X(s_{-}),\theta _{0})dW(s)+\sum_{j=1}^{J_{t}}y_{j},
\label{difmod1}
\end{equation}%
where $X(t_{-})$ is a cadlag process (right continuous with left limit) for $%
t$ $\in \Re ^{+},$ and is an $N$ dimensional vector of variables, $W(t)$ is
an $N-$dimensional Brownian motion, $b(\cdot )$ is $N-$dimensional function
of $X(t_{-}),$ and $\sigma (\cdot )$ is an $N$x$N$ matrix-valued function of 
$X(t_{-}),$ where $\theta _{0}$ is an unknown true parameter. $J_{t}$ is a
Poisson process with intensity parameter $\lambda _{0},$ $\lambda _{0}$
finite, and the $N-$dimensional jump size, $y_{j}$, is $iid$ with marginal
distribution given by $\phi .$ Both $J_{t}$ and $y_{j}$ are assumed to be
independent of the driving Brownian motion, $W\left( t\right) $.\footnote{%
Hereafter, $X(t_{-})$ denotes the cadlag, while $X_{t}$ denotes discrete
skeleton for $t=1,2,...$ .}Also, note that $\int_{Y}y\phi (y)dy$ denotes the
mean jump size, hereafter denoted by $\mu _{0}$. Over a unit time interval,
there are on average $\lambda _{0}$ jumps; so that over the time span $[0,t],
$ there are on average $\lambda _{0}t$ jumps. The dynamics of $X(t_{-})$ is
then given by:%
\begin{equation}
dX(t)=\left( b(X(t_{-}),\theta _{0})-\lambda _{0}\mu _{y,0}\right) dt+\sigma
(X(t_{-}),\theta _{0})dW(t)+\int_{Y}yp(dy,dt),  \label{difmod2}
\end{equation}%
where $p(dy,dt)$ is a random Poisson measure giving point mass at $y$ if a
jump occurs in the interval $dt$, and $b(\cdot ),\sigma (\cdot )$ are the
\textquotedblleft drift" and \textquotedblleft volatility" functions
defining the parametric specification of the model. Hereafter, the same (or
similar) notation is used throughout when models are specified.

Though not an exhaustive list, we review some popular models. Models are
presented with the "true" parameters.

\textbf{Diffusion Models Without Jumps: }

\textit{Geometric Brownian Motion (log normal model). }In this set-up, $%
b(X(t_{-}),\theta _{0})=b_{0}X(t)$ and $\sigma (X(t_{-}),\theta _{0})=\sigma
_{0}X(t)$%
\begin{equation*}
dX(t)=b_{0}X(t)dt+\sigma _{0}X(t)dW(t),
\end{equation*}%
where $b_{0}$ and $\sigma _{0}$ are constants and and $W(t)$ is a one
dimensional standard Brownian motion. (Below, other constants such as $%
\alpha _{0}$ , $\beta _{0},$ $\lambda _{0},$ $\gamma _{0},\delta _{0},$ $%
\eta _{0}$, $\kappa _{0}$, and $\Omega _{0}$\ are also used in model
specifications.)

This model is popular in the asset pricing literature. For example, one can
model equity prices according to this process, especially in the
Black-Scholes option set-up or in structured corporate finance.\footnote{%
See Black and Scholes (1973) for details.} The main drawback of this model
is that the return process (log(price)) has constant volatility, and is not
time varying. However, it is widely used as a convenient \textquotedblleft
first" econometric model.

\textit{Vasicek (1977) and Ornstein-Uhlenbeck process.} The process is used
to model asset prices, specifically in term structure modelling, and the
specification is:%
\begin{equation*}
dX(t)=(\alpha _{0}+\beta _{0}X(t))dt+\sigma _{0}dW(t)
\end{equation*}%
where $W(t)$ is a standard Brownian motion, and $\alpha _{0}$, $\beta _{0}$
and $\sigma _{0}$ are constants. $\beta _{0}$ is negative to ensure the mean
reversion of $X(t)$.

\textit{Cox, Ingersoll and Ross (1995)} use the following square root
process to model the term structure of interest rates:%
\begin{equation*}
dX(t)=\kappa (\alpha _{0}-X(t))dt+\sigma _{0}\sqrt{X(t)}dW(t)
\end{equation*}%
where $W(t)$ is a standard Brownian motion, $\alpha _{0}$ is the long-run
mean of $X(t),$ $\kappa $ measures the speed of mean-reversion, and $\sigma
_{0}$ is a standard deviation parameter and is assumed to be fixed. Also,
non-negativity of the process is imposed, as $2\kappa \beta _{0}>\sigma
_{0}^{2}.$

\textit{Wong (1964)} points out that in the CIR model, $X(t)$ with the
dynamics evolving according to:%
\begin{equation}
dX(t)=((\alpha _{0}-\lambda _{0})-X(t))dt+\sqrt{\alpha _{0}X(t)}dW(t),\text{ 
}\alpha _{0}>0\text{ and }\alpha _{0}-\lambda _{0}>0  \label{wong}
\end{equation}%
belongs to the linear exponential (or Pearson) family with a closed form
cumulative distribution. $\alpha _{0}$ and $\lambda _{0}$ are fixed
parameters of the model.

\textit{The Constant Elasticity of Variance}, or CEV model is specified as
follows: \ 
\begin{equation*}
dX(t)=\alpha _{0}X(t)dt+\sigma _{0}X(t)^{\beta _{0}/2}dW(t)
\end{equation*}%
where $W(t)$ is a standard Brownian motion and $\alpha _{0},\sigma _{0}$ and 
$\beta _{0}$ are fixed constants.

Of note is that the interpretation of this model depends on $\beta _{0},$
i.e. in the case of stock prices, if $\beta _{0}=2$, then the price process $%
X(t)$ follows a lognormal diffusion; if $\beta _{0}<2$ , then the model
captures exactly the leverage effect as price and volatility are inversely
correlated.

Among other authors, \textit{Beckers (1980)} uses this CEV model for stocks, 
\textit{Marsha and Rosenfeld (1983)} apply a CEV parametrization to interest
rates and \textit{Emanuel and Macbeth (1982)} utilize this set-up for option
pricing.

The \textit{Generalized constant elasticity of variance model is defined as
follows:}%
\begin{equation*}
dX(t)=(\alpha _{0}X(t)^{-(1-\beta _{0})}+\lambda _{0}X(t))dt+\sigma
_{0}X(t)^{\beta _{0}/2}dW(t)
\end{equation*}%
where the notation follows the CEV\ case. $\lambda _{0}$ is another
parameter of the model. This process nests log diffusion when $\beta _{0}=2,$
and nests square root diffusion when $\beta _{0}=1.$

\textit{Brennan \ and Schwartz (1979) and Courtadon (1982)} analyze the
model: 
\begin{equation*}
dX(t)=(\alpha _{0}+\beta _{0}X(t))dt+\sigma _{0}X(t)^{2}dW(t)
\end{equation*}%
where $\alpha _{0},\beta _{0},\sigma _{0}$ are fixed constants and $W(t)$ is
a standard Brownian motion.

\textit{Duffie and Kan (1993)} study the specification:%
\begin{equation*}
dX(t)=(\alpha _{0}-X(t))dt+\sqrt{\beta _{0}+\gamma _{0}X(t)}dW(t)
\end{equation*}%
where $W(t)$ is a standard Brownian motion and $\alpha _{0},\beta _{0}$ and $%
\gamma _{0}$ are fixed parameters.

\textit{A\"{\i}t-Sahalia (1996)} looks at a general case with general drift
and CEV diffusion: 
\begin{equation*}
dX(t)=(\alpha _{0}+\beta _{0}X(t)+\gamma _{0}X(t)^{2}+\eta
_{0}/X(t))dt+\sigma _{0}X(t)^{\beta _{0}/2}dW(t)
\end{equation*}%
In the above expression, $\alpha _{0},\beta _{0},\gamma _{0},\eta
_{0},\sigma _{0}$ and $\beta _{0}$ are fixed constants and $W(t)$ is again a
standard Brownian motion.

\textbf{Diffusion Models with Jumps}\textit{:}

For term structure modeling in empirical finance, the most widely studied
class of models is the family of affine processes, including diffusion
processes that incorporate jumps.

\textit{Affine Jump Diffusion Model}: $X(t_{-})$ is defined to follow an
affine jump diffusion if \ 
\begin{equation*}
dX(t)=\kappa _{0}(\alpha _{0}-X(t))dt+\Omega _{0}\sqrt{D(t)}dW(t)+dJ(t)
\end{equation*}%
where $X(t_{-})$ is an $N-$dimensional vector of variables of interest and
is a cadlag process, $W(t)$ is an $N-$dimensional independent standard
Brownian motion, $\kappa _{0}$ and $\Omega _{0}$ are square $N$ $\times $ $N$
matrices, $\alpha _{0}$ is a fixed long-run mean, $D(t)$ is a diagonal
matrix with $ith$ diagonal element given by%
\begin{equation*}
d_{ii}(t)=\theta _{0i}+\delta _{0i}^{\prime }X(t)
\end{equation*}

In the above expressions, $\theta _{0i}$ and $\delta _{0i}^{\prime }$ are
constants. The jump intensity is assumed to be a positive, affine function
of $X(t)$ and the jump size distribution is assumed to be determined by it's
conditional characteristic function. The attractive feature of this class of
affine jump diffusions is that, as shown in Duffie, Pan and Singleton
(2000), it has an exponential affine structure that can be derived in closed
form, i.e.%
\begin{equation*}
\Phi (X(t))=\exp (a(t)+b(t)^{\prime }X(t))
\end{equation*}

where the functions $a(t)$ and $b(t)$ can be derived from Riccati equations.%
\footnote{%
For details, see Singleton (2006), page 102.} Given a known characteristic
function, one can use either GMM to estimate the parameters of this jump
diffusion, or one can use quasi-maximum likelihood (QML), once the first two
moments are obtained. In the univariate case without jumps, as a special
case, this corresponds to the above general CIR\ model with jumps.

\textbf{Multifactor and} \textbf{Stochastic Volatility Model: }Multifactor
models have been widely used in the literature; particularly in option
pricing, term structure, and asset pricing.\ One general set-up has $%
(X(t),V(t))^{\prime }=\left( X(t),V^{1}(t),...,V^{d}(t)\right) ^{\prime }$
where only the first element, the diffusion process $X_{t},$ is observed
while $V(t)=(V^{1}(t),...,V^{d}(t))_{d\text{x}1}^{\prime }$ is latent. In
addition, $X(t)$ can be dependent on $V(t).$ For instance, in empirical
finance, the most well-known class of the multifactor models is the
stochastic volatility model expressed as:%
\begin{equation}
\left( 
\begin{array}{c}
dX(t) \\ 
dV(t)%
\end{array}%
\right) =\left( 
\begin{array}{c}
b_{1}(X(t),\theta _{0}) \\ 
b_{2}(V(t),\theta _{0})%
\end{array}%
\right) dt+\left( 
\begin{array}{c}
\sigma _{11}(V(t),\theta _{0}) \\ 
0%
\end{array}%
\right) dW_{1}(t)+\left( 
\begin{array}{c}
\sigma _{12}(V(t),\theta _{0}) \\ 
\sigma _{22}(V(t),\theta _{0})%
\end{array}%
\right) dW_{2}(t),  \label{SVI}
\end{equation}%
where $W_{1}(t)_{1\text{x}1}$ and $W_{2}(t)_{1\text{x}1}$ are independent
standard Brownian motions and $V(t)$ is latent volatility process. $%
b_{1}(\cdot )$ is a function of $X(t)$ and $b_{2}(\cdot ),\sigma _{11}(\cdot
),\sigma _{22}(\cdot )$ and $\sigma _{22}(\cdot )$ are general functions of $%
V(t),$ such that system of equations (\ref{SVI}) is well-defined. Popular
specifications are the square-root model of Heston (1993), the GARCH
diffusion model of Nelson (1990), lognormal model of Hull and White (1987)
and the eigenfunction models of Meddahi (2001). Note that in this stochastic
volatility case, the dimension of volatility is $d=1.$ More general set-up
can involve $d$ driving Brownian motions in $V(t)$ equation.

As an example, \textit{Andersen and Lund (1997)} study the generalized CIR\
model with stochastic volatility, specifically%
\begin{equation*}
dX(t)=\kappa _{x0}(\overline{x}_{0}-X(t))dt+\sqrt{V(t)}dW_{1}(t)
\end{equation*}%
\begin{equation*}
dV(t)=\kappa _{v0}(\overline{v}_{0}-V(t))dt+\sigma _{v0}\sqrt{V(t)}dW_{2}(t)
\end{equation*}%
where $X(t)$ and $V(t)$ are price and volatility processes, respectively, $%
\kappa _{x0},\kappa _{v0}$ $>0$ to ensure stationarity, $\overline{x}_{0}$
is the long-run mean of (log) price process, and $\overline{v}_{0}$ and $%
\sigma _{v0}$ are constants. $W_{1}(t)$ and $W_{2}(t)$ are scalar Brownian
motions. However, $W_{1}(t)$ and $W_{2}(t)$ are correlated such that $%
dW_{1}(t)dW_{2}(t)=\rho dt$ where the correlation $\rho $ is some constant $%
\rho \in \lbrack -1,1]$. Finally, note that volatility is a square-root
diffusion process, which requires that $\kappa _{v0}\overline{v}_{0}>\sigma
_{v0}^{2}.$

\textit{Stochastic Volatility Model with Jumps (SVJ):}\ A standard
specification is:\textbf{\ }%
\begin{eqnarray*}
dX\left( t\right) &=&\kappa _{x0}\left( \overline{x}_{0}-X(t)\right) dt+%
\sqrt{V(t)}dW_{1}\left( t\right) +J_{u}dq_{u}-J_{d}dq_{d}, \\
dV(t) &=&\kappa _{v0}\left( \overline{v}_{0}-V(t)\right) dt+\sigma _{v0}%
\sqrt{V(t)}dW_{2}\left( t\right) ,
\end{eqnarray*}%
where $q_{u}$ and $q_{d}$ are Poisson processes with jump intensity
parameters $\lambda _{u}$ and $\lambda _{d}$ respectively, and are
independent of the Brownian motions $W_{1}\left( t\right) $ and $W_{2}\left(
t\right) .$ In particular, $\lambda _{u}$ is the probability of a jump up, $%
\Pr \left( dq_{u}\left( t\right) =1\right) =\lambda _{u}$ and $\lambda _{d}$
is the probability of a jump down, $\Pr \left( dq_{d}\left( t\right)
=1\right) =\lambda _{d}.$ $J_{u}$ and $J_{d}$ are jump up and jump down
sizes and have exponential distributions: $f\left( J_{u}\right) =\frac{1}{%
\zeta _{u}}\exp \left( -\frac{J_{u}}{\zeta _{u}}\right) $ and $f\left(
J_{d}\right) =\frac{1}{\zeta _{d}}\exp \left( -\frac{J_{d}}{\zeta _{d}}%
\right) ,$ where $\zeta _{u},$ $\zeta _{d}>0$ are the jump magnitudes, which
are the means of the jumps, $J_{u}$ and $J_{d}.$

\textit{Three Factor Model (CHEN): }The three factor model combines various
features of the above models, by considering a version of the oft examined
3-factor model due to Chan, Karolyi, Longstaff and Sanders (1992), which is
discussed in detail in Dai and Singleton (2000). In particular, 
\begin{eqnarray}
dX\left( t\right) &=&\kappa _{x0}\left( \theta \left( t\right) -X\left(
t\right) \right) dt+\sqrt{V(t)}dW_{1}\left( t\right) ,  \notag \\
dV(t) &=&\kappa _{v0}\left( \overline{v}-V(t)\right) dt+\sigma _{v0}\sqrt{%
V(t)}dW_{2}\left( t\right) ,  \label{CHEN} \\
d\theta \left( t\right) &=&\kappa _{\theta 0}\left( \overline{\theta }%
_{0}-\theta \left( t\right) \right) dt+\sigma _{\theta 0}\sqrt{\theta \left(
t\right) }dW_{3}\left( t\right) ,  \notag
\end{eqnarray}%
where $W_{1}\left( t\right) ,$ $W_{2}\left( t\right) $ and $W_{3}\left(
t\right) $ are independent Brownian motions, and $V$ and $\theta $ are the
stochastic volatility and stochastic mean of $X(t)$, respectively. $\kappa
_{x0},\kappa _{v0},\kappa _{\theta 0},\overline{v_{0}},\overline{\theta _{0}}%
,\sigma _{v0},$ $\sigma _{\theta 0}$ are constants. As discussed above,
non-negativity for $V\left( t\right) $ and $\theta \left( t\right) $\
requires that $2\kappa _{v0}\overline{v}_{0}>\sigma _{v0}^{2}$\ and $2\kappa
_{\theta 0}\overline{\theta _{0}}>\sigma _{\theta 0}^{2}.$

\textit{Three Factor Jump Diffusion Model (CHENJ):}\textbf{\ }Andersen,
Benzoni and Lund (2004) extend the three factor Chen (1996) model by
incorporating jumps in the short rate process, hence improving the ability
of the model to capture the effect of outliers, and to address the finding
by Piazzesi (2004, 2005) that violent discontinuous movements in underlying
measures may arise from monetary policy regime changes. The model is defined
as follows:%
\begin{eqnarray}
dX\left( t\right) &=&\kappa _{x0}\left( \theta \left( t\right) -X\left(
t\right) \right) dt+\sqrt{V(t)}dW_{1}\left( t\right)
+J_{u}dq_{u}-J_{d}dq_{d},  \label{CHENJ} \\
dV(t) &=&\kappa _{v0}\left( \overline{v_{0}}-V(t)\right) dt+\sigma _{v0}%
\sqrt{V(t)}dW_{2}\left( t\right) ,  \notag \\
d\theta \left( t\right) &=&\kappa _{\theta 0}\left( \overline{\theta _{0}}%
-\theta \left( t\right) \right) dt+\sigma _{\theta 0}\sqrt{\theta \left(
t\right) }dW_{3}\left( t\right)
\end{eqnarray}

where all parameters are similar as in (\ref{CHEN}), $W_{1}\left( t\right) ,$
$W_{2}\left( t\right) $ and $W_{3}\left( t\right) $ are independent Brownian
motions, $q_{u}$ and $q_{d}$ are Poisson processes with jump intensities $%
\lambda _{u0}$ and $\lambda _{d0},$ respectively, and are independent of the
Brownian motions $W_{r}\left( t\right) $, $W_{v}\left( t\right) $ and $%
W_{\theta }\left( t\right) .$ In particular, $\lambda _{u0}$ is the
probability of a jump up, $\Pr \left( dq_{u}\left( t\right) =1\right)
=\lambda _{u0}$ and $\lambda _{d0}$ is the probability of a jump down, $\Pr
\left( dq_{d}\left( t\right) =1\right) =\lambda _{d0}.$ $J_{u}$ and $J_{d}$
are jump up and jump down sizes and have exponential distributions $f\left(
J_{u}\right) =\frac{1}{\zeta _{u0}}\exp \left( -\frac{J_{u}}{\zeta _{u0}}%
\right) $ and $f\left( J_{d}\right) =\frac{1}{\zeta _{d0}}\exp \left( -\frac{%
J_{d}}{\zeta _{d0}}\right) ,$ where $\zeta _{u0},$ $\zeta _{d0}>0$ are the
jump magnitudes, which are the means of the jumps $J_{u}$ and $J_{d}.$

\subsection{Overview on Specification Tests and Model Selection }

The focus in this paper is specification testing and model selection. The
\textquotedblleft tools\textquotedblright\ used in this literature have been
long established. Several key classical contributions include the
Kolmogorov-Smirnov test (see e.g. Kolmogorov (1933) and Smirnov (1939)),
various results on empirical processes (see e.g. Andrews (1993) and the
discussion in chapter 19 of van der Vaart (1998) on the contributions of
Glivenko, Cantelli, Doob, Donsker and others), the probability integral
transform (see e.g. Rosenblatt (1952)), and the Kullback-Leibler Information
Criterion (see e.g. White (1982) and Vuong (1989)). For illustration, the
empirical distribution mentioned above is crucial in our discussion of
predictive densities because it is useful in estimation, testing, and model
evaluation. Let $Y_{t}$ is a variable of interest with distribution $F$ and
parameter $\theta _{0}$. The theory of empirical distributions provides a
result that%
\begin{equation*}
\frac{1}{\sqrt{T}}\sum_{t=1}^{T}(1\left\{ Y_{t}\leq u\right\} -F(u|\theta
_{0}))
\end{equation*}%
satisfies a central limit theorem (with a parametric rate) if $T$ is large
(i.e., asymptotically). In the above expression, $1\left\{ Y_{t}\leq
u\right\} $ is the indicator function which takes value $1$ if $Y_{t}\leq u$
and $0$ otherwise. In the case where there is parameter estimation error, we
can use more general results in chapter 19 of van der Vaart (1998). Define%
\begin{equation*}
P_{T}(f)=\frac{1}{T}\sum_{i=1}^{T}f(Y_{i})\text{ and }P(f)=\int fdP
\end{equation*}

where $P$ is a probability measure associated with $F.$ Here, $P_{n}(f)$
converges to $P(f)$ almost surely for all the measurable functions $f$ for
which $P(f)$ is defined .\ Suppose one wants to test the null hypothesis
that $P$ belongs to a certain family $\{P_{\theta _{0}}:\theta _{0}\in
\Theta \},$ where $\theta _{0}$ is unknown; it is natural to use a measure
of the discrepancy between $P_{n}$ and $P_{\widehat{\theta }}$ for a
reasonable estimator $\widehat{\theta }_{t}$ of $\theta _{0}.$ In
particular, if $\widehat{\theta }_{t}$ converges to $\theta _{0}$ at a root-$%
T$ rate, $\frac{1}{\sqrt{T}}(P_{T}-P_{\widehat{\theta }_{t}})$ has been
shown to satisfy a central limit theorem.\footnote{%
See Theorem 19.23 in van der Vaart (1998) for details.}

With regard to dynamic misspecification and parameter estimation error, the
approach discussed for the class of tests in this paper allows for the
construction of statistics that admit for dynamic misspecification under
both hypotheses. This differs from other classes of tests such as the
framework used by Diebold, Gunther and Tay (DGT: 1998), Hong (2001), and Bai
(2003) in which correction dynamic specification under the null hypothesis
is assumed. In particular, DGT use the probability integral transform to
show that $F_{t}(Y_{t}|\Im _{t-1},\theta _{0})=\int_{-\infty
}^{Y_{t}}f_{t}(y|\Im _{t-1},\theta _{0})dy$ is identically and independently
distributed as a uniform random variable on $[0;1]$, where $F_{t}(\cdot )$
and $f_{t}$ $(\cdot )$ are a parametric distribution and density with
underlying parameter $\theta _{0}$, $Y_{t}$ is again our random variable of
interest, and $\Im _{t}$ is the information set containing all
\textquotedblleft relevant\textquotedblright\ past information. They thus
suggest using the difference between the empirical distribution of $%
F_{t}(Y_{t}|\Im _{t-1},\widehat{\theta }_{t})$ and the $45^{\circ }$ -
degree line as a measure of \textquotedblleft goodness of
fit\textquotedblright , where $\widehat{\theta }_{t}$ is some estimator of $%
\theta _{0}$. This approach has been shown to be very useful for financial
risk management (see e.g. Diebold, Hahnand, Tay (1999)), as well as for
macroeconomic forecasting (see e.g. Diebold, Tay and Wallis (1998) and
Clements and Smith (2000,2002)). Similarly, Bai (2003) develops a Kolmogorov
type test of $F_{t}(Y_{t}|\Im _{t-1},\theta _{0})$ on the basis of the
discrepancy between $F_{t}(Y_{t}|\Im _{t-1},\widehat{\theta }_{t})$ and the
CDF of a uniform on $[0;1]$. As the test involves estimator $\widehat{\theta 
}_{t}$, the limiting distribution reflects the contribution of parameter
estimation error and is not nuisance parameter free. To overcome this
problem, Bai (2003) proposes a novel approach based on a martingalization
argument to construct a modified Kolmogorov test which has a nuisance
parameter free limiting distribution. This test has power against violations
of uniformity but not against violations of independence. Hong (2001)
proposes another related interesting test, based on the generalized
spectrum, which has power against both uniformity and independence
violations, for the case in which the contribution of parameter estimation
error vanishes asymptotically. If the null is rejected, Hong (2001) also
proposes a test for uniformity robust to non independence, which is based on
the comparison between a kernel density estimator and the uniform density.
Two features differentiate the tests surveyed in this paper from the tests
outlined in the other papers mentioned above. First, the tests discussed
here assume strict stationarity. Second, they allow for dynamic
misspecification under the null hypothesis. The second feature allows us to
obtain asymptotically valid critical values even when the conditioning
information set does not contain all of the relevant past history. More
precisely, assume that we are interested in testing for correct
specification, given a particular information set which may or may not
contain all of the relevant past information. This is important when a
Kolmogorov test is constructed, as one is generally faced with the problem
of defining $\Im _{t-1}.$ If enough history is not included, then there may
be dynamic misspecification. Additionally, finding out how much information
(e.g. how many lags) to include may involve pre-testing, hence leading to a
form of sequential test bias. By allowing for dynamic misspecification, such
pre-testing is not required. Also note that critical values derived under
correct specification given $\Im _{t-1}$ are not in general valid in the
case of correct specification given a subset of $\Im _{t-1}$. Consider the
following example. Assume that we are interested in testing whether the
conditional distribution of $Y_{t}|Y_{t-1}$ follows normal distribution $%
N(\alpha _{1}Y_{t-1},\sigma _{1})$. Suppose also that in actual fact the
\textquotedblleft relevant\textquotedblright\ information set has $\Im _{t-1}
$ including both $Y_{t-1}$and $Y_{t-2}$, so that the true conditional model
is $Y_{t}|\Im _{t-1}=$ $Y_{t}|Y_{t-1},$ $Y_{t-2}=N(\alpha _{1}Y_{t-1}+\alpha
_{2}Y_{t-2},\sigma _{2}).$ In this case, correct specification holds with
respect to the information contained in $X_{t-1}$; but there is dynamic
misspecification with respect to $Y_{t-1}$and $Y_{t-2}$. Even without taking
account of parameter estimation error, the critical values obtained assuming
correct dynamic specification are invalid, thus leading to invalid
inference. Stated differently, tests that are designed to have power against
both uniformity and independence violations (i.e. tests that assume correct
dynamic specification under the null) will reject; an inference which is
incorrect, at least in the sense that the \textquotedblleft
normality\textquotedblright\ assumption is not false. In summary, if one is
interested in the particular problem of testing for correct specification
for a given information set, then the approach of tests in this paper is
appropriate

\section{Consistent Distribution-Based Specification Tests and Predictive
Density Type Model Selection{\protect\Huge \ }for Diffusion Processes}

\subsection{One Factor Models}

In this Section we outline the set-up for the general class of one factor
jump diffusion specifications. All analysis carry through to the more
complicated case of multi-factor stochastic volatility models which we will
elaborate upon in the next Subsection. In the presentation of the tests, we
follow a view that all candidate models, either single or multiple
dimensional ones, are approximations of reality, and can thus be
misspecified. The issue of correct specification (or misspecification) of a
single model and the model selection test for choosing amongst multiple
competing models allow for this feature. 

To begin, fix the time interval $[0,T],$ consider\ a given single one factor
candidate model the same as (\ref{difmod1}), with the true parameters $%
\theta _{0},\lambda _{0},$ $\mu _{0\text{ }}$to be replaced by it's the
pseudo true analogs $\theta ^{\dagger },\lambda ,\mu ,$ respectively and $%
0\leq t\leq T$: 
\begin{equation*}
X(t_{-})=\int_{0}^{t}b(X(s_{-}),\theta ^{\dagger })ds-\lambda t\int_{Y}y\phi
(y)dy+\int_{0}^{t}\sigma (X(s_{-}),\theta ^{\dagger
})dW(s)+\sum_{j=1}^{J_{t}}y_{j},
\end{equation*}%
or 
\begin{equation}
dX(t-)=\left( b(X(t-),\theta ^{\dagger })-\lambda \mu \right) dt+\sigma
(X(t-),\theta ^{\dagger })dW(t)+\int_{Y}yp(dy,dt),  \label{base}
\end{equation}%
where variables are defined the same as in (\ref{difmod1}) and (\ref{difmod2}%
). Note that as the above model is the one factor version of (\ref{difmod1})
and (\ref{difmod2}) where the dimension of $X(t_{-})$ is $1$x$1$, $W(t)$ is
a one-dimensional standard Brownian motion and jump size, and $y_{j}$ is one
dimensional variable for all $j$. Also note that both $J_{t}$ and $y_{j}$
are assumed to be independent of the driving Brownian motion.

If the single model is correctly specified, then $b(X(t-),\theta ^{\dagger
})=b_{0}(X(t-),\theta _{0})$, $\sigma (X(t-),\theta ^{\dagger })=\sigma
_{0}(X(t-),\theta _{0}),$ $\lambda =\lambda _{0},$ $\mu $ $=\mu _{0\text{ }}$%
and $\phi =\phi _{0}$ where $b_{0}(X(t-),\theta _{0}),\sigma
_{0}(X(t-),\theta _{0}),\lambda _{0},\mu _{0\text{ }},\phi _{0}$ are unknown
and belong to the true specification.

Now consider a different case (not a single model) where $m$ candidate
models are involved. For model $k$ with $1\leq k\leq m,$ denote it's
corresponding specification to be $(b_{k}(X(t_{-}),\theta _{k}^{\dagger }),$ 
$\sigma _{k}(X(t_{-}),\theta _{k}^{\dagger }),$ $\lambda _{k},$ $\mu _{k},$ $%
\phi _{k}).$ Two scenarios immediate arise. Firstly, if the model $k$ is
correctly specified, then $b_{k}(X(t_{-}),\theta _{k}^{\dagger
})=b_{0}(X(t_{-}),\theta _{0}),$ $\sigma _{k}(X(t_{-}),\theta _{k}^{\dagger
})=\sigma _{0}(X(t_{-}),\theta _{0}),$ $\lambda _{k}=\lambda _{0},$ $\mu
_{k} $ $=\mu _{0\text{ }}$ and $\phi _{k}=\phi _{0}$ which are similar to
the case of a single model. In the second scenario, all the models are
likely to be misspecified and modelers are faced with the choice of
selecting the "best" one. This type of problem is well-fitted into the class
of accuracy assessment tests initiated earlier by Diebold and Mariano (1995)
or White (2000).

The tests discussed hereafter are Kolomogorov type tests based on the
construction of cumulative distribution functions (CDFs). In a few cases,
the CDF is known in closed form. For instance, for the simplified version of
the CIR\ model as in (\ref{wong}), $X(t)$ belongs to the linear exponential
(or Pearson) family\ with the gamma CDF of the form:\footnote{%
See Wong (1964) for details.}%
\begin{equation}
F(u,\alpha ,\lambda )=\frac{\int_{0}^{u}(\frac{\lambda }{2})^{-2(1-\alpha
/\lambda )-1}\exp (-x/(\frac{\lambda }{2}))dx}{\Gamma (2(1-\alpha /\lambda ))%
},\text{ where }\Gamma (x)=\int_{0}^{\infty }t^{x}\exp (-t)dt,  \label{wongd}
\end{equation}

and $\alpha ,\lambda $ are constants.

Furthermore, if we look at the pure diffusion process without jumps:%
\begin{equation}
dX(t)=b(X(t),\theta ^{\dagger })dt+\sigma (X(t),\theta ^{\dagger })dW(t),
\label{est_diff}
\end{equation}%
{\large \ } \ where $b(\cdot )$ and $\sigma =\allowbreak \sigma (\cdot )$
are drift and volatility functions, it is known that the stationary density,
say $f(x,\theta ^{\dagger }),$ associated with the invariant probability
measure can be expressed explicitly as:\footnote{%
See Karlin and Taylor (1981) for details.}%
\begin{equation*}
f(x,\theta ^{\dagger })=\frac{c(\theta ^{\dagger })}{\sigma ^{2}(x,\theta
^{\dagger })}\exp \left( \int^{x}\frac{2b(u,\theta ^{\dagger })}{\sigma
^{2}(u,\theta ^{\dagger })}du\right) 
\end{equation*}%
where $c(\theta ^{\dagger })$ is a constant ensuring that $f$ integrates to
one. The CDF, say $F(u,\theta ^{\dagger })=\int^{u}f(x,\theta ^{\dagger })dx,
$ can then be obtained using available numerical integration procedures.

However, in most cases, it is impossible to derive the CDFs in closed form.
To obtain a CDF in such cases, a more general approach is to use simulation.
Instead of estimating the CDF directly, simulation techniques estimates the
CDF indirectly utilizing it's generated sample paths and the theory of
empirical distributions. The specification of a specific diffusion process
will\ dictate the sample paths and thereby corresponding test outcomes.

Note that in the historical context, many early papers in this literature
are probability density-based. For example, in a seminal paper, Ait-Sahalia
(1996) compares the marginal densities implied by hypothesized null models
with nonparametric estimates thereof. Following the same framework of
correct specification tests, CS(2005) and BCS (2008), however, do not look
at densities. Instead, they compare the cumulative distribution (marginal or
joint) implied by a hypothesized null model with the corresponding empirical
distribution. While CS (2005) focus on the known unconditional distribution,
BCS (2008) look at the conditional simulated distributions. CS (2011) make
extensions to multiple models in the context of out of sample accuracy
assessment tests. This approach is somewhat novel to this continuous time
model testing literature.

Now suppose we observe a discrete sample path $X_{1,}X_{2},...,X_{T}$ (also
referred as skeletons).\footnote{%
As mentioned earlier, we follow CS (2005) by using notation $X(\cdot )$ when
defining continuous time processes and $X_{t}$ for a skeleton.} The
corresponding hypotheses can be set up as follows:

\textbf{Hypothesis 1: Unconditional Distribution Specification Test of a
Single Model}

$H_{0}:F(u,\theta ^{\dagger })=F_{0}(u,\theta _{0}),$ for all $u,$ a.s.

$H_{A}:\Pr \left( F(u,\theta ^{\dagger })-F_{0}(u,\theta _{0})\neq 0\right)
>0$, for some $u\in U,$ with non-zero Lebesgue measure.

where $F_{0}(u,\theta _{0})$ is the true cumulative distribution implied by
the above density, i.e. $F_{0}(u,\theta _{0})=\Pr (X_{t}\leq u)$. $%
F(u,\theta ^{\dagger })=\Pr \left( X_{t}^{\theta ^{\dagger }}\leq u\right) $
is the cumulative distribution of the proposed model. $X_{t}^{\theta
^{\dagger }}$ is a skeleton implied by model (\ref{base}).

\textbf{Hypothesis 2: Conditional Distribution Specification Test of A
Single Model}

$H_{0}:F_{\tau }(u|X_{t},\theta ^{\dagger })=F_{0,\tau }(u|X_{t},\theta
_{0}),$ for all $u,$ a.s.

$H_{A}:\Pr \left( F_{\tau }(u|X_{t},\theta ^{\dagger })-F_{0,\tau
}(u|X_{t},\theta _{0})\neq 0\right) >0$, for some $u\in U,$ with non-zero
Lebesgue measure.

where $F_{\tau }(u|X_{t},\theta ^{\dagger })=$ $\Pr \left( X_{t+\tau
}^{\theta ^{\dagger }}\leq u|X_{t}^{\theta ^{\dagger }}=X_{t}\right) $ is $%
\tau $-Step ahead conditional distributions and $t=1,...,T-\tau $. $%
F_{0,\tau }(u|X_{t},\theta _{0})$ is $\tau $-Step ahead true conditional
distributions .

\textbf{Hypothesis 3: Predictive Density Test for Choosing Amongst Multiple
Competing Models }

The null hypothesis is that no model can outperform model $1$ which is the
benchmark model.\footnote{%
See White (2000) for a discussion of a discrete time series analog to this
case, whereby point rather than density-based loss is considered; Corradi
and Swanson (2007b) for an extension of White (2000) that allows for
parameter estimation error; and Corradi and Swanson (2006a) for an extension
of Corradi and Swanson (2007b) that allows for the comparison of conditional
distributions and densities in a discrete time series context.}

$H_{0}:\max_{k=2,...,m}E_{X}\left( \left( F_{X_{1,t+\tau }^{\theta
_{1}^{\dagger }}(X_{t})}(u_{2})-F_{X_{1,t+\tau }^{\theta _{1}^{\dagger
}}(X_{t})}(u_{1})\right) -\left(
F_{0}(u_{2}|X_{t})-F_{0}(u_{1}|X_{t})\right) \right) ^{2}$

$\ \ \ \ \ \ -E_{X}\left( \left( F_{_{X_{k,t+\tau }^{\theta _{k}^{\dagger
}}(X_{t})}}(u_{2})-F_{X_{k,t+\tau }^{\theta _{k}^{\dagger
}}(X_{t})}(u_{1})\right) -\left(
F_{0}(u_{2}|X_{t})-F_{0}(u_{1}|X_{t})\right) \right) ^{2}$

$H_{A}:$ negation of $H_{0}^{\prime }$

where $F_{_{X_{k,t+\tau }^{\theta _{k}^{\dagger }}(X_{t})}}(u)=F_{k}^{\tau
}(u|X_{t},\theta _{k}^{\dagger })=P_{\theta _{k}^{\dagger }}^{\tau }\left(
X_{t+\tau }^{\theta _{k}^{\dagger }}\leq u|X_{t}^{\theta _{k}^{\dagger
}}=X_{t}\right) ,$ which is the conditional distribution of $X_{t+\tau },$
given $X_{t}$, and evaluated at $u$ under the probability law generated by
model $k.$ $X_{k,t+\tau }^{\theta _{k}^{\dagger }}(X_{t})$ with $1\leq \tau
\leq T-t$ is the skeleton implied by model $k$, parameter $\theta
_{k}^{\dagger }$ and initial value $X_{t}.$ Analogously, define $F_{0}^{\tau
}(u|X_{t},\theta _{0})=P_{\theta _{0}}^{\tau }(X_{t+\tau }\leq u|X_{t})$ to
be the \textquotedblleft true\textquotedblright\ conditional distribution.

Note that the three hypotheses expressed above apply exactly the same to the
case of multifactor diffusions. Now, before moving to the statistics
description Section, we briefly explain the intuitions in facilitating
construction of the tests:

In the first case (Hypothesis 1), CS (2005) construct a Kolomogorov type
test based on comparison of the empirical distribution and the unconditional
CDF implied by the specification of the drift, variance and jumps.
Specifically, one can look at the scaled difference between%
\begin{equation*}
F(u,\theta ^{\dagger })=\Pr \left( X_{t}^{\theta ^{\dagger }}\leq u\right)
=\int^{u}f(x,\theta ^{\dagger })dx
\end{equation*}%
and estimator of the true $F_{0}(u|X_{t},\theta _{0}),$ the empirical
distribution of $X_{t}$ defined as:%
\begin{equation*}
\frac{1}{T}\sum_{t=1}^{T}1\left\{ X_{t}\leq u\right\} 
\end{equation*}%
where $1\left\{ Y_{t}\leq u\right\} $ is indicator function which takes
value $1$ if $Y_{t}\leq u$ and $0$ otherwise.

Similarly for the second case of conditional distribution (Hypothesis 2),
the test statistic $V_{T}$ can be a measure of the distance between the $%
\tau -Step$ ahead conditional distribution of $X_{t+\tau }^{\theta ^{\dagger
}},$ given $X_{t}^{\theta ^{\dagger }}=X_{t},$ as:%
\begin{equation*}
F_{\tau }(u|X_{t},\theta ^{\dagger })=\Pr \left( X_{t+\tau }^{\theta
^{\dagger }}\leq u|X_{t}^{\theta ^{\dagger }}=X_{t}\right) ,
\end{equation*}%
to an estimator of the true $F_{0,\tau }(u|X_{t},\theta _{0}),$ the
conditional empirical distribution of $X_{t+\tau }$ conditional on the
initial value $X_{t}$ defined as:%
\begin{equation*}
\frac{1}{T-\tau }\sum_{t=1}^{T-\tau }1\left\{ X_{t+\tau }\leq u\right\} ,
\end{equation*}%
In the third case (Hypothesis 3), model accuracy is measured in terms of a
distributional analog of mean square error. As is commonplace in the
out-of-sample evaluation literature, the sample of $T$ observations is
divided into two subsamples, such that $T=R+P,$ where only the last $P$
observations are used for predictive evaluation. A $\tau -$Step ahead
prediction error under model $k$ is $1\{u_{1}\leq X_{t+\tau }\leq
u_{2}\}-\left( F_{k}^{\tau }(u_{2}|X_{t},\theta _{k}^{\dagger })-F_{k}^{\tau
}(u_{1}|X_{t},\theta _{k}^{\dagger })\right) $ where $2\leq k\leq m$ and
similarly for model $1$ by replacing index $k$ with index $1.$ Suppose we
can simulate $P-\tau $ paths of $\tau -$Step ahead skeleton\footnote{%
See Section 3.3.1 for model simulation details.} using $X_{t}$ as starting
values where $t=R,...,R+P-\tau ,$ from which we can construct a sample of $%
\allowbreak P-\tau $ prediction errors. Then, these prediction errors can be
used to construct a test statistic for model comparison. In particular,
model $1$ is defined to be more accurate than model $k$ if: 
\begin{eqnarray*}
&&E\left( \left( (F_{1}^{\tau }(u_{2}|X_{t},\theta _{1}^{\dagger
})-F_{1}^{\tau }(u_{1}|X_{t},\theta _{1}^{\dagger }))-\left( F_{0}^{\tau
}(u_{2}|X_{t},\theta _{0})-F_{0}^{\tau }(u_{1}|X_{t},\theta _{0})\right)
\right) ^{2}\right)  \\
&<&E\left( \left( (F_{k}^{\tau }(u_{2}^{\tau }|X_{t},\theta _{k}^{\dagger
})-F_{k}^{\tau }(u_{1}^{\tau }|X_{t},\theta _{k}^{\dagger }))-\left(
F_{0}^{\tau }(u_{2}|X_{t},\theta _{0})-F_{0}^{\tau }(u_{1}|X_{t},\theta
_{0})\right) \right) ^{2}\right) .
\end{eqnarray*}%
where $E(\cdot )$ is an expectation operator and $E\left( 1\{u_{1}\leq
X_{t+\tau }\leq u_{2}\}|X_{t}\right) =F_{0}^{\tau }(u_{2}|X_{t},\theta
_{0})-F_{0}^{\tau }(u_{1}|X_{t},\theta _{0}).$ Concretely, model $k$ is
worse than model $1$ if on average $\tau -$Step ahead prediction errors
under model $k$ is larger than that of model $1$.

Finally, it is important to point out some main features characterized by
all the three test statistics. Processes $X(t)$ hereafter is required to
satisfy the regular conditions, i.e. assumptions A1-A8 in CS (2011).
Regarding model estimation (in Section 3.3), $\theta ^{\dagger }$ and $%
\theta _{k}^{\dagger }$ are unobserved and need to be estimated. While CS
(2005), BCS (2008) utilize (recursive) Simulated General Method of Moments
(SGMM), CS (2011) make extension to (recursive) Nonparametric Simulated
Quasi Maximum Likelihood (NPSQML). For the unknown distribution and
conditional distribution, it will be pointed out in Section 3.3.2 that $%
F(u,\theta ^{\dagger })$, $F_{\tau }(u|X_{t},\theta ^{\dagger })$ and $%
F_{_{X_{k,t+\tau }^{\theta _{k}^{\dagger }}(X_{t})}}(u)$ can be replaced by
their simulated counterparts using the (recursive) SGMM and NPSQML parameter
estimators. In addition, test statistics converge to functional of Gaussian
processes with covariance kernels that reflect time dependence of the data
and the contribution of parameter estimation error (PEE). Limiting
distributions are not nuisance parameter free and critical values thereby
cannot be tabulated by the standard approach. All the tests discussed in
this paper rely on the bootstrap procedures for obtaining the asymptotically
valid critical values, which we will describe in Section 3.4.

\subsubsection{Unconditional Distribution Tests}

For one-factor diffusions, we outline the construction of unconditional test
statistics in the context where CDF is known in closed form. In order to
test the \textbf{Hypothesis 1}, consider the following statistic:%
\begin{equation*}
V_{T,N,h}^{2}=\dint\limits_{U}V_{T,N,h}^{2}(u)\pi (u)du,
\end{equation*}%
where%
\begin{equation*}
V_{T,N,h}=\frac{1}{\sqrt{T}}\dsum\limits_{t=1}^{T}\left( 1\{X_{t}\leq
u\}-F(u,\widehat{\theta }_{T,N,h})\right) 
\end{equation*}%
In the above expression, $U$ is a compact interval and $\dint\limits_{U}\pi
(u)du=1,$ $1\{X_{t}\leq u\}$ is again the indicator function which returns
value $1$ if $X_{t}\leq u$ and $0$ otherwise. Further, as defined in Section
3.3, $\widehat{\theta }_{T,N,h}$ hereafter is a simulated estimator where $T$
is sample size and $h$ is the discretization interval used in simulation. In
addition, with the abuse of notation, $N$ is a generic notation throughout
this paper, i.e. $N=L$, the length of each simulation path for (recursive)
SGMM and $N=M,$ the number of random draws (simulated paths) for (recursive)
NPQML estimator.\footnote{$M$ is often chosen to coincide with $S,$ the
number of simulated paths used when simulating distributions.} Also note in
our notation that as the above test is in sample specification test, the
estimator and the statistics are constructed using the entire sample, i.e. $%
\widehat{\theta }_{T,N,h}$.

It has been shown in CS (2005) that under regular conditions and if the
estimator is estimated by SGMM, the \ above statistics converges to a
functional of Gaussian process.\footnote{%
For details and the proof, see Theorem 1 in CS (2005).} In particular, pick
the choice $T,N\rightarrow \infty ,h\rightarrow 0,T/N\rightarrow 0$ and $%
Th^{2}\rightarrow 0$

Under the null,%
\begin{equation*}
V_{T,N,h}^{2}\rightarrow \int_{U}Z^{2}(u)\pi (u)
\end{equation*}%
where $Z$ is a Gaussian process with covariance kernel. Hence, the limiting
distribution of $V_{T,N,h}^{2}$ is a functional of a Gaussian process with a
covariance kernel that reflects both PEE\ and the time series nature of the
data. As $\widehat{\theta }_{T,N,h}$ is root-T consistent, PEE\ does not
disappear in the asymptotic covariance kernel.

Under $H_{A}$, there exists an $\varepsilon $ $>0$ such that%
\begin{equation*}
\lim_{T\rightarrow \infty }\Pr (\frac{1}{T}V_{T,N,h}^{2}>\varepsilon )=1
\end{equation*}

For the asymptotic critical value tabulation, we use the bootstrap
procedure. In order to establish validity of the block bootstrap under SGMM
with the presence of PEE, the simulated sample size should be chosen to grow
at a faster rate than the historical sample, i.e. $T/N$ $\rightarrow 0.$

Thus, we can follow Steps in appropriate bootstrap procedure in Section 3.4.
For instance, if the SGMM estimator is used, the bootstrap statistic is  
\begin{equation*}
V_{T,N,h}^{2\ast }=\dint\limits_{U}V_{T,N,h}^{2\ast }(u)\pi (u)du,
\end{equation*}%
where%
\begin{equation*}
V_{T,N,h}^{\ast }=\frac{1}{\sqrt{T}}\dsum\limits_{t=1}^{T}\left(
(1\{X_{t}^{\ast }\leq u\}-1\{X_{t}\leq u\})-(F(u,\widehat{\theta }%
_{T,N,h}^{\ast })-F(u,\widehat{\theta }_{T,N,h}))\right) .
\end{equation*}%
In the above expression, $\widehat{\theta }_{T,N,h}^{\ast }$ is the
bootstrap analog of $\widehat{\theta }_{T,N,h}$ and is estimated by the
bootstrap sample $X_{1}^{\ast },...,X_{T}^{\ast }$ (see Section 3.4)$.$ With
appropriate conditions, CS\ (2005) show that under the null, $%
V_{T,N,h}^{2\ast }$ has a well defined limiting distribution which coincides
with that of $V_{T,N,h}^{2}.$ We then can straightforwardly derive the
bootstrap critical value by following Step 1-5 Section 3.4. In particular,
in Step 5, the idea is to perform $B$ bootstrap replications ($B$ large) and
compute the percentiles of the empirical distribution of the $B$ bootstrap
statistics. Reject $H_{0}$ if $V_{T,N,h}^{2}$ is greater than the $(1-\alpha
)th-$percentile of this empirical distribution. Otherwise, do not reject $%
H_{0}.$

\subsubsection{Conditional Distribution Tests}

\textbf{Hypothesis 2} tests correct specification of the conditional
distribution, implied by a proposed diffusion model. In practice, the
difficulty arises from the fact that the functional form of neither $\tau $%
-Step ahead conditional distributions $F_{\tau }(u|X_{t},\theta ^{\dagger })$
nor $F_{0,\tau }(u|X_{t},\theta _{0})$ is unknown in most cases. Therefore,
BCS (2008) develop bootstrap specification test on the basis of simulated
distribution using the SGMM estimator.\footnote{%
In this paper, we assume that $X(\cdot )$ satisfies the regularity
conditions stated in CS (2011), i.e. assuptions A1-A8. Those conditions also
reflect requirements A1-A2 in BCS (2008). Note that, the SGMM estimator used
in BCS (2008) satisfies the root-N consistency condition that CS (2011)
impose on their parameter estimator (See Assumption 4).
\par
{}} With the important inputs leading to the test such as simulated
estimator, distribution simulation and bootstrap procedures to be presented
in the next Section\footnote{%
See Sections 3.3 and 3.4 for further details.}, the test statistic is
defined as:%
\begin{equation*}
Z_{T}=\sup_{u\times v\in U\times V}\left\vert Z_{T}(u,v)\right\vert 
\end{equation*}%
where%
\begin{equation*}
Z_{T}(u,v)=\frac{1}{\sqrt{T-\tau }}\sum_{t=1}^{T-\tau }\left( \frac{1}{S}%
\sum_{s=1}^{S}1\left\{ X_{s,t+\tau }^{\widehat{\theta }_{T,N,h}}\leq
u\right\} -1\{X_{t+\tau }\leq u\}\right) 1\left\{ X_{t}\leq v\right\} ,
\end{equation*}%
with $U$ and $V$ compact sets on the real line. $\widehat{\theta }_{T,N,h}$
is the simulated estimator using entire sample $X_{1,}...,X_{T}$ and\ $S$ is
the number of simulated replications used in the estimation of conditional
distributions as described in Section 3.3. If SGMM\ estimator is used
(similar to unconditional distribution case and the same as in BCS (2008)),
then $N=L$, where $L$ is the simulation length used in parameter estimation.

The above statistic is a simulation-based version of the conditional
Kolmogorov test of Andrews (1997), which compare the joint empirical
distribution 
\begin{equation*}
\frac{1}{T-\tau }\sum_{t=1}^{T-\tau }1\{X_{t+\tau }\leq u\}1\left\{
X_{t}\leq v\right\}
\end{equation*}%
with its semi-empirical/semi-parametric analog given by the product of 
\begin{equation*}
\frac{1}{T-\tau }\sum_{t=1}^{T-\tau }F_{0,\tau }(u|X_{t},\theta
_{0})1\left\{ X_{t}\leq v\right\} .
\end{equation*}%
Intuitively, if the null is not rejected, the metric distance between the
two should asymptotically disappear. In the simulation context with
parameter estimation error, the asymptotic limit of $Z_{T}$ however is a
nontrivial one.\ BCS (2008) show that with the proper choice of $T,N,S,h$,
i.e. $T,N,S,T^{2}/S\rightarrow \infty $ and $h,T/N,T/S,Nh,h^{2}T\rightarrow
0,$ then 
\begin{equation*}
Z_{T}\overset{d}{\rightarrow }\sup_{u\times v\in U\times V}|Z(u,v)|,
\end{equation*}%
where $Z(u,v)$ is a Gaussian process with a covariance kernel that
characterizes: 1) long-run variance we would have if we knew $F_{0,\tau
}(u|X_{1},\theta _{0})$; 2) the contribution of parameter estimation error;\
3) The correlation between the first two.

Furthermore, under $H_{A},$ there exists some $\varepsilon >0$ such that:%
\begin{equation*}
\lim_{P\rightarrow \infty }\Pr \left( \frac{1}{\sqrt{T}}Z_{T}>\varepsilon
\right) =1.
\end{equation*}%
\ As $T/S\rightarrow 0,$ the contribution of simulation error is
asymptotically negligible.\ The limiting distribution is not nuisance
parameter free and hence critical values cannot be tabulated directly from
it. The appropriate bootstrap statistic in this context is:%
\begin{equation*}
Z_{T}^{\ast }=\sup_{u\times v\in U\times V}\left\vert Z_{T}^{\ast
}(u,v)\right\vert ,
\end{equation*}%
where%
\begin{eqnarray*}
Z_{T}^{\ast }(u,v) &=&\frac{1}{\sqrt{T-\tau }}\sum_{t=1}^{T-\tau }\left( 
\frac{1}{S}\sum_{s=1}^{S}1\left\{ X_{s,t+\tau }^{\widehat{\theta }%
_{T,N,h}^{\ast }}\leq u\right\} -1\{X_{t+\tau }^{\ast }\leq u\}\right)
1\left\{ X_{t}^{\ast }\leq v\right\}  \\
&&-\frac{1}{\sqrt{T-\tau }}\sum_{t=1}^{T-\tau }\left( \frac{1}{S}%
\sum_{s=1}^{S}1\left\{ X_{s,t+\tau }^{\widehat{\theta }_{T,N,h}}\leq
u\right\} -1\{X_{t+\tau }\leq u\}\right) 1\left\{ X_{t}\leq v\right\} 
\end{eqnarray*}%
In the above expression, $\widehat{\theta }_{T,N,h}^{\ast }$ is the
bootstrap parameter estimated using the resampled data $X_{t}^{\ast }$ for $%
t=1,...,T-\tau $. $X_{s,t+\tau }^{\widehat{\theta }_{T,N,h}^{\ast }},$ $%
s=1,...,S$ and $t=1,...,T-\tau $ is the simulated data under $\widehat{%
\theta }_{T,N,h}^{\ast }$ and $X_{t}^{\ast },$ $t=1,...,T-\tau $ is a
resampled series constructed using standard block-bootstrap methods as
described in 3.4. Note that in the original paper, BCS (2008) propose
bootstrap SGMM estimator for conditional distribution of diffusion
processes. CS (2011) extend the test to the case of simulated recursive
NPSQML estimator.\ Regarding the generation of the empirical distribution of 
$Z_{T}^{\ast }$ (asthmatically the same as $Z_{T}),$ follow Step 1-5 in the
bootstrap procedure in Section 3.4.\ This yields $B$ bootstrap replications (%
$B$ large) of $Z_{T}^{\ast }$. One can then compare $Z_{T}$ with the
percentiles of the empirical distribution of $Z_{T}^{\ast },$ and reject $%
H_{0}$ if $Z_{T}$ is greater than the $(1-\alpha )th$-percentile. Otherwise,
do not reject  $H_{0}$. Tests carried out in this manner are correctly
asymptotically sized, and have unit asymptotic power.

\subsubsection{\textbf{Predictive Density Tests for Multiple Competing
Models }}

In many circumstances, one might want to compare one (benchmark) model
(model $1$) against multiple competing models (models $k,$ $2\leq k\leq m$).
In this case, recall in the null in \textbf{Hypothesis 3} is that no model
can outperform the benchmark model. In testing the null, we first choose a
particular interval i.e., $(u_{1},u_{2})\in U$x$U$ where $U$ is a compact
set so that the objective is evaluation of predictive densities for a given
range of values. In addition, in the recursive setting (not full sample is
used to estimate parameters), if we use the recursive NPSQML estimator, say $%
\widehat{\theta }_{1,t,N,h}$ and $\widehat{\theta }_{k,t,N,h},$ for models $1
$ and $k$, respectively, then the test statistic is defined as 
\begin{equation*}
D_{k,P,S}^{Max}(u_{1},u_{2})=\max_{k=2,...,m}D_{k,P,S}(u_{1},u_{2}).
\end{equation*}%
where%
\begin{eqnarray*}
&&D_{k,P,S}(u_{1},u_{2}) \\
&=&\frac{1}{\sqrt{P}}\sum_{t=R}^{T-\tau }\left( \left[ \frac{1}{S}%
\sum_{i=1}^{S}{\small 1}\left\{ u_{1}\leq X_{1,i,t+\tau }^{\widehat{\theta }%
_{1,t,N,h}}(X_{t})\leq u_{2}\right\} {\small -}1\{u_{1}\leq X_{t+\tau }\leq
u_{2}\}\right] ^{2}\right. 
\end{eqnarray*}%
\begin{equation*}
\left. -\left[ \frac{1}{S}\sum_{i=1}^{S}{\small 1}\left\{ u_{1}\leq
X_{k,i,t+\tau }^{\widehat{\theta }_{k,t,N,h}}(X_{t})\leq u_{2}\right\} 
{\small -}1\{u_{1}\leq X_{t+\tau }\leq u_{2}\}\right] ^{2}\right) .
\end{equation*}%
All notation is consistent with previous Sections where $S$ is the number of
simulated replications used in the estimation of conditional distributions. $%
X_{1,i,t+\tau }^{\widehat{\theta }_{1,t,N,h}}(X_{t})$ and $X_{k,i,t+\tau }^{%
\widehat{\theta }_{k,t,N,h}}$ , $i=1,...,S,$ $t=1,...,T-\tau ,$ are the $ith$
simulated path under $\widehat{\theta }_{1,t,N,h}$ and $\widehat{\theta }%
_{k,t,N,h}.$ If models $1$\ and $k$\ are nonnested for at least one $%
k=2,...,m$. Under regular conditions and if $P,R,S,h$ are chosen such as $%
P,R,N\rightarrow \infty $ and $h,P/N,h^{2}P\rightarrow 0$, $P/R\rightarrow
\pi $ where $\pi $ is finite then%
\begin{equation*}
\max_{k=2,..,m}\left( D_{k,P,N}(u_{1},u_{2})-\mu _{k}(u_{1},u_{2})\right) 
\overset{d}{\rightarrow }\max_{k=2,...,m}Z_{k}(u_{1},u_{2}),
\end{equation*}%
where, with an abuse of notation, $\mu _{k}(u_{1},u_{2})=\mu
_{1}(u_{1},u_{2})-\mu _{k}(u_{1},u_{2}),$ and%
\begin{equation*}
\mu _{j}(u_{1},u_{2})=E\left( \left( \left( F_{X_{j,t+\tau }^{\theta
_{j}^{\dagger }}(X_{t})}(u_{2})-F_{X_{j,t+\tau }^{\theta _{j}^{\dagger
}}(X_{t})}(u_{1})\right) -\left(
F_{0}(u_{2}|X_{t})-F_{0}(u_{1}|X_{t})\right) \right) ^{2}\right) ,
\end{equation*}%
for $j=1,...,m,$\ and where $(Z_{1}(u_{1},u_{2}),...,Z_{m}(u_{1},u_{2}))$\
is an $m-$dimensional Gaussian random variable the covariance kernels that
involves error in parameter estimation. Bootstrap statistics are therefore
required to reflect this parameter estimation error issue.\footnote{%
See CS (2011) for further discussion.}

In the implementation, we can obtain the asymptotic critical value using a
recursive version of the block bootstrap. The idea is that when forming
block bootstrap samples in the recursive setting, observations at the
beginning of the sample are used more frequently than observations at the
end of the sample. We can replicate the Step 1-5 in bootstrap procedure in
Section 3.4. It should be stressed the re-sampling in the Step $1$ is the
recursive one. Specifically, begin by resampling $b$ blocks of length $l$
from the full sample, with $lb=T.$ For any given\ $\tau ,$ it is necessary
to jointly resample\ $X_{t},X_{t+1},...,X_{t+\tau }.$ More precisely, let $%
Z^{t,\tau }=(X_{t},X_{t+1},...,X_{t+\tau }),$ $t=1,...,T-\tau .$ Now,
resample $b$ overlapping blocks of length $l$ from $Z^{t,\tau }.$ This
yields $Z^{t,\ast }=(X_{t}^{\ast },X_{t+1}^{\ast },...,X_{t+\tau }^{\ast }),$
$t=1,...,T-\tau .$ Use these data to construct bootstrap estimator $\widehat{%
\theta }_{k,t,N,h}^{\ast }$. Recall that $N$ is chosen in CS (2011) as the
number of simulated series used to estimate the parameters ($N=M=S$) and
such as $N/R,N/P\rightarrow \infty .$ Under this condition, simulation error
vanishes and there is no need to resample the simulated series.

CS (2011) show that 
\begin{equation*}
\frac{1}{\sqrt{P}}\sum_{t=R}^{T}\left( \widehat{\theta }_{k,t,N,h}^{\ast }-%
\widehat{\theta }_{k,t,N,h}\right)
\end{equation*}%
has the same limiting distribution as 
\begin{equation*}
\frac{1}{\sqrt{P}}\sum_{t=R}^{T}\left( \widehat{\theta }_{k,t,N,h}-\theta
_{k}^{\dagger }\right) ,
\end{equation*}%
conditional on all samples except a set with probability measure approaching
zero. Given this, the appropriate bootstrap statistic is:%
\begin{eqnarray*}
&&D_{k,P,S}^{\ast }(u_{1},u_{2}) \\
&=&\frac{1}{\sqrt{P}}\sum_{t=R}^{T-\tau }\left\{ \left( \left[ \frac{1}{S}%
\sum_{i=1}^{S}{\small 1}\left\{ u_{1}\leq X_{1,i,t+\tau }^{\widehat{\theta }%
_{1,t,N,h}^{\ast }}(X_{t}^{\ast })\leq u_{2}\right\} -1\{u_{1}\leq X_{t+\tau
}^{\ast }\leq u_{2}\}\right] ^{2}\right. \right. \\
&&\left. -\left( \frac{1}{T}\sum_{j=1}^{T}\left[ \frac{1}{S}\sum_{i=1}^{S}%
{\small 1}\left\{ u_{1}\leq X_{1,i,t+\tau }^{\widehat{\theta }%
_{1,t,N,h}}(X_{j})\leq u_{2}\right\} -1\{u_{1}\leq X_{j+\tau }\leq u_{2}\}%
\right] ^{2}\right) \right) \\
&&-\left( \left[ \frac{1}{S}\sum_{i=1}^{S}{\small 1}\left\{ u_{1}\leq
X_{k,i,t+\tau }^{\widehat{\theta }_{k,t,N,h}^{\ast }}(X_{t}^{\ast })\leq
u_{2}\right\} -1\{u_{1}\leq X_{t+\tau }^{\ast }\leq u_{2}\}\right]
^{2}\right. \\
&&\left. \left. -\left( \frac{1}{S}\sum_{j=1}^{S}\left[ \frac{1}{S}%
\sum_{i=1}^{S}{\small 1}\left\{ u_{1}\leq X_{k,i,t+\tau }^{\widehat{\theta }%
_{k,t,N,h}}(X_{j})\leq u_{2}\right\} -1\{u_{1}\leq X_{j+\tau }\leq u_{2}\}%
\right] ^{2}\right) \right) \right\} .
\end{eqnarray*}%
As the bootstrap statistic is calculated from the last $P$ resampled
observations, it is necessary to have each bootstrap term recentered around
the (full) sample mean. This is true even in the case there is no need to
mimic PEE, i.e. the choice of $P,R$ is such that $P/R\rightarrow 0.$ In such
a case, above statistic can be formed using $\widehat{\theta }_{k,t,N,h}$
rather than $\widehat{\theta }_{k,t,N,h}^{\ast }.$

For any bootstrap replication, repeat $B$ times ($B$ large) ) bootstrap
replications which yield $B$ bootstrap statistics $D_{k,P,S}^{\ast }$.
Reject $H_{0}$ if $D_{k,P,S}$ is greater than the $(1-\alpha )th$-percentile
of the bootstrap empirical distribution. For numerical implementation, it is
of importance to note that in the case where $P/R\rightarrow
0,P,T,R\rightarrow \infty ,$ there is no need to re-estimate $\widehat{%
\theta }_{1,t,N,h}^{\ast }$ $(\widehat{\theta }_{k,t,N,h}^{\ast }).$ Namely, 
$\widehat{\theta }_{_{1,t,N,h}}(\widehat{\theta }_{_{k,t,N,h}})$ can be used
in all bootstrap experiments.

Of course, the above framework can also be applied using entire simulated
distributions rather than predictive densities, by simply estimating
parameters once, using the entire sample, as opposed to using recursive
estimation techniques, say, as when forming predictions and associated
predictive densities.

\subsection{Multifactor Models}

Now, let us turn our attention to multifactor diffusion models of the form $%
(X(t),V(t))^{\prime }=\left( X(t),V^{1}(t),...,V^{d}(t)\right) ^{\prime },$
where only the first element, the diffusion process $X_{t},$ is observed
while $V(t)=(V^{1}(t),...,V^{d}(t))^{\prime }$ is latent. The most popular
class of the multifactor models is stochastic volatility model expressed as
below:%
\begin{equation}
\left( 
\begin{array}{c}
dX(t) \\ 
dV(t)%
\end{array}%
\right) =\left( 
\begin{array}{c}
b_{1}(X(t),\theta ^{\dagger }) \\ 
b_{2}(V(t),\theta ^{\dagger })%
\end{array}%
\right) dt+\left( 
\begin{array}{c}
\sigma _{11}(V(t),\theta ^{\dagger }) \\ 
0%
\end{array}%
\right) dW_{1}(t)+\left( 
\begin{array}{c}
\sigma _{12}(V(t),\theta ^{\dagger }) \\ 
\sigma _{22}(V(t),\theta ^{\dagger })%
\end{array}%
\right) dW_{2}(t),  \label{SVN}
\end{equation}%
where $W_{1}(t)_{1\text{x}1}$ and $W_{2}(t)_{1\text{x}1}$ are independent
Brownian Motions.\footnote{%
Note that the dimension of $X(\cdot )$ can be higher and we can add jumps to
the above specification such that it satisfies the regularity conditions
outlined in the one factor case. In addition, CS (2005), provide a detailed
discussion of approximation schemes in the context of stochastic volatility
models.} For instance, many term structure models require the multifactor
specification of the above form (see Dai and Singleton (2000)). In a more
complicated case, the drift function can also be specified to be a
stochastic process which poses even more challenges to testing. As mentioned
earlier, the hypotheses (\textbf{Hypothesis 1,2,3}) and the test
construction strategy for multifactor models are the same as for one factor
model. All theory essentially applies immediately to multifactor cases. In
implementation, the key difference is in the simulated approximation scheme
facilitating parameter and CDF estimation. $X(t)$ cannot simply be expressed
as a function of $d+1$ driving Brownian motions but instead involves a
function of ($W_{jt},\int_{0}^{t}W_{js}dW_{is}$ ), $i,j=1,...,d+1$ (see e.g.
Pardoux and Talay (1985) p.30-32 and CS(2005)).

For illustration, we hereafter focus on the analysis of a stochastic
volatility model (\ref{SVN}) where drift and diffusion coefficients can be
written as

\begin{equation*}
b=\binom{b_{1}(X(t),\theta ^{\dagger }))}{b_{2}(V(t),\theta ^{\dagger }))},%
\text{ }\sigma =\left( 
\begin{array}{cc}
\sigma _{11}(V(t),\theta ^{\dagger }) & \sigma _{12}(V(t),\theta ^{\dagger })
\\ 
0 & \sigma _{22}(V(t),\theta ^{\dagger })%
\end{array}%
\right) 
\end{equation*}%
We also examine a three factor model (i.e., the  Chen Model as in (\ref{CHEN}%
)) and a three factor model with jumps, (i.e., CHENJ as in (\ref{CHENJ})).
By presenting two and three factor models as an extension of our above
discussion, we make it clear that specification tests of multiple factor
diffusions with $d\geq 3$ can be easily constructed in similar manner.

In distribution estimation, the important challenge for multifactor models
lies in the missing variable issue. In particular, for simulation of $X_{t}$%
, one needs initial values of the latent processes $V_{1,}...,V_{d},$ which
are unobserved. To overcome this problem, it suffices to simulate the
process using different random initial values for the volatility process,
then construct the simulated distribution using those initial values and
average them out. This allows one to integrate out the effect of a
particular choice of volatility initial value. 

For clarity of exposition, we sketch out a simulation strategy for a general
model of $d$ latent variables in Section 3.3. This generalizes the
simulation scheme of three factor models in the Cai and Swanson (2011). As a
final remark before moving to the statistic presentation, note that the
class of multifactor diffusion processes considered in this paper is
required to match the regular conditions as in previous Section (assumption
from A1-A8 in CS (2011) with A4 being replaced by A4').

\subsubsection{Unconditional Distribution Tests}

Following the above discussion on test construction, we specialize to the
case of two-factor stochastic volatility models. Extension to general
multidimensional and multifactor models follows similarly. As the CDF is
rarely known in closed form for stochastic volatility models, we rely on
simulation technique. With the simulation scheme,\ estimators, simulated
distributed and bootstrap procedures to be presented in the next sections
(see Section 3.3 and 3.4), the test statistics for \textbf{Hypothesis 1}
turns out to be:%
\begin{equation*}
SV_{T,S,h}=\frac{1}{\sqrt{T}}\dsum\limits_{t=1}^{T}\left( 1\{X_{t}\leq u\}-%
\frac{1}{S}\dsum\limits_{t=1}^{S}1(X_{t,h}^{\widehat{\theta }_{T,N,L,h}}\leq
u)\right) 
\end{equation*}%
In the above expression, recall that $S$ is the number of simulation paths
used in distribution simulation, $\widehat{\theta }_{T,N,L,h}$ is a
simulated estimator (see Section 3.3). $N$ is a generic notation throughout
this paper, i.e. $N=L$, the length of each simulation path for SGMM and $N=M,
$ the number of random draws (simulated paths) for NPQML estimator. $h$ is
the discretization interval used in simulation. Note that $\widehat{\theta }%
_{T,N,L,h}$ is chosen in CS (2005) to be SGMM estimator using full sample
and therefore $N=L=S$.\footnote{%
As seen in assumption A4' in CS (2011) and Section 3.3 of this paper, $%
\widehat{\theta }_{T,N,L,h}$ can be other estimators such as the NPSQML
estimator. Importantly, $\widehat{\theta }_{T,N,L,h}$ satisfies condition
A4' in CS (2011).} To put it simply, one can write $\widehat{\theta }%
_{T,S,h}=\widehat{\theta }_{T,N,L,h}$.

Under the null, choose $T,S$ to satisfy $T,S\rightarrow \infty
,Sh\rightarrow 0,T/S\rightarrow 0$ then:%
\begin{equation*}
SV_{T,S,h}^{2}\rightarrow \int_{U}SV^{2}(u)\pi (u)
\end{equation*}%
where $Z$ is a Gaussian process with covariance kernel that reflects both
PEE\ and the time dependent nature of the data. The relevant bootstrap
statistic is:%
\begin{equation*}
SV_{T,S,h}^{2\ast }=\frac{1}{\sqrt{T}}\dsum\limits_{t=1}^{T}\left(
(1\{X_{t}^{\ast }\leq u\}-1\{X_{t}\leq u\})-\frac{1}{S}(\dsum%
\limits_{t=1}^{S}1(X_{t,h}^{\widehat{\theta }_{T,N,L,h}^{\ast }}\leq
u)-1(X_{t,h}^{\widehat{\theta }_{T,N,L,h}}\leq u))\right) 
\end{equation*}%
where $\widehat{\theta }_{T,S,h}^{\ast }$ is the bootstrap analogue of $%
\widehat{\theta }_{T,S,h}$. Repeat the Step 1-5 in the bootstrap procedure
in Section 3.4 to obtain critical value which are the percentiles of the
empirical distribution of $Z_{T}^{\ast }$. Compare $SV_{T,S,h}$ with the
percentiles of the empirical distribution of the bootstrap statistic and
reject $H_{0}$ if $SV_{T,S,h}$ is greater than the $(1-\alpha )th$%
-percentile thereof. Otherwise, do not reject $H_{0}$.

\subsubsection{Conditional Distribution Tests}

To test \textbf{Hypothesis 2} for the multifactor models, first we present
the test statistic for the case of the stochastic volatility model ($%
X_{t},V_{t}$) in (\ref{SVN}), (i.e., for two factor diffusion), and then we
discuss testing with the three factor model ($X_{t},V_{t}^{1},V_{t}^{2}$) as
in (\ref{CHEN}). Other multiple factor models can be tested analogously.
Note that for illustration, we again assume use of the SGMM estimator $%
\widehat{\theta }_{T,N,L,h},$ as in the original work of BCS (2008) (namely, 
$\widehat{\theta }_{T,N,L,h}$ is the simulated estimator described in
Section 3.3). Specifically, $N$ is chosen as the length of sample path $L$
used in parameter estimation. The associated test statistic is:%
\begin{equation*}
SZ_{T}=\sup_{u\times v\in U\times V}|SZ_{T}(u,v)|
\end{equation*}%
\begin{equation*}
SZ_{T}(u,v)=\frac{1}{\sqrt{T-\tau }}\sum_{t=1}^{T-\tau }\left( \frac{1}{NS}%
\sum_{j=1}^{N}\sum_{i=1}^{S}1\left\{ X_{j,i,t+\tau }^{\widehat{\theta }%
_{T,N,,L,h}}\leq u\right\} -1\{X_{t+\tau }\leq u\}\right) 1\left\{ X_{t}\leq
v\right\} .
\end{equation*}%
where $X_{j,i,t+\tau }^{\widehat{\theta }_{T,N,,L,h}}$ is is $\tau $ - Step
ahead simulated skeleton obtained by simulation procedure for multi-factor
model in Subsection 3.4.1.

In a similar manner, the bootstrap statistic analogous to $SZ_{T}$ is%
\begin{equation*}
SZ_{T}^{\ast }=\sup_{u\times v\in U\times V}\left\vert SZ_{T}^{\ast
}(u,v)\right\vert ,
\end{equation*}%
\begin{eqnarray*}
SZ_{T}^{\ast }(u,v) &=&\frac{1}{\sqrt{T-\tau }}\sum_{t=1}^{T-\tau }\left( 
\frac{1}{NS}\sum_{j=1}^{N}\sum_{i=1}^{S}1\left\{ X_{j,i,t+\tau }^{\widehat{%
\theta }_{T,N,L,h}^{\ast }}\leq u\right\} -1\{X_{t+\tau }^{\ast }\leq
u\}\right) 1\left\{ X_{t}^{\ast }\leq v\right\}  \\
&&-\frac{1}{\sqrt{T-\tau }}\sum_{t=1}^{T-\tau }\left( \frac{1}{NS}%
\sum_{j=1}^{N}\sum_{i=1}^{S}1\left\{ X_{j,i,t+\tau }^{\widehat{\theta }%
_{T,N,L,h}}\leq u\right\} -1\{X_{t+\tau }\leq u\}\right) 1\left\{ X_{t}\leq
v\right\} .
\end{eqnarray*}%
where $\widehat{\theta }_{T,N,L,h}^{\ast }$ is the bootstrap estimator
described in Section 3.4. For the three factor model, the test statistic is
defined as%
\begin{equation*}
MZ_{T}=\sup_{u\times v\in U\times V}\left\vert MZ_{T}(u,v)\right\vert ,
\end{equation*}%
\begin{equation*}
MZ_{T}(u,v)=\frac{1}{\sqrt{T-\tau }}\sum_{t=1}^{T-\tau }\left( \frac{1}{%
L^{2}S}\sum_{j=1}^{L}\sum_{k=1}^{L}\sum_{i=1}^{S}1\left\{ X_{s,t+\tau }^{%
\widehat{\theta }_{T,N,L,h}}\leq u\right\} -1\{X_{t+\tau }\leq u\}\right)
1\left\{ X_{t}\leq v\right\} 
\end{equation*}%
and bootstrap statistics is:%
\begin{eqnarray*}
MZ_{T}^{\ast }(u,v) &=&\frac{1}{\sqrt{T-\tau }}\sum_{t=1}^{T-\tau }\left( 
\frac{1}{L^{2}S}\sum_{j=1}^{L}\sum_{k=1}^{L}\sum_{i=1}^{S}1\left\{
X_{s,t+\tau }^{\widehat{\theta }_{t,N,L,h}^{\ast }}\leq u\right\}
-1\{X_{t+\tau }^{\ast }\leq u\}\right) 1\left\{ X_{t}^{\ast }\leq v\right\} 
\\
&&-\frac{1}{\sqrt{T-\tau }}\sum_{t=1}^{T-\tau }\left( \frac{1}{L^{2}S}%
\sum_{j=1}^{L}\sum_{k=1}^{L}\sum_{i=1}^{S}1\left\{ X_{s,t+\tau }^{\widehat{%
\theta }_{t,N,L,h}}\leq u\right\} -1\{X_{t+\tau }\leq u\}\right) 1\left\{
X_{t}\leq v\right\} .
\end{eqnarray*}%
where $X_{s,t+\tau }^{\widehat{\theta }_{T,N,L,h}}$ $=$ $X_{s,t+\tau }^{%
\widehat{\theta }_{T,N,L,h}}(X_{t},V_{j}^{1,\widehat{\theta }%
_{T,N,L,h}},V_{k}^{2,\widehat{\theta }_{T,N,L,h}})$ and $X_{s,t+\tau }^{%
\widehat{\theta }_{t,N,L,h}^{\ast }}=X_{s,t+\tau }^{\widehat{\theta }%
_{t,N,L,h}^{\ast }}(X_{t},V_{j}^{1,\widehat{\theta }_{t,N,L,h}^{\ast
}},V_{k}^{2,\widehat{\theta }_{t,N,L,h}^{\ast }}).$

The first order asymptotic validity of inference carried out using bootstrap
statistics formed as outlined above follows immediately from BCS (2008). For
testing decisions, one compares the test statistics $SZ_{T,S,h}$ and $%
MZ_{T,S,h}$ with the percentiles or the empirical distributions of $%
SZ_{T}^{\ast }$ and $MZ_{T,S,h}^{\ast },$ respectively. Then, reject $H_{0}$
if the actual statistic is greater than the $(1-\alpha )th$-percentile of
the empirical distribution of the bootstrap statistic, as in Section 3.4.
Otherwise, do not reject $H_{0}$.

\subsubsection{\textbf{Predictive Density Tests for Multiple Competing
Models }}

For illustration, we present the test for the stochastic volatility model
(two factor model). Again, note that extension to other multi-factor models
follows immediately. In particular, all steps in the construction of the
test in the one factor model case carry through immediately to the
stochastic volatility case with the statistic defined as:%
\begin{equation*}
DV_{P,L,S}=\max_{k=2,...,m}DV_{k,P,L,S}(u_{1},u_{2})
\end{equation*}%
where%
\begin{eqnarray*}
&&DV_{k,P,L,S}(u_{1},u_{2}) \\
&=&\frac{1}{\sqrt{P}}\sum_{t=R}^{T-\tau }\left( \left( \frac{1}{SL}%
\sum_{j=1}^{L}\sum_{i=1}^{S}{\small 1}\left\{ u_{1}\leq X_{1,t+\tau ,i,j}^{%
\widehat{\theta }_{1,t,N,L,h}}(X_{t},V_{1,j}^{\widehat{\theta }%
_{1,t,N,L,h}})\leq u_{2}\right\} {\small -}1\{u_{1}\leq X_{t+\tau }\leq
u_{2}\}\right) ^{2}\right. 
\end{eqnarray*}%
\begin{equation*}
\left. -\left( \frac{1}{SL}\sum_{j=1}^{L}\sum_{i=1}^{S}{\small 1}\left\{
u_{1}\leq X_{k,t+\tau ,i,j}^{\widehat{\theta }_{k,t,N,L,h}}(X_{t},V_{k,j}^{%
\widehat{\theta }_{k,t,N,L,h}})\leq u_{2}\right\} {\small -}1\{u_{1}\leq
X_{t+\tau }\leq u_{2}\}\right) ^{2}\right) ,
\end{equation*}%
Critical values for these tests can be obtained using a recursive version of
the block bootstrap. The corresponding bootstrap test statistic is:%
\begin{equation*}
DV_{P,L,S}^{\ast }=\max_{k=2,...,m}DV_{k,P,L,S}^{\ast }(u_{1},u_{2})
\end{equation*}

where%
\begin{eqnarray*}
&&DV_{k,P,L,S}^{\ast }(u_{1},u_{2}) \\
&=&\frac{1}{\sqrt{P}}\sum_{t=R}^{T-\tau }\left\{ \left( \left[ \frac{1}{SL}%
\sum_{j=1}^{L}\sum_{i=1}^{S}{\small 1}\left\{ u_{1}\leq X_{1,t+\tau ,i,j}^{%
\widehat{\theta }_{1,t,N,L,h}^{\ast }}(X_{t}^{\ast },V_{1,j}^{\widehat{%
\theta }_{1,t,N,L,h}^{\ast }})\leq u_{2}\right\} -1\{u_{1}\leq X_{t+\tau
}^{\ast }\leq u_{2}\}\right] ^{2}\right. \right.  \\
&&\left. -\left( \frac{1}{T}\sum_{l=1}^{T}\left[ \frac{1}{SL}%
\sum_{j=1}^{L}\sum_{i=1}^{S}{\small 1}\left\{ u_{1}\leq X_{1,t+\tau ,i,j}^{%
\widehat{\theta }_{1,t,N,L,h}}(X_{l},V_{1,j}^{\widehat{\theta }%
_{1,t,N,L,h}})\leq u_{2}\right\} -1\{u_{1}\leq X_{l+\tau }\leq u_{2}\}\right]
^{2}\right) \right)  \\
&&-\left( \left[ \frac{1}{SL}\sum_{j=1}^{L}\sum_{i=1}^{S}{\small 1}\left\{
u_{1}\leq X_{k,t+\tau ,i,j}^{\widehat{\theta }_{k,t,N,L,h}^{\ast
}}(X_{t}^{\ast },V_{k,j}^{\widehat{\theta }_{k,t,N,L,h}^{\ast }})\leq
u_{2}\right\} -1\{u_{1}\leq X_{t+\tau }^{\ast }\leq u_{2}\}\right]
^{2}\right.  \\
&&\left. \left. -\left( \frac{1}{T}\sum_{l=1}^{T}\left[ \frac{1}{SL}%
\sum_{j=1}^{L}\sum_{i=1}^{S}{\small 1}\left\{ u_{1}\leq X_{k,t+\tau ,i,j}^{%
\widehat{\theta }_{k,t,N,L,h}}(X_{l},V_{k,j}^{\widehat{\theta }%
_{k,t,N,L,h}})\leq u_{2}\right\} -1\{u_{1}\leq X_{l+\tau }\leq u_{2}\}\right]
^{2}\right) \right) \right\} .
\end{eqnarray*}%
Of note is that we follow CS (2011) by adopting the recursive NPSQML
estimator $\widehat{\theta }_{1,t,N,L,h}$ and $\widehat{\theta }_{k,t,N,L,h}$
for model $1$ and $k$, respectively, as introduced in Section 3.3.4 with the
choice $N=M=S.$ $\widehat{\theta }_{1,t,N,L,h}^{\ast }$ and $\widehat{\theta 
}_{k,t,N,L,h}^{\ast }$ are bootstrap analogs of $\widehat{\theta }%
_{1,t,N,L,h}$ and $\widehat{\theta }_{k,t,N,L,h}$ respectively (see Section
3.4). In addition, we do not need to resample the volatility process,
although volatility is simulated under both $\widehat{\theta }_{k,t,N,L,h}$
and $\widehat{\theta }_{k,t,N,L,h}^{\ast },$ $k=1,...,m.$

Repeat Steps 1-5 in the bootstrap procedure in Section 3.4 to obtain
critical values. Compare $DV_{P,L,S}$ with the percentiles of the empirical
distribution of $DV_{P,L,S}^{\ast },$ and reject $H_{0}$ if $DV_{P,L,S}$ is
greater than the $(1-\alpha )th$-percentile. Otherwise, do not reject $H_{0}.
$ Again, in implementation, there is no need to re-estimate $\widehat{\theta 
}_{k,t,N,L,h}^{\ast }$ for each bootstrap replications if  $P/R\rightarrow
0,P,T,R\rightarrow \infty $, ass parameter estimation error vanishes
asymptotically in this case$.$

\subsection{Model Simulation and Estimation}

\subsubsection{Simulating Data}

Approximation schemes are used to obtain simulated distributions and
simulated parameter estimators, which are needed in order to construct the
tests statistics outlined in previous sections. We therefore devote the
first part of this section to a discussion of the Milstein approximation
schemes that have been used in CS (2005), BCS (2008) and CS (2011). Let $L$
be the length of each simulation path and $h$ be the discretization
interval, $L=Qh$ and $\theta $ be a generic parameter in simulation
expression.\ We consider three cases: \ 

\textit{The pure diffusion process }as in (\ref{est_diff})\textit{:}%
\begin{eqnarray*}
X_{qh}^{\theta }-X_{(q-1)h}^{\theta } &=&b(X_{(q-1)h}^{\theta },\theta
)h+\sigma (X_{(q-1)h}^{\theta },\theta )\epsilon _{qh} \\
&&-\frac{1}{2}\sigma (X_{(q-1)h}^{\theta },\theta )^{\prime }\sigma
(X_{(q-1)h}^{\theta },\theta )h \\
&&+\frac{1}{2}\sigma (X_{(q-1)h}^{\theta },\theta )^{\prime }\sigma
(X_{(q-1)h}^{\theta },\theta )\epsilon _{qh}^{2},
\end{eqnarray*}

where 
\begin{equation*}
\left( W_{qh}-W_{(q-1)h}\right) =\epsilon _{qh}\overset{iid}{\sim }N(0,h),
\end{equation*}%
$q=1,\ldots ,Q,$\ with $\epsilon _{qh}\overset{iid}{\sim }N(0,h);$ and where 
$\sigma ^{\prime }$ is the derivative of $\sigma (\cdot )$ with respect to
its first argument. Hereafter, $X_{qh}^{\theta }$ denotes the values of the
diffusion at time $qh,$ simulated under generic $\theta ,$ and with a
discrete interval equal to $h,$ and so is a fine grain analog of $%
X_{t,h}^{\theta }$.

\textit{The pure jump diffusion process without stochastic volatility }as%
\textit{\ }in (\ref{difmod})\textit{:}%
\begin{eqnarray*}
&&X_{(q+1)h}^{\theta }-X_{qh}^{\theta } \\
&=&b(X_{qh}^{\theta },\theta )h+\sigma (X_{qh}^{\theta },\theta )\epsilon
_{(q+1)h}-\frac{1}{2}\sigma (X_{qh}^{\theta },\theta )^{\prime }\sigma
(X_{qh}^{\theta },\theta )h
\end{eqnarray*}%
\begin{equation}
+\frac{1}{2}\sigma (X_{qh}^{\theta },\theta )^{\prime }\sigma
(X_{qh}^{\theta },\theta )\epsilon _{(q+1)h}^{2}-\lambda \mu
_{y}h+\sum_{j=1}^{\mathcal{J}}y_{j}1\left\{ qh\leq \mathcal{U}_{j}\leq
(q+1)h\right\} ,  \label{mil}
\end{equation}%
The only difference between this approximation and that used for the pure
diffusion is the jump part. Note that the last term on the right-hand-side
(RHS) of (\ref{mil}) is nonzero whenever we have one (or more) jump
realization(s) in the interval $[(q-1)h,qh].$ Moreover, as neither the
intensity nor the jump size is state dependent, the jump component can be
simulated without any discretization error, as follows. Begin by making a
draw from a Poisson distribution with intensity parameter $\widehat{\lambda }%
\tau ,$ say $\mathcal{J}$. This gives a realization for the number of jumps
over the simulation time span. Then, draw $\mathcal{J}$ uniform random
variables over $[0,L],$ and sort them in ascending order so that $\mathcal{U}%
_{1}\leq \mathcal{U}_{2}\leq ...\leq \mathcal{U}_{\mathcal{J}}.$ These
provide realizations for the $\mathcal{J}$ jump times. Then, make $\mathcal{J%
}$ independent draws from $\phi ,$ say $y_{1},...,y_{\mathcal{J}}$.

\textit{SV models without jumps }as in (\ref{SVI})\textit{\ (}using a
generalized Milstein scheme): 
\begin{eqnarray}
X_{(q+1)h}^{\theta } &=&X_{qh}^{\theta }+\widetilde{b}_{1}(X_{qh}^{\theta
},\theta )h+\sigma _{11}(V_{qh}^{\theta },\theta )\epsilon _{1,(q+1)h} 
\notag \\
&&+\sigma _{12}(V_{qh}^{\theta },\theta )\epsilon _{2,(q+1)h}+\frac{1}{2}%
\sigma _{22}(V_{qh}^{\theta },\theta )\frac{\partial \sigma
_{12,k}(V_{qh}^{\theta },\theta )}{\partial V}\epsilon _{2,(q+1)h}^{2} 
\notag \\
&&+\sigma _{22}(V_{qh}^{\theta },\theta )\frac{\partial \sigma
_{11}(V_{qh}^{\theta },\theta )}{\partial V}\int_{qh}^{(q+1)h}\left(
\int_{qh}^{s}dW_{1,\tau }\right) dW_{2,s}  \label{gmil1a}
\end{eqnarray}%
\begin{eqnarray}
V_{(q+1)h}^{\theta } &=&V_{qh}^{\theta }+\widetilde{b}_{2}(V_{qh}^{\theta
},\theta )h+\sigma _{22}(V_{qh}^{\theta },\theta )\epsilon _{2,(q+1)h} 
\notag \\
&&+\frac{1}{2}\sigma _{22}(V_{qh}^{\theta },\theta )\frac{\partial \sigma
_{22}(V_{qh}^{\theta },\theta )}{\partial V}\epsilon _{2,(q+1)h}^{2}
\label{gmil1b}
\end{eqnarray}%
where $h^{-1/2}\epsilon _{i,qh}\sim N(0,1),$ $i=1,2$, $E(\epsilon
_{1,qh}\epsilon _{2,q^{\prime }h})=0$ for all $q\neq q^{\prime },$ and%
\begin{equation*}
\widetilde{b}(V,\theta )=\left( 
\begin{array}{c}
\widetilde{b}_{1}(V,\theta ) \\ 
\widetilde{b}_{2}(V,\theta )%
\end{array}%
\right) =\left( 
\begin{array}{c}
b_{1}(V,\theta )-\frac{1}{2}\sigma _{22}(V,\theta )\frac{\partial \sigma
_{12}(V,\theta )}{\partial V} \\ 
b_{2}(V,\theta )-\frac{1}{2}\sigma _{22}(V,\theta )\frac{\partial \sigma
_{22}(V,\theta )}{\partial V}%
\end{array}%
\right) .
\end{equation*}

The last terms on the RHS of (\ref{gmil1a}) involve stochastic integrals and
cannot be explicitly computed. However, they can be approximated, up to an
error of order $o(h)$ by (see, for example, equation (3.7), pp. 347 in
Kloeden and Platen (1999)): 
\begin{equation*}
\int_{qh}^{(q+1)h}\left( \int_{qh}^{s}dW_{1,\tau }\right) dW_{2,s}\approx
h\left( \frac{1}{2}\xi _{1}\xi _{2}+\sqrt{\rho _{p}}\left( \mu _{1,p}\xi
_{2}-\mu _{2,p}\xi _{1}\right) \right)
\end{equation*}%
\begin{equation*}
+\frac{h}{2\pi }\sum_{r=1}^{p}\frac{1}{r}\left( \varsigma _{1,r}\left( \sqrt{%
2}\xi _{2}+\eta _{2,r}\right) -\varsigma _{2,r}\left( \sqrt{2}\xi _{1}+\eta
_{1,r}\right) \right) ,
\end{equation*}%
where for $j=1,2,$ $\xi _{j},\mu _{j,p},\varsigma _{j,r},\eta _{j,r}$ are $%
iid$ $N(0,1)$ random variables, $\rho _{p}=\frac{1}{12}-\frac{1}{2\pi ^{2}}%
\sum_{r=1}^{p}\frac{1}{r^{2}},$ and $p$ is such that as $h\rightarrow 0,$ $%
p\rightarrow \infty .$

\textit{Stochastic Volatility with Jumps}

Simulation of sample paths of diffusion processes with stochastic volatility
and jumps follows straightforwardly from the previous two cases. Whenever
both intensity and jump size are not state dependent, a jump component can
be simulated and added to either $X(t)$ and/or the $V(t)$ in the same manner
as above. Extension to general multidimensional and multifactor models both
with and without jumps also follows directly.

\subsubsection{Simulating Distributions}

In this section we sketch out methods used to construct $\tau -$step ahead
simulated conditional distributions using simulated data. In applications,
simulation techniques are needed when the functional form conditional
distribution is unknown. We first illustrate the technique for one factor
models and then discuss multifactor models.

\textit{One factor models:} 

Consider the one factor model as in (\ref{base}). To estimate the simulated
CDFs,

\textbf{Step 1:} Obtain $\widehat{\theta }_{T,N,h}$ (using the entire
sample) or $\widehat{\theta }_{t,N,h}$ (recursive estimator) where $\widehat{%
\theta }_{T,N,h}$ and $\widehat{\theta }_{t,N,h}$ are estimators as
discussed in Section 3.3.3 and 3.3.4.

\textbf{Step 2:} Under $\widehat{\theta }_{T,N,h}$ or $\widehat{\theta }%
_{t,N,h}$\footnote{%
Note that $N=L$ for the SGMM estimator while $N=M=S$ \ for NSQML estimator.}%
, simulate $S$ paths of length $\tau ,$ all having the same starting value, $%
X_{t}.$ In particular, for each path $i=1,...S$ of length $\tau ,$ generate $%
X_{i,t+\tau }^{\widehat{\theta }_{T,N,h}}(X_{t})$ according to a Milstein
schemes detailed in previous section, with $\theta =\widehat{\theta }_{T,N,h}
$ or $\widehat{\theta }_{t,N,h}.$ The errors used in simulation are $\
\epsilon _{qh}\overset{iid}{\sim }N(0,h)$, and $Qh=\tau $. $\epsilon _{qh}$
is assumed to be independent across simulations, so that $E(\epsilon
_{i,qh}\epsilon _{j,qh})=0,$for all $i\neq j$ and $E(\epsilon
_{i,qh}\epsilon _{i,qh})=h,$ for any $i,j.$ In addition, as the simulated
diffusion is ergodic, the effect of the starting value approaches zero at an
exponential rate, as $\tau \rightarrow \infty $.

\textbf{Step 3:} If $\widehat{\theta }_{T,N,h}$ ($\widehat{\theta }_{t,N,h}$%
) is used, an estimate for the distribution$,$ at time $t+\tau ,$
conditional on $X_{t},$ with estimator $\widehat{\theta }_{T,N,h}(\widehat{%
\theta }_{t,N,h})$, is defined as: 
\begin{equation*}
\widehat{F}_{\tau }(u|X_{t},\widehat{\theta }_{T,N,h})=\frac{1}{S}%
\sum_{i=1}^{S}1\left\{ X_{i,t+\tau }^{\widehat{\theta }_{T,N,h}}(X_{t})\leq
u\right\} 
\end{equation*}%
BCS (2008) show that if the model is correctly specified, then $\frac{1}{S}%
\sum_{i=1}^{S}1\left\{ X_{i,t+\tau }^{\widehat{\theta }_{T,N,h}}(X_{t})\leq
u\right\} $ provides a consistent estimate of the\ conditional distribution $%
F_{\tau }(u|X_{t},\theta ^{\dagger })=\Pr \left( X_{t+\tau }^{\theta
^{\dagger }}\leq u|X_{t}^{\theta ^{\dagger }}=X_{t}\right) .$

Specifically, assume that $T,N,S\rightarrow \infty .$ Then, for the case of
SGMM estimator, if $h\rightarrow 0,$ $T/N\rightarrow 0,$ and $%
h^{2}T\rightarrow 0,$ $T^{2}/S\rightarrow \infty ,$ the following result
holds for any $X_{t},$ $t\geq 1,$ uniformly in $u$%
\begin{equation*}
\widehat{F}_{\tau }(u|X_{t},\widehat{\theta }_{T,N,h})-F_{\tau
}(u|X_{t},\theta ^{\dagger })\overset{pr}{\rightarrow }0,
\end{equation*}%
In addition, if the model is correctly specified (i.e. if $\mu (\cdot ,\cdot
)=\mu _{0}(\cdot ,\cdot )$ and $\sigma (\cdot ,\cdot )=\sigma _{0}(\cdot
,\cdot ))$ then:%
\begin{equation*}
\widehat{F}_{\tau }(u|X_{t},\widehat{\theta }_{T,N,h})-F_{0,\tau
}(u|X_{t},\theta _{0})\overset{pr}{\rightarrow }0,
\end{equation*}%
\textbf{Step 4:} Repeat Steps 1-3 for $t=1,...,T-\tau .$ This yields $T-\tau 
$ conditional distributions that are $\tau -$Steps ahead \ which will be
used in the construction of the specification tests.

The CDF simulation in the case selection test of multiple models with
recursive estimator is similar. For model $k,$ let $\widehat{\theta }%
_{k,t,N,h}$ be the recursive estimator of "pseudo true" $\theta
_{k}^{\dagger }$ computed using all observations up to varying time $t.$
Then, $X_{k,i,t+\tau }^{\widehat{\theta }_{k,t,N,h}}(X_{t})$ is generated
according to a Milstein schemes as in Section 3.3.1, with $\theta =\widehat{%
\theta }_{k,t,N,h}$ and the initial value $X_{t},$ $Qh=\tau $. The
corresponding empirical distribution of the simulated series $X_{k,i,t+\tau
}^{\widehat{\theta }_{k,t,N,h}}(X_{t})$ can then be constructed$.$ Under
some regularity conditions,%
\begin{equation*}
\frac{1}{S}\sum_{i=1}^{S}1\left\{ u_{1}\leq X_{k,i,t+\tau }^{\widehat{\theta 
}_{k,t,N,h}}(X_{t})\leq u_{2}\right\} \overset{pr}{\rightarrow }%
F_{X_{k,t+\tau }^{\theta _{k}^{\dagger }}(X_{t})}(u_{2})-F_{X_{k,t+\tau
}^{\theta _{k}^{\dagger }}(X_{t})}(u_{1}),\text{ }t=R,...,T-\tau ,
\end{equation*}

where $F_{X_{k,t+\tau }^{\theta _{k}^{\dagger }}(X_{t})}(u)$ is the marginal
distribution of $X_{t+\tau }^{\theta _{k}^{\dagger }}(X_{t})$ implied by $k$
model (i.e., by the model used to simulate the series), conditional on the
(simulation) starting value $X_{t}.$ Furthermore, the marginal distribution
of $X_{t+\tau }^{\theta ^{\dagger }}(X_{t})$ is the distribution of $%
X_{t+\tau }$ conditional on the values observed at time $t.$ Thus, $%
F_{X_{k,t+\tau }^{\theta _{k}^{\dagger }}(X_{t})}(u)=F_{k}^{\tau
}(u|X_{t},\theta _{k}^{\dagger }).$

Of important note is that in the simulation of $X_{k,i,t+\tau }^{\widehat{%
\theta }_{k,t,N,h}}(X_{t}),$ $i=1,...,S$, for each $t,$ $t=R,...,T-\tau ,$
we must use the same set of randomly drawn errors and similarly the same
draws for numbers of jumps, jump times and jump sizes. Thus, we only allow
for the starting value to change. In particular, for each $i=1,...,S,$ we
generate $X_{k,i,R+\tau }^{\widehat{\theta }%
_{k,R,N,h}}(X_{R}),...,X_{k,i,T}^{\widehat{\theta }_{k,T-\tau
,N,h}}(X_{T-\tau }).$ This yields an $S$x$P$ matrix of simulated values,
where $P=T-R-\tau +1$ refers to the length of the out-of-sample period. $%
X_{k,i,R+j+\tau }^{\widehat{\theta }_{k,R+j,N,h}}(X_{R+j})$ (at time $%
R+j+\tau )$ can be seen as $\tau $ periods ahead value "predicted" by model $%
k$ using all available information up to time $R+j_{R+j}$, $j=1,...,P$ (the
initial value $X_{R+j}$ and $\widehat{\theta }_{k,R+j,N,h}$ estimated using $%
X_{1},...,X_{R+j}).$ The key feature of this setup is that it enables us to
compare "predicted " $\tau $ periods ahead values (i.e. $X_{k,i,R+j+\tau }^{%
\widehat{\theta }_{k,R+j,N,h}}(X_{R+j})$) with actual values that are $\tau $
periods ahead (i.e., $X_{R+j+\tau }$), for $j=1,...,P$. In this manner,
simulation based tests under \textit{ex-ante} predictive density comparison
framework can be constructed.

\textit{Multifactor model:}

Consider the multi-factor model with a skeleton $\left(
X_{t},V_{t}^{1},...,V_{t}^{d}\right) ^{\prime }$ (e.g. stochastic mean,
stochastic volatility models, stochastic volatility of volatility, etc.)
where only the first element $X_{t}$ is observed. For simulation of the CDF$,
$ the difficulty arises as we do not know the initial values of latent
variables ($V_{t}^{1},...,V_{t}^{d})^{\prime }$ at each point in time$.$ We
generalize the simulation plan of BCS (2008) and Cai and Swanson (2011) to
the case of $d$ dimensions. Specifically, to overcome the initial value
difficulty, a natural strategy is to simulate a long path of length $L$ for
each latent variable $V_{t}^{1},...,V_{t}^{d}$ , use them to construct $%
X_{t+\tau }$ and the corresponding simulated CDF of $X_{t+\tau }$; and
finally, we average out the volatility values. Note that there are $L^{d}$
combinations of the initial values $V_{t}^{1},...,V_{t}^{d}.$ For
illustration, consider the case of stochastic volatility ($d=1$) and the
Chen three factor model as in (\ref{CHEN}) ($d=2),$ using recursive
estimators.

For the case of stochastic volatility ($d=1$), i.e. $(X_{t},V_{t})^{\prime },
$ the steps are as follows:

\textbf{Step 1:} Estimate $\widehat{\theta }_{t,N,L,h}$ using recursive SGMM
or NSQML\ estimation methods.

\textbf{Step 2. }Using the scheme in (\ref{gmil1b}) with $\theta =$ $%
\widehat{\theta }_{t,N,L,h},$ generate the path $V_{qh}^{\widehat{\theta }%
_{t,N,L,h}}$ for $q=1/h,...,Qh$ with $Qh=L$ and hence obtain $V_{j}^{%
\widehat{\theta }_{t,N,L,h}}$ $j=1,...L$.

\textbf{Step 3:} Using schemes in (\ref{gmil1a}), (\ref{gmil1b}), simulate $L
$x$S$ paths of length $\tau $, setting the initial value for the observable
state variable to be $X_{t}.$ For the initial values of unobserved
volatility, use $V_{j,qh}^{\widehat{\theta }_{t,N,L,h}},$ $j=1,...L$ as
retrieved in Step 2. Also, keep the simulated random innovations (i.e.$%
\epsilon _{1,qh},$.$\epsilon _{1,qh},$ $\int_{qh}^{(q+1)h}\left(
\int_{qh}^{s}dW_{1,\tau }\right) dW_{2,s}$) to be constant across each $j$
and $t$. Hence, for each replication $i,$using initial values $X_{t}$ and $%
V_{j,qh}^{\widehat{\theta }_{T,N,h}},$ we obtain $X_{j,i,t+\tau }^{\widehat{%
\theta }_{t,N,L,h}}(X_{t})$ which is a $\tau $ - step ahead simulated value.

\textbf{Step 4: }Now the estimator of $F_{\tau }(u|X_{t},\theta ^{\dagger })$
is defined as:

\begin{equation*}
\widehat{F}_{\tau }(u|X_{t},\widehat{\theta }_{t,N,L,h})=\frac{1}{LS}%
\sum_{j=1}^{L}\sum_{i=1}^{S}1\left\{ X_{j,i,t+\tau }^{\widehat{\theta }%
_{t,N,h}}(X_{t})\leq u\right\}
\end{equation*}

Note that, by averaging over the initial value of the volatility process, we
have integrated out it's effect. In other words, $\frac{1}{S}%
\sum_{i=1}^{S}1\left\{ X_{j,i,t+\tau }^{\widehat{\theta }_{t,N,h}}(X_{t})%
\leq u\right\} $ is an estimate of $F_{\tau }(u|X_{t},V_{j,h}^{\widehat{%
\theta }_{t,N,h}},\theta ^{\dagger }).$

\textbf{Step 5}: Repeat the Steps 1-4 for $t=1,...,T-\tau .$ This yields $%
T-\tau $ conditional distributions that are $\tau -$steps ahead \ which will
be used in the construction of the specification tests.

For three factor model ($d=2$), i.e., ($X_{t},V_{t}^{1},V_{t}^{2})$, consider%
{\large \ }model (\ref{CHEN}), where $W_{t}=\left(
W_{t}^{1},W_{t}^{2},W_{t}^{3}\right) $ are mutually independent standard
Brownian motions.

\textbf{Step 1:} Estimate $\widehat{\theta }_{t,N,L,h}$ using SGMM or NSQML\
estimation methods.

\textbf{Step 2}\textit{: }Given the estimated parameter $\widehat{\theta }%
_{_{t,N,L,h}},$\ generate the path $V_{qh}^{1,\widehat{\theta }_{t,N,L,h}}$
and $V_{ph}^{2,\widehat{\theta }_{t,N,L,h}}$ for $q,p=1/h,...,Qh$ with $Qh=L$
and hence obtain $V_{j}^{1,\widehat{\theta }_{T,N,L,h}}$,$V_{k}^{2,\widehat{%
\theta }_{T,N,L,h}}$ $j,k=1,...,L$.

\textbf{Step 3:} Given the observable $X_{t}$\ and the $L\times L$\
simulated latent paths ($V_{j}^{1,\widehat{\theta }_{t,N,L,h}}$ and $%
V_{k}^{2,\widehat{\theta }_{t,N,L,h}}$\ $j,k=1,...,L)$\ as the start values
, we simulate $\tau $-Step ahead $X_{t+\tau }^{\widehat{\theta }%
_{t,N,L,h}}(X_{t},V_{j}^{1,\widehat{\theta }_{t,N,L,h}},V_{k}^{2,\widehat{%
\theta }_{t,N,L,h}})$.\ Since the start values for the two latent variables
are $L\times L$\ length, so for each $X_{t}$\ we have $N^{2}$\ path. Now to
integrate out the initial effect of latent variables, form the estimate of
conditional distribution as

\begin{equation*}
\widehat{F}_{\tau ,s}(u|X_{t},\widehat{\theta })=\frac{1}{L^{2}}%
\sum_{j=1}^{L}\sum_{k=1}^{L}1\left\{ X_{s,t+\tau }^{\widehat{\theta }%
_{t,N,L,h}}(X_{t},V_{j}^{1,\widehat{\theta }_{t,N,L,h}},V_{k}^{2,\widehat{%
\theta }_{t,N,L,h}})\leq u\right\} ,
\end{equation*}

where $s$\ denotes the $sth$\ simulation.

\textbf{Step 4}\textit{:} Simulate $X_{s,t+\tau }^{\widehat{\theta }%
_{t,N,L,h}}$\ S times, that is, repeat Step 3 $S$ times i.e. $s=1,...,S$.
The estimate of $F_{\tau }(u|X_{t},\theta ^{\dagger })$ is

\begin{equation*}
\widehat{F}_{\tau }(u|X_{t},\widehat{\theta })=\frac{1}{S}\sum_{i=1}^{S}%
\widehat{F}_{\tau ,s}(u|X_{t},\widehat{\theta }_{T,N,h})
\end{equation*}

\textbf{Step 5: }Repeat the Steps 1-4 for $t=1,...,T-\tau .$ This yields $%
T-\tau $ conditional distributions that are $\tau -$steps ahead \ which will
be used in the construction of the specification tests.

As a final remark, for the case of multiple competing models, we can proceed
similarly. In addition, in the next two subsections, we present the exactly
identified simulated (recursive) general method of moments and recursive
nonparametric simulated quasi-maximum likelihood estimators that can be used
in simulating distributions as well as constructing test statistics
described in Section 3.2. The bootstrap analogs of those estimators will be
discussed in Section 3.4.

\subsubsection{Estimation: (Recursive) Simulated General Method of Moments
(SGMM) Estimators}

Suppose that we observe a discrete sample (skeleton) of $T$ observations,
say $(X_{1},X_{2},...,X_{T})^{\prime },$ from the underlying diffusion in (%
\ref{base}). The (recursive) SGMM estimator $\widehat{\theta }_{t,L,h}$ with 
$1\leq t\leq T$ is specified as: 
\begin{eqnarray}
\widehat{\theta }_{t,L,h} &=&\arg \min_{\theta \in \Theta }\left( \frac{1}{t}%
\sum_{j=1}^{t}g(X_{j})-\frac{1}{L}\sum_{j=1}^{L}g(X_{j,h}^{\theta })\right)
^{\prime }W_{t}^{-1}\left( \frac{1}{t}\sum_{j=1}^{t}g(X_{j})-\frac{1}{L}%
\sum_{j=1}^{L}g(X_{j,h}^{\theta })\right)   \notag \\
&& \\
&=&\arg \min_{\theta \in \Theta }G_{t,L,h}(\theta )^{\prime
}W_{t}G_{t,L,h}(\theta ),  \label{thetahat}
\end{eqnarray}%
where $g$ is a vector of $p$ moment conditions, $\Theta \subset \Re ^{p}$
(so that we have as many moment conditions as parameters), and $%
X_{j,h}^{\theta }=X_{[Qjh/L]}^{\theta },$ with $L=Qh$ is the simulated path
under generic parameter $\theta $ and with discrete interval $h$. $%
X_{j,h}^{\theta }$ is simulated using the Milstein schemes.

Note that in the above expression, in the context of the specification test $%
\widehat{\theta }_{t,L,h}$ is estimated using the whole sample, i.e. $t=T$.
In the out of sample context, the recursive SGMM estimator $\widehat{\theta }%
_{t,L,h}$ is estimated recursively using the using\ sample from $1$ up to $%
t. $

Typically, the $p$ moment conditions are based on the difference between
sample moments of historical and simulated data or, between sample moments
and model implied moments, whenever the latter are known in closed form.
Finally, $W_{t}$ is the heteroskedasticity and autocorrelation (HAC) robust
covariance matrix estimator, defined as 
\begin{equation}
W_{t}^{-1}=\frac{1}{t}\sum_{\nu =-l_{t}}^{l_{t}}w_{\nu }\sum_{j=\nu
+1+l_{t}}^{t-l_{t}}\left( g(X_{j})-\frac{1}{t}\sum_{j=1}^{t}g(X_{j})\right)
\left( g(X_{j-\nu })-\frac{1}{t}\sum_{j=1}^{t}g(X_{j})\right) ^{\prime },
\label{NW}
\end{equation}%
where $w_{v}=1-v/(l_{T}+1).$ Further, the pseudo true value, $\theta
^{\dagger }$, is defined to be: 
\begin{equation*}
\theta ^{\dagger }=\arg \min_{\theta \in \Theta }G_{\infty }(\theta
)^{\prime }W_{0}G_{\infty }(\theta ),
\end{equation*}%
where 
\begin{equation*}
G_{\infty }(\theta )^{\prime }W_{0}G_{\infty }(\theta
)=p\lim_{L,T\rightarrow \infty ,h\rightarrow 0}G_{T,L,h}(\theta )^{\prime
}W_{T}G_{T,L,h}(\theta );
\end{equation*}%
and where $\theta ^{\dagger }=\theta _{0},$ if the model is correctly
specified.

In the above set up, the exactly identified case is considered rather than
the overidentified (S)GMM. This choice guarantees that $G_{\infty }(\theta
^{\dagger })=0$\ even under misspecification, in the sense that the model
differs from the underlying DGP. As pointed out by Hall and Inoue (2003),
the root-N consistency does not hold for overidentified (S)GMM estimators of
misspecified models. In addition,

\begin{equation*}
\nabla _{\theta }G_{\infty }(\theta ^{\dagger })^{\prime }W^{^{\dagger
}}G_{\infty }(\theta ^{\dagger })=0.
\end{equation*}%
However, in the case for which the number of parameters and the number of
moment conditions is the same, $\nabla _{\theta }G_{\infty }(\theta
^{\dagger })^{\prime }W^{^{\dagger }}$\ is invertible, and so the first
order conditions also imply that $G_{\infty }(\theta ^{\dagger })=0.$

Also note that other available estimation methods using moments include the
efficient method of moments (EMM) estimator as proposed by Gallant and
Tauchen (1996, 1997), which calculates moment functions by simulating the
expected value of the score implied by an auxiliary model. In their setup,
parameters are then computed by minimizing a chi-square criterion function.

\subsubsection{Estimation: Recursive Nonparametric Simulated Quasi Maximum
Likelihood Estimators}

In this section we outline a recursive version of the NPSQML estimator of
Fermanian and Salani%
%TCIMACRO{\U{b4}}%
%BeginExpansion
\'{}%
%EndExpansion
e (2004), proposed by CS (2011). The bootstrap counterpart of the recursive
NPSQML estimator will be presented in the next section.

\textit{One factor models:}

Hereafter, let $f\left( X_{t}|X_{t-1},\theta ^{\dagger }\right) $ be the
conditional density associated with the above jump diffusion$.$ If $f$ is
known in closed form, we can just estimate $\theta ^{\dagger }$ recursively,
using standard QML as:\footnote{%
Note that as model $k$ is, in general, misspecified, \ $%
\sum_{t=1}^{T-1}f_{k}\left( X_{t}|X_{t-1},\theta _{k}\right) $ is a
quasi-likelihood and $f_{k}\left( X_{t}|X_{t-1},\theta _{k}^{\dagger
}\right) $ is not necessarily a martingale difference sequence.}

\begin{equation}
\widehat{\theta }_{t}=\arg \max_{\theta \in \Theta }\frac{1}{t}%
\sum_{j=2}^{t}\ln f\left( X_{j}|X_{j-1},\theta \right) ,\text{ }%
t=R,...,R+P-1..  \label{ftrue}
\end{equation}

Note that, similarly to the case of SGMM, the pseudo true value $\theta
^{\dagger }$ is optimal in the sense:%
\begin{equation}
\theta ^{\dagger }=\arg \max_{\theta \in \Theta }E\left( \ln f\left(
X_{t}|X_{t-1},\theta \right) \right) .  \label{theta-ddag}
\end{equation}%
for the case $f$ is not known in closed form, we can follow Kristensen and
Shin (2008) and CS (2011) to construct the simulated analog $\widehat{f}$ of 
$f$ and then use it to estimate $\theta ^{\dagger }$. $\widehat{f}$ is
estimated as function of the simulated sample paths $X_{t,i}^{\theta
}(X_{t-1}),$ for $t=2,...,T-1,$ $i=1,...,M.$ First, generate $T-1$ paths of
length one for each simulation replication, using $X_{t-1}$ with $t=1,...T$
as starting values. Hence, at time $t$ and simulation replication $i$ we
obtain skeletons $X_{t,i}^{\theta }(X_{t-1}),$ for $t=2,...,T-1,$ $i=1,...,M$
where $M$ is the number of simulation paths (number of random draws or $%
X_{t,j}^{\theta }(X_{t-1})$ and $X_{t,l}^{\theta }(X_{t-1})$ are i.i.d.) for
each simulation replication. $M$ is fixed across all initial values. Then
the recursive NPSQML estimator is defined as follows:

\begin{equation*}
\widehat{\theta }_{t,M,h}=\arg \max_{\theta \in \Theta }\frac{1}{t}%
\sum_{i=2}^{t}\ln \widehat{f}_{M,h}\left( X_{i}|X_{i-1},\theta \right) \tau
_{M}\left( \widehat{f}_{M,h}\left( X_{i}|X_{i-1},\theta \right) \right) ,%
\text{ }t\geq R,
\end{equation*}

where 
\begin{equation*}
\widehat{f}_{M,h}\left( X_{t}|X_{t-1},\theta \right) =\frac{1}{M\xi _{M}}%
\sum_{i=1}^{M}K\left( \frac{X_{t,i,h}^{\theta }(X_{t-1})-X_{t}}{\xi _{M}}%
\right) .
\end{equation*}

Note that with abuse of notation, we define $\widehat{\theta }_{t,L,h}$ for
SGMM and $\widehat{\theta }_{t,M,h}$ for NPSQML estimators where $L$ and $M$
have different interpretations ($L$ is the length of each simulation path
and $M$ is number of random draws).

The function $\tau _{M}\left( \widehat{f}_{M,h}\left( X_{t}|X_{t-1},\theta
\right) \right) $ is a trimming function. It has some characteristics such
as positive and increasing, $\tau _{M}\left( \widehat{f}_{M,h}\left(
X_{t},X_{t-1},\theta \right) \right) =0,$ if $\widehat{f}_{M,h}\left(
X_{t},X_{t-1},\theta \right) <\xi _{M}^{\delta },$ and

\noindent $\tau _{M}\left( \widehat{f}_{M,h}\left( X_{t},X_{t-1},\theta
\right) \right) =1,$ if $\widehat{f}_{M,h}\left( X_{t},X_{t-1},\theta
\right) >2\xi _{M}^{\delta },$ for some $\delta >0$.\footnote{%
Fermanian and Salanie (2004) suggest using the following triming function:%
\begin{equation*}
\tau _{N}(x)=\frac{4(x-a_{N})^{3}}{a_{N}^{3}}-\frac{3(x-a_{N})4}{a_{N}^{4}},
\end{equation*}%
for $a_{N}\leq x\leq 2a_{N}.$} Note that when the log density is close to
zero, the derivative tends to infinity and thus even very tiny simulation
errors can have a large impact on the likelihood. The introduction of the
trimming parameter into the optimization function ensures the impact of this
case to be minimal asymptotically.

\textit{Multifactor Models:}

Since volatility is not observable, we cannot proceed as in the single
factor case when estimating the SV model using NPSQML estimator. Instead,
let $V_{j}^{\theta }$ be generated according to (\ref{gmil1b}), setting $%
qh=j,$ and $j=1,...,L.$ The idea is to simulate $L$ different starting
values for unobservable volatility, construct the simulated likelihood
functions accordingly and then average them out. For each simulation
replication at time $t$, we simulate $L$ different values of $X_{t}$ $%
(X_{t-1},V_{j}^{\theta })$ by generating $L$ paths of length one, using
fixed observable $X_{t-1}$ and $\,$ unobservable $V_{j}^{\theta }$, $%
j=1,...,L$ as starting values$.$ Repeat this procedure for any $t=1,...,T-1$%
, and for any set $j,$ $j=1,...,L$ of random errors $\epsilon _{1,t+(q+1)h,j}
$ and $\epsilon _{2,t+(q+1)h,j},$ $q=1,...,1/h.$ Note that it is important
to use the same set of random errors $\epsilon _{1,t+(q+1)h,j}$ and $%
\epsilon _{2,t+(q+1)h,j}$ across different initial values for volatility.
Denote the simulated value at time $t,$ simulation replication $i,$ under
generic parameter $\theta ,$ using $X_{t-1},V_{j}^{\theta }$ as starting
values as $X_{t,i,h}^{\theta }(X_{t-1},V_{j}^{\theta }).$ Then:%
\begin{equation*}
\widehat{f}_{M,L,h}\left( X_{t}|X_{t-1},\theta \right) =\frac{1}{L}%
\sum_{j=1}^{L}\frac{1}{M\xi _{M}}\sum_{i=1}^{M}K\left( \frac{%
X_{t,i,h}^{\theta }(X_{t-1},V_{j}^{\theta })-X_{t}}{\xi _{M}}\right) ,
\end{equation*}%
and note that by averaging over the initial values for the unobservable
volatility, its effect is integrated out. Finally, define:\footnote{%
For discussion of asymptotic properties of $\widehat{\theta }_{k,t,M,L,h},$
as well as of regularity conditions, see CS(2011).}%
\begin{equation*}
\widehat{\theta }_{t,M,L,h}=\arg \min_{\theta \in \Theta }\frac{1}{t}%
\sum_{s=2}^{t}\ln \widehat{f}_{M,L,h}\left( X_{s}|X_{s-1},\theta \right)
\tau _{M}\left( \widehat{f}_{M,L,h}\left( X_{s}|X_{s-1},\theta \right)
\right) ,\text{ }t\geq R.
\end{equation*}

Note that in this case, $X_{t}$ is no longer Markov (i.e., $X_{t}$ and $%
V_{t} $ are jointly Markovian, but $X_{t}$ is not). Therefore, even in the
case of true data generating process, the joint likelihood cannot be
expressed as the product of the conditional and marginal distributions$.$
Thus, $\widehat{\theta }_{t,M,L,h}$ is necessarily a QML estimator.
Furthermore, note that $\nabla _{\theta }f(X_{t}|X_{t-1},\theta ^{\dagger })$
is no longer a martingale difference sequence; therefore, we need to use HAC
robust covariance matrix estimators, regardless of whether the model is the
\textquotedblleft correct\textquotedblright\ model or not.

\subsection{Bootstrap Critical Value Procedures}

The test statistics presented in Section 3.1 and 3.2 are implemented using
critical values constructed via the bootstrap.\ As mentioned earlier,
motivation for using the bootstrap is clear. The covariance kernel of the
statistics limiting distributions contain both parameter estimation error\
and the data related time dependence components. Asymptotic critical value
cannot thus be tabulated in a usual way. Several methods have been proposed
to tackle this issue. One is the block bootstrap procedures which we
discuss. Others have been mentioned above.

With regarding to the validity of the bootstrap, note that, in the case of
dependent observations without PEE, we can tabulate valid critical value
using a simple empirical version of the K\"{u}nsch (1989) block bootstrap.
Now, the difficulty in our context lies in accounting for parameter
estimation error. Goncalves and White (2002) establish the first order
validity of the block bootstrap for QMLE (or m-estimator) for dependent and
heterogeneous data. This is an important result for the class of SGMM\ and
NSQML estimators surveyed in this paper, and allows Corradi and Swanson in
CS (2011) and elsewhere to develop asymptotically valid version of the
bootstrap that can be applied under generic model misspecification, as
assumed throughout this paper.

For the SGMM estimator, as shown in CS (2005) the first order validity of
the block bootstrap is valid in the exact identification case, and when $%
T/S\rightarrow 0$. \ In this case, SGMM\ is asymptotically equivalent to
GMM, and consequently there is no need to bootstrap the simulated series. In
addition, in the exact identification case, GMM estimators can be treated
the same way that QMLE estimators are treated. For the NSQML estimator, CS
(2011) point out that the NPSQML estimator is asymptotically equivalent to
the QML\ estimator. Thus, we do not need to resample the simulated
observations as the negligible contribution of simulation errors.

Also note that critical values for these tests can be obtained using a
recursive version of the block bootstrap. When forming block bootstrap
samples in the recursive case, observations at the beginning of the sample
are used more frequently than observations at the end of the sample. This
introduces a location bias to the usual block bootstrap, as under standard
resampling with replacement, all blocks from the original sample have the
same probability of being selected. Also, the bias term varies across
samples and can be either positive or negative, depending on the specific
sample{\small . }A first-order valid bootstrap procedure for non simulation
based $m-$estimators constructed using a recursive estimation scheme is
outlined in Corradi and Swanson (2007a). Here we extend the results of
Corradi and Swanson (2007a) by establishing asymptotic results for cases in
which simulation-based estimators are bootstrapped in a recursive setting.

Now the details of bootstrap procedure for critical value tabulation can be
outlined in 5 steps as follows:

\textbf{Step 1:} Let $T=bl,$ where $b$ denotes the number of blocks and $l$
denotes the length of each block.\ We first draw a discrete uniform random
variable, $I_{1},$ that can take values $0,1,...,T-l$ with probability $%
1/(T-l+1).$ The first block is given by $X_{I_{1+1}},...,X_{I_{1}+l}.$ We
then draw another discrete uniform random variable, say $I_{2},$ and a
second block of length $l$ is formed, say $X_{I_{2}+1},...,X_{I_{2}+l}.$
Continue in the same manner, until you draw the last discrete uniform say $%
I_{b},$ and so the last block is $X_{I_{b}+1},...,X_{I_{b}+l}.$ Let's call
the $X_{t}^{\ast }$ the resampled series, and note that $X_{1}^{\ast
},X_{2}^{\ast },...,X_{T}^{\ast }$ corresponds to $%
X_{I_{1}+1},X_{I_{1}+2},...,X_{I_{b}+l}.$ Thus, conditional on the sample,
the only random element is the beginning of each block. In particular 
\begin{equation*}
X_{1}^{\ast },...,X_{l}^{\ast },X_{l+1}^{\ast },...,X_{2l}^{\ast
},X_{T-l+1}^{\ast },...,X_{T}^{\ast },
\end{equation*}%
conditional on the sample, can be treated as $b$ $iid$ blocks of discrete
uniform random variables. For a simple illustration the link between the
bootstrap sample and the original sample. Note that it can be shown that
except a set of probability measure approaching zero,%
\begin{equation}
E^{\ast }\left( \frac{1}{T}\sum_{t=1}^{T}X_{t}^{\ast }\right) =\frac{1}{T}%
\sum_{t=1}^{T}X_{t}+O_{P}^{\ast }(l/T)  \label{E}
\end{equation}

\begin{eqnarray}
Var^{\ast }\left( \frac{1}{T^{1/2}}\sum_{t=1}^{T}X_{t}^{\ast }\right) &=&%
\frac{1}{T}\sum_{t=l}^{T-l}\sum_{i=-l}^{l}(X_{t}-\frac{1}{T}%
\sum_{t=1}^{T}X_{t})(X_{t+i}-\frac{1}{T}\sum_{t=1}^{T}X_{t})  \notag \\
&&+O_{P^{\ast }}(l^{2}/T),  \label{Var}
\end{eqnarray}%
where $E^{\ast }$ and $Var^{\ast }$ denotes the expectation and the variance
operators with respect to $P^{\ast }$ (the probability law governing the
resampled series or the probability law governing the $iid$ uniform random
variables, conditional on the sample), and where $O_{P^{\ast }}(l/T)\
(O_{P^{\ast }}(l^{2}/T))$ denotes a term converging in probability $P^{\ast
} $ to zero, as $l/T\rightarrow 0$ ($l^{2}/T\rightarrow 0).$

In the case of recursive estimators, we proceed the bootstrap similarly as
follows. Begin by resampling $b$ blocks of length $l$ from the full sample,
with $lb=T.$ For any given\ $\tau ,$ it is necessary to jointly resample\ $%
X_{t},X_{t+1},...,X_{t+\tau }.$ More precisely, let $Z^{t,\tau
}=(X_{t},X_{t+1},...,X_{t+\tau }),$ $t=1,...,T-\tau .$ Now, resample $b$
overlapping blocks of length $l$ from $Z^{t,\tau }.$ This yields $Z^{t,\ast
}=(X_{t}^{\ast },X_{t+1}^{\ast },...,X_{t+\tau }^{\ast }),$ $t=1,...,T-\tau .
$

\textbf{Step 2:}\textit{\ }Re-estimate $\widehat{\theta }_{t,N,h}^{\ast }(%
\widehat{\theta }_{T,N,L,h}^{\ast })$ using the bootstrap sample $Z^{t,\ast
}=(X_{t}^{\ast },X_{t+1}^{\ast },...,X_{t+\tau }^{\ast }),$ $t=1,...,T-\tau $
(or full sample $X_{1}^{\ast },X_{2}^{\ast },...,X_{T}^{\ast }$ ). Recall
that if we use the entire sample for the estimation, as the specification
test in CS(2005) and BCS(2008), then $\widehat{\theta }_{t,N,h}^{\ast }$ is
denoted as $\widehat{\theta }_{T,N,h}^{\ast }.$ The bootstrap estimators for
SGMM and NPSQML are presented below:

\textit{Bootstrap (recursive) SGMM Estimators}

If the full sample is used in the specification test as in CS (2005) and
BCS(2008), the bootstrap estimator is constructed straightforward as

\begin{equation*}
\widehat{\theta }_{T,L,h}^{\ast }=\arg \min_{\theta \in \Theta }\left( \frac{%
1}{T}\sum_{j=1}^{T}g(X_{j}^{\ast })-\frac{1}{L}\sum_{i=1}^{L}g(X_{j,h}^{%
\theta })\right) ^{\prime }W_{T}^{\ast -1}\left( \frac{1}{T}%
\sum_{j=1}^{T}g(X_{j}^{\ast })-\frac{1}{L}\sum_{i=1}^{L}g(X_{j,h}^{\theta
})\right) ,
\end{equation*}

where $W_{T}^{-1}$ and $g(.)$ are defined in (\ref{NW}) and $L$ is the
length of each simulation path.

Note that it is convenient not to resample the simulated series as the
simulation error vanishes asymptotically. In implementation, we do not have
mimic its contribution to the covariate kernel.

In the case of predictive density type model selection where recursive
estimators are needed, define the bootstrap analog as

\begin{eqnarray*}
\widehat{\theta }_{t,L,h}^{\ast } &=&\arg \min_{\theta \in \Theta }\left( 
\frac{1}{t}\sum_{j=1}^{t}\left( \left( g(X_{j}^{\ast })-\frac{1}{T}%
\sum_{j^{\prime }=1}^{T}g(X_{j^{\prime }})\right) -\left( \frac{1}{L}%
\sum_{i=1}^{L}g(X_{j,h}^{\theta })-\frac{1}{L}\sum_{i=1}^{L}g(X_{j,h}^{%
\widehat{\theta }_{t,L,h}})\right) \right) \right) ^{\prime } \\
&&\Omega _{t}^{\ast -1}\left( \frac{1}{t}\sum_{j=1}^{t}\left( \left(
g(X_{j}^{\ast })-\frac{1}{T}\sum_{j^{\prime }=1}^{T}g(X_{j^{\prime
}})\right) -\left( \frac{1}{L}\sum_{i=1}^{L}g(X_{j,h}^{\theta })-\frac{1}{L}%
\sum_{i=1}^{L}g(X_{j,h}^{\widehat{\theta }_{t,L,h}})\right) \right) \right)
\\
&=&\arg \min_{\theta \in \Theta }G_{t,L,h}^{\ast }(\theta )^{\prime }\Omega
_{t}^{\ast -1}G_{t,L,h}^{\ast }(\theta ),
\end{eqnarray*}

where 
\begin{equation*}
\Omega _{t}^{\ast -1}=\frac{1}{t}\sum_{\nu =-l_{t}}^{l_{t}}w_{\nu
,t}\sum_{j=\nu +1+l_{t}}^{t-l_{t}}\left( g(X_{j}^{\ast })-\frac{1}{T}%
\sum_{j^{\prime }=1}^{T}g(X_{j^{\prime }})\right) \left( g(X_{j-\nu }^{\ast
})-\frac{1}{T}\sum_{j^{\prime }=1}^{T}g(X_{j^{\prime }})\right)
\end{equation*}

Note that each bootstrap term is recentered around the (full) sample mean.
The intuition behind the particular recentering in bootstrap recursive SGMM\
estimator is that it ensures that the mean of the bootstrap moment
conditions, evaluated at $\widehat{\theta }_{t,L,h}$ is zero, up to a
negligible term. Specifically, we have

\begin{eqnarray*}
&&E^{\ast }\left( \frac{1}{t}\sum_{j=1}^{t}\left( g(X_{j}^{\ast })-\frac{1}{T%
}\sum_{j^{\prime }=1}^{T}g(X_{j^{\prime }})\right) -\left( \frac{1}{L}%
\sum_{i=1}^{L}g(X_{j,h}^{\widehat{\theta }_{t,L,h}^{\ast }})-\frac{1}{L}%
\sum_{i=1}^{L}g(X_{j,h}^{\widehat{\theta }_{t,L,h}})\right) \right) \\
&=&E^{\ast }(g(X_{j}^{\ast }))-\frac{1}{T}\sum_{j^{\prime
}=1}^{T}g(X_{j^{\prime }})=O(l/T),\text{ with }l=o(T^{1/2}),
\end{eqnarray*}

where the $O(l/T)$ term is due to the end block effect (see Corradi and
Swanson (2007b) for further discussion).

\textit{Bootstrap Recursive NPSQML Estimators}

Let $Z^{t,\ast }=(X_{t}^{\ast },X_{t+1}^{\ast },...,X_{t+\tau }^{\ast }),$ $%
t=1,...,T-\tau $. For each simulation replication, generate $T-1$ paths of
length one, using $X_{1}^{\ast },...,X_{T-1}^{\ast }$ as starting values,
and so obtaining $X_{t,j}^{\theta }(X_{t-1}^{\ast }),$ for $t=2,...,T-1,$ $%
i=1,...,M.$ Further, let:%
\begin{equation*}
\widehat{f}_{M,h}^{\ast }\left( X_{t}^{\ast }|X_{t-1}^{\ast },\theta \right)
=\frac{1}{M\xi _{M}}\sum_{i=1}^{M}K\left( \frac{X_{t,i,h}^{\theta
}(X_{t-1}^{\ast })-X_{t}^{\ast }}{\xi _{M}}\right) ,
\end{equation*}%
Now, for $t=R,...,R+P-1$, define:%
\begin{eqnarray*}
\widehat{\theta }_{t,M,h}^{\ast } &=&\arg \max_{\theta \in \Theta }\frac{1}{t%
}\sum_{l=2}^{t}\left( \ln \widehat{f}_{M,h}\left( X_{l}^{\ast
}|X_{l-1}^{\ast },\theta \right) \tau _{M}\left( \widehat{f}_{M,h}\left(
X_{l}^{\ast }|X_{l-1}^{\ast },\theta \right) \right) \right. \\
&&-\theta ^{\prime }\left( \frac{1}{T}\sum_{l^{\prime }=2}^{T}\frac{\nabla
_{\theta }\widehat{f}_{M,h}\left( X_{l^{\prime }}|X_{l^{\prime }-1},\theta
\right) }{\widehat{f}_{M,h}\left( X_{l^{\prime }}^{\ast }|X_{t-l^{\prime
}}^{\ast },\theta \right) }\left\vert _{\theta =\widehat{\theta }%
_{t,M,h}}\right. \tau _{M}\left( \widehat{f}_{M,h}\left( X_{l^{\prime
}}|X_{l^{\prime }-1},\widehat{\theta }_{t,M,h}\right) \right) \right. \\
&&\left. \left. +\tau _{M}^{\prime }\left( \widehat{f}_{M,h}\left(
X_{l^{\prime }}|X_{l^{\prime }-1},\widehat{\theta }_{t,M,h}\right) \right)
\nabla _{\theta }\widehat{f}_{M,h}\left( X_{l^{\prime }}|X_{l^{\prime
}-1},\theta \right) \left\vert _{\widehat{\theta }_{t,M,h}}\right. \ln 
\widehat{f}_{M,h}\left( X_{l^{\prime }}|X_{l^{\prime }-1},\widehat{\theta }%
_{t,M,h}\right) \right) \right) ,
\end{eqnarray*}%
where $\tau _{M}^{\prime }(\cdot )$ denotes the derivative of \ $\tau
_{M}(\cdot )$ with respect to its argument. Note that each term in the
simulated likelihood is recentered around the (full) sample mean of the
score, evaluated at $\widehat{\theta }_{t,M,h}.$ This ensures that the
bootstrap score has mean zero, conditional on the sample. The recentering
term requires computation of $\nabla _{\theta }\widehat{f}_{M,h}\left(
X_{l^{\prime }}|X_{l^{\prime }-1},\widehat{\theta }_{t,M,h}\right) ,$ which
is not known in closed form. Nevertheless, it can be computed numerically,
by simply taking the numerical derivative of the simulated likelihood.

\textit{Bootstrap Estimators for Multifactor Model}

The SGMM and the bootstrap SGMM estimators in the case of multifactor model
are similar as in one factor model. The difference is that the simulation
scheme (\ref{gmil1a}) and (\ref{gmil1b}) are used instead of (\ref{mil}).

For recursive NPSQML estimators, to construct the bootstrap counterpart $%
\widehat{\theta }_{_{t,M,L,h}}^{\ast }$ of $\widehat{\theta }_{t,M,L,h},$
since $M/T$ $\rightarrow \infty $ and $L/T\rightarrow \infty ,$ the
contribution of simulation error is asymptotically negligible. Hence, there
is no need to resample the simulated observations or the simulated initial
values for volatility. Define:%
\begin{equation*}
\widehat{f}_{M,L,h}\left( X_{t}^{\ast }|X_{t-1}^{\ast },\theta \right) =%
\frac{1}{L}\sum_{j=1}^{L}\frac{1}{M\xi _{M}}\sum_{i=1}^{M}K\left( \frac{%
X_{t,i,h}^{\theta }(X_{t-1}^{\ast },V_{j}^{\theta })-X_{t}^{\ast }}{\xi _{M}}%
\right) .
\end{equation*}%
Now, for $t=R,...,R+P-1$, define:%
\begin{eqnarray*}
&&\widehat{\theta }_{_{t,M,L,h}}^{\ast } \\
&=&\arg \max_{\theta \in \Theta }\frac{1}{t}\sum_{l=2}^{t}\left( \log 
\widehat{f}_{_{t,M,L,h}}\left( X_{l}^{\ast }|X_{l-1}^{\ast },\theta \right)
\tau _{M}\left( \widehat{f}_{_{t,M,L,h}}\left( X_{l}^{\ast }|X_{l-1}^{\ast
},\theta \right) \right) \right. \\
&&-\theta ^{\prime }\left( \frac{1}{T}\sum_{l^{\prime }=2}^{T}\frac{\nabla
_{\theta }\widehat{f}_{_{t,M,L,h}}\left( X_{l^{\prime }}|X_{l^{\prime
}-1},\theta \right) }{\widehat{f}_{_{t,M,L,h}}\left( X_{l^{\prime }}^{\ast
}|X_{t-l^{\prime }}^{\ast },\theta \right) }\left\vert _{\theta
_{_{t,M,L,h}}}\right. \tau _{M}\left( \widehat{f}_{_{t,M,L,h}}\left(
X_{l^{\prime }}|X_{l^{\prime }-1},\widehat{\theta }_{_{t,M,L,h}}\right)
\right) \right. \\
&&\left. \left. +\tau _{M}^{\prime }\left( \widehat{f}_{_{t,M,L,h}}\left(
X_{l^{\prime }}|X_{l^{\prime }-1},\widehat{\theta }_{t,M,L,h}\right) \right)
\nabla _{\theta }\widehat{f}_{_{t,M,L,h}}\left( X_{l^{\prime }}|X_{l^{\prime
}-1},\theta \right) \left\vert _{\widehat{\theta }_{t,M,L,h}}\right. \ln 
\widehat{f}_{_{t,M,L,h}}\left( X_{l^{\prime }}|X_{l^{\prime }-1},\widehat{%
\theta }_{_{t,M,L,h}}\right) \right) \right) ,
\end{eqnarray*}%
where $\tau _{M}^{\prime }(\cdot )$ denotes the derivative with respect to
its argument.

Of note is that each bootstrap term is recentered around the (full) sample
mean. This is necessary because the bootstrap statistic is constructed using
the last $P$ resampled observations, which in turn have been resampled from
the full sample. In particular, this is necessary regardless of the ratio, $%
P/R$. In addition, in the case $P/R\rightarrow 0$, so that there is no need
to mimic parameter estimation error, the bootstrap statistics can be
constructed using $\widehat{\theta }_{_{t,M,L,h}}$ instead of $\widehat{%
\theta }_{t,M,L,h}^{\ast }.$

\textbf{Step 3:} Using the same set of random variables used in the
construction of the actual statistics, construct $X_{i,t+\tau ,\ast }^{%
\widehat{\theta }_{t,N,h}^{\ast }}$ or $X_{k,i,t+\tau ,\ast }^{\widehat{%
\theta }_{t,N,h}^{\ast }},$ $i=1,....,S$ and $t=1,...,T-\tau .$ Note that we
do not need resample the simulated series (as $L/T\rightarrow \infty ,$%
simulation error is asymptotically negligible). Instead, simulate the series
using bootstrap estimators and using bootstrapped values as starting values.

\textbf{Step 4: }Corresponding bootstrap statistics $V_{T,N,h}^{2\ast }$ (or 
$Z_{T,N,h}^{\ast },$ $D_{k,P,S}^{\ast },$ $SV_{T,N,h}^{2\ast },$ $%
SZ_{T,N,h}^{\ast },$ $SD_{k,P,S}^{\ast }$ depending on the types of tests)\
which are built on $\widehat{\theta }_{t,N,h}^{\ast }$ $(\widehat{\theta }%
_{t,N,L,h}^{\ast })$ then are followed correspondingly$.$ For the numerical
implementation, again, of importance note is that in the case where we pick
the choice $P/R\rightarrow 0,P,T,R\rightarrow \infty ,$ there is no need to
re-estimate $\widehat{\theta }_{t,N,h}^{\ast }(\widehat{\theta }%
_{t,N,L,h}^{\ast }).$ $\widehat{\theta }_{_{t,N,h}}$ $(\widehat{\theta }%
_{t,N,L,h}^{\ast })$ can be used in all the bootstrap replications.

\textbf{Step 5:} Repeat the bootstrap Steps 1-4 $B$ times, and generate the
empirical distribution of the $B$ bootstrap statistics.

\section{Summary of Empirical Applications of the Tests}

In this section, we briefly review some empirical applications of the
methods discussed above. We start with unconditional distribution test, as
in CS (2005), then give a specific empirical example using the conditional
distribution test from BCS (2008). Finally, we briefly discuss on
conditional distribution specification test applied to multiple competing
models. The list of the diffusion models considered are provided in Table 1.

\begin{center}
Table 1: Specification Test Hypotheses of Continuous Time Spot Rate Process%
\footnote{%
Note that the 3rd column, "Reference \& Data" provides the referenced papers
and data used in empirical applications. In the 4th column, H1, H2 and H3
denote Hypothesis 1, Hypothesis 2 and Hypothesis 3, respectively. The
hypotheses are presented corresponding to the references in the third
column. For example, for CIR\ model, H2 corresponds to BCS (2008) and H2, H3
correspond to Cai and Swanson (2011).}

$%
\begin{array}{cccc}
\hline\hline
\text{{\small Model}} & \text{{\small Specification }} & \text{{\small %
Reference \& Data}} & \text{{\small Hypothesis }} \\ 
\text{{\small Wong\QQfnmark{%
This model is a simplified version of the CIR model. For convenience, we
refer to this model as Wong. }}} & 
\begin{array}{c}
{\footnotesize dr(t)=(\alpha -\lambda -r(t))dt+}\sqrt{{\footnotesize \alpha
r(t)}}{\footnotesize dW}_{r}{\footnotesize (t)} \\ 
\end{array}
& 
\begin{array}{c}
\text{{\small CS (2005)}} \\ 
\text{{\small Simulated Data}}%
\end{array}
& \text{{\small H1}} \\ 
\text{{\small CIR}} & 
\begin{array}{c}
{\footnotesize dr(t)=k}_{r}\left( \overline{{\footnotesize r}}-%
{\footnotesize r(t)}\right) {\footnotesize dt+}\sqrt{{\footnotesize V(t)}}%
{\footnotesize dW}_{r}{\footnotesize (t),} \\ 
\end{array}
& 
\begin{array}{c}
\begin{array}{c}
\text{{\small BCS (2008)}} \\ 
\text{{\small Eurodollar Rate }} \\ 
\text{({\small 1971-2005)}}%
\end{array}
\\ 
\begin{array}{c}
\text{{\small Cai \& Swanson (2011)}} \\ 
\text{{\small Eurodollar Rate }} \\ 
\text{({\small 1971-2008)}}%
\end{array}%
\end{array}
& 
\begin{array}{c}
\text{{\small H2}} \\ 
\\ 
\\ 
\text{{\small H2,H3}}%
\end{array}
\\ 
\text{{\small CEV}} & {\footnotesize dr(t)=k}_{r}\left( \overline{%
{\footnotesize r}}-{\footnotesize r(t)}\right) {\footnotesize dt+\sigma
_{r}r(t)}^{\rho }{\footnotesize dW}_{r}{\footnotesize (t)} & 
\begin{array}{c}
\text{{\small Cai \& Swanson (2011)}} \\ 
\text{{\small Eurodollar Rate }} \\ 
\text{({\small 1971-2008)}}%
\end{array}
& \text{{\small H2,H3}} \\ 
\text{{\small SV\QQfnmark{%
Data used for Stochatic Volatility (SV) model is the same as in CIR Model.}}}
& 
\begin{array}{c}
{\footnotesize dr(t)=k}_{r}\left( \overline{{\footnotesize r}}-%
{\footnotesize r(t)}\right) {\footnotesize dt+}\sqrt{{\footnotesize V(t)}}%
{\footnotesize dW}_{r}{\footnotesize (t),} \\ 
{\footnotesize dV(t)=k}_{v}\left( \overline{{\footnotesize v}}-%
{\footnotesize V(t)}\right) {\footnotesize dt+\sigma }_{v}\sqrt{%
{\footnotesize V(t)}}{\footnotesize dW}_{v}{\footnotesize (t),}%
\end{array}
& 
\begin{array}{c}
\text{{\small BCS (2008)}} \\ 
\text{{\small Cai \& Swanson (2011)}}%
\end{array}
& 
\begin{array}{c}
\text{{\small H2}} \\ 
\text{{\small H2,H3}}%
\end{array}
\\ 
&  &  &  \\ 
\text{{\small SVJ\QQfnmark{%
Data used for Stochatic Volatility and Jump (SVJ) model is the same as in
CIR Model.}}} & 
\begin{array}{c}
{\footnotesize dr(t)=k}_{r}\left( \overline{{\footnotesize r}}-%
{\footnotesize r(t)}\right) {\footnotesize dt+}\sqrt{{\footnotesize V}%
{\small ({\footnotesize t})}}{\footnotesize dW}_{r}{\footnotesize (t)+J}_{u}%
{\footnotesize dq}_{u}{\footnotesize -J}_{d}{\footnotesize dq}_{d}%
{\footnotesize ,} \\ 
{\footnotesize dV(t)=k}_{v}\left( \overline{{\footnotesize v}}-%
{\footnotesize V(t)}\right) {\footnotesize dt+\sigma }_{v}\sqrt{%
{\footnotesize V(t)}}{\footnotesize dW}_{v}\left( {\footnotesize t}\right) 
{\footnotesize ,}%
\end{array}
& 
\begin{array}{c}
\text{{\small BCS (2008)}} \\ 
\text{{\small Cai \& Swanson (2011)}}%
\end{array}
& 
\begin{array}{c}
\text{{\small H2}} \\ 
\text{{\small H2,H3}}%
\end{array}
\\ 
&  &  &  \\ 
\text{{\small CHEN}} & 
\begin{array}{c}
{\footnotesize dr}\left( {\footnotesize t}\right) {\footnotesize =\kappa }%
_{r}\left( {\footnotesize \theta }\left( {\footnotesize t}\right) -%
{\footnotesize r}\left( {\footnotesize t}\right) \right) {\footnotesize dt+}%
\sqrt{{\footnotesize V}\left( {\footnotesize t}\right) }{\footnotesize dW}%
_{r}{\footnotesize ,} \\ 
{\footnotesize dV}\left( t\right) {\footnotesize =\kappa }_{v}\left( 
\overline{{\footnotesize v}}-{\footnotesize V(t)}\right) {\footnotesize %
dt+\sigma }_{v}\sqrt{{\footnotesize V(t)}}{\footnotesize dW}_{v}\left( 
{\small t}\right) {\footnotesize ,} \\ 
{\footnotesize d\theta }\left( {\footnotesize t}\right) {\footnotesize %
=\kappa }_{\theta }\left( \overline{{\footnotesize \theta }}{\small -}%
{\footnotesize \theta }\left( {\footnotesize t}\right) \right) 
{\footnotesize dt+\sigma }_{\theta }\sqrt{{\small \theta }\left( {\small t}%
\right) }{\footnotesize dW}_{\theta }\left( {\small t}\right) {\footnotesize %
,}%
\end{array}
& 
\begin{array}{c}
\text{{\small Cai \& Swanson (2011)}} \\ 
\text{{\small Euro Dollar Rate}} \\ 
\text{({\small 1971-2008)}}%
\end{array}
& \text{{\small H2, H3}} \\ 
&  &  &  \\ 
\text{{\small CHENJ}} & 
\begin{array}{c}
{\footnotesize dr}\left( {\footnotesize t}\right) {\footnotesize =\kappa }%
_{r}\left( {\footnotesize \theta }\left( {\footnotesize t}\right) -%
{\footnotesize r}\left( {\footnotesize t}\right) \right) {\footnotesize dt+}%
\sqrt{{\footnotesize V(t)}}{\footnotesize dW}_{r}\left( {\small t}\right) 
{\footnotesize +J}_{u}{\footnotesize dq}_{u}{\footnotesize -J}_{d}%
{\footnotesize dq}_{d}{\footnotesize ,} \\ 
{\footnotesize dV}\left( {\small t}\right) {\footnotesize =\kappa }%
_{v}\left( \overline{{\footnotesize v}}-{\footnotesize V(t)}\right) 
{\footnotesize dt+\sigma }_{v}\sqrt{{\footnotesize V(t)}}{\footnotesize dW}%
_{v}\left( {\small t}\right) {\footnotesize ,} \\ 
{\footnotesize d\theta }\left( {\small t}\right) {\footnotesize =\kappa }%
_{\theta }\left( \overline{{\footnotesize \theta }}{\small -}{\footnotesize %
\theta }\left( {\footnotesize t}\right) \right) {\footnotesize dt+\sigma }%
_{\theta }\sqrt{{\small \theta }\left( {\small t}\right) }{\footnotesize dW}%
_{\theta }\left( {\small t}\right) {\footnotesize ,}%
\end{array}
& 
\begin{array}{c}
\text{{\small Cai \& Swanson (2011}} \\ 
\text{{\small Euro Dollar Rate }} \\ 
\text{({\small 1971-2008)}}%
\end{array}
& \text{{\small H2, H3}}%
\end{array}%
\QQfntext{-2}{
This model is a simplified version of the CIR model. For convenience, we
refer to this model as Wong. }
\QQfntext{1}{
Data used for Stochatic Volatility (SV) model is the same as in CIR Model.}
\QQfntext{1}{
Data used for Stochatic Volatility and Jump (SVJ) model is the same as in
CIR Model.}$
\end{center}

Note that specification testing of the first model - a simplified version of
the CIR model (we refer to this model as Wong) is carried out using the
unconditional distribution test. With the cumulative distribution function
known in closed form as in (\ref{wongd}), the test statistic can be
straightforwardly calculated. It is also convenient to use GMM\ estimation
in this case as the first two moments are known in closed form, i.e. $\alpha
-\lambda $ and $\alpha /2(\alpha -\beta ),$ respectively. CS (2005) examine
Hypothesis 1 using simulated data. Their Monte Carlo experiments suggest
that the test is useful, even for samples as small as 400 observations.

\textbf{Hypothesis 2} is tested in BCS (2008) and Cai and Swanson (2011).
For illustration, we focus on the results in BCS (2008) where CIR, SV and
SVJ models are empirically tested using the one-month Eurodollar deposit
rate (as a proxy for short rate) for the sample period January 6, 1971 -
September 30, 2005, which yields 1,813 weekly observations. Note that one
might apply these tests to other datasets including the monthly federal
funds rate, the weekly 3-month T-bill rate, the weekly US dollar swap rate,
the monthly yield on zero-coupon bonds with different maturities, and the
6-month LIBOR. Some of these variables have been examined elsewhere, for
example in Ait-Sahalia (1999), Andersen, Benzoni and Lund (2004), Dai and
Singleton (2000), Diebold and Li (2006, 2007), and Piazzesi (2001).

The statistic needed to apply the test discussed in Section 3.1.2 is: 
\begin{equation*}
Z_{T}=\sup_{v\in V}\left\vert Z_{T}(v)\right\vert ,
\end{equation*}%
where 
\begin{equation*}
Z_{T}(v)=\frac{1}{\sqrt{T-\tau }}\sum_{t=1}^{T-\tau }\left( \frac{1}{S}%
\sum_{s=1}^{S}1\left\{ \underline{u}\leq X_{s,t+\tau }^{\widehat{\theta }%
_{T,N,h}}\leq \overline{u}\right\} -1\{\underline{u}\leq X_{t+\tau }\leq 
\overline{u}\}\right) 1\left\{ X_{t}\leq v\right\} ;
\end{equation*}%
and 
\begin{equation*}
Z_{T}^{\ast }=\sup_{v\in V}\left\vert Z_{T}^{\ast }(v)\right\vert ,
\end{equation*}%
where 
\begin{eqnarray*}
Z_{T}^{\ast }(v) &=&\frac{1}{\sqrt{T-\tau }}\sum_{t=1}^{T-\tau }\left( \frac{%
1}{S}\sum_{s=1}^{S}1\left\{ \underline{u}\leq X_{s,t+\tau ,\ast }^{\widehat{%
\theta }_{T,N,h}^{\ast }}\leq \overline{u}\right\} -1\{\underline{u}\leq
X_{t+\tau }^{\ast }\leq \overline{u}\}\right) 1\left\{ X_{t}^{\ast }\leq
v\right\}  \\
&&-\frac{1}{\sqrt{T-\tau }}\sum_{t=1}^{T-\tau }\left( \frac{1}{S}%
\sum_{s=1}^{S}1\left\{ \underline{u}\leq X_{s,t+\tau }^{\widehat{\theta }%
_{T,N,h}}\leq \overline{u}\right\} -1\{\underline{u}\leq X_{t+\tau }\leq 
\overline{u}\}\right) 1\left\{ X_{t}\leq v\right\} .
\end{eqnarray*}%
For the case of stochastic volatility models, similarly we have: 
\begin{equation*}
SZ_{T}=\sup_{v\in V}\left\vert SZ_{T}(v)\right\vert ,
\end{equation*}%
where 
\begin{equation*}
SZ_{T}(v)=\frac{1}{\sqrt{T-\tau }}\sum_{t=1}^{T-\tau }\left( \frac{1}{LS}%
\sum_{j=1}^{L}\sum_{s=1}^{S}1\left\{ \underline{u}\leq X_{j,s,t+\tau }^{%
\widehat{\theta }_{T,N,h}}\leq \overline{u}\right\} -1\{\underline{u}\leq
X_{t+\tau }\leq \overline{u}\}\right) 1\left\{ X_{t}\leq v\right\} ;
\end{equation*}%
and its bootstrap analog 
\begin{equation*}
SZ_{T}^{\ast }=\sup_{v\in V}\left\vert SZ_{T}^{\ast }(v)\right\vert ,
\end{equation*}%
where 
\begin{eqnarray*}
SZ_{T}^{\ast }(v) &=&\frac{1}{\sqrt{T-\tau }}\sum_{t=1}^{T-\tau }\left( 
\frac{1}{LS}\sum_{j=1}^{L}\sum_{s=1}^{S}1\left\{ \underline{u}\leq
X_{j,s,t+\tau ,\ast }^{\widehat{\theta }_{i,T,N,h}^{\ast }}\leq \overline{u}%
\right\} -1\{\underline{u}\leq X_{t+\tau }^{\ast }\leq \overline{u}\}\right)
1\left\{ X_{t}^{\ast }\leq v\right\}  \\
&&-\frac{1}{\sqrt{T-\tau }}\sum_{t=1}^{T-\tau }\left( \frac{1}{LS}%
\sum_{j=1}^{L}\sum_{s=1}^{S}1\left\{ \underline{u}\leq X_{j,s,t+\tau }^{%
\widehat{\theta }_{i,T,N,h}}\leq \overline{u}\right\} -1\{\underline{u}\leq
X_{t+\tau }\leq \overline{u}\}\right) 1\left\{ X_{t}\leq v\right\} .
\end{eqnarray*}

BCS (2008) carry out these tests using $\tau -Step$ ahead confidence
intervals. They set $\tau =\left\{ 1,2,4,12\right\} $ which corresponds to
one week, two week, one month, and one quarter ahead intervals and set $(%
\underline{u},\overline{u})$ $=$ $(\overline{X}\pm 0.5\sigma _{X},$ $%
\overline{X}\pm \sigma _{X}),$ covering 46.3\% and 72.4\% coverage,
respectively. $\overline{X}\ $and $\sigma _{X}$ are the mean and variance of
an initial sample of data. In addition, $S=\{10T,20T\}$ and $%
l=\{5,10,20,50\}.$

For illustrative purposes, we report one case from BCS (2008). The test is
implemented by setting $S=10T$ and $l=25$ for the calculation of both $Z_{T}$
and $SZ_{T}.$ In the Table 2, single, double, and triple starred entries
represent rejection using $20\%,10\%,$ and $5\%$ size tests, respectively.
Not surprisingly, the findings are consistent with some other papers in the
specification test literature such as such as A\"{\i}t-Sahalia (1996) and
Bandi (2002). Namely, the CIR model is rejected using $5\%$ size tests in
almost all cases. When considering SV and SVJ models, smaller confidence
intervals appear to lead to more model rejections. Moreover, results are
somewhat mixed when evaluating the SVJ model, with a slightly higher
frequency of rejection than in the case of SV models.\bigskip 

\begin{center}
Table 2: Empirical Illustration of Specification Testing - $CIR$, $SV,$ $SVJ$
Models

{\scriptsize \ }

{\scriptsize 
\begin{tabular}{lllllllllll}
\hline\hline
& $(\underline{u},\overline{u})$ & \multicolumn{3}{c}{$CIR$} & 
\multicolumn{3}{c}{$SV$} & \multicolumn{3}{c}{$SVJ$} \\ 
&  & $Z_{T}$ & {\tiny 5\% CV} & {\tiny 10\% CV} & SZ$_{T}$ & {\tiny 5\% CV}
& {\tiny 10\% CV} & S$Z_{T}$ & {\tiny 5\% CV} & {\tiny 10\% CV} \\ \hline
\multicolumn{11}{c}{{\tiny \ }$l=25$} \\ 
1 & $\overline{X}\pm 0.5\sigma _{X}$ & 0.5274*** & 0.2906 & 0.3545 & 
0.9841*** & 0.8729 & 0.9031 & 1.1319 & 1.8468 & 2.1957 \\ 
& $\overline{X}\pm \sigma _{X}$ & 0.4289*** & 0.2658 & 0.3178 & 0.6870 & 
0.6954 & 0.7254 & 1.2272* & 1.1203 & 1.3031 \\ 
2 & $\overline{X}\pm 0.5\sigma _{X}$ & 0.6824*** & 0.4291 & 0.4911 & 0.4113
& 1.3751 & 1.4900 & 0.9615* & 0.8146 & 1.1334 \\ 
& $\overline{X}\pm \sigma _{X}$ & 0.4897* & 0.4264 & 0.5182 & 0.3682 & 1.1933
& 1.2243 & 1.2571 & 1.3316 & 1.4096 \\ 
4 & $\overline{X}\pm 0.5\sigma _{X}$ & 0.8662** & 0.7111 & 0.8491 & 1.2840 & 
2.3297 & 2.6109 & 1.5012* & 1.1188 & 1.6856 \\ 
& $\overline{X}\pm \sigma _{X}$ & 0.8539* & 0.7521 & 0.9389 & 1.0472 & 2.2549
& 2.2745 & 0.9901* & 0.9793 & 1.0507 \\ 
12 & $\overline{X}\pm 0.5\sigma _{X}$ & 1.1631* & 1.0087 & 1.3009 & 1.7687 & 
4.9298 & 5.2832 & 2.4237* & 2.0818 & 3.0640 \\ 
& $\overline{X}\pm \sigma _{X}$ & 1.0429 & 1.4767 & 2.0222 & 1.7017 & 5.2601
& 5.6522 & 1.4522 & 1.7400 & 2.1684 \\ \hline\hline
\end{tabular}
}

{\scriptsize \ }{\footnotesize 
\begin{minipage}{1\columnwidth}

\footnotesize
$^{(*)}$ Notes: Tabulated entries are test statistics and 5$\%$, 10$\%$ and 20$\%$ level critical values.
Test intervals are given in the second column of the table, for $\tau = ${$1,2,4,12$}. All tests are carried out using historical one-month Eurodollar deposit rate data for the period January 1971 - September 2005,
measured at a weekly frequency.
Single, double, and triple starred entries denote rejection at the 20$\%$,10$\%$, and 5$\%$ levels, respectively.
Additionally, $\overline{X}$ and $\sigma_{X}$
are the mean and standard deviation of the historical data.
See above for complete details.
\end{minipage}\bigskip }
\end{center}

Finally, turning to \textbf{Hypothesis 3, }Cai and Swanson (2011) use an
extended version of the above dataset, i.e. the one-month Eurodollar deposit
rate from January 1971 to April 2008 (1,996 weekly observations).
Specifically, they examine whether the CHEN model is the \textquotedblleft
best\textquotedblright\ model amongst multiple alternative models including
those outlined in Table 1. The answer is "yes". In this example, the test
was implemented using $D_{k,p,N}(u1,u2),$ as described in Sections 3.1 and
3.2, where $P=T/2$ and predictions are constructed using recursively
estimated models and the simulation sample length used to address latent
variable  initial values is set at $L=10T$. The choice of other inputs to
the test such as $\tau $ and interval $(\underline{u},\overline{u})$ are the
same as in BCS (2008). The number of replications $S,$ the block length $l$
and number of bootstrap replications are $S=10T,$ $l=20$ and $B=100$. 

Cai and Swanson (2011) also compare the Chen model with the so called Smooth
Transition Autoregression Model (STAR) model defined as follows:%
\begin{equation*}
r_{t}=(\theta _{1}+\beta _{1}r_{t-1})G(\gamma ,z_{t},c)+(\theta _{1}+\beta
_{2}r_{t-1})(1-G(\gamma ,z_{t},c))+u_{t}
\end{equation*}

where $u_{t}$ is a disturbance term, $\theta _{1},$ $\beta _{1},$ $\gamma $
, $\beta _{2},$ and $c$ are constants, $G(\cdot )$ is the logistic CDF (i.e. 
$G(\gamma ,z_{t},c)=\frac{1}{1+\exp (\gamma (z_{t}-c))}$ ), and the number
of lags, $p$ is selected via the use of Schwarz information criterion. Test
statistics and predictive density type \textquotedblleft mean square
forecast errors\textquotedblright\ (MSFEs) values are again calculated as in
Section 3.1 and 3.2.\footnote{%
See Table 6 in Cai and Swanson (2011) for complete details.}\ Their results
indicate that at a 90\% level of confidence, one cannot reject the null
hypothesis that the CHEN model generates predictive densities at least as
accurate as the STAR model, regardless of forecast horizon and confidence
interval width. Moreover, in almost all cases, the CHEN model has lower
MSFE, and the magnitude of the MSFE differential between the CHEN model and
STAR model rises as the forecast horizon increases. This confirms their
in-sample findings that the CHEN model also wins when carrying out in-sample
tests.

\section{Conclusion}

This paper reviews a class of specification and model selection type tests
developed by CS (2005), BCS (2008) and CS (2011) for continuous time models.
We begin with outlining\ the setup used to specify the types of diffusion
models considered in this paper. Thereafter, diffusion models in finance are
discussed, and testing procedures are outlined. Related testing procedures
are also discussed, both in contexts where models are assumed to be either
correctly specified under the null hypothesis or generically misspecified
under both the null and alternative test hypotheses. In addition to
discussing tests of correct specification and test for selecting amongst
alternative competing models, using both in-sample methods and via
comparison of predictive accuracy, methodology is outlined allowing for
parameter estimation, model and data simulation, and bootstrap critical
value construction. 

Several extensions that are left to future research are as follows. First,
it remains to construct specification tests that do not integrate out the
effects of latent factors. Additionally, it remains to examine the finite
sample properties of the estimators and bootstrap methods discussed in this
paper.

\newpage

\begin{center}
\bigskip
\end{center}

\section{References$^{{}}$}

\baselineskip=12pt

\renewcommand{\baselinestretch}{1.1}

\noindent A\"{\i}t-Sahalia, Y., (1996), Testing Continuous Time Models of
the Spot Interest Rate, \textit{Review of Financial Studies},\ 9,
385-426.\smallskip

\noindent A\"{\i}t-Sahalia, Y., (2002), Maximum Likelihood Estimation of
Discretely Sampled Diffusions: A Closed Form Approximation Approach, \textit{%
Econometrica}, 70, 223-262.\smallskip

\noindent A\"{\i}t-Sahalia, Y., (2007), Estimating Continuous-Time Models
Using Discretely Sampled Data, Econometric Society World Congress Invited
Lecture, in Advances in Economics and Econometrics, Theory and Applications,
Ninth World Congress, edited by Richard Blundell, Persson Torsten and
Whitney K. Newey, Econometric Society Monographs, Cambridge University
Press.\smallskip

\noindent A\"{\i}t-Sahalia, Y., J. Fan and H. Peng, (2009), Nonparametric
Transition-Based Tests for Diffusions, \textit{Journal of the American
Statistical Association}, 104, 1102-1116.\smallskip

\noindent Altissimo, F. and A. Mele, (2009), Simulated Nonparametric
Estimation of Dynamic Models with Application in Finance, \textit{Review of
Economic Studies}, 76, 413-450.\smallskip

\noindent Andersen, T.G., and J. Lund, (1997), Estimating Continuous Time
Stochastic Volatility Models of Short Term Interest Rates, \textit{Journal
of Econometrics}, 72, 343-377.\smallskip

\noindent Andersen, T.G., L. Benzoni and J. Lund, (2004), Stochastic
Volatility, Mean Drift, and Jumps in the Short-Term Interest Rate, Working
Paper, Northwestern University.\smallskip

\noindent Andrews, D.W.K., (1993), An Introduction to Econometric
Applications of Empirical Process Theory for Dependent Random Variables, 
\textit{Econometric Reviews}, 12, 183-216.\smallskip

\noindent Andrews, D.W.K., (1997), A Conditional Kolmogorov Test, \textit{%
Econometrica,}\emph{\ }65, 1097-1128. \smallskip

\noindent Bai, J., (2003), Testing Parametric Conditional Distributions of
Dynamic Models, \textit{Review of Economics and Statistics, }85,
531-549.\smallskip

\noindent Bandi, F., (2002), Short-Term Interest Rate Dynamics: A Spatial
Approach, \textit{Journal of Financial Economics,} 65, 73-110.\smallskip

\noindent Beckers, S. (1980), The Constant Elasticity of Variance Model and
US\ Implications for Option Pricing. \textit{Journal of Finance,} 35,
661-673.\smallskip

\noindent Bhardwaj, G., V. Corradi and N.R. Swanson, (2008), A Simulation
Based Specification Test for Diffusion Processes, \textit{Journal of
Business and Economic Statistics}, 26, 176-93.\smallskip

\noindent Black, F. and M. Scholes, (1973), The pricing of options and
corporate liabilities, \textit{Journal of Political Economy}, 81,
637--654.\smallskip

\noindent Bontemps, C., and N. Meddahi, (2005), Testing Normality: a GMM
Approach, \textit{Journal of Econometrics}, 124, 149-186.\smallskip

\noindent Brennan, M.J., and E.S. Schwartz, (1979), A Continuous-Time
Approach to the Pricing Bonds. \textit{Journal of Banking and Finance, }3,
133-155.\smallskip

\noindent Cai, L. and N.R. Swanson, (2011), An Empirical Assessment of Spot
Rate Model Stability. \textit{Journal of Empirical Finance}, 18,
743--764.\smallskip

\noindent Chan, K.C., G. A. Karolyi, F. A. Longstaff and A. B. Sanders,
(1992), An Empirical Comparison of Alternative Models of the Short-Term
Interest Rate, \textit{Journal of Finance,} 47, 1209-1227.\smallskip

\noindent Chen, L, (1996), Stochastic Mean and Stochastic Volatility - A
Three-Factor Model of Term Structure of Interest Rates and its Application
to the Pricing of Interest Rate Derivatives, Blackwell Publishers, Oxford,
UK.\smallskip

\noindent Chen, B. and Y. Hong, (2005), Diagnostics of Multivariate
Continuous-Time Models with Application to Affine Term Structure Models,
Working Paper, Cornell University.\smallskip

\noindent Clements, M.P. and J. Smith, (2000), Evaluating the Forecast
Densities of Linear and Nonlinear Models: Applications to Output Growth and
Unemployment, \textit{Journal of Forecasting}, 19, 255-276.\smallskip

\noindent Clements, M.P. and J. Smith, (2002), Evaluating Multivariate
Forecast Densities: A Comparison of Two Approaches, \textit{International
Journal of Forecasting}, 18, 397-407.\smallskip

\noindent Corradi, V. and N.R. Swanson, (2005), A Bootstrap Specification
Test for Diffusion Processes, \textit{Journal of Econometrics}, 124, 117-148.

\noindent Corradi, V. and N.R. Swanson, (2006), Predictive Density and
Conditional Confidence Interval AccuracyTests, \textit{Journal of
Econometrics}, 135, 187-228.\smallskip

\noindent Corradi, V. and N.R. Swanson, (2007a), Nonparametric Bootstrap
Procedures for Predictive Inference Based on Recursive Estimation Schemes, 
\textit{International Economic Review}, February 2007, 48, 67-109. \smallskip

\noindent Corradi, V. and N.R. Swanson, (2007b), Nonparametric Bootstrap
Procedures for Predictive Inference Based on Recursive Estimation Schemes, 
\textit{International Economic Review}, 48, 67-109.\smallskip

\noindent Corradi, V. and N.R. Swanson, (2011), Predictive Density
Construction and Accuracy Testing with Multiple Possibly Misspecified
Diffusion Models, \textit{Journal of Econometrics, }161,304 - 324..\smallskip

\noindent Courtadon, G., (1982), The Pricing of Options on Default-Free
Bonds. \textit{Journal of Financial and Quantitative Analysis}, 17,
75-100.\smallskip

\noindent Cox, J.C., J.E. Ingersoll and S.A. Ross, (1985), A Theory of the
Term Structure of Interest Rates, \textit{Econometrica}, 53,
385-407.\smallskip

\noindent Dai, Q. and Kenneth J. Singleton, (2000), Specification Analysis
of Affine Term Structure Models, \textit{Journal of Finance,} 55,
1943-1978.\smallskip

\noindent Diebold, F.X., T. Gunther and A.S. Tay, (1998), Evaluating Density
Forecasts with Applications to Finance and Management, \textit{International
Economic Review}, 39, 863-883.\smallskip

\noindent Diebold, F.X., J. Hahn and A.S. Tay, (1999), Multivariate Density
Forecast Evaluation and Calibrationin Financial Risk Management: High
Frequency Returns on Foreign Exchange, \textit{Review of Economics and
Statistics}, 81, 661-673.\smallskip

\noindent Diebold, F.X. and C. Li, (2006), Forecasting the Term Structure of
Government Bond Yields, \textit{Journal of Econometrics}, 130,
337-364.\smallskip

\noindent Diebold, F.X. and R.S. Mariano, (1995), Comparing Predictive
Accuracy, \textit{Journal of Business and Economic Statistics}, 13,
253-263.\smallskip

\noindent Diebold, F.X., A.S. Tay and K.D. Wallis, (1998), Evaluating
Density Forecasts of Inflation: The Survey of Professional Forecasters, in
Festschrift in Honor of C.W.J. Granger, eds. R.F. Engle and H. White, Oxford
University Press, Oxford.\smallskip

\noindent Duan, J.C., (2003), A Specification Test for Time Series Models by
a Normality Transformation, Working Paper, University of Toronto.\smallskip

\noindent Duffie, D. and R. Kan, (1996), A Yield-Factor Model of Interest
Rates. \textit{Mathematical Finance,} 6:379-406.\smallskip

\noindent Duffie, D., J. Pan, and K. Singleton, (2000), Transform Analysis
and Asset Pricing for Affine Junp-Diffusions. \textit{Econometrica}, 68,
1343--1376.\smallskip

\noindent Duffie, D. and K. Singleton, (1993), Simulated Moment Estimation
of Markov Models of Asset Prices, \textit{Econometrica,} 61,
929-952.\smallskip

\noindent Duong, D., and N.R. Swanson, (2010), Empirical Evidence on Jumps
and Large Fluctuations in Individual Stocks. Working Paper, Rutgers
University.\smallskip

\noindent Duong, D., and N.R. Swanson, (2011), Volatility in Discrete and
Continuous Time Models: A Survey with New Evidence on Large and Small Jumps, 
\textit{Missing Data Methods: Advances in Econometrics, }27B,
179-233.\smallskip

\noindent Emanuel, D. and J. Mcbeth, (1982), Further Results on the Constant
Elasticity of Variance Call Option Pricing Model, \textit{Journal of
Financial Quantitative Analysis,} 17, 533-554.\smallskip

\noindent Fermanian, J.-D. and B. Salani\'{e}, (2004), A Nonparametric
Simulated Maximum Likelihood Estimation Method, \textit{Econometric Theory},
20, 701-734.\smallskip

\noindent Gallant, A.R. and G. Tauchen, (1996), Which Moments to Match, 
\textit{Econometric Theory}, 12, 657-681.\smallskip

\noindent Gallant, A.R., and G. Tauchen (1997), Estimation of Continuous
Time Models for Stock Returns and Interest Rates, \textit{Macroeconomic
Dynamics,} 1, 135-168.\smallskip

\noindent Goncalves, S. and H. White, (2002), The Bootstrap of the Mean for
Dependent Heterogeneous Arrays, \textit{Econometric Theory}, 18,
1367-1384.\smallskip

\noindent Hall, A.R. and A. Inoue, (2003), The Large Sample Behavior of the
Generalized Method of Moments Estimator in Misspecified Models, J\textit{%
ournal of Econometrics}, 361-394.\smallskip

\noindent Hansen, B.E., (1996), Inference when a Nuisance Parameter is Not
Identified Under the Null Hypothesis, \textit{Econometrica}, 64,
413-430.\smallskip

\noindent Heston, S.L., (1993), A Closed Form Solution for Option with
Stochastic Volatility with Applications toBond and Currency Options, \textit{%
Review of Financial Studies}, 6, 327-344.\smallskip

\noindent Hong, Y., (2001), Evaluation of out-of-sample Probability Density
Forecasts with Applications to S\&P 500 Stock Prices, Working Paper, Cornell
University. \smallskip

\noindent Hong, Y.M., and H.T. Li, (2005), Nonparametric Specification
Testing for Continuous Time Models with Applications to Term Structure
Interest Rates, \textit{Review of Financial Studies},\ 18, 37-84.\smallskip

\noindent Hong, Y.M., H.T. Li, and F. Zhao, (2007), Can the random walk
model be beaten in out-of-sample density forecasts? Evidence from intraday
foreign exchange rates, \textit{Journal of Econometrics}, 141, 736-776.

\noindent Hull, J., and A. White, (1987), The Pricing of Options on Assets
with Stochastic Volatility, \textit{Journal of Finance}, 42,
281-300.\smallskip

\noindent Inoue, (2001), Testing for Distributional Change in Time Series, 
\textit{Econometric Theory}, 17, 156-187.\smallskip

\noindent Jiang, G.J., (1998), Nonparametric Modelling of US Term Structure
of Interest Rates and Implications on the Prices of Derivative Securities, 
\textit{Journal of Financial and Quantitative Analysis}, 33,
465-497.\smallskip

\noindent Karlin, S. and H.M. Taylor, (1981), \textbf{A Second Course in
Stochastic Processes}. Academic Press, San Diego.\smallskip

\noindent Kloeden, P.E. and E. Platen, (1999), \textbf{Numerical Solution of
Stochastic Differential Equations}, Springer and Verlag, New York.\smallskip

\noindent Kolmogorov A.N., (1933), Sulla Determinazione Empirica di una
Legge di Distribuzione, Giornale dell'Istitutodegli Attuari, 4,
83-91.\smallskip

\noindent Kristensen, D. and Y. Shin, (2008), Estimation of Dynamic Models
with Nonparametric Simulated Maximum Likelihood, CREATES Research Paper
2008-58, University of Aarhus and Columbia University.\smallskip

\noindent K\"{u}nsch H.R., (1989), The Jackknife and the Bootstrap for
General Stationary Observations, \textit{Annals of Statistics},\ 17,
1217-1241.\smallskip

\noindent Marsh, T. and E. Rosenfeld, (1983), Stochastic Processes for
Interest Rates and Equilibrium Bond Prices. \textit{Journal of Finance,} 38,
635-646.\smallskip

\noindent Meddahi, N., (2001), An Eigenfunction Approach for Volatility
Modeling, Working Paper, University of Montreal

\noindent Merton, C.R., (1973), Theory of Rational Option Pricing, \textit{%
The Bell Journal of Economics and Management Science,} 4, 141-183\smallskip

\noindent Nelson, D.B., (1990), ARCH as Diffusion Approximations, \textit{%
Journal of Econometrics}, 45, 7-38\smallskip

\noindent Pardoux, E. and D. Talay, (1985), Discretization and Simulation of
Stochastic Differential Equations, \textit{Acta Applicandae Mathematicae,}%
\emph{\ }3, 23-47.\smallskip

\noindent Piazzesi, M., (2001), Macroeconomic jump effects and the yield
curve, working paper, UCLA.\smallskip

\noindent Piazzesi, M., (2004), Affine Term Structure Models, Manuscript
prepared for the Handbook of Financial Econometrics, University of
California at Los Angeles.\smallskip

\noindent Piazzesi, M., (2005), Bond Yields and the Federal Reserve, \textit{%
Journal of Political Economy,} 113, 311-344.\smallskip

\noindent Politis, D.N., J.P. Romano and M. Wolf, (1999), \textbf{Subsampling%
}\textit{, }Springer and Verlag, New York.\smallskip

\noindent Pritsker, M., (1998), Nonparametric Density Estimators and Tests
of Continuous Time Interest Rate Models, \textit{Review of Financial Studies}%
, 11, 449-487.\smallskip

\noindent Rosenblatt, M., (1952), Remarks on a Multivariate Transformation, 
\textit{Annals of Mathematical Statistics}, 23,470-472.\smallskip

\noindent Singleton, K.J. (2006), \textbf{Empirical Dynamic Asset Pricing -
Model Specification and Econometric Assessment}, Princeton University Press:
Princeton and Oxford.\smallskip

\noindent Smirnov N., (1939), On the Estimation of the Discrepancy Between
Empirical Curves of Distribution for Two Independent Samples, \textit{%
Bulletin Mathematique de l'Universite' de Moscou}, 2, fasc. 2.\smallskip

\noindent Sullivan R., A. Timmermann, and H. White, (1999), Data-Snooping,
Technical Trading Rule Performance, and the Bootstrap, \textit{The Journal
of Finance}, vol. 54, issue 5, 1647-1691.\smallskip

\noindent Sullivan, R., A. Timmermann, and H. White, (2001), Dangers of
data-driven inference: The case of calendar effects in stock returns, 
\textit{Journal of Econometrics}, 249-- 286.\smallskip

\noindent Thompson, S.B., (2008), Identifying Term Structure Volatility from
the LIBOR-swap Curve, \textit{Review of Financial Studies}, 21,
819-854.\smallskip

\noindent Vasicek, O. A., (1977), An Equilibrium Characterization of the
Term Structure, \textit{Journal of Financial Economics,} 5,
177-188.\smallskip

\noindent Vuong, Q., (1989), Likelihood Ratio Tests for Model Selection and
Non-Nested Hypotheses, \textit{Econometrica}, 57, 307-333

\noindent White, H., (1982), Maximum Likelihood Estimation of Misspecified
Models, \textit{Econometrica}, 50, 1-25\smallskip

\noindent White, H., (2000), A Reality Check for Data Snooping, \textit{%
Econometrica}, 68, 1097-1126.\smallskip

\noindent Wong, E., (1960), The Construction of a Class of Stationary Markov
Processes, in Sixtenn Symposia in Applied Mathematics: Stochastic Processes
in Mathematical Physics and Engineering, ed. by R. Bellman, American
Mathematical Society, Providence, R.I.\smallskip 

\baselineskip=20pt

\renewcommand{\baselinestretch}{1.0}

\end{document}
