%2multibyte Version: 5.50.0.2960 CodePage: 936

\documentclass[12pt]{article}
%%%%%%%%%%%%%%%%%%%%%%%%%%%%%%%%%%%%%%%%%%%%%%%%%%%%%%%%%%%%%%%%%%%%%%%%%%%%%%%%%%%%%%%%%%%%%%%%%%%%%%%%%%%%%%%%%%%%%%%%%%%%%%%%%%%%%%%%%%%%%%%%%%%%%%%%%%%%%%%%%%%%%%%%%%%%%%%%%%%%%%%%%%%%%%%%%%%%%%%%%%%%%%%%%%%%%%%%%%%%%%%%%%%%%%%%%%%%%%%%%%%%%%%%%%%%
\usepackage{amssymb}
\usepackage{amsfonts}
\usepackage{amsmath}
\usepackage{geometry}
\usepackage[onehalfspacing]{setspace}

\setcounter{MaxMatrixCols}{10}
%TCIDATA{OutputFilter=LATEX.DLL}
%TCIDATA{Version=5.50.0.2960}
%TCIDATA{Codepage=936}
%TCIDATA{<META NAME="SaveForMode" CONTENT="1">}
%TCIDATA{BibliographyScheme=Manual}
%TCIDATA{Created=Sunday, July 18, 2004 16:10:34}
%TCIDATA{LastRevised=Monday, April 11, 2022 10:17:17}
%TCIDATA{<META NAME="GraphicsSave" CONTENT="32">}
%TCIDATA{<META NAME="DocumentShell" CONTENT="General\Blank Document">}
%TCIDATA{CSTFile=article.cst}
%TCIDATA{PageSetup=72,72,72,72,0}
%TCIDATA{AllPages=
%F=36,\PARA{038<p type="texpara" tag="Body Text" >\hfill \thepage}
%}


\newtheorem{theorem}{Theorem}
\newtheorem{acknowledgement}[theorem]{Acknowledgement}
\newtheorem{algorithm}[theorem]{Algorithm}
\newtheorem{axiom}[theorem]{Axiom}
\newtheorem{case}[theorem]{Case}
\newtheorem{claim}[theorem]{Claim}
\newtheorem{conclusion}[theorem]{Conclusion}
\newtheorem{condition}[theorem]{Condition}
\newtheorem{conjecture}[theorem]{Conjecture}
\newtheorem{corollary}[theorem]{Corollary}
\newtheorem{criterion}[theorem]{Criterion}
\newtheorem{definition}[theorem]{Definition}
\newtheorem{example}[theorem]{Example}
\newtheorem{exercise}[theorem]{Exercise}
\newtheorem{lemma}[theorem]{Lemma}
\newtheorem{notation}[theorem]{Notation}
\newtheorem{problem}[theorem]{Problem}
\newtheorem{proposition}[theorem]{Proposition}
\newtheorem{remark}[theorem]{Remark}
\newtheorem{solution}[theorem]{Solution}
\newtheorem{summary}[theorem]{Summary}
\newenvironment{proof}[1][Proof]{\noindent\textbf{#1.} }{\ \rule{0.5em}{0.5em}}
\renewcommand{\baselinestretch}{1.3} 
\textwidth=6.8in
\textheight=8.7in
\oddsidemargin=0in
\evensidemargin=0in
\topmargin=0in
\baselineskip=10pt
\linespread{1.0}
\input{tcilatex}
\geometry{left=1in,right=1in,top=1.25in,bottom=1.25in}

\begin{document}

\title{Online Appendix to\\
\textquotedblleft Selecting the Relevant Variables for Factor Estimation in
a Factor-Augmented VAR Model"}
\author{John C. Chao\thanks{%
Department of Economics, 7343 Preinkert Drive, University of Maryland,
jcchao@umd.edu} and Norman R. Swanson\thanks{%
Department of Economics, 9500 Hamilton Street, Rutgers University,
nswanson@econ.rutgers.edu.}\thanks{%
The authors are grateful to Simon Freyaldenhoven, Yuan Liao, Minchul Shin,
Xiye Yang, and seminar participants at the Federal Reserve Bank of
Philadelphia for useful comments received on earlier versions of this paper.
Chao thanks the University of Maryland for reseach support.}}
\date{April 11, 2022}
\maketitle

\begin{abstract}
This Online Appendix is comprised of two sections. In the first section, we
give a theorem which shows that, even in the case where the conventional
factor pervasiveness assumption does not hold, consistent estimation of the
factors can be achieved if the variables used for factor estimation are
selected based on the variable selection procedure introduced in our main
paper. Section 2 of this Online Appendix then gives a proof of this theorem
as well as the proofs of some supporting lemmas.
\end{abstract}

\noindent \noindent \setcounter{page}{1}

\section{Consistent Estimation of Factors after Variable Pre-Screening}

\noindent \qquad In this section of the Online Appendix, we give a theorem
which shows that if the variable selection procedure introduced in our main
paper is used to pre-screen the variables prior to factor estimation; then,
consistent factor estimation, up to an invertible matrix transformation, can
be achieved. It should be noted that being able to estimate the factors
consistently up to an invertible matrix transformation is already sufficient
for us to be able to consistently estimate the conditional mean function of
a factor-augmented forecast equation. This has been shown in an earlier
version of our paper, Chao and Swanson (2022a); see Theorem 5 of that paper.
Hence, we do not make further identifying assumptions here to try to
estimate the factors consistently, say, up to a sign change. However,
consistent factor estimation up to a sign change could be achieved within
our framework, if we were to make additional identifying assumptions such as
those given in Stock and Watson (2002).

For the factor estimation problem, we need to impose some further
conditions, in addition to the assumptions given in the body of the main
paper. These additional assumptions are stated below.

\noindent \textbf{Assumption OA-1: }There exists a constant $\underline{C}>0$
such that $\inf_{t}\lambda _{\min }\left\{ E\left[ \varepsilon
_{t}\varepsilon _{t}^{\prime }\right] \right\} \geq \underline{C}>0$.

\noindent \textbf{Assumption OA-2: }There exists a positive constant $C$
such that $\sup_{t}\left( \frac{1}{N_{1}}\dsum\limits_{i\in H^{{\large c}%
}}\dsum\limits_{k\in H^{{\large c}}}\left\vert E\left[ u_{i,t}u_{k,t}\right]
\right\vert \right) $

\noindent $\leq C<\infty $ for every positive integer $N_{1}$, where $H^{%
{\large c}}=\left\{ k\in \left\{ 1,....,N\right\} :\gamma _{k}\neq 0\right\} 
$.

\noindent \textbf{Assumption OA-3: }There exists a positive constant $%
\overline{C},$ such that:%
\begin{equation*}
0<\frac{1}{\overline{C}}\leq \lambda _{\min }\left( \frac{\Gamma ^{\prime
}\Gamma }{N_{1}}\right) \leq \lambda _{\max }\left( \frac{\Gamma ^{\prime
}\Gamma }{N_{1}}\right) \leq \overline{C}<\infty \text{ for all }N_{1}\text{%
, }N_{2}\text{ sufficiently large,}
\end{equation*}%
where $N_{1}$ is the number of relevant variables or the number of
components of the subvector $Z_{t}^{\left( 1\right) }$ and $N_{2}$ is the
number of irrelvant variables or the number of components of the subvector $%
Z_{t}^{\left( 2\right) }$, with $Z_{t}^{\left( 1\right) }$ and $%
Z_{t}^{\left( 2\right) }$ as defined in expressions (9) and (10) of the main
paper.

\noindent \textbf{Assumption OA-4: }Let $\mathbb{S}_{i,T}^{+}$ denote either
the statistic $\max_{1\leq \ell \leq d}\left\vert S_{i,\ell ,T}\right\vert $
or the statistic $\dsum\nolimits_{\ell =1}^{d}\varpi _{\ell }\left\vert
S_{i,\ell ,T}\right\vert $, and let $\varphi $ be the tuning parameter of
the variable selection decision rule%
\begin{equation}
i\in \left\{ 
\begin{array}{cc}
\widehat{H}^{c} & \text{ if }\mathbb{S}_{i,T}^{+}\geq \Phi ^{-1}\left( 1-%
\frac{\varphi }{2N}\right) \\ 
\widehat{H} & \text{if }\mathbb{S}_{i,T}^{+}<\Phi ^{-1}\left( 1-\frac{%
\varphi }{2N}\right)%
\end{array}%
\right. ,  \label{var selection decision rule}
\end{equation}%
as described in section 2 of the main paper. In addition, let $N=N_{1}+N_{2}$%
, and assume that $\varphi $ satisfies the following three conditions: (a) $%
\varphi \rightarrow 0$ as $N_{1},N_{2}\rightarrow \infty $, (b)\ there
exists some constant $a>0,$ such that $\varphi \geq \frac{1}{N^{a}}$ for all 
$N_{1},N_{2}$ sufficiently large, and (c) 
\begin{equation*}
\max \left\{ \frac{N^{\frac{{\large 2}}{{\large 7}}}\varphi ^{\frac{{\large 5%
}}{{\large 7}}}}{N_{1}},\frac{N^{\frac{{\large 1}}{{\large 3}}}\varphi }{%
N_{1}T}\right\} \rightarrow 0\text{ as }N_{1},N_{2},T\rightarrow \infty .
\end{equation*}

\medskip

\noindent \textbf{Remark OA1.1: }

\noindent \textbf{(a) }Since the factor loading matrix $\Gamma $ is an $%
N\times Kp$ matrix, the matrix $\Gamma ^{\prime }\Gamma $ will have order of
magnitude equal to $N$ under the conventional assumption of factor
pervasiveness. Much of the factor analysis literature in both econometrics
and statistics has studied the case where factors are pervasive in this
sense. Assumption OA-3 above allows for possible violations of this
conventional pervasiveness assumption, which will occur in our setup when $%
N_{1}/N\rightarrow 0$.

\noindent \textbf{(b) }Note that\textbf{\ }the rate condition given in part
(c) of Assumption OA-4 depends on $N_{1}$. However, if we choose $\varphi $
so that $\varphi N^{\frac{{\large 2}}{{\large 5}}}=O\left( 1\right) $; then, 
\begin{equation*}
\frac{N^{\frac{{\large 2}}{{\large 7}}}\varphi ^{\frac{{\large 5}}{{\large 7}%
}}}{N_{1}}=O\left( \frac{1}{N_{1}}\right) =o\left( 1\right) \text{ and }%
\frac{N^{\frac{{\large 1}}{{\large 3}}}\varphi }{N_{1}T}=O\left( \frac{1}{%
N_{1}N^{\frac{{\large 1}}{{\large 15}}}T}\right) =o\left( \frac{1}{N_{1}}%
\right) \text{.}
\end{equation*}%
Hence, with this choice of $\varphi $, Assumption OA-4 part (c) will be
satisfied as long as $N_{1}\rightarrow \infty $, and there is no need to
impose any further condition on the rate at which $N_{1}$ grows. Requiring
that $N_{1}\rightarrow \infty $ is a minimal condition, since if $%
N_{1}\nrightarrow \infty $; then consistent factor estimation, even up to an
invertible matrix transformation, is impossible. Moreover, Monte Carlo
results reported in Section 3 of the main paper show that our variable
selection procedure performs very well in finite samples, under the tuning
parameter choice $\varphi =N^{-\frac{{\large 2}}{{\large 5}}}$, both in
terms of controlling the probability of a false positive (or Type I) error
and in terms of controlling the probability of a false negative (or Type II)
error.

\medskip

Next, consider the post-variable-selection principal component estimator

\noindent of $\underline{F}_{t}=\left( F_{t}^{\prime },F_{t-1}^{\prime
},...,F_{t-p{\LARGE +}1}^{\prime }\right) $ given by 
\begin{equation}
\widehat{\underline{F}}_{t}=\frac{\widehat{\Gamma }^{\prime }Z_{t,N}\left( 
\widehat{H^{c}}\right) }{\widehat{N}_{1}}\text{,}  \label{factor estimator}
\end{equation}%
where $Z_{t,N}\left( \widehat{H^{c}}\right) =\left[ 
\begin{array}{cccc}
Z_{1,t}\mathbb{I}\left\{ 1\in \widehat{H^{c}}\right\} , & Z_{2,t}\mathbb{I}%
\left\{ 2\in \widehat{H^{c}}\right\} , & \cdots , & Z_{N,t}\mathbb{I}\left\{
N\in \widehat{H^{c}}\right\}%
\end{array}%
\right] ^{\prime }$, with%
\begin{equation*}
\mathbb{I}\left\{ i\in \widehat{H^{c}}\right\} =\left\{ 
\begin{array}{cc}
1 & \text{if }i\in \widehat{H^{c}}\text{, i.e., if }\mathbb{S}_{i,T}^{+}\geq
\Phi ^{-1}\left( 1-\frac{\varphi }{2N}\right) \\ 
0 & \text{if }i\in \widehat{H}\text{, i.e., if }\mathbb{S}_{i,T}^{+}<\Phi
^{-1}\left( 1-\frac{\varphi }{2N}\right)%
\end{array}%
\right. ,
\end{equation*}%
and where $\widehat{N}_{1}=\#\left( \widehat{H^{c}}\right) $, i.e., the
cardinality of the set $\widehat{H^{c}}$. Here, $\widehat{\Gamma }$ denotes
the principal component estimator of the loading matrix $\Gamma $,
constructed from taking $\sqrt{\widehat{N}_{1}}$ times a matrix whose
columns are the eigenvectors associated with the $Kp$ largest eigenvalues of
the post-variable-selection sample covariance matrix $\widehat{\Sigma }%
\left( \widehat{H^{c}}\right) $, where, in this case,%
\begin{equation}
\widehat{\Sigma }\left( \widehat{H^{c}}\right) =\frac{Z\left( \widehat{H^{c}}%
\right) ^{\prime }Z\left( \widehat{H^{c}}\right) }{\widehat{N}_{1}T_{0}}=%
\frac{1}{\widehat{N}_{1}T_{0}}\dsum\limits_{t=p}^{T}Z_{t,N}\left( \widehat{%
H^{c}}\right) Z_{t,N}\left( \widehat{H^{c}}\right) ^{\prime },\text{ }
\label{post-selection sample cov matrix}
\end{equation}%
with $T_{0}=T-p+1$.

Our next result shows that the estimator given in expression (\ref{factor
estimator}) consistently estimates the unobserved factors $\underline{F}_{t}$%
, up to an invertible $Kp\times Kp$ matrix transformation.

\medskip

\noindent \textbf{Theorem 3: }\textit{Suppose that Assumptions 2-1, 2-2,
2-3, 2-4, 2-5, 2-6, 2-7, 2-8, and 2-9 of the main paper hold. Suppose, in
addition, that Assumptions OA-1, OA-2, OA-3, and OA-4 also hold. Let }$%
\widehat{\underline{F}}_{t}$\textit{\ be as defined in expression (\ref%
{factor estimator}). Then,}%
\begin{equation*}
\left\Vert \widehat{\underline{F}}_{t}-Q^{\prime }\underline{F}%
_{t}\right\Vert _{2}=o_{p}\left( 1\right) ,\text{ for all fixed }t\text{,}
\end{equation*}%
\textit{where }$Q=\left( \Gamma ^{\prime }\Gamma /N_{1}\right) ^{\frac{%
{\Large 1}}{{\Large 2}}}\Xi \widehat{V}$, \textit{and where }$\widehat{V}$%
\textit{\ is the }$Kp\times Kp$\textit{\ orthogonal matrix given in
expression (\ref{V eigenvector matrix}) below, and }$\Xi $\textit{\ is a }$%
Kp\times Kp$\textit{\ orthogonal matrix whose columns are the eigenvectors
of the matrix }%
\begin{equation*}
M_{FF}^{\ast }=\left( \frac{\Gamma ^{\prime }\Gamma }{N_{1}}\right)
^{1/2}M_{FF}\left( \frac{\Gamma ^{\prime }\Gamma }{N_{1}}\right)
^{1/2}=\left( \frac{\Gamma ^{\prime }\Gamma }{N_{1}}\right) ^{1/2}\frac{1}{%
T_{0}}\dsum\limits_{t=p}^{T}E\left[ \underline{F}_{t}\underline{F}%
_{t}^{\prime }\right] \left( \frac{\Gamma ^{\prime }\Gamma }{N_{1}}\right)
^{1/2}\text{.}
\end{equation*}

\noindent \textbf{Remark OA1.2:}\noindent \noindent

\noindent \textbf{(a)} If we examine the proof of Theorem 3 as well as the
supporting arguments given in the proof of Lemma OA-2 (both of which can be
found in the next section of this Online Appendix), we see that two of the
key components of the proof involve showing that $\left\Vert \left( \Gamma
\left( \widehat{H^{c}}\right) -\Gamma \right) /\sqrt{N_{1}}\right\Vert _{2}%
\overset{p}{\rightarrow }0$ and that $\left( \widehat{N}_{1}-N_{1}\right)
/N_{1}\overset{p}{\rightarrow }0$. This is one of the reasons why in Remark
2.3(b) of the main paper, we have argued that initial variable selection
should focus on determining which variables load strongly on the factors
without worrying specifically at that stage about the related issues of
predictability or, for that matter, any other issue. By contrast, if we make
our initial variable selection based on some more stringent criterion that
takes into consideration not only variable relevance but also other concerns
such as predictability, then, we may end up with a much smaller set $%
\widetilde{H}^{c}$ of selected variables relative to the set $\widehat{H^{c}}
$ selected under our procedure. In particular, in this case, it may be
possible that even in large samples a significant number of the rows of $%
\Gamma \left( \widetilde{H^{c}}\right) $ may contain only zero elements even
though the corresponding rows of $\Gamma $ are not zero vectors, so that the
desired result $\left\Vert \left( \Gamma \left( \widetilde{H^{c}}\right)
-\Gamma \right) /\sqrt{N_{1}}\right\Vert _{2}\overset{p}{\rightarrow }0$ may
not hold. For the same reason, if we let $\widetilde{N}_{1}$ denote the
cardinality of the set of selected indices based on an alternative, more
stringent variable selection procedure, then, the result $\left( \widetilde{N%
}_{1}-N_{1}\right) /N_{1}\overset{p}{\rightarrow }0$ also may not hold,
since, by definition, $N_{1}$ is the number of rows of $\Gamma $ which have
at least one non-zero element.

\noindent \textbf{(b)} Note that, with the proper specification of the
tuning parameter $\varphi $, the consistent factor estimation result given
in Theorem 3 can be attained without the condition $N/\left( TN_{1}\right)
\rightarrow 0$ given in Bai and Ng (2021). Theorem 3 does require that $%
N_{1}\rightarrow \infty $, which is a minimal condition as previously argued
in Remark OA1.1(b).

\section{Proofs of Theorem 3 and of Supporting Lemmas}

\noindent \qquad This section gives the proof of Theorem 3 followed by the
statement and proof of two supporting lemmas: Lemmas OA-1 and Lemma
OA-2.\newpage

\noindent \textbf{Proof of Theorem 3: }

To proceed, note first that the principal component estimator of $\underline{%
F}_{t}$ can be written as $\widehat{\underline{F}}_{t}=\widehat{\Gamma }%
^{\prime }Z_{t,N}\left( \widehat{H^{c}}\right) /\widehat{N}_{1}$, where $%
\widehat{\Gamma }=\sqrt{\widehat{N}_{1}}\widehat{B}$ and where the columns
of the matrix $\widehat{B}$ are the eigenvectors associated with the $Kp$
largest eigenvalues of the (post-variable-selection) sample covariance
matrix $\widehat{\Sigma }\left( \widehat{H^{c}}\right) =Z\left( \widehat{%
H^{c}}\right) ^{\prime }Z\left( \widehat{H^{c}}\right) /\left( \widehat{N}%
_{1}T_{0}\right) $. Moreover, by the result of part (d) of Lemma OA-1, the
matrix $\widehat{B}$ has the representation $\widehat{B}=\widehat{G}_{1}%
\widehat{V}$, where $\widehat{G}_{1}$ is an $N\times Kp$ matrix, whose
columns define an orthonormal basis for an invariant subspace of $\widehat{%
\Sigma }\left( \widehat{H^{c}}\right) $ and where $\widehat{V}$ is a $%
Kp\times Kp$ orthogonal matrix as defined in expression (\ref{V eigenvector
matrix}) in part (c) of Lemma OA-1. Making use of this representation, we
can further write%
\begin{eqnarray*}
\widehat{\underline{F}}_{t}-Q^{\prime }\underline{F}_{t} &=&\frac{\sqrt{%
\widehat{N}_{1}}\widehat{V}^{\prime }\widehat{G}_{1}^{{\Large \prime }%
}Z_{t,N}\left( \widehat{H^{c}}\right) }{\widehat{N}_{1}}-Q^{\prime }%
\underline{F}_{t} \\
&=&\frac{\widehat{V}^{\prime }\widehat{G}_{1}^{{\Large \prime }}\Gamma
\left( \widehat{H^{c}}\right) \underline{F}_{t}}{\sqrt{\widehat{N}_{1}}}+%
\frac{\widehat{V}^{\prime }\widehat{G}_{1}^{{\Large \prime }}U_{t,N}\left( 
\widehat{H^{c}}\right) }{\sqrt{\widehat{N}_{1}}}-Q^{\prime }\underline{F}_{t}
\\
&=&\left( \frac{\widehat{V}^{\prime }\widehat{G}_{1}^{{\Large \prime }%
}\Gamma \left( \widehat{H^{c}}\right) }{\sqrt{\widehat{N}_{1}}}-Q^{\prime
}\right) \underline{F}_{t}+\frac{\widehat{V}^{\prime }\widehat{G}_{1}^{%
{\Large \prime }}U_{t,N}\left( \widehat{H^{c}}\right) }{\sqrt{\widehat{N}_{1}%
}}
\end{eqnarray*}%
where $\Gamma \left( \widehat{H^{c}}\right) =\left( \mathbb{I}\left\{ 1\in 
\widehat{H^{c}}\right\} \gamma _{1},...,\mathbb{I}\left\{ N\in \widehat{H^{c}%
}\right\} \gamma _{N}\right) ^{\prime }$ and

\noindent $U_{t,N}\left( \widehat{H^{c}}\right) =\left( \mathbb{I}\left\{
1\in \widehat{H^{c}}\right\} u_{1,t},...,\mathbb{I}\left\{ N\in \widehat{%
H^{c}}\right\} u_{N,t}\right) ^{\prime }$. Next, write

\noindent $\left( \widehat{V}^{\prime }\widehat{G}_{1}^{{\Large \prime }%
}\Gamma /\sqrt{\widehat{N}_{1}}\right) -Q^{\prime }=\left[ \left( 1+\left[ 
\widehat{N}_{1}-N_{1}\right] /N_{1}\right) ^{-\frac{{\large 1}}{{\large 2}}%
}-1\right] \left( \widehat{V}^{\prime }\widehat{G}_{1}^{{\Large \prime }%
}\Gamma /\sqrt{N_{1}}\right) +\left( \widehat{V}^{\prime }\widehat{G}_{1}^{%
{\Large \prime }}\Gamma /\sqrt{N_{1}}\right) -Q^{\prime }$

\noindent and $\left( \Gamma \left( \widehat{H^{c}}\right) -\Gamma \right) /%
\sqrt{\widehat{N}_{1}}=\left( 1+\left[ \widehat{N}_{1}-N_{1}\right]
/N_{1}\right) ^{-\frac{{\large 1}}{{\large 2}}}\left( \left[ \Gamma \left( 
\widehat{H^{c}}\right) -\Gamma \right] /\sqrt{N_{1}}\right) $, so that%
\begin{eqnarray*}
\frac{\widehat{V}^{\prime }\widehat{G}_{1}^{{\Large \prime }}\Gamma \left( 
\widehat{H^{c}}\right) \underline{F}_{t}}{\sqrt{\widehat{N}_{1}}}
&=&Q^{\prime }\underline{F}_{t}+\left( \frac{\widehat{V}^{\prime }\widehat{G}%
_{1}^{{\Large \prime }}\Gamma }{\sqrt{\widehat{N}_{1}}}-Q^{\prime }\right) 
\underline{F}_{t}+\widehat{V}^{\prime }\widehat{G}_{1}^{{\Large \prime }%
}\left( \frac{\Gamma \left( \widehat{H^{c}}\right) -\Gamma }{\sqrt{\widehat{N%
}_{1}}}\right) \underline{F}_{t} \\
&=&Q^{\prime }\underline{F}_{t}+\left( \frac{\widehat{V}^{\prime }\widehat{G}%
_{1}^{{\Large \prime }}\Gamma }{\sqrt{N_{1}}}-Q^{\prime }\right) \underline{F%
}_{t}+\left[ \left( 1+\frac{\widehat{N}_{1}-N_{1}}{N_{1}}\right) ^{-\frac{%
{\large 1}}{{\large 2}}}-1\right] \frac{\widehat{V}^{\prime }\widehat{G}%
_{1}^{{\Large \prime }}\Gamma }{\sqrt{N_{1}}}\underline{F}_{t} \\
&&+\left[ \left( 1+\frac{\widehat{N}_{1}-N_{1}}{N_{1}}\right) ^{-\frac{%
{\large 1}}{{\large 2}}}\right] \widehat{V}^{\prime }\widehat{G}_{1}^{%
{\Large \prime }}\left( \frac{\Gamma \left( \widehat{H^{c}}\right) -\Gamma }{%
\sqrt{N}_{1}}\right) \underline{F}_{t}
\end{eqnarray*}%
It follows that%
\begin{eqnarray*}
\widehat{\underline{F}}_{t}-Q^{\prime }\underline{F}_{t} &=&\left( \frac{%
\widehat{V}^{\prime }\widehat{G}_{1}^{{\Large \prime }}\Gamma \left( 
\widehat{H^{c}}\right) }{\sqrt{\widehat{N}_{1}}}-Q^{\prime }\right) 
\underline{F}_{t}+\frac{\widehat{V}^{\prime }\widehat{G}_{1}^{{\Large \prime 
}}U_{t,N}\left( \widehat{H^{c}}\right) }{\sqrt{\widehat{N}_{1}}} \\
&=&\left( \frac{\widehat{V}^{\prime }\widehat{G}_{1}^{{\Large \prime }%
}\Gamma }{\sqrt{N_{1}}}-Q^{\prime }\right) \underline{F}_{t}+\left[ \left( 1+%
\frac{\widehat{N}_{1}-N_{1}}{N_{1}}\right) ^{-\frac{{\large 1}}{{\large 2}}%
}-1\right] \frac{\widehat{V}^{\prime }\widehat{G}_{1}^{{\Large \prime }%
}\Gamma }{\sqrt{N_{1}}}\underline{F}_{t} \\
&&+\left[ \left( 1+\frac{\widehat{N}_{1}-N_{1}}{N_{1}}\right) ^{-\frac{%
{\large 1}}{{\large 2}}}\right] \widehat{V}^{\prime }\widehat{G}_{1}^{%
{\Large \prime }}\left( \frac{\Gamma \left( \widehat{H^{c}}\right) -\Gamma }{%
\sqrt{N}_{1}}\right) \underline{F}_{t}+\frac{\widehat{V}^{\prime }\widehat{G}%
_{1}^{{\Large \prime }}U_{t,N}\left( \widehat{H^{c}}\right) }{\sqrt{\widehat{%
N}_{1}}}
\end{eqnarray*}%
Hence, applying the triangle inequality as well as parts (a)-(c), (g), and
(h) of Lemma OA-2 along with the Slutsky's theorem, we obtain 
\begin{eqnarray*}
&&\left\Vert \widehat{\underline{F}}_{t}-Q^{\prime }\underline{F}%
_{t}\right\Vert _{2} \\
&\leq &\left\Vert \frac{\widehat{V}^{\prime }\widehat{G}_{1}^{{\Large \prime 
}}\Gamma }{\sqrt{N_{1}}}-Q^{\prime }\right\Vert _{2}\left\Vert \underline{F}%
_{t}\right\Vert _{2}+\left\vert \left( 1+\frac{\widehat{N}_{1}-N_{1}}{N_{1}}%
\right) ^{-\frac{{\large 1}}{{\large 2}}}-1\right\vert \left\Vert \frac{%
\widehat{V}^{\prime }\widehat{G}_{1}^{{\Large \prime }}\Gamma }{\sqrt{N_{1}}}%
\right\Vert _{2}\left\Vert \underline{F}_{t}\right\Vert _{2} \\
&&+\left\vert \left( 1+\frac{\widehat{N}_{1}-N_{1}}{N_{1}}\right) ^{-\frac{%
{\large 1}}{{\large 2}}}\right\vert \left\Vert \widehat{V}^{\prime }\widehat{%
G}_{1}^{{\Large \prime }}\right\Vert _{2}\left\Vert \frac{\Gamma \left( 
\widehat{H^{c}}\right) -\Gamma }{\sqrt{N}_{1}}\right\Vert _{2}\left\Vert 
\underline{F}_{t}\right\Vert _{2}+\left\Vert \frac{\widehat{V}^{\prime }%
\widehat{G}_{1}^{{\Large \prime }}U_{t,N}\left( \widehat{H^{c}}\right) }{%
\sqrt{\widehat{N}_{1}}}\right\Vert _{2} \\
&=&\left\Vert \frac{\widehat{V}^{\prime }\widehat{G}_{1}^{{\Large \prime }%
}\Gamma }{\sqrt{N_{1}}}-Q^{\prime }\right\Vert _{2}\left\Vert \underline{F}%
_{t}\right\Vert _{2}+\left\vert \left( 1+\frac{\widehat{N}_{1}-N_{1}}{N_{1}}%
\right) ^{-\frac{{\large 1}}{{\large 2}}}-1\right\vert \left\Vert \frac{%
\widehat{V}^{\prime }\widehat{G}_{1}^{{\Large \prime }}\Gamma }{\sqrt{N_{1}}}%
\right\Vert _{2}\left\Vert \underline{F}_{t}\right\Vert _{2} \\
&&+\left\vert \left( 1+\frac{\widehat{N}_{1}-N_{1}}{N_{1}}\right) ^{-\frac{%
{\large 1}}{{\large 2}}}\right\vert \left\Vert \frac{\Gamma \left( \widehat{%
H^{c}}\right) -\Gamma }{\sqrt{N}_{1}}\right\Vert _{2}\left\Vert \underline{F}%
_{t}\right\Vert _{2}+\left\Vert \frac{\widehat{V}^{\prime }\widehat{G}_{1}^{%
{\Large \prime }}U_{t,N}\left( \widehat{H^{c}}\right) }{\sqrt{\widehat{N}_{1}%
}}\right\Vert _{2} \\
&&\left( \text{since }\left\Vert \widehat{V}^{\prime }\widehat{G}_{1}^{%
{\Large \prime }}\right\Vert _{2}=\sqrt{\lambda _{\max }\left( \widehat{G}%
_{1}\widehat{V}\widehat{V}^{\prime }\widehat{G}_{1}^{{\Large \prime }%
}\right) }=\sqrt{\lambda _{\max }\left( \widehat{V}^{\prime }\widehat{G}%
_{1}^{{\Large \prime }}\widehat{G}_{1}\widehat{V}\right) }=\sqrt{\lambda
_{\max }\left( I_{Kp}\right) }=1\right) \\
&=&o_{p}\left( 1\right) O_{p}\left( 1\right) +o_{p}\left( 1\right)
O_{p}\left( 1\right) O_{p}\left( 1\right) +O_{p}\left( 1\right) o_{p}\left(
1\right) O_{p}\left( 1\right) +o_{p}\left( 1\right) \\
&=&o_{p}\left( 1\right) \text{. }\square
\end{eqnarray*}

\medskip

\noindent \textbf{Lemma OA-1: }Let $\widehat{\Sigma }\left( \widehat{H^{c}}%
\right) $ be the post-variable-selection sample covariance matrix as defined
in expression (\ref{post-selection sample cov matrix}), and let $T_{0}=T-p+1$%
, as before. Decompose $\widehat{\Sigma }\left( \widehat{H^{c}}\right) $ as $%
\widehat{\Sigma }\left( \widehat{H^{c}}\right) =A+E$, where $A=\Gamma
M_{FF}\Gamma ^{\prime }/N_{1}$ and where 
\begin{eqnarray}
E &=&\widehat{\Sigma }\left( \widehat{H^{c}}\right) -\frac{\Gamma
M_{FF}\Gamma ^{\prime }}{N_{1}}  \notag \\
&=&\left( \Gamma \left( \widehat{H^{c}}\right) M_{FF}\Gamma \left( \widehat{%
H^{c}}\right) ^{\prime }/\widehat{N}_{1}-\Gamma M_{FF}\Gamma ^{\prime
}/N_{1}\right) +\frac{1}{\widehat{N}_{1}}\Gamma \left( \widehat{H^{c}}%
\right) \left[ \frac{\underline{F}^{\prime }\underline{F}}{T_{0}}-M_{FF}%
\right] \Gamma \left( \widehat{H^{c}}\right) ^{\prime }  \notag \\
&&+\frac{U\left( \widehat{H^{c}}\right) ^{\prime }\underline{F}\Gamma \left( 
\widehat{H^{c}}\right) ^{\prime }}{\widehat{N}_{1}T_{0}}+\frac{\Gamma \left( 
\widehat{H^{c}}\right) \underline{F}^{\prime }U\left( \widehat{H^{c}}\right) 
}{\widehat{N}_{1}T_{0}}+\frac{U\left( \widehat{H^{c}}\right) ^{\prime
}U\left( \widehat{H^{c}}\right) }{\widehat{N}_{1}T_{0}},  \label{E matrix}
\end{eqnarray}%
with $U\left( \widehat{H^{c}}\right) =$ $\left( \mathbb{I}\left\{ 1\in 
\widehat{H^{c}}\right\} u_{1\cdot },...,\mathbb{I}\left\{ N\in \widehat{H^{c}%
}\right\} u_{N\cdot }\right) $ and $M_{FF}=T_{0}^{-1}\dsum%
\nolimits_{t=p}^{T}E\left[ \underline{F}_{t}\underline{F}_{t}^{\prime }%
\right] $. Let Assumptions 2-1, 2-2, 2-3, 2-4 2-5, 2-6, 2-7, 2-9, OA-1,
OA-2, OA-3, and OA-4 all hold; and define $\underset{N\times N}{G}=\left[ 
\begin{array}{cc}
\underset{N\times Kp}{G_{1}} & \underset{N\times \left( N-Kp\right) }{G_{2}}%
\end{array}%
\right] $ to be an orthogonal matrix whose columns are the eigenvectors of
the matrix $A$. Without loss of generality, suppose also that the columns of 
$G_{1}$ are the eigenvectors associated with the non-zero eigenvalues of $A$%
, whereas $G_{2}$ contains the eigenvectors associated with the zero
eigenvalue which has an algebraic multiplicity of $N-Kp$ in this case%
\footnote{%
An explicit proof that $0$ is an eigenvalue of the matrix $A=\Gamma
M_{FF}\Gamma ^{\prime }/N_{1}$ with algebraic multiplicity equaling $N-Kp$
is given in the Technical Appendix of an earlier version of this paper, Chao
and Swanson (2022b) which is available at http://econweb.umd.edu/\symbol{126}%
chao/Research/research\_files/AppConFacVarSel-03-18-2022.pdf. In particular,
this result is shown in part (a) of Lemma D-11 in Appendix D of Chao and
Swanson (2022b).}. Partition the matrices $G^{\prime }AG$ and $G^{\prime }EG$
as follows:%
\begin{eqnarray*}
G^{\prime }AG &=&\left( 
\begin{array}{cc}
\underset{Kp\times Kp}{\Lambda _{1}} & \underset{Kp\times \left( N-Kp\right) 
}{0} \\ 
\underset{\left( N-Kp\right) \times Kp}{0} & \underset{\left( N-Kp\right)
\times \left( N-Kp\right) }{\Lambda _{2}}%
\end{array}%
\right) =\left( 
\begin{array}{cc}
\underset{Kp\times Kp}{\Lambda _{1}} & \underset{Kp\times \left( N-Kp\right) 
}{0} \\ 
\underset{\left( N-Kp\right) \times Kp}{0} & \underset{\left( N-Kp\right)
\times \left( N-Kp\right) }{0}%
\end{array}%
\right) \text{ and} \\
G^{\prime }EG &=&\left( 
\begin{array}{cc}
\underset{Kp\times Kp}{E_{11}} & \underset{Kp\times \left( N-Kp\right) }{%
E_{21}^{\prime }} \\ 
\underset{\left( N-Kp\right) \times Kp}{E_{21}} & \underset{\left(
N-Kp\right) \times \left( N-Kp\right) }{E_{22}}%
\end{array}%
\right) \text{,}
\end{eqnarray*}%
where $\Lambda _{1}$ is a diagonal matrix whose diagonal elements are the $%
Kp $ largest eigevalues of the matrix $A$.\footnote{%
An explicit proof showing that $G^{\prime }AG$ can be partitioned in the
manner given here is also provided in the proof of Lemma D-11 of Chao and
Swanson (2022b).}\qquad\ 

Under the assumed conditions, the following statements are true.

\begin{enumerate}
\item[(a)] There exists a $\left( N-Kp\right) \times Kp$ matrix $R$ such
that the columns of the matrix $\widehat{G}_{1}=\left( G_{1}+G_{2}R\right)
\left( I_{Kp}+R^{\prime }R\right) ^{-1/2}$ define an orthonormal basis for a
subspace that is

invariant for $\widehat{\Sigma }\left( \widehat{H^{c}}\right) =A+E$ w.p.a.1.
Moreover, $\left\Vert R\right\Vert _{2}=o_{p}\left( 1\right) $ as $%
N_{1},N_{2},$ and $T\rightarrow \infty $.

\item[(b)] $\left\Vert \widehat{G}_{1}-G_{1}\right\Vert _{2}=o_{p}\left(
1\right) $ as $N_{1},$ $N_{2}$, and $T\rightarrow \infty $.

\item[(c)] The exists a unique symmetric matrix $L$ such that $\left(
A+E\right) \widehat{G}_{1}=\widehat{G}_{1}L$. Moreover, let 
\begin{equation}
\widehat{\Lambda }=diag\left( \widehat{\lambda }_{1},...,\widehat{\lambda }%
_{Kp}\right)  \label{LambdaL}
\end{equation}%
denote a diagonal matrix whose diagonal elements are the eigenvalues of the
matrix $L$, and let%
\begin{equation}
\widehat{V}=\left( 
\begin{array}{cccc}
\widehat{v}_{1} & \widehat{v}_{2} & \cdots & \widehat{v}_{Kp}%
\end{array}%
\right)  \label{V eigenvector matrix}
\end{equation}%
be a $Kp\times Kp$ matrix whose $\ell ^{th}$ column (i.e., $\widehat{v}%
_{\ell }$) is an eigenvector of $L$ associated with the eigenvalue $\widehat{%
\lambda }_{\ell }$ for $\ell =1,...,Kp$. Then, $\widehat{V}$ is an
orthogonal matrix and $\left( \widehat{G}_{1}\widehat{v}_{\ell },\widehat{%
\lambda }_{\ell }\right) $ is an eigenpair for the matrix $A+E$ for $\ell
=1,...,Kp$.

\item[(d)] The columns of the matrix $\widehat{G}_{1}\widehat{V}=\left( 
\begin{array}{cccc}
\widehat{G}_{1}\widehat{v}_{1} & \widehat{G}_{1}\widehat{v}_{2} & \cdots & 
\widehat{G}_{1}\widehat{v}_{Kp}%
\end{array}%
\right) $ are the eigenvectors associated with the $Kp$ largest eigenvalues
of the post-variable-selection sample covariance matrix $A+E=\widehat{\Sigma 
}\left( \widehat{H^{c}}\right) $.
\end{enumerate}

\medskip

\noindent \textbf{Proof of Lemm OA-1: }

To show part (a), we first verify that two key conditions of Theorem 8.1.10
of Golub and van Loan (1996), i.e.,%
\begin{equation}
\text{sep}\left( \Lambda _{1},\Lambda _{2}\right) =\min_{X{\Large \neq }0}%
\frac{\left\Vert \Lambda _{1}X-X\Lambda _{2}\right\Vert _{F}}{\left\Vert
X\right\Vert _{F}}>0  \label{sep bd from zero}
\end{equation}%
and 
\begin{equation}
\left\Vert E\right\Vert _{2}\leq \frac{\text{sep}\left( \Lambda _{1},\Lambda
_{2}\right) }{5}=\frac{1}{5}\min_{X{\Large \neq }0}\frac{\left\Vert \Lambda
_{1}X-X\Lambda _{2}\right\Vert _{F}}{\left\Vert X\right\Vert _{F}}\text{,}
\label{upper bd E}
\end{equation}%
are satisfied here\footnote{%
It should be noted that Golub and van Loan (1996) use slightly different
notations than we do here. In particular, the two conditions given in
expressions (\ref{sep bd from zero}) and (\ref{upper bd E}), if stated in
their notations, would be sep$\left( D_{1},D_{2}\right) >0$ and $\left\Vert
E\right\Vert _{2}\leq \frac{sep\left( D_{1},D_{2}\right) }{5}$.}. To
proceed, let ran$\left( G_{1}\right) $ denote the range space of $G_{1}$,
i.e., ran$\left( G_{1}\right) =\left\{ g\in \mathbb{R}^{N}:g=G_{1}b\text{
for some }b\in \mathbb{R}^{Kp}\right\} $; and, by definition, $\Lambda _{1}$
is a $Kp\times Kp$ diagonal matrix whose diagonal elements are the non-zero
eigenvalues of the matrix $A=\Gamma M_{FF}\Gamma ^{\prime }/N_{1}$. Now, for
any $\widetilde{g}\in $ ran$\left( G_{1}\right) $, note that there exists $%
b\in \mathbb{R}^{Kp}$ such that $g^{\ast }=A\widetilde{g}=\left( \Gamma
M_{FF}\Gamma ^{\prime }/N_{1}\right) G_{1}b=G_{1}\Lambda _{1}b=G_{1}b^{\ast
} $, where $b^{\ast }=\Lambda _{1}b$, from which it follows that $g^{\ast
}\in $ ran$\left( G_{1}\right) $, so that ran$\left( G_{1}\right) $ is an
invariant subspace of $A$. Next, given Assumptions 2-1, 2-2(a)-(b), 2-5,
2-6, OA-1, and OA-3, it is straightforward to show that there exists a
positive constant $\underline{c}$ such that 
\begin{equation}
\text{sep}\left( \Lambda _{1},\Lambda _{2}\right) =\text{sep}\left( \Lambda
_{1},0\right) =\min_{X{\Large \neq }0}\frac{\left\Vert \Lambda
_{1}X\right\Vert _{F}}{\left\Vert X\right\Vert _{F}}\geq \lambda _{\min
}\left( \Lambda _{1}\right) =\lambda _{\min }\left( \frac{M_{FF}^{1/2}\Gamma
^{\prime }\Gamma M_{FF}^{1/2}}{N_{1}}\right) \geq \underline{c}>0\text{ }
\label{lower bd for sep}
\end{equation}%
for all $N_{1}$ and $N_{2}$ sufficiently large\footnote{%
A more detailed proof of the result that sep$\left( \Lambda _{1},\Lambda
_{2}\right) \geq \underline{c}>0$ can be found in the Technical Appendix of
an earlier version of this paper, Chao and Swanson (2022b). In particular,
this result is shown in part (c) of Lemma D-11 in Appendix D of Chao and
Swanson (2022b).}, so that the condition given in expression (\ref{sep bd
from zero}) is fulfilled. In addition, by the result given in Lemma D-10 of
Chao and Swanson (2022b), we have $\left\Vert E\right\Vert _{2}=\left\Vert 
\widehat{\Sigma }\left( \widehat{H^{c}}\right) -\left( \Gamma M_{FF}\Gamma
^{\prime }/N_{1}\right) \right\Vert _{2}=o_{p}\left( 1\right) $ \ as $N_{1}$%
, $N_{2}$, and $T\rightarrow \infty $; from which it follows that $%
\left\Vert E\right\Vert _{2}\leq sep\left( \Lambda _{1},0\right) /5$ w.p.a.1
as $N_{1}$, $N_{2}$, and $T\rightarrow \infty $, so that the condition given
in expression (\ref{upper bd E}) is also satisfied w.p.a.1. Hence,
application of Theorem 8.1.10 of Golub and van Loan (1996) allows us to
conclude that there exists a $\left( N-Kp\right) \times Kp$ matrix $R$ such
that the columns of the matrix $\widehat{G}_{1}=\left( G_{1}+G_{2}R\right)
\left( I_{Kp}+R^{\prime }R\right) ^{-1/2}$ define an orthonormal basis for a
subspace that is invariant for $A+E$ w.p.a.1. In addition,%
\begin{eqnarray*}
\left\Vert R\right\Vert _{2} &\leq &\frac{4}{\text{sep}\left( \Lambda
_{1},0\right) }\left\Vert E\right\Vert _{2}\leq 4\left[ \lambda _{\min
}\left( \frac{M_{FF}^{1/2}\Gamma ^{\prime }\Gamma M_{FF}^{1/2}}{N_{1}}%
\right) \right] ^{-1}\left\Vert E\right\Vert _{2} \\
&\leq &\frac{4}{\underline{c}}\left\Vert E\right\Vert _{2}\text{ \ }\left( 
\text{for some }\underline{c}>0\text{ given expression (\ref{lower bd for
sep})}\right) \\
&=&o_{p}\left( 1\right) \text{,}
\end{eqnarray*}%
which shows result (a).

To show that $\left\Vert \widehat{G}_{1}-G_{1}\right\Vert _{2}=o_{p}\left(
1\right) $, we first show that an explicit representation for $G_{1}$ can be
given as $G_{1}=\left( \Gamma /\sqrt{N_{1}}\right) \left( \Gamma ^{\prime
}\Gamma /N_{1}\right) ^{-1/2}\Xi =\Gamma \left( \Gamma ^{\prime }\Gamma
\right) ^{-1/2}\Xi $ , where $\Xi $ is an orthogonal matrix whose columns
are eigenvectors of the matrix $M_{FF}^{\ast }=\left( \Gamma ^{\prime
}\Gamma /N_{1}\right) ^{1/2}M_{FF}\left( \Gamma ^{\prime }\Gamma
/N_{1}\right) ^{1/2}$. To see that this representation satisfies the various
properties we require of $G_{1}$, note first that $G_{1}^{\prime }G_{1}=\Xi
^{\prime }\left( \Gamma ^{\prime }\Gamma /N_{1}\right) ^{-1/2}\left( \Gamma
^{\prime }\Gamma /N_{1}\right) \left( \Gamma ^{\prime }\Gamma /N_{1}\right)
^{-1/2}\Xi $ $=I_{Kp}$; hence, $G_{1}$ so represented does have orthonormal
columns. Moreover, note that%
\begin{eqnarray}
\frac{\Gamma M_{FF}\Gamma ^{\prime }}{N_{1}}G_{1} &=&\frac{\Gamma }{\sqrt{%
N_{1}}}M_{FF}\frac{\Gamma ^{\prime }\Gamma }{N_{1}}\left( \frac{\Gamma
^{\prime }\Gamma }{N_{1}}\right) ^{-1/2}\Xi =\frac{\Gamma }{\sqrt{N_{1}}}%
\left( \frac{\Gamma ^{\prime }\Gamma }{N_{1}}\right) ^{-1/2}M_{FF}^{\ast
}\Xi \text{ \ }=\Gamma \left( \Gamma ^{\prime }\Gamma \right) ^{-1/2}\Xi
\Lambda _{1}  \notag \\
&=&G_{1}\Lambda _{1}  \label{eigen eqn for A}
\end{eqnarray}%
where $\Lambda _{1}$ is a $Kp\times Kp$ diagonal matrix whose diagonal
elements are the eigenvalues of the matrix $M_{FF}^{\ast }$, which also
happen to be the non-zero eigenvalues of the matrix $A=\Gamma M_{FF}\Gamma
^{\prime }/N_{1}$. Pre-multiplying the above equation by $G_{1}^{\prime }$,
we obtain $G_{1}^{\prime }\left( \Gamma M_{FF}\Gamma ^{\prime }/N_{1}\right)
G_{1}=G_{1}^{\prime }G_{1}\Lambda _{1}=\Lambda _{1}$. Since equation (\ref%
{eigen eqn for A}) shows that the columns of $\Gamma \left( \Gamma ^{\prime
}\Gamma \right) ^{-1/2}\Xi $ are indeed the eigenvectors of the matrix $%
A=\Gamma M_{FF}\Gamma ^{\prime }/N_{1}$, by the argument given previously in
the proof of part (a) above, we can then deduce that ran$\left( G_{1}\right) 
$, the range space of $G_{1}$ with $G_{1}=\Gamma \left( \Gamma ^{\prime
}\Gamma \right) ^{-1/2}\Xi $, is an invariant subspace of $A$. It follows
that setting $G_{1}=\Gamma \left( \Gamma ^{\prime }\Gamma \right) ^{-1/2}\Xi 
$ \ fulfills all the required properties which we have previously specified
for $G_{1}$.

Next, write%
\begin{eqnarray*}
\widehat{G}_{1}-G_{1} &=&\left( G_{1}+G_{2}R\right) \left( I_{Kp}+R^{\prime
}R\right) ^{-1/2}-G_{1} \\
&=&G_{1}\left[ \left( I_{Kp}+R^{\prime }R\right) ^{-1/2}-I_{Kp}\right]
+G_{2}R\left( I_{Kp}+R^{\prime }R\right) ^{-1/2}
\end{eqnarray*}%
\begin{equation*}
=\frac{\Gamma }{\sqrt{N_{1}}}\left( \frac{\Gamma ^{\prime }\Gamma }{N_{1}}%
\right) ^{-1/2}\Xi \left[ \left( I_{Kp}+R^{\prime }R\right) ^{-1/2}-I_{Kp}%
\right] +G_{2}R\left( I_{Kp}+R^{\prime }R\right) ^{-1/2}
\end{equation*}%
Applying the submultiplicative property of matrix norms and the triangle
inequality, we obtain%
\begin{eqnarray*}
\left\Vert \widehat{G}_{1}-G_{1}\right\Vert _{2} &\leq &\left\Vert \frac{%
\Gamma }{\sqrt{N_{1}}}\left( \frac{\Gamma ^{\prime }\Gamma }{N_{1}}\right)
^{-1/2}\right\Vert _{2}\left\Vert \Xi \right\Vert _{2}\left\Vert \left(
I_{Kp}+R^{\prime }R\right) ^{-1/2}-I_{Kp}\right\Vert _{2} \\
&&+\left\Vert G_{2}\right\Vert _{2}\left\Vert R\right\Vert _{2}\left\Vert
\left( I_{Kp}+R^{\prime }R\right) ^{-1/2}\right\Vert _{2} \\
&=&\left\Vert I_{Kp}-\left( I_{Kp}+R^{\prime }R\right) ^{-1/2}\right\Vert
_{2}+\left\Vert R\right\Vert _{2}\left\Vert \left( I_{Kp}+R^{\prime
}R\right) ^{-1/2}\right\Vert _{2}
\end{eqnarray*}%
where the last equality follows from the fact that $\left\Vert \Xi
\right\Vert _{2}=\sqrt{\lambda _{\max }\left( \Xi ^{\prime }\Xi \right) }=%
\sqrt{\lambda _{\max }\left( I_{Kp}\right) }=1$, $\left\Vert
G_{2}\right\Vert _{2}=\sqrt{\lambda _{\max }\left( G_{2}^{\prime
}G_{2}\right) }=\sqrt{\lambda _{\max }\left( I_{N-Kp}\right) }=1$,

\noindent $\left\Vert \left( I_{Kp}+R^{\prime }R\right)
^{-1/2}-I_{Kp}\right\Vert _{2}=\left\Vert I_{Kp}-\left( I_{Kp}+R^{\prime
}R\right) ^{-1/2}\right\Vert _{2}$, and 
\begin{equation*}
\text{ }\left\Vert \frac{\Gamma }{\sqrt{N_{1}}}\left( \frac{\Gamma ^{\prime
}\Gamma }{N_{1}}\right) ^{-1/2}\right\Vert _{2}=\sqrt{\lambda _{\max
}\left\{ \left( \frac{\Gamma ^{\prime }\Gamma }{N_{1}}\right) ^{-1/2}\frac{%
\Gamma ^{\prime }\Gamma }{N_{1}}\left( \frac{\Gamma ^{\prime }\Gamma }{N_{1}}%
\right) ^{-1/2}\right\} }=\sqrt{\lambda _{\max }\left\{ I_{Kp}\right\} }=1%
\text{.}
\end{equation*}%
Now, , if $\left( \lambda ,\rho \right) $ is an eigen-pair of $R^{\prime }R$%
; then, by definition, $R^{\prime }R\rho =\lambda \rho $ with $\lambda \geq
0 $, since $R^{\prime }R$ is a positive semidefinite matrix. It follows by
elementary properties of eigenvalues and eigenvectors that, in this case, $%
\left( \sqrt{1+\lambda }-1\right) /\sqrt[\backslash ]{1+\lambda }$ will be
an eigenvalue of the matrix $I_{Kp}-\left( I_{Kp}+R^{\prime }R\right)
^{-1/2} $ associated with the eigenvector $\rho $, so that 
\begin{equation*}
\left[ I_{Kp}-\left( I_{Kp}+R^{\prime }R\right) ^{-1/2}\right] \rho =\frac{%
\sqrt{1+\lambda }-1}{\sqrt[\backslash ]{1+\lambda }}\rho
\end{equation*}%
Next, let $g\left( \lambda \right) =\left( \sqrt{1+\lambda }-1\right) /\sqrt[%
\backslash ]{1+\lambda }$; and, by taking derivative of $g\left( \lambda
\right) $ with respect to $\lambda $, we obtain%
\begin{equation*}
g^{\prime }\left( \lambda \right) =\frac{1}{2}\frac{1}{1+\lambda }-\frac{1}{2%
}\frac{\sqrt{1+\lambda }-1}{\left( 1+\lambda \right) ^{3/2}}=\frac{1}{%
2\left( 1+\lambda \right) ^{3/2}}>0\text{ for all }\lambda \geq 0\text{,}
\end{equation*}%
so that, in particular, $g\left( \lambda \right) $ is an increasing function
of $\lambda $ for $\lambda \geq 0$. Making use of these results, we see that%
\begin{eqnarray*}
&&\left\Vert \widehat{G}_{1}-G_{1}\right\Vert _{2} \\
&\leq &\left\Vert I_{Kp}-\left( I_{Kp}+R^{\prime }R\right)
^{-1/2}\right\Vert _{2}+\left\Vert R\right\Vert _{2}\left\Vert \left(
I_{Kp}+R^{\prime }R\right) ^{-1/2}\right\Vert _{2}
\end{eqnarray*}%
\begin{eqnarray*}
&=&\sqrt{\lambda _{\max }\left( \left[ I_{Kp}-\left( I_{Kp}+R^{\prime
}R\right) ^{-1/2}\right] ^{\prime }\left[ I_{Kp}-\left( I_{Kp}+R^{\prime
}R\right) ^{-1/2}\right] \right) } \\
&&+\left\Vert R\right\Vert _{2}\sqrt{\lambda _{\max }\left( \left(
I_{Kp}+R^{\prime }R\right) ^{-1/2\prime }\left( I_{Kp}+R^{\prime }R\right)
^{-1/2}\right) } \\
&=&\lambda _{\max }\left[ I_{Kp}-\left( I_{Kp}+R^{\prime }R\right) ^{-1/2}%
\right] +\left\Vert R\right\Vert _{2}\lambda _{\max }\left[ \left(
I_{Kp}+R^{\prime }R\right) ^{-1/2}\right] \\
&\leq &\frac{\sqrt{1+\lambda _{\max }\left( R^{\prime }R\right) }-1}{\sqrt[%
\backslash ]{1+\lambda _{\min }\left( R^{\prime }R\right) }}+\frac{%
\left\Vert R\right\Vert _{2}}{\sqrt[\backslash ]{1+\lambda _{\min }\left(
R^{\prime }R\right) }} \\
&\leq &\sqrt{1+\left\Vert R\right\Vert _{2}^{2}}-1+\left\Vert R\right\Vert
_{2}\text{ \ } \\
&=&o_{p}\left( 1\right) \text{ as }N_{1},\text{ }N_{2}\text{, and }%
T\rightarrow \infty \text{ }\left( \text{since }\left\Vert R\right\Vert
_{2}=o_{p}\left( 1\right) \right) \text{. }
\end{eqnarray*}

To show part (c), note that, by the result given in part (a) above, the
columns of $\widehat{G}_{1}=\left( G_{1}+G_{2}R\right) \left(
I_{r}+R^{\prime }R\right) ^{-1/2}$ form an orthonormal basis for a subspace
that is invariant for $A+E$. It then follows from applying Theorem 3.9 on
page 22 of Stewart and Sun (1990) that there exists a unique matrix $L$ such
that $\left( A+E\right) \widehat{G}_{1}=\left( A+E\right) \left(
G_{1}+G_{2}R\right) \left( I_{r}+R^{\prime }R\right) ^{-1/2}=\left(
G_{1}+G_{2}R\right) \left( I_{r}+R^{\prime }R\right) ^{-1/2}L=\widehat{G}%
_{1}L$. Note further that, since by assumption $G=\left[ 
\begin{array}{cc}
G_{1} & G_{2}%
\end{array}%
\right] $ is an orthogonal matrix, we have%
\begin{eqnarray*}
\widehat{G}_{1}^{\prime }\widehat{G}_{1} &=&\left( I_{Kp}+R^{\prime
}R\right) ^{-1/2}\left( G_{1}^{\prime }+R^{\prime }G_{2}^{\prime }\right)
\left( G_{1}+G_{2}R\right) \left( I_{Kp}+R^{\prime }R\right) ^{-1/2} \\
&=&\left( I_{Kp}+R^{\prime }R\right) ^{-1/2}\left( G_{1}^{\prime
}G_{1}+R^{\prime }G_{2}^{\prime }G_{1}+G_{1}^{\prime }G_{2}R+R^{\prime
}G_{2}^{\prime }G_{2}R\right) \left( I_{Kp}+R^{\prime }R\right) ^{-1/2} \\
&=&\left( I_{Kp}+R^{\prime }R\right) ^{-1/2}\left( I_{Kp}+R^{\prime
}R\right) \left( I_{Kp}+R^{\prime }R\right) ^{-1/2} \\
&=&I_{Kp}
\end{eqnarray*}%
which, in turn, implies that $\widehat{G}_{1}^{\prime }\left( A+E\right) 
\widehat{G}_{1}=\widehat{G}_{1}^{\prime }\left( \left[ \Gamma M_{FF}\Gamma
^{\prime }/N_{1}\right] +E\right) \widehat{G}_{1}=\widehat{G}_{1}^{\prime }%
\widehat{G}_{1}L=L$, so that $L$ must be symmetric since, in our case here, $%
A+E=\left( \Gamma M_{FF}\Gamma ^{\prime }N_{1}\right) +\widehat{\Sigma }%
\left( \widehat{H^{c}}\right) -\left( \Gamma M_{FF}\Gamma ^{\prime
}/N_{1}\right) =\widehat{\Sigma }\left( \widehat{H^{c}}\right) =Z\left( 
\widehat{H^{c}}\right) ^{\prime }Z\left( \widehat{H^{c}}\right) /\left(
N_{1}T_{0}\right) $ is a symmetric matrix. Now, let $\widehat{\Lambda }%
=diag\left( \widehat{\lambda }_{1},...,\widehat{\lambda }_{Kp}\right) $ and $%
\widehat{V}=\left( 
\begin{array}{cccc}
\widehat{v}_{1} & \widehat{v}_{2} & \cdots & \widehat{v}_{Kp}%
\end{array}%
\right) $ be as defined in expressions (\ref{LambdaL}) and (\ref{V
eigenvector matrix}). The fact that $L$ is symmetric implies that $\widehat{V%
}$ is an orthogonal matrix. In addition, further application of Theorem 3.9
of Stewart and Sun (1990) shows that $\left( \widehat{G}_{1}\widehat{v}_{g},%
\widehat{\lambda }_{g}\right) $ is an eigenpair for the matrix $A+E$ for $%
g=1,...,Kp$.

Finally, to show part (d), let $G=\left( 
\begin{array}{cc}
G_{1} & G_{2}%
\end{array}%
\right) $, and note that, by assumption, 
\begin{equation*}
G^{\prime }AG=\left( 
\begin{array}{cc}
G_{1}^{\prime }AG_{1} & G_{1}^{\prime }AG_{2} \\ 
G_{2}^{\prime }AG_{1} & G_{2}^{\prime }AG_{2}%
\end{array}%
\right) =\left( 
\begin{array}{cc}
\Lambda _{1} & 0 \\ 
0 & 0%
\end{array}%
\right) =\Lambda \text{,}
\end{equation*}%
where $\Lambda _{1}=diag\left( \lambda _{1,1.},...,\lambda _{1,Kp}\right) $
contains the $Kp$ largest eigenvalues of $A$. Without loss of generality, we
can further assume that $\lambda _{1,1.},...,\lambda _{1,Kp}$ are ordered,
so that $\lambda _{1,j}=\lambda _{\left( j\right) }\left( A\right) $, i.e., $%
\lambda _{1,j}$ is the $j^{th}$ largest eigenvalue of $A$.\footnote{%
If this is not the case; then, we can always define a permutation matrix $%
\mathcal{P}$ such that $\Lambda ^{\ast }=\mathcal{P}^{\prime }\Lambda 
\mathcal{P}$ results in a diagonal matrix whose diagonal elements are
repermutated in such a way, so that the required ordering of the eigenvalues
is satisfied. Moreover, since $\mathcal{P}$ is an orthogonal matrix, it
further follows that $A=G\mathcal{PP}^{\prime }\Lambda \mathcal{PP}^{\prime
}G^{\prime }=G\mathcal{P}\Lambda ^{\ast }\mathcal{P}^{\prime }G^{\prime }$.
Now, define $\widetilde{G}=G\mathcal{P}$, and note that $\widetilde{G}$ is
an orthogonal matrix whose columns are just the columns of $G$ repermutated.
Hence, we can simply proceed with our analysis using $\widetilde{G}$ in lieu
of $G$, and the associated eigenvalues will be in the order which we have
assumed.} Given that, $G^{\prime }G=GG^{\prime }=I_{N}$, we have $\left( 
\begin{array}{cc}
AG_{1} & AG_{2}%
\end{array}%
\right) =AG=G\Lambda =\left( 
\begin{array}{cc}
G_{1}\Lambda _{1} & 0%
\end{array}%
\right) $, from which it follows that%
\begin{equation}
AG_{1}G_{1}^{\prime }\widehat{G}_{1}\widehat{v}_{\ell }=G_{1}\Lambda
_{1}G_{1}^{\prime }\widehat{G}_{1}\widehat{v}_{\ell }\text{, for }\ell \in
\left\{ 1,...,Kp\right\} \text{.}  \label{AG1G1'G1hatv1hat}
\end{equation}%
Now, the result of part (c) above shows $\left( \widehat{G}_{1}\widehat{v}%
_{\ell },\widehat{\lambda }_{\ell }\right) $ to be an eigenpair of the
matrix $A+E$ for $\ell \in \left\{ 1,...,Kp\right\} $, so that 
\begin{equation}
\left( A+E\right) \widehat{G}_{1}\widehat{v}_{\ell }=\widehat{\lambda }%
_{\ell }\widehat{G}_{1}\widehat{v}_{\ell }\text{ for }\ell \in \left\{
1,...,Kp\right\}  \label{A+E eigen eqn}
\end{equation}%
where $\widehat{G}_{1}=\left( G_{1}+G_{2}R\right) \left( I_{Kp}+R^{\prime
}R\right) ^{-1/2}$. Multiplying both sides of expression (\ref{A+E eigen eqn}%
) by $\widehat{v}_{\ell }^{\prime }\widehat{G}_{1}^{\prime
}G_{1}G_{1}^{\prime }$, we get 
\begin{eqnarray}
\widehat{\lambda }_{\ell }\widehat{v}_{\ell }^{\prime }\widehat{G}%
_{1}^{\prime }G_{1}G_{1}^{\prime }\widehat{G}_{1}\widehat{v}_{\ell } &=&%
\widehat{v}_{\ell }^{\prime }\widehat{G}_{1}^{\prime }G_{1}G_{1}^{\prime
}\left( A+E\right) \widehat{G}_{1}\widehat{v}_{\ell }  \notag \\
&=&\widehat{v}_{\ell }^{\prime }\widehat{G}_{1}^{\prime }G_{1}G_{1}^{\prime
}A\widehat{G}_{1}\widehat{v}_{\ell }+\widehat{v}_{\ell }^{\prime }\widehat{G}%
_{1}^{\prime }G_{1}G_{1}^{\prime }E\widehat{G}_{1}\widehat{v}_{\ell }
\label{lambdavGhatG1G1'Ghatv}
\end{eqnarray}%
Since $A=\Gamma M_{FF}\Gamma ^{\prime }/N_{1}$ is symmetric, it further
follows by expression (\ref{AG1G1'G1hatv1hat}) that%
\begin{equation}
\widehat{v}_{\ell }^{\prime }\widehat{G}_{1}^{\prime }G_{1}G_{1}^{\prime }A=%
\widehat{v}_{\ell }^{\prime }\widehat{G}_{1}^{\prime }G_{1}G_{1}^{\prime
}A^{\prime }=\widehat{v}_{\ell }^{\prime }\widehat{G}_{1}^{\prime
}G_{1}\Lambda _{1}G_{1}^{\prime }  \label{vGhatG1G1'A}
\end{equation}%
Moreover, note that%
\begin{eqnarray*}
0 &\leq &\left( \widehat{v}_{\ell }^{\prime }\widehat{G}_{1}^{\prime
}G_{1}G_{1}^{\prime }E\widehat{G}_{1}\widehat{v}_{\ell }\right) ^{2} \\
&\leq &\left( \widehat{v}_{\ell }^{\prime }\widehat{G}_{1}^{\prime
}G_{1}G_{1}^{\prime }\widehat{G}_{1}\widehat{v}_{\ell }\right) \left( 
\widehat{v}_{\ell }^{\prime }\widehat{G}_{1}^{\prime }E^{\prime }E\widehat{G}%
_{1}\widehat{v}_{\ell }\right) \text{ }\left( \text{by CS inequality and the
fact that }G_{1}^{\prime }G_{1}=I_{Kp}\right) \\
&=&\left[ \widehat{v}_{\ell }^{\prime }\left( I_{Kp}+R^{\prime }R\right) ^{-%
\frac{1}{2}}\left( G_{1}^{\prime }+R^{\prime }G_{2}^{\prime }\right)
G_{1}G_{1}^{\prime }\left( G_{1}+G_{2}R\right) \left( I_{Kp}+R^{\prime
}R\right) ^{-\frac{1}{2}}\widehat{v}_{\ell }\right] \left( \widehat{v}_{\ell
}^{\prime }\widehat{G}_{1}^{\prime }E^{\prime }E\widehat{G}_{1}\widehat{v}%
_{\ell }\right) \\
&\leq &\left[ \widehat{v}_{\ell }^{\prime }\left( I_{Kp}+R^{\prime }R\right)
^{-1}\widehat{v}_{\ell }\right] \lambda _{\max }\left( E^{\prime }E\right)
\end{eqnarray*}%
from which it follows that%
\begin{eqnarray}
-\sqrt{\widehat{v}_{\ell }^{\prime }\left( I_{Kp}+R^{\prime }R\right) ^{-1}%
\widehat{v}_{\ell }}\left\Vert E\right\Vert _{2} &=&-\sqrt{\widehat{v}_{\ell
}^{\prime }\left( I_{Kp}+R^{\prime }R\right) ^{-1}\widehat{v}_{\ell }}\sqrt{%
\lambda _{\max }\left( E^{\prime }E\right) }  \notag \\
&\leq &-\sqrt{\left( \widehat{v}_{\ell }^{\prime }\widehat{G}_{1}^{\prime
}G_{1}G_{1}^{\prime }E\widehat{G}_{1}\widehat{v}_{\ell }\right) ^{2}}  \notag
\\
&\leq &-\left\vert \widehat{v}_{\ell }^{\prime }\widehat{G}_{1}^{\prime
}G_{1}G_{1}^{\prime }E\widehat{G}_{1}\widehat{v}_{\ell }\right\vert  \notag
\end{eqnarray}%
\begin{equation}
\leq \widehat{v}_{\ell }^{\prime }\widehat{G}_{1}^{\prime
}G_{1}G_{1}^{\prime }E\widehat{G}_{1}\widehat{v}_{\ell }\text{.}
\label{vG1hatG1G1'EG1hatv}
\end{equation}%
Combining expressions (\ref{lambdavGhatG1G1'Ghatv}), (\ref{vGhatG1G1'A}),
and (\ref{vG1hatG1G1'EG1hatv}), we see that%
\begin{eqnarray}
\widehat{\lambda }_{\ell }\widehat{v}_{\ell }^{\prime }\widehat{G}%
_{1}^{\prime }G_{1}G_{1}^{\prime }\widehat{G}_{1}\widehat{v}_{\ell } &=&%
\widehat{v}_{\ell }^{\prime }\widehat{G}_{1}^{\prime }G_{1}G_{1}^{\prime }A%
\widehat{G}_{1}\widehat{v}_{\ell }+\widehat{v}_{\ell }^{\prime }\widehat{G}%
_{1}^{\prime }G_{1}G_{1}^{\prime }E\widehat{G}_{1}\widehat{v}_{\ell }  \notag
\\
&\geq &\widehat{v}_{\ell }^{\prime }\widehat{G}_{1}^{\prime }G_{1}\Lambda
_{1}G_{1}^{\prime }\widehat{G}_{1}\widehat{v}_{\ell }-\sqrt{\widehat{v}%
_{\ell }^{\prime }\left( I_{Kp}+R^{\prime }R\right) ^{-1}\widehat{v}_{\ell }}%
\left\Vert E\right\Vert _{2}  \label{lambdavGhatG1G1'Ghatv bd}
\end{eqnarray}%
for $\ell \in \left\{ 1,...,Kp\right\} $. In addition, note that%
\begin{eqnarray*}
\widehat{v}_{\ell }^{\prime }\widehat{G}_{1}^{\prime }G_{1}G_{1}^{\prime }%
\widehat{G}_{1}\widehat{v}_{\ell } &=&\widehat{v}_{\ell }^{\prime }\widehat{G%
}_{1}^{\prime }G_{1}G_{1}^{\prime }\left( G_{1}+G_{2}R\right) \left(
I_{Kp}+R^{\prime }R\right) ^{-1/2}\widehat{v}_{\ell } \\
&=&\widehat{v}_{\ell }^{\prime }\widehat{G}_{1}^{\prime }G_{1}\left(
I_{Kp}+R^{\prime }R\right) ^{-1/2}\widehat{v}_{\ell } \\
&=&\widehat{v}_{\ell }^{\prime }\left( I_{Kp}+R^{\prime }R\right)
^{-1/2}\left( G_{1}^{\prime }+R^{\prime }G_{2}^{\prime }\right) G_{1}\left(
I_{Kp}+R^{\prime }R\right) ^{-1/2}\widehat{v}_{\ell } \\
&=&\widehat{v}_{\ell }^{\prime }\left( I_{Kp}+R^{\prime }R\right) ^{-1}%
\widehat{v}_{\ell } \\
&>&0
\end{eqnarray*}%
Hence, dividing both sides of expression (\ref{lambdavGhatG1G1'Ghatv bd}) by 
$\widehat{v}_{\ell }^{\prime }\widehat{G}_{1}^{\prime }G_{1}G_{1}^{\prime }%
\widehat{G}_{1}\widehat{v}_{\ell }$, we obtain%
\begin{eqnarray*}
\widehat{\lambda }_{\ell } &\geq &\frac{\widehat{v}_{\ell }^{\prime }%
\widehat{G}_{1}^{\prime }G_{1}\Lambda _{1}G_{1}^{\prime }\widehat{G}_{1}%
\widehat{v}_{\ell }}{\widehat{v}_{\ell }^{\prime }\widehat{G}_{1}^{\prime
}G_{1}G_{1}^{\prime }\widehat{G}_{1}\widehat{v}_{\ell }}-\frac{\sqrt{%
\widehat{v}_{\ell }^{\prime }\left( I_{Kp}+R^{\prime }R\right) ^{-1}\widehat{%
v}_{\ell }}\left\Vert E\right\Vert _{2}}{\widehat{v}_{\ell }^{\prime }%
\widehat{G}_{1}^{\prime }G_{1}G_{1}^{\prime }\widehat{G}_{1}\widehat{v}%
_{\ell }}\text{ } \\
&=&\widetilde{v}_{\ell }^{\prime }\Lambda _{1}\widetilde{v}_{\ell }-\frac{%
\sqrt{\widehat{v}_{\ell }^{\prime }\left( I_{Kp}+R^{\prime }R\right) ^{-1}%
\widehat{v}_{\ell }}\left\Vert E\right\Vert _{2}}{\widehat{v}_{\ell
}^{\prime }\left( I_{Kp}+R^{\prime }R\right) ^{-1}\widehat{v}_{\ell }} \\
&=&\widetilde{v}_{\ell }^{\prime }\Lambda _{1}\widetilde{v}_{\ell }-\frac{%
\left\Vert E\right\Vert _{2}}{\sqrt{\widehat{v}_{\ell }^{\prime }\left(
I_{Kp}+R^{\prime }R\right) ^{-1}\widehat{v}_{\ell }}} \\
&=&\dsum\limits_{j=1}^{Kp}\widetilde{v}_{\ell ,j}^{2}\lambda _{1,j}-\frac{%
\left\Vert E\right\Vert _{2}}{\sqrt{\widehat{v}_{\ell }^{\prime }\left(
I_{Kp}+R^{\prime }R\right) ^{-1}\widehat{v}_{\ell }}}
\end{eqnarray*}%
where $\widetilde{v}_{\ell }=G_{1}^{\prime }\widehat{G}_{1}\widehat{v}_{\ell
}/\left( \widehat{v}_{\ell }^{\prime }\widehat{G}_{1}^{\prime
}G_{1}G_{1}^{\prime }\widehat{G}_{1}\widehat{v}_{\ell }\right) $ so that $%
\left\Vert \widetilde{v}_{\ell }\right\Vert _{2}^{2}=\dsum\limits_{\ell
=1}^{Kp}\widetilde{v}_{\ell ,j}^{2}=1$. Note also that, by applying Theorem
8.1.10 of Golub and van Loan (1996) (or GvL for short), we have%
\begin{eqnarray*}
\widehat{v}_{\ell }^{\prime }\left( I_{Kp}+R^{\prime }R\right) ^{-1}\widehat{%
v}_{\ell } &\geq &\lambda _{\min }\left\{ \left( I_{Kp}+R^{\prime }R\right)
^{-1}\right\} \widehat{v}_{\ell }^{\prime }\widehat{v}_{\ell } \\
\text{ \ } &=&\frac{1}{\lambda _{\max }\left( I_{Kp}+R^{\prime }R\right) }%
\text{ }\left( \text{given that }\left\Vert \widehat{v}_{\ell }\right\Vert
_{2}^{2}=1\right) \\
&\geq &\frac{1}{1+\lambda _{\max }\left( R^{\prime }R\right) }=\frac{1}{%
1+\left\Vert R\right\Vert _{2}^{2}}
\end{eqnarray*}%
\begin{eqnarray*}
&\geq &\left[ 1+\frac{16\left\Vert E_{21}\right\Vert _{2}^{2}}{\left( \text{%
sep}\left( \Lambda _{1},\Lambda _{2}\right) \right) ^{2}}\right] ^{-1}\text{
\ }\left( \text{by Theorem 8.1.10 of GvL}\right) \\
&\geq &\left[ 1+\frac{16\left\Vert E\right\Vert _{2}^{2}}{\left( \text{sep}%
\left( \Lambda _{1},\Lambda _{2}\right) \right) ^{2}}\right] ^{-1}\text{ \ }%
\left( \text{since }\left\Vert E_{21}\right\Vert _{2}^{2}\leq \left\Vert
E\right\Vert _{2}^{2}\right) \\
&\geq &\left[ 1+\frac{16\left( \text{sep}\left( \Lambda _{1},\Lambda
_{2}\right) \right) ^{2}/25}{\left( \text{sep}\left( \Lambda _{1},\Lambda
_{2}\right) \right) ^{2}}\right] ^{-1}=\frac{25}{41}\text{.}
\end{eqnarray*}%
Making use of this lower bound, we obtain $\widehat{\lambda }_{\ell }\geq
\dsum\nolimits_{j=1}^{Kp}\widetilde{v}_{\ell ,j}^{2}\lambda _{1,j}-\left(
\left\Vert E\right\Vert _{2}/\sqrt{\widehat{v}_{\ell }^{\prime }\left(
I_{Kp}+R^{\prime }R\right) ^{-1}\widehat{v}_{\ell }}\right) $

\noindent $=\dsum\nolimits_{j=1}^{Kp}\widetilde{v}_{\ell ,j}^{2}\lambda
_{1,j}-\frac{25}{41}\left\Vert E\right\Vert _{2}$. Next, the ordering of the
eigenvalues of the matrices $A+E$ and $A$ be given by $\lambda _{\left(
1\right) }\left( A+E\right) \geq \cdot \cdot \cdot \geq \lambda _{\left(
Kp\right) }\left( A+E\right) \geq \lambda _{\left( Kp+1\right) }\left(
A+E\right) \geq \cdot \cdot \cdot \geq \lambda _{\left( N\right) }\left(
A+E\right) $ and $\lambda _{\left( 1\right) }\left( A\right) \geq \cdot
\cdot \cdot \geq \lambda _{\left( Kp\right) }\left( A\right) \geq \lambda
_{\left( Kp+1\right) }\left( A\right) \geq \cdot \cdot \cdot \geq \lambda
_{\left( N\right) }\left( A\right) $. Since $A=\Gamma M_{FF}\Gamma ^{\prime
}/N_{1}$ and since $Rank\left( A\right) =Kp$ for all $N_{1}$, $N_{2}$, and $%
T $ sufficiently large\footnote{%
An explicit proof that $Rank\left( A\right) =Kp$ for all $N_{1}$, $N_{2}$,
and $T$ sufficiently large is given in the Technical Appendix of an earlier
version of this paper, Chao and Swanson (2022b). In particular, this result
is shown in part (a) of Lemma D-11 in Appendix D of Chao and Swanson (2022b).%
}; it follows that $\lambda _{\left( Kp+1\right) }\left( A\right) =\cdot
\cdot \cdot =\lambda _{\left( N\right) }\left( A\right) =0$. In addition, by
Corollary 8.1.6 of Golub and van Loan (1996), we have the inequality $%
\lambda _{\left( Kp+1\right) }\left( A+E\right) \leq \lambda _{\left(
Kp+1\right) }\left( A\right) +\left\Vert E\right\Vert _{2}$. Making use of
these results, we see that, for any $\ell \in \left\{ 1,...,Kp\right\} $,%
\begin{eqnarray*}
\widehat{\lambda }_{\ell }-\lambda _{\left( Kp+1\right) }\left( A+E\right)
&\geq &\dsum\limits_{j=1}^{Kp}\widetilde{v}_{\ell ,j}^{2}\lambda _{1,j}-%
\frac{25}{41}\left\Vert E\right\Vert _{2}-\left\{ \lambda _{\left(
Kp+1\right) }\left( A\right) +\left\Vert E\right\Vert _{2}\right\} \\
&=&\dsum\limits_{j=1}^{Kp}\widetilde{v}_{\ell ,j}^{2}\lambda _{\left(
j\right) }\left( A\right) -\frac{66}{41}\left\Vert E\right\Vert _{2}\text{ \ 
}\left( \text{since }\lambda _{\left( Kp+1\right) }\left( A\right) =0\text{
and }\lambda _{1,j}=\lambda _{\left( j\right) }\left( A\right) \right) \\
&\geq &\dsum\limits_{j=1}^{Kp}\widetilde{v}_{\ell ,j}^{2}\lambda _{\left(
j\right) }\left( A\right) -\frac{66}{41}\frac{\text{sep}\left( \Lambda
_{1},\Lambda _{2}\right) }{5}\text{ \ \ \ \ }\left( \text{by Theorem 8.1.10
of \ GvL}\right) \\
&=&\dsum\limits_{j=1}^{Kp}\widetilde{v}_{\ell ,j}^{2}\lambda _{\left(
j\right) }\left( A\right) -\frac{66}{205}\text{sep}\left( \Lambda
_{1},0\right) \text{ }\left( \text{since }\Lambda _{2}=0\text{ here}\right)
\\
&\geq &\lambda _{\min }\left( \Lambda _{1}\right) -\frac{66}{205}\text{sep}%
\left( \Lambda _{1},0\right) \text{ }\left( \text{since }\Lambda
_{1}=diag\left( \lambda _{\left( 1\right) }\left( A\right) ,...,\lambda
_{\left( Kp\right) }\left( A\right) \right) \right) \\
&=&\frac{139}{205}\text{sep}\left( \Lambda _{1},0\right) \text{ } \\
&\geq &\frac{139}{205}\underline{c}>0\text{ \ }\left( \text{by expression (%
\ref{lower bd for sep})}\right) \text{. }
\end{eqnarray*}%
where the last equality above follows from the fact that sep$\left( \Lambda
_{1},0\right) =\lambda _{\min }\left( \Lambda _{1}\right) $ by Theorem 3.1
on page 247 of Stewart and Sun (1990) since $A$ is symmetric in this case.
This shows that the set $\left\{ \widehat{\lambda }_{1},...,\widehat{\lambda 
}_{Kp}\right\} $ contains the $Kp$ largest eigenvalues of the matrix $A+E$.
It further follows from the result given in part (c) that the columns of the
matrix $\widehat{G}_{1}\widehat{V}=\left( 
\begin{array}{cccc}
\widehat{G}_{1}\widehat{v}_{1} & \widehat{G}_{1}\widehat{v}_{2} & \cdots & 
\widehat{G}_{1}\widehat{v}_{Kp}%
\end{array}%
\right) $ are the eigenvectors associated with the $Kp$ largest eigenvalues
of the matrix $A+E$. $\square $

\medskip

\noindent \textbf{Lemma OA-2: }Suppose that Assumptions 2-1, 2-2, 2-3, 2-4,
2-5, 2-6, 2-7, 2-8, 2-9, OA-3, and OA-4 hold. Then, the following statements
are true.

\begin{enumerate}
\item[(a)] $\left( \widehat{N}_{1}-N_{1}\right) /N_{1}\overset{p}{%
\rightarrow }0$

\item[(b)] $\left\Vert \left[ \Gamma \left( \widehat{H^{c}}\right) -\Gamma %
\right] /\sqrt{N_{1}}\right\Vert _{2}\overset{p}{\rightarrow }0$

\item[(c)] Let $\widehat{G}_{1}=\left( G_{1}+G_{2}R\right) \left(
I_{Kp}+R^{\prime }R\right) ^{-1/2}$, where $G_{1}$, $G_{2}$, and $R$ are as
defined in Lemma OA-1 above. Also, let $\widehat{V}$ be the $Kp\times Kp$
orthogonal matrix given in expression (\ref{V eigenvector matrix}) of Lemma
OA-1. Then, there exists some positive constant $\overline{C}$ such that $%
\left\Vert \widehat{V}^{\prime }\widehat{G}_{1}^{\prime }\Gamma /\sqrt{N_{1}}%
\right\Vert _{2}\leq \sqrt{\lambda _{\max }\left( \Gamma ^{\prime }\Gamma
/N_{1}\right) }\leq \overline{C}<\infty $ for $N_{1},N_{2}$, and $T$
sufficiently large. In addition,

$\left\Vert \widehat{V}^{\prime }\widehat{G}_{1}^{\prime }\Gamma /\sqrt{N_{1}%
}-Q^{\prime }\right\Vert _{2}\overset{p}{\rightarrow }0$, where $Q=\left(
\Gamma ^{\prime }\Gamma /N_{1}\right) ^{\frac{{\Large 1}}{{\Large 2}}}\Xi 
\widehat{V}$, with $\Xi $ being the $Kp\times Kp$ orthogonal matrix whose
columns are the eigenvectors of the matrix

$M_{FF}^{\ast }=\left( \Gamma ^{\prime }\Gamma /N_{1}\right) ^{1/2}\frac{1}{%
T-p+1}\dsum\nolimits_{t=p}^{T}E\left[ \underline{F}_{t}\underline{F}%
_{t}^{\prime }\right] \left( \Gamma ^{\prime }\Gamma /N_{1}\right) ^{1/2}$.

\item[(d)] For all fixed index $t$, $\left\Vert G_{1}^{\prime }U_{t,N}\left( 
\widehat{H^{c}}\right) /\sqrt{N_{1}}\right\Vert _{2}=o_{p}\left( 1\right) $.

\item[(e)] For all fixed index $t$, $\left\Vert U_{t,N}\left( \widehat{H^{c}}%
\right) /\sqrt{N_{1}}\right\Vert _{2}^{2}=O_{p}\left( 1\right) $.

\item[(f)] For all fixed index $t$, $\left\Vert G_{2}^{\prime }U_{t,N}\left( 
\widehat{H^{c}}\right) /\sqrt{N_{1}}\right\Vert _{2}=O_{p}\left( 1\right) $.

\item[(g)] Let $\widehat{G}_{1}=\left( G_{1}+G_{2}R\right) \left(
I_{Kp}+R^{\prime }R\right) ^{-1/2}$, where $G_{1}$, $G_{2}$, and $R$ are as
defined in Lemma OA-1 above. Also, let $\widehat{V}$ be the $Kp\times Kp$
orthogonal matrix given in expression (\ref{V eigenvector matrix}) of Lemma
OA-1. Then, for all fixed index $t$, $\left\Vert \widehat{V}^{\prime }%
\widehat{G}_{1}^{\prime }U_{t,N}\left( \widehat{H^{c}}\right) /\sqrt{%
\widehat{N}_{1}}\right\Vert _{2}\overset{p}{\rightarrow }0$ as $N_{1}$, $%
N_{2}$, and $T\rightarrow \infty $.

\item[(h)] $\left\Vert \underline{F}_{t}\right\Vert _{2}=O_{p}\left(
1\right) $ for all $t$.
\end{enumerate}

\medskip

\noindent \textbf{Proof of Lemma OA-2:}

To show part (a), note first that, for any $\epsilon >0$,%
\begin{eqnarray*}
\left\{ \left\vert \frac{\widehat{N}_{1}-N_{1}}{N_{1}}\right\vert \geq
\epsilon \right\} &=&\left\{ \left\vert \frac{1}{N_{1}}\dsum\limits_{i=1}^{N}%
\mathbb{I}\left\{ i\in \widehat{H^{c}}\right\} -1\right\vert \geq \epsilon
\right\} \\
&=&\left\{ \left\vert \frac{1}{N_{1}}\dsum\limits_{i\in H^{{\large c}%
}}\left( \mathbb{I}\left\{ i\in \widehat{H^{c}}\right\} -1\right) +\frac{1}{%
N_{1}}\dsum\limits_{i\in H}\mathbb{I}\left\{ i\in \widehat{H^{c}}\right\}
\right\vert \geq \epsilon \right\} \\
&\subseteq &\left\{ \left\vert \frac{1}{N_{1}}\dsum\limits_{i\in H^{{\large c%
}}}\left( \mathbb{I}\left\{ i\in \widehat{H^{c}}\right\} -1\right)
\right\vert +\left\vert \frac{1}{N_{1}}\dsum\limits_{i\in H}\mathbb{I}%
\left\{ i\in \widehat{H^{c}}\right\} \right\vert \geq \epsilon \right\} \\
&\subseteq &\left\{ \left\vert \frac{1}{N_{1}}\dsum\limits_{i\in H^{{\large c%
}}}\left( \mathbb{I}\left\{ i\in \widehat{H^{c}}\right\} -1\right)
\right\vert \geq \frac{\epsilon }{2}\right\} \cup \left\{ \left\vert \frac{1%
}{N_{1}}\dsum\limits_{i\in H}\mathbb{I}\left\{ i\in \widehat{H^{c}}\right\}
\right\vert \geq \frac{\epsilon }{2}\right\}
\end{eqnarray*}%
By Markov's inequality, we have%
\begin{eqnarray*}
&&\Pr \left( \left\vert \frac{1}{N_{1}}\dsum\limits_{i\in H^{{\large c}%
}}\left( \mathbb{I}\left\{ i\in \widehat{H^{c}}\right\} -1\right)
\right\vert \geq \frac{\epsilon }{2}\right) \\
&=&\Pr \left( \left\vert \frac{1}{N_{1}}\dsum\limits_{i\in H^{{\large c}%
}}\left( \mathbb{I}\left\{ i\in \widehat{H^{c}}\right\} -1\right)
\right\vert ^{2}\geq \frac{\epsilon ^{2}}{4}\right) \\
&\leq &\frac{4}{\epsilon ^{2}}E\left\{ \left\vert \frac{1}{N_{1}}%
\dsum\limits_{i\in H^{{\large c}}}\left( \mathbb{I}\left\{ i\in \widehat{%
H^{c}}\right\} -1\right) \right\vert ^{2}\right\} \\
&=&\frac{4}{\epsilon ^{2}}\frac{1}{N_{1}^{2}}\dsum\limits_{i\in H^{{\large c}%
}}\dsum\limits_{k\in H^{{\large c}}}E\left[ \left( \mathbb{I}\left\{ i\in 
\widehat{H^{c}}\right\} -1\right) \left( \mathbb{I}\left\{ k\in \widehat{%
H^{c}}\right\} -1\right) \right] \\
&=&\frac{4}{\epsilon ^{2}}\frac{1}{N_{1}^{2}}\dsum\limits_{i\in H^{{\large c}%
}}\dsum\limits_{k\in H^{{\large c}}}\left\{ \Pr \left( \left\{ i\in \widehat{%
H^{c}}\right\} \cap \left\{ k\in \widehat{H^{c}}\right\} \right) -\Pr \left(
k\in \widehat{H^{c}}\right) \right\} \\
&&+\frac{4}{\epsilon ^{2}}\frac{1}{N_{1}^{2}}\dsum\limits_{i\in H^{{\large c}%
}}\dsum\limits_{k\in H^{{\large c}}}\left\{ 1-\Pr \left( i\in \widehat{H^{c}}%
\right) \right\} \\
&\leq &\frac{4}{\epsilon ^{2}}\frac{1}{N_{1}^{2}}\dsum\limits_{i\in H^{%
{\large c}}}\dsum\limits_{k\in H^{{\large c}}}\left\{ \Pr \left( k\in 
\widehat{H^{c}}\right) -\Pr \left( k\in \widehat{H^{c}}\right) \right\} +%
\frac{4}{\epsilon ^{2}}\frac{1}{N_{1}}\dsum\limits_{i\in H^{{\large c}%
}}\left\{ 1-\Pr \left( i\in \widehat{H^{c}}\right) \right\} \\
&\leq &\frac{4}{\epsilon ^{2}}\frac{1}{N_{1}}\dsum\limits_{i\in H^{{\large c}%
}}\left\{ 1-\min_{i\in H^{{\large c}}}\Pr \left( i\in \widehat{H^{c}}\right)
\right\} \rightarrow 0\text{ as }N_{1},N_{2}\text{, and }T\rightarrow \infty 
\text{.}
\end{eqnarray*}%
where the last line above follows from the fact that, for $i\in H^{c}$ and
for either the case where $\mathbb{S}_{i,T}^{+}=\dsum\nolimits_{\ell
=1}^{d}\varpi _{\ell }\left\vert S_{i,\ell ,T}\right\vert $ or the case
where $\mathbb{S}_{i,T}^{+}=\max_{1\leq \ell \leq d}\left\vert S_{i,\ell
,T}\right\vert $, we can apply the results of Theorem 2 of the main paper to
obtain $\min_{i\in H^{{\large c}}}\Pr \left( i\in \widehat{H^{c}}\right)
\geq \Pr \left( \dbigcap\limits_{i\in H^{{\large c}}}\left\{ \mathbb{S}%
_{i,T}^{+}\geq \Phi ^{-1}\left( 1-\frac{\varphi }{2N}\right) \right\}
\right) $

\noindent $=P\left( \min_{{\large i\in }H^{{\large c}}}\mathbb{S}%
_{i,T}^{+}\geq \Phi ^{-1}\left( 1-\frac{\varphi }{2N}\right) \right)
\rightarrow 1$. Also, making use of Markov's inequality, we obtain, for
either the case where $\mathbb{S}_{i,T}^{+}=\dsum\nolimits_{\ell
=1}^{d}\varpi _{\ell }\left\vert S_{i,\ell ,T}\right\vert $ or the case
where $\mathbb{S}_{i,T}^{+}=\max_{1\leq \ell \leq d}\left\vert S_{i,\ell
,T}\right\vert $,%
\begin{eqnarray*}
\Pr \left( \left\vert \frac{1}{N_{1}}\dsum\limits_{i\in H}\mathbb{I}\left\{
i\in \widehat{H^{c}}\right\} \right\vert \geq \frac{\epsilon }{2}\right)
&=&\Pr \left( \frac{1}{N_{1}}\dsum\limits_{i\in H}\mathbb{I}\left\{ i\in 
\widehat{H^{c}}\right\} \geq \frac{\epsilon }{2}\right) \\
&\leq &\frac{2}{\epsilon }E\left[ \frac{1}{N_{1}}\dsum\limits_{i\in H}%
\mathbb{I}\left\{ i\in \widehat{H^{c}}\right\} \right] \\
&=&\frac{2}{\epsilon }\frac{1}{N_{1}}\dsum\limits_{i\in H}\Pr \left( i\in 
\widehat{H^{c}}\right) \\
&=&\frac{2}{\epsilon }\frac{1}{N_{1}}\dsum\limits_{i\in H}\Pr \left( \mathbb{%
S}_{i,T}^{+}\geq \Phi ^{-1}\left( 1-\frac{\varphi }{2N}\right) \right) \\
&\leq &\frac{2}{\epsilon }\frac{dN_{2}\varphi }{NN_{1}}\left[ 1+o\left(
1\right) \right] \\
&\rightarrow &0\text{ }\left( \text{since }\frac{\varphi }{N_{1}}\rightarrow
0\text{ and }\frac{N_{2}}{N}=O\left( 1\right) \right)
\end{eqnarray*}%
where the last inequality above follows by an argument similar to that given
in the proof of Theorem 1 of the main paper. Combining these results, we
have that%
\begin{eqnarray*}
&&\Pr \left( \left\vert \frac{\widehat{N}_{1}-N_{1}}{N_{1}}\right\vert \geq
\epsilon \right) \\
&\leq &\Pr \left( \left\{ \left\vert \frac{1}{N_{1}}\dsum\limits_{i\in H^{%
{\large c}}}\left( \mathbb{I}\left\{ i\in \widehat{H^{c}}\right\} -1\right)
\right\vert \geq \frac{\epsilon }{2}\right\} \cup \left\{ \left\vert \frac{1%
}{N_{1}}\dsum\limits_{i\in H}\mathbb{I}\left\{ i\in \widehat{H^{c}}\right\}
\right\vert \geq \frac{\epsilon }{2}\right\} \right) \\
&\leq &\Pr \left( \left\vert \frac{1}{N_{1}}\dsum\limits_{i\in H^{{\large c}%
}}\left( \mathbb{I}\left\{ i\in \widehat{H^{c}}\right\} -1\right)
\right\vert \geq \frac{\epsilon }{2}\right) +\Pr \left( \left\vert \frac{1}{%
N_{1}}\dsum\limits_{i\in H}\mathbb{I}\left\{ i\in \widehat{H^{c}}\right\}
\right\vert \geq \frac{\epsilon }{2}\right) \\
&\rightarrow &0
\end{eqnarray*}

For part (b), note that%
\begin{eqnarray*}
\left\Vert \frac{\Gamma \left( \widehat{H^{c}}\right) -\Gamma }{\sqrt{N_{1}}}%
\right\Vert _{F}^{2} &=&\frac{1}{N_{1}}tr\left\{ \left( \Gamma \left( 
\widehat{H^{c}}\right) -\Gamma \right) ^{\prime }\left( \Gamma \left( 
\widehat{H^{c}}\right) -\Gamma \right) \right\} \\
&=&\frac{1}{N_{1}}\dsum\limits_{i=1}^{N}tr\left\{ \left( \mathbb{I}\left\{
i\in \widehat{H^{c}}\right\} \gamma _{i}-\gamma _{i}\right) \left( \mathbb{I}%
\left\{ i\in \widehat{H^{c}}\right\} \gamma _{i}-\gamma _{i}\right) ^{\prime
}\right\} \\
&=&\frac{1}{N_{1}}\dsum\limits_{i=1}^{N}\left( \mathbb{I}\left\{ i\in 
\widehat{H^{c}}\right\} \gamma _{i}-\gamma _{i}\right) ^{\prime }\left( 
\mathbb{I}\left\{ i\in \widehat{H^{c}}\right\} \gamma _{i}-\gamma _{i}\right)
\end{eqnarray*}%
\begin{eqnarray*}
&=&\frac{1}{N_{1}}\dsum\limits_{i=1}^{N}\gamma _{i}^{\prime }\gamma _{i}%
\left[ 1-\mathbb{I}\left\{ i\in \widehat{H^{c}}\right\} \right] \\
&=&\frac{1}{N_{1}}\dsum\limits_{i\in H^{{\large c}}}\gamma _{i}^{\prime
}\gamma _{i}\left[ 1-\mathbb{I}\left\{ i\in \widehat{H^{c}}\right\} \right] 
\text{ }\left( \text{since }\gamma _{i}=0\text{ for }i\in H\right)
\end{eqnarray*}%
Applying Markov's inequality, we have, for any $\epsilon >0$,%
\begin{eqnarray*}
\Pr \left( \left\Vert \frac{\Gamma \left( \widehat{H^{c}}\right) -\Gamma }{%
\sqrt{N_{1}}}\right\Vert _{F}^{2}\geq \epsilon \right) &\leq &\frac{1}{%
\epsilon }E\left\{ \frac{1}{N_{1}}\dsum\limits_{i\in H^{{\large c}}}\gamma
_{i}^{\prime }\gamma _{i}\left[ 1-\mathbb{I}\left\{ i\in \widehat{H^{c}}%
\right\} \right] \right\} \\
&=&\frac{1}{\epsilon }\frac{1}{N_{1}}\dsum\limits_{i\in H^{{\large c}%
}}\gamma _{i}^{\prime }\gamma _{i}\left[ 1-\Pr \left( i\in \widehat{H^{c}}%
\right) \right] \\
&\leq &\frac{1}{\epsilon }\left[ 1-\min_{i\in H^{{\large c}}}\Pr \left( i\in 
\widehat{H^{c}}\right) \right] \frac{1}{N_{1}}\dsum\limits_{i\in H^{{\large c%
}}}\gamma _{i}^{\prime }\gamma _{i} \\
&\leq &\frac{1}{\epsilon }\left[ 1-\min_{i\in H^{{\large c}}}\Pr \left( i\in 
\widehat{H^{c}}\right) \right] \left( \sup_{i\in H^{{\large c}}}\left\Vert
\gamma _{i}\right\Vert _{2}\right) ^{2} \\
&\leq &\frac{1}{\epsilon }\left[ 1-\min_{i\in H^{{\large c}}}\Pr \left( i\in 
\widehat{H^{c}}\right) \right] \overline{C}^{2}\text{ \ }\left( \text{by
Assumption 2-5}\right) \\
&\rightarrow &0
\end{eqnarray*}%
where the last line follows from the fact that $\min_{i\in H^{{\large c}%
}}\Pr \left( i\in \widehat{H^{c}}\right) \rightarrow 1$ by Theorem 2 of the
main paper. Using this result, we further deduce that $\left\Vert \left(
\Gamma \left( \widehat{H^{c}}\right) -\Gamma \right) /\sqrt{N_{1}}%
\right\Vert _{2}\leq \left\Vert \left( \Gamma \left( \widehat{H^{c}}\right)
-\Gamma \right) /\sqrt{N_{1}}\right\Vert _{F}\overset{p}{\rightarrow }0.$

Turning our attention to part (c), note that since, by definition,

\noindent $\widehat{G}_{1}=\left( G_{1}+G_{2}R\right) \left(
I_{Kp}+R^{\prime }R\right) ^{-1/2}$, where $G_{1}^{\prime }G_{1}=I_{Kp}$, $%
G_{2}^{\prime }G_{2}=I_{N-Kp}$, and $G_{1}^{\prime }G_{2}=0$; it follows
that $\widehat{G}_{1}^{\prime }\widehat{G}_{1}=\left( I_{Kp}+R^{\prime
}R\right) ^{-1/2}\left( G_{1}^{\prime }+R^{\prime }G_{2}^{\prime }\right)
\left( G_{1}+G_{2}R\right) \left( I_{Kp}+R^{\prime }R\right) ^{-1/2}$

\noindent $=\left( I_{Kp}+R^{\prime }R\right) ^{-1/2}\left( I_{Kp}+R^{\prime
}R\right) \left( I_{Kp}+R^{\prime }R\right) ^{-1/2}=I_{Kp}$. Hence, by
Assumption OA-3,%
\begin{eqnarray*}
\left\Vert \frac{\widehat{V}^{\prime }\widehat{G}_{1}^{{\Large \prime }%
}\Gamma }{\sqrt{N_{1}}}\right\Vert _{2} &\leq &\left\Vert \widehat{V}%
^{\prime }\widehat{G}_{1}^{{\Large \prime }}\right\Vert _{2}\left\Vert \frac{%
\Gamma }{\sqrt{N_{1}}}\right\Vert _{2}=\sqrt{\lambda _{\max }\left( \widehat{%
G}_{1}\widehat{V}\widehat{V}^{\prime }\widehat{G}_{1}^{{\Large \prime }%
}\right) }\sqrt{\lambda _{\max }\left( \frac{\Gamma ^{\prime }\Gamma }{N_{1}}%
\right) } \\
&=&\sqrt{\lambda _{\max }\left( \widehat{V}^{\prime }\widehat{G}_{1}^{%
{\Large \prime }}\widehat{G}_{1}\widehat{V}\right) }\sqrt{\lambda _{\max
}\left( \frac{\Gamma ^{\prime }\Gamma }{N_{1}}\right) } \\
&=&\sqrt{\lambda _{\max }\left( I_{Kp}\right) }\sqrt{\lambda _{\max }\left( 
\frac{\Gamma ^{\prime }\Gamma }{N_{1}}\right) }\text{ \ }\left( \text{since }%
\widehat{V}\text{ is an orthogonal matrix}\right)
\end{eqnarray*}%
\begin{equation*}
=\sqrt{\lambda _{\max }\left( \frac{\Gamma ^{\prime }\Gamma }{N_{1}}\right) }%
\leq \overline{C}<\infty \text{ for }N_{1},N_{2}\text{ sufficiently large\ }
\end{equation*}%
Now, to show the second result in part (c), note that, since $Q=\left(
\Gamma ^{\prime }\Gamma /N_{1}\right) ^{\frac{{\Large 1}}{{\Large 2}}}\Xi 
\widehat{V}$ and $G_{1}=\left( \Gamma /\sqrt{N_{1}}\right) \left( \Gamma
^{\prime }\Gamma /N_{1}\right) ^{-1/2}\Xi =\Gamma \left( \Gamma ^{\prime
}\Gamma \right) ^{-1/2}\Xi $ , we can write%
\begin{eqnarray*}
\frac{\widehat{V}^{\prime }\widehat{G}_{1}^{\prime }\Gamma }{\sqrt{N_{1}}}%
-Q^{\prime } &=&\frac{\widehat{V}^{\prime }\widehat{G}_{1}^{\prime }\Gamma }{%
\sqrt{N_{1}}}-\widehat{V}^{\prime }\Xi ^{\prime }\left( \frac{\Gamma
^{\prime }\Gamma }{N_{1}}\right) ^{\frac{{\Large 1}}{{\Large 2}}}=\frac{%
\widehat{V}^{\prime }\widehat{G}_{1}^{\prime }\Gamma }{\sqrt{N_{1}}}-%
\widehat{V}^{\prime }\Xi ^{\prime }\left( \frac{\Gamma ^{\prime }\Gamma }{%
N_{1}}\right) ^{-1/2}\frac{\Gamma ^{\prime }\Gamma }{N_{1}} \\
&=&\frac{\widehat{V}^{\prime }\widehat{G}_{1}^{\prime }\Gamma }{\sqrt{N_{1}}}%
-\frac{\widehat{V}^{\prime }G_{1}^{\prime }\Gamma }{\sqrt{N_{1}}}=\widehat{V}%
^{\prime }\left( \widehat{G}_{1}-G_{1}\right) ^{\prime }\frac{\Gamma }{\sqrt{%
N_{1}}}
\end{eqnarray*}%
from which it follows that%
\begin{eqnarray*}
\left\Vert \frac{\widehat{V}^{\prime }\widehat{G}_{1}^{\prime }\Gamma }{%
\sqrt{N_{1}}}-Q^{\prime }\right\Vert _{2} &\leq &\left\Vert \widehat{V}%
^{\prime }\right\Vert _{2}\left\Vert \left( \widehat{G}_{1}-G_{1}\right)
^{\prime }\right\Vert _{2}\left\Vert \frac{\Gamma }{\sqrt{N_{1}}}\right\Vert
_{2} \\
&=&\sqrt{\lambda _{\max }\left( \widehat{V}\widehat{V}^{\prime }\right) }%
\sqrt{\lambda _{\max }\left\{ \left( \widehat{G}_{1}-G_{1}\right) \left( 
\widehat{G}_{1}-G_{1}\right) ^{\prime }\right\} }\sqrt{\lambda _{\max
}\left( \frac{\Gamma ^{\prime }\Gamma }{N_{1}}\right) } \\
&=&\sqrt{\lambda _{\max }\left( I_{Kp}\right) }\sqrt{\lambda _{\max }\left\{
\left( \widehat{G}_{1}-G_{1}\right) ^{\prime }\left( \widehat{G}%
_{1}-G_{1}\right) \right\} }\sqrt{\lambda _{\max }\left( \frac{\Gamma
^{\prime }\Gamma }{N_{1}}\right) } \\
&\leq &\sqrt{\overline{C}}\left\Vert \widehat{G}_{1}-G_{1}\right\Vert _{2}%
\text{ \ }\left( \text{by Assumption OA-3}\right) \\
&=&o_{p}\left( 1\right) \text{ \ as }N_{1},\text{ }N_{2}\text{, and }%
T\rightarrow \infty \text{\ }\left( \text{by part (b) of Lemma OA-1}\right) 
\text{.}
\end{eqnarray*}

Next, to show part (d), we first write%
\begin{eqnarray}
&&\left\Vert \frac{G_{1}^{\prime }U_{t,N}\left( \widehat{H^{c}}\right) }{%
\sqrt{N_{1}}}\right\Vert _{2}^{2}  \notag \\
&=&\dsum\limits_{k=1}^{Kp}\left( \dsum\limits_{i\in H^{{\large c}}}\mathbb{I}%
\left\{ i\in \widehat{H^{c}}\right\} \frac{g_{1,ik}u_{i,t}}{\sqrt{N_{1}}}%
+\dsum\limits_{i\in H}\mathbb{I}\left\{ i\in \widehat{H^{c}}\right\} \frac{%
g_{1,ik}u_{i,t}}{\sqrt{N_{1}}}\right) ^{2}  \notag \\
&\leq &2\dsum\limits_{k=1}^{Kp}\left( \dsum\limits_{i\in H^{{\large c}}}%
\mathbb{I}\left\{ i\in \widehat{H^{c}}\right\} \frac{g_{1,ik}u_{i,t}}{\sqrt{%
N_{1}}}\right) ^{2}+2\dsum\limits_{k=1}^{Kp}\left( \dsum\limits_{i\in H}%
\mathbb{I}\left\{ i\in \widehat{H^{c}}\right\} \frac{g_{1,ik}u_{i,t}}{\sqrt{%
N_{1}}}\right) ^{2}  \label{GU/N1^0.5}
\end{eqnarray}%
where $g_{1,ik}$ denotes the $\left( i,k\right) ^{th}$ element of $G_{1}$.
Now, consider the first term on the right-hand side of expression (\ref%
{GU/N1^0.5}). Write%
\begin{equation*}
2\dsum\limits_{k=1}^{Kp}\left( \dsum\limits_{i\in H^{{\large c}}}\mathbb{I}%
\left\{ i\in \widehat{H^{c}}\right\} \frac{g_{1,ik}u_{i,t}}{\sqrt{N_{1}}}%
\right) ^{2}
\end{equation*}%
\begin{eqnarray*}
&=&\frac{2}{N_{1}}\dsum\limits_{k=1}^{Kp}\left( \dsum\limits_{i\in H^{%
{\large c}}}\left[ \mathbb{I}\left\{ i\in \widehat{H^{c}}\right\} -1+1\right]
g_{1,ik}u_{i,t}\right) ^{2} \\
&=&\frac{2}{N_{1}}\dsum\limits_{k=1}^{Kp}\left( \dsum\limits_{i\in H^{%
{\large c}}}\left( \mathbb{I}\left\{ i\in \widehat{H^{c}}\right\} -1\right)
g_{1,ik}u_{i,t}\right) ^{2}+\frac{2}{N_{1}}\dsum\limits_{k=1}^{Kp}\left(
\dsum\limits_{i\in H^{{\large c}}}g_{1,ik}u_{i,t}\right) ^{2} \\
&&+\frac{2}{N_{1}}\dsum\limits_{k=1}^{Kp}\dsum\limits_{i\in H^{{\large c}%
}}\left( \mathbb{I}\left\{ i\in \widehat{H^{c}}\right\} -1\right)
g_{1,ik}u_{i,t}\dsum\limits_{j\in H^{{\large c}}}g_{1,jk}u_{j,t} \\
&&+\frac{2}{N_{1}}\dsum\limits_{k=1}^{Kp}\dsum\limits_{i\in H^{{\large c}%
}}g_{1,ik}u_{i,t}\dsum\limits_{j\in H^{{\large c}}}\left( \mathbb{I}\left\{
j\in \widehat{H^{c}}\right\} -1\right) g_{1,jk}u_{j,t} \\
&=&\mathcal{E}_{1,1,t}+\mathcal{E}_{1,2,t}+\mathcal{E}_{1,3,t}+\mathcal{E}%
_{1,4,t},\text{ }\left( say\right) \text{.}
\end{eqnarray*}

Focusing first on the term $\mathcal{E}_{1,1,t}$, we have%
\begin{eqnarray*}
\mathcal{E}_{1,1,t} &=&\frac{2}{N_{1}}\dsum\limits_{k=1}^{Kp}\left(
\dsum\limits_{i\in H^{{\large c}}}\left( \mathbb{I}\left\{ i\in \widehat{%
H^{c}}\right\} -1\right) g_{1,ik}u_{i,t}\right) ^{2} \\
&=&\frac{2}{N_{1}}\dsum\limits_{k=1}^{Kp}\left( \left\vert
\dsum\limits_{i\in H^{{\large c}}}\left( \mathbb{I}\left\{ i\in \widehat{%
H^{c}}\right\} -1\right) g_{1,ik}u_{i,t}\right\vert \right) ^{2} \\
&\leq &2\dsum\limits_{k=1}^{Kp}\left( \frac{1}{N_{1}}\dsum\limits_{i\in H^{%
{\large c}}}\left( \mathbb{I}\left\{ i\in \widehat{H^{c}}\right\} -1\right)
^{2}\right) \left( \dsum\limits_{i\in H^{{\large c}}}g_{1,ik}^{2}u_{i,t}^{2}%
\right) \text{ }\left( \text{by CS inequality}\right) \\
&=&2\dsum\limits_{k=1}^{Kp}\left[ \frac{1}{N_{1}}\dsum\limits_{i\in H^{%
{\large c}}}\left( 1-\mathbb{I}\left\{ i\in \widehat{H^{c}}\right\} \right) %
\right] \left( \dsum\limits_{i\in H^{{\large c}}}g_{1,ik}^{2}u_{i,t}^{2}%
\right)
\end{eqnarray*}%
Now, for either the case where $\mathbb{S}_{i,T}^{+}=\dsum\nolimits_{\ell
=1}^{d}\varpi _{\ell }\left\vert S_{i,\ell ,T}\right\vert $ or the case
where $\mathbb{S}_{i,T}^{+}=\max_{1\leq \ell \leq d}\left\vert S_{i,\ell
,T}\right\vert $, we have%
\begin{eqnarray*}
0 &\leq &E\left[ \frac{1}{N_{1}}\dsum\limits_{i\in H^{{\large c}}}\left( 1-%
\mathbb{I}\left\{ i\in \widehat{H^{c}}\right\} \right) \right] =\frac{1}{%
N_{1}}\dsum\limits_{i\in H^{{\large c}}}\left[ 1-\Pr \left( i\in \widehat{%
H^{c}}\right) \right] \\
&=&\frac{1}{N_{1}}\dsum\limits_{i\in H^{{\large c}}}\left[ 1-P\left( \mathbb{%
S}_{i,T}^{+}\geq \Phi ^{-1}\left( 1-\frac{\varphi }{2N}\right) \right) %
\right] \\
&\leq &1-P\left( \min_{i\in H^{{\large c}}}\mathbb{S}_{i,T}^{+}\geq \Phi
^{-1}\left( 1-\frac{\varphi }{2N}\right) \right) \text{ }\left( \text{given
that }N_{1}=\#\left\{ H^{c}\right\} \right) \\
&\rightarrow &0\text{ }\left( \text{since }P\left( \min_{i\in H^{{\large c}}}%
\mathbb{S}_{i,T}^{+}\geq \Phi ^{-1}\left( 1-\frac{\varphi }{2N}\right)
\right) \rightarrow 1\text{ by Theorem 2 of the main paper}\right) \text{.}
\end{eqnarray*}%
Moreover, by part (b) of Assumption 2-3, we have $E\left[ \dsum\nolimits_{i%
\in H^{{\large c}}}g_{1,ik}^{2}u_{i,t}^{2}\right] =\dsum\nolimits_{i\in H^{%
{\large c}}}g_{1,ik}^{2}E\left[ u_{i,t}^{2}\right] \leq
C\dsum\nolimits_{i=1}^{N}g_{1,ik}^{2}\leq C$. It follows by Markov's
inequality that $\frac{1}{N_{1}}\dsum\nolimits_{i\in H^{{\large c}}}\left( 1-%
\mathbb{I}\left\{ i\in \widehat{H^{c}}\right\} \right) =o_{p}\left( 1\right) 
$ and $\dsum\nolimits_{i\in H^{{\large c}}}g_{1,ik}^{2}u_{i,t}^{2}=O_{p}%
\left( 1\right) $ from which we deduce that

\noindent $\mathcal{E}_{1,1,t}=2N_{1}^{-1}\dsum\nolimits_{k=1}^{Kp}\left(
\dsum\nolimits_{i\in H^{{\large c}}}\left( \mathbb{I}\left\{ i\in \widehat{%
H^{c}}\right\} -1\right) g_{1,ik}u_{i,t}\right) ^{2}$

\noindent $\leq 2\dsum\nolimits_{k=1}^{Kp}\left[ N_{1}^{-1}\dsum\nolimits_{i%
\in H^{{\large c}}}\left( 1-\mathbb{I}\left\{ i\in \widehat{H^{c}}\right\}
\right) \right] \left( \dsum\nolimits_{i\in H^{{\large c}%
}}g_{1,ik}^{2}u_{i,t}^{2}\right) =o_{p}\left( 1\right) $. Consider next the
term $\mathcal{E}_{1,2,t}$. To proceed, let $U_{t,N}\left( H^{c}\right) $
denote an $N\times 1$ vector whose $i^{th}$ component $U_{i,t,N}\left(
H^{c}\right) $ is given by $U_{i,t,N}\left( H^{c}\right) =u_{i,t}\mathbb{I}%
\left\{ i\in H^{c}\right\} $, and we can write%
\begin{eqnarray*}
\mathcal{E}_{1,2,t} &=&\frac{2}{N_{1}}\dsum\limits_{k=1}^{Kp}\left(
\dsum\limits_{i\in H^{{\large c}}}g_{1,ik}u_{i,t}\right) ^{2}=2\left\Vert 
\frac{G_{1}^{\prime }U_{t,N}\left( H^{c}\right) }{\sqrt{N_{1}}}\right\Vert
_{2}^{2} \\
&\leq &2tr\left\{ \frac{G_{1}^{\prime }U_{t,N}\left( H^{c}\right)
U_{t,N}\left( H^{c}\right) ^{\prime }G_{1}}{N_{1}}\right\} \\
&=&2tr\left\{ \Xi ^{\prime }\left( \frac{\Gamma ^{\prime }\Gamma }{N_{1}}%
\right) ^{-1/2}\frac{\Gamma ^{\prime }}{\sqrt{N_{1}}}\frac{U_{t,N}\left(
H^{c}\right) U_{t,N}\left( H^{c}\right) ^{\prime }}{N_{1}}\frac{\Gamma }{%
\sqrt{N_{1}}}\left( \frac{\Gamma ^{\prime }\Gamma }{N_{1}}\right) ^{-1/2}\Xi
\right\} \\
&=&2tr\left\{ \left( \frac{\Gamma ^{\prime }\Gamma }{N_{1}}\right) ^{-1/2}%
\frac{\Gamma ^{\prime }}{\sqrt{N_{1}}}\frac{U_{t,N}\left( H^{c}\right)
U_{t,N}\left( H^{c}\right) ^{\prime }}{N_{1}}\frac{\Gamma }{\sqrt{N_{1}}}%
\left( \frac{\Gamma ^{\prime }\Gamma }{N_{1}}\right) ^{-1/2}\right\} \\
&=&2tr\left\{ \frac{\Gamma _{\ast }^{\prime }U_{t,N}\left( H^{c}\right)
U_{t,N}\left( H^{c}\right) ^{\prime }\Gamma _{\ast }}{N_{1}^{2}}\right\} 
\text{ }\left( \text{letting }\Gamma _{\ast }=\Gamma \left( \frac{\Gamma
^{\prime }\Gamma }{N_{1}}\right) ^{-1/2}\right) \\
&=&\frac{2}{N_{1}^{2}}U_{t,N}\left( H^{c}\right) ^{\prime }\Gamma _{\ast
}\Gamma _{\ast }^{\prime }U_{t,N}\left( H^{c}\right) =\frac{2}{N_{1}^{2}}%
\dsum\limits_{i\in H^{{\large c}}}\dsum\limits_{j\in H^{{\large c}}}\gamma
_{\ast ,i}^{\prime }\gamma _{\ast ,j}u_{i,t}u_{j,t}
\end{eqnarray*}%
where $\gamma _{\ast ,i}^{\prime }$ denotes the $i^{th}$ row of $\Gamma
_{\ast }=\Gamma \left( \Gamma ^{\prime }\Gamma /N_{1}\right) ^{-1/2}$.
Hence, by Assumptions 2-5, OA-2, and OA-3, there exist positive constants $%
\overline{c}$, $\underline{C}$, and $\overline{C}$ such that%
\begin{eqnarray*}
0 &\leq &E\left[ \mathcal{E}_{1,2,t}\right] \\
&=&\frac{2}{N_{1}}\dsum\limits_{k=1}^{Kp}\dsum\limits_{i\in H^{{\large c}%
}}\dsum\limits_{j\in H^{{\large c}}}g_{1,ik}g_{1,jk}E\left[ u_{i,t}u_{j,t}%
\right] \\
&=&\frac{2}{N_{1}^{2}}\dsum\limits_{i\in H^{{\large c}}}\dsum\limits_{j\in
H^{{\large c}}}\gamma _{\ast ,i}^{\prime }\gamma _{\ast ,j}E\left[
u_{i,t}u_{j,t}\right] =\frac{2}{N_{1}^{2}}\dsum\limits_{i\in H^{{\large c}%
}}\dsum\limits_{j\in H^{{\large c}}}\gamma _{i}^{\prime }\left( \frac{\Gamma
^{\prime }\Gamma }{N_{1}}\right) ^{-1/2}\left( \frac{\Gamma ^{\prime }\Gamma 
}{N_{1}}\right) ^{-1/2}\gamma _{j}E\left[ u_{i,t}u_{j,t}\right] \\
&\leq &\frac{2}{N_{1}^{2}}\dsum\limits_{i\in H^{{\large c}%
}}\dsum\limits_{j\in H^{{\large c}}}\left\vert \gamma _{i}^{\prime }\left( 
\frac{\Gamma ^{\prime }\Gamma }{N_{1}}\right) ^{-1/2}\left( \frac{\Gamma
^{\prime }\Gamma }{N_{1}}\right) ^{-1/2}\gamma _{j}\right\vert \left\vert E%
\left[ u_{i,t}u_{j,t}\right] \right\vert \\
&\leq &\frac{2}{N_{1}^{2}}\dsum\limits_{i\in H^{{\large c}%
}}\dsum\limits_{j\in H^{{\large c}}}\sqrt{\gamma _{i}^{\prime }\left( \frac{%
\Gamma ^{\prime }\Gamma }{N_{1}}\right) ^{-1}\gamma _{i}}\sqrt{\gamma
_{i}^{\prime }\left( \frac{\Gamma ^{\prime }\Gamma }{N_{1}}\right)
^{-1}\gamma _{i}}\left\vert E\left[ u_{i,t}u_{j,t}\right] \right\vert
\end{eqnarray*}%
\begin{eqnarray*}
&\leq &2\overline{C}\frac{1}{N_{1}^{2}}\dsum\limits_{i\in H^{{\large c}%
}}\dsum\limits_{j\in H^{{\large c}}}\left\vert E\left[ u_{i,t}u_{j,t}\right]
\right\vert \text{ \ }\left( \text{by Assumption 2-5 and OA-3}\right) \\
&\leq &2\overline{C}\frac{C}{N_{1}}\text{ }\rightarrow 0\text{ as }%
N_{1}\rightarrow \infty \text{ }\left( \text{by Assumption OA-2}\right) 
\text{.}
\end{eqnarray*}%
It follows by Markov's inequality that $\mathcal{E}_{1,2,t}=o_{p}\left(
1\right) $. Now, for $\mathcal{E}_{1,3,t}$, note that, by applying the
triangle and CS inequalities, we have

\noindent $\left\vert \mathcal{E}_{1,3,t}\right\vert \leq
2N_{1}^{-1}\dsum\nolimits_{k=1}^{Kp}\left\vert \dsum\nolimits_{i\in H^{%
{\large c}}}\left( \mathbb{I}\left\{ i\in \widehat{H^{c}}\right\} -1\right)
g_{1,ik}u_{i,t}\dsum\nolimits_{j\in H^{{\large c}}}g_{1,jk}u_{j,t}\right%
\vert $

\noindent $\leq \sqrt{2N_{1}^{-1}\dsum\nolimits_{k=1}^{Kp}\left[
\dsum\nolimits_{i\in H^{{\large c}}}\left( \mathbb{I}\left\{ i\in \widehat{%
H^{c}}\right\} -1\right) g_{1,ik}u_{i,t}\right] ^{2}}\sqrt{%
2N_{1}^{-1}\dsum\nolimits_{k=1}^{Kp}\left[ \dsum\nolimits_{j\in H^{{\large c}%
}}g_{1,jk}u_{j,t}\right] ^{2}}$

\noindent $=\sqrt{\mathcal{E}_{1,1,t}}\sqrt{\mathcal{E}_{1,2,t}}=o_{p}\left(
1\right) $. Similarly, we also have

\noindent $\left\vert \mathcal{E}_{1,4,t}\right\vert \leq
2N_{1}^{-1}\dsum\nolimits_{k=1}^{Kp}\left\vert \dsum\nolimits_{i\in H^{%
{\large c}}}g_{1,ik}u_{i,t}\dsum\nolimits_{j\in H^{{\large c}}}\left( 
\mathbb{I}\left\{ j\in \widehat{H^{c}}\right\} -1\right)
g_{1,jk}u_{j,t}\right\vert $

\noindent $\leq \sqrt{\mathcal{E}_{1,2,t}}\sqrt{\mathcal{E}_{1,1,t}}%
=o_{p}\left( 1\right) $.

\noindent \qquad Application of the Slutsky's theorem then allows us to
deduce that%
\begin{equation*}
\frac{2}{N_{1}}\dsum\limits_{k=1}^{Kp}\dsum\limits_{i\in H^{{\large c}%
}}\dsum\limits_{j\in H^{{\large c}}}\mathbb{I}\left\{ i\in \widehat{H^{c}}%
\right\} \mathbb{I}\left\{ j\in \widehat{H^{c}}\right\}
g_{1,ik}g_{1,jk}u_{i,t}u_{j,t}=\mathcal{E}_{1,1,t}+\mathcal{E}_{1,2,t}+%
\mathcal{E}_{1,3,t}+\mathcal{E}_{1,4,t}=o_{p}\left( 1\right) \text{.}
\end{equation*}

Next, consider the second term on the right-hand side of expression (\ref%
{GU/N1^0.5}). In this case, write%
\begin{eqnarray*}
&&\frac{2}{N_{1}}\dsum\limits_{k=1}^{Kp}\dsum\limits_{i\in
H}\dsum\limits_{j\in H}\mathbb{I}\left\{ i\in \widehat{H^{c}}\right\} 
\mathbb{I}\left\{ j\in \widehat{H^{c}}\right\} g_{1,ik}g_{1,jk}u_{i,t}u_{j,t}
\\
&=&\frac{2}{N_{1}}\dsum\limits_{k=1}^{Kp}\left( \dsum\limits_{i\in H}\mathbb{%
I}\left\{ i\in \widehat{H^{c}}\right\} g_{1,ik}u_{i,t}\right) ^{2}=\frac{2}{%
N_{1}}\dsum\limits_{k=1}^{Kp}\left( \left\vert \dsum\limits_{i\in H}\mathbb{I%
}\left\{ i\in \widehat{H^{c}}\right\} g_{1,ik}u_{i,t}\right\vert \right) ^{2}
\\
&\leq &2\dsum\limits_{k=1}^{Kp}\left[ \frac{1}{N_{1}}\dsum\limits_{i\in H}%
\mathbb{I}\left\{ i\in \widehat{H^{c}}\right\} \right] \left[
\dsum\limits_{i\in H}g_{1,ik}^{2}u_{i,t}^{2}\right]
\end{eqnarray*}%
Note that, for either the case where $\mathbb{S}_{i,T}^{+}=\dsum\nolimits_{%
\ell =1}^{d}\varpi _{\ell }\left\vert S_{i,\ell ,T}\right\vert $ or the case
where $\mathbb{S}_{i,T}^{+}=\max_{1\leq \ell \leq d}\left\vert S_{i,\ell
,T}\right\vert $, we have, by applying an argument similar to that given in
the proof of Theorem 1 of the main paper,%
\begin{eqnarray*}
0 &\leq &E\left[ \frac{1}{N_{1}}\dsum\limits_{i\in H}\mathbb{I}\left\{ i\in 
\widehat{H^{c}}\right\} \right] =\frac{1}{N_{1}}\dsum\limits_{i\in H}\Pr
\left( i\in \widehat{H^{c}}\right) =\frac{1}{N_{1}}\dsum\limits_{i\in
H}P\left( \mathbb{S}_{i,T}^{+}\geq \Phi ^{-1}\left( 1-\frac{\varphi }{2N}%
\right) \right) \\
&\leq &\frac{dN_{2}\varphi }{NN_{1}}\left\{ 1+2^{2}A\left[ 1+\left( \Phi
^{-1}\left( 1-\frac{\varphi }{2N}\right) \right) ^{3}\right] T^{-\left(
1-\alpha _{{\large 1}}\right) \frac{{\large 1}}{{\large 2}}}\right\} =\frac{%
dN_{2}\varphi }{N_{1}\left( N_{1}+N_{2}\right) }\left[ 1+o\left( 1\right) %
\right] \rightarrow 0\text{, }
\end{eqnarray*}%
as $N_{1},N_{2},T\rightarrow \infty $. Moreover, making use of part (b) of
Assumption 2-3, we have

\noindent $E\left[ \dsum\nolimits_{i\in H}g_{1,ik}^{2}u_{i,t}^{2}\right]
=\dsum\nolimits_{i\in H}g_{1,ik}^{2}E\left[ u_{i,t}^{2}\right] \leq
C\dsum\nolimits_{i=1}^{N}g_{1,ik}^{2}\leq C$. It follows by Markov's
inequality that

\noindent $N_{1}^{-1}\dsum\nolimits_{i\in H}\mathbb{I}\left\{ i\in \widehat{%
H^{c}}\right\} =o_{p}\left( 1\right) $ and $\dsum\nolimits_{i\in
H}g_{1,ik}^{2}u_{i,t}^{2}=O_{p}\left( 1\right) $, from which we deduce that $%
2N^{-1}\dsum\nolimits_{k=1}^{Kp}\dsum\nolimits_{i\in H}\dsum\nolimits_{j\in
H}\mathbb{I}\left\{ i\in \widehat{H^{c}}\right\} \mathbb{I}\left\{ j\in 
\widehat{H^{c}}\right\} g_{1,ik}g_{1,jk}u_{i,t}u_{j,t}$

\noindent $\leq 2\dsum\nolimits_{k=1}^{Kp}\left[ N^{-1}\dsum\nolimits_{i\in
H}\mathbb{I}\left\{ i\in \widehat{H^{c}}\right\} \right] \left[
\dsum\nolimits_{i\in H}g_{1,ik}^{2}u_{i,t}^{2}\right] =o_{p}\left( 1\right) $%
.

Combining these results and using the inequality $\sqrt{a_{1}+a_{2}}\leq 
\sqrt{a_{1}}+\sqrt{a_{2}}$ for $a_{1}\geq 0$ and $a_{2}\geq 0$, we further
obtain, for all $t$,%
\begin{eqnarray*}
\left\Vert \frac{G_{1}^{\prime }U_{t,N}\left( \widehat{H^{c}}\right) }{\sqrt{%
N_{1}}}\right\Vert _{2} &\leq &\sqrt{\frac{2}{N_{1}}\dsum\limits_{k=1}^{K}%
\dsum\limits_{i\in H^{{\large c}}}\dsum\limits_{j\in H^{{\large c}}}\mathbb{I%
}\left\{ i\in \widehat{H^{c}}\right\} \mathbb{I}\left\{ j\in \widehat{H^{c}}%
\right\} g_{1,ik}g_{1,jk}u_{i,t}u_{j,t}} \\
&&+\sqrt{\frac{2}{N_{1}}\dsum\limits_{k=1}^{K}\dsum\limits_{i\in
H}\dsum\limits_{j\in H}\mathbb{I}\left\{ i\in \widehat{H^{c}}\right\} 
\mathbb{I}\left\{ j\in \widehat{H^{c}}\right\} g_{1,ik}g_{1,jk}u_{i,t}u_{j,t}%
} \\
&=&o_{p}\left( 1\right) \text{.}
\end{eqnarray*}

For part (e), write%
\begin{eqnarray*}
\left\Vert U_{t,N}\left( \widehat{H^{c}}\right) /\sqrt{N_{1}}\right\Vert
_{2}^{2} &=&N_{1}^{-1}\dsum\nolimits_{i=1}^{N}\mathbb{I}\left\{ i\in 
\widehat{H^{c}}\right\} u_{i,t}^{2} \\
&=&N_{1}^{-1}\dsum\nolimits_{i\in H^{{\large c}}}\mathbb{I}\left\{ i\in 
\widehat{H^{c}}\right\} u_{i,t}^{2}+N_{1}^{-1}\dsum\nolimits_{i\in H}\mathbb{%
I}\left\{ i\in \widehat{H^{c}}\right\} u_{i,t}^{2} \\
&\leq &N_{1}^{-1}\dsum\nolimits_{i\in H^{{\large c}}}u_{i,t}^{2}+N_{1}^{-1}%
\dsum\nolimits_{i\in H}\mathbb{I}\left\{ i\in \widehat{H^{c}}\right\}
u_{i,t}^{2}
\end{eqnarray*}%
Note that, by Assumption 2-3(b), $E\left[ N_{1}^{-1}\dsum\nolimits_{i\in H^{%
{\large c}}}u_{i,t}^{2}\right] =N_{1}^{-1}\dsum\nolimits_{i\in H^{{\large c}%
}}E\left[ u_{i,t}^{2}\right] \leq C$

\noindent $\left( \text{since }N_{1}=\#\left\{ H^{c}\right\} \right) $, so
that, by applying Markov's inequality, we obtain $N_{1}^{-1}\dsum%
\nolimits_{i\in H^{{\large c}}}u_{i,t}^{2}=O_{p}\left( 1\right) $. Moreover,
note that, for any $\epsilon >0$, $\dbigcap\nolimits_{i\in H}\left\{ i\notin 
\widehat{H^{c}}\right\} \subseteq \left\{ N_{1}^{-1}\dsum\nolimits_{i\in H}%
\mathbb{I}\left\{ i\in \widehat{H^{c}}\right\} u_{i,t}^{2}<\epsilon \right\} 
$, so that by DeMorgan's law $\left\{ N_{1}^{-1}\dsum\nolimits_{i\in H}%
\mathbb{I}\left\{ i\in \widehat{H^{c}}\right\} u_{i,t}^{2}\geq \epsilon
\right\} \subseteq \left\{ \dbigcap\nolimits_{i\in H}\left\{ i\notin 
\widehat{H^{c}}\right\} \right\} ^{c}$

\noindent $=\dbigcup\nolimits_{i\in H}\left\{ i\in \widehat{H^{c}}\right\} $%
. Hence, for either the case where $\mathbb{S}_{i,T}^{+}=\dsum\nolimits_{%
\ell =1}^{d}\varpi _{\ell }\left\vert S_{i,\ell ,T}\right\vert $ or the case
where $\mathbb{S}_{i,T}^{+}=\max_{1\leq \ell \leq d}\left\vert S_{i,\ell
,T}\right\vert $, we have, by applying an argument similar to that given in
the proof of Theorem 1 of the main paper,%
\begin{equation*}
\Pr \left\{ \frac{1}{N_{1}}\dsum\limits_{i\in H}\mathbb{I}\left\{ i\in 
\widehat{H^{c}}\right\} u_{i,t}^{2}\geq \epsilon \right\} \leq \Pr \left\{
\dbigcup\limits_{i\in H}\left\{ i\in \widehat{H^{c}}\right\} \right\}
\end{equation*}%
\begin{eqnarray*}
&\leq &\dsum\limits_{i\in H}\Pr \left\{ i\in \widehat{H^{c}}\right\}
=\dsum\limits_{i\in H}P\left( \mathbb{S}_{i,T}^{+}\geq \Phi ^{-1}\left( 1-%
\frac{\varphi }{2N}\right) \right) \\
&\leq &\frac{dN_{2}\varphi }{N}\left\{ 1+2^{2}A\left[ 1+\left( \Phi
^{-1}\left( 1-\frac{\varphi }{2N}\right) \right) ^{3}\right] T^{-\left(
1-\alpha _{{\large 1}}\right) \frac{{\large 1}}{{\large 2}}}\right\} \\
&=&\frac{dN_{2}\varphi }{N_{1}+N_{2}}\left[ 1+o\left( 1\right) \right]
\rightarrow 0\text{ as }N_{1},N_{2},T\rightarrow \infty
\end{eqnarray*}%
Hence, $N_{1}^{-1}\dsum\nolimits_{i\in H}\mathbb{I}\left\{ i\in \widehat{%
H^{c}}\right\} u_{i,t}^{2}=o_{p}\left( 1\right) $ from which it further
follows that

\noindent $\left\Vert U_{t,N}\left( \widehat{H^{c}}\right) /\sqrt{N_{1}}%
\right\Vert _{2}^{2}\leq \frac{1}{N_{1}}\dsum\limits_{i\in H^{{\large c}%
}}u_{i,t}^{2}+\frac{1}{N_{1}}\dsum\limits_{i\in H}\mathbb{I}\left\{ i\in 
\widehat{H^{c}}\right\} u_{i,t}^{2}=O_{p}\left( 1\right) +o_{p}\left(
1\right) =O_{p}\left( 1\right) $.

Turning our attention to part (f), note first that since $G=\left[ 
\begin{array}{cc}
G_{1} & G_{2}%
\end{array}%
\right] $ is an orthogonal matrix, we have $I_{N}=GG^{\prime
}=G_{1}G_{1}^{\prime }+G_{2}G_{2}^{\prime }$ or $G_{2}G_{2}^{\prime
}=I_{N}-G_{1}G_{1}^{\prime }$. Hence, we can write $\left\Vert G_{2}^{\prime
}U_{t,N}\left( \widehat{H^{c}}\right) /\sqrt{N_{1}}\right\Vert
_{2}^{2}=U_{t,N}\left( \widehat{H^{c}}\right) ^{\prime }U_{t,N}\left( 
\widehat{H^{c}}\right) /N_{1}-U_{t,N}\left( \widehat{H^{c}}\right) ^{\prime
}G_{1}G_{1}^{\prime }U_{t,N}\left( \widehat{H^{c}}\right) /N_{1}$

\noindent $\leq \left\Vert U_{t,N}\left( \widehat{H^{c}}\right) /\sqrt{N_{1}}%
\right\Vert _{2}^{2}+\left\Vert G_{1}^{\prime }U_{t,N}\left( \widehat{H^{c}}%
\right) /\sqrt{N_{1}}\right\Vert _{2}^{2}$. Applying the results from parts
(d) and (e) above, we then obtain $\left\Vert G_{2}^{\prime }U_{t,N}\left( 
\widehat{H^{c}}\right) /\sqrt{N_{1}}\right\Vert _{2}^{2}\leq \left\Vert
U_{t,N}\left( \widehat{H^{c}}\right) /\sqrt{N_{1}}\right\Vert
_{2}^{2}+\left\Vert G_{1}^{\prime }U_{t,N}\left( \widehat{H^{c}}\right) /%
\sqrt{N_{1}}\right\Vert _{2}^{2}$

\noindent $=O_{p}\left( 1\right) +o_{p}\left( 1\right) =O_{p}\left( 1\right) 
$, from which it follows by Slutsky't theorem that

\noindent $\left\Vert G_{2}^{\prime }U_{t,N}\left( \widehat{H^{c}}\right) /%
\sqrt{N_{1}}\right\Vert _{2}=O_{p}\left( 1\right) $.

Now, to show part (g), first write $\widehat{V}^{\prime }\widehat{G}_{1}^{%
{\Large \prime }}U_{t,N}\left( \widehat{H^{c}}\right) /\sqrt{\widehat{N}_{1}}
$

\noindent $=\left( 1+\left[ \widehat{N}_{1}-N_{1}\right] /N_{1}\right) ^{-%
\frac{{\large 1}}{{\large 2}}}\widehat{V}^{\prime }\widehat{G}_{1}^{{\Large %
\prime }}U_{t,N}\left( \widehat{H^{c}}\right) /\sqrt{N_{1}}$. Using the fact
that $\widehat{V}^{\prime }\widehat{V}=I_{Kp}$ so that $\left\Vert \widehat{V%
}\right\Vert _{2}=1$, we have%
\begin{eqnarray*}
\left\Vert \frac{\widehat{V}^{\prime }\widehat{G}_{1}^{{\Large \prime }%
}U_{t,N}\left( \widehat{H^{c}}\right) }{\sqrt{N_{1}}}\right\Vert _{2}
&=&\left\Vert \frac{\widehat{V}^{\prime }\left( I_{Kp}+R^{\prime }R\right)
^{-1/2}\left[ G_{1}^{\prime }U_{t,N}\left( \widehat{H^{c}}\right) +R^{\prime
}G_{2}^{\prime }U_{t,N}\left( \widehat{H^{c}}\right) \right] }{\sqrt{N_{1}}}%
\right\Vert _{2} \\
&\leq &\left\Vert \widehat{V}\right\Vert _{2}\left\Vert \left(
I_{Kp}+R^{\prime }R\right) ^{-\frac{1}{2}}\right\Vert _{2}\left\{ \left\Vert 
\frac{G_{1}^{\prime }U_{t,N}\left( \widehat{H^{c}}\right) }{\sqrt{N_{1}}}%
\right\Vert _{2}+\left\Vert R\right\Vert _{2}\left\Vert \frac{G_{2}^{\prime
}U_{t,N}\left( \widehat{H^{c}}\right) }{\sqrt{N_{1}}}\right\Vert _{2}\right\}
\\
&\leq &\frac{1}{\sqrt[\backslash ]{1+\lambda _{\min }\left( R^{\prime
}R\right) }}\left\{ \left\Vert \frac{G_{1}^{\prime }U_{t,N}\left( \widehat{%
H^{c}}\right) }{\sqrt{N_{1}}}\right\Vert _{2}+\left\Vert R\right\Vert
_{2}\left\Vert \frac{G_{2}^{\prime }U_{t,N}\left( \widehat{H^{c}}\right) }{%
\sqrt{N_{1}}}\right\Vert _{2}\right\} \text{.}
\end{eqnarray*}%
It follows that%
\begin{equation*}
\left\Vert \frac{\widehat{V}^{\prime }\widehat{G}_{1}^{{\Large \prime }%
}U_{t,N}\left( \widehat{H^{c}}\right) }{\sqrt{\widehat{N}_{1}}}\right\Vert
_{2}=\left\vert \left( 1+\frac{\widehat{N}_{1}-N_{1}}{N_{1}}\right) ^{-\frac{%
{\large 1}}{{\large 2}}}\right\vert \left\Vert \frac{\widehat{V}^{\prime }%
\widehat{G}_{1}^{{\Large \prime }}U_{t,N}\left( \widehat{H^{c}}\right) }{%
\sqrt{N_{1}}}\right\Vert _{2}
\end{equation*}%
\begin{eqnarray*}
&\leq &\left\vert \left( 1+\frac{\widehat{N}_{1}-N_{1}}{N_{1}}\right) ^{-%
\frac{{\large 1}}{{\large 2}}}\right\vert \frac{1}{\sqrt[\backslash ]{%
1+\lambda _{\min }\left( R^{\prime }R\right) }}\left\{ \left\Vert \frac{%
G_{1}^{\prime }U_{t,N}\left( \widehat{H^{c}}\right) }{\sqrt{N_{1}}}%
\right\Vert _{2}+\left\Vert R\right\Vert _{2}\left\Vert \frac{G_{2}^{\prime
}U_{t,N}\left( \widehat{H^{c}}\right) }{\sqrt{N_{1}}}\right\Vert _{2}\right\}
\\
&\leq &\left\vert \left( 1+\frac{\widehat{N}_{1}-N_{1}}{N_{1}}\right) ^{-%
\frac{{\large 1}}{{\large 2}}}\right\vert \left\{ \left\Vert \frac{%
G_{1}^{\prime }U_{t,N}\left( \widehat{H^{c}}\right) }{\sqrt{N_{1}}}%
\right\Vert _{2}+\left\Vert R\right\Vert _{2}\left\Vert \frac{G_{2}^{\prime
}U_{t,N}\left( \widehat{H^{c}}\right) }{\sqrt{N_{1}}}\right\Vert
_{2}\right\} =o_{p}\left( 1\right)
\end{eqnarray*}

\noindent where the last equality follows from the fact that $\left\Vert
R\right\Vert _{2}\overset{p}{\rightarrow }0$, $\left\vert \left( 1+\left[ 
\widehat{N}_{1}-N_{1}\right] /N_{1}\right) ^{-\frac{{\large 1}}{{\large 2}}%
}\right\vert \overset{p}{\rightarrow }1$, $\left\Vert G_{1}^{\prime
}U_{t,N}\left( \widehat{H^{c}}\right) /\sqrt{N_{1}}\right\Vert _{2}\overset{p%
}{\rightarrow }0$, and $\left\Vert G_{2}^{\prime }U_{t,N}\left( \widehat{%
H^{c}}\right) /\sqrt{N_{1}}\right\Vert _{2}=O_{p}\left( 1\right) $, given
results previously provided in part (a) in Lemma OA-1 and in parts (a), (d),
and (f) of this lemma.

To show part (h), note first that, under Assumptions 2-1, 2-2(a)-(b), 2-5,
and OA-3; there exists a positive constant $\overline{C}$ such that $%
E\left\Vert \underline{F}_{t}\right\Vert _{2}^{6}\leq \overline{C}<\infty $%
\footnote{%
An explicit proof that, under Assumptions 2-1, 2-2(a)-(b), 2-5, and 2-6;
there exists some positive constant $\overline{C}$ such that $E\left\Vert 
\underline{F}_{t}\right\Vert _{2}^{6}\leq $ $\overline{C}$ is given in the
Technical Appendix of an earlier version of this paper, Chao and Swanson
(2022b). In particular, this result is shown in Lemma C-5 in Appendix C of
Chao and Swanson (2022b).}. Hence, for any $\epsilon >0$, let $C_{\epsilon }=%
\overline{C}^{\frac{{\Large 1}}{{\Large 6}}}/\sqrt{\epsilon }$; and we can
apply Markov's inequality and then Liapunov's inequality to obtain $\Pr
\left( \left\Vert \underline{F}_{t}\right\Vert _{2}\geq C_{\epsilon }\right)
\leq \Pr \left( \left\Vert \underline{F}_{t}\right\Vert _{2}^{2}\geq
C_{\epsilon }^{2}\right) \leq C_{\epsilon }^{-2}E\left\Vert \underline{F}%
_{t}\right\Vert _{2}^{2}\leq C_{\epsilon }^{-2}\left( E\left\Vert \underline{%
F}_{t}\right\Vert _{2}^{6}\right) ^{\frac{{\Large 1}}{{\Large 3}}}\leq
\epsilon \overline{C}^{-\frac{{\Large 1}}{{\Large 3}}}\overline{C}^{\frac{%
{\Large 1}}{{\Large 3}}}\leq \epsilon $, from which it follows that $%
\left\Vert \underline{F}_{t}\right\Vert _{2}=O_{p}\left( 1\right) $ for all $%
t$. $\square $

\medskip

\begin{thebibliography}{9}
\bibitem{} Bai, J. and S. Ng (2021): \textquotedblleft Approximate Factor
Models with Weaker Loading,\textquotedblright\ Working Paper, Columbia
University.

\bibitem{} Chao, J. C. and N. R. Swanson (2022a): \textquotedblleft
Consistent Estimation, Variable Selection, and Forecasting in
Factor-Augmented VAR Models," Working Paper, Rutgers University and
University of Maryland.

\bibitem{} Chao, J. C. and N. R. Swanson (2022b): Technical Appendix to
\textquotedblleft Consistent Estimation, Variable Selection, and Forecasting
in Factor-Augmented VAR Models," Working Paper, Rutgers University and
University of Maryland.

\bibitem{} Golub, G. H. and C. F. van Loan (1996): \textit{Matrix
Computations}, 3rd Edition. Baltimore: The Johns Hopkins University Press.

\bibitem{} Stewart, G.W. and J. Sun (1990): \textit{Matrix Perturbation
Theory}. Boston: Academic Press.

\bibitem{} Stock, J. H. and M. W. Watson (2002): \textquotedblleft
Forecasting Using Principal Components from a Large Number of
Predictors,\textquotedblright\ \textit{Journal of the American Statistical
Association}, 97, 1167-1179.
\end{thebibliography}

\end{document}
