\documentclass[a4paper,11pt]{article}%
\usepackage[top=1.25in, bottom=1.25in, left=1.25in, right=1.25in]{geometry}
\usepackage[utf8]{inputenc}
\usepackage[T1]{fontenc}
\usepackage{lmodern}
\usepackage[english]{babel}
\usepackage{color}
\usepackage{csvsimple}
\usepackage{tabularx}
\usepackage{subcaption}
\usepackage{tabulary}
\usepackage{array}
\newcolumntype{L}[1]{>{\raggedright\let\newline\\\arraybackslash\hspace{0pt}}m{#1}}
\newcolumntype{C}[1]{>{\centering\let\newline\\\arraybackslash\hspace{0pt}}m{#1}}
\newcolumntype{R}[1]{>{\raggedleft\let\newline\\\arraybackslash\hspace{0pt}}m{#1}}
\usepackage{booktabs}
\usepackage[flushleft]{threeparttable}
\usepackage{multirow}
\usepackage{siunitx}
\usepackage{graphicx}
\usepackage[singlelinecheck=false ]{caption}
\usepackage[font={small,it}]{caption}
\usepackage{mdframed}
\usepackage{times}
\usepackage{pdflscape}
\usepackage{adjustbox}
\usepackage{ragged2e}
\usepackage{hanging}
\usepackage[bottom]{footmisc}
\usepackage{geometry}
\geometry{
	a4paper,
	total={6in,7in},
	left=30mm,
	top=30mm,
	bottom=40mm
}
\newenvironment{sciabstract}{
\begin{quote} \bf}
{\end{quote}}

\renewcommand\refname{References and Notes}
\newcommand{\head}[1]{\textnormal{\textbf{#1}}}
\newcounter{lastnote}
\newenvironment{scilastnote}{
\setcounter{lastnote}{\value{enumiv}}
\addtocounter{lastnote}{+1}
\begin{list}
{\arabic{lastnote}.}
{\setlength{\leftmargin}{.22in}}
{\setlength{\labelsep}{.5em}}}
{\end{list}}

\begin{document}
	
\begin{center}

{\huge{New Evidence of the Marginal Predictive Content of Small and Large Jumps$^*$}}

\bigskip
\bigskip

{\Large


Bo Yu, Bruce Mizrach, and Norman R. Swanson
\bigskip 
\bigskip

Rutgers University

\bigskip

August 2019}

\end{center}	

\thispagestyle{empty}

\bigskip
\bigskip
	
\noindent {\footnotesize In an important paper, Bollerslev, Li, and Zhao (2019) find that there is significant predictive content in signed total jump variation. In this paper, we further decompose jumps into small and large (signed) variation. Our primary purpose is to investigate the marginal predictive content of small versus large jump variation, when forecasting one-week ahead cross-sectional equity returns. However, we also examine earnings announcements, in order to shed new light on the linkages between (small and large) jumps and news. One of our key findings is that sorting on signed small jump variation leads to 
greater value-weighted return differentials between stocks in our highest and lowest quintile portfolios (i.e., high-low spreads)	
than when either signed total jump or signed large jump variation is sorted on. Moreover, in a key case, the high-low spread
is not significantly different from zero when signed large jump variation is sorted on. Indeed, including large jump variation can actually decrease predictive accuracy, in the sense that average returns and alphas for high-low portfolios
are lower when total jump variation is utilized in our prediction experiments rather than small jump variation. These results suggest that there may be a threshold, beyond which ``large'' jump variation contains no marginal predictive ability, relative to that contained in small jump variation. Analysis of returns and alphas based on industry double-sorts indicates that the benefit of small signed jump variation investing is driven by stock selection within an industry, rather than industry bets. Investors prefer stocks with a high probability of large positive jump variation, but they also tend to overweight safer industries. Additionally, the fact that large and small (signed) jump variation have differing marginal predictive content is explained at least in part by our observation that in double-sorted portfolios, the content of signed large jump variation is negligible when controlling for either signed total jump variation or realized skewness. By contrast, signed small jump variation has unique information for predicting future returns, even when controlling for total jump variation or realized skewness. Further, we find that large jumps are closely associated with ``big'' news, as might be expected. In particular, large earning announcement surprises increase both the magnitude and occurrence of large jumps. While such news related information is embedded in large jump variation, the information is generally short-lived, and dissipates too quickly to provide marginal predictive content for subsequent weekly returns. Finally, we find that while large jump variation is closely associated with large earnings surprises (``big'' news), small jumps tend to be more closely associated with idiosyncratic risks, and can be diversified away.}


\bigskip 
\bigskip
%\bigskip

\vspace{.1in}

\noindent {\it Keywords}: Forecasting, Integrated Volatility, High-Frequency Data, Jumps, Realized Skewness, Cross-Sectional Stock Returns, Signed Jump Variation.
	
\noindent {\it JEL Classification}: C22, C52, C53, C58.	

\renewcommand{\baselinestretch}{1.1}%
\normalsize{}%
----------------------------------------

\noindent {\footnotesize $^*$ Bo Yu, Dept. of Economics, Rutgers University, 75 Hamilton Street, New Brunswick, NJ 08901, USA - byu@econ.rutgers.edu. Bruce Mizrach, Dept. of Economics, Rutgers University - mizrach@econ.rutgers.edu. Norman R. Swanson, Dept. of Economics, Rutgers University - nswanson@economics.rutgers.edu. The authors are grateful to Mingmian Cheng, Yuan Liao, Xiye Yang and seminar participants at the 2019 SoFiE Conference, the 2019 Asia Meeting of the Econometric Society, and the 2019 Midwest Economics Association Conference for numerous useful comments.}

\setlength{\baselineskip}{1.5\baselineskip}
\hyphenpenalty=5000 \tolerance=1000


\setcounter {page} {0}\newpage

\newpage

\section*{1  { } Introduction}

Theoretical models of the risk-return relationship anticipate that volatility should be priced, and that investors should demand higher expected returns for more volatile assets.  However, ex-ante risk measures are not directly observable, and must be estimated (see e.g., Rossi and Timmermann (2015)). Given the necessity of estimating volatility, various different risk estimators have been utilized in the empirical literature studying the strength and sign of the risk-return relationship. Unfortunately, the evidence from the literature is mixed, in the sense that researchers have found both negative and positive relationships between return and volatility. One possible reason for these surprisingly contradictory findings is that  
the risk-return relationship is nonlinear. Examples of papers pursuing this hypothesis include Campbell and Vuolteenaho (2004), who incorporate different factor betas based on good and bad news about cash flows and discount rates; and Woodward and Anderson (2009) who find that bull and bear market betas differ substantially across most industries. This research has helped to spawn the ``smart-beta'' approach to factor investing.\footnote{In 2017, Morningstar reported that this approach to investing has attracted over one trillion dollars in assets (see e.g., Jennifer Thompson, Financial Times, December 27, 2017).} In related research, Feunou, Jahan-Parvar, and T\'edongap (2012) model the effects of volatility in positive and negative return states separately. They define so-called disappointment aversion preferences, and show that investors should demand a higher return for downside variability. These authors find empirical support for their model in the U.S. and several foreign markets using a bi-normal GARCH process to estimate volatility.

In this paper, we focus on the importance of jumps in volatility for understanding the risk-return relationship. We do this by assessing the marginal predictive content of small versus large jump variation, when forecasting one-week ahead cross-sectional equity returns. We also examine earnings announcements as well as carry out various Fama-MacBeth type regressions in order to uncover the linkages between (small and large) jumps and news. Finally, we examine the importance of control variates, including skewness and other firm characteristics, when undertaking to disentangle the relative importance of small, large, positive, and negative jumps for the dynamic evolution of firm specific returns. Much of the empirical research that explores the importance of jumps in this context focuses on estimation of continuous and jump variation components using nonparameteric realized measures constructed with high frequency financial data. A key paper in this area is Bollerslev, Li and Zhao (BLZ: 2019), who examine the relationship between signed jumps and future stock returns in the cross-section. They document that signed jump variation, which captures the asymmetric impact of upside and downside jump risks, are good predictors of returns for small and illiquid stocks.\footnote{ In a related paper, Duong and Swanson (2015) construct both small and large jump measures based on some fixed truncation levels. They exploit the risk predictabilities of different jump measures using both index data and Dow 30 stocks and find that small jump variation has more volatility predictability than large jump variation.}
In the current paper, we add to this literature by decomposing jump variation into signed small and large components and evaluating the importance of these elements in a cross-section of stock returns. We utilize the cross-section of individual stocks because aggregate index returns may mask small jump effects on return predictability. Indeed, many studies document that aggregation  may diversify away idiosyncratic small jumps in the cross-section (see e.g., A\"{\i}t-Sahalia and Jacod (2012) and Duong and Swanson (2015)).

The motivation for our paper can be traced back to Yan (2011) and Jiang and Yao (2013), who show that large, infrequent jumps are priced in the cross-section of returns. Feunou, Jahan-Parvar, and Okou (2018) take the decomposition used by these authors one step further, and model jumps in the realized semi-variances of market returns. They construct a new measure of the variance risk premium, and find a strong positive premium for downside risk. Fang, Jian and Luo (2017) find a similar result for Chinese market returns. In a related line of research, various authors study 
the information content in the upside and downside jump variation. For example, Guo, Wang, and Zhou (2015) document that at the market level, a negative jump component in realized volatility predicts an increase in future equity premia. Bollerslev, Todorov, and Xu (2015) identify both left and right jump tail risks under the risk-neutral measure. They find that the left jump tail risk is an appropriate proxy for market fear. Additionally, they find that including a variance risk premium together with jump tail risk measures as predictors significantly improves market return forecasts. Finally, they show that jump risk helps explain the high-low book-to-market and winners versus losers portfolio returns. 

Building on the above literature, we decompose jump variation into four distinct components depending on both the direction (semi-variances) and magnitude (small and large) of the jumps.\footnote{The methods that we implement to separate jump variation utilize recent advances in financial econometrics due to Andersen, Bollerslev, Diebold and Labys (2003), Andersen, Bollerslev, and Diebold (2007), Jacod (2008), Mancini (2009), Barndorff, Kinnebrock, and Shephard (2010), Todorov and Tauchen (2010), A\"{\i}t-Sahalia and Jacod (2012), and Patton and Shephard (2015).} Specifically, we decompose individual stock jump semi-variances into small and large components. High frequency intraday data are used to construct various realized jump variation measures, including large upside/downside, small upside/downside, and the difference between upside large (small) and downside large (small) jump variation. We then investigate the relationship between these various jump measures and future returns, using sorted and double-sorted stock portfolios, and using regression analysis. The reason that we decompose jump semi-variances into small and large components is that this decomposition allows us to explore the possibility that they contain different information relevant to investing and return predictability. As Maheu and McCurdy (2004) note, large jumps may reflect important individual stock and market news announcements.  Smaller jumps (or continuous variation) may result from liquidity and strategic trading.

Our key findings can be summarized as follows. First, we find that both small and large upside (downside) jump variation negatively (positively) predict subsequent weekly returns. However, portfolios sorted using signed total jump variation are associated with increased average returns and risk adjusted alphas
for high-low portfolios, relative to the cases where upside or downside jump variation is sorted on. This finding is in accord with the findings of BLZ (2019). 

Our second finding involves the case where jump variation is further decomposed into ``small'' and ``large'' components. In this case, sorting on signed small jump variation leads to value weighted high-low
portfolios with greater average returns and alphas than when either signed total jump or signed large jump variation is sorted on. Indeed, when the truncation parameter used to differentiate small from large jumps is based on a 5 standard deviation cut-off, we find that average
return spreads
are 10\% higher when signed small jump variation is sorted on rather than signed total jump variation. Moreover, these average return
spreads
are statistically significant in both cases. However, average return
spreads
are not significantly different from zero when signed large jump variation is sorted on. Indeed, including large jump variation is actually detrimental to predictive accuracy, as average returns and alphas 
for high-low portfolios
actually decline when total variation is instead utilized in our prediction experiments. These results suggest that there may be a ``jump-threshold'', beyond which ``large'' jump variation contains no marginal predictive ability, relative to that contained in small jump variations.\footnote{When equal weighted portfolios are instead examined, sorting on total jump variation yields higher average returns and alphas than when sorting on small or large jump variation. However, deeper inspection of our tabulated results in this case reveals that average returns associated with large jump variation sorts are much smaller (around 1/2 the magnitude) of small and total jump variation sorts, and that the magnitude of average returns associated with small jump variation sorts is much closer (within 10\%) to the average returns associated with total jump variation sorts when our truncation parameter uses a 5 standard deviation cut-off instead of a 4 standard-deviation cut-off. This suggests that the ``jump-threshold'' differs depending upon portfolio type, and indicates that our findings based on equal weighted portfolios are largely in accord with the findings elucidated above.} In summary, we find that large jump variation has little to no marginal predictive content, beyond a certain threshold. Indeed, when said threshold is judiciously selected, one can actually improve predictive performance in our experiments, leading to increased 
high-low portfolio average returns and alphas, when sorting portfolios based on small jump variation rather than total jump variation.

Third, industry double-sorts indicate that the benefit of small signed jump
variation investing is driven by stock selection within an industry, rather than industry bets. Investors prefer stocks with a high probability of large positive jump variation, but they also tend to overweight safer industries.

Fourth, the reason why small and large (signed) jump variation measures have differing marginal predictive content for returns is associated with the importance of realized skewness as a control variable in our experiments. Namely, we find that in double-sorted portfolios, the content of signed large jump variation is negligible when controlling for either signed total jump variation or realized skewness. By contrast, signed small jump variation has unique information for predicting future returns, even when controlling for total jump variation or realized skewness. This finding is consistent with the results from a series of Fama-MacBeth regressions, in which we control for multiple firm characteristics and risk measures.

Finally, small and large jump variation measures are driven by different economic factors and contain different information for predicting future returns. For example, large jumps are closely associated with ``big'' news. In particular, large earning announcement surprises increase both the magnitude and occurrence of large jumps. While such news related information is embedded in large jump variation, the information is generally short-lived, and dissipates too quickly to provide marginal predictive content for subsequent
weekly returns. This is consistent with our finding that filtering out signed small jump variation, which we know to be useful, from signed total jump variation, results in increased 
predictive ability, relative to the case where only signed total jump variation is utilized in return forecasting, especially for big firms. Additionally, this finding is interesting, given that comparison of aggregated and weighted jump variation measures indicates that small jump variation captures idiosyncratic risks and can be diversified away.\footnote{This result is consistent with the finding of Amaya, Christoffersen, Jacobs,
and Vasquez (2015) that preference for positive asymmetry (skewness) may partially explain
the idiosyncratic volatility puzzle, especially for small firms.}

The rest of this paper is organized as follows. In Section 2 we discuss the model setup and define the jump risk measures that we utilize. Section 3 contains a discussion of the data used in our empirical analysis, and highlights key summary statistics taken from our dataset. Section 4 presents our main empirical findings, including discussions of results based on single portfolio sorts, double-sorts, cumulative return and Sharpe ratio analysis, firm-level Fama MacBeth regressions, and finally, jumps and news announcements. Section 5 concludes.   	  

\section*{2  { } Model Setup and Estimation Methodology}

Following A\"{\i}t-Sahalia and Jacod (2012), assume that the log price, $X_t$, of a security follows an It\^o semimartingale, formally defined as: 
$$X_t=X_0+\int_0^tb_sds+\int_0^t\sigma_sdW_s+\int_0^t\int_{\{|x|\le\epsilon\}}x(\mu-\nu)(ds,dx)+\int_0^t\int_{\{|x|\ge\epsilon\}}x\mu(ds,dx),$$
where $b$ and $\sigma$ denote the drift and diffusive volatility processes, respectively; $W$ is a standard Brownian motion; $\mu$ is a random positive measure with its compensator $\nu$; and $\epsilon$ is the (arbitrary) fixed cutoff level (threshold) used to distinguish between small and large jumps. As pointed out in A\"{\i}t-Sahalia and Jacod (2012), the continuous part of this model (i.e., the $\int_0^t\sigma_sdW_s$ term) captures normal hedgeable risk of the asset. The ``big jumps'' part of the model (i.e., the $\int_0^t\int_{\{|x|\ge\epsilon\}}x\mu(ds,dx)$ term) may capture big news-related events such as default risk, and the ``small jumps'' part of the model (i.e., the $\int_0^t\int_{\{|x|\le\epsilon\}}x(\mu-\nu)(ds,dx)$ term) may capture large price movements on the time scale of a few seconds. If jumps are summable (e.g., when jumps have finite activity, so that $\sum_{s\le t}\Delta X_s < \inf$, for all $t$), then   
the size of a jump at time $s$ is defined as $\Delta X_s = X_s-X_{s-}.$\footnote{A jump process has finite activity when it makes a finite number of jumps, almost surely, in each finite time interval, otherwise it is said to have infinite activity.} 
In this context, the ``true'' price of risk is often defined by the quadratic variation, $QV_t$, of the process $X_t$. Namely,  
$$QV_t=\int_0^t\sigma_s^2ds+\sum_{s\le t}\Delta X^2_s,$$
where the variation of the continuous component (i.e., the integrated volatility) is given by $IV_t=\int_0^t\sigma_s^2ds$, and the variation of the price jump component is given by  $QJ_t=\sum_{s\le t}\Delta X^2_s $. 

In the sequel, intraday stock returns are assumed to be observed over equally spaced time intervals in a given day, where the sampling interval is denoted by $\Delta_n$, and the number of intraday observations is $n$. Thus the intraday log-return over the $i$th interval is defined as  $$r_{i,t}=X_{i\Delta_n,t}-X_{(i-1)\Delta_n,t}.$$
It is well known that when the sampling interval goes to zero, the realized volatility, $RV_t$, which is calculated by summing up all successive intraday squared returns, converges to $QV_t$, as $n \rightarrow \infty$, where 
$$RV_{t}=\sum_{i=1}^{n}r_{i,n}^{2}\rightarrow_{u} QV_t=IV_t+QJ_t,$$
where $\rightarrow_{u}$ denotes convergence in probability, uniformly in time.

To separate jump variation from integrated volatility, Andersen, Bollerslev and Diebold (2007) show that the jump and continuous components of realized variance can be constructed as: 
$$RVJ_t=max(RV_t-\widehat{IV_t},0)$$
and
$$RVC_t=RV_t-RVJ_t,$$
respectively, where $\widehat{IV_t}$ is an estimator of $\int_0^t\sigma_s^2ds$. Following Barndorff-Nielsen and Shephard (2004), and Barndorff-Nielsen, Graverson, Jacod, Podolskij, and Shephard (2006), we use tripower variation to estimate the integrated volatility. In particular, define  $$\widehat{IV_t}=V_{\frac{2}{3},\frac{2}{3},\frac{2}{3}}\mu_{\frac{2}{3}}^{-3},$$
where $\mu_q=E(|Z|^q)$ is the $q$th absolute moment of the standard normal distribution, and $$V_{m_1,m_2,...m_k}=\sum_{i=k}^n |r_{i,t}|^{m_1}|r_{i-1,t}|^{m_2}...|r_{i-k+1,t}|^{m_k},$$ where $m_1,$ $m_2$ ...$m_k$ are positive, such that $\sum_1^km_i=q$. Based on the above decomposition approach, Duong and Swanson (2011, 2015) separate jump variation into small and large variation measures, using various truncation levels, $\gamma$. In particular, they define realized small and large jump variation measures as follows: 
$$RVLJ_{\gamma,t}=min(RVJ_t, \sum_{i=1}^n r_{i,t}^2I_{\{|r_{i,t}|\ge\gamma \}})$$
and
$$RVSJ_{\gamma,t}=RVJ_t-RVLJ_{\gamma,t},$$
respectively, where $I(\cdot )$ denotes the indicator function, which equals one if the absolute return is larger than the truncation level, and is otherwise equal to zero. 
We are also interested in upside and downside variation measures associated with positive and negative returns. Thus, following, Barndorff-Nielsen, Kinnebrock, and Shephard (2010) we construct realized semi-variances, defined as: 
$RS_t^+=\sum_{i=1}^n r_{i,t}^2I_{\{r_{i,t}>0\}},$  
$RS_t^-=\sum_{i=1}^n r_{i,t}^2I_{\{r_{i,t}<0\}},$
and $RV_t=RS_t^++RS_t^-.$ They show that the upside and downside semi-variances ($RS_t^+$ and $RS_t^-$, respectively) each converge to the sum of one-half of the integrated volatility and the corresponding signed jump variation. Namely, 
$$RS_t^+\rightarrow_u \frac{1}{2}\int_0^t\sigma_s^2ds+\sum_{s \le t}\Delta X_s^2I_{\{\Delta X_s>0\}}$$
and 
$$RS_t^-\rightarrow_u \frac{1}{2}\int_0^t\sigma_s^2ds+\sum_{s \le t}\Delta X_s^2I_{\{\Delta X_s<0\}}.$$ 
We construct upside and downside jump variation measures as follows: 
\begin{equation} 
RVJP_t=max(RS_t^+-\frac{1}{2}\widehat{IV_t},0) \label{eq1}
\end{equation}
and
\begin{equation}
RVJN_t=max(RS_t^--\frac{1}{2}\widehat{IV_t},0). \label{eq2}
\end{equation}

In addition, signed jump variation can be calculated as the difference between these upside and downside jump measures,
\begin{equation}
SRVJ_t=RVJP_t-RVJN_t. \label{eq3}
\end{equation} This measure captures asymmetry in upside and downside jump variation. 

In our analysis, we further decompose upside and downside jump variation measures into small and large components using thresholding method (see Mancini (2009), Duong and Swanson (2015), Li, Todorov, Tauchen and Chen (2017), and the references cited therein for discussion of thresholding methods). In particular, upside large jump variation based on fixed truncation level, $\gamma$, is defined as follows:
\begin{equation}
RVLJP_{\gamma,t}=min(RVJP_t, \sum_{i=1}^n r_{i,t}^2I_{\{r_{i,t}>\gamma \}}) \label{eq4}
\end{equation}
and
\begin{equation}
RVLJN_{\gamma,t}=min(RVJN_t, \sum_{i=1}^n r_{i,t}^2I_{\{r_{i,t}< -\gamma \}}). \label{eq5}
\end{equation}
We use a truncation level, $\gamma$, that is constructed by estimating $\alpha\sqrt{\frac{1}{t}\widehat{IV}_t^{(i)}}\Delta_n^{0.49}$, and is data-driven, accounting for the time-varying diffusive spot volatility of different stocks in the cross-section. \footnote{For each stock, Li, Todorov, Tauchen and Chen (2017) use bipower variation as the fixed value for $\widehat{IV}_t^{(i)}$. We instead use bipower variation as the initial value for the integrated volatility $\widehat{IV}_t^{(0)}$, say, and $\widehat{IV}_t^{(i)}$ is estimated using truncated bipower variation with threshold $\gamma^{(i-1)}$, say, where $\gamma^{(i-1)}$ is fixed only when $|\widehat{IV}_t^{(i)}-\widehat{IV}_t^{(i-1)}|$ is smaller than $5\%\times\widehat{IV}_t^{(i-1)}$.} In the sequel, we consider three values for $\gamma$, say $\gamma^1$ (with $\alpha=4$), $\gamma^2$ (with $\alpha=5$), and $\gamma^3$ (with $\alpha=6$). Signed large jump variation (i.e., large jump asymmetry) is defined as follows:
\begin{equation}
SRVLJ_t=RVLJP_t-RVLJN_t. \label{eq6}
\end{equation} Our corresponding small jump variation measure is defined as the difference between total jump variation and large jump variation. Namely,
\begin{equation} 
RVSJP_t=RVJP_t-RVLJP_t \label{eq7}
\end{equation}
and
\begin{equation}
RVSJN_t=RVJN_t-RVLJN_t. \label{eq8}
\end{equation}
 Signed small jump variation is defined as:
\begin{equation}
SRVSJ_t=RVSJP_t-RVSJN_t. \label{eq9}
\end{equation} 
 In order to analyze the predictability of various jump measures in the cross-section, we normalize each of the jump variation measures discussed above by total realized variation. 

Of note, is that a natural alternative to our approach for calculating the upside and downside jump variation measures in (1) and (2) is to use thresholding. Namely, instead of using tripower variation, one can use truncated realized variation (TRV) as a consistent estimator of integrated volatility, where $TRV_t=\sum_{i=1}^n r_{i,t}^2I_{\{|r_{i,t}|\le\alpha_n \}}\rightarrow_{u} IV_t=\int_0^t\sigma_s^2ds.\>$ Upside and downside jump variation measures can then be calculated using:
\begin{equation}
RVJP_t=RS_t^+-\sum_{i=1}^n r_{i,t}^2I_{\{0<r_{i,t}\le\alpha_n \}} \label{eq 10}
\end{equation}
and 
\begin{equation}
RVJN_t=RS_t^--\sum_{i=1}^n r_{i,t}^2I_{\{-\alpha_n\le r_{i,t}<0 \}} , \label{eq 11}
\end{equation}
where $\alpha_n$ is the truncation level.\footnote{Here, the threshold, $\alpha_n=3\sqrt{\frac{1}{t}\widehat{IV}_t^{(i)}}\Delta_n^{0.49}$, is estimated using the same procedure as in footnote 5.} Our empirical findings based on the use of (10) and (11) to define $RVJP_t$ and $RVJN_t$ are qualitatively the same as those reported in Section 4 based on the use of (1) and (2). 

In order to measure skewness and kurtosis, we also construct higher order realized return moments. Following Amaya, Christoffersen, Jacobs, and Vasquez (2015),
\begin{equation}
RSK_t=\frac{\sqrt{n}\sum_{i=1}^n r_{i,t}^3}{RV_t^{\frac{3}{2}}}, \label{eq 12}
\end{equation} standardized daily skewness is defined as: 
and normalized daily realized kurtosis is defined as:
\begin{equation}
RKT_t=\frac{n\sum_{i=1}^n r_{i,t}^4}{RV_t^{2}}. \label{eq 13}
\end{equation} 

Finally, it should be noted that we follow Amaya, Christoffersen, Jacobs, and Vasquez (2015) and BLZ (2019), and conduct our cross-sectional analysis at the weekly frequency. In particular, on each Tuesday, we compute the following weekly realized measures: 
$RV_t^W=(\frac{252}{5}\sum_{i=0}^4RV_{t-i})^{1/2}$ and
$RM_t^W=\frac{1}{5}(\sum_{i=0}^4RM_{t-i}),$
where $RV_t$ is defined above, and where $RM_t$ denotes any of the realized measures defined above other than $RV_t$ (e.g., $RVJP_t$, $RVJN_t$, $SRVJ_t$, etc.)
Hereafter, we shall drop the superscript ``W'' for the sake of notational brevity. All of the descriptors used to denote the various realized measures constructed in our empirical analysis are summarized in Table 1. 

As described in detail in Section 4, the realized measures outlined above are used in a number of different ways in our empirical analysis. First, we carry out single portfolio sorts, in which we sort stock portfolios on the above realized jump measures, and predict average excess returns, one-week ahead. In these experiments, we also calculate alphas based on regressions that utilize the Fama-French and Carhart factors. In this first part of our analysis, we also examine cumulative returns and Sharpe ratios. In addition to the single portfolio sorts, we carry out double portfolio sorts, in which we sort not only on realized jump risk measures, but also on various control variables, including realized skewness and other firm specific characteristics. Using these double sorts, we also examine the inter-play between individual stock-level jump variation and industry-level jump variation. Needless to say, the purpose of our double-sorts is to examine the robustness of our findings based on single sorts, after controlling for other realized measures. Next, we carry out a series of Fama-MacBeth regressions, in order to check the robustness of our findings to the inclusion of various firm specific characteristics. Finally, we carry out an event study in which the effect of earning surprises on realized jump measures is examined. For complete details, see Section 4.

\section*{3  { } Data}
We utilize high frequency trading data obtained from the consolidated Trade and Quote (TAQ) database. In particular, we analyze all stocks in the TAQ database that are listed on the NYSE, Amex, and NASDAQ stock exchanges. There are 15,585 unique stocks during the 1,246 weeks analyzed in this paper.\footnote {In some cases, multiple TAQ symbols are matched with a unique Center for Research in Security Prices (CRSP) PERMNO. Over each quarter, the TAQ symbol which has the most observations is kept and the other overlapping observations are dropped.} The sample period is from January 4, 1993 to December 31, 2016. Intraday prices are sampled at five minute intervals from 9:30 a.m. to 4:00 p.m. from Monday to Friday. Overnight returns are not considered in this paper, and days with less than 80 transactions at a 5 minute frequency are eliminated. For example, if AAPL has less than 80 trades on a particular day, then AAPL is dropped from our sample, but only for that day. All high frequency data used in this paper are cleaned to remove trades outside of exchange hours, negative or zero prices or volumes, trade corrections and non-standard sale conditions, using the methodology described in Appendix A.1 in BLZ (2019).

We constructed two variants of our dataset. The first is cleaned as discussed above. The second classifies five minute intraday returns greater than 15\% as abnormal and replaces them with zeros. In the sequel, results based on analysis of the second dataset are reported. However, results based on utilization of the first dataset are qualitatively the same; and indeed key return results reported in this paper generally change by 1 basis point or less when the former dataset is used in our analysis. Complete results are available upon request from the authors. 
 
Daily and monthly returns, and adjusted numbers of shares for individual securities are collected from the CRSP database. Delisting returns in CRSP are used as returns after the last trading day. Daily Fama-French and Carhart four factor (FFC) portfolio returns are obtained from Kenneth R. French's website. 

Following Amaya, Christoffersen, Jacobs, and Vasquez (2015) and BLZ (2019), we also construct various lower frequency firm level variables that might be related to future returns, such as the market beta (BETA), the firm size, the book-to-market ratio (BEME), momentum (MOM), short-term reversals (REV), idiosyncratic volatility (IVOL), co-skewness (CSK), co-kurtosis (CKT), maximum (MAX) and minimum (MIN) daily return in the previous week, and the Amihud (2002) illiquidity measure (ILLIQ). For a complete list of these firm specific control variables, refer to Table 1. For a detailed description of these variables, including the methodology used to construct them, see Appendix A.2 in BLZ (2019). 

Note that while the majority of our analysis is based on the examination of individual stocks, in our double sorts, there are some cases (that are reported in Section 4.4) where we examine the inter-play between individual stock-level jump variation and industry-level jump variation. In this case, we follow the Fama-French industry classification approach, and group stocks into 49 industries based on their SIC codes, which are obtained from CRSP. 
 
\subsection*{3.1  { }  Unconditional Distributions of Realized Measures}

Figure 1 displays kernel density estimates of the unconditional distributions of each of our realized measures, across all firms and weeks. The top two panels in the figure show the distributions of signed jump variation and realized skewness. Both of these distributions are approximately symmetric and peaked around zero. The skewness distribution is more fat-tailed, however.\footnote{The kurtosis of signed jump variation is 4.36. For realized skewness, the analogous statistic is 12.04.} The middle two panels of Figure 1 display the distributions of signed small and large jump variation. Similar to signed jump variation, both signed small and large jump variation mesaures are approximately symmetric around zero, but signed small jump variation is less fat-tailed.\footnote{The unconditional kurtosis is 6.43 and 3.51, for signed small and large jump variation, based on truncation level $\gamma^1 $; and 8.87 and 3.09 based on truncation level $\gamma^2$, respectively.} Consistent with the results in Amaya, Christoffersen, Jacobs, and Vasquez (2015) and BLZ (2019), realized volatility and realized kurtosis are both right skewed and very fat-tailed, as shown in the bottom two panels of the figure.\footnote{The kurtosis is 15.85 and 27.24, for the unconditional distribution of realized volatility and realized kurtosis, respectively.} 

Figure 2 shows the time variation in the cross-sectional distribution of each realized measure using 10-week moving averages. In particular, 10th, 50th, and 90th percentiles for each realized measure in the cross-section are plotted. Thus, dispersion at any given time in these plots reflects information about the cross-sectional distribution of the realized measure. Inspection of Panels A and B in the figure reveal that signed jump variation and realized skewness have stable dispersion, for all three cross-sectional percentiles, over time, while the cross-sectional dispersion in realized volatility and kurtosis are rather time-dependent (see Panels C and D). Additionally, similar to the cross-sectional distribution of signed jump variation, the percentiles for signed small and large jump variation measures are quite steady over time, as indicated in Panels E-H. 


\subsection*{3.2  { } Summary Statistics and Portfolio Characteristics}

Table 2 contains various summary statistics for all of the realized measures summarized in Table 1. In Panel A, the cross-sectional means and standard errors for each of the realized measures is given. This is done for two different truncation levels, denoted as $\gamma^{1}=4\sqrt{\frac{1}{t}\widehat{IV}_t^{(i)}}\Delta_n^{0.49}$ and 
$\gamma^{2}=5\sqrt{\frac{1}{t}\widehat{IV}_t^{(i)}}\Delta_n^{0.49}$. As might be expected, jump variation is quite sensitive to the choice of $\gamma$. For example, the (normalized) mean of RVSJP (positive (upside) small jump variation) increases from 0.1180 to 0.1715 when the threshold is increased from $\gamma^{1}$ to $\gamma^{2} $. Needless to say, various measures remains the same, as they are independent of $\gamma$. 

Panel B of Table 2 contains cross-sectional correlations for all of the realized measures. In accord with the findings reported by Amaya, Christoffersen, Jacobs, and Vasquez (2015) and BLZ (2019), signed jump variation (SRVJ) and realized skewness (RSK) are highly correlated with each other and have significantly positive correlations with the short term reversal variable (REV); as well as with maximum (MAX) and minimum (MIN) daily returns in the previous week. 

Interestingly, we also find that signed large jump variation (SRVLJ) is highly correlated with SRVJ and with RSK. However, signed small jump variation (SRVSJ) has lower correlation with SRVJ and much smaller positive correlation with RSK. This finding is consistent with our finding discussed below that realized skewness captures information that is primarily contained in large jumps; and serves as an important distinction between the findings in this paper and those reported in the papers discussed above.

Table 3 complements Table 2 by sorting stocks into quintile portfolios based on different realized measures. On each Tuesday, stocks are ranked by the realized variation measures, and we calculate the equal-weighted averages of each firm characteristic in the same week. Panels A, B, C and E report summary statistics for portfolios sorted by SRVJ, SRVLJ, SRVSJ, and RSK, respectively. Consistent with the correlations contained in Table 2, firms with larger signed small and large jump variation measures tend to have higher signed jump variation, realized skewness, REV, MAX and MIN. Firms with high realized volatility and realized kurtosis (see Panels D and F) tend to be illiquid and small. 
\footnote{See the Supplementary Appendix for results based on the examination of additional quintile portfolios that are constructed based on ex-ante risk measures and displayed with ex-post risk measures. It is clear that sorting stocks based on jump risk measures results in portfolios with the desired risk exposures.}

\color{black}
\section*{4  { } Empirical Results }

In this section, results based on stocks that are sorted into quintile portfolios based on a single different realized measure are first reported. These single (univariate) portfolio sort results are collected in Tables 4 to 7. Results based on double sorts are reported (in Tables 8 to 15). We assume a weekly holding period, and return calculations reported in the tables are carried out as follows. At the end of each Tuesday, stocks are sorted into quintile portfolios based on different realized variation measures (see Panel A of Table 1). We then calculate equal-weighted and value-weighted portfolio returns over the subsequent week. We report the time series average of these weekly returns for each portfolio (these returns are called ``Mean Return'' in the tables) In addition, we regress excess return of each portfolio on the Fama-French and Carhart (FFC4) factors to control for systematic risks, using regression of the form
\begin{equation}
r_{i,t}-r_{f,t}=\alpha_i+\beta_{i}^{MKT}(MKT_t-r_{f,t})+\beta_{i}^{SMB}SMB_t +\beta_{i}^{HML} HML_t +\beta_{i}^{UMD} UMD_t+\epsilon_{i,t} \label{eq12}
\end{equation}
where $r_{i,t}$ denotes the weekly return for firm i, $ r_{f,t}$ is the risk-free rate; and $MKT_t$, $SMB_t$, $HML_t$, and $UMD_t$ denote FFC4 market, size, value and momentum factors, respectively. The intercepts from these regressions (called ``Alpha'' in our tabulated results), measure risk-adjusted excess returns, and are also reported in Tables 4 to 15. Needless to say, our objective in these tables is to assess whether predictability exists, after controlling for various systematic risk factors. Finally, in Tables 16 and 17, we report the results of cross-sectional (firm level) Fama-MacBeth regressions used to investigate return predictability when simultaneously controlling for multiple realized measures and firm characteristics. 

\subsection*{4.1  { }  Single (Univariate) Portfolio Sorts Based on Realized Measures}

In this section, we first discuss the results contained in Table 4. Recall that the ``Mean-Return'' in this table is an average taken over our entire time series of equal-weighted and value-weighted portfolio returns, for single sorted portfolios based on positive jump variation (RVJP), negative jump variation (RVJN) and signed jump variation (SRVJ). Values in parentheses are Newey-West $t$-statistics (see Bollerslev, Todorov, and Xu (2015) and Petersen (2009) for further discussion). Panel A provides results for portfolios sorted by RVJP. Inspection of the entries in this panel indicate that mean returns and alphas of high-low portfolios (i.e., the difference in returns (alphas) between the fifth and first quintiles) are all negative, indicating a negative association between RVJP and subsequent stock returns. Interestingly, the alpha of -7.71 basis points (bps) is insignificant for the high-low spread for the equal weighted portfolio, while the mean return of -5.63 bps is only significant at a 10\% level for the high-low spread for the value weighted portfolio. 

The lack of statistical significance for some of the mean return values reported in Panel A does not characterize our findings when negative and signed jump variation measures are utilized for sorting. Moreover, the magnitudes of the mean returns and alphas are usually three or more times larger when sorting on negative and signed jump variation (to see this, turn to Panels B and C of Table 4). In Panel B, the high-low spread of mean returns equals 36.06 bps, with a $t$-statistic of 6.47 for the equal-weighted portfolio, and 15.13 bps with a $t$-statistic of 3.75 for the value-weighted portfolio. Moreover, both equal-weighted and value-weighted portfolios generate significant positive abnormal future returns measured by the alphas. These results clearly point to a statistically significant positive association between negative jump variation and the following week's returns. 

Panel C in Table 4 contains results for portfolios sorted by signed jump variation. The negative high-low spreads indicate a statistically significant negative association between signed jump variation and future returns. In particular, a strategy buying stocks in the lowest signed jump variation quintile and selling stocks in the highest signed jump variation quintile earns a mean return of 40.82 bps with a $t$-statistic of 9.85 each week for the equal-weighted portfolio and 25.02 bps with a $t$-statistic of 5.78 for the value-weighted portfolio. These results are consistent with the results reported in BLZ (2019). Interestingly, almost all of the mean returns listed in Table 4 are ``alpha'' (see tabulated average alphas in the table), and cannot be explained by standard portfolio risk factors using regressions of the type given above as equation (\ref{eq12}). 

A key question that we provide evidence on in this paper is whether the results summarized in Table 4 carry over to the case where small and large jump variation is separately sorted on. First, consider large jumps. Table 5 reports the results for portfolios sorted by positive, negative and signed large jump variation, respectively. Similar to positive jump variation, positive large jumps negatively predict subsequent returns, but the predictability is not significant, regardless of the truncation level ($\gamma$) used to separate small and large jumps, and regardless of portfolio weighting used. This is evidenced by the fact that the $t$-statistics for mean returns and alphas of high-low portfolios all indicate insignificance, at a 5\% testing level, regardless of truncation level. Thus, there is no ambiguity, as in Panel A of Table 4. Positive jump variation is not a significant predictor, under our large jump scenario. On the other hand, we shall see that sorting on small and large negative variation measures yields significant excess returns, as does sorting on small positive jump variation, under both equal and value weighting schemes. 

As just noted, equal-weighted high-low portfolios sorted on large negative jump variation generate significant positive returns and alphas (see Panel B of Table 5). However, analogous returns and alphas under value weighting are not significant. Signed large jump variation is sorted on in Panel C of Table 5. Signed large jump variation is useful for undertaking a long-short trading strategy based on the difference between large upside and downside jump variation measures. Inspection of the results in this panel of the table reveals that the high-low spread for the equal weighted portfolio generates an average risk-adjusted weekly return of -28.36 bps (with a $t$-statistics of -9.39) and -9.25 bps (with a $t$-statistics of -2.87) for the value-weighted portfolio, for truncation level equal to $\gamma^1$. Results based on $\gamma^2$ (i.e., our larger truncation level) are also significant, although magnitudes are lesser and only for our equal weighted portfolio.\footnote{Empirical findings based on $\gamma^3$ are similar to those discussed above, and hence are not reported. This robustness of our findings to the choice of $\gamma$ also characterizes the other empirical findings discussed in the sequel.} In particular, observe that when large jump variation is constructed using $\gamma^2$, the high-low spreads for value-weighted portfolios sorted by downside or signed large jump variation measures are insignificant, suggesting that small firms have stronger relationships (than larger firms) between signed (or negative) large jump variation and subsequent returns. This may be due to the fact that smaller firms are in some ways more susceptible to changing market conditions than larger firms.

Table 6 summarizes results analogous to those reported in Table 5, but for positive, negative and signed small jump variation measures. Similar to large jump measures, positive and signed small jump variation measures negatively predict future returns, and negative small jump variation measures positively predict returns in the following week. By contrast, the differences in average (risk-adjusted) returns between equal-weighted and value-weighted long-short portfolios based on RVSJP and RVSJN are smaller than those for portfolios based on large jumps (compare the entries for the high-low quintiles under the two weighting schemes in Panels A and B of Table 6 with like entries in Panels A and B of Table 5). These results indicate that big firms have a stronger relationship between small jump variation and future returns than that between large jumps and subsequent weekly returns. Since stocks for big firms are more liquid and price discovery more rapid, the predictabilities of large jumps are much weaker or insignificant for big firms. This finding is in line with BLZ (2019), who document that the predictability of signed jump variation is stronger for small and illiquid firms and is driven by investor overreaction. In addition, when using our larger truncation level, $\gamma^2$, value-weighted high-low spreads based on signed small jump variation are larger than those based on signed total jump variation and signed large jump variation. This result implies that a long-short strategy associated with signed small jump variation generates the highest value-weighted risk-adjusted returns, given the use of an appropriate truncation level to separate small and large jumps. 

Table 7 reports results for portfolios sorted by realized volatility, realized skewness, realized kurtosis, and continuous variance. Consistent with the results in Amaya, Christoffersen, Jacobs, and Vasquez (2015) and BLZ (2019), there is a significant negative relationship between realized skewness and future returns, while the association is not significant between either realized volatility or realized kurtosis and returns in the following week, regardless of portfolio weighting scheme. In addition, continuous variance significantly and negatively predicts one-week ahead returns for equal-weighted portfolios, but this negative association is not significant for value-weighted portfolios.  

\subsection*{4.2  { }  Cumulative Returns and Sharpe Ratios}

Not surprisingly, our findings based on univariate portfolio sorts suggest that strategies that utilize different realized measures deliver different risk-adjusted average returns.  In order to investigate this result further, we calculate cumulative returns and Sharpe ratios for short-long portfolios, sorted on various risk measures that are described in Table 1, including SRVJ, RSJ, SRVLJ, SRVSJ, and RSK. In addition, for comparison purposes, we also carry out our analysis using the relative signed jump variation measure (called RSJ) that is examined by BLZ (2019). Our experiments are carried out as follows. Beginning in January 1993, various short-long portfolios are constructed, with an initial investment of \$1. These portfolios are re-balanced and accumulated at a weekly frequency, until the end of 2016.\footnote{Cumulative returns calculations do not include the risk-free rate. For a definition of cumulative returns both with and without the weekly risk-free rate, see BLZ (2019).}  
Figure 3 plots portfolio values over time. Consistent with our results based on single portfolio sorts, inspection of the plots in this figure indicates that for equal-weighted portfolios sorting on signed jump variation (SRVJ) yields the largest portfolio accumulations; and for value-weighted portfolios, sorting on signed small jump variation (SRVSJ) yields the largest portfolio accumulations.
\footnote{Note that RSJ, which measures the same signed jump variation as SRVJ, although using different estimation methodology, generates the highest cumulative return for equal-weighted portfolios, but is dominated by SRVSJ for value-weighted portfolios.}

Now, consider the Sharpe ratios reported below, which are reported for various jump measures, and are constructed based on truncation level $\gamma^{2}=5\sqrt{\frac{1}{t}\widehat{IV}_t^{(i)}}\Delta_n^{0.49}$.


\begin{table}[htbp!]
	\centering
	Sharpe Ratios
	
	\resizebox{0.6\textwidth}{!}{
		\begin{tabular}{lrccccc}
			\toprule
			\multicolumn{2}{r}{} & SRVJ  & RSJ & SRVLJ & SRVSJ & RSK    \\
			\midrule
			\multicolumn{2}{l}{Equal-Weighted} & 2.1342 & 2.1363 & 1.8556 & 1.8161 & 2.2234 \\
			\multicolumn{2}{l}{Value-Weighted } & 1.1322 & 1.1310 & 0.1611 & 1.2755 & 0.8665 \\
			\bottomrule\\
	\end{tabular}}\\	 

\end{table}

The entries in this table are Sharpe ratios for equal and value-weighted short-long portfolios constructed using SRVJ, RSJ, SRVLJ, SRVSJ, and RSK. Recall that RSK is realized skewness (see Table 1 for definitions of these measures). The sample of stocks used for Sharpe ratio calculations includes all NYSE, NASDAQ and AMEX listed stocks for the period January 1993 to December 2016. At the end of each Tuesday, all of the stocks in the sample are sorted into quintile portfolios based on ascending values of various realized risk measures. A high-low spread portfolio is then formed as the difference between portfolio 1 and portfolio 5, and held for one week, where 1 and 5 refer to quintiles, as in Tables 1 to 7. The Sharpe ratio is calculated with the one-week ahead returns. 

Interestingly, for equal-weighted portfolios, the RSK-based short-long strategy yields the highest Sharpe ratio (i.e., 2.2234), although the ratio of 2.1342 for SRVJ is approximately the same. \color{black}  
Still, the success of the RSK measure is likely due to its relatively stable performance, compared with other jump-based strategies. This finding is similar to the findings discussed in Xiong, Idzorek, and Ibbotson (2016), who show that tail-risks can be substantially reduced by forecasting skewness. Note also that the signed small jump variation (SRVSJ) based portfolio has the highest Sharpe ratio, among all value-weighted portfolios. However, it is clear that all equal-weighted portfolios outperform their corresponding value-weighted counterparts. This result is consistent with the finding discussed above that small and illiquid firms tend to react more strongly to realized risk measures. 

\subsection*{4.3  { }  Double Portfolio Sorts Based on Realized Measures}

To further investigate whether small and large jumps are priced differently, we utilize double portfolio sorts. In particular, we carry out double sorts in order to examine the robustness of our findings based on single sorts, after controlling for other realized measures. Table 8 reports returns and alphas from various of these sorts in which we alternate the sorting order among SRVJ, SRVLJ and SRVSJ. When we first sort by total jump variation, and then sort stocks based on SRVLJ or SRVSJ, a negative relation only exists between SRVSJ and subsequent weekly returns (see Panels A and B of the table). This result indicates that there is no marginal predictive content associated with large jumps, when conditioning on the predictive content associated with total jump variation, while small jumps have unique information for predicting future returns, even compared to total jumps. 

Panel C reports returns and alphas based on sorting on SRVSJ after controlling for SRVLJ. Both the equal- and value-weighted high-low spreads and alphas are statistically significant in this case, while this is not the case if stocks are first sorted by SRVSJ and then by SRVLJ, as shown in Panel D. More specifically, the high-low return is -25.38 bps (with a t-statistic of -6.61), for the value-weighted portfolio in Panel C, and is -3.77 bps with a $t$-statistics of -1.49 in Panel D, for the value-weighted portfolio. This indicates that the predictable content in large jumps becomes negligible after controlling for small jumps.

BLZ (2019) document that the negative association between realized skewness and one-week ahead returns is reversed when controlling for the signed jump variation. To further investigate the relationship between skewness and different jump variation measures, we use double portfolio sorts to control for different effects that are associated with cross-sectional variation in future returns. 

Panel A of Table 9 reports average returns and corresponding $t$-statistics for 25 portfolios sorted by SRVJ (signed jump variation), controlling for realized skewness (RSK). At the end of each Tuesday, stocks are first sorted into quintiles based on realized skewness, and then within each quintile portfolio, we further sort stocks into quintiles based on signed jump variation. We also report the equal- and value-weighted returns in the following week and Fama-French and Carhart four-factor alphas for the long-short portfolios and the averaged portfolios across quintiles. Inspection of the results in this table indicates that the negative association between SRVJ and future returns still exists, after controlling for RSK, indicating that there is unique predictive information contained in signed jump variation. Panel B in this table reports results for portfolios sorted first by SRVJ and then by RSK. The high-low spreads of the averaged portfolios are positive after controlling for SRVJ, confirming the results reported in BLZ (2019). 

Panel A of Table 10 contains results for portfolios sorted by SRVLJ (signed large jump variation) after controlling for RSK. As noted above, the negative association between SRVLJ and future returns is reversed after controlling for skewness. By contrast, this issue doesn't exist for portfolios sorted by SRVSJ (signed small jump variation) when controlling for skewness, as shown in panel B of Table 10, indicating that signed small jump variation has unique information about future return premia. However, first accounting for skewness negates the usefulness that signed large jump variation has for predicting future returns. This finding serves as an important distinction between the predictive content of small and large jumps. Whereas the former can be forecast by realized skewness, the latter cannot.

Finally, Tables 11 contains results for portfolios sorted on RSK, after controlling for SRVLJ and SRVSJ, respectively. Inspection of the entries in this table indicates that the high-low spreads are negative, except in select value weighted portfolio cases, when controlling for SRVSJ. This is not surprising since skewness captures information from both SRVLJ and SRVSJ, while the negative association between realized skewness and subsequent returns remains, when controlling for either SRVLJ or SRVSJ, in most cases. Of note is that this negative association disappears for some value-weighted portfolios, when controlling for SRVSJ, suggesting that signed small jump variation (especially for big firms) is the main driver of the signed total jump variation. These findings are consistent with the findings documented by Bollerslev and Todorov (2011) that S\&P 500 market portfolios tend to have symmetric jump tails (large jumps). 
  
\subsection*{4.4  { }  Using Double Portfolio Sorts to Examine Stock-Level Versus Industry-Level Predictability}

In this section, we carry out an additional set of double portfolio sort experiments, in which industry based investing is compared with individual stock based investing. Our earlier findings indicate that low signed jump variation investing (buying stocks with low signed jump variation and shorting stocks with high signed jump variation) can deliver significant risk-adjusted returns (this is similar to low risk investing, and is a result also found by BLZ (2019), for example). In order to examine whether this investment strategy relies on industry betting or stock selection within industries (or both), we form double sorted portfolios based on industry-level and stock-level signed jump risk variation. In particular, each Tuesday we group stocks into 49 industries based on SIC codes. Industry-level signed jump risk is calculated as the value-weighted average of signed (large/small) jump variation measures for stocks within each industry. Thus, stocks in the same industry have the same industry signed (large/small) jump variation during a given week. Stock-level signed jump risk is calculated as outlined in the above. Double sorts are then used to investigate the selection effects at industry- and stock-level. Namely, stocks are sorted into 25 portfolios based on industry- and stock-level signed (large/small) jump variation quintiles. With this particular variety of sorting, results are independent of the order in which stocks are sorted.  

Figure 4 depicts the percentage of stocks in each portfolio (see Panel A), and the market capitalization in these portfolios (see Panel B). If industry-level selection and stock-level selection lead to different quintile portfolios (i.e. off-diagonal portfolios in the figures have non-zero membership), it is possible to separate these two effects using double sorts. Namely, there are different industry- and stock-level effects. Both panels indicate this to be the case. 

Tables 12 to 13 report our empirical findings based on our double portfolio sort experiments. In particular, Table 12 reports results for sorting done on signed jump variation (SRVJ), while Table 13 reports results for sorting done on signed large jump variation (SRVLJ) and signed small jump variation (SRVSJ), respectively.  Entries in the tables are mean returns and alphas, as in previous tables. However, in these tables we also report industry-level effects and stock-level effects. These are reported in the last two rows of entries in each panel of the tables. The first of these two rows, called ``Industry-Level Effect'' reports average high-low returns and alphas by averaging across quintiles in the high-low and alpha columns of the table (these are industry-level results).  The second of these two rows, called ``Stock-Level Effect'' reports average high-low returns and alphas by averaging across quintiles in the high-low and alpha rows of the table (these are stock-level results).
Summarizing, rows in these tables display portfolios formed by stocks in the same stock-level SRVJ, SRVLJ, or SRVSJ quintiles, while columns report results for portfolios formed by stocks in the same industry-level SRVJ, SRVLJ, or SRVSJ quintiles.

Turning to Table 12, notice, for example, that a strategy of buying stocks in the highest industry SRVJ quintile and selling stocks in the lowest industry SRVJ quintile generates an equal-weighted average return of 29.63 bps with a $t$-statistic of 5.66, and the corresponding value-weighed average return is 11.48 bps with a $t$-statistics of 2.23 (see Table 12). This finding is interesting, as it suggests that the negative association between SRVJ and future returns is reversed in the industry level. The equal-weighted average of the high-low row (i.e., the average stock-level effect) is -45.28 bps with a $t$-statistic of -11.39 and the alpha is -44.70 bps with a $t$-statistic of -11.50, indicating that the stock-level effect is economically significant. At the stock-level, investors prefer stocks with high SRVJ, requiring lower returns under higher SRVJ, given that there is a large probability of extremely large positive jumps. By contrast, when sorting at the industry-level, investors are more interested in industry exposure with lower SRVJ, or in return distributions concentrated to the right. Lottery-like payoff exposure comes from individual stocks, not from industry bets. These results are mirrored in Table 13, where SRVLJ and SRVSJ are the sorting measures. However, average stock- and industry-level returns and alphas are much higher under SRVSJ sorting than under SRVLJ sorting. For example, buying stocks in the highest industry SRVLJ quintile and selling stocks in the lowest industry SRVLJ quintile generates an equal-weighted average return of 14.83 bps with a $t$-statistic of 3.77 under SRVLJ sorting (see panel A of Table 13), versus an equal-weighted average return of 26.69 bps with a $t$-statistic of 5.02 under SRVSJ sorting (see panel B of Table 13). 

\subsection*{4.5  { }  Firm-Level Fama-MacBeth Regressions}

Table 14 gathers results based on firm-level Fama-MacBeth regressions, which we run in order to investigate the return predictability associated with variation measures, when controlling for multiple firm specific characteristics. Regressions are carried out as follows. At the end of each Tuesday, we run the cross-sectional regression, 
\begin{equation}
r_{i,t+1} =\gamma_{0,t} +\sum_{j=1}^{K_1}\gamma_{j,t}X_{i,j,t}+\sum_{s=1}^{K_2}\phi_{s,t}Z_{i,s,t}+\epsilon_{i,t+1},      \ \ \ t=1,...,T, \label{eq13} 
\end{equation}
where $r_{i,t+1}$ denotes the stock return for firm $i$ in week $t+1$, $K_1$ is the number of potential variation measures, and $X_{i,j,t}$ denotes a relevant realized measure at the end of week $t$. In addition, there are $K_2$ variables measuring firm characteristics, which are denoted by $Z_{i,j,t}$ (see Section 3 for details). After estimating the cross-sectional regression coefficients on a weekly basis, we form the time series average of the resulting $T$ weekly $\widehat{\gamma}_{j,t}$ and $\widehat{\phi}_{s,t}$ values, in order to estimate the average risk premium associated with each risk measure. Namely, we construct
$$\widehat{\gamma}_{j}=\frac{1}{T}\sum_{t=1}^T \widehat{\gamma}_{j,t}, \ \ \ {\textrm{and}} \ \ \    \widehat{\phi}_{s}=\frac{1}{T}\sum_{t=1}^T \widehat{\phi}_{s,t}, \ \ \  {\textrm{for}} \ \ \ j=1,...,K_1, {\  }s=1,...,K_2.$$ 

Panel A of Table 14 reports results for regressions on various realized variation measures, without controlling for firm specific characteristics. Consistent with our results based univariate sorting, signed jump variation (SRVJ) significantly negatively predicts cross-sectional variation, in these weekly returns regressions. Additionally, both signed small and large jump variation measures negatively predict future weekly returns. Finally, both small and large upside (downside) jump variation measures negatively (positively) predict subsequent weekly returns. However, when including measures that contain information from both small and large jump variation measures, as well as realized skewness, the negative association between skewness and future returns is reversed (see the results for the regressions labeled IX, XII, XV, XVI). In particular, skewness drives out signed large jump variation in regression XIII by reverting the negative association between the latter and future returns. If only small jumps are considered as control variables, skewness still negatively predicts future returns. This again indicates that signed small jump variation has unique and significant information about future returns.

Panel B of Table 14 reports regression results for the same set of regressions in panel A, but controlling for various firm specific characteristics, ranging from BETA to ILLIQ (see Table 1 for details). In these regressions, signed (small) jump variation is always significant. Additionally, skewness significantly negatively predicts future returns in regressions that only include small jump variation. This provides yet further evidence that signed small jump variation has unique and significant information about future returns, while large jumps have information in common with realized skewness.

\subsection*{4.6  { } Pricing Distinctions Between Small and Large Jumps}
The results in previous sections show that small and large jump variation measures contain different information, and thus have different predictive content. To further investigate whether the differences are driven by distinct economic factors, we provide empirical evidence on the inter-relationship between jumps and news.
 
\subsubsection*{4.6.1  { } Jumps and News Announcement }

We begin by examining the relationship between jumps and firm-level news announcements. In order to do this, we construct event windows using the approach of Bernard and Thomas (1989). We then plot the dynamics of SRVJ, SRVLJ, and SRVSJ around earnings announcements. In particular, following Livnat and Mendenhall (2006), the earning surprise (SUE) for each stock is defined as
\begin{equation}
SUE_{j,t} = \frac{(X_{j,t}-X_{j,t-4})}{P_{j,t}}, \label{eq16} 
\end{equation} 
where $X_{j,t}$ and $X_{j,t-4}$ denote the analysts' expectations and reported actual earnings per share, respectively. Here, $P_{j,t}$ is the price per share for stock j at the end of quarter t. In a [-12,12] week event window, where week zero denotes the earning announcement week, stocks are sorted into tertile portfolios by the value of SUE at the end of each event week. We then calculate the equal-weighted and value-weighted average of jump measures for each tertile portfolio at each week. Figure 5 displays various jump variation measures of portfolios with the most negative, median, and positive earning surprises. It turns out that large (both positive and negative) jump variation measures are higher during announcement weeks, regardless of news sentiment (i.e., regardless of whether SUE is positive or negative). However, positive large jump variation (RVLJP) is higher on days with the most positive earning surprises, and negative large jump variation (RVLJN) reaches its peak on days with the most negative earning surprises. In contrast, both small positive and negative jump variation measures (RVSJP and RVSJN) have lower magnitudes during announcement weeks. The size of the reduction associated with small positive jump variation (RVSJP) is larger on days with the most negative earning surprises, while small negative jump variation (RVSJN) decreases the most on days with the most positive surprises. For signed jump variation, jump magnitudes increase (relative to non-earnings-surprise weeks) on positive surprise days and decrease on negative surprising days. These results indicate that big news, regardless of sentiment, simultaneously leads to increases in the magnitude of large jump variation, and reductions in the level of small jump variation. 

The other direction in which we investigate the linkage between news announcements and jump variation is based on an exploration of whether news announcements affect the frequency of occurrence of either small or large jumps. 
Table 15 reports the average percentage of firms exhibiting particular types of jumps on days with and without earning surprises. Specifically, on each announcement date, all stocks exhibiting earnings are sorted into tertile portfolios based on the absolute value of the earning surprise (SUE). The categories sorted on are denoted as ``small'', ``medium'', and ``large'', with tertiles calculated by appropriate sorting of the firms based on the absolute values of the firms' earnings surprise magnitudes. Then, within each tertile, the percentage of firms exhibiting a particular type of jump (averaged across all earnings surprise days) is calculated and reported. For these calculations, only days in which at least 3 firms report earning surprises and included in our sample.\footnote{Results are virtually identical if we only include days in which at least 12 or 24 firms report earnings surprises.} Thus, for example, if 12 firms report earning surprises, then 4 firms will be represented in each of the 3 tertiles. Turning to the results in the table, note, for example, that the entry 0.3042 in the sixth column of Panel A indicates that 30.42\% of firms in the ``small surprise'' tertile portfolio recorded a large jump (measured by SRVLJ) on small surprise days, on average, across the entire daily sample. By contrast, 89.83\% of firms exhibit small jumps (measured by SRVSJ) on days with small surprises. 

Two clear conclusions emerge upon examination of the results in this table. First, when the magnitude of earning surprises increases, the average percentage of firms with large jumps (SRVLJ) increases from 30.42\% to 37.37\%. In particular, in Panel A, note that for the ``Small'' tertile, the percentage of firms exhibiting large jumps (SRVLJ) is 30.42\%, while for the ``Large'' tertile, the percentage is 37.37\%. By contrast, the percentage of firms with small jumps decreases as the relative magnitude of earnings surprises increases (i.e., the percentage of firms associated with SRVSJ decreases from 89.83\% to 88.29\%). This result indicates that ``big news'' is associated with an increase in the prevalence of large jumps. Second, the prevalence of jumps differs depending upon whether one tabulates results on earnings surprise days (Panel A) or on non-earnings surprise days (Panel B). For example, large news surprises are associated with large jumps for 31.07\% of firms on non-announcement days (see Panel B) and 37.38\% of firms on announcement days (Panel A). This result is consistent with event study finding that jump magnitudes are larger on announcement days than non-announcement days. 

It is also worth noting that Panel C of Table 15 reports t-statistics that test whether the differences in percentages of jumps in different portfolios are significant. In this table, ``None'' refers to the case where percentages are calculated on non-earnings-announcement days. Thus, the fact that the ``Large-None'' t-statistic associated with SRVLJ is 16.85, indicates that the percentage of large jumps on ``large-surprise'' earnings announcement days is significantly greater than the percentage of large jumps on non-earnings-announcement days. This in turn implies that large jumps tend to occur on ``large-surprise'' earnings announcement days. On the other hand, the reverse is true in the case of small jumps. In particular, the ``Large-None'' t-statistic associated with SRVSJ is -10.85, indicating that small jumps tend to occur on non-earnings-announcement days.

\subsubsection*{4.6.2  { } Systematic Versus Idiosyncratic Risks }

To further explore the unique information embedded in either large or small jump variation measures, and examine their association with systematic and idiosyncratic risks, we identify the effect of diversification on both small and large jumps. In order to do this, we construct two alternative measures of SRVLJ and SRVSL. The ratio of these is plotted in Figure 6.

$Method$ 1: For jump measures using this method, we simply construct SRVLJ and SRVSJ as done earlier in the paper. Namely, we sort stocks into quintiles based on either weekly SRVLJ or SRVSJ. Then, we construct daily ratios of SRVLJ to SRVSJ for each individual stock in a given quintile. Finally, these ratios are aggregated, forming weekly measures of SRVLJ/SRVSJ. These measures are then used to form equal- or value-weighted ratios of SRVLJ to SRVSJ. These values are depicted in red (solid line) in Figure 6.

$Method$ 2: For jump measures using this method, we start by constructing the same quintiles (based on weekly SRVLJ and SRVSJ) as done above. Then, we use the 5-minute returns for each stock in a given quintile in order to construct 5-minute aggregate portfolio returns for that quintile. We then construct daily jump measures using these portfolio returns (called SRVLJ and SRVSJ, and SRVLJ/SRVSJ), which are porfolio versions of the similar measures constructed using Method 1. Finally, daily measures are aggregated into weekly measures.  These value are depicted in blue (dotted line) in Figure 6.

Comparing jump variation ratios constructed in these two different ways allows us to explore the importance of diversification when measuring jump variation. Turning to our findings, Figure 6 shows the time series of aggregated (Method 2) and weighted average (Method 1) jump variation measures for the first quintile portfolios. The fact that Method 1 (red line) is much smoother than Method 2 (blue line) means that the small jump component in the ratio of SRVLJ/SRVSJ reamins much larger than in the other case. Thus, the obvious difference between aggregated and weighted averages of SRVLJ/SRVSJ indicates that small jump variation is more likely to be diversified away than large jump variation. This can be immediately seen upon examination of the plots in any of the four panels in the figure. Small jump variation is therefore more closely related to firm specific or idiosyncratic risks, while large jump variation is more likely to be systematic risks. 
\footnote{See the Supplementary Appendix for plots of jump variation measures for the other quintile portfolios.}

Another way to explore the relationship between systematic and idiosyncratic risks is to carry out Fama-MacBeth type regressions where the dependent variable is one of our jump variation measures and the independent variables are firm characteristics.\footnote{Specifically, our objective in this section is to discuss regressions of the form given in equation (\ref{eq13}), with the dependent variable replaced by various realized variables.} The results from a number of these sorts of regressions are reported in Table 16. Evidently, the firm characteristics always explain more of the dynamics associated with small jumps than with large jumps. This finding is supported by the fact that adjusted $R^2$ are higher when the dependent variable is a small jump variation measure (compare the results of regressions I and II with III and IV). This again suggests that small jump variation is more likely to be associated with idiosyncratic risks.
\footnote{See the Supplementary Appendix for results from double-sorted portfolios that condition on various control variables. In these tables, it is noteworthy that when stocks are first sorted by a control variable (e.g., illiquidity, volatility, firm size and reversal), the SRVJ (SRVLJ and SRVSJ) effect is much higher within quintile portfolios with high illiquidity, high volatility, small firm size, and low reversal. This result suggests that all of these control variables significantly contribute to the predictability of jump variation measures. This result provides additional confirmation to earlier findings reported in BLZ (2019).}

\section*{5  { } Concluding Remarks }

In this paper, we add to the literature that explores the relationship between equity returns and volatility. In particular, we focus on the strand of this literature that explores the data for evidence of asymmetry (non-linearity) in the return volatility trade-off. Following BLZ (2019), we decompose realized variation into upside and downside semi-variances (good and bad volatilities).  We then take the additional step of partitioning the semi-variances into small and large components. Within this context, we examine the marginal predictive content of small and large jump variation measures. We also examine the importance of earnings announcements for examining the linkages between small and large jumps and news.

We find that sorting on signed small jump variation leads to value weighted 
high-low portfolios 
with greater average returns and alphas than when either signed total jump or signed large jump variation is sorted on. We also find that there is a threshold, beyond which ``large'' jump variation contains no marginal predictive ability, relative to that contained in small jump variation. Indeed, including large jump variation can actually be detrimental to predictive accuracy, as average returns and alphas 
for high-low portfolios
actually decline when total variation is instead utilized in some of our prediction experiments. Analysis of returns and alphas based on industry double-sorts indicate that the benefit of small signed jump variation investing is driven by stock selection within an industry, rather than industry bets. Investors prefer stocks with a high probability of large positive jump variation, but they also tend to overweight safer industries. Additionally, we find that the content of signed large jump variation is negligible when controlling for either signed total jump variation or realized skewness. By contrast, signed small jump variation has unique information for predicting future returns, even when controlling for total jump variation or realized skewness. Finally, we find that large jumps are closely associated with ``big'' news, as might be expected. In particular, large earning announcement surprises increase both the magnitude and occurrence of large jumps. While such news related information is embedded in large jump variation, the information is generally short-lived, and dissipates too quickly to provide marginal predictive content for subsequent weekly returns. Moreover, while large jump variation is closely associated with large earnings surprises (``big'' news), small jumps tend to be more closely associated with idiosyncratic risks, and can be diversified away.

\newpage
\section*{References}

\begin{hangparas}{1em}{1} 

A\"{\i}t-Sahalia, Y., Jacod, J., 2012. Analyzing the spectrum of asset returns: Jump and volatility components in high frequency data. \textit{Journal of Economic Literature} 50, 1007-1050.

%A\"{\i}t-Sahalia, Y., Xiu, D., 2016. Increased correlation among asset classes: Are volatility or jumps to blame, or both? \textit{Journal of Econometrics} 194, 205-219.

Amaya, D., Christoffersen, P., Jacobs, K., Vasquez, A., 2015. Does realized skewness predict the cross-section of equity returns? \textit{Journal of Financial Economics} 118, 135-167.

Amihud, Y., 2002. Illiquidity and stock returns: Cross-section and time-series effects. \textit{Journal of Financial Markets} 5, 31-56.

Andersen, T. G., Bollerslev, T., Diebold, F. X., 2007. Roughing it up: Including jump components in the measurement, modeling, and forecasting of return volatility. \textit{Review of Economics and Statistics} 89, 701-720.

Andersen, T. G., Bollerlsev, T., Diebold, F. X., Labys, P., 2003. Modeling and forecasting realized volatility. \textit{Econometrica} 71, 579-625.

%Andersen, T. G., Dobrev, D., Schaumburg, E., 2012. Jump-robust volatility estimation using nearest neighbor truncation. \textit{Journal of Econometrics} 169, 75-93.

%Ang, A., Chen, J. and Xing, Y., 2006. Downside risk. \textit{Review of Financial Studies} 19 4, 1191-1239.

%Baker, M., Bradley, B., Taliaferro, R., 2014. The low-risk anomaly: A decomposition into micro and macro effects. \textit{Financial Analysts Journal} 70, 43-58.

Barndorff-Nielsen, O. E., Graversen, S. E., Jacod, J., Podolskij, M., Shephard,N., 2006. A central limit theorem for realised power and bipower variations of continuous semimartingales. In \textit{Stochastic Analysis to Mathematical Finance, The Shiryaev Festschrift} (Edited by Kabanov, Y., Liptser, R., and Stoyanov, J.), Springer Verlag, 33-68.

Barndorff-Nielsen, O. E., Kinnebrock, S., Shephard, N., 2010. Measuring downside risk: realised semivariance. In \textit{Volatility and Time Series Econometrics: Essays in Honor of Robert F. Engle} (Edited by T. Bollerslev, J. Russell and M. Watson), Oxford University Press, 117-136.

Barndorff-Nielsen, O. E., Shephard, N., 2004. Power and bipower variation with stochastic volatility and jumps. \textit{Journal of Financial Econometrics} 2, 1-37.

%Barndorff-Nielsen, O. E., Shephard, N., 2006. Econometrics of testing for jumps in financial economics using bipower variation. \textit{Journal of Financial Econometrics} 4, 1-30.

%Bollerslev, T., Li, S. Z., Todorov, V., 2016. Roughing up beta: Continuous vs. discontinuous betas, and the cross section of expected stock returns. \textit{Journal of Financial Economics} 120, 464-490.

Bernard, V., Thomas, J. K. , 1989, Post-earnings-announcement drift: delayed price response or risk premium? \textit{Journal of Accounting Research} 27, 1-36.

Bollerslev, T., Li, S. Z., Zhao, B., 2019. Good volatility, bad volatility, and the cross-section of stock returns.\textit{Journal of Financial and Quantitative Analysis}, forthcoming. 

%Bollerslev, T., Patton, A.J., Quaedvlieg, R., 2017. Realized semicovariances: Looking for signs of direction inside the covariance matrix. Working Paper, Duke University. 
%
%Bollerslev, T., Tauchen, G., Zhou, H., 2009. Expected stock returns and variance risk premia. \textit{Review of Financial Studies} 22, 4463-4492.

Bollerslev, T., Todorov, V., 2011. Estimation of jump tails. \textit{Econometrica} 79, 1727-1783.

%Bollerslev, T., Todorov, V., 2011b. Tails, fears, and risk premia. \textit{Journal of Finance} 66, 2165-2211.

Bollerslev, T., Todorov, V., Xu, L., 2015. Tail risk premia and return predictability. \textit{Journal of Financial Economics} 118, 113-134.

Campbell, J.Y. and Vuolteenaho, T., 2004, Bad beta, good beta.
\textit{American Economic Review} 94, 1249-1275.

%Chang, B. Y., Christoffersen, P., Jacobs, K., 2013. Market skewness risk and the cross section of stock returns. \textit{Journal of Financial Economics} 107, 46-68.

%Corsi, F., 2009. A simple approximate long-memory model of realized volatility. \textit{Journal of Financial Econometrics} 7, 174-196.

%Drechsler, I., Yaron, A., 2011.  What’s vol got to do with it. \textit{Review of Financial Studies} 24, 1–45.

Duong, D., Swanson, N. R., 2011, Volatility in discrete and continuous-time models: A survey with new evidence on small and large jumps. \textit{Missing Data Methods: Advances in Econometrics} 27, 179-233.

Duong, D., Swanson, N. R., 2015. Empirical evidence on the importance of aggregation, asymmetry, and jumps for volatility prediction. \textit{Journal of Econometrics} 187, 606-621.

%Fama, E. F., French, K. R., 1993. Common risk factors in the returns on stocks and bonds. \textit{Journal of Financial Economics} 33, 3-56.

%Fan, J., Furger, A., Xiu, D., 2016. Incorporating global industrial classification standard into portfolio allocation: A simple factor-based large covariance matrix estimator with high-frequency data. \textit{Journal of Business and Economic Statistics} 34, 489-503.

Fang, N., Wen, J., and Luo, R., 2017, Realized semivariances and the variation of signed jumps in China's stock market, \textit{Emerging Markets Finance and Trade} 53, 563-586.

Feunou, B., Jahan-Pravar, M. R., Okou, C., 2018. Downside variance risk premium. \textit{Journal of Financial Econometrics} 16, 341-83.

%Feunou, B., Okou, C., 2017. Good volatility, bad volatility, and option pricing. \textit{Journal of Financial and Quantitative Analysis}, forthcoming.

Feunou, B., Jahan-Pravar, M. R., T\'edongap, R., 2012. Modeling market downside volatility. \textit{Review of Finance} 17, 443-481.

Guo, H.,  Wang, K., Zhou H., 2015.  Good jumps, bad jumps, and conditional equity premium.  Working Paper, University of Cincinnati.

%Han, B., Zhou, Y., 2011. Variance risk premium and cross-section of stock returns. Working Paper, University of Texas, Austin.

%Hansen, P. R., Lunde, A., 2006. Realized variance and market microstructure noise. \textit{Journal of Business and Economic Statistics} 24, 127-161.

Jacod J., 2008. Asymptotic properties of realized power variations and related functionals of semimartingales. \textit{Stochastic Processes and Their Applications} 118, 517-559. 

%Jacod, J., Todorov, V., 2009. Testing for common arrivals of jumps for discretely observed multidimensional processes. \textit{Annals of Statistics} 37, 1792-1838.

Jiang, G., Yao, T., 2013. Stock price jumps and cross-sectional return predictability. \textit{Journal of Financial and Quantitative Analysis} 48, 1519-1544.

%Li, J., Todorov, V., Tauchen, G., 2017. Jump regressions. \textit{Econometrica} 85, 173-195.

Li, J., Todorov, V., Tauchen, G., Chen, R., 2017. Mixed-scale jump regressions with bootstrap inference. \textit{Journal of Econometrics} 201, 417-432.

Livnat, J., Mendenhall, R., 2006, Comparing the post-earnings announcement drift for surprises calculated from analyst and time-series forecasts. \textit{Journal of Accounting Research} 44, 177-205.

%Lo, A. W., MacKinlay, A. C., 1990. When are contrarian profits due to stock market overreaction? \textit{Review of Financial Studies} 3, 175-205.

Mancini, C., 2009. Non-parametric threshold estimation for models with stochastic diffusion coefficient and jumps. \textit{Scandinavian Journal of Statistics} 36, 270-296.

Maheu, J.M. McCurdy, T.H. (2004), News arrival, jump dynamics, and volatility components for individual stock returns. 
\textit{Journal of Finance} 59, 755-93.

%Nagel, S., 2012. Evaporating liquidity. \textit{Review of Financial Studies} 25, 2005-2039.

%Neuberger, A., 2012. Realized skewness. R\textit{eview of Financial Studies} 25, 3423-3455.

%Newey, W. K., West, K. D., 1987. A simple, positive semi-definite, heteroskedasticity and autocorrelation consistent covariance matrix. \textit{Econometrica} 55, 703-708.

Patton, A. J., Sheppard, K., 2015. Good volatility, bad volatility: Signed jumps and the persistence of volatility. \textit{Review of Economics and Statistics} 97, 683-697.

Petersen, M.A., 2009. Estimating standard errors in financial panel data sets: Comparing approaches. \textit{Review of Financial Studies} 22, 435-480.

Rossi, A., Timmermann, A., 2015, Modeling covariance risk in Merton's ICAPM. \textit{Review of Financial Studies} 28, 1428-61.

%Tauchen, G., Zhou, H., 2011. Realized jumps on financial markets and predicting credit spreads. \textit{Journal of Econometrics} 160, 102–118.

Todorov, V., Tauchen, G., 2010. Activity signature functions with application for high-frequency data analysis. \textit{Journal of Econometrics} 154, 125-138. 

%Todorov, V., Bollerslev, T., 2010. Jumps and betas: A new framework for disentangling and estimating
%systematic risks. \textit{Journal of Econometrics} 157, 220-235.

Woodward, G. and Anderson, H., 2009. Does beta react to market conditions? Estimates of bull and bear betas using a nonlinear market model with an endogenous threshold parameter. \textit{Quantitative Finance} 9, 913-924.

Xiong, J.X., Idzorek, T.M. and Ibbotson, R.G., 2016. The economic value of forecasting left-tail risk. \textit{The Journal of Portfolio Management} 42, 114-123.

Yan, S., 2011. Jump risk, stock returns, and slope of implied volatility smile. \textit{Journal of Financial Economics} 99, 216-233.
\end{hangparas}
%%%%%%%%%%%%%%%%%%%%%%%%%%%%%%%%%%%%%%%%%%%%%%%%%%%%%%%%%%%%%%%%%%%%%%%%%%%%%%%%%%%%%%%%%%%%%%%%%%%%%%%%%%%%%%%%%%%%%%%%%%%%%%%%%%%%%%%%%

\clearpage
%\newpage
%\section*{  { } Appendix }
%Controlling variables: 
%
%Beta: a firm's beta at the end of day t is computed using a regression based on daily returns over the past 252  days. 
%
%Size: market value of equity (in millions). Measured by the product of closing price and the number of outstanding shares. The market cap is updated daily. 
%
%BEME: following Fama and French (1993),it's the ratio of book common equity in fiscal year t-1 over market value of equity in December of year t-1. 
%
%Mom: a firm's momenturm at the end of day t is the compound gross return from day t-252 to day t-21.
%
%REV: a firm's short term reversal variable is measured by the weekly return in the past week from Tuesday to Monday.
%
%IVOL: idiosyncratic volatility is measured as the standard deviation of the error terms from the three-factor Fama French regresson using daily returns over the past 21 trading days. 
%$$r_{i,t}-r_{f,t}=\alpha_i+\beta_i(MKT_t-r_{f,t})+\eta_iSMB_t +\kappa_i HML_t +\epsilon_{i,t}$$ 
%where $r_{i,t}$ is the daily return of firm i at day t, therefore $IVOL_{i,t}=\sqrt{var(\epsilon_{i,t})}$. 
%
%CSK: coskewness of firm i at the end of day t is computed with daily returns over the past 252 trading days as 
%$$\widehat{CSK}_{i,t}=\frac{\frac{1}{N}\sum (r_{i,t}-\bar{r}_i)(r_{m,t}-\bar{r}_m)^2}{\sqrt{\frac{1}{N}\sum (r_{i,t}-\bar{r}_i)^2}(\frac{1}{N}\sum (r_{m,t}-\bar{r}_m)^2)}$$ 
%
%where $r_{i,t}$ is the daily return of firm i at day t, and $r_{m,t}$ is the daily return of market portfolio at day t. $\bar{r}_i$ and $\bar{r}_m$ denote the average daily return of stock i and the market portfolio, respectively. 
%
%CKT: cokurtosis of firm i at the end of day t is computed with daily returns over the past 252 trading days as 
%$$\widehat{CKT}_{i,t}=\frac{\frac{1}{N}\sum (r_{i,t}-\bar{r}_i)(r_{m,t}-\bar{r}_m)^3}{\sqrt{\frac{1}{N}\sum (r_{i,t}-\bar{r}_i)^2}(\frac{1}{N}\sum (r_{m,t}-\bar{r}_m)^2)^{\frac{3}{2}}}$$
%
%MAX: maximum return is measured as the maximum daily return over the past week. 
%
%MIN: minimum return is defined as the mimimum daily return over the past week. 
%
%ILLIQ: illiquidity of firm i at day t is measured as the average of the ratio of absolute daily return over the dollar trading volume over the past 252 trading days: 
%$$ILLIQ=\frac{1}{N}\sum_{d=0}^{N}(\frac{|r_{i,t-d}|}{volume_{i,t-d}*price_{i,t-d}})$$
%
%where $volume_{i,t-d}$ is the daily trading volume of stock i at day t-d. 

\newpage

\centering
\noindent Table 1: Realized Measures and Firm Characteristics$^*\>$
% Table generated by Excel2LaTeX from sheet 'notation'
\begin{table}[htbp!]
	\centering
	\resizebox{1.04\textwidth}{!}{
		\begin{threeparttable}
			\begin{tabular}{llrrrrrr}
				\multicolumn{8}{l}{Panel A: Realized Measures Used in Portfolio Sorts and Fama-MacBeth Regressions} \\
				&  &       &       &       &       &       &  \\
				\midrule
				RVJP  & Positive (upside) jump variation, see (\ref{eq1}).  &       &       &       &       &       &  \\
				RVJN  & Negative (downside) jump variation, see (\ref{eq2}).&       &       &       &       &       &  \\
				SRVJ  & Signed jump variation, $RVJP-RVJN$, see (\ref{eq3}).&     &       &       &       &       &    \\
				RVLJP & Positive (upside) large jump variation, see (\ref{eq4}). &       &       &       &       &       &  \\
				RVLJN & Negative (downside) large jump variation, see (\ref{eq5}). &       &       &       &       &       &  \\
				SRVLJ & Signed large jump variation, $RVLJP-RVLJN$, see (\ref{eq6}). &       &       &       &       &       &  \\
				RVSJP & Positive (upside) small jump variation, see (\ref{eq7}). &       &       &       &       &       &  \\
				RVSJN & Negative (downside) small jump variation, see (\ref{eq8}). &       &       &       &       &       &  \\
				SRVSJ & Signed small jump variation, $RVSJP-RVSJN$, see (\ref{eq9}). &       &       &       &       &       &  \\
				RVOL  & Realized volatility &       &       &       &       &       &  \\
				RSK   & Realized skewness, see (\ref{eq 12}). &       &       &       &       &       &  \\
				RKT   & Realized kurtosis, see (\ref{eq 13}). &       &       &       &       &       &  \\
				\midrule
				&       &       &       &       &       &       &  \\
				
				\multicolumn{8}{l}{Panel B: Explanatory Variables and Firm Characteristics Used in Fama-MacBeth Regressions} \\
				&  &       &       &       &       &       &  \\
				\midrule
				BETA  & Market beta &       &       &       &       &       &  \\
				log(Size)  & Natural logarithm of firm size &       &       &       &       &       &  \\
				BEME  & Book-to-market ratio &       &       &       &       &       &  \\
				MOM   & Momentum &       &       &       &       &       &  \\
				REV   & Short-term reversal &       &       &       &       &       &  \\
				IVOL  & Idiosyncratic volatility &       &       &       &       &       &  \\
				CSK   & Coskewness &       &       &       &       &       &  \\
				CKT   & Cokurtosis &       &       &       &       &       &  \\
				MAX   & Maximum daily return &       &       &       &       &       &  \\
				MIN   & Minimum daily return &       &       &       &       &       &  \\
				ILLIQ  & Illiquidity &       &       &       &       &       &  \\
				\midrule
			\end{tabular}
			\begin{tablenotes}
				\item {}
				\item {$^*\>$ Notes: The realized measures listed in Panel A of this table are defined and discussed in Section 2. For detailed descriptions of the explanatory variables and firm characteristics listed in Panel B of this table, refer to Bollerslev, Li, and Zhao (2017), and the references cited therein.}
			\end{tablenotes}
	\end{threeparttable}}
\end{table}
\newpage
\noindent Table 2: Summary  Statistics for Various Realized Measures and Firm Characteristics Based on Two Jump Truncation Levels$^*\>$
%\footnote{We measure the cross-sectional bid-ask spreads using one month quote data. The resutls show that the bid-ask spreads for S\&P 500 stocks are much smaller than those for non S\&P stocks, indicating that stocks for large firms are more liquid. } 

\large
\begin{table}[htbp!]
	\centering
	%\caption*{\normalfont Panel A: Cross-Sectional Summary Statistics}
	\resizebox{1.04\textwidth}{!}{
		\begin{tabular}{lccccccccccccccccccccccc}
			\multicolumn{24}{l}{Panel A: Cross-Sectional Summary Statistics}\\
			&      &      &      &      &      &  & & & & & & & & & &
			&       &     &      &      &      &      & 
			\\
			\midrule
			& SRVJ  & RVJP  & RVJN  & SRVLJ & RVLJP & RVLJN & SRVSJ & RVSJP & RVSJN & RVOL  & RSK   & RKT   & BETA  & log(Size) & BEME  & MOM   & REV   & IVOL  & CSK   & CKT   & MAX   & MIN   & ILLIQ \\
			&       &       &       &       &       &       &       &       &       &       &       &       &       &       &       &       &       &       &       &       &       &       &  \\
			\multicolumn{24}{l}{Part I: Jump Truncation Level=   $\gamma^{1}$}\\
			Mean & 0.0061  & 0.2698  & 0.2637  & 0.0045  & 0.1518  & 0.1472  & 0.0015  & 0.1180  & 0.1165  & 0.9489  & 0.0288  & 8.2569  & 1.0794  & 6.5280  & 0.5969  & 2023.8456  & 70.6077  & 0.0293  & -0.0263  & 1.1438  & 412.1094  & -346.7608  & -5.2826  \\
		Std & 0.1537  & 0.1350  & 0.1347  & 0.1424  & 0.1555  & 0.1523  & 0.0635  & 0.0783  & 0.0783  & 2.1211  & 0.8159  & 4.5706  & 0.5566  & 1.8359  & 0.7224  & 7464.5273  & 927.3551  & 0.0250  & 0.3283  & 0.8474  & 572.1454  & 359.6789  & 2.4047  \\
			&       &       &       &       &       &       &       &       &       &       &       &       &       &       &       &       &       &       &       &       &       &       &  \\
			\multicolumn{24}{l}{Part II: Jump Truncation Level=   $\gamma^{2}$}\\
				Mean & 0.0061  & 0.2698  & 0.2637  & 0.0029  & 0.0983  & 0.0954  & 0.0031  & 0.1715  & 0.1684  & 0.9489  & 0.0288  & 8.2569  & 1.0794  & 6.5280  & 0.5969  & 2023.8456  & 70.6077  & 0.0293  & -0.0263  & 1.1438  & 412.1094  & -346.7608  & -5.2826  \\
			Std & 0.1537  & 0.1350  & 0.1347  & 0.1303  & 0.1401  & 0.1368  & 0.0859  & 0.0911  & 0.0909  & 2.1211  & 0.8159  & 4.5706  & 0.5566  & 1.8359  & 0.7224  & 7464.5273  & 927.3551  & 0.0250  & 0.3283  & 0.8474  & 572.1454  & 359.6789  & 2.4047  \\
			\bottomrule
	\end{tabular}}
\end{table}
% Table generated by Excel2LaTeX from sheet 'mean_t4_15t80'
\begin{table}[htbp!]
	
	\centering
	\resizebox{1.04\textwidth}{!}{
		\begin{threeparttable}
			\begin{tabular}{lccccccccccccccccccccccc}
				\multicolumn{24}{l}{Panel B: Cross-Sectional Correlations}\\
				&      &      &      &      &      &  & & & & & & & & & &
				&       &     &      &      &      &      & 
				\\
				\midrule
				& SRVJ  & RVJP  & RVJN  & SRVLJ & RVLJP & RVLJN & SRVSJ & RVSJP & RVSJN & RVOL  & RSK   & RKT   & BETA  & log(Size) & BEME  & MOM   & REV   & IVOL  & CSK   & CKT   & MAX   & MIN   & ILLIQ \\
				&       &       &       &       &       &       &       &       &       &       &       &       &       &       &       &       &       &       &       &       &       &       &  \\
				\multicolumn{24}{l}{Part I: Jump Truncation Level=   $\gamma^{1}$}\\
				SRVJ & 1.00 & 0.57 & -0.57 & 0.91 & 0.43 & -0.40 & 0.37 & 0.13 & -0.18 & -0.02 & 0.94 & 0.03 & -0.03 & 0.01 & 0.01 & 0.01 & 0.30 & -0.03 & 0.09 & 0.00 & 0.17 & 0.22 & 0.00 \\
				RVJP &    & 1.00 & 0.33 & 0.52 & 0.85 & 0.37 & 0.21 & 0.04 & -0.13 & 0.22 & 0.54 & 0.45 & -0.26 & -0.49 & 0.14 & -0.10 & 0.15 & 0.12 & 0.04 & -0.24 & 0.15 & 0.06 & 0.56 \\
				RVJN &    &    & 1.00 & -0.52 & 0.35 & 0.84 & -0.22 & -0.10 & 0.08 & 0.24 & -0.54 & 0.41 & -0.23 & -0.49 & 0.13 & -0.11 & -0.19 & 0.15 & -0.06 & -0.24 & -0.05 & -0.20 & 0.55 \\
				SRVLJ &    &    &    & 1.00 & 0.48 & -0.44 & -0.04 & -0.05 & -0.01 & -0.01 & 0.92 & 0.03 & -0.02 & 0.00 & 0.01 & 0.00 & 0.20 & -0.02 & 0.05 & 0.00 & 0.12 & 0.16 & 0.00 \\
				RVLJP &    &    &    &    & 1.00 & 0.57 & -0.02 & -0.46 & -0.45 & 0.23 & 0.44 & 0.61 & -0.25 & -0.47 & 0.12 & -0.06 & 0.09 & 0.13 & 0.02 & -0.24 & 0.12 & -0.02 & 0.54 \\
				RVLJN &    &    &    &    &    & 1.00 & 0.01 & -0.44 & -0.45 & 0.24 & -0.41 & 0.59 & -0.23 & -0.48 & 0.11 & -0.06 & -0.10 & 0.15 & -0.03 & -0.24 & 0.01 & -0.17 & 0.54 \\
				SRVSJ &    &    &    &    &    &    & 1.00 & 0.42 & -0.42 & -0.02 & 0.19 & 0.00 & -0.03 & 0.01 & 0.01 & 0.01 & 0.26 & -0.03 & 0.08 & 0.00 & 0.14 & 0.18 & 0.00 \\
				RVSJP &    &    &    &    &    &    &    & 1.00 & 0.64 & -0.04 & 0.06 & -0.40 & 0.06 & 0.03 & 0.01 & -0.05 & 0.10 & -0.03 & 0.04 & 0.04 & 0.02 & 0.11 & -0.06 \\
				RVSJN &    &    &    &    &    &    &    &    & 1.00 & -0.03 & -0.10 & -0.40 & 0.08 & 0.02 & 0.01 & -0.06 & -0.12 & -0.01 & -0.03 & 0.03 & -0.09 & -0.04 & -0.06 \\
				RVOL &    &    &    &    &    &    &    &    &    & 1.00 & -0.01 & 0.22 & -0.05 & -0.55 & 0.08 & -0.12 & 0.06 & 0.56 & -0.01 & -0.27 & 0.44 & -0.47 & 0.56 \\
				RSK &    &    &    &    &    &    &    &    &    &    & 1.00 & 0.04 & -0.02 & 0.00 & 0.01 & 0.00 & 0.22 & -0.02 & 0.06 & 0.00 & 0.13 & 0.17 & 0.00 \\
				RKT &    &    &    &    &    &    &    &    &    &    &    & 1.00 & -0.20 & -0.34 & 0.09 & -0.02 & 0.00 & 0.10 & -0.01 & -0.19 & 0.08 & -0.10 & 0.40 \\
				BETA &    &    &    &    &    &    &    &    &    &    &    &    & 1.00 & 0.10 & -0.09 & 0.00 & -0.04 & 0.06 & 0.01 & 0.30 & 0.03 & -0.09 & -0.16 \\
				ME &    &    &    &    &    &    &    &    &    &    &    &    &    & 1.00 & -0.19 & 0.11 & -0.05 & -0.52 & 0.01 & 0.40 & -0.32 & 0.35 & -0.93 \\
				BEME &    &    &    &    &    &    &    &    &    &    &    &    &    &    & 1.00 & 0.03 & 0.02 & 0.05 & 0.00 & -0.06 & 0.05 & -0.03 & 0.18 \\
				MOM &    &    &    &    &    &    &    &    &    &    &    &    &    &    &    & 1.00 & 0.00 & -0.08 & -0.07 & 0.06 & -0.05 & 0.05 & -0.15 \\
				REV &    &    &    &    &    &    &    &    &    &    &    &    &    &    &    &    & 1.00 & 0.12 & 0.16 & -0.04 & 0.49 & 0.29 & 0.05 \\
				IVOL &    &    &    &    &    &    &    &    &    &    &    &    &    &    &    &    &    & 1.00 & 0.02 & -0.35 & 0.50 & -0.47 & 0.47 \\
				CSK &    &    &    &    &    &    &    &    &    &    &    &    &    &    &    &    &    &    & 1.00 & 0.01 & 0.07 & 0.07 & 0.00 \\
				CKT &    &    &    &    &    &    &    &    &    &    &    &    &    &    &    &    &    &    &    & 1.00 & -0.16 & 0.15 & -0.37 \\
				MAX &    &    &    &    &    &    &    &    &    &    &    &    &    &    &    &    &    &    &    &    & 1.00 & -0.28 & 0.34 \\
				MIN &    &    &    &    &    &    &    &    &    &    &    &    &    &    &    &    &    &    &    &    &    & 1.00 & -0.35 \\
				ILLIQ &    &    &    &    &    &    &    &    &    &    &    &    &    &    &    &    &    &    &    &    &    &    & 1.00 \\
				\midrule
				%				&       &       &       &       &       &       &       &       &       &       &       &       &       &       &       &       &       &       &       &       &       &       &  \\
				%				II:  $\gamma^{(2)}$ &       &       &       &       &       &       &       &       &       &       &       &       &       &       &       &       &       &       &       &       &       &       &  \\
				&       &       &       &       &       &       &       &       &       &       &       &       &       &       &       &       &       &       &       &       &       &       &  \\
				&       &       &       &       &       &       &       &       &       &       &       &       &       &       &       &       &       &       &       &       &       &       &  \\
				\midrule
				& SRVJ  & RVJP  & RVJN  & SRVLJ & RVLJP & RVLJN & SRVSJ & RVSJP & RVSJN & RVOL  & RSK   & RKT   & BETA  & log(Size) & BEME  & MOM   & REV   & IVOL  & CSK   & CKT   & MAX   & MIN   & ILLIQ \\
				&       &       &       &       &       &       &       &       &       &       &       &       &       &       &       &       &       &       &       &       &       &       &  \\
				\multicolumn{24}{l}{Part II: Jump Truncation Level=   $\gamma^{2}$}\\
				SRVJ & 1.00 & 0.57 & -0.57 & 0.83 & 0.40 & -0.37 & 0.52 & 0.23 & -0.27 & -0.02 & 0.94 & 0.03 & -0.03 & 0.01 & 0.01 & 0.01 & 0.30 & -0.03 & 0.09 & 0.00 & 0.17 & 0.22 & 0.00 \\
				RVJP &    & 1.00 & 0.33 & 0.48 & 0.77 & 0.33 & 0.30 & 0.30 & 0.01 & 0.22 & 0.54 & 0.45 & -0.26 & -0.49 & 0.14 & -0.10 & 0.15 & 0.12 & 0.04 & -0.24 & 0.15 & 0.06 & 0.56 \\
				RVJN &    &    & 1.00 & -0.47 & 0.31 & 0.75 & -0.31 & 0.03 & 0.33 & 0.24 & -0.54 & 0.41 & -0.23 & -0.49 & 0.13 & -0.11 & -0.19 & 0.15 & -0.06 & -0.24 & -0.05 & -0.20 & 0.55 \\
				SRVLJ &    &    &    & 1.00 & 0.49 & -0.44 & -0.04 & -0.05 & -0.01 & -0.01 & 0.89 & 0.04 & -0.01 & 0.00 & 0.01 & 0.00 & 0.16 & -0.02 & 0.04 & 0.00 & 0.09 & 0.13 & 0.00 \\
				RVLJP &    &    &    &    & 1.00 & 0.56 & -0.02 & -0.36 & -0.34 & 0.20 & 0.43 & 0.64 & -0.24 & -0.40 & 0.11 & -0.05 & 0.06 & 0.11 & 0.01 & -0.22 & 0.10 & -0.02 & 0.47 \\
				RVLJN &    &    &    &    &    & 1.00 & 0.01 & -0.34 & -0.35 & 0.21 & -0.40 & 0.62 & -0.23 & -0.41 & 0.10 & -0.06 & -0.09 & 0.13 & -0.03 & -0.22 & 0.01 & -0.14 & 0.47 \\
				SRVSJ &    &    &    &    &    &    & 1.00 & 0.47 & -0.47 & -0.02 & 0.32 & 0.00 & -0.03 & 0.01 & 0.01 & 0.01 & 0.30 & -0.03 & 0.09 & 0.00 & 0.16 & 0.21 & 0.00 \\
				RVSJP &    &    &    &    &    &    &    & 1.00 & 0.55 & 0.03 & 0.13 & -0.30 & -0.02 & -0.14 & 0.04 & -0.07 & 0.13 & 0.02 & 0.04 & -0.04 & 0.07 & 0.10 & 0.13 \\
				RVSJN &    &    &    &    &    &    &    &    & 1.00 & 0.05 & -0.17 & -0.30 & 0.01 & -0.15 & 0.04 & -0.07 & -0.15 & 0.05 & -0.04 & -0.04 & -0.08 & -0.09 & 0.13 \\
				RVOL &    &    &    &    &    &    &    &    &    & 1.00 & -0.01 & 0.22 & -0.05 & -0.55 & 0.08 & -0.12 & 0.06 & 0.56 & -0.01 & -0.27 & 0.44 & -0.47 & 0.56 \\
				RSK &    &    &    &    &    &    &    &    &    &    & 1.00 & 0.04 & -0.02 & 0.00 & 0.01 & 0.00 & 0.22 & -0.02 & 0.06 & 0.00 & 0.13 & 0.17 & 0.00 \\
				RKT &    &    &    &    &    &    &    &    &    &    &    & 1.00 & -0.20 & -0.34 & 0.09 & -0.02 & 0.00 & 0.10 & -0.01 & -0.19 & 0.08 & -0.10 & 0.40 \\
				BETA &    &    &    &    &    &    &    &    &    &    &    &    & 1.00 & 0.10 & -0.09 & 0.00 & -0.04 & 0.06 & 0.01 & 0.30 & 0.03 & -0.09 & -0.16 \\
				ME &    &    &    &    &    &    &    &    &    &    &    &    &    & 1.00 & -0.19 & 0.11 & -0.05 & -0.52 & 0.01 & 0.40 & -0.32 & 0.35 & -0.93 \\
				BEME &    &    &    &    &    &    &    &    &    &    &    &    &    &    & 1.00 & 0.03 & 0.02 & 0.05 & 0.00 & -0.06 & 0.05 & -0.03 & 0.18 \\
				MOM &    &    &    &    &    &    &    &    &    &    &    &    &    &    &    & 1.00 & 0.00 & -0.08 & -0.07 & 0.06 & -0.05 & 0.05 & -0.15 \\
				REV &    &    &    &    &    &    &    &    &    &    &    &    &    &    &    &    & 1.00 & 0.12 & 0.16 & -0.04 & 0.49 & 0.29 & 0.05 \\
				IVOL &    &    &    &    &    &    &    &    &    &    &    &    &    &    &    &    &    & 1.00 & 0.02 & -0.35 & 0.50 & -0.47 & 0.47 \\
				CSK &    &    &    &    &    &    &    &    &    &    &    &    &    &    &    &    &    &    & 1.00 & 0.01 & 0.07 & 0.07 & 0.00 \\
				CKT &    &    &    &    &    &    &    &    &    &    &    &    &    &    &    &    &    &    &    & 1.00 & -0.16 & 0.15 & -0.37 \\
				MAX &    &    &    &    &    &    &    &    &    &    &    &    &    &    &    &    &    &    &    &    & 1.00 & -0.28 & 0.34 \\
				MIN &    &    &    &    &    &    &    &    &    &    &    &    &    &    &    &    &    &    &    &    &    & 1.00 & -0.35 \\
				ILLIQ &    &    &    &    &    &    &    &    &    &    &    &    &    &    &    &    &    &    &    &    &    &    & 1.00 \\
				\bottomrule
			\end{tabular}
			
			\begin{tablenotes}
				\item {}
				\item {\large $^*\>$ Notes: See notes to Table 1. This table presents cross-sectional summary statistics and correlations for all realized measures and control variables based on two truncation levels:   $\>\gamma^{1}=4\sqrt{\frac{1}{t}\widehat{IV}_t^{(i)}}\Delta_n^{0.49} \>$ and $\>\gamma^{2}=5\sqrt{\frac{1}{t}\widehat{IV}_t^{(i)}}\Delta_n^{0.49}\>$. The entries in the table for realized measures (see columns 2-13) are constructed using 5-min intraday high frequency data. Entries for firm characteristics (see columns 14-24) are constructed using daily data, with the exception of BEME, which is constructed using monthly data. For complete details, see Sections 3 and 4.}
			\end{tablenotes}
	\end{threeparttable}}
	
\end{table}

\newpage
\large
\centering
\noindent Table 3: Realized Measures and Firm Characteristics of Portfolios Sorted by Various Realized Measures
% Table generated by Excel2LaTeX from sheet '4srvj'

\begin{table}[htbp!]
	\centering
	%\caption*{\normalfont Panel A: Stocks Sorted by SRVJ}
	\large
	\resizebox{1.04\textwidth}{!}{
		\begin{tabular}{lccccccccccccccccccccccc}
			\multicolumn{24}{l}{Panel A: Stocks Sorted by SRVJ}\\
			&      &      &      &      &      &  & & & & & & & & & &
			&       &     &      &      &      &      & 
			\\
			\toprule
			Quintile & RVJP  & RVJN  & RVLJP & RVLJN & RVSJP & RVSJN & SRVLJ & SRVSJ & SRVJ  & RVOL  & RSK   & RKT   & BETA  & log(Size) & BEME  & MOM   & REV   & IVOL  & CSK   & CKT   & MAX   & MIN   & ILLIQ \\
			%\midrule
				1  & 0.2021 & 0.3959 & 0.1161 & 0.2723 & 0.0860 & 0.1235 & -0.1563 & -0.0375 & -0.1938 & 0.9394 & -0.9324 & 9.8720 & 1.0369 & 6.1326 & 0.6235 & 0.2006 & -0.0363 & 0.0317 & -0.0708 & 1.0482 & 0.0295 & -0.0494 & -4.6903 \\
		2  & 0.2200 & 0.2777 & 0.1015 & 0.1391 & 0.1185 & 0.1385 & -0.0376 & -0.0200 & -0.0576 & 0.9513 & -0.2504 & 7.1729 & 1.1351 & 6.7041 & 0.5711 & 0.2051 & -0.0111 & 0.0293 & -0.0443 & 1.1987 & 0.0346 & -0.0383 & -5.5835 \\
		3  & 0.2435 & 0.2399 & 0.1127 & 0.1103 & 0.1308 & 0.1296 & 0.0023 & 0.0012 & 0.0036 & 1.0360 & 0.0194 & 6.9162 & 1.1301 & 6.8171 & 0.5695 & 0.2030 & 0.0077 & 0.0283 & -0.0242 & 1.2222 & 0.0397 & -0.0328 & -5.7301 \\
		4  & 0.2801 & 0.2138 & 0.1441 & 0.1005 & 0.1360 & 0.1132 & 0.0436 & 0.0227 & 0.0663 & 0.9162 & 0.2954 & 7.2497 & 1.1096 & 6.7855 & 0.5778 & 0.2097 & 0.0266 & 0.0278 & -0.0066 & 1.2055 & 0.0454 & -0.0282 & -5.6668 \\
		5  & 0.4035 & 0.1914 & 0.2846 & 0.1138 & 0.1189 & 0.0775 & 0.1708 & 0.0413 & 0.2121 & 0.9018 & 1.0138 & 10.0739 & 0.9851 & 6.2007 & 0.6427 & 0.1936 & 0.0485 & 0.0293 & 0.0145 & 1.0443 & 0.0569 & -0.0247 & -4.7419 \\
			\bottomrule
	\end{tabular}}
\end{table}




\begin{table}[htbp!]
	\centering
	%\caption*{\normalfont Panel D: Stocks Sorted by SRVLJ}
	\large
	\resizebox{1.04\textwidth}{!}{
		\begin{tabular}{lccccccccccccccccccccccc}
			\multicolumn{24}{l}{Panel B: Stocks Sorted by SRVLJ}\\
			&      &      &      &      &      &  & & & & & & & & & &
			&       &     &      &      &      &      & 
			\\
			\toprule
			Quintile & RVJP  & RVJN  & RVLJP & RVLJN & RVSJP & RVSJN & SRVLJ & SRVSJ & SRVJ  & RVOL  & RSK   & RKT   & BETA  & log(Size) & BEME  & MOM   & REV   & IVOL  & CSK   & CKT   & MAX   & MIN   & ILLIQ \\
			\multicolumn{24}{l}{Part I: Jump Truncation Level=   $\gamma^{1}$}\\
			1  & 0.2255 & 0.3959 & 0.1250 & 0.3016 & 0.1005 & 0.0943 & -0.1766 & 0.0062 & -0.1703 & 0.9429 & -0.9075 & 10.2967 & 1.0163 & 6.0403 & 0.6329 & 0.1966 & -0.0221 & 0.0316 & -0.0548 & 1.0357 & 0.0347 & -0.0461 & -4.5515 \\
			2  & 0.2222 & 0.2603 & 0.0874 & 0.1297 & 0.1347 & 0.1307 & -0.0422 & 0.0041 & -0.0382 & 0.9128 & -0.2003 & 6.9378 & 1.1314 & 6.7639 & 0.5678 & 0.2104 & -0.0011 & 0.0288 & -0.0329 & 1.2050 & 0.0374 & -0.0357 & -5.6652 \\
			3  & 0.2296 & 0.2252 & 0.0968 & 0.0936 & 0.1329 & 0.1316 & 0.0032 & 0.0013 & 0.0044 & 1.1510 & 0.0218 & 6.5571 & 1.1308 & 6.9059 & 0.5648 & 0.1997 & 0.0069 & 0.0282 & -0.0245 & 1.2380 & 0.0390 & -0.0326 & -5.8581 \\
			4  & 0.2726 & 0.2229 & 0.1493 & 0.0981 & 0.1233 & 0.1248 & 0.0512 & -0.0016 & 0.0497 & 0.8314 & 0.2602 & 7.2170 & 1.1237 & 6.7993 & 0.5764 & 0.2118 & 0.0169 & 0.0280 & -0.0166 & 1.2086 & 0.0421 & -0.0304 & -5.6962 \\
			5  & 0.4041 & 0.2153 & 0.3149 & 0.1234 & 0.0891 & 0.0920 & 0.1915 & -0.0028 & 0.1887 & 0.9158 & 0.9900 & 10.5179 & 0.9873 & 6.0946 & 0.6471 & 0.1887 & 0.0354 & 0.0299 & -0.0021 & 1.0376 & 0.0533 & -0.0284 & -4.6003 \\
			&       &       &       &       &       &       &       &       &       &       &       &       &       &       &       &       &       &       &       &       &       &       &  \\
			\multicolumn{24}{l}{Part II: Jump Truncation Level=   $\gamma^{2}$}\\
			1  & 0.2412 & 0.3897 & 0.0841 & 0.2403 & 0.1571 & 0.1494 & -0.1561 & 0.0076 & -0.1485 & 0.9355 & -0.8465 & 10.5599 & 1.0128 & 6.0603 & 0.6360 & 0.1928 & -0.0152 & 0.0313 & -0.0478 & 1.0403 & 0.0366 & -0.0438 & -4.5663 \\
			2  & 0.2189 & 0.2314 & 0.0341 & 0.0519 & 0.1848 & 0.1794 & -0.0178 & 0.0054 & -0.0124 & 0.9321 & -0.0762 & 6.4172 & 1.1366 & 6.8798 & 0.5622 & 0.2132 & 0.0045 & 0.0283 & -0.0278 & 1.2310 & 0.0388 & -0.0340 & -5.8322 \\
			3  & 0.2438 & 0.2405 & 0.0805 & 0.0770 & 0.1633 & 0.1635 & 0.0035 & -0.0001 & 0.0033 & 1.6847 & 0.0202 & 7.7437 & 1.0936 & 6.5831 & 0.6248 & 0.1664 & 0.0063 & 0.0322 & -0.0021 & 1.2281 & 0.0426 & -0.0360 & -5.3661 \\
			4  & 0.2985 & 0.2611 & 0.1233 & 0.0854 & 0.1752 & 0.1756 & 0.0379 & -0.0004 & 0.0375 & 1.0724 & 0.2140 & 8.0742 & 1.0912 & 6.4686 & 0.5995 & 0.1944 & 0.0125 & 0.0294 & -0.0213 & 1.1478 & 0.0418 & -0.0324 & -5.2346 \\
			5  & 0.3985 & 0.2321 & 0.2526 & 0.0837 & 0.1459 & 0.1484 & 0.1690 & -0.0025 & 0.1664 & 0.9201 & 0.9332 & 10.8242 & 0.9935 & 6.1033 & 0.6466 & 0.1896 & 0.0285 & 0.0300 & -0.0084 & 1.0418 & 0.0506 & -0.0301 & -4.6103 \\
			\bottomrule
	\end{tabular}}
\end{table}

\begin{table}[htbp!]
	
	\centering
	%\caption*{\normalfont Panel G: Stocks Sorted by SRVSJ}
	\large
	\resizebox{1.04\textwidth}{!}{
		\begin{tabular}{lccccccccccccccccccccccc}
			\multicolumn{24}{l}{Panel C: Stocks Sorted by SRVSJ}\\
			&      &      &      &      &      &  & & & & & & & & & &
			&       &     &      &      &      &      & 
			\\
			\toprule
			Quintile & RVJP  & RVJN  & RVLJP & RVLJN & RVSJP & RVSJN & SRVLJ & SRVSJ & SRVJ  & RVOL  & RSK   & RKT   & BETA  & log(Size) & BEME  & MOM   & REV   & IVOL  & CSK   & CKT   & MAX   & MIN   & ILLIQ \\
			\multicolumn{24}{l}{Part I: Jump Truncation Level=   $\gamma^{1}$}\\
			1  & 0.2060 & 0.2784 & 0.1085 & 0.0968 & 0.0975 & 0.1815 & 0.0116 & -0.0840 & -0.0724 & 0.7622 & -0.1832 & 7.1817 & 1.1514 & 6.7378 & 0.5654 & 0.2183 & -0.0270 & 0.0292 & -0.0637 & 1.2017 & 0.0302 & -0.0438 & -5.6348 \\
			2  & 0.2711 & 0.2845 & 0.1758 & 0.1649 & 0.0952 & 0.1197 & 0.0110 & -0.0244 & -0.0135 & 1.0260 & -0.0157 & 8.6959 & 1.0715 & 6.4139 & 0.6062 & 0.1933 & -0.0046 & 0.0302 & -0.0417 & 1.1242 & 0.0380 & -0.0375 & -5.1168 \\
			3  & 0.3077 & 0.3003 & 0.2186 & 0.2126 & 0.0891 & 0.0877 & 0.0060 & 0.0014 & 0.0074 & 1.2309 & 0.0345 & 9.7978 & 1.0053 & 6.1262 & 0.6382 & 0.1754 & 0.0066 & 0.0309 & -0.0275 & 1.0469 & 0.0423 & -0.0356 & -4.6619 \\
			4  & 0.2743 & 0.2492 & 0.1413 & 0.1438 & 0.1330 & 0.1053 & -0.0026 & 0.0277 & 0.0251 & 0.9562 & 0.0709 & 8.0830 & 1.0906 & 6.6614 & 0.5854 & 0.2098 & 0.0190 & 0.0283 & -0.0099 & 1.1752 & 0.0436 & -0.0305 & -5.4837 \\
			5  & 0.2830 & 0.1981 & 0.1004 & 0.1051 & 0.1826 & 0.0930 & -0.0047 & 0.0896 & 0.0849 & 0.7380 & 0.2395 & 7.1974 & 1.0947 & 6.8064 & 0.5783 & 0.2217 & 0.0428 & 0.0272 & 0.0136 & 1.1959 & 0.0517 & -0.0251 & -5.6801 \\
			&       &       &       &       &       &       &       &       &       &       &       &       &       &       &       &       &       &       &       &       &       &       &  \\
			\multicolumn{24}{l}{Part II: Jump Truncation Level=   $\gamma^{2}$}\\
			1  & 0.2058 & 0.3081 & 0.0749 & 0.0650 & 0.1309 & 0.2431 & 0.0099 & -0.1121 & -0.1023 & 0.8104 & -0.3171 & 7.6436 & 1.1207 & 6.5435 & 0.5792 & 0.2179 & -0.0318 & 0.0301 & -0.0682 & 1.1558 & 0.0296 & -0.0461 & -5.3493 \\
			2  & 0.2561 & 0.2841 & 0.1103 & 0.1027 & 0.1458 & 0.1814 & 0.0077 & -0.0356 & -0.0280 & 1.0119 & -0.0747 & 8.4222 & 1.0918 & 6.5103 & 0.5957 & 0.1964 & -0.0079 & 0.0298 & -0.0427 & 1.1482 & 0.0363 & -0.0379 & -5.2661 \\
			3  & 0.2906 & 0.2847 & 0.1423 & 0.1381 & 0.1484 & 0.1465 & 0.0041 & 0.0019 & 0.0060 & 1.1729 & 0.0308 & 9.3041 & 1.0385 & 6.3412 & 0.6204 & 0.1788 & 0.0065 & 0.0301 & -0.0273 & 1.0955 & 0.0413 & -0.0346 & -4.9812 \\
			4  & 0.2821 & 0.2444 & 0.0950 & 0.0979 & 0.1871 & 0.1465 & -0.0029 & 0.0405 & 0.0377 & 0.9731 & 0.1235 & 8.1884 & 1.0851 & 6.6358 & 0.5919 & 0.2046 & 0.0220 & 0.0284 & -0.0086 & 1.1700 & 0.0447 & -0.0299 & -5.4350 \\
			5  & 0.3137 & 0.1959 & 0.0671 & 0.0713 & 0.2466 & 0.1246 & -0.0043 & 0.1220 & 0.1178 & 0.7734 & 0.3845 & 7.6692 & 1.0627 & 6.6238 & 0.5954 & 0.2153 & 0.0469 & 0.0278 & 0.0158 & 1.1527 & 0.0543 & -0.0245 & -5.4040 \\
			\bottomrule
	\end{tabular}}
\end{table}


\begin{table}[htbp!]
	\centering
	%\caption*{\normalfont Panel J: Stocks Sorted by RVOL}
	\large
	\resizebox{1.04\textwidth}{!}{
		\begin{tabular}{lccccccccccccccccccccccc}
			\multicolumn{24}{l}{Panel D: Stocks Sorted by RVOL}\\
			&      &      &      &      &      &  & & & & & & & & & &
			&       &     &      &      &      &      & 
			\\
			\toprule
			Quintile & RVJP  & RVJN  & RVLJP & RVLJN & RVSJP & RVSJN & SRVLJ & SRVSJ & SRVJ  & RVOL  & RSK   & RKT   & BETA  & log(Size) & BEME  & MOM   & REV   & IVOL  & CSK   & CKT   & MAX   & MIN   & ILLIQ \\
			%\midrule
				1  & 0.2255 & 0.2140 & 0.1013 & 0.0943 & 0.1241 & 0.1197 & 0.0070 & 0.0044 & 0.0114 & 0.2290 & 0.0485 & 6.8794 & 0.8390 & 8.3393 & 0.5407 & 0.1686 & 0.0040 & 0.0137 & -0.0213 & 1.4325 & 0.0177 & -0.0154 & -7.4961 \\
			2  & 0.2375 & 0.2282 & 0.1136 & 0.1071 & 0.1240 & 0.1211 & 0.0065 & 0.0028 & 0.0094 & 0.3596 & 0.0403 & 7.2835 & 1.0471 & 7.4505 & 0.5506 & 0.1853 & 0.0044 & 0.0187 & -0.0213 & 1.3708 & 0.0257 & -0.0227 & -6.4401 \\
			3  & 0.2567 & 0.2493 & 0.1387 & 0.1327 & 0.1180 & 0.1166 & 0.0060 & 0.0014 & 0.0074 & 0.5331 & 0.0341 & 7.9018 & 1.2246 & 6.6274 & 0.5581 & 0.2598 & 0.0051 & 0.0253 & -0.0249 & 1.2373 & 0.0346 & -0.0304 & -5.5465 \\
			4  & 0.2864 & 0.2823 & 0.1717 & 0.1675 & 0.1147 & 0.1148 & 0.0042 & -0.0001 & 0.0041 & 0.8136 & 0.0216 & 8.7067 & 1.2761 & 5.7649 & 0.5997 & 0.3071 & 0.0058 & 0.0338 & -0.0308 & 1.0491 & 0.0464 & -0.0405 & -4.4310 \\
			5  & 0.3429 & 0.3449 & 0.2336 & 0.2347 & 0.1093 & 0.1102 & -0.0011 & -0.0010 & -0.0021 & 2.8115 & -0.0003 & 10.5156 & 1.0101 & 4.4547 & 0.7443 & 0.0910 & 0.0160 & 0.0548 & -0.0331 & 0.6286 & 0.0817 & -0.0645 & -2.4951 \\
			\bottomrule
	\end{tabular}}
\end{table}

\begin{table}[htbp!]
	\centering
	%\caption*{\normalfont Panel K: Stocks Sorted by RSK}
	\large
	\resizebox{1.04\textwidth}{!}{
		\begin{tabular}{lccccccccccccccccccccccc}
			\multicolumn{24}{l}{Panel E: Stocks Sorted by RSK}\\
			&      &      &      &      &      &  & & & & & & & & & &
			&       &     &      &      &      &      & 
			\\
			\toprule
			Quintile & RVJP  & RVJN  & RVLJP & RVLJN & RVSJP & RVSJN & SRVLJ & SRVSJ & SRVJ  & RVOL  & RSK   & RKT   & BETA  & log(Size) & BEME  & MOM   & REV   & IVOL  & CSK   & CKT   & MAX   & MIN   & ILLIQ \\
			%\midrule
			1  & 0.2077 & 0.3914 & 0.1200 & 0.2822 & 0.0877 & 0.1091 & -0.1622 & -0.0215 & -0.1837 & 0.9182 & -0.9829 & 10.3573 & 1.0293 & 6.1615 & 0.6212 & 0.2118 & -0.0274 & 0.0312 & -0.0611 & 1.0555 & 0.0324 & -0.0472 & -4.7275 \\
			2  & 0.2203 & 0.2758 & 0.0967 & 0.1366 & 0.1235 & 0.1393 & -0.0398 & -0.0157 & -0.0556 & 0.9332 & -0.2596 & 6.8740 & 1.1303 & 6.7033 & 0.5719 & 0.2052 & -0.0077 & 0.0291 & -0.0412 & 1.1981 & 0.0353 & -0.0374 & -5.5757 \\
			3  & 0.2430 & 0.2394 & 0.1057 & 0.1035 & 0.1373 & 0.1359 & 0.0023 & 0.0014 & 0.0037 & 1.1222 & 0.0189 & 6.5111 & 1.1288 & 6.8070 & 0.5748 & 0.1932 & 0.0073 & 0.0283 & -0.0238 & 1.2155 & 0.0391 & -0.0325 & -5.7121 \\
			4  & 0.2786 & 0.2142 & 0.1418 & 0.0960 & 0.1368 & 0.1182 & 0.0459 & 0.0186 & 0.0644 & 0.8769 & 0.3040 & 6.9460 & 1.1139 & 6.7671 & 0.5790 & 0.2056 & 0.0231 & 0.0280 & -0.0098 & 1.2017 & 0.0444 & -0.0290 & -5.6422 \\
			5  & 0.3996 & 0.1978 & 0.2947 & 0.1179 & 0.1049 & 0.0800 & 0.1769 & 0.0249 & 0.2018 & 0.8942 & 1.0656 & 10.5964 & 0.9944 & 6.2012 & 0.6374 & 0.1963 & 0.0400 & 0.0297 & 0.0045 & 1.0482 & 0.0549 & -0.0272 & -4.7552 \\
			\bottomrule
	\end{tabular}}
\end{table}

\begin{table}[htbp!]
	\centering
	%\caption*{\normalfont Panel L: Stocks Sorted by RKT}
	\large
	\resizebox{1.04\textwidth}{!}{
		\begin{threeparttable}	
			\begin{tabular}{lccccccccccccccccccccccc}
				\multicolumn{24}{l}{Panel F: Stocks Sorted by RKT}\\
				&      &      &      &      &      &  & & & & & & & & & &
				&       &     &      &      &      &      & 
				\\
				\toprule
				Quintile & RVJP  & RVJN  & RVLJP & RVLJN & RVSJP & RVSJN & SRVLJ & SRVSJ & SRVJ  & RVOL  & RSK   & RKT   & BETA  & log(Size) & BEME  & MOM   & REV   & IVOL  & CSK   & CKT   & MAX   & MIN   & ILLIQ \\
				%\midrule
				1  & 0.1804 & 0.1785 & 0.0257 & 0.0248 & 0.1548 & 0.1536 & 0.0008 & 0.0011 & 0.0019 & 0.6884 & 0.0110 & 4.4470 & 1.1920 & 7.6130 & 0.5303 & 0.1864 & 0.0054 & 0.0248 & -0.0222 & 1.3529 & 0.0339 & -0.0290 & -6.7592 \\
				2  & 0.2242 & 0.2206 & 0.0757 & 0.0738 & 0.1484 & 0.1468 & 0.0020 & 0.0017 & 0.0036 & 0.7285 & 0.0167 & 5.7679 & 1.1586 & 6.9411 & 0.5522 & 0.2203 & 0.0077 & 0.0276 & -0.0252 & 1.2505 & 0.0387 & -0.0324 & -5.9541 \\
				3  & 0.2630 & 0.2582 & 0.1340 & 0.1310 & 0.1291 & 0.1272 & 0.0030 & 0.0019 & 0.0048 & 0.8130 & 0.0215 & 7.0711 & 1.1103 & 6.5028 & 0.5817 & 0.2175 & 0.0083 & 0.0296 & -0.0265 & 1.1627 & 0.0417 & -0.0347 & -5.3464 \\
				4  & 0.3070 & 0.3004 & 0.2067 & 0.2017 & 0.1004 & 0.0987 & 0.0049 & 0.0017 & 0.0066 & 0.9433 & 0.0292 & 8.9744 & 1.0438 & 6.0841 & 0.6186 & 0.2085 & 0.0076 & 0.0313 & -0.0281 & 1.0644 & 0.0440 & -0.0370 & -4.6922 \\
				5  & 0.3745 & 0.3612 & 0.3171 & 0.3051 & 0.0574 & 0.0561 & 0.0119 & 0.0013 & 0.0133 & 1.5722 & 0.0660 & 15.0322 & 0.8918 & 5.4974 & 0.7022 & 0.1793 & 0.0064 & 0.0332 & -0.0295 & 0.8881 & 0.0478 & -0.0403 & -3.6585 \\
				\bottomrule
			\end{tabular}
			\begin{tablenotes}
				\large
				\item {}
				\item {\large $^*\>$ Notes: See notes to Table 2. Entries in this table are time series averages of equal-weighted realized measures and firm characteristics of stocks sorted by various realized measures. The sample includes all NYSE, NASDAQ and AMEX listed stocks for the period January 1993 to December 2016. At the end of each Tuesday, all of the stocks in the sample are sorted into quintile portfolios, based on ascending values of various realized measures. The equal-weighted realized measures and firm characteristics of each quintile portfolio are calculated over the same week. Additionally, $\>\gamma^{1}=4\sqrt{\frac{1}{t}\widehat{IV}_t^{(i)}}\Delta_n^{0.49} \>$ and $\>\gamma^{2}=5\sqrt{\frac{1}{t}\widehat{IV}_t^{(i)}}\Delta_n^{0.49}\>$ are jump truncation levels. See Sections 2 and 4 for further details.}
			\end{tablenotes}
	\end{threeparttable}}
\end{table}
\newpage
\centering 
Table 4: Univariate Portfolio Sorts Based on Positive, Negative, and Signed Total Jump Variation$^*\>$

\small
% Table generated by Excel2LaTeX from sheet 'r_rvjp'
\begin{table}[htbp!]
	\centering
	\resizebox{1.04\textwidth}{!}{
		\begin{threeparttable}
			%	\begin{tabular}{llllllllllllll}
			\begin{tabular}{lrrrrrrrrrrrrr}	
				%				Panel A: RVJP &      &      &      &      &      & 
				%				&       &     &      &      &      &      & 
				%				\\
				\multicolumn{14}{l}{Panel A: Stocks Sorted by RVJP}\\
				&      &      &      &      &      & 
				&       &     &      &      &      &      & 
				\\
				&    & \multicolumn{4}{c}{Equal-Weighted Returns and Alphas }     &       &       &  & \multicolumn{4}{c}{Value-Weighted Returns and Alphas}         \\
				\midrule
				Quintile & 1(Low) & 2     & 3     & 4     & 5(High) & High-Low &       & 1(Low) & 2     & 3     & 4     & 5(High) & High-Low \\
			\multicolumn{1}{l}{Mean Return} & 33.65 & 30.50 & 33.04 & 28.08 & 20.64 & -13.01*** &    & 23.52 & 19.48 & 17.93 & 20.91 & 18.27 & -5.25 \\
			& (3.54) & (3.28) & (3.49) & (2.87) & (2.19) & (-2.75) &    & (3.54) & (3.27) & (2.93) & (3.35) & (2.83) & (-1.35) \\
			\multicolumn{1}{l}{Alpha} & 10.59 & 7.47 & 11.24 & 7.52 & 2.88 & -7.71 &    & 2.88 & -0.63 & -2.30 & 0.67 & -2.75 & -5.63* \\
			& (4.16) & (3.64) & (4.34) & (2.33) & (0.72) & (-1.64) &    & (2.31) & (-0.44) & (-1.22) & (0.32) & (-1.19) & (-1.87) \\
				\midrule
				&       &       &       &       &       &       &       &       &       &       &       &       &  \\
				\multicolumn{14}{l}{Panel B: Stocks Sorted by RVJN}\\
				&      &      &      &      &      & 
				&       &     &      &      &      &      & 
				\\
				&    & \multicolumn{4}{c}{Equal-Weighted Returns and Alphas }     &       &       &  & \multicolumn{4}{c}{Value-Weighted Returns and Alphas}         \\
				\midrule
				Quintile & 1(Low) & 2     & 3     & 4     & 5(High) & High-Low &       & 1(Low) & 2     & 3     & 4     & 5(High) & High-Low \\
				\multicolumn{1}{l}{Mean Return} & 13.17 & 23.01 & 26.77 & 33.79 & 49.23 & 36.06*** &    & 16.23 & 25.44 & 26.41 & 26.62 & 31.36 & 15.13*** \\
			& (1.52) & (2.59) & (2.84) & (3.36) & (4.62) & (6.47) &    & (2.55) & (4.11) & (4.08) & (3.93) & (4.29) & (3.75) \\
			\multicolumn{1}{l}{Alpha} & -9.36 & -0.03 & 4.18 & 13.24 & 31.71 & 41.07*** &    & -3.55 & 4.94 & 5.27 & 5.49 & 10.05 & 13.60*** \\
			& (-4.46) & (-0.02) & (1.86) & (3.93) & (6.34) & (7.51) &    & (-3.08) & (3.02) & (2.64) & (2.37) & (4.13) & (4.52) \\
				\midrule
				&       &       &       &       &       &       &       &       &       &       &       &       &  \\
				\multicolumn{14}{l}{Panel C: Stocks Sorted by SRVJ}\\
				&      &      &      &      &      & 
				&       &     &      &      &      &      & 
				\\
				&    & \multicolumn{4}{c}{Equal-Weighted Returns and Alphas }     &       &       &  & \multicolumn{4}{c}{Value-Weighted Returns and Alphas}         \\
				\midrule
				Quintile & 1(Low) & 2     & 3     & 4     & 5(High) & High-Low &       & 1(Low) & 2     & 3     & 4     & 5(High) & High-Low \\
					\multicolumn{1}{l}{Mean Return} & 51.85 & 39.02 & 26.15 & 17.86 & 11.02 & -40.82*** &    & 34.67 & 27.43 & 19.93 & 13.64 & 9.65 & -25.02*** \\
				& (5.14) & (3.85) & (2.70) & (1.98) & (1.33) & (-9.85) &    & (4.85) & (4.12) & (3.10) & (2.16) & (1.59) & (-5.78) \\
				\multicolumn{1}{l}{Alpha} & 30.54 & 17.81 & 4.56 & -3.58 & -9.64 & -40.18*** &    & 13.44 & 6.94 & -0.52 & -6.53 & -10.25 & -23.69*** \\
				& (8.40) & (5.78) & (1.74) & (-1.56) & (-4.05) & (-10.10) &    & (5.01) & (3.95) & (-0.40) & (-4.48) & (-4.47) & (-5.56) \\
				\bottomrule
			\end{tabular}
			\begin{tablenotes}
				
				\item {}
				\item { $^*\>$ Notes: Entries in this table are average returns and risk-adjusted alphas for single-sorted portfolios based on RVJP, RVJN and SRVJ, which are described in Table 2. The sample includes all NYSE, NASDAQ and AMEX listed stocks for the period January 1993 to December 2016. At the end of each Tuesday, all the stocks in the sample are sorted into quintile portfolios based on ascending values of the various jump variation measures listed in the titel of each panel. Each portfolio is held for one week. The row labeled ``Mean Return'' reports the time series average values of one-week ahead equal-weighted and value-weighted returns for quintile portfolios. The row labeled ``Alpha'' reports Fama-French-Carhart four-factor alphas, based on the model (\ref{eq12}), for each of the quintile portfolios, as well as for the difference between portfolio 5 and portfolio 1. Newey-West $t$-statistics are given in parentheses; and
					*, **, and *** denote means and alphas that are significant at the 10\%, 5\%, and 1\% levels, respectively. }
			\end{tablenotes}
	\end{threeparttable}}
\end{table}

\newpage
\noindent Table 5: Univariate Portfolio Sorts Based on Positive, Negative, and Signed Large Jump Variation$^*\>$ 




\small
% Table generated by Excel2LaTeX from sheet 'r_rvljp'
\begin{table}[htbp!]
	\centering
	\resizebox{1.04\textwidth}{!}{
		\begin{threeparttable}
			%\begin{tabular}{llllllllllllll}
			\begin{tabular}{lrrrrrrrrrrrrr}
				\multicolumn{14}{l}{Panel A: Stocks Sorted by RVLJP}\\
				&      &      &      &      &      & 
				&       &     &      &      &      &      & 
				\\
				&    & \multicolumn{4}{c}{Equal-Weighted Returns and Alphas }     &       &       &  & \multicolumn{4}{c}{Value-Weighted Returns and Alphas}         \\
				\midrule
				Quintile & 1(Low) & 2     & 3     & 4     & 5(High) & High-Low &       & 1(Low) & 2     & 3     & 4     & 5(High) & High-Low \\
				\multicolumn{5}{l}{Part I: Jump Truncation Level=$\gamma^{1}$} &       &       &       &       &       &       &       &         \\
					\multicolumn{1}{l}{Mean Return} & 31.40 & 30.00 & 29.17 & 31.04 & 23.78 & -7.61** &    & 21.78 & 20.61 & 20.72 & 18.56 & 17.75 & -4.03 \\
				& (3.31) & (3.19) & (3.10) & (3.28) & (2.48) & (-2.05) &    & (3.41) & (3.21) & (3.33) & (2.95) & (2.66) & (-1.09) \\
				\multicolumn{1}{l}{Alpha} & 10.07 & 7.64 & 6.65 & 9.04 & 6.08 & -3.98 &    & 1.97 & 0.68 & -0.42 & -2.32 & -2.83 & -4.80* \\
				& (4.20) & (3.44) & (2.81) & (3.35) & (1.53) & (-1.08) &    & (1.95) & (0.52) & (-0.26) & (-1.27) & (-1.21) & (-1.71) \\
				&       &       &       &       &       &       &       &       &       &       &       &       &  \\
				\multicolumn{5}{l}{Part II: Jump Truncation Level=$\gamma^{2}$} &       &       &       &       &       &       &       &         \\
				\multicolumn{1}{l}{Mean Return} & 29.42 & 43.86 & 30.00 & 29.12 & 25.77 & -3.65 &    & 20.42 & 27.59 & 20.12 & 22.50 & 20.39 & -0.03 \\
				& (3.15) & (2.16) & (3.04) & (3.07) & (2.74) & (-1.11) &    & (3.25) & (1.77) & (2.98) & (3.63) & (3.14) & (-0.01) \\
				\multicolumn{1}{l}{Alpha} & 7.74 & 32.93 & 7.28 & 7.02 & 7.70 & -0.04 &    & 0.45 & 13.80 & -0.27 & 1.60 & -0.52 & -0.97 \\
				& (3.70) & (3.23) & (2.92) & (2.62) & (2.02) & (-0.01) &    & (0.72) & (1.35) & (-0.13) & (0.83) & (-0.22) & (-0.37) \\
				\midrule
				&       &       &       &       &       &       &       &       &       &       &       &       &  \\
				\multicolumn{14}{l}{Panel B: Stocks Sorted by RVLJN}\\
				&      &      &      &      &      & 
				&       &     &      &      &      &      & 
				\\
				&    & \multicolumn{4}{c}{Equal-Weighted Returns and Alphas }     &       &       &  & \multicolumn{4}{c}{Value-Weighted Returns and Alphas}         \\
				\midrule
				Quintile & 1(Low) & 2     & 3     & 4     & 5(High) & High-Low &       & 1(Low) & 2     & 3     & 4     & 5(High) & High-Low \\
				\multicolumn{5}{l}{Part I: Jump Truncation Level=$\gamma^{1}$} &       &       &       &       &       &       &       &         \\
			\multicolumn{1}{l}{Mean Return} & 23.80 & 23.68 & 27.64 & 28.85 & 41.56 & 17.76*** &    & 19.65 & 19.58 & 25.63 & 22.38 & 25.24 & 5.59 \\
			& (2.65) & (2.59) & (2.98) & (2.95) & (4.10) & (4.47) &    & (3.10) & (3.10) & (4.00) & (3.27) & (3.44) & (1.41) \\
			\multicolumn{1}{l}{Alpha} & 2.31 & 1.11 & 5.00 & 7.10 & 23.74 & 21.43*** &    & -0.20 & -1.32 & 4.52 & 0.89 & 4.34 & 4.54 \\
			& (1.11) & (0.55) & (2.32) & (2.47) & (5.39) & (5.53) &    & (-0.18) & (-0.92) & (2.47) & (0.40) & (1.82) & (1.61) \\
				&       &       &       &       &       &       &       &       &       &       &       &       &  \\
				\multicolumn{5}{l}{Part II: Jump Truncation Level=$\gamma^{2}$} &       &       &       &       &       &       &       &         \\
			\multicolumn{1}{l}{Mean Return} & 24.76 & 6.46 & 27.89 & 29.12 & 38.62 & 13.86*** &    & 19.69 & 13.21 & 21.46 & 23.20 & 22.34 & 2.66 \\
			& (2.71) & (0.33) & (2.75) & (3.03) & (3.88) & (4.01) &    & (3.13) & (0.93) & (3.05) & (3.61) & (3.15) & (0.75) \\
			\multicolumn{1}{l}{Alpha} & 3.02 & 6.75 & 6.58 & 7.06 & 20.62 & 17.60*** &    & -0.30 & 6.52 & 1.26 & 2.28 & 1.58 & 1.88 \\
			& (1.63) & (1.08) & (2.50) & (2.67) & (4.81) & (5.06) &    & (-0.45) & (0.91) & (0.57) & (1.06) & (0.66) & (0.70) \\
				\midrule
				&       &       &       &       &       &       &       &       &       &       &       &       &  \\
				\multicolumn{14}{l}{Panel C: Stocks Sorted by SRVLJ}\\
				&      &      &      &      &      & 
				&       &     &      &      &      &      & 
				\\
				&    & \multicolumn{4}{c}{Equal-Weighted Returns and Alphas }     &       &       &  & \multicolumn{4}{c}{Value-Weighted Returns and Alphas}         \\
				\midrule
				Quintile & 1(Low) & 2     & 3     & 4     & 5(High) & High-Low &       & 1(Low) & 2     & 3     & 4     & 5(High) & High-Low \\
				\multicolumn{5}{l}{Part I: Jump Truncation Level=$\gamma^{1}$} &       &       &       &       &       &       &       &         \\
				\multicolumn{1}{l}{Mean Return} & 44.35 & 32.94 & 31.08 & 22.72 & 16.04 & -28.31*** &    & 26.27 & 22.99 & 22.36 & 17.77 & 16.27 & -10.01*** \\
				& (4.52) & (3.36) & (3.13) & (2.44) & (1.88) & (-9.00) &    & (3.92) & (3.58) & (3.29) & (2.79) & (2.71) & (-3.09) \\
				\multicolumn{1}{l}{Alpha} & 23.47 & 11.38 & 8.91 & 0.96 & -4.90 & -28.36*** &    & 5.00 & 2.42 & 1.83 & -2.64 & -4.26 & -9.25*** \\
				& (7.13) & (4.20) & (3.04) & (0.40) & (-2.17) & (-9.39) &    & (2.24) & (1.64) & (1.01) & (-1.82) & (-2.24) & (-2.87) \\
				&       &       &       &       &       &       &       &       &       &       &       &       &  \\
				\multicolumn{5}{l}{Part II: Jump Truncation Level=$\gamma^{2}$} &       &       &       &       &       &       &       &         \\
					\multicolumn{1}{l}{Mean Return} & 40.55 & 28.37 & 33.05 & 24.16 & 19.03 & -21.52*** &    & 22.59 & 20.48 & 16.45 & 18.86 & 20.14 & -2.45 \\
				& (4.19) & (2.91) & (1.48) & (2.55) & (2.19) & (-8.22) &    & (3.40) & (3.18) & (1.14) & (3.02) & (3.27) & (-0.80) \\
				\multicolumn{1}{l}{Alpha} & 19.59 & 8.18 & 24.26 & 2.23 & -1.97 & -21.55*** &    & 1.82 & 0.76 & 6.85 & -2.41 & -0.26 & -2.08 \\
				& (6.15) & (3.29) & (2.16) & (0.82) & (-0.84) & (-8.33) &    & (0.86) & (0.68) & (1.04) & (-1.23) & (-0.13) & (-0.69) \\
				\bottomrule
			\end{tabular}
			\begin{tablenotes}
				
				\item{}
				\item { $^*\>$ Notes: See notes to Table 4. Entries are average returns and risk-adjusted alphas for single-sorted portfolios based on RVLJP, RVLJN and SRVLJ. Jump truncation levels are  $\>\gamma^{1}=4\sqrt{\frac{1}{t}\widehat{IV}_t^{(i)}}\Delta_n^{0.49} \>$ and $\>\gamma^{2}=5\sqrt{\frac{1}{t}\widehat{IV}_t^{(i)}}\Delta_n^{0.49}\>$. }
			\end{tablenotes}
	\end{threeparttable}}
\end{table}
\newpage
\noindent Table 6: Univariate Portfolio Sorts Based on Positive, Negative, and Signed Small Jump Variation$^*\>$ 



\small
% Table generated by Excel2LaTeX from sheet 'r_rvsjp'
% Table generated by Excel2LaTeX from sheet 'r_rvsjp'
\begin{table}[htbp!]
	\centering
	\resizebox{1.04\textwidth}{!}{
		\begin{threeparttable} 
			%\begin{tabular}{llllllllllllll}
			\begin{tabular}{lrrrrrrrrrrrrr}
				\multicolumn{14}{l}{Panel A: Stocks Sorted by RVSJP}\\
				&      &      &      &      &      & 
				&       &     &      &      &      &      & 
				\\
				&    & \multicolumn{4}{c}{Equal-Weighted Returns and Alphas }     &       &       &  & \multicolumn{4}{c}{Value-Weighted Returns and Alphas}         \\
				\midrule
				Quintile & 1(Low) & 2     & 3     & 4     & 5(High) & High-Low &       & 1(Low) & 2     & 3     & 4     & 5(High) & High-Low \\
				\multicolumn{5}{l}{Part I: Jump Truncation Level=$\gamma^{1}$} &       &       &       &       &       &       &       &         \\
				\multicolumn{1}{l}{Mean Return} & 32.01 & 32.51 & 29.03 & 26.59 & 25.72 & -6.29** &    & 27.93 & 26.10 & 18.44 & 16.52 & 15.37 & -12.55** \\
				& (3.44) & (3.43) & (3.09) & (2.85) & (2.65) & (-2.14) &    & (3.65) & (3.84) & (2.91) & (2.75) & (2.45) & (-2.54) \\
				\multicolumn{1}{l}{Alpha} & 13.11 & 9.39 & 6.37 & 4.77 & 6.13 & -6.98*** &    & 7.40 & 5.21 & -1.83 & -2.89 & -4.95 & -12.35*** \\
				& (3.74) & (4.21) & (3.02) & (2.08) & (1.82) & (-2.65) &    & (2.44) & (3.28) & (-1.39) & (-1.55) & (-2.32) & (-2.93) \\
				&       &       &       &       &       &       &       &       &       &       &       &       &  \\
				\multicolumn{5}{l}{Part II: Jump Truncation Level=$\gamma^{2}$} &       &       &       &       &       &       &       &         \\
					\multicolumn{1}{l}{Mean Return} & 34.25 & 31.55 & 28.43 & 27.41 & 24.23 & -10.02*** &    & 29.37 & 18.76 & 17.76 & 14.75 & 18.98 & -10.40** \\
				& (3.72) & (3.37) & (3.04) & (2.90) & (2.48) & (-3.26) &    & (4.09) & (2.95) & (2.94) & (2.39) & (2.93) & (-2.25) \\
				\multicolumn{1}{l}{Alpha} & 14.08 & 8.85 & 5.90 & 5.75 & 5.08 & -9.00*** &    & 8.52 & -2.00 & -2.01 & -5.30 & -1.51 & -10.02** \\
				& (4.93) & (4.12) & (2.73) & (2.32) & (1.42) & (-3.13) &    & (3.87) & (-1.52) & (-1.24) & (-2.67) & (-0.62) & (-2.54) \\
				\midrule
				&       &       &       &       &       &       &       &       &       &       &       &       &  \\
				\multicolumn{14}{l}{Panel B: Stocks Sorted by RVSJN}\\
				&      &      &      &      &      & 
				&       &     &      &      &      &      & 
				\\
				&    & \multicolumn{4}{c}{Equal-Weighted Returns and Alphas }     &       &       &  & \multicolumn{4}{c}{Value-Weighted Returns and Alphas}         \\
				\midrule
				Quintile & 1(Low) & 2     & 3     & 4     & 5(High) & High-Low &       & 1(Low) & 2     & 3     & 4     & 5(High) & High-Low \\
				\multicolumn{5}{l}{Part I: Jump Truncation Level=$\gamma^{1}$} &       &       &       &       &       &       &       &         \\
				\multicolumn{1}{l}{Mean Return} & 23.66 & 21.60 & 27.21 & 31.73 & 41.77 & 18.10*** &    & 6.94 & 15.90 & 21.97 & 27.77 & 31.26 & 24.32*** \\
				& (2.64) & (2.38) & (2.97) & (3.35) & (3.93) & (5.07) &    & (1.00) & (2.42) & (3.46) & (4.34) & (4.68) & (5.00) \\
				\multicolumn{1}{l}{Alpha} & 5.26 & -1.04 & 4.35 & 9.56 & 21.62 & 16.36*** &    & -13.39 & -4.28 & 1.80 & 7.41 & 10.52 & 23.91*** \\
				& (1.59) & (-0.51) & (2.22) & (4.02) & (5.49) & (5.46) &    & (-4.54) & (-2.80) & (1.33) & (3.72) & (4.07) & (5.21) \\
				&       &       &       &       &       &       &       &       &       &       &       &       &  \\
				\multicolumn{5}{l}{Part II: Jump Truncation Level=$\gamma^{2}$} &       &       &       &       &       &       &       &         \\
				\multicolumn{1}{l}{Mean Return} & 19.42 & 23.04 & 26.65 & 32.48 & 44.37 & 24.96*** &    & 14.22 & 18.78 & 25.82 & 29.05 & 32.60 & 18.38*** \\
				& (2.23) & (2.60) & (2.93) & (3.35) & (4.09) & (6.07) &    & (2.13) & (3.00) & (4.13) & (4.42) & (4.43) & (3.80) \\
				\multicolumn{1}{l}{Alpha} & -0.37 & 0.70 & 4.18 & 10.38 & 24.84 & 25.22*** &    & -5.47 & -1.28 & 5.81 & 7.80 & 10.85 & 16.31*** \\
				& (-0.14) & (0.37) & (2.18) & (4.02) & (5.85) & (7.21) &    & (-2.72) & (-1.07) & (3.39) & (3.67) & (4.02) & (4.15) \\
				\midrule
				&       &       &       &       &       &       &       &       &       &       &       &       &  \\
				\multicolumn{14}{l}{Panel C: Stocks Sorted by SRVSJ}\\
				&      &      &      &      &      & 
				&       &     &      &      &      &      & 
				\\
				&    & \multicolumn{4}{c}{Equal-Weighted Returns and Alphas }     &       &       &  & \multicolumn{4}{c}{Value-Weighted Returns and Alphas}         \\
				\midrule
				Quintile & 1(Low) & 2     & 3     & 4     & 5(High) & High-Low &       & 1(Low) & 2     & 3     & 4     & 5(High) & High-Low \\
				\multicolumn{5}{l}{Part I: Jump Truncation Level=$\gamma^{1}$} &       &       &       &       &       &       &       &         \\
					\multicolumn{1}{l}{Mean Return} & 46.41 & 40.51 & 25.12 & 19.00 & 13.74 & -32.67*** &    & 34.54 & 24.08 & 18.26 & 17.84 & 10.42 & -24.12*** \\
				& (4.57) & (4.04) & (2.67) & (2.06) & (1.62) & (-8.60) &    & (5.00) & (3.58) & (2.81) & (2.85) & (1.72) & (-6.60) \\
				\multicolumn{1}{l}{Alpha} & 23.64 & 19.68 & 5.23 & -2.09 & -8.14 & -31.78*** &    & 13.77 & 3.25 & -2.37 & -2.39 & -9.27 & -23.04*** \\
				& (7.62) & (6.10) & (1.53) & (-0.85) & (-4.17) & (-9.01) &    & (6.18) & (1.91) & (-1.07) & (-1.52) & (-5.00) & (-6.54) \\
				&       &       &       &       &       &       &       &       &       &       &       &       &  \\
				\multicolumn{5}{l}{Part II: Jump Truncation Level=$\gamma^{2}$} &       &       &       &       &       &       &       &         \\
					\multicolumn{1}{l}{Mean Return} & 47.90 & 41.62 & 27.23 & 17.90 & 11.20 & -36.70*** &    & 36.88 & 25.13 & 18.45 & 14.98 & 9.41 & -27.47*** \\
				& (4.70) & (4.13) & (2.87) & (1.98) & (1.34) & (-9.06) &    & (5.31) & (3.79) & (2.86) & (2.39) & (1.52) & (-6.94) \\
				\multicolumn{1}{l}{Alpha} & 25.37 & 20.76 & 6.99 & -3.03 & -10.51 & -35.88*** &    & 16.07 & 4.52 & -1.87 & -5.39 & -10.34 & -26.41*** \\
				& (7.79) & (6.65) & (2.23) & (-1.25) & (-5.21) & (-9.49) &    & (6.72) & (2.60) & (-1.23) & (-3.30) & (-5.00) & (-6.72) \\
				\bottomrule
			\end{tabular}
			\begin{tablenotes}
				
				\item{}
				\item { $^*\>$ Notes: See notes to Table 5.}
			\end{tablenotes}
	\end{threeparttable}}
\end{table}


\newpage
\small
\noindent Table 7: Univariate Portfolio Sorts Based on Realized Volatility, Skewness, Kurtosis and Continuous Variance$^*\>$ 

\small
% Table generated by Excel2LaTeX from sheet 'rvol'
% Table generated by Excel2LaTeX from sheet 'rvol'
\begin{table}[htbp!]
	\centering
	\resizebox{1.04\textwidth}{!}{
		\begin{threeparttable}
			\begin{tabular}{lrrrrrrrrrrrrr}
				\multicolumn{14}{l}{Panel A: Stocks Sorted by RVOL}\\
				&      &      &      &      &      & 
				&       &     &      &      &      &      & 
				\\
				&    & \multicolumn{4}{c}{Equal-Weighted Returns and Alphas }     &       &       &  & \multicolumn{4}{c}{Value-Weighted Returns and Alphas}         \\
				\midrule
				Quintile & 1(Low) & 2     & 3     & 4     & 5(High) & High-Low &       & 1(Low) & 2     & 3     & 4     & 5(High) & High-Low \\
					\multicolumn{1}{l}{Mean Return} & 23.36 & 28.00 & 28.89 & 31.78 & 33.91 & 10.55 &    & 20.72 & 21.42 & 19.75 & 26.78 & 29.19 & 8.47 \\
				& (4.47) & (3.92) & (2.96) & (2.59) & (2.24) & (0.81) &    & (4.09) & (2.83) & (1.84) & (1.98) & (1.92) & (0.64) \\
				\multicolumn{1}{l}{Alpha} & 4.50 & 5.01 & 5.15 & 8.74 & 16.33 & 11.83 &    & 1.95 & -1.39 & -3.94 & 2.44 & 5.44 & 3.49 \\
				& (2.07) & (2.94) & (2.57) & (2.54) & (2.11) & (1.37) &    & (1.35) & (-0.67) & (-1.01) & (0.43) & (0.67) & (0.40) \\
				\midrule
				&       &       &       &       &       &       &       &       &       &       &       &       &  \\
				\multicolumn{14}{l}{Panel B: Stocks Sorted by RSK}\\
				&      &      &      &      &      & 
				&       &     &      &      &      &      & 
				\\
				&    & \multicolumn{4}{c}{Equal-Weighted Returns and Alphas }     &       &       &  & \multicolumn{4}{c}{Value-Weighted Returns and Alphas}         \\
				\midrule
				Quintile & 1(Low) & 2     & 3     & 4     & 5(High) & High-Low &       & 1(Low) & 2     & 3     & 4     & 5(High) & High-Low \\
					\multicolumn{1}{l}{Mean Return} & 47.56 & 38.06 & 27.86 & 19.44 & 12.98 & -34.58*** &    & 29.45 & 27.52 & 19.27 & 14.68 & 13.21 & -16.24*** \\
				& (4.85) & (3.82) & (2.86) & (2.12) & (1.54) & (-9.94) &    & (4.27) & (4.22) & (2.98) & (2.32) & (2.18) & (-4.29) \\
				\multicolumn{1}{l}{Alpha} & 26.22 & 16.77 & 6.73 & -2.15 & -7.90 & -34.12*** &    & 7.87 & 7.02 & -0.82 & -5.38 & -6.77 & -14.64*** \\
				& (7.93) & (5.66) & (2.41) & (-0.96) & (-3.51) & (-10.08) &    & (3.30) & (4.44) & (-0.60) & (-3.73) & (-3.23) & (-3.85) \\
				\midrule
				&       &       &       &       &       &       &       &       &       &       &       &       &  \\
				\multicolumn{14}{l}{Panel C: Stocks Sorted by RKT}\\
				&      &      &      &      &      & 
				&       &     &      &      &      &      & 
				\\
				&    & \multicolumn{4}{c}{Equal-Weighted Returns and Alphas }     &       &       &  & \multicolumn{4}{c}{Value-Weighted Returns and Alphas}         \\
				\midrule
				Quintile & 1(Low) & 2     & 3     & 4     & 5(High) & High-Low &       & 1(Low) & 2     & 3     & 4     & 5(High) & High-Low \\
					\multicolumn{1}{l}{Mean Return} & 28.95 & 28.59 & 29.91 & 29.42 & 29.07 & 0.12 &    & 19.87 & 21.57 & 21.12 & 22.17 & 19.55 & -0.32 \\
				& (3.07) & (3.00) & (3.13) & (3.06) & (3.24) & (0.04) &    & (3.12) & (3.42) & (3.37) & (3.38) & (2.96) & (-0.10) \\
				\multicolumn{1}{l}{Alpha} & 8.55 & 6.42 & 7.47 & 7.92 & 9.36 & 0.81 &    & 0.21 & 0.65 & -0.10 & 0.94 & -1.92 & -2.13 \\
				& (3.21) & (2.87) & (3.05) & (2.87) & (3.07) & (0.28) &    & (0.20) & (0.46) & (-0.06) & (0.49) & (-0.91) & (-0.81) \\
				\midrule
				&       &       &       &       &       &       &       &       &       &       &       &       &  \\
				\multicolumn{14}{l}{Panel D: Stocks Sorted by RVC}\\
				&      &      &      &      &      & 
				&       &     &      &      &      &      & 
				\\
				&    & \multicolumn{4}{c}{Equal-Weighted Returns and Alphas }     &       &       &  & \multicolumn{4}{c}{Value-Weighted Returns and Alphas}         \\
				\midrule
				Quintile & 1(Low) & 2     & 3     & 4     & 5(High) & High-Low &       & 1(Low) & 2     & 3     & 4     & 5(High) & High-Low \\
				\multicolumn{1}{l}{Mean Return} & 36.18 & 31.82 & 28.23 & 27.47 & 22.21 & -13.97** &    & 24.36 & 24.80 & 22.94 & 23.40 & 20.41 & -3.95 \\
			& (3.54) & (3.22) & (3.00) & (3.05) & (2.42) & (-2.58) &    & (3.53) & (3.65) & (3.59) & (3.87) & (3.20) & (-1.00) \\
			\multicolumn{1}{l}{Alpha} & 19.82 & 10.62 & 5.54 & 4.66 & -0.94 & -20.76*** &    & 4.27 & 3.29 & 2.42 & 3.27 & 0.06 & -4.22 \\
			& (4.04) & (3.36) & (2.48) & (2.47) & (-0.41) & (-3.87) &    & (1.76) & (1.62) & (1.30) & (2.10) & (0.07) & (-1.46) \\
				\bottomrule
			\end{tabular}
			\begin{tablenotes}
				
				\item{}
				\item { $^*\>$ Notes: See notes to Tables 5. }
			\end{tablenotes}
	\end{threeparttable}}
\end{table}

\newpage
\small
\noindent Table 8: Double-Sorted Portfolios: Portfolios Sorted by Various Jump Variation Measures$^*\>$
\begin{table}[htbp!]
	\centering
	\resizebox{1.04\textwidth}{!}{
		\begin{tabular}{lrrrrrrrrrrrrr}
			\multicolumn{14}{l}{Panel A: Stocks Sorted by SRVLJ, Controlling for SRVJ Based on $\gamma^{2}$}\\
			&      &      &      &      &      & 
			&       &     &      &      &      &      & 
			\\
			&    & \multicolumn{5}{c}{Equal-Weighted Returns and Alphas } &    &    & \multicolumn{5}{c}{Value-Weighted Returns and Alphas} \\
			\midrule
			&    &    & \multicolumn{2}{c}{SRVJ Quintile } &    &    &    &    &    & \multicolumn{2}{c}{SRVJ Quintile } &    &  \\
			SRVLJ Quintile & \multicolumn{1}{l}{1(Low)} & 2  & 3  & 4  & \multicolumn{1}{l}{5(High)} & \multicolumn{1}{l}{Average} &    & \multicolumn{1}{l}{1(Low)} & 2  & 3  & 4  & \multicolumn{1}{l}{5(High)} & \multicolumn{1}{l}{Average} \\
			Part I: Mean Return and Alpha &    &    &    &    &    &    &    &    &    &    &    &    &  \\
			1(Low) & 47.99 & 28.73 & 21.12 & 12.90 & 7.82 & 23.71 &    & 41.88 & 18.83 & 19.24 & 13.59 & 5.04 & 16.61 \\
			2  & 50.55 & 37.24 & 22.38 & 26.91 & 12.99 & 33.21 &    & 29.61 & 25.26 & 18.26 & 17.39 & 14.08 & 22.70 \\
			3  & 53.07 & 38.57 & 23.49 & 21.73 & 12.37 & 29.75 &    & 36.31 & 26.28 & 8.10 & 15.50 & 17.40 & 24.19 \\
			4  & 56.29 & 21.42 & 35.09 & 14.59 & 17.86 & 27.51 &    & 36.38 & 7.94 & 18.53 & 13.64 & 16.59 & 20.33 \\
			5(High) & 47.70 & 45.92 & 32.70 & 26.10 & 7.59 & 32.00 &    & 32.64 & 39.20 & 23.00 & 20.20 & 19.84 & 27.00 \\
			High-Low & -0.24 & 17.19 & 11.58 & 13.20 & -0.23 & 8.30 &    & 6.38 & 20.37 & 3.76 & 6.61 & 14.81 & 10.38 \\
			Alpha & -6.48 & 16.64 & 10.80 & 13.88 & 2.60 & 7.49 &    & 3.69 & 19.41 & 2.46 & 6.06 & 13.84 & 9.09 \\
			&    &    &    &    &    &    &    &    &    &    &    &    &  \\
			Part II: t-Statistics &    &    &    &    &    &    &    &    &    &    &    &    &  \\
			1(Low) & 5.15 & 2.90 & 2.26 & 1.48 & 0.92 & 2.68 &    & 3.45 & 2.66 & 2.89 & 2.08 & 0.75 & 2.67 \\
			2  & 4.81 & 3.62 & 1.94 & 0.81 & 1.43 & 3.49 &    & 3.97 & 3.40 & 2.28 & 0.88 & 1.99 & 3.49 \\
			3  & 4.87 & 3.03 & 0.75 & 1.89 & 1.39 & 3.03 &    & 4.65 & 3.06 & 0.37 & 1.92 & 2.65 & 3.73 \\
			4  & 4.64 & 0.63 & 3.07 & 1.53 & 2.04 & 2.81 &    & 4.26 & 0.27 & 2.18 & 2.05 & 2.50 & 3.04 \\
			5(High) & 4.39 & 4.29 & 3.23 & 2.71 & 0.89 & 3.38 &    & 3.77 & 4.89 & 3.16 & 2.78 & 2.99 & 4.01 \\
			High-Low & -0.04 & 4.24 & 3.31 & 3.87 & -0.06 & 4.18 &    & 1.09 & 3.71 & 0.79 & 1.54 & 3.09 & 4.15 \\
			Alpha & -1.19 & 4.11 & 3.06 & 4.09 & 0.67 & 3.80 &    & 0.62 & 3.40 & 0.49 & 1.44 & 2.96 & 3.53 \\
			\bottomrule
		\end{tabular}%
	}
\end{table}%



% Table generated by Excel2LaTeX from sheet 'r_srvj_r_srvsj'
\begin{table}[htbp!]
	\centering
	\resizebox{1.04\textwidth}{!}{
		\begin{tabular}{lrrrrrrrrrrrrr}
			\multicolumn{14}{l}{Panel B: Stocks Sorted by SRVSJ, Controlling for SRVJ Based on $\gamma^{2}$}\\
			&      &      &      &      &      & 
			&       &     &      &      &      &      & 
			\\
			&    & \multicolumn{5}{c}{Equal-Weighted Returns and Alphas } &    &    & \multicolumn{5}{c}{Value-Weighted Returns and Alphas} \\
			\midrule
			&    &    & \multicolumn{2}{c}{SRVJ Quintile } &    &    &    &    &    & \multicolumn{2}{c}{SRVJ Quintile } &    &  \\
			SRVSJ Quintile & \multicolumn{1}{l}{1(Low)} & 2  & 3  & 4  & \multicolumn{1}{l}{5(High)} & \multicolumn{1}{l}{Average} &    & \multicolumn{1}{l}{1(Low)} & 2  & 3  & 4  & \multicolumn{1}{l}{5(High)} & \multicolumn{1}{l}{Average} \\
			Part I: Mean Return and Alpha &    &    &    &    &    &    &    &    &    &    &    &    &  \\
			1(Low) & 56.90 & 44.50 & 34.78 & 26.13 & 19.50 & 36.36 &    & 41.88 & 35.91 & 26.98 & 18.28 & 22.43 & 29.73 \\
			2  & 55.66 & 47.02 & 30.91 & 18.59 & 10.34 & 32.50 &    & 40.76 & 34.11 & 20.84 & 13.23 & 16.52 & 25.09 \\
			3  & 56.24 & 41.20 & 28.46 & 19.74 & 10.02 & 31.13 &    & 30.72 & 23.04 & 20.39 & 13.82 & 14.29 & 20.45 \\
			4  & 53.58 & 37.21 & 16.75 & 12.62 & 10.19 & 26.07 &    & 28.23 & 23.82 & 12.02 & 14.94 & 6.93 & 17.19 \\
			5(High) & 34.30 & 25.12 & 19.72 & 12.12 & 5.79 & 19.41 &    & 17.03 & 16.65 & 18.03 & 12.69 & 3.77 & 13.63 \\
			High-Low & -22.60 & -19.38 & -15.06 & -14.01 & -13.71 & -16.95 &    & -28.00 & -19.26 & -8.95 & -5.59 & -18.66 & -16.09 \\
			Alpha & -19.26 & -18.83 & -14.19 & -14.86 & -16.20 & -16.67 &    & -25.86 & -20.41 & -6.87 & -4.79 & -18.22 & -15.23 \\
			&    &    &    &    &    &    &    &    &    &    &    &    &  \\
			Part II: t-Statistics &    &    &    &    &    &    &    &    &    &    &    &    &  \\
			1(Low) & 5.44 & 4.22 & 3.40 & 2.67 & 2.16 & 3.76 &    & 5.47 & 4.75 & 3.58 & 2.54 & 3.33 & 4.44 \\
			2  & 5.20 & 4.21 & 3.08 & 1.93 & 1.17 & 3.35 &    & 5.19 & 4.65 & 3.00 & 1.95 & 2.33 & 3.86 \\
			3  & 5.15 & 3.99 & 2.77 & 2.14 & 1.17 & 3.26 &    & 4.04 & 3.21 & 2.89 & 2.02 & 2.20 & 3.20 \\
			4  & 5.33 & 3.58 & 1.71 & 1.42 & 1.23 & 2.86 &    & 3.39 & 3.41 & 1.72 & 2.22 & 1.02 & 2.68 \\
			5(High) & 3.46 & 2.60 & 2.12 & 1.39 & 0.70 & 2.19 &    & 2.22 & 2.32 & 2.67 & 1.89 & 0.56 & 2.16 \\
			High-Low & -4.99 & -5.04 & -4.14 & -3.79 & -3.45 & -7.22 &    & -5.37 & -3.66 & -1.79 & -1.19 & -3.80 & -5.77 \\
			Alpha & -4.15 & -4.90 & -3.91 & -4.16 & -4.12 & -7.42 &    & -4.93 & -3.65 & -1.30 & -1.02 & -3.78 & -5.32 \\
			\bottomrule
		\end{tabular}%
	}
\end{table}%
\newpage
\noindent Table 8 (Continued)$^*\>$
\begin{table}[htbp!]
	\centering
	\resizebox{1.04\textwidth}{!}{
		\begin{tabular}{lrrrrrrrrrrrrr}
			\multicolumn{14}{l}{Panel C: Stocks Sorted by SRVSJ, Controlling for SRVLJ Based on $\gamma^{2}$}\\
			&      &      &      &      &      & 
			&       &     &      &      &      &      & 
			\\
			&    & \multicolumn{5}{c}{Equal-Weighted Returns and Alphas } &    &    & \multicolumn{5}{c}{Value-Weighted Returns and Alphas} \\
			\midrule
			&    &    & \multicolumn{2}{c}{SRVLJ Quintile } &    &    &    &    &    & \multicolumn{2}{c}{SRVLJ Quintile } &    &  \\
			SRVSJ Quintile & \multicolumn{1}{l}{1(Low)} & 2  & 3  & 4  & \multicolumn{1}{l}{5(High)} & \multicolumn{1}{l}{Average} &    & \multicolumn{1}{l}{1(Low)} & 2  & 3  & 4  & \multicolumn{1}{l}{5(High)} & \multicolumn{1}{l}{Average} \\
			Part I: Mean Return and Alpha &    &    &    &    &    &    &    &    &    &    &    &    &  \\
			1(Low) & 60.34 & 51.38 & 40.69 & 44.93 & 31.91 & 47.28 &    & 41.88 & 35.33 & 35.90 & 34.10 & 30.94 & 36.43 \\
			2  & 57.30 & 38.41 & 58.52 & 28.26 & 28.35 & 40.27 &    & 23.32 & 25.07 & 37.49 & 18.00 & 25.93 & 25.12 \\
			3  & 35.25 & 26.42 & 4.21 & 28.76 & 14.62 & 25.17 &    & 16.19 & 21.74 & 9.63 & 20.36 & 21.71 & 19.86 \\
			4  & 28.42 & 16.02 & 48.13 & 10.74 & 10.78 & 19.49 &    & 17.78 & 15.39 & 97.02 & 14.92 & 14.12 & 22.20 \\
			5(High) & 18.79 & 9.45 & -1.20 & 8.01 & 8.26 & 10.89 &    & 14.98 & 8.09 & 6.98 & 7.78 & 13.31 & 11.30 \\
			High-Low & -41.55 & -41.93 & -40.72 & -38.27 & -23.65 & -36.71 &    & -26.63 & -27.24 & -27.87 & -27.67 & -17.63 & -25.38 \\
			Alpha & -40.15 & -41.33 & -40.17 & -37.25 & -22.90 & -35.45 &    & -25.59 & -25.58 & -26.01 & -26.35 & -17.73 & -24.07 \\
			&    &    &    &    &    &    &    &    &    &    &    &    &  \\
			Part II: t-Statistics &    &    &    &    &    &    &    &    &    &    &    &    &  \\
			1(Low) & 5.60 & 4.76 & 1.56 & 4.18 & 3.28 & 4.58 &    & 5.38 & 4.76 & 1.72 & 4.12 & 4.37 & 5.13 \\
			2  & 5.40 & 3.61 & 2.17 & 2.64 & 3.02 & 3.97 &    & 3.03 & 3.41 & 1.61 & 2.34 & 3.60 & 3.62 \\
			3  & 3.58 & 2.60 & 0.19 & 2.85 & 1.62 & 2.65 &    & 2.27 & 3.21 & 0.53 & 2.93 & 3.11 & 3.15 \\
			4  & 2.83 & 1.70 & 1.20 & 1.10 & 1.23 & 2.01 &    & 2.49 & 2.30 & 1.11 & 2.03 & 2.07 & 2.35 \\
			5(High) & 2.12 & 1.06 & -0.06 & 0.93 & 1.01 & 1.31 &    & 2.12 & 1.22 & 0.46 & 1.08 & 2.02 & 1.82 \\
			High-Low & -8.29 & -8.49 & -2.04 & -6.08 & -5.33 & -8.59 &    & -5.15 & -5.54 & -1.50 & -4.14 & -3.64 & -6.61 \\
			Alpha & -8.24 & -8.75 & -2.03 & -6.22 & -5.50 & -8.80 &    & -4.90 & -5.18 & -1.39 & -3.87 & -3.66 & -6.13 \\
			\bottomrule
		\end{tabular}%
	}
\end{table}%

% Table generated by Excel2LaTeX from sheet 'r_srvsj_r_srvLj'
\begin{table}[htbp!]
	\centering
	\resizebox{1.04\textwidth}{!}{
		\begin{threeparttable}
		\begin{tabular}{lrrrrrrrrrrrrr}
			\multicolumn{14}{l}{Panel D: Stocks Sorted by SRVLJ, Controlling for SRVSJ Based on $\gamma^{2}$}\\
			&      &      &      &      &      & 
			&       &     &      &      &      &      & 
			\\
			&    & \multicolumn{5}{c}{Equal-Weighted Returns and Alphas } &    &    & \multicolumn{5}{c}{Value-Weighted Returns and Alphas} \\
			\midrule
			&    &    & \multicolumn{2}{c}{SRVSJ Quintile } &    &    &    &    &    & \multicolumn{2}{c}{SRVSJ Quintile } &    &  \\
			SRVLJ Quintile & \multicolumn{1}{l}{1(Low)} & 2  & 3  & 4  & \multicolumn{1}{l}{5(High)} & \multicolumn{1}{l}{Average} &    & \multicolumn{1}{l}{1(Low)} & 2  & 3  & 4  & \multicolumn{1}{l}{5(High)} & \multicolumn{1}{l}{Average} \\
			Part I: Mean Return and Alpha &    &    &    &    &    &    &    &    &    &    &    &    &  \\
			1(Low) & 59.99 & 60.01 & 41.84 & 27.66 & 16.62 & 41.23 &    & 41.88 & 30.53 & 19.67 & 16.32 & 12.01 & 23.80 \\
			2  & 49.75 & 41.95 & 27.20 & 17.65 & 11.51 & 29.31 &    & 35.69 & 25.91 & 18.85 & 17.68 & 8.92 & 20.89 \\
			3  & 76.41 & 38.05 & 24.45 & 6.48 & 14.41 & 32.06 &    & 45.15 & 16.78 & 11.30 & 10.47 & 21.45 & 21.03 \\
			4  & 43.99 & 30.88 & 23.29 & 13.82 & 11.84 & 24.86 &    & 38.54 & 18.80 & 17.59 & 15.79 & 10.15 & 20.55 \\
			5(High) & 32.38 & 28.64 & 10.59 & 8.97 & 6.78 & 17.47 &    & 31.35 & 25.93 & 18.89 & 13.97 & 10.03 & 20.03 \\
			High-Low & -27.61 & -31.38 & -31.25 & -18.69 & -9.84 & -23.75 &    & -9.14 & -4.60 & -0.78 & -2.35 & -1.98 & -3.77 \\
			Alpha & -26.51 & -31.63 & -31.28 & -19.37 & -9.80 & -23.72 &    & -8.62 & -3.72 & -0.59 & -2.79 & -1.66 & -3.47 \\
			&    &    &    &    &    &    &    &    &    &    &    &    &  \\
			Part II: t-Statistics &    &    &    &    &    &    &    &    &    &    &    &    &  \\
			1(Low) & 5.51 & 5.59 & 4.22 & 2.85 & 1.85 & 4.25 &    & 5.37 & 4.03 & 2.69 & 2.23 & 1.71 & 3.59 \\
			2  & 4.51 & 3.85 & 2.62 & 1.87 & 1.29 & 3.05 &    & 4.71 & 3.54 & 2.63 & 2.71 & 1.32 & 3.28 \\
			3  & 2.31 & 1.95 & 1.61 & 0.41 & 0.80 & 2.49 &    & 1.53 & 1.16 & 1.08 & 0.78 & 1.49 & 2.25 \\
			4  & 4.05 & 2.90 & 2.28 & 1.44 & 1.31 & 2.63 &    & 4.83 & 2.51 & 2.47 & 2.31 & 1.39 & 3.23 \\
			5(High) & 3.36 & 2.99 & 1.21 & 1.03 & 0.83 & 2.02 &    & 4.32 & 3.63 & 2.69 & 2.11 & 1.55 & 3.23 \\
			High-Low & -6.48 & -7.66 & -6.92 & -4.65 & -2.73 & -8.80 &    & -1.83 & -0.98 & -0.16 & -0.46 & -0.44 & -1.49 \\
			Alpha & -6.16 & -7.50 & -6.94 & -4.94 & -2.83 & -8.94 &    & -1.70 & -0.79 & -0.12 & -0.53 & -0.37 & -1.38 \\
			\bottomrule
		\end{tabular}%
			\begin{tablenotes}
	
	\item{}
	\item { $^*\>$ Notes: See notes to Table 5. This table presents average returns (called ``Mean Return'') and risk-adjusted alphas (called ``Alpha'') for portfolios sorted by various jump variation measures. The sample includes NYSE, NASDAQ and AMEX listed stocks for the period January 1993 to December 2016. At the end of each Tuesday, all the stocks in the sample are sorted into quintile portfolios based on ascending values of SRVJ (SRVLJ/SRVSJ). Then, within each quintile portfolio, stocks are further sorted based on the values of SRVLJ/SRVSJ (SRVSJ/SRVLJ), resulting in 25 portfolios. Each portfolio is held for one week. The row labeled ``High-Low'' reports the average values of one-week ahead returns in Part I (corresponding Newey-West $t$-statistics are given in Part II of the panel). The row labeled ``Alpha'' reports Fama-French-Carhart four-factor alphas in Part I (corresponding Newey-West $t$-statistics are again given in Part II of the panel) for the double-sorted High-Low portfolios. Note that entries given in the ``Average'' column of the table, are average returns across the 5 quintiles. Finally, note that SRVLJ and SRVSJ are constructed based on jump truncation level $\gamma^{2}=5\sqrt{\frac{1}{t}\widehat{IV}_t^{(i)}}\Delta_n^{0.49}$.}
\end{tablenotes}
\end{threeparttable}
	}
\end{table}%


\newpage
\small
\noindent Table 9: Double-Sorted Portfolios: Portfolios Sorted by SRVJ and RSK$^*\>$

\small
% Table generated by Excel2LaTeX from sheet '4RSK_SRV'
\begin{table}[htbp!]
	\centering
	%\caption*{\normalfont Panel A: Sorted by SRVJ Controlling for RSK}
	\resizebox{1.04\textwidth}{!}{
		\begin{tabular}{lrrrrrrrrrrrrr}
			\multicolumn{14}{l}{Panel A: Stocks Sorted by SRVJ, Controlling for RSK}\\
			&      &      &      &      &      & 
			&       &     &      &      &      &      & 
			\\
			&    & \multicolumn{4}{c}{Equal-Weighted Returns and Alphas }     &       &       &  & \multicolumn{4}{c}{Value-Weighted Returns and Alphas}         \\
			\midrule
			&    & \multicolumn{4}{c}{RSK Quintile }     &       &       &  & \multicolumn{4}{c}{RSK Quintile}         \\
			SRVJ Quintile & 1(Low) & 2     & 3     & 4     & 5(High) & Average &       & 1(Low) & 2     & 3     & 4     & 5(High) & Average \\
			%			&       &       & \multicolumn{1}{l}{Equal } & \multicolumn{1}{l}{Weighted} &       &       &       &       &       & \multicolumn{1}{l}{Value} & \multicolumn{1}{l}{Weighted} &       &  \\
			%			\midrule
			%			&       &       & \multicolumn{1}{l}{RSK} & \multicolumn{1}{l}{Quintiles} &       &       &       &       &       & \multicolumn{1}{l}{RSK} & \multicolumn{1}{l}{Quintiles} &       &  \\
			%			& \multicolumn{1}{c}{1(Low)} & \multicolumn{1}{c}{2} & \multicolumn{1}{c}{3} & \multicolumn{1}{c}{4} & \multicolumn{1}{c}{5(High)} & \multicolumn{1}{c}{Average} &       & \multicolumn{1}{c}{1(Low)} & \multicolumn{1}{c}{2} & \multicolumn{1}{c}{3} & \multicolumn{1}{c}{4} & \multicolumn{1}{c}{5(High)} & \multicolumn{1}{c}{Average} \\
			%			I: Point Estimates &       &       &       &       &       &       &       &       &       &       &       &       &  \\
			\multicolumn{14}{l}{Part I: Mean Return and Alpha   }                         \\
			1(Low) & 56.92 & 55.49 & 42.81 & 31.69 & 20.91 & 41.57 &    & 41.88 & 39.12 & 30.49 & 25.47 & 17.56 & 30.90 \\
			\multicolumn{1}{l}{2} & 56.77 & 46.17 & 33.30 & 22.89 & 15.27 & 34.88 &    & 41.63 & 35.10 & 21.38 & 17.20 & 13.71 & 25.81 \\
			\multicolumn{1}{l}{3} & 49.97 & 38.92 & 23.47 & 17.47 & 11.61 & 28.29 &    & 37.18 & 27.01 & 22.24 & 14.45 & 10.52 & 22.28 \\
			\multicolumn{1}{l}{4} & 42.67 & 29.57 & 21.68 & 13.20 & 12.18 & 23.86 &    & 28.53 & 20.24 & 15.10 & 11.99 & 11.47 & 17.47 \\
			5(High) & 31.39 & 20.03 & 17.94 & 11.86 & 4.85 & 17.21 &    & 18.32 & 21.67 & 12.23 & 7.31 & 12.60 & 14.42 \\
			High-Low & -25.54 & -35.46 & -24.87 & -19.83 & -16.06 & -24.35 &    & -23.56 & -17.46 & -18.27 & -18.16 & -4.95 & -16.48 \\
			Alpha & -28.79 & -36.20 & -24.40 & -18.40 & -12.75 & -24.11 &    & -24.22 & -18.52 & -18.40 & -16.82 & -4.88 & -16.57 \\
			&       &       &       &       &       &       &       &       &       &       &       &       &  \\
			\multicolumn{5}{l}{Part II: t-Statistics} &       &       &       &       &       &       &       &         \\
			1(Low) & 6.13 & 5.04 & 3.97 & 3.04 & 2.22 & 4.21 &    & 5.72 & 4.87 & 3.97 & 3.35 & 2.51 & 4.58 \\
			\multicolumn{1}{l}{2} & 5.34 & 4.12 & 3.14 & 2.32 & 1.70 & 3.52 &    & 5.11 & 4.57 & 2.96 & 2.44 & 2.02 & 3.89 \\
			\multicolumn{1}{l}{3} & 4.67 & 3.80 & 2.30 & 1.90 & 1.35 & 2.99 &    & 4.90 & 3.78 & 3.07 & 2.14 & 1.63 & 3.51 \\
			\multicolumn{1}{l}{4} & 4.09 & 2.98 & 2.27 & 1.46 & 1.44 & 2.61 &    & 3.67 & 2.84 & 2.17 & 1.76 & 1.72 & 2.73 \\
			5(High) & 3.27 & 2.20 & 2.02 & 1.36 & 0.59 & 2.01 &    & 2.59 & 3.24 & 1.76 & 1.07 & 1.90 & 2.36 \\
			High-Low & -5.36 & -7.09 & -5.12 & -4.32 & -3.40 & -7.70 &    & -4.70 & -3.17 & -3.32 & -3.28 & -0.89 & -5.24 \\
			Alpha & -6.18 & -7.57 & -5.25 & -4.32 & -2.80 & -8.07 &    & -4.81 & -3.50 & -3.36 & -3.04 & -0.90 & -5.45 \\	
			\bottomrule
	\end{tabular}}
\end{table}


\begin{table}[htbp!]
	\centering
	%\caption*{Panel B: Sorted by RSK Controlling for SRVJ}
	\resizebox{1.04\textwidth}{!}{
		\begin{threeparttable}
			\begin{tabular}{lrrrrrrrrrrrrr}		
				\multicolumn{14}{l}{Panel B: Stocks Sorted by RSK, Controlling for SRVJ}\\
				&      &      &      &      &      & 
				&       &     &      &      &      &      & 
				\\
				&    & \multicolumn{4}{c}{Equal-Weighted Returns and Alphas }     &       &       &  & \multicolumn{4}{c}{Value-Weighted Returns and Alphas}         \\
				\midrule
				&    & \multicolumn{4}{c}{SRVJ Quintile }     &       &       &  & \multicolumn{4}{c}{SRVJ Quintile}         \\
				RSK Quintile & 1(Low) & 2     & 3     & 4     & 5(High) & Average &       & 1(Low) & 2     & 3     & 4     & 5(High) & Average \\
				\multicolumn{14}{l}{Part I: Mean Return and Alpha   }                         \\
					1(Low) & 51.14 & 29.34 & 21.71 & 17.58 & 12.40 & 26.43 &    & 41.88 & 18.33 & 20.91 & 12.07 & 4.09 & 18.48 \\
				2  & 49.50 & 39.74 & 23.05 & 14.15 & 11.46 & 27.58 &    & 37.29 & 33.29 & 14.95 & 10.36 & 10.00 & 21.18 \\
				3  & 49.96 & 37.41 & 27.26 & 18.14 & 12.93 & 29.14 &    & 28.38 & 23.82 & 22.52 & 14.80 & 15.23 & 20.95 \\
				4  & 53.85 & 43.31 & 28.11 & 18.79 & 12.52 & 31.32 &    & 36.37 & 28.73 & 19.57 & 17.15 & 10.67 & 22.50 \\
				5(High) & 54.80 & 45.32 & 30.65 & 20.66 & 5.78 & 31.44 &    & 36.75 & 32.79 & 24.63 & 18.92 & 14.26 & 25.47 \\
				High-Low & 3.66 & 15.98 & 8.94 & 3.08 & -6.63 & 5.01 &    & -0.25 & 14.46 & 3.72 & 6.85 & 10.17 & 6.99 \\
				Alpha & 0.54 & 16.64 & 8.57 & 2.34 & -4.54 & 4.71 &    & -0.71 & 15.66 & 4.39 & 6.30 & 9.15 & 6.96 \\
				&       &       &       &       &       &       &       &       &       &       &       &       &  \\
				\multicolumn{5}{l}{Part II: t-Statistics} &       &       &       &       &       &       &       &         \\
					1(Low) & 5.56 & 3.07 & 2.36 & 1.99 & 1.42 & 3.02 &    & 4.98 & 2.61 & 3.15 & 1.74 & 0.60 & 2.97 \\
				2  & 4.76 & 3.75 & 2.33 & 1.57 & 1.30 & 2.94 &    & 4.89 & 4.50 & 2.06 & 1.53 & 1.52 & 3.29 \\
				3  & 4.76 & 3.60 & 2.69 & 1.94 & 1.50 & 3.07 &    & 3.59 & 3.33 & 3.29 & 2.24 & 2.33 & 3.30 \\
				4  & 4.99 & 3.98 & 2.77 & 1.97 & 1.49 & 3.27 &    & 4.70 & 3.89 & 2.69 & 2.39 & 1.63 & 3.45 \\
				5(High) & 5.00 & 4.19 & 2.99 & 2.17 & 0.68 & 3.27 &    & 4.50 & 4.20 & 3.29 & 2.83 & 2.09 & 3.87 \\
				High-Low & 0.85 & 3.98 & 2.31 & 0.84 & -1.61 & 2.35 &    & -0.05 & 2.85 & 0.74 & 1.48 & 2.19 & 2.87 \\
				Alpha & 0.14 & 4.40 & 2.24 & 0.66 & -1.09 & 2.42 &    & -0.14 & 3.05 & 0.89 & 1.38 & 2.01 & 2.92 \\
				\bottomrule
			\end{tabular}
			\begin{tablenotes}
				
				\item{}
				\item { $^*\>$ Notes: See notes to Table 5. This table presents average returns (called ``Mean Return'') and risk-adjusted alphas (called ``Alpha'') for portfolios sorted by SRVJ controlling for RSK, and vice versa. The sample includes NYSE, NASDAQ and AMEX listed stocks for the period January 1993 to December 2016. At the end of each Tuesday, all the stocks in the sample are sorted into quintile portfolios based on ascending values of RSK (SRVJ), and then within each quintile portfolio, stocks are further sorted using values of SRVJ (RSK), resulting in 25 portfolios. Each portfolio is held for one week. The row labeled ``High-Low'' reports the average values of one-week ahead returns in Part I (corresponding Newey-West $t$-statistics are given in Part II of the panel). The row labeled ``Alpha'' reports Fama-French-Carhart four-factor alphas in Part I (corresponding Newey-West $t$-statistics are again given in Part II of the panel) for each of the quintile portfolios, as well as for the average across 5 RSK (SRVJ) portfolios.}
			\end{tablenotes}
	\end{threeparttable}}
\end{table}
\newpage
\noindent Table 10: Double-Sorted Portfolios: Portfolios Sorted by SRVLJ/SRVSJ, Controlling for RSK$^*\>$ 


% Table generated by Excel2LaTeX from sheet '4RSK_L'
\begin{table}[htbp!]
	\centering
	%\caption*{Add caption}
	\resizebox{1.04\textwidth}{!}{
		\begin{threeparttable}
			\begin{tabular}{lrrrrrrrrrrrrr}
				\multicolumn{14}{l}{Panel A: Stocks Sorted by SRVLJ, Controlling for RSK Based on $\gamma^{2}$}\\
				&      &      &      &      &      & 
				&       &     &      &      &      &      & 
				\\
				&    & \multicolumn{4}{c}{Equal-Weighted Returns and Alphas }     &       &       &  & \multicolumn{4}{c}{Value-Weighted Returns and Alphas}         \\
				\midrule
				&    & \multicolumn{4}{c}{RSK Quintile }     &       &       &  & \multicolumn{4}{c}{RSK Quintile}         \\
				SRVLJ Quintile & 1(Low) & 2     & 3     & 4     & 5(High) & Average &       & 1(Low) & 2     & 3     & 4     & 5(High) & Average \\
				
				\multicolumn{14}{l}{Part I: Mean Return and Alpha   }                         \\
				1(Low) & 47.31 & 31.04 & 20.83 & 13.46 & 9.23 & 24.38 &    & 41.88 & 20.74 & 17.63 & 10.28 & 10.95 & 17.04 \\
				2  & 48.24 & 35.53 & 18.15 & 41.82 & 10.40 & 30.72 &    & 27.52 & 25.93 & 13.06 & 22.40 & 13.21 & 21.88 \\
				3  & 46.79 & 36.39 & -2.22 & 17.53 & 16.01 & 27.86 &    & 27.77 & 24.25 & 23.68 & 13.64 & 18.71 & 21.29 \\
				4  & 47.56 & 9.61 & 27.36 & 19.04 & 20.92 & 27.46 &    & 27.71 & 8.29 & 20.27 & 17.45 & 16.06 & 20.34 \\
				5(High) & 43.55 & 43.48 & 34.14 & 26.41 & 7.84 & 31.13 &    & 35.80 & 32.07 & 20.61 & 23.70 & 18.89 & 26.28 \\
				High-Low & -3.57 & 12.44 & 13.30 & 12.95 & -1.39 & 6.75 &    & 10.42 & 11.33 & 2.98 & 13.42 & 7.94 & 9.22 \\
				Alpha & -8.76 & 11.52 & 13.22 & 13.09 & 1.59 & 6.13 &    & 8.26 & 11.20 & 1.91 & 12.69 & 7.30 & 8.27 \\
				&       &       &       &       &       &       &       &       &       &       &       &       &  \\
				\multicolumn{5}{l}{Part II: t-Statistics} &       &       &       &       &       &       &       &         \\
				1(Low) & 5.10 & 3.13 & 2.17 & 1.51 & 1.08 & 2.73 &    & 3.34 & 2.85 & 2.51 & 1.59 & 1.63 & 2.68 \\
			2  & 4.67 & 3.47 & 1.41 & 1.61 & 1.16 & 3.21 &    & 3.71 & 3.71 & 1.47 & 1.12 & 1.98 & 3.42 \\
			3  & 4.48 & 2.95 & -0.07 & 1.54 & 1.79 & 2.91 &    & 3.81 & 2.87 & 0.85 & 1.56 & 2.89 & 3.30 \\
			4  & 4.32 & 0.37 & 2.27 & 1.96 & 2.38 & 2.89 &    & 3.68 & 0.43 & 2.12 & 2.52 & 2.45 & 3.17 \\
			5(High) & 4.22 & 4.03 & 3.42 & 2.75 & 0.93 & 3.34 &    & 4.08 & 3.80 & 2.81 & 3.23 & 2.79 & 3.90 \\
			High-Low & -0.66 & 2.93 & 4.28 & 4.11 & -0.37 & 3.64 &    & 1.71 & 1.89 & 0.66 & 3.03 & 1.62 & 3.76 \\
			Alpha & -1.62 & 2.63 & 4.14 & 4.13 & 0.42 & 3.18 &    & 1.35 & 1.76 & 0.41 & 2.90 & 1.51 & 3.22 \\
			\hline
			&       &       &       &       &       &       &       &       &       &       &       &       &  \\
			&       &       &       &       &       &       &       &       &       &       &       &       &  \\
				\multicolumn{14}{l}{Panel B: Stocks Sorted by SRVSJ, Controlling for RSK Based on $\gamma^{2}$}\\
			&      &      &      &      &      & 
			&       &     &      &      &      &      & 
			\\
			&    & \multicolumn{4}{c}{Equal-Weighted Returns and Alphas }     &       &       &  & \multicolumn{4}{c}{Value-Weighted Returns and Alphas}         \\
			\midrule
			&    & \multicolumn{4}{c}{RSK Quintile }     &       &       &  & \multicolumn{4}{c}{RSK Quintile}         \\
			SRVSJ Quintile & 1(Low) & 2     & 3     & 4     & 5(High) & Average &       & 1(Low) & 2     & 3     & 4     & 5(High) & Average \\
			
			\multicolumn{14}{l}{Part I: Mean Return and Alpha   }                         \\
			1(Low) & 56.15 & 51.19 & 42.56 & 31.93 & 24.16 & 41.20 &    & 41.88 & 40.69 & 31.16 & 31.49 & 21.39 & 33.55 \\
			2  & 54.34 & 50.15 & 35.79 & 24.76 & 12.89 & 35.59 &    & 32.45 & 33.20 & 21.93 & 19.73 & 23.95 & 26.25 \\
			3  & 55.24 & 38.53 & 26.29 & 18.62 & 11.09 & 29.95 &    & 27.91 & 22.64 & 18.61 & 12.36 & 16.37 & 19.58 \\
			4  & 40.20 & 31.45 & 18.77 & 14.26 & 12.14 & 23.35 &    & 18.13 & 22.01 & 15.30 & 13.17 & 10.40 & 15.79 \\
			5(High) & 30.21 & 18.90 & 15.72 & 7.49 & 4.76 & 15.42 &    & 18.59 & 19.31 & 12.59 & 8.23 & 6.56 & 13.06 \\
			High-Low & -25.94 & -32.29 & -26.84 & -24.43 & -19.40 & -25.78 &    & -24.44 & -21.37 & -18.56 & -23.26 & -14.84 & -20.49 \\
			Alpha & -23.86 & -31.76 & -26.49 & -23.94 & -20.71 & -25.35 &    & -22.27 & -22.89 & -18.30 & -21.83 & -14.44 & -19.95 \\
			&       &       &       &       &       &       &       &       &       &       &       &       &  \\
			\multicolumn{5}{l}{Part II: t-Statistics} &       &       &       &       &       &       &       &         \\
			1(Low) & 5.43 & 4.75 & 3.98 & 3.16 & 2.62 & 4.15 &    & 5.11 & 5.25 & 4.16 & 4.39 & 3.15 & 4.92 \\
			2  & 5.17 & 4.48 & 3.42 & 2.47 & 1.43 & 3.61 &    & 4.30 & 4.53 & 2.96 & 2.68 & 3.39 & 3.95 \\
			3  & 5.25 & 3.76 & 2.60 & 1.96 & 1.25 & 3.14 &    & 3.75 & 3.12 & 2.66 & 1.72 & 2.46 & 3.03 \\
			4  & 4.11 & 3.13 & 1.96 & 1.58 & 1.45 & 2.59 &    & 2.50 & 3.20 & 2.27 & 1.98 & 1.57 & 2.55 \\
			5(High) & 3.18 & 2.06 & 1.71 & 0.87 & 0.58 & 1.79 &    & 2.56 & 2.86 & 1.79 & 1.23 & 0.97 & 2.09 \\
			High-Low & -5.78 & -7.15 & -6.04 & -5.58 & -4.72 & -8.41 &    & -4.21 & -4.04 & -3.67 & -4.56 & -2.93 & -6.31 \\
			Alpha & -5.36 & -7.20 & -6.12 & -5.75 & -5.27 & -8.89 &    & -3.83 & -4.29 & -3.51 & -4.09 & -2.83 & -6.01 \\
		
				\hline
			\end{tabular}
			\begin{tablenotes}
				
				\item{}
				\item { $^*\>$ Notes: See notes to Table 8. Portfolios are sorted by SRVLJ/SRVSJ, controlling for RSK, and using truncation level $\gamma^{2}$, as discussed in the footnote to Table 2.}
			\end{tablenotes}
	\end{threeparttable}}
	
\end{table}

\newpage
\noindent Table 11: Double-Sorted Portfolios: Portfolios Sorted by RSK, Controlling for SRVLJ or SRVSJ$^*\>$

\begin{table}[htbp!]
	\centering
	%\caption*{Add caption}
	\resizebox{1.04\textwidth}{!}{
		\begin{threeparttable}
			\begin{tabular}{lrrrrrrrrrrrrr}
				\multicolumn{14}{l}{Panel A: Stocks Sorted by RSK, Controlling for SRVLJ Based on $\gamma^{2}$}\\
				&      &      &      &      &      & 
				&       &     &      &      &      &      & 
				\\
				&    & \multicolumn{4}{c}{Equal-Weighted Returns and Alphas }     &       &       &  & \multicolumn{4}{c}{Value-Weighted Returns and Alphas}         \\
				\midrule
				&    & \multicolumn{4}{c}{SRVLJ Quintile }     &       &       &  & \multicolumn{4}{c}{SRVLJ Quintile}         \\
				RSK Quintile & 1(Low) & 2     & 3     & 4     & 5(High) & Average &       & 1(Low) & 2     & 3     & 4     & 5(High) & Average \\
				
				\multicolumn{14}{l}{Part I: Mean Return and Alpha   }                         \\
				1(Low) & 51.98 & 46.26 & 32.88 & 42.99 & 35.40 & 43.72 &    & 41.88 & 34.45 & 32.97 & 31.19 & 33.46 & 34.64 \\
				2  & 48.38 & 38.05 & 55.49 & 31.86 & 20.58 & 37.03 &    & 30.03 & 23.16 & 48.38 & 17.47 & 16.06 & 24.56 \\
				3  & 43.16 & 26.83 & 8.78 & 20.69 & 16.54 & 25.94 &    & 25.91 & 21.10 & 10.09 & 19.92 & 19.16 & 21.15 \\
				4  & 37.70 & 22.51 & 55.61 & 11.20 & 14.84 & 24.67 &    & 18.23 & 18.69 & 8.48 & 13.82 & 14.50 & 16.23 \\
				5(High) & 21.43 & 8.10 & -0.87 & 14.29 & 7.74 & 12.81 &    & 13.59 & 7.52 & -8.20 & 15.74 & 15.50 & 12.31 \\
				High-Low & -30.55 & -38.16 & -30.25 & -30.06 & -27.66 & -31.07 &    & -23.12 & -26.93 & -37.56 & -16.82 & -17.96 & -22.43 \\
				Alpha & -32.89 & -37.85 & -28.79 & -30.01 & -24.81 & -30.46 &    & -23.34 & -24.74 & -36.67 & -14.80 & -18.66 & -21.39 \\
				&       &       &       &       &       &       &       &       &       &       &       &       &  \\
				\multicolumn{5}{l}{Part II: t-Statistics} &       &       &       &       &       &       &       &         \\
				1(Low) & 5.63 & 4.46 & 1.33 & 4.16 & 3.61 & 4.48 &    & 4.94 & 4.65 & 1.67 & 4.00 & 4.80 & 5.06 \\
				2  & 4.68 & 3.68 & 2.16 & 3.04 & 2.22 & 3.72 &    & 3.99 & 3.28 & 2.04 & 2.34 & 2.31 & 3.63 \\
				3  & 4.14 & 2.66 & 0.40 & 2.04 & 1.84 & 2.69 &    & 3.32 & 3.22 & 0.59 & 2.66 & 2.87 & 3.29 \\
				4  & 3.62 & 2.29 & 1.37 & 1.11 & 1.73 & 2.50 &    & 2.54 & 2.82 & 0.48 & 1.80 & 2.22 & 2.54 \\
				5(High) & 2.27 & 0.89 & -0.04 & 1.63 & 0.92 & 1.49 &    & 1.91 & 1.08 & -0.52 & 2.33 & 2.24 & 1.98 \\
				High-Low & -7.91 & -9.22 & -1.71 & -5.63 & -5.85 & -9.46 &    & -4.56 & -6.02 & -2.29 & -2.75 & -3.47 & -6.54 \\
				Alpha & -8.43 & -9.50 & -1.64 & -5.86 & -5.51 & -9.57 &    & -4.59 & -5.54 & -2.22 & -2.37 & -3.65 & -6.10 \\
				\hline
				&       &       &       &       &       &       &       &       &       &       &       &       &  \\
				&       &       &       &       &       &       &       &       &       &       &       &       &  \\
				\multicolumn{14}{l}{Panel B: Stocks Sorted by RSK, Controlling for SRVSJ Based on $\gamma^{2}$}\\
				&      &      &      &      &      & 
				&       &     &      &      &      &      & 
				\\
				&    & \multicolumn{4}{c}{Equal-Weighted Returns and Alphas }     &       &       &  & \multicolumn{4}{c}{Value-Weighted Returns and Alphas}         \\
				\midrule
				&    & \multicolumn{4}{c}{SRVSJ Quintile }     &       &       &  & \multicolumn{4}{c}{SRVSJ Quintile}         \\
				RSK Quintile & 1(Low) & 2     & 3     & 4     & 5(High) & Average &       & 1(Low) & 2     & 3     & 4     & 5(High) & Average \\
				
				\multicolumn{14}{l}{Part I: Mean Return and Alpha   }                         \\
				1(Low) & 57.49 & 56.03 & 40.07 & 26.79 & 17.75 & 39.62 &    & 41.88 & 24.28 & 18.84 & 16.57 & 12.21 & 23.20 \\
				2  & 51.21 & 42.88 & 31.02 & 15.27 & 12.69 & 30.61 &    & 35.14 & 26.67 & 14.77 & 21.10 & 9.49 & 21.43 \\
				3  & 54.96 & 41.34 & 25.81 & 20.97 & 10.12 & 30.64 &    & 37.20 & 24.66 & 20.50 & 10.79 & 7.58 & 20.15 \\
				4  & 42.52 & 38.93 & 29.50 & 16.21 & 9.58 & 27.35 &    & 37.22 & 23.08 & 17.57 & 15.29 & 11.02 & 20.84 \\
				5(High) & 33.23 & 28.85 & 9.62 & 10.20 & 5.82 & 17.54 &    & 31.94 & 30.24 & 21.43 & 16.36 & 8.89 & 21.77 \\
				High-Low & -24.26 & -27.17 & -30.45 & -16.59 & -11.93 & -22.08 &    & -12.15 & 5.96 & 2.59 & -0.21 & -3.33 & -1.43 \\
				Alpha & -22.78 & -27.83 & -30.17 & -17.16 & -11.85 & -21.96 &    & -10.86 & 7.74 & 2.94 & -0.96 & -3.03 & -0.83 \\
				&       &       &       &       &       &       &       &       &       &       &       &       &  \\
				\multicolumn{5}{l}{Part II: t-Statistics} &       &       &       &       &       &       &       &         \\
					1(Low) & 5.53 & 5.32 & 4.11 & 2.82 & 1.99 & 4.19 &    & 5.79 & 3.21 & 2.52 & 2.21 & 1.73 & 3.49 \\
				2  & 4.84 & 4.00 & 3.06 & 1.62 & 1.44 & 3.20 &    & 4.45 & 3.56 & 2.02 & 3.25 & 1.42 & 3.30 \\
				3  & 4.96 & 3.79 & 2.58 & 2.21 & 1.15 & 3.15 &    & 4.86 & 3.35 & 3.01 & 1.55 & 1.13 & 3.12 \\
				4  & 4.01 & 3.79 & 2.86 & 1.73 & 1.12 & 2.87 &    & 5.03 & 3.26 & 2.44 & 2.17 & 1.62 & 3.22 \\
				5(High) & 3.41 & 3.05 & 1.09 & 1.17 & 0.72 & 2.03 &    & 4.19 & 4.35 & 2.91 & 2.49 & 1.38 & 3.48 \\
				High-Low & -5.79 & -6.79 & -6.74 & -4.40 & -3.12 & -8.56 &    & -2.27 & 1.17 & 0.48 & -0.04 & -0.69 & -0.49 \\
				Alpha & -5.33 & -6.78 & -6.68 & -4.61 & -3.19 & -8.57 &    & -2.06 & 1.50 & 0.53 & -0.17 & -0.61 & -0.29 \\
				
				\hline
			\end{tabular}
			\begin{tablenotes}
				
				\item{}
				\item { $^*\>$ Notes: See notes to Table 8. Portfolios are sorted by RSK, controlling for SRVLJ/SRVSJ, and using truncation level $\gamma^{2}$, as discussed in the footnote to Table 2.}
			\end{tablenotes}
	\end{threeparttable}}
	
\end{table}				
				
				
\newpage
\noindent Table 12: Double-Sorted Portfolios: Portfolios Independently Sorted by Stock- and Industry-Level SRVJ$^*\>$


% Table generated by Excel2LaTeX from sheet '4srvj'
\begin{table}[htbp!]
	\centering
	%\caption{Add caption}
	\resizebox{1.04\textwidth}{!}{
		\begin{threeparttable}
			\begin{tabular}{llllllrrrrrrrrrrr}
				%\begin{tabular}{lrrrrrrrrrrrrrrrr}
				&    & \multicolumn{4}{c}{Equal-Weighted Returns and Alphas }     &       &       &  & & & \multicolumn{4}{c}{Value-Weighted Returns and Alphas}   &     \\
				\midrule
				&    & \multicolumn{4}{c}{Industry-Level Quintile }     &       &   &  &  &  & \multicolumn{4}{c}{Industry-Level Quintile}   &      \\
				
				Stock-Level Quintile & \multicolumn{1}{c}{1(Low)} & \multicolumn{1}{c}{2} & \multicolumn{1}{c}{3} & \multicolumn{1}{c}{4} & \multicolumn{1}{c}{5(High)} & \multicolumn{1}{c}{High-Low} & \multicolumn{1}{c}{Alpha} &       &       & \multicolumn{1}{c}{1(Low)} & \multicolumn{1}{c}{2} & \multicolumn{1}{c}{3} & \multicolumn{1}{c}{4} & \multicolumn{1}{c}{5(High)} & \multicolumn{1}{c}{High-Low} & \multicolumn{1}{c}{Alpha} \\
				\multicolumn{17}{l}{Part I: Mean Return and Alpha   }                         \\
			1(Low) & \multicolumn{1}{r}{39.23} & \multicolumn{1}{r}{47.42} & \multicolumn{1}{r}{53.00} & \multicolumn{1}{r}{61.43} & \multicolumn{1}{r}{72.18} & 32.94 & 34.74 &    &    & 36.50 & 33.40 & 32.78 & 36.58 & 25.03 & 10.19 & 9.28 \\
			2  & \multicolumn{1}{r}{30.55} & \multicolumn{1}{r}{33.28} & \multicolumn{1}{r}{38.69} & \multicolumn{1}{r}{48.44} & \multicolumn{1}{r}{55.77} & 25.22 & 25.86 &    &    & 28.37 & 27.31 & 20.24 & 37.10 & 14.64 & 7.20 & 6.68 \\
			3  & \multicolumn{1}{r}{14.07} & \multicolumn{1}{r}{21.47} & \multicolumn{1}{r}{24.27} & \multicolumn{1}{r}{30.98} & \multicolumn{1}{r}{43.15} & 29.07 & 29.60 &    &    & 18.16 & 24.85 & 17.56 & 20.35 & 8.69 & 10.64 & 11.98 \\
			4  & \multicolumn{1}{r}{6.90} & \multicolumn{1}{r}{12.45} & \multicolumn{1}{r}{15.65} & \multicolumn{1}{r}{19.51} & \multicolumn{1}{r}{34.51} & 27.61 & 28.30 &    &    & 12.97 & 11.96 & 10.38 & 14.58 & 2.67 & 9.54 & 9.99 \\
			5(High) & \multicolumn{1}{r}{-7.35} & \multicolumn{1}{r}{1.99} & \multicolumn{1}{r}{9.90} & \multicolumn{1}{r}{16.40} & \multicolumn{1}{r}{25.94} & 33.29 & 32.92 &    &    & -4.30 & 9.86 & 4.86 & 13.25 & -3.95 & 19.82 & 21.72 \\
			High-Low & \multicolumn{1}{r}{-46.58} & \multicolumn{1}{r}{-45.43} & \multicolumn{1}{r}{-43.09} & \multicolumn{1}{r}{-45.03} & \multicolumn{1}{r}{-46.24} &    &    &    &    & -40.80 & -23.54 & -27.92 & -23.33 & -31.16 &    &  \\
			Alpha & \multicolumn{1}{r}{-44.64} & \multicolumn{1}{r}{-45.23} & \multicolumn{1}{r}{-41.18} & \multicolumn{1}{r}{-45.96} & \multicolumn{1}{r}{-46.46} &    &    &    &    & -41.42 & -22.73 & -26.52 & -23.09 & -28.98 &    &  \\
			\multicolumn{6}{l}{Industry-Level Effect (average of High-Low column; Alpha column)} & 29.63 & 30.29 &    &    &    &    &    &    &    & 11.48 & 11.93 \\
			\multicolumn{6}{l}{Stock-Level Effect (average of High-Low row; Alpha row)} & -45.28 & -44.70 &    &    &    &    &    &    &    & -29.35 & -28.55 \\
				&       &       &       &       &       &       &       &       &       &       &       &       &       &       &       &  \\
				&       &       &       &       &       &       &       &       &       &       &       &       &       &       &       &  \\
				\multicolumn{17}{l}{Part II: t-Statistics}         \\
				1(Low) & \multicolumn{1}{r}{3.82} & \multicolumn{1}{r}{4.37} & \multicolumn{1}{r}{4.98} & \multicolumn{1}{r}{5.87} & \multicolumn{1}{r}{6.83} & 5.20 & 5.35 &    &    & 4.61 & 3.98 & 3.93 & 4.55 & 4.71 & 1.40 & 1.27 \\
			2  & \multicolumn{1}{r}{2.90} & \multicolumn{1}{r}{3.13} & \multicolumn{1}{r}{3.49} & \multicolumn{1}{r}{4.48} & \multicolumn{1}{r}{5.22} & 3.50 & 3.35 &    &    & 3.82 & 3.56 & 2.35 & 4.58 & 3.16 & 1.14 & 1.00 \\
			3  & \multicolumn{1}{r}{1.39} & \multicolumn{1}{r}{2.09} & \multicolumn{1}{r}{2.28} & \multicolumn{1}{r}{3.11} & \multicolumn{1}{r}{4.43} & 4.60 & 4.53 &    &    & 2.34 & 3.23 & 2.25 & 2.76 & 1.87 & 1.54 & 1.73 \\
			4  & \multicolumn{1}{r}{0.70} & \multicolumn{1}{r}{1.28} & \multicolumn{1}{r}{1.60} & \multicolumn{1}{r}{2.12} & \multicolumn{1}{r}{3.93} & 4.40 & 4.47 &    &    & 1.69 & 1.62 & 1.36 & 1.87 & 0.68 & 1.50 & 1.54 \\
			5(High) & \multicolumn{1}{r}{-0.79} & \multicolumn{1}{r}{0.22} & \multicolumn{1}{r}{1.07} & \multicolumn{1}{r}{1.88} & \multicolumn{1}{r}{3.16} & 5.88 & 5.93 &    &    & -0.51 & 1.27 & 0.62 & 1.93 & -1.00 & 2.82 & 3.06 \\
			High-Low & \multicolumn{1}{r}{-9.33} & \multicolumn{1}{r}{-8.12} & \multicolumn{1}{r}{-8.27} & \multicolumn{1}{r}{-7.80} & \multicolumn{1}{r}{-8.01} &    &    &    &    & -7.19 & -3.92 & -5.22 & -4.31 & -5.33 &    &  \\
			Alpha, FFC4 & \multicolumn{1}{r}{-9.00} & \multicolumn{1}{r}{-8.16} & \multicolumn{1}{r}{-7.97} & \multicolumn{1}{r}{-7.93} & \multicolumn{1}{r}{-8.34} &    &    &    &    & -7.37 & -3.73 & -5.06 & -4.31 & -5.47 &    &  \\
			\multicolumn{6}{l}{Industry-Level Effect (average of High-Low column; Alpha column)} & 5.66 & 5.57 &    &    &    &    &    &    &    & 2.23 & 2.20 \\
			\multicolumn{6}{l}{Stock-Level Effect (average of High-Low row; Alpha row)} & -11.39 & -11.50 &    &    &    &    &    &    &    & -8.88 & -8.91 \\
				\bottomrule
			\end{tabular}
			\begin{tablenotes}	
				\item{}
				\item { $^*\>$ Notes: See notes to Table 8. This table presents average returns and risk-adjusted alphas for portfolios sorted by stock-level and industry-level SRVJ. The sample includes all NYSE, NASDAQ and AMEX listed stocks for the period January 1993 to December 2016. A stock's industry signed jump variation (SRVJ) is the capitalization-weighted average of the SRVJ of all stocks within the industry. At the end of each Tuesday, all stocks in the sample are sorted into quintile portfolios based on stock-level and industry-level SRVJ, independently, resulting in 25 portfolios. Each portfolio is held for one week. The row labeled ``Industry-Level Effect'' reports average values of one-week ahead returns (and Fama-French-Carhart four-factor alphas in the High-Low (Alpha) column) in Part I (corresponding Newey-West $t$-statistics are given in Part II). The row labeled ``Stock-Level Effect'' reports the average values of one-week 
					ahead returns (and alphas) in Part I (corresponding Newey-West $t$-statistics are again given in Part II).  }
			\end{tablenotes}
	\end{threeparttable}}
\end{table}
\newpage
\noindent Table 13: Double-Sorted Portfolios: Portfolios Sorted by Stock- and Industry-Level SRVLJ/SRVSJ Independently$^*\>$

\begin{table}[htbp!]
	\centering
	%\caption*{\normalfont Panel B: Portfolios Sorted Based on Truncation Level $\gamma^{(2)}$}
	\resizebox{1.04\textwidth}{!}{
		\begin{threeparttable}
			\begin{tabular}{llllllrrrrrrrrrrr}
				\multicolumn{17}{l}{Panel A: Portfolios Sorted Based on SRVLJ}\\
				&      &      &      &      &      &  & & &
				&       &     &      &      &      &      & 
				\\
				&    & \multicolumn{4}{c}{Equal-Weighted Returns and Alphas }     &       &       &  & & & \multicolumn{4}{c}{Value-Weighted Returns and Alphas}   &     \\
				\midrule
				&    & \multicolumn{4}{c}{Industry-Level Quintile }     &       &   &  &  &  & \multicolumn{4}{c}{Industry-Level Quintile}   &      \\
				
				Stock-Level Quintile & \multicolumn{1}{c}{1(Low)} & \multicolumn{1}{c}{2} & \multicolumn{1}{c}{3} & \multicolumn{1}{c}{4} & \multicolumn{1}{c}{5(High)} & \multicolumn{1}{c}{High-Low} & \multicolumn{1}{c}{Alpha} &       &       & \multicolumn{1}{c}{1(Low)} & \multicolumn{1}{c}{2} & \multicolumn{1}{c}{3} & \multicolumn{1}{c}{4} & \multicolumn{1}{c}{5(High)} & \multicolumn{1}{c}{High-Low} & \multicolumn{1}{c}{Alpha} \\
				\multicolumn{17}{l}{Part I: Mean Return and Alpha   }                         \\
				1(Low) & \multicolumn{1}{r}{34.64} & \multicolumn{1}{r}{35.34} & \multicolumn{1}{r}{35.04} & \multicolumn{1}{r}{51.12} & \multicolumn{1}{r}{51.00} & 16.36 & 17.03 &    &    & 22.26 & 24.61 & 20.46 & 26.39 & 25.19 & 2.92 & 0.73 \\
				2  & \multicolumn{1}{r}{25.37} & \multicolumn{1}{r}{25.36} & \multicolumn{1}{r}{22.38} & \multicolumn{1}{r}{34.21} & \multicolumn{1}{r}{36.00} & 10.64 & 9.43 &    &    & 18.78 & 22.97 & 19.49 & 23.51 & 27.29 & 8.51 & 7.46 \\
				3  & \multicolumn{1}{r}{17.55} & \multicolumn{1}{r}{0.89} & \multicolumn{1}{r}{26.31} & \multicolumn{1}{r}{34.26} & \multicolumn{1}{r}{31.56} & 14.40 & 11.15 &    &    & 18.90 & 15.95 & 11.59 & 4.81 & 17.06 & -6.76 & -10.28 \\
				4  & \multicolumn{1}{r}{13.59} & \multicolumn{1}{r}{20.91} & \multicolumn{1}{r}{21.20} & \multicolumn{1}{r}{27.71} & \multicolumn{1}{r}{34.08} & 19.69 & 19.75 &    &    & 11.26 & 21.71 & 10.01 & 12.25 & 25.88 & 13.89 & 15.35 \\
				5(High) & \multicolumn{1}{r}{12.22} & \multicolumn{1}{r}{11.90} & \multicolumn{1}{r}{14.43} & \multicolumn{1}{r}{25.57} & \multicolumn{1}{r}{25.57} & 13.36 & 12.93 &    &    & 14.16 & 21.60 & 16.13 & 20.63 & 22.85 & 8.69 & 11.30 \\
				High-Low & \multicolumn{1}{r}{-22.42} & \multicolumn{1}{r}{-23.44} & \multicolumn{1}{r}{-20.61} & \multicolumn{1}{r}{-25.55} & \multicolumn{1}{r}{-25.42} &    &    &    &    & -8.10 & -3.00 & -4.33 & -5.75 & -2.33 &    &  \\
				Alpha & \multicolumn{1}{r}{-22.02} & \multicolumn{1}{r}{-23.37} & \multicolumn{1}{r}{-20.45} & \multicolumn{1}{r}{-25.74} & \multicolumn{1}{r}{-26.12} &    &    &    &    & -9.58 & -3.52 & -4.34 & -5.40 & 1.00 &    &  \\
				\multicolumn{6}{l}{Industry-Level Effect (average of High-Low column; Alpha column)} & 14.83 & 14.31 &    &    &    &    &    &    &    & 7.24 & 6.90 \\
				\multicolumn{6}{l}{Stock-Level Effect (average of High-Low row; Alpha row)} & -23.49 & -23.54 &    &    &    &    &    &    &    & -4.70 & -4.37 \\
				&       &       &       &       &       &       &       &       &       &       &       &       &       &       &       &  \\
				&       &       &       &       &       &       &       &       &       &       &       &       &       &       &       &  \\
				\multicolumn{17}{l}{Part II: t-Statistics}         \\
					1(Low) & \multicolumn{1}{r}{3.63} & \multicolumn{1}{r}{3.39} & \multicolumn{1}{r}{3.33} & \multicolumn{1}{r}{4.94} & \multicolumn{1}{r}{5.23} & 3.40 & 3.41 &    &    & 3.25 & 3.06 & 2.48 & 3.25 & 3.16 & 0.49 & 0.12 \\
				2  & \multicolumn{1}{r}{2.55} & \multicolumn{1}{r}{2.44} & \multicolumn{1}{r}{2.14} & \multicolumn{1}{r}{3.46} & \multicolumn{1}{r}{3.73} & 2.17 & 1.90 &    &    & 2.70 & 3.17 & 2.57 & 3.20 & 3.73 & 1.60 & 1.32 \\
				3  & \multicolumn{1}{r}{0.78} & \multicolumn{1}{r}{0.04} & \multicolumn{1}{r}{0.93} & \multicolumn{1}{r}{1.41} & \multicolumn{1}{r}{1.46} & 0.85 & 0.65 &    &    & 1.08 & 0.73 & 0.51 & 0.25 & 1.05 & -0.37 & -0.54 \\
				4  & \multicolumn{1}{r}{1.32} & \multicolumn{1}{r}{1.95} & \multicolumn{1}{r}{1.94} & \multicolumn{1}{r}{2.74} & \multicolumn{1}{r}{3.73} & 3.67 & 3.73 &    &    & 1.49 & 2.62 & 1.20 & 1.45 & 3.69 & 2.43 & 2.65 \\
				5(High) & \multicolumn{1}{r}{1.31} & \multicolumn{1}{r}{1.24} & \multicolumn{1}{r}{1.50} & \multicolumn{1}{r}{2.84} & \multicolumn{1}{r}{2.99} & 2.97 & 2.91 &    &    & 1.83 & 2.83 & 2.12 & 2.83 & 3.56 & 1.50 & 2.09 \\
				High-Low & \multicolumn{1}{r}{-6.14} & \multicolumn{1}{r}{-5.13} & \multicolumn{1}{r}{-4.66} & \multicolumn{1}{r}{-5.85} & \multicolumn{1}{r}{-6.19} &    &    &    &    & -1.57 & -0.59 & -0.90 & -1.19 & -0.45 &    &  \\
				Alpha, FFC4 & \multicolumn{1}{r}{-6.09} & \multicolumn{1}{r}{-5.03} & \multicolumn{1}{r}{-4.58} & \multicolumn{1}{r}{-5.91} & \multicolumn{1}{r}{-6.38} &    &    &    &    & -1.93 & -0.68 & -0.87 & -1.09 & 0.21 &    &  \\
				\multicolumn{6}{l}{Industry-Level Effect (average of High-Low column; Alpha column)} & 3.77 & 3.55 &    &    &    &    &    &    &    & 1.77 & 1.57 \\
				\multicolumn{6}{l}{Stock-Level Effect (average of High-Low row; Alpha row)} & -9.05 & -9.14 &    &    &    &    &    &    &    & -2.01 & -1.91 \\
				\bottomrule
			\end{tabular}
	\end{threeparttable}}
\end{table}

\begin{table}[htbp!]
	\centering
	%\caption*{\normalfont Panel B: Portfolios Sorted Based on Truncation Level $\gamma^{(2)}$}
	\resizebox{1.04\textwidth}{!}{
		\begin{threeparttable}
			\begin{tabular}{llllllrrrrrrrrrrr}
				\multicolumn{17}{l}{Panel B: Portfolios Sorted Based on SRVSJ}\\
				&      &      &      &      &      &  & & &
				&       &     &      &      &      &      & 
				\\
				&    & \multicolumn{4}{c}{Equal-Weighted Returns and Alphas }     &       &       &  & & & \multicolumn{4}{c}{Value-Weighted Returns and Alphas}   &     \\
				\midrule
				&    & \multicolumn{4}{c}{Industry-Level Quintile }     &       &   &  &  &  & \multicolumn{4}{c}{Industry-Level Quintile}   &      \\
				
				Stock-Level Quintile & \multicolumn{1}{c}{1(Low)} & \multicolumn{1}{c}{2} & \multicolumn{1}{c}{3} & \multicolumn{1}{c}{4} & \multicolumn{1}{c}{5(High)} & \multicolumn{1}{c}{High-Low} & \multicolumn{1}{c}{Alpha} &       &       & \multicolumn{1}{c}{1(Low)} & \multicolumn{1}{c}{2} & \multicolumn{1}{c}{3} & \multicolumn{1}{c}{4} & \multicolumn{1}{c}{5(High)} & \multicolumn{1}{c}{High-Low} & \multicolumn{1}{c}{Alpha} \\
				\multicolumn{17}{l}{Part I: Mean Return and Alpha   }                         \\
				1(Low) & \multicolumn{1}{r}{40.39} & \multicolumn{1}{r}{42.50} & \multicolumn{1}{r}{43.55} & \multicolumn{1}{r}{55.02} & \multicolumn{1}{r}{68.45} & 28.07 & 29.88 &    &    & 44.09 & 37.87 & 32.07 & 36.14 & 45.53 & 1.44 & 0.72 \\
			2  & \multicolumn{1}{r}{32.52} & \multicolumn{1}{r}{40.60} & \multicolumn{1}{r}{41.29} & \multicolumn{1}{r}{45.73} & \multicolumn{1}{r}{59.67} & 27.15 & 28.20 &    &    & 24.11 & 24.09 & 23.82 & 27.45 & 42.09 & 17.99 & 18.36 \\
			3  & \multicolumn{1}{r}{15.39} & \multicolumn{1}{r}{22.96} & \multicolumn{1}{r}{24.90} & \multicolumn{1}{r}{30.69} & \multicolumn{1}{r}{45.29} & 29.90 & 30.43 &    &    & 20.22 & 15.93 & 21.91 & 17.63 & 25.37 & 5.16 & 4.88 \\
			4  & \multicolumn{1}{r}{9.07} & \multicolumn{1}{r}{9.08} & \multicolumn{1}{r}{17.26} & \multicolumn{1}{r}{17.20} & \multicolumn{1}{r}{32.70} & 23.63 & 23.94 &    &    & 11.38 & 12.88 & 22.53 & 11.01 & 25.43 & 14.04 & 15.23 \\
			5(High) & \multicolumn{1}{r}{-2.44} & \multicolumn{1}{r}{3.16} & \multicolumn{1}{r}{7.13} & \multicolumn{1}{r}{12.21} & \multicolumn{1}{r}{22.30} & 24.73 & 26.19 &    &    & 8.91 & 7.91 & 11.60 & 9.76 & 11.23 & 2.32 & 5.33 \\
			High-Low & \multicolumn{1}{r}{-42.83} & \multicolumn{1}{r}{-39.34} & \multicolumn{1}{r}{-36.43} & \multicolumn{1}{r}{-42.82} & \multicolumn{1}{r}{-46.16} &    &    &    &    & -35.18 & -29.95 & -20.47 & -26.38 & -34.30 &    &  \\
			Alpha & \multicolumn{1}{r}{-42.27} & \multicolumn{1}{r}{-39.03} & \multicolumn{1}{r}{-34.75} & \multicolumn{1}{r}{-42.27} & \multicolumn{1}{r}{-45.96} &    &    &    &    & -36.51 & -30.26 & -18.68 & -25.89 & -31.90 &    &  \\
			\multicolumn{6}{l}{Industry-Level Effect (average of High-Low column; Alpha column)} & 26.69 & 27.73 &    &    &    &    &    &    &    & 8.19 & 8.90 \\
			\multicolumn{6}{l}{Stock-Level Effect (average of High-Low row; Alpha row)} & -41.51 & -40.86 &    &    &    &    &    &    &    & -29.26 & -28.65 \\
				&       &       &       &       &       &       &       &       &       &       &       &       &       &       &       &  \\
				&       &       &       &       &       &       &       &       &       &       &       &       &       &       &       &  \\
				\multicolumn{17}{l}{Part II: t-Statistics}         \\
					1(Low) & \multicolumn{1}{r}{3.91} & \multicolumn{1}{r}{3.90} & \multicolumn{1}{r}{3.98} & \multicolumn{1}{r}{5.06} & \multicolumn{1}{r}{6.20} & 4.04 & 3.99 &    &    & 5.58 & 4.66 & 3.66 & 4.40 & 5.62 & 0.21 & 0.10 \\
				2  & \multicolumn{1}{r}{3.00} & \multicolumn{1}{r}{3.83} & \multicolumn{1}{r}{3.88} & \multicolumn{1}{r}{4.42} & \multicolumn{1}{r}{5.68} & 3.89 & 3.88 &    &    & 3.19 & 3.04 & 2.90 & 3.50 & 5.39 & 2.84 & 2.77 \\
				3  & \multicolumn{1}{r}{1.55} & \multicolumn{1}{r}{2.26} & \multicolumn{1}{r}{2.46} & \multicolumn{1}{r}{3.11} & \multicolumn{1}{r}{4.68} & 4.91 & 4.94 &    &    & 2.52 & 1.97 & 2.73 & 2.33 & 3.43 & 0.70 & 0.68 \\
				4  & \multicolumn{1}{r}{0.92} & \multicolumn{1}{r}{0.93} & \multicolumn{1}{r}{1.74} & \multicolumn{1}{r}{1.84} & \multicolumn{1}{r}{3.64} & 3.72 & 3.74 &    &    & 1.46 & 1.78 & 2.85 & 1.47 & 3.72 & 2.18 & 2.43 \\
				5(High) & \multicolumn{1}{r}{-0.25} & \multicolumn{1}{r}{0.33} & \multicolumn{1}{r}{0.77} & \multicolumn{1}{r}{1.41} & \multicolumn{1}{r}{2.70} & 3.94 & 4.15 &    &    & 1.06 & 0.96 & 1.49 & 1.36 & 1.74 & 0.34 & 0.76 \\
				High-Low & \multicolumn{1}{r}{-8.38} & \multicolumn{1}{r}{-7.85} & \multicolumn{1}{r}{-7.04} & \multicolumn{1}{r}{-7.99} & \multicolumn{1}{r}{-7.60} &    &    &    &    & -6.21 & -5.45 & -3.70 & -5.15 & -6.15 &    &  \\
				Alpha, FFC4 & \multicolumn{1}{r}{-8.38} & \multicolumn{1}{r}{-7.95} & \multicolumn{1}{r}{-6.85} & \multicolumn{1}{r}{-8.19} & \multicolumn{1}{r}{-8.02} &    &    &    &    & -6.55 & -5.68 & -3.41 & -5.15 & -5.69 &    &  \\
				\multicolumn{6}{l}{Industry-Level Effect (average of High-Low column; Alpha column)} & 5.02 & 5.01 &    &    &    &    &    &    &    & 1.60 & 1.68 \\
				\multicolumn{6}{l}{Stock-Level Effect (average of High-Low row; Alpha row)} & -10.65 & -11.14 &    &    &    &    &    &    &    & -8.78 & -8.75 \\
				\bottomrule
			\end{tabular}
			\begin{tablenotes}	
				\item{}
				\item { $^*\>$ Notes: See notes to Table 12. Jumps are decomposed using  truncation level $\gamma^{2}$, as discussed in the footnote to Table 2.}
			\end{tablenotes}
	\end{threeparttable}}
\end{table}
\newpage
\noindent Table 14: Fama-MacBeth Cross-Sectional Regressions$^*\>$
\begin{table}[htbp!]
	\centering
	%\caption*{\normalfont Panel B: Regressions Based on Truncation Level $\gamma^{(2)}$}
	\resizebox{1.0\textwidth}{!}{
		\begin{threeparttable}
			\begin{tabular}{lcccccccccccccccc}
				\multicolumn{17}{l}{Panel A: Regressions Without Control Variables}\\
				&      &      &      &      &      &  & & &
				&       &     &      &      &      &      & 
				\\
				\toprule
				& I     & II    & III   & IV    & V     & VI    & VII   & VIII  & IX    & X     & XI    & XII   & XIII  & XIV   & XV    & XVI \\
				Intercept & 18.54 & 27.95 & 23.99 & 15.77 & 30.31 & 31.03 & 31.46 & 31.20 & 19.74 & 32.04 & 20.01 & 20.04 & 28.88 & 30.01 & 30.48 & 30.17 \\
			& (1.94) & (3.07) & (2.82) & (1.69) & (3.32) & (3.37) & (3.41) & (3.39) & (2.03) & (3.73) & (2.14) & (2.08) & (3.29) & (3.40) & (3.44) & (3.41) \\
			RVJP & -63.86 &    &    &    &    &    &    &    & -128.25 &    &    &    &    &    &    &  \\
			& (-6.00) &    &    &    &    &    &    &    & (-6.24) &    &    &    &    &    &    &  \\
			RVJN & 107.11 &    &    &    &    &    &    &    & 196.57 &    &    &    &    &    &    &  \\
			& (8.29) &    &    &    &    &    &    &    & (8.98) &    &    &    &    &    &    &  \\
			RVLJP &    & -53.42 &    & -44.85 &    &    &    &    &    & 76.84 &    & -79.63 &    &    &    &  \\
			&    & (-6.46) &    & (-4.46) &    &    &    &    &    & (6.58) &    & (-3.94) &    &    &    &  \\
			RVLJN &    & 71.27 &    & 83.09 &    &    &    &    &    & -30.83 &    & 149.92 &    &    &    &  \\
			&    & (8.12) &    & (7.45) &    &    &    &    &    & (-2.40) &    & (7.18) &    &    &    &  \\
			RVSJP &    &    & -130.77 & -99.16 &    &    &    &    &    &    & -88.56 & -129.39 &    &    &    &  \\
			&    &    & (-8.97) & (-6.24) &    &    &    &    &    &    & (-6.64) & (-6.33) &    &    &    &  \\
			RVSJN &    &    & 165.05 & 161.39 &    &    &    &    &    &    & 129.24 & 195.31 &    &    &    &  \\
			&    &    & (8.22) & (7.19) &    &    &    &    &    &    & (7.94) & (8.26) &    &    &    &  \\
			SRVLJ &    &    &    &    & -50.07 &    & -53.94 &    &    &    &    &    & 72.19 &    & -82.69 &  \\
			&    &    &    &    & (-7.98) &    & (-8.37) &    &    &    &    &    & (6.60) &    & (-4.48) &  \\
			SRVSJ &    &    &    &    &    & -141.69 & -144.75 &    &    &    &    &    &    & -103.72 & -149.56 &  \\
			&    &    &    &    &    & (-9.25) & (-9.32) &    &    &    &    &    &    & (-8.25) & (-7.66) &  \\
			SRVJ &    &    &    &    &    &    &    & -81.15 &    &    &    &    &    &    &    & -150.59 \\
			&    &    &    &    &    &    &    & (-10.15) &    &    &    &    &    &    &    & (-7.80) \\
			RVOL &    &    &    &    &    &    &    &    & -8.94 & -7.46 & -6.74 & -9.08 & -5.90 & -6.31 & -6.40 & -6.38 \\
			&    &    &    &    &    &    &    &    & (-1.60) & (-1.32) & (-1.21) & (-1.62) & (-1.05) & (-1.12) & (-1.14) & (-1.13) \\
			RSK &    &    &    &    &    &    &    &    & 16.12 & -22.16 & -9.87 & 9.12 & -24.75 & -10.15 & 4.08 & 14.02 \\
			&    &    &    &    &    &    &    &    & (5.59) & (-9.55) & (-9.49) & (3.07) & (-10.41) & (-9.72) & (1.39) & (4.91) \\
			RKT &    &    &    &    &    &    &    &    & -0.68 & -0.68 & 0.46 & -0.68 & 0.12 & 0.09 & 0.08 & 0.09 \\
			&    &    &    &    &    &    &    &    & (-2.25) & (-2.27) & (1.45) & (-2.24) & (0.42) & (0.30) & (0.28) & (0.32) \\
			Adjusted $R^2$ & 0.0063 & 0.0033 & 0.0035 & 0.0082 & 0.0005 & 0.0019 & 0.0024 & 0.0016 & 0.0204 & 0.0175 & 0.0185 & 0.0214 & 0.0160 & 0.0168 & 0.0172 & 0.0168 \\
				\bottomrule
			\end{tabular}
	\end{threeparttable}}
\end{table}
\newpage
\noindent Table 14 (Continued)$^*\>$
\begin{table}[htbp!]
	\centering
	%\caption*{\normalfont Panel B: Regressions Based on Truncation Level $\gamma^{(2)}$}
	\resizebox{1.0\textwidth}{!}{	
		\begin{threeparttable}
			\begin{tabular}{lcccccccccccccccc}
				\multicolumn{17}{l}{Panel B: Regressions with Control Variables}\\
				&      &      &      &      &      &  & & &
				&       &     &      &      &      &      & 
				\\	
				\toprule
				& I     & II    & III   & IV    & V     & VI    & VII   & VIII  & IX    & X     & XI    & XII   & XIII  & XIV   & XV    & XVI \\
			Intercept & 100.76 & 100.26 & 92.97 & 98.02 & 97.60 & 97.72 & 98.57 & 98.26 & 89.27 & 92.87 & 93.62 & 89.60 & 94.59 & 94.82 & 95.19 & 94.94 \\
			& (4.24) & (5.45) & (5.25) & (4.15) & (5.67) & (5.65) & (5.69) & (5.67) & (3.30) & (4.26) & (3.93) & (3.27) & (4.37) & (4.33) & (4.30) & (4.29) \\
			RVJP & -30.35 &    &    &    &    &    &    &    & -33.59 &    &    &    &    &    &    &  \\
			& (-3.04) &    &    &    &    &    &    &    & (-1.87) &    &    &    &    &    &    &  \\
			RVJN & 28.77 &    &    &    &    &    &    &    & 50.58 &    &    &    &    &    &    &  \\
			& (3.26) &    &    &    &    &    &    &    & (3.48) &    &    &    &    &    &    &  \\
			RVLJP &    & -27.56 &    & -27.07 &    &    &    &    &    & 11.55 &    & -28.36 &    &    &    &  \\
			&    & (-4.46) &    & (-2.81) &    &    &    &    &    & (1.20) &    & (-1.53) &    &    &    &  \\
			RVLJN &    & 16.42 &    & 23.05 &    &    &    &    &    & -0.32 &    & 48.59 &    &    &    &  \\
			&    & (2.67) &    & (2.61) &    &    &    &    &    & (-0.03) &    & (2.89) &    &    &    &  \\
			RVSJP &    &    & -26.94 & -34.15 &    &    &    &    &    &    & -24.07 & -38.20 &    &    &    &  \\
			&    &    & (-2.78) & (-2.51) &    &    &    &    &    &    & (-2.20) & (-2.04) &    &    &    &  \\
			RVSJN &    &    & 44.62 & 45.83 &    &    &    &    &    &    & 26.48 & 52.34 &    &    &    &  \\
			&    &    & (4.25) & (3.81) &    &    &    &    &    &    & (2.67) & (3.42) &    &    &    &  \\
			SRVLJ &    &    &    &    & -22.63 &    & -25.76 &    &    &    &    &    & 9.90 &    & -31.02 &  \\
			&    &    &    &    & (-5.18) &    & (-5.67) &    &    &    &    &    & (1.10) &    & (-1.93) &  \\
			SRVSJ &    &    &    &    &    & -33.16 & -38.71 &    &    &    &    &    &    & -23.75 & -41.74 &  \\
			&    &    &    &    &    & (-3.92) & (-4.45) &    &    &    &    &    &    & (-2.74) & (-2.83) &  \\
			SRVJ &    &    &    &    &    &    &    & -28.64 &    &    &    &    &    &    &    & -39.38 \\
			&    &    &    &    &    &    &    & (-6.26) &    &    &    &    &    &    &    & (-2.69) \\
			RVOL &    &    &    &    &    &    &    &    & 4.79 & 5.07 & 4.59 & 4.68 & 4.94 & 4.87 & 4.86 & 4.84 \\
			&    &    &    &    &    &    &    &    & (0.79) & (0.84) & (0.76) & (0.77) & (0.82) & (0.80) & (0.80) & (0.79) \\
			RSK &    &    &    &    &    &    &    &    & 3.02 & -4.90 & -3.58 & 2.67 & -5.53 & -3.67 & 1.42 & 2.46 \\
			&    &    &    &    &    &    &    &    & (1.23) & (-3.01) & (-4.69) & (0.98) & (-3.45) & (-4.79) & (0.50) & (0.95) \\
			RKT &    &    &    &    &    &    &    &    & -0.53 & -0.61 & -0.38 & -0.56 & -0.49 & -0.46 & -0.44 & -0.44 \\
			&    &    &    &    &    &    &    &    & (-2.00) & (-2.27) & (-1.08) & (-1.87) & (-1.78) & (-1.63) & (-1.49) & (-1.46) \\
			Beta & -8.28 & -8.11 & -8.09 & -8.27 & -8.07 & -8.26 & -8.29 & -8.16 & -7.71 & -7.75 & -8.18 & -7.72 & -8.10 & -8.18 & -8.22 & -8.14 \\
			& (-1.46) & (-1.40) & (-1.38) & (-1.46) & (-1.37) & (-1.41) & (-1.41) & (-1.39) & (-1.36) & (-1.35) & (-1.42) & (-1.37) & (-1.40) & (-1.42) & (-1.43) & (-1.41) \\
			log(Size) & -14.93 & -14.77 & -14.73 & -14.94 & -14.76 & -14.76 & -14.67 & -14.67 & -14.36 & -14.25 & -14.29 & -14.48 & -14.13 & -14.11 & -14.04 & -14.05 \\
			& (-5.24) & (-5.25) & (-5.21) & (-5.32) & (-5.15) & (-5.14) & (-5.13) & (-5.13) & (-5.19) & (-5.15) & (-5.16) & (-5.26) & (-5.08) & (-5.07) & (-5.06) & (-5.07) \\
			BE/ME & -0.76 & -0.75 & -0.65 & -0.76 & -0.67 & -0.61 & -0.59 & -0.62 & -0.72 & -0.57 & -0.57 & -0.69 & -0.56 & -0.54 & -0.56 & -0.58 \\
			& (-0.37) & (-0.36) & (-0.32) & (-0.37) & (-0.33) & (-0.30) & (-0.29) & (-0.30) & (-0.34) & (-0.27) & (-0.27) & (-0.33) & (-0.27) & (-0.26) & (-0.27) & (-0.28) \\
			MOM & 0.00 & 0.00 & 0.00 & 0.00 & 0.00 & 0.00 & 0.00 & 0.00 & 0.00 & 0.00 & 0.00 & 0.00 & 0.00 & 0.00 & 0.00 & 0.00 \\
			& (0.96) & (0.97) & (0.94) & (0.98) & (0.93) & (0.93) & (0.93) & (0.93) & (1.26) & (1.19) & (1.18) & (1.22) & (1.19) & (1.21) & (1.20) & (1.21) \\
			REV & -0.01 & -0.01 & -0.01 & -0.01 & -0.01 & -0.01 & -0.01 & -0.01 & -0.01 & -0.01 & -0.01 & -0.01 & -0.01 & -0.01 & -0.01 & -0.01 \\
			& (-5.63) & (-5.83) & (-5.63) & (-5.53) & (-5.84) & (-5.66) & (-5.55) & (-5.64) & (-5.74) & (-5.97) & (-5.82) & (-5.70) & (-5.99) & (-5.83) & (-5.77) & (-5.80) \\
			IVOL & -301.46 & -293.89 & -297.02 & -304.39 & -291.92 & -292.47 & -298.32 & -298.14 & -425.97 & -416.77 & -422.83 & -424.54 & -419.05 & -421.52 & -424.24 & -425.20 \\
			& (-2.18) & (-2.15) & (-2.17) & (-2.21) & (-2.13) & (-2.12) & (-2.16) & (-2.17) & (-4.69) & (-4.57) & (-4.63) & (-4.67) & (-4.59) & (-4.62) & (-4.65) & (-4.67) \\
			CSK & -7.52 & -8.67 & -8.13 & -7.29 & -8.69 & -8.23 & -7.62 & -7.87 & -7.00 & -7.84 & -7.44 & -6.89 & -7.91 & -7.45 & -7.35 & -7.52 \\
			& (-1.75) & (-2.01) & (-1.89) & (-1.70) & (-2.01) & (-1.91) & (-1.77) & (-1.82) & (-1.66) & (-1.86) & (-1.77) & (-1.64) & (-1.88) & (-1.77) & (-1.74) & (-1.78) \\
			CKT & 2.34 & 2.29 & 2.24 & 2.36 & 2.32 & 2.32 & 2.43 & 2.38 & 1.74 & 1.72 & 1.69 & 1.74 & 1.71 & 1.77 & 1.83 & 1.81 \\
			& (1.19) & (1.15) & (1.13) & (1.20) & (1.15) & (1.16) & (1.21) & (1.18) & (0.91) & (0.89) & (0.87) & (0.91) & (0.88) & (0.91) & (0.94) & (0.93) \\
			MAX & -0.03 & -0.03 & -0.03 & -0.03 & -0.03 & -0.03 & -0.03 & -0.03 & -0.03 & -0.03 & -0.03 & -0.03 & -0.03 & -0.03 & -0.03 & -0.03 \\
			& (-5.33) & (-5.66) & (-5.57) & (-5.31) & (-5.64) & (-5.71) & (-5.55) & (-5.54) & (-7.37) & (-7.55) & (-7.45) & (-7.35) & (-7.59) & (-7.55) & (-7.57) & (-7.55) \\
			MIN & -0.02 & -0.02 & -0.02 & -0.02 & -0.02 & -0.02 & -0.02 & -0.02 & -0.01 & -0.02 & -0.01 & -0.01 & -0.01 & -0.01 & -0.01 & -0.01 \\
			& (-2.79) & (-3.11) & (-3.14) & (-2.79) & (-3.11) & (-3.12) & (-2.83) & (-2.85) & (-2.81) & (-2.94) & (-2.80) & (-2.81) & (-2.88) & (-2.80) & (-2.73) & (-2.75) \\
			ILLIQ & -7.84 & -7.68 & -8.08 & -7.86 & -7.99 & -8.03 & -7.87 & -7.88 & -8.94 & -8.87 & -8.68 & -9.10 & -8.54 & -8.51 & -8.41 & -8.43 \\
			& (-5.24) & (-5.12) & (-5.22) & (-5.26) & (-5.15) & (-5.16) & (-5.08) & (-5.10) & (-4.79) & (-5.12) & (-4.96) & (-4.87) & (-4.97) & (-4.95) & (-4.86) & (-4.88) \\
			Adjusted $R^2$ & 0.0602 & 0.0597 & 0.0597 & 0.0609 & 0.0590 & 0.0592 & 0.0594 & 0.0592 & 0.0647 & 0.0641 & 0.0642 & 0.0652 & 0.0636 & 0.0637 & 0.0639 & 0.0638 \\
				\bottomrule
			\end{tabular}
			\begin{tablenotes}	
				\item{}
				\item {\large $^*\>$ Notes: See notes to Tables 1 and 5. This table reports results for cross-sectional Fama-MacBeth regressions, based on the regression model depicted as equation (\ref{eq13}) in Section 4.5. In these regression models, future weekly returns are regressed on various realized measures and control variates.  The two panels utilize jump truncation level $\gamma^{2}$, as discussed in the footnote to Table 2. The regressions that are reported on are of the form:
					$r_{i,t+1} =\gamma_{0,t} +\sum_{j=1}^{K_1}\gamma_{j,t}X_{i,j,t}+\sum_{s=1}^{K_2}\phi_{s,t}Z_{i,s,t}+\epsilon_{i,t+1},$  $t=1,...,T,$
					where $r_{i,t+1}$ denotes the stock return for firm $i$ in week $t+1$, $K_1$ is the number of potential variation measures, and $X_{i,j,t}$ denotes a relevant realized measure at the end of week $t$. In addition, there are $K_2$ variables measuring firm characteristics, which are denoted by $Z_{i,j,t}$ (see Section 3 for details). In the table, time series averages of the coefficient estimates ($\frac{1}{T}\sum_{t=1}^T \widehat{\gamma}_{j,t}$ and $\frac{1}{T}\sum_{t=1}^T \widehat{\phi}_{j,t}$) are reported, along with Newey-West $t$-statistics (in parentheses). For complete details, see Section 4.}
			\end{tablenotes}
	\end{threeparttable}}
\end{table}

\newpage
\noindent Table 15: Jumps Associated with (Absolute) Magnitude of Earning Surprises$^*\>$
% Table generated by Excel2LaTeX from sheet 'divf5'

\begin{table}[htbp!]
	
	\centering
	
	\resizebox{0.9\textwidth}{!}{	
		
		\begin{threeparttable}
			
			\begin{tabular}{lccccccccc}
				
				\multicolumn{10}{l}{Panel A: Daily Average Percentage of Firms Exhibiting Various Types of Jumps, on Days}\\
				
				\multicolumn{8}{l}{{ }{ }{ }{ }{ }{ }{ }{ }{ }{ }{ }{ }{ }{ }{ }Characterized by Earnings Surprises}\\
				
				\midrule
				
				A-SUE & RVJP & RVJN & SRVJ & RVLJP & RVLJN & SRVLJ & RVSJP & RVSJN & SRVSJ \\
				
				\midrule
				
				Small & 0.8099 & 0.8180 & 0.9849 & 0.1951 & 0.2004 & 0.3042 & 0.7233 & 0.7258 & 0.8983 \\
				
				Medium & 0.8310 & 0.8232 & 0.9841 & 0.2216 & 0.2173 & 0.3289 & 0.7319 & 0.7232 & 0.8928 \\
				
				Large & 0.8605 & 0.8621 & 0.9909 & 0.2618 & 0.2572 & 0.3737 & 0.7455 & 0.7488 & 0.8829 \\
				
				\midrule
				
				&&&&&&&&&\\
				
				\multicolumn{10}{l}{Panel B: Daily Average Percentage of Firms Exhibiting Various Types of Jumps, on Days}\\
				
				\multicolumn{8}{l}{{ }{ }{ }{ }{ }{ }{ }{ }{ }{ }{ }{ }{ }{ }{ }Characterized by No Earnings Surprises}\\
				
				\cmidrule{1-10}
				
				& RVJP & RVJN & SRVJ & RVLJP & RVLJN & SRVLJ & RVSJP & RVSJN & SRVSJ \\
				
				\cmidrule{2-10}       & 0.8836 & 0.8786 & 0.9884 & 0.2252 & 0.2220 & 0.3107 & 0.7941 & 0.7900 & 0.9095 \\
				
				\cmidrule{1-10}
				
				
				&&&&&&&&&\\
				
				\multicolumn{10}{l}{Panel C: t-Statistics Associated with the Difference in Jump Size Percentages Between Portfolios}\\
				
				
				\cmidrule{4-7}
				
				& & & Difference  & SRVJ & SRVLJ & SRVSJ \\
				
				\cmidrule{4-7}
				
				& & & Medium-Small & -0.55 & 4.98 & -1.68 \\
				
				& & & Large-Medium  & 5.76 & 9.02 & -3.06 \\
				
				& & & Large-None & 3.54 & 16.85 & -10.85 \\
				
				\cmidrule{4-7}	
			\end{tabular}%
			
			\begin{tablenotes}	
				\item{}
				\item {\large $^*\>$ Notes: See notes to Tables 1. Panels A and B of this table report daily average percentages of firms exhibiting various types of jumps, on days with (Panel A) and without (Panel B) earnings surprises. On earning announcement dates for which at least 3 stocks report earning, the ``reporting'' stocks are sorted into tertile portfolios (called ``Small'', ``Medium'', and ``Large''), based on the absolute value of earning surprise (A-SUE), where SUE is defined in equation (\ref{eq16}). Thus, small, medium and large portfolios are only constructed on days for which at least 3 firms are characterized by an earnings surprise. Then, the percentage of firms exhibiting jumps in each of the three earnings surprise size categories is calculated, for various different jump types (i.e., RVJP, RVJN, etc.) Finally, percentages are averages over all reporting days in the sample. Finally, various Newey-West t-statistics measuring the significance of the differences in jump size percentages for SRVJ, SRVLJ, and SRVSJ type jumps are reported in Panel C of the table.}
			\end{tablenotes}
			
	\end{threeparttable}}
	
\end{table}%

\newpage
\large
\noindent Table 16: Fama-MacBeth Type Regressions Using Various Jump Variation Measures as Dependent Variable$^*\>$

% Table generated by Excel2LaTeX from sheet 'divf5'

\begin{table}[htbp!]

	\centering

	\resizebox{0.75\textwidth}{!}{	

		\begin{threeparttable}

			\begin{tabular}{lrrrrrr}

				& \multicolumn{1}{l}{SRVLJ} & \multicolumn{1}{l}{SRVLJ} & \multicolumn{1}{l}{SRVSJ} & \multicolumn{1}{l}{SRVSJ} & \multicolumn{1}{l}{SRVJ} & \multicolumn{1}{l}{SRVJ} \\

				\midrule

				& \multicolumn{1}{l}{I} & \multicolumn{1}{l}{II} & \multicolumn{1}{l}{III} & \multicolumn{1}{l}{IV} & \multicolumn{1}{l}{V} & \multicolumn{1}{l}{VI} \\

 Intercept & 0.0080 & 0.0176 & 0.0034 & 0.0159 & 0.0115 & 0.0335 \\
& (5.81) & (10.78) & (2.59) & (11.64) & (4.63) & (12.31) \\
RVOL & -0.0050 & 0.0025 & -0.0065 & 0.0006 & -0.0115 & 0.0032 \\
& (-11.02) & (5.36) & (-15.37) & (1.88) & (-14.81) & (4.71) \\
Beta &    & 0.0014 &    & -0.0011 &    & 0.0003 \\
&    & (4.40) &    & (-3.76) &    & (0.79) \\
log(Size) & 0.0013 & 0.0003 & 0.0030 & 0.0017 & 0.0043 & 0.0020 \\
& (5.79) & (1.62) & (15.19) & (9.21) & (12.71) & (6.67) \\
BE/ME &    & 0.0007 &    & 0.0003 &    & 0.0010 \\
&    & (4.11) &    & (2.72) &    & (4.58) \\
MOM &    & 0.0007 &    & 0.0011 &    & 0.0018 \\
&    & (3.54) &    & (5.94) &    & (6.13) \\
REV & 0.25166 & 0.1176 & 0.3172 & 0.1882 & 0.5688 & 0.3058 \\
& (61.05) & (31.43) & (65.55) & (41.52) & (69.66) & (39.62) \\
IVOL &    & -0.1424 &    & -0.1880 &    & -0.3303 \\
&    & (-17.05) &    & (-23.23) &    & (-24.43) \\
CSK &    & 0.0142 &    & 0.0206 &    & 0.0349 \\
&    & (19.14) &    & (28.33) &    & (26.38) \\
CKT &    & -0.0008 &    & -0.0008 &    & -0.0015 \\
&    & (-2.28) &    & (-1.96) &    & (-2.31) \\
MAX &    & 0.2456 &    & 0.2248 &    & 0.4704 \\
&    & (43.77) &    & (25.28) &    & (35.84) \\
MIN &    & 0.5115 &    & 0.4365 &    & 0.9480 \\
&    & (54.63) &    & (48.47) &    & (58.64) \\
ILLIQ & 0.0020 & 0.0016 & 0.0032 & 0.0023 & 0.0052 & 0.0039 \\
& (6.33) & (5.32) & (18.60) & (16.36) & (12.20) & (10.28) \\
Adjusted $R^2$ & 0.0322 & 0.0492 & 0.1070 & 0.1473 & 0.1049 & 
0.1517 \\

				\bottomrule

			\end{tabular}%

			\begin{tablenotes}	

				\item{}

				\item {\large $^*\>$ Notes: See notes to Tables 1, 5 and 14. This table reports results for cross-sectional Fama-MacBeth type regressions using various jump variation measures (listed across the first row of entries in the table) as dependent variables, and for various control variables (listed in the first column of the table). Thus, the regressions in this table mirror those reported in Table 14, with one difference. Namely, the dependent variable in the regressions is either SRVLJ, SRVSJ, or SRVJ. Here, SRVLJ and SRVSJ are contructed using jump truncation level $\gamma^{2}=5\sqrt{\frac{1}{t}\widehat{IV}_t^{(i)}}\Delta_n^{0.49}.$}

			\end{tablenotes}

	\end{threeparttable}}

\end{table}%


\small
\newpage
\noindent Figure 1: Unconditional Distributions of Realized Measures$^*\>$  

\begin{figure}[!htb]
	\minipage{0.49\textwidth}
	\caption*{\centering \normalfont Panel A: SRVJ Kernel Density Estimate }
	\includegraphics[width=\linewidth]{density_srvj.JPG}
	\endminipage\hfill
	\minipage{0.49\textwidth}
	\caption*{\centering \normalfont Panel B: RSK Kernel Density Estimate}
	\includegraphics[width=\linewidth]{density_rsk.JPG}
	\endminipage
\end{figure}
\begin{figure}[!htb]
	\minipage{0.49\textwidth}
	\caption*{\centering \normalfont Panel C: SRVLJ Kernel Density Estimate }
	\includegraphics[width=\linewidth]{density_srvlj.JPG}
	\endminipage\hfill
	\minipage{0.49\textwidth}
	\caption*{\centering \normalfont Panel D: SRVSJ Kernel Density Estimate}
	\includegraphics[width=\linewidth]{density_srvsj.JPG}
	\endminipage
\end{figure}
\begin{figure}[!htb]
	\minipage{0.49\textwidth}
	\caption*{\centering \normalfont Panel E: RKT Kernel Density Estimate }
	\includegraphics[width=\linewidth,height=0.21\textheight]{density_rkt.JPG}
	\endminipage\hfill
	\minipage{0.49\textwidth}
	\caption*{\centering \normalfont Panel F: RVOL Kernel Density Estimate}
	\includegraphics[width=\linewidth, height=0.21\textheight ]{density_rvol.JPG}
	\endminipage
\end{figure}
%\raggedright 
\justify
$^*\>$ Notes: See notes to Table 1. Panels A-F display unconditional distribution kernel density estimates of various realized measures, for the cross-section of stock returns for the period January 1993 to December 2016. Signed small and large jump variation measures are constructed using truncation levels  $\gamma^{1}=4\sqrt{\frac{1}{t}\widehat{IV}_t^{(i)}}\Delta_n^{0.49}$. Distributions are similar when using $\gamma^{2}=5\sqrt{\frac{1}{t}\widehat{IV}_t^{(i)}}\Delta_n^{0.49}$. 

\newpage
\centering
Figure 2: Percentiles of Realized Measures  \\

\begin{figure}[!htb]
	\minipage{0.5\textwidth}
	\caption*{\centering \normalfont Panel A: Percentiles of SRVJ }
	\includegraphics[width=\linewidth]{per_srvj.JPG}
	\endminipage\hfill
	\minipage{0.5\textwidth}
	\caption*{\centering \normalfont Panel B: Percentiles of RSK}
	\includegraphics[width=\linewidth]{per_rsk.JPG}
	\endminipage
\end{figure}
\begin{figure}[!htb]
	\minipage{0.5\textwidth}
	\caption*{\centering \normalfont Panel C: Percentiles of RKT }
	\includegraphics[width=\linewidth]{per_rkt.JPG}
	\endminipage\hfill
	\minipage{0.5\textwidth}
	\caption*{\centering \normalfont Panel D: Percentiles of RVOL}
	\includegraphics[width=\linewidth]{per_rvol.JPG}
	\endminipage
\end{figure}
\begin{figure}[!htb]
	\minipage{0.5\textwidth}
	\caption*{\centering \normalfont Panel E: Percentiles of SRVLJ Based on $\gamma^{2}$}
	\includegraphics[width=\linewidth]{srvlj_alpha2.png}
	\endminipage\hfill
	\minipage{0.5\textwidth}
	\caption*{\centering \normalfont Panel F: Percentiles of SRVSJ Based on $\gamma^{2}$}
    \includegraphics[width=\linewidth]{srvsj_alpha2.png}
	\endminipage
\end{figure}

\justify
$^*\>$ Notes: See notes to Table 1. Panels A-H display 10-week moving averages of percentiles of realized measures, for the cross-section of stocks, for the period January 1993 to December 2016. Signed small and large jump variation measures are contructed based on jump truncation level $\gamma^{2}=5\sqrt{\frac{1}{t}\widehat{IV}_t^{(i)}}\Delta_n^{0.49}$.

\newpage
\centering
\noindent Figure 3: Cumulative Gains of Short-Long Portfolios$^*\>$\\

\begin{figure}[!htb]
	\centering
	\caption*{\centering \normalfont Panel A: Equal-Weighted Mean Return}
	\includegraphics[width=0.7\linewidth]{CR2_EW_RSJ.png}
	
\end{figure}
\begin{figure}[!htb]
	\centering
	\caption*{\centering \normalfont Panel B: Value-Weighted Mean Return}
	\includegraphics[width=0.7\linewidth]{CR2_VW_RSJ.png}
\end{figure}
 
\justify
$^*\>$ Notes: Panels A-B display cumulative gains of equal-weighted and value-weighted short-long portfolios constructed using SRVJ, SRVLJ, SRVSJ, and RSK (see Table 1 and Section 2 for a discussion of these measures). RSJ is the relative signed jump variation measure defined and analyzed in Bollerslev, Li, and Zhao (2017), who include the risk-free rate in all of their calculations, while we do not (refer to Bollerslev, Li, and Zhao (2017) for complete details). In all experiments, the initial investment, made on January 1993,  is \$1. Each portfolio is re-balanced and accumulated on a weekly basis, through  2016. Signed small and large jump variation measures used in the experiment reported on in this figure are constructed based on truncation level $\gamma^{2}=5\sqrt{\frac{1}{t}\widehat{IV}_t^{(i)}}\Delta_n^{0.49}$. See Section 4.2 for further discussion.
\newpage
\centering
Figure 4: Distribution of Stocks in Portfolios Formed Based on Stocks' Signed Jump Variation (SRVJ) and Industry Signed Jump Variation$^*\>$ \\
\bigskip
%\raggedright  
\normalfont Panel A: Average Distribution of Stocks Across Double-Sorted Portfolios
\begin{figure}[!ht]
	\centering
	\includegraphics[width=0.70\textwidth]{frac25_t5.png}	
	
\end{figure}\\
%\raggedright
\normalfont Panel B: Average Distribution of Market Capitalization Across Double-Sorted Portfolios

\begin{figure}[!ht]
	\centering
	\includegraphics[width=0.68\textwidth]{frac25_me_t5.png}
	
\end{figure}

\justify
$^*\>$ Notes: See notes to Table 13. The vertical axis in Panels A and B measures time series average proportions of stocks and market capitalizations, across double sorted portfolios.  

\newpage
\centering
Figure 5: Jump Variation Measures Around Earnings Announcement  \\
\begin{figure}[htbp!]
	\minipage{0.49\textwidth}
	\caption*{\centering \normalfont Panel A1: Equal-Weighted RVJP}
	\includegraphics[width=\linewidth]{sue_ew_rvjp.png}
	\endminipage\hfill
	\minipage{0.49\textwidth}
	\caption*{\centering \normalfont Panel A2: Value-Weighted RVJP}
	\includegraphics[width=\linewidth]{sue_vw_rvjp.png}
	\endminipage
\end{figure}
\begin{figure}[htbp!]
	\minipage{0.49\textwidth}
	\caption*{\centering \normalfont Panel B1: Equal-Weighted RVJN}
	\includegraphics[width=\linewidth]{sue_ew_rvjn.png}
	\endminipage\hfill
	\minipage{0.49\textwidth}
	\caption*{\centering \normalfont Panel B2: Value-Weighted RVJN}
	\includegraphics[width=\linewidth]{sue_vw_rvjn.png}
	\endminipage
\end{figure}

\begin{figure}[htbp!]
	\minipage{0.49\textwidth}
	\caption*{\centering \normalfont Panel C1: Equal-Weighted SRVJ}
	\includegraphics[width=\linewidth]{sue_ew_srvj.png}
	\endminipage\hfill
	\minipage{0.49\textwidth}
	\caption*{\centering \normalfont Panel C2: Value-Weighted SRVJ}
	\includegraphics[width=\linewidth]{sue_vw_srvj.png}
	\endminipage
\end{figure}
		

\newpage

\centering
Figure 5 (Continued)
\begin{figure}[htbp!]
	\minipage{0.49\textwidth}
	\caption*{\centering \normalfont Panel D1: Equal-Weighted RVLJP}
	\includegraphics[width=\linewidth]{sue_ew_rvljp.png}
	\endminipage\hfill
	\minipage{0.49\textwidth}
	\caption*{\centering \normalfont Panel D2: Value-Weighted RVLJP}
	\includegraphics[width=\linewidth]{sue_vw_rvljp.png}
	\endminipage
\end{figure}
\begin{figure}[htbp!]
	\minipage{0.49\textwidth}
	\caption*{\centering \normalfont Panel E1: Equal-Weighted RVLJN}
	\includegraphics[width=\linewidth]{sue_ew_rvljn.png}
	\endminipage\hfill
	\minipage{0.49\textwidth}
	\caption*{\centering \normalfont Panel E2: Value-Weighted RVLJN}
	\includegraphics[width=\linewidth]{sue_vw_rvljn.png}
	\endminipage
\end{figure}

\begin{figure}[htbp!]
	\minipage{0.49\textwidth}
	\caption*{\centering \normalfont Panel F1: Equal-Weighted SRVLJ}
	\includegraphics[width=\linewidth]{sue_ew_srvlj.png}
	\endminipage\hfill
	\minipage{0.49\textwidth}
	\caption*{\centering \normalfont Panel F2: Value-Weighted SRVLJ}
	\includegraphics[width=\linewidth]{sue_vw_srvlj.png}
	\endminipage
\end{figure}

\newpage

\centering
Figure 5 (Continued)$^*$
\begin{figure}[htbp!]
	\minipage{0.49\textwidth}
	\caption*{\centering \normalfont Panel G1: Equal-Weighted RVSJP}
	\includegraphics[width=\linewidth]{sue_ew_rvsjp.png}
	\endminipage\hfill
	\minipage{0.49\textwidth}
	\caption*{\centering \normalfont Panel G2: Value-Weighted RVSJP}
	\includegraphics[width=\linewidth]{sue_vw_rvsjp.png}
	\endminipage
\end{figure}
\begin{figure}[htbp!]
	\minipage{0.49\textwidth}
	\caption*{\centering \normalfont Panel H1: Equal-Weighted RVSJN}
	\includegraphics[width=\linewidth]{sue_ew_rvsjn.png}
	\endminipage\hfill
	\minipage{0.49\textwidth}
	\caption*{\centering \normalfont Panel H2: Value-Weighted RVSJN}
	\includegraphics[width=\linewidth]{sue_vw_rvsjn.png}
	\endminipage
\end{figure}

\begin{figure}[htbp!]
	\minipage{0.49\textwidth}
	\caption*{\centering \normalfont Panel I1: Equal-Weighted SRVSJ}
	\includegraphics[width=\linewidth]{sue_ew_srvsj.png}
	\endminipage\hfill
	\minipage{0.49\textwidth}
	\caption*{\centering \normalfont Panel I2: Value-Weighted SRVSJ}
	\includegraphics[width=\linewidth]{sue_vw_srvsj.png}
	\endminipage
\end{figure}
%\raggedright
\justify
$^*\>$ Notes: See notes to Table 1. Panels A-I display equal- or value-weighted averages of various weekly jump variation measures in a [-12, 12] week window around earnings announcement. 

\newpage
\newpage
\centering
Figure 6: Aggregated and Weighted Average of Jump Variation Measures$^*$  \\
\begin{figure}[htbp!]
	\minipage{0.49\textwidth}
	\caption*{\centering \normalfont Panel A: Equal-Weighted SRVLJ/SRVSJ \\(Stocks Sorted by SRVLJ)}
	\includegraphics[width=\linewidth]{srvlj_ew_1.png}
	\endminipage\hfill
	\minipage{0.49\textwidth}
	\caption*{\centering \normalfont Panel B: Value-Weighted SRVLJ/SRVSJ \\(Stocks Sorted by SRVLJ)}
	\includegraphics[width=\linewidth]{SRVLJ_vw_1.png}
	\endminipage
\end{figure}

\begin{figure}[htbp!]
	\minipage{0.49\textwidth}
	\caption*{\centering \normalfont Panel C: Equal-Weighted SRVLJ/SRVSJ \\(Stocks Sorted by SRVSJ) }
	\includegraphics[width=\linewidth]{srvsj_ew_1.png}
	\endminipage\hfill
	\minipage{0.49\textwidth}
	\caption*{\centering \normalfont Panel D: Value-Weighted SRVLJ/SRVSJ \\(Stocks Sorted by SRVSJ)}
	\includegraphics[width=\linewidth]{srvsj_vw_1.png}
	\endminipage
\end{figure}

\justify
$^*\>$ Notes: See notes to Table 1. Panels A-D display weekly aggregated and weighted averages of the ratio of SRVLJ to SRVSJ for 1st quintile stocks, sorted on  SRVLJ to SRVSJ. Aggregated jump measures are depicted in blue (dotted line), and are constructed using 5-minute portfolio returns. Weighted average jump measures are depicted in red (solid line) and are constructed using individual daily jump measures, and then aggregating to weekly. All calculation utilize jump truncation level $\gamma^{1}=4\sqrt{\frac{1}{t}\widehat{IV}_t^{(i)}}\Delta_n^{0.49}$. For complete details, refer to Section 4.6.2.\\


\end{document}