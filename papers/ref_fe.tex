%%%%%%%%%%%%%%%%%%%%%%%% referenc.tex %%%%%%%%%%%%%%%%%%%%%%%%%%%%%%
% sample references
% %
% Use this file as a template for your own input.
%
%%%%%%%%%%%%%%%%%%%%%%%% Springer-Verlag %%%%%%%%%%%%%%%%%%%%%%%%%%
%
% BibTeX users please use
% \bibliographystyle{}
% \bibliography{}
%

\begin{thebibliography}{99.}%

% Journal article
%\bibitem{JCS2017} Jin, S., Corradi, V., and Swanson, N. R. (2017). Robust forecast comparison. Econometric Theory, 33(6), 1306-1351.

\bibitem{DM2002} Diebold, F.X. and Mariano, R.S. (2002). Comparing predictive accuracy. \textit{Journal of Business \& Economic Statistics}, 20(1), 134--144.  

\bibitem{W1996} West, K.D. (1996). Asymptotic inference about predictive ability. \textit{Econometrica}, 64, 1067--1084

%\bibitem{CS2006a} Corradi, V. and Swanson, N.R. (2006a). Predictive density evaluation. Handbook of economic forecasting, 1, 197-284.

\bibitem{CSO2001} Corradi, V., Swanson, N.R. and Olivetti, C. (2001). Predictive ability with cointegrated variables. \textit{Journal of Econometrics}, 104(2), 315--358.

\bibitem{R2005} Rossi, B. (2005). Testing long-horizon predictive ability with high persistence, and the Meese–Rogoff puzzle. \textit{International Economic Review}, 46(1), 61--92.

\bibitem{MR1983} Meese, R.A. and Rogoff, K. (1983). Empirical exchange rate models of the seventies: Do they fit out-of-sample? \textit{Journal of International Economics}, 14, 3--24.

\bibitem{CM2001} Clark, T.E. and McCracken, M.W. (2001). Tests of equal forecast accuracy and encompassing for nested models. \textit{Journal of Econometrics}, 105(1), 85--110.

\bibitem{CM2003} Clark, T.E. and McCracken, M.W. (2003). Evaluating long horizon forecasts. \textit{Working Paper, University of Missouri-Columbia}.

\bibitem{K1999} Kilian, L. (1999). Exchange rates and monetary fundamentals: What do we learn from long‐horizon regressions? \textit{Journal of Applied Econometrics}, 14(5), 491--510.

\bibitem{CCS2001} Chao, J., Corradi, V. and Swanson, N.R. (2001). Out-of-sample tests for Granger causality. \textit{Macroeconomic Dynamics}, 5(4), 598--620.

\bibitem{W2000} White, H. (2000). A reality check for data snooping. \textit{Econometrica}, 68(5), 1097--1126.

\bibitem{CS2007} Corradi, V. and Swanson, N. R. (2007). Nonparametric bootstrap procedures for predictive inference based on recursive estimation schemes. \textit{International Economic Review}, 48(1), 67--109.

\bibitem{CS2006} Corradi, V. and Swanson, N. R. (2006). Predictive density and conditional confidence interval accuracy tests. \textit{Journal of Econometrics}, 135(1), 187--228.

\bibitem{H2005} Hansen, R.P. (2005). A test for superior predictive ability. \textit{Journal of Business \& Economic Statistics}, 23(4), 365--380.

\bibitem{CD2011} Corradi, V. and Distaso, W. (2011). Multiple forecast model evaluation. \textit{The Oxford Handbook of Economic Forecasting, Oxford University Press, USA}, 391--414.

\bibitem{AS2010} Andrews, D.W. and Soares, G. (2010). Inference for parameters defined by moment inequalities using generalized moment selection. \textit{Econometrica}, 78(1), 119--157.

\bibitem{RW2005} Romano, J.P. and Wolf, M. (2005). Stepwise multiple testing as formalized data snooping. \textit{Econometrica}, 73(4), 1237--1282.

\bibitem{PRW1999} Politis, D.N., Romano, J.P. and Wolf, M. (1999). Subsampling. \textit{Springer Series in Statistics. New York}.

\bibitem{B1990} Bierens, H.J. (1990). A consistent conditional moment test of functional form. \textit{Econometrica}, 58, 1443--1458.

\bibitem{BP1997} Bierens, H.J. and Ploberger, W. (1997). Asymptotic theory of integrated conditional moment tests. \textit{Econometrica}, 65, 1129--1151.

\bibitem{DJ1996} De Jong, R.M. (1996). The Bierens test under data dependence. \textit{Journal of Econometrics}, 72(1), 1--32.

\bibitem{H1996a} Hansen, B.E. (1996). Inference when a nuisance parameter is not identified under the null hypothesis. \textit{Econometrica}, 64, 413--430.

\bibitem{LWG1993} Lee, T.H., White, H. and Granger, C.W.J. (1993). Testing for neglected nonlinearity in time series models: A comparison of neural network methods and alternative tests. \textit{Journal of Econometrics}, 56(3), 269--290.

\bibitem{SW1998} Stinchcombe, M.B. and White, H. (1998). Consistent specification testing with nuisance parameters present only under the alternative. \textit{Econometric Theory}, 14(3), 295--325.

\bibitem{CS2002} Corradi, V. and Swanson, N.R. (2002). A consistent test for out of sample nonlinear predictive ability. \textit{Journal of Econometrics}, 110, 353--381.

\bibitem{G1993} Granger, C.W.J. (1993). On the limitations of comparing mean square forecast errors: A comment. \textit{Journal of Forecasting}, 12(8), 651--652.

\bibitem{WA1996} Weiss, A. (1996). Estimating time series models using the relevant cost function. \textit{Journal of Applied Econometrics}, 11(5), 539--560.

\bibitem{White1982} White, H. (1982). Maximum likelihood estimation of misspecified models. \textit{Econometrica}, 50(1), 1--25. 

\bibitem{Vuong1989} Vuong, Q. H. (1989). Likelihood ratio tests for model selection and non-nested hypotheses. \textit{Econometrica}, 57(2), 307--333.

\bibitem{Giacomini2007} Gianni, A. and Giacomini R. (2007). Comparing density forecasts via weighted likelihood ratio tests. \textit{Journal of Business \& Economic Statistics}, 25(2), 177--190.

\bibitem{Kitamura2002} Kitamura, Y. (2002). Econometric comparisons of conditional models. \textit{Working Paper, University of Pennsylvania}.

\bibitem{Schorfheide2000} Schorfheide, F. (2010). Loss function-based evaluation of DSGE models. \textit{Journal of Applied Econometrics}, 15(6), 645--670.

\bibitem{Fernandez2004} Fern\'{a}ndez-Villaverde, J. and Rubio-Ram\'{i}Rez, J. F. (2004). Comparing dynamic equilibrium models to data: A bayesian approach. \textit{Journal of Econometrics}, 123(1), 153--187.

\bibitem{Chang2002} Chang, Y., Gomes, J. F. and Schorfheide, F. (2002). Learning-by-doing as a propagation mechanism. \textit{American Economic Review}, 92(5), 1498--1520.

\bibitem{Corradi2005a} Corradi, V. and Swanson, N.R. (2005). A test for comparing multiple misspecified conditional interval models. \textit{Econometric Theory}, 21(5), 991--1016.

\bibitem{Corradi2006b} Corradi, V. and Swanson, N.R. (2006). Predictive density and conditional confidence interval accuracy tests. \textit{Journal of Econometrics}, 135(1), 187--228.

\bibitem{Hall1996} Hall, P. and Horowitz, J.L. (1996). Bootstrap critical values for tests based on generalized-method-of-moments estimators. \textit{Econometrica}, 64(4), 891--916.

\bibitem{Andrews2002} Andrews, D.W.K. (2002). Higher-order improvements of a computationally attractive k-step bootstrap for extremum estimators. \textit{Econometrica}, 70(1), 119--162.

\bibitem{Andrews2004} Andrews, D.W.K. (2004). The block–block bootstrap: Improved asymptotic refinements. \textit{Econometrica}, 72(3), 673-700.

\bibitem{Inoue2006} Inoue, A. and Shintani, M. (2006). Bootstrapping GMM estimators for time series. \textit{Journal of Econometrics}, 133(2), 531--555.

\bibitem{Linton2004} Linton, O.B., Maasoumi, E. and Whang, Y.J. (2002). Consistent testing for stochastic dominance: A subsampling approach. \textit{Social Science Electronic Publishing}, 72(3), 735--765.

\bibitem{JCS2017} Jin, S., Corradi V. and Swanson, N.R. (2017). Robust forecast comparison. \textit{Econometric Theory}, 33(6), 1306--1351.

\bibitem{LMW2005} Linton, O., Maassoumi E. and Whang Y.J. (2005). Consistent tesing for stochastic dominance: A subsampling approach. \textit{Review of Economic Studies}, 72, 735--765.

\bibitem{G1999} Granger, C.W.J (1999). Outline of forecast theory using generalized cost function. \textit{Spanish Economic Review}, 1, 161--173.

\bibitem{H1979} Holm, S. (1979) A simple sequentially rejective multiple test procedure. \textit{Scandinavian Journal of Statistics}, 6, 65--70.

\bibitem{H1996b} Hansen, B.E. (1996) Stochastic equicontinuity for unbounded dependent heterogeneous arrays. \textit{Econometric Theory}, 12, 347--359.

\bibitem{Mc2000} McCracken, M.W. (2000) Robust out-of-sample inference. \textit{Journal of Econometrics}, 99, 195--223.

%
\end{thebibliography}
