%2multibyte Version: 5.50.0.2960 CodePage: 936
%% This document created by Scientific Word (R) Version 3.5

\documentclass[a4paper,amstex,11pt]{article}%
\usepackage{multirow}
\usepackage{enumerate}
\usepackage{graphics}
\usepackage{epsfig}
\usepackage{verbatim}
\usepackage{bm}
\usepackage{latexsym}
\usepackage{amsmath}
\usepackage{natbib}
\usepackage{amsmath,amsthm,amsfonts,natbib,bm,latexsym,enumerate,url}
\usepackage{graphicx}
\usepackage{amsfonts}
\usepackage{setspace}
\usepackage{amssymb}%
\setcounter{MaxMatrixCols}{30}
%TCIDATA{OutputFilter=latex2.dll}
%TCIDATA{Version=5.50.0.2960}
%TCIDATA{Codepage=936}
%TCIDATA{CSTFile=LaTeX article (bright).cst}
%TCIDATA{Created=Tue Nov 04 20:07:01 2003}
%TCIDATA{LastRevised=Saturday, December 05, 2015 16:43:13}
%TCIDATA{<META NAME="GraphicsSave" CONTENT="32">}
%TCIDATA{<META NAME="SaveForMode" CONTENT="1">}
%TCIDATA{BibliographyScheme=Manual}
%TCIDATA{<META NAME="DocumentShell" CONTENT="General\Blank Document">}
%TCIDATA{Language=American English}
%BeginMSIPreambleData
\providecommand{\U}[1]{\protect\rule{.1in}{.1in}}
%EndMSIPreambleData
\providecommand{\U}[1]{\protect \rule{.1in}{.1in}}
\providecommand{\U}[1]{\protect \rule{.1in}{.1in}}
\providecommand{\U}[1]{\protect \rule{.1in}{.1in}}
\renewcommand{\baselinestretch}{0.9}
\textwidth=6.5in \textheight=9in \oddsidemargin=0in
\evensidemargin=0in \topmargin=-0.25in
\renewcommand {\baselinestretch}{1.3}
\newtheorem{theorem}{Theorem}
\newtheorem{acknowledgement}[theorem]{Acknowledgement}
\newtheorem{algorithm}[theorem]{Algorithm}
\newtheorem{axiom}[theorem]{Axiom}
\newtheorem{case}[theorem]{Case}
\newtheorem{claim}[theorem]{Claim}
\newtheorem{conclusion}[theorem]{Conclusion}
\newtheorem{condition}[theorem]{Condition}
\newtheorem{conjecture}[theorem]{Conjecture}
\newtheorem{corollary}{Corollary}
\newtheorem{criterion}[theorem]{Criterion}
\newtheorem{definition}[theorem]{Definition}
\newtheorem{example}[theorem]{Example}
\newtheorem{exercise}[theorem]{Exercise}
\newtheorem{lemma}{Lemma}
\newtheorem{assumption}{Assumption}
\newtheorem{notation}[theorem]{Notation}
\newtheorem{problem}[theorem]{Problem}
\newtheorem{proposition}{Proposition}
\theoremstyle{remark}
\newtheorem{remark}{Remark}
\newtheorem{solution}[theorem]{Solution}
\newtheorem{summary}[theorem]{Summary}
\begin{document}

\begin{center}
\bigskip

\bigskip\bigskip{\LARGE Testing for Jumps and Jump Intensity Path Dependence*
}\bigskip\bigskip

\bigskip\bigskip

{\Large Valentina Corradi}$^{1}${\Large , Mervyn J. Silvapulle}$^{2}%
${\Large \ and Norman R. Swanson}$^{3}$

$^{1}${\Large University of Surrey,\ }$^{2}${\Large Monash University\ and
}$^{3}${\Large Rutgers University}{\large \bigskip\bigskip}

{\large December 2015}
\end{center}

\bigskip\bigskip\bigskip\bigskip

\begin{center}
\bigskip Abstract
\end{center}

{\small In this paper, we fill a gap in the financial econometrics literature,
by developing a \textquotedblleft jump test\textquotedblright\ for the null
hypothesis that the probability of a jump is zero. The test is based on
realized third moments, and uses observations over an increasing time span.
The test offers an alternative to standard finite time span tests, and is
designed to detect jumps in the data generating process rather than detecting
realized jumps over a fixed time span. More specifically, we make two
contributions. First, we introduce our largely model free jump\ test\ for the
null hypothesis of zero jump intensity. Second, under the maintained
assumption of strictly positive jump intensity, we introduce a
\textquotedblleft self excitement\ test\textquotedblright\ for the null of
constant jump intensity against the alternative of path dependent intensity.
The latter test has power against autocorrelation in the jump component, and
is a direct test for Hawkes diffusions (see e.g., A\"{\i}t-Sahalia, Cacho-Diaz
and Laeven (2015)). The limiting distributions of the proposed statistics are
analyzed via use of a double asymptotic scheme, wherein the time span goes to
infinity and the discrete interval approaches zero; and the distributions of
the tests are normal and half normal, respectively. The results from a Monte
Carlo study indicate that the tests have good finite sample properties.}

\bigskip

\bigskip

\bigskip

\bigskip

\noindent\textit{Keywords}: diffusion model, jump intensity, jump size
density, tricity.

\noindent\textit{JEL classification}: C12, C22, C52, C55.

\renewcommand {\baselinestretch}{1.0}

\_ \_ \_ \_ \_ \_ \_ \_ \_ \_ \_ \_ \_ \_ \_ \_ \_ \_ \_ \_ \_ \_ \_ \_ \_ \_ \_

{\small \renewcommand {\baselinestretch}{1.0} }

\begin{spacing}{1.0}
{\small $^{\ast}$ }{\footnotesize Valentina Corradi, School of Economics,
University of Surrey, Guildford, Surrey, GU2 7XH, UK, v.corradi@surrey.ac.uk.\ Mervyn
Silvapulle, Department of Econometrics and Business Statistics, Monash
University Caulfield, 26 Sir John Monash Drive, Caulfield East, Victoria 3145,
Australia, mervyn.silvapulle@monash.edu. Norman Swanson, Department of
Economics, Rutgers University, 75 Hamilton Street, New Brunswick, NJ 08901,
USA, nswanson@econ.rutgers.edu. We are grateful to Brendan Beare, Francesco Bravo,
Bertrand Candelon, Laura
Coroneo, Prosper Dovonon, Rustam Ibragimov, Tatsushi Oka, Peter Spencer, and to
seminar participants at the National University of Singapore, Notre Dame Univeristy, the University of York,
the 2013 Computational and Financial Econometrics Conference in London,
the 2014 CIREQ Time Series and Financial Econometrics conference in Montreal,
the 2014 UK Econometric Study Group in Bristol,
and the 2015 ESRC conference on Financial Contagion in Nice
for useful comments
and suggestions.}
\end{spacing}


\renewcommand {\baselinestretch}{1.3}

\newpage

\clearpage


\allowdisplaybreaks


\section{Introduction}

Jump diffusions are widely used in the financial econometrics literature when
analyzing returns or exchange rates, as discussed in Duffie, Pan and Singleton
(2000), Singleton (2001), Anderson, Benzoni and Lund (2002), Jiang and Knight
(2002), Chacko and Viceira (2003) and Eraker, Johannes and Polson (2003),
among others. In this context, various estimation techniques have been
developed, and the common practice is to jointly estimate the parameters of
both the continuous time and the jump components of models. Thus, parameters
characterizing the drift, variance, jump intensity, and jump size probability
density are jointly estimated. However, an obvious non-standard feature of
this class of models is that the parameters characterizing the jump size
density are not identified when the jump intensity is identically zero. This
is an issue both when the intensity parameter is constant, as in standard
stochastic volatility models with jumps (see, e.g. Andersen, Benzoni and Lund
(2002)) as well as when the intensity follows a diffusion process, as in the
important case of the Hawkes diffusion models analyzed by A\"{\i}t-Sahalia,
Cacho-Diaz and Laeven (2015). If one estimates a jump diffusion model that
contains a jump intensity parameter and if the population jump intensity
happens to be zero, then a subset of the parameters in the model is not
identified, which in turn precludes consistent estimation of other parameters
(see Andrews and Cheng (2012)).

The above estimation problem serves to underscore the importance of pretesting
for jumps. In the extant literature, there are a large variety of tests for
the null of no jumps versus the alternative of jumps. Tests include those
based on the comparison of two realized volatility measures, one which is
robust, and the other which is not robust to the presence of jumps (see, e.g.
Barndorff-Nielsen, Shephard and Winkel (2006) and Podolskji and Vetter
(2009a)), tests based on a thresholding approach (see, e.g. Corsi, Pirino, and
Ren\`{o} (2010), Lee and Mykland (2008), and Lee, Loretan and Ploberger
(2013)), and tests based on power variation, as discussed in A\"{\i}t-Sahalia
and Jacod (2009). Such tests are consistent against realized\ jumps. One
feature of these tests is that they are based on observations drawn on a given
finite time span, and they can thus only detect whether jumps occurred during
this given time span. While this is hardly a weakness of the existing tests,
there are clearly situations for which interest lies in testing for the
existence of jumps in the data generating process, or within a class of
models. For example, this is the case if one is interested in using
(transformations of) jump diffusion processes in a variety of valuation
problems, such as option pricing and default modelling (see, e.g., Duffie, Pan
and Singleton (2000)). It follows that tests for jumps using an increasing
time span are needed, although such tests have not previously been proposed.

In this paper, we fill the above gap in the literature by developing a
\textquotedblleft jump test\textquotedblright\ of the null hypothesis that the
probability of a jump is zero. In addition, under the maintained assumption of
strictly positive jump intensity, we introduce a \textquotedblleft self
excitement\ test\textquotedblright\ for the null of constant jump intensity
against the alternative of path dependent intensity. This test is found to
have power against autocorrelation in the jump component, and is a direct test
for Hawkes diffusions (see A\"{\i}t-Sahalia, Cacho-Diaz and Laeven 2015), in
which jump intensity is modeled as a mean-reverting diffusion process. When
the proposed tests are implemented prior to model specification, standard
estimation of jump diffusions can be subsequently carried out, avoiding the
identification problems discussed above.

Our jump test is based on realized third moments, or so-called tricity.
Various realized tricity-type statistics over a finite time span have already
been examined in the literature in order to: detect realized jumps, as in
Jacod (2012); study the contribution of realized skewness when predicting the
cross-section of equity returns, as in Amaya, Christoffersen, Jacobs and
Vasquez (2015); and to test for the endogeneity of sampling times, as in Li,
Mykland, Renault, Zhang and Zheng (2014). What distinguishes our tricity-type
test from these is that it is analyzed using both in-fill and long-span
asymptotics. The use of long-span asymptotics ensures that the suggested
statistic has power against jump intensity rather than against realized jumps.
Importantly, our test is also robust to the presence of leverage. The limiting
behavior of the proposed statistic is readily analyzed via use of a double
asymptotic scheme wherein the time span goes to infinity and the discrete
interval approaches zero. Under the null hypothesis of zero intensity, the
statistic has a normal limiting distribution. Under the alternative, it is
necessary to distinguish between jumps with zero or non-zero third moment. In
the latter case, the proposed test has a well defined Pitman drift and has
power against $\sqrt{T}-$local alternatives, where $T$ is the time span, in
days. In the former case, the sample third moment approaches zero, but the
probability order of the statistic is larger than that which obtains under the
null, since the jump component does not contribute to the mean, while it does
contribute to the variance. As the order of magnitude of the variance depends
on whether the null hypothesis is true or not, we introduce a threshold
estimator for the variance, which is consistent under the null of zero
intensity, and bounded in probability under the alternative. Thus, inference
can be performed via use of a simple $t$-statistic.

As alluded to above, our self excitement test is based on the autocorrelation
function of the jump component. A necessary condition for test consistency is
that the mean of the jump size is non-zero. We thus additionally propose a
simple \textquotedblleft zero mean jump test\textquotedblright\ for the null
hypothesis of zero jump mean. The self excitement test is implemented as
follows. First, carry out the jump test. If the null of zero intensity is
rejected, then carry out a simple test in order to ascertain whether the jump
process is zero mean. If this test is in turn rejected, carry out the self
excitement test, in order to ascertain whether the jump intensity is a
constant, or is path dependent.

In principle, one might consider testing for the null of zero intensity using
a score, Wald or likelihood ratio test, based on discrete observations (see.,
e.g., Andrews (2001)). This approach requires treating jump size density
parameters as nuisance parameters unidentified under the null, and requires
correct specification of both the continuous and the jump components of the
diffusion. Misspecification of one or both components will invalidate the
test. Additionally, the likelihood function of a jump diffusion is not
generally known in closed form, and therefore estimation (which is needed for
test statistic construction) is usually based on either simulated GMM (see
Duffie and Singleton (1993) and Anderson, Benzoni and Lund (2002)), indirect
inference (see Gourieroux and Monfort (1993) and Gallant and Tauchen (1996)),
or nonparametric simulated maximum likelihood (see Fermanian and Salani\'{e}
(2004) and Corradi and Swanson (2011)). However, it goes without saying that
one cannot simulate a diffusion with a negative intensity parameter. This, in
turn, precludes the existence of a quadratic approximation around the null
parameters of the criterion function to be maximized (minimized). Given that
the existence of such quadratic approximations is a necessary condition for
estimation and inference about parameters on the boundary (see Andrews (1999,
2001), Beg, Silvapulle and Silvapulle (2001), and Chapter 4 in Silvapulle and
Sen (2005)), we cannot rely on simulation-based estimators when testing using
standard score, Wald or likelihood ratio tests.

The finite sample behavior of the tests is studied in a Monte Carlo
experiment. Since the tests are not robust to microstructure noise, one needs
to choose a frequency for which the noise is not too binding. For this reason,
in our Monte Carlo exercise, we set the discretization interval $\Delta=1/78$
and $\Delta=1/156,$ which correspond to moderate frequencies. The empirical
size of the jump test is sensitive to the smallest values of $T$ and
$\Delta^{-1}$, but performance is markedly better as the magnitude of these
parameters is increased. Moreover, the power is quite good across all
parameterizations, even in the case of jumps with zero third moment. The zero
mean jump test is well sized for all parameterizations, and power is quite
good, other than for parameter combinations with the smallest values of the
jump intensity, the shortest time spans, and the widest discretization
intervals. The self excitement test likewise has very good size properties,
for all parameterizations, and has good power, which increases with the level
of path dependence.

The rest of the paper is organized as follows. Section 2 describes the set-up.
Section 3 and Section 4 discuss the jump intensity and self excitement tests,
and derive their asymptotic properties, respectively. Section 5 reports the
findings of a Monte Carlo experiment designed to examine the finite sample
properties of the tests, and concluding remarks are gathered in Section 6. All
proofs are collected in an appendix.

\section{Set-Up}

\noindent We consider stochastic volatility jump diffusions, with either
constant or path dependent intensity. For $t\in\mathbb{R}^{+},$ consider%
\begin{equation}
\mathrm{d\ln}X_{t}=\mu\mathrm{d}t+V_{t}^{1/2}\sqrt{1-\rho^{2}}\mathrm{d}%
W_{1,t}+V_{t}^{1/2}\rho\mathrm{d}W_{2,t}+Z_{t}\mathrm{d}N_{t}, \label{X}%
\end{equation}
and%
\begin{equation}
\mathrm{d}V_{t}=\mu(V_{t},\theta)\mathrm{d}t+g\left(  V_{t},\theta\right)
\mathrm{d}W_{2,t}, \label{vc}%
\end{equation}
with $-1\leq\rho\leq1.$ Additionally, $W_{1,t}$ and $W_{2,t}$ are independent
standard Brownian motions. From (\ref{X}) and (\ref{vc}), it is immediate to
see that the specification of the volatility process is rather general, as the
drift and variance terms in (\ref{vc}) need only to ensure the existence of a
strong solution, $V_{t}>0.$ For example, $V_{t}$ can be generated by a square
root process, which is the case considered in the Monte Carlo study. On the
other hand, we are imposing functional structure on the variance process in
the asset process, as well as a constant drift. While, the first restriction
can be relaxed at the cost of more notation, and some additional assumptions,
the constancy of the drift is used in the proofs, and cannot be relaxed in the
test for (no) path dependence in the jump intensity.

In the sequel, we assume that the jump process, $N_{t},$ is a finite activity
process. Namely, we focus on the case of a small number of large jumps. More
precisely,
\begin{equation}
\Pr\left(  N_{t+\Delta}-N_{t}=1|\mathcal{F}_{t}\right)  =\lambda_{t}%
\Delta+o\left(  \Delta\right)  , \label{J1}%
\end{equation}%
\begin{equation}
\Pr\left(  N_{t+\Delta}-N_{t}=0|\mathcal{F}_{t}\right)  =1-\lambda_{t}%
\Delta+o\left(  \Delta\right)  , \label{J2}%
\end{equation}
and%
\begin{equation}
\Pr\left(  N_{t+\Delta}-N_{t}>1|\mathcal{F}_{t}\right)  =o\left(
\Delta\right)  , \label{J3}%
\end{equation}
where $\mathcal{F}_{t}=\sigma\left(  N_{s},\text{ }0\leq s\leq t\right)  .$
Additionally, in the sequel $1_{\Delta_{N_{(k+1)\Delta}}}=1$ if a jump occurs
between time $k\Delta$ and $(k+1)\Delta$, and the associated jump size,
$Z_{k},$ is identically and independently distributed with density
$f(z;\gamma)$.

We consider two general cases. The first is that of Poisson jumps, in which
$\lambda_{t}=\lambda,$ for all $t.$ The second is that of a Hawkes diffusion,
in which the intensity is an increasing function of past jumps (see Bowsher
(2007) and A\"{\i}t-Sahalia, Cacho-Diaz and Laeven (2015)). In this case:%

\[
\lambda_{t}=\lambda_{\infty}+\beta\int_{0}^{t}\exp\left(  -a\left(
t-s\right)  \right)  \mathrm{d}N_{s},
\]
with $\lambda_{\infty}\geq0,$ $\beta\geq0,$ $a>0,$ and $a>\beta$ (in order to
ensure intensity mean reversion). Thus,%
\begin{equation}
\mathrm{d}\lambda_{t}=a\left(  \lambda_{\infty}-\lambda_{t}\right)
\mathrm{d}t+\beta\mathrm{d}N_{s} \label{lambda}%
\end{equation}
and
\[
\mathrm{E}\left(  \lambda_{t}\right)  =\frac{a\lambda_{\infty}}{a-\beta
}=\lambda.
\]
If $\lambda_{\infty}=0,$ then \textrm{E}$\left(  \lambda_{t}\right)  =0;$ and
since $\lambda_{t}$ can never be negative, this in turn implies that
$\lambda_{t}=0$ a.s., for all $t$ (i.e., $N_{t}=0$ a.s., for all $t).$ But, if
$N_{t}=0$ a.s., for all $t,$ then $\beta$ cannot be identified, and
consequently $a$ is not identified. Furthermore, if $N_{t}=0$ a.s., for all
$t,$ then $\gamma$ cannot be identified.

In summary, if $\lambda_{\infty}=0,$ then $\beta,\alpha,$and $\gamma$ are not
identified. By contrast, if $\lambda_{\infty}>0,$ then $\gamma$ and $\beta$
are identified. However, if $\lambda_{\infty}>0$ but $\beta=0,$ then $a$ is
not identified.

These observations highlight the importance of being very clear as to which of
the two assumptions, $\lambda_{\infty}=0$ or $\lambda_{\infty}>0,$ is made for
statistical inference in the foregoing Hawkes diffusion model. In practice,
thus, we are concerned with the following jump test hypotheses: $H_{0}%
:\lambda=0$ versus $H_{A}:\lambda>0,$ where hereafter $\lambda$ denotes
expected intensity (i.e. $\lambda=\mathrm{E}\left(  \lambda_{t}\right)
).$\footnote{Note that testing for $\lambda=0$ $(>0)$ implies and is implied
by $\lambda_{\infty}=0$ $(>0).$ Also, if $\beta=0,$ $\lambda=\lambda_{\infty
}.$} This is a nonstandard inference problem because, under $H_{0},$ some
parameters are not identified and a parameter lies on the boundary of the null
parameter space. Additionally, depending upon the outcome of tests of the
above hypotheses, we are also interested in the following self excitement test
hypotheses: $H_{0}:\beta=0$ versus $H_{A}:\beta>0.$

Another important class of jump diffusion is the affine jump diffusion, in
which intensity is an affine function of a state variable (see e.g., Duffie,
Pan and Singleton (2000) and Singleton (2001)). For example, in our set-up one
can define intensity to be an affine function of the volatility process,%
\begin{equation}
\lambda_{t}=\lambda_{0}+\lambda_{1}V_{t},\text{ }\lambda_{0},\lambda_{1}\geq0,
\label{aff}%
\end{equation}
so that the probability of a jump is positively correlated with the volatility
level. Whenever $\lambda_{t}$ is generated as in (\ref{lambda}) or as in
(\ref{aff}), it displays autocorrelation. However, in the former case, the
positive autocorrelation in $\lambda_{t}$ is reflected in a positive
autocorrelation in $\left(  N_{t+\Delta}-N_{t}\right)  .$ This is because the
realization of a jump at time $t$ increases the probability of a jump
realization at time $t+\Delta.$ By contrast, in the affine case, a jump
realization at time $t,$ does not affect the probability of having another
jump at time $t+\Delta.$ However, given (\ref{aff}), jumps appear positively
correlated during periods of increasing volatility and vice-versa. Thus, jumps
are not independently distributed. Nevertheless we cannot observe a clear
pattern in their autocorrelation structure. As a consequence, our test for
zero intensity has power against affine jumps. However, the test for
self-excitation has power against positive autocorrelation in jumps, and may
have no power against affine jumps.

If we are willing to parametrically specify the continuous and the jump
components of the model, and most importantly if the transition density is
known in closed form, then it is easy to construct a consistent test for
jumps, based only on a long time span of discrete observations. In particular
one can easily test $H_{0}:\lambda=0$ versus $H_{A}:\lambda>0$. This fact can
be illustrated by considering a score test. Suppose that the skeleton of the
process, $\ln X_{t}$ in (\ref{X}) is observed. Namely, $\ln X_{1},\ln
X_{2},...,\ln X_{T},$ is observed. Now, using the notation in (\ref{X}%
)-(\ref{J3}), let $\delta=\left(  \theta,\mu,\rho,\lambda,\gamma\right)
=\left(  \vartheta,\gamma\right)  .$ It follows immediately that, provided the
transition density is known in closed form, the likelihood can be written as:%
\[
l_{T}(\vartheta,\gamma)=\frac{1}{T}\sum_{t=1}^{T-1}l_{t}(\vartheta
,\gamma)=\frac{1}{T}\sum_{t=1}^{T-1}\ln f_{t+1|t}\left(  Y_{t+1}%
|Y_{t},\vartheta,\gamma\right)  .
\]
The score statistic for testing $H_{0}$ is thus\footnote{If $\lambda$ is not
scalar (for example, consider allowing for different up and down jump
intensities, as in Chacko and Viceira (2003)), then the score statistic can be
written as:%
\begin{align*}
K\left(  \gamma\right)   &  =U_{T}\left(  \gamma\right)  ^{\prime}\left(
R\widehat{\mathcal{I}}_{T}(\gamma)^{-1}\widehat{V}_{T}(\gamma
)\widehat{\mathcal{I}}_{T}(\gamma)^{-1}R^{\prime}\right)  ^{-1}U_{T}\left(
\gamma\right) \\
&  -\inf_{\lambda\geq0}\left(  U_{T}\left(  \gamma\right)  -\lambda\right)
^{\prime}\left(  R\widehat{\mathcal{I}}_{T}(\gamma)^{-1}\widehat{V}_{T}%
(\gamma)\widehat{\mathcal{I}}_{T}(\gamma)^{-1}R^{\prime}\right)  ^{-1}\left(
U_{T}\left(  \gamma\right)  -\lambda\right)
\end{align*}
\par
{}}:%
\[
K_{T}\left(  \gamma\right)  =\max\left\{  0,\left(  R\widehat{\mathcal{I}}%
_{T}(\gamma)^{-1}\widehat{V}_{T}(\gamma)\widehat{\mathcal{I}}_{T}(\gamma
)^{-1}R^{\prime}\right)  ^{-1/2}U_{T}\left(  \gamma\right)  \right\}  ,
\]
where $R$ is a $1\times p$ matrix, with $p$ denoting the dimension of
$\vartheta$, and where%

\[
U_{T}\left(  \gamma\right)  =\sqrt{T}\left(  R\widehat{\mathcal{I}}_{T}%
(\gamma)^{-1}\nabla_{\vartheta}l_{T}\left(  \widehat{\vartheta}_{T}%
,\gamma\right)  \right)  ,
\]%
\begin{equation}
\widehat{\mathcal{I}}_{T}(\gamma)=\frac{1}{T}\sum_{t=1}^{T}\nabla
_{\vartheta\vartheta}l_{t}\left(  \widehat{\vartheta}_{T},\gamma\right)  ,
\label{HAT-HES}%
\end{equation}%
\[
\widehat{\vartheta}_{T}=\arg\max_{\vartheta}l_{T}\left(  \vartheta
,\gamma\right)  \text{ s.t. }R\vartheta=\lambda_{\infty}=0,
\]
and%
\begin{equation}
\widehat{V}_{T}(\gamma)=\frac{1}{T}\sum_{j=-\tau_{T}}^{\tau_{T}}\sum
_{t=\tau_{T}}^{T-\tau_{T}}\omega_{j}\nabla_{\vartheta}l_{t}\left(
\widehat{\vartheta}_{T},\gamma\right)  \nabla_{\vartheta}l_{t+j}\left(
\widehat{\vartheta}_{T},\gamma\right)  ^{\prime},\text{ }\omega_{j}=1-\frac
{j}{1+\tau_{T}}. \label{HAT-HAC}%
\end{equation}
Now, given mild regularity assumptions controlling the smoothness of the
likelihood, under the null of $\lambda=0,$%
\[
\sup_{\gamma\in\Gamma}K\left(  \gamma\right)  \overset{d}{\rightarrow}%
\sup_{\gamma\in\Gamma}\max\left\{  0,\left(  R\mathcal{I}(\gamma)^{-1}%
V(\gamma)\mathcal{I}(\gamma)^{-1}R^{\prime}\right)  ^{-1/2}Z\left(
\gamma\right)  \right\}  ,
\]
where $\sup_{\gamma\in\Gamma}\left\vert \widehat{\mathcal{I}}_{T}%
(\gamma)-\mathcal{I}(\gamma)\right\vert =o_{p}(1),$ $\sup_{\gamma\in\Gamma
}\left\vert \widehat{V}_{T}(\gamma)-V(\gamma)\right\vert =o_{p}(1),$ and
$Z(\cdot)$ is a Gaussian process with covariance kernel,%
\[
C(\gamma_{1},\gamma_{2})=\left(
\begin{array}
[c]{cc}%
R\mathcal{I}(\gamma_{1})^{-1}V(\gamma_{1},\gamma_{1})\mathcal{I}(\gamma
_{1})^{-1}R^{\prime} & R\mathcal{I}(\gamma_{1})^{-1}V(\gamma_{1},\gamma
_{2})\mathcal{I}(\gamma_{2})^{-1}R^{\prime}\\
R\mathcal{I}(\gamma_{2})^{-1}V(\gamma_{1},\gamma_{2})\mathcal{I}(\gamma
_{1})^{-1}R^{\prime} & R\mathcal{I}(\gamma_{2})^{-1}V(\gamma_{2},\gamma
_{2})\mathcal{I}(\gamma_{2})^{-1}R^{\prime}%
\end{array}
\right)  ,
\]
where $V(\gamma_{1},\gamma_{2})=\mathrm{p\lim_{T\rightarrow\infty}}%
\widehat{V}_{T}(\gamma_{1},\gamma_{2}).$

Note also that $\sup_{\gamma\in\Gamma}K\left(  \gamma\right)  $ diverges to
infinity under the alternative$.$ This test has power against $\sqrt{T}-$local
alternatives. Additionally, the limiting behavior of the test depends on the
quadratic approximation of the likelihood around $\lambda=0$ (see Andrews
(2001)). Hence, if the likelihood is known in closed form, and if both the
continuous and the jump components of the model, including the density of the
jumps size, are correctly specified, then inference can be easily carried out
using this score test, or using analogous Wald or likelihood ratio tests.
However, it is well known that for most empirically relevant models the
likelihood is usually not known in closed form. In such cases, as discussed in
the introduction, one often relies on simulation based estimation techniques
such as simulated GMM, indirect inference, or nonparametric simulated maximum
likelihood. However, as one cannot simulate observations with negative
intensity, a quadratic approximation of the criterion function cannot be
constructed, and these sorts of tests are not applicable. It is for this
reason that we instead focus on simple moment based jump and self-excitement tests.

\section{Test of $\lambda=0$ (Jump Test)}

As mentioned in the introduction, tests based on high frequency observations
over a finite time span are model free, but have power only against realized
jumps, and thus cannot be consistent against the alternative $\lambda>0$. On
the other hand, tests based on discrete observations over a long time span are
consistent against $\lambda>0,$ but require correct specification of both the
continuous and jump components, as well as knowledge of the transition
density. In order to have tests that are consistent against $\lambda>0,$ but
are still to a large extent model free, we use functions of sample moments and
rely on double in-fill and long-time span asymptotic approximations.

In the sequel, assume the existence of a sample of $n^{+}$ observations over
an increasing time span $T^{+}$ and a shrinking discrete interval $\Delta,$ so
that $n^{+}=\frac{T^{+}}{\Delta},$ with $T^{+}\rightarrow\infty$ and
$\Delta\rightarrow0.$ Define $n=\frac{T}{\Delta}=n^{+}-\frac{T^{+}-T}{\Delta
},$ with $T^{+}>T,$ and $T^{+}/T\rightarrow\infty.$ We first test for zero
jump intensity ($\lambda=0)$. The hypotheses of interest are:%
\[
H_{0}^{\lambda}:\lambda=0
\]
versus%
\[
H_{A}^{\lambda}:\lambda>0,
\]
where%
\[
H_{A}^{\lambda}=H_{A}^{\lambda(1)}\cup H_{A}^{\lambda(2)}:\left(
\lambda>0\text{ and \textrm{E}}\left(  Z_{k}^{3}\right)  \neq0\right)
\cup\left(  \lambda>0\text{ and \textrm{E}}\left(  Z_{k}^{3}\right)
=0\right)  .
\]
Notice that the alternative hypothesis is the union of two different
alternatives, depending on whether \textrm{E}$\left(  Z_{k}^{3}\right)  \neq0$
or \textrm{E}$\left(  Z_{k}^{3}\right)  =0.$ Let:%
\begin{align}
\widehat{\mu}_{3,T,\Delta}  &  =\frac{1}{T}\sum_{k=1}^{n}\left(  \ln
X_{k\Delta}-\ln X_{(k-1)\Delta}-\frac{\ln X_{n\Delta}-\ln X_{\Delta}}%
{n}\right)  ^{3}\nonumber\\
&  -\frac{1}{T^{+}}\sum_{k=1}^{n^{+}}\left(  \ln X_{k\Delta}-\ln
X_{(k-1)\Delta}-\frac{\ln X_{n^{+}\Delta}-\ln X_{\Delta}}{n^{+}}\right)
^{3}1\left\{  \left\vert \ln X_{k\Delta}-\ln X_{(k-1)\Delta}\right\vert
\leq\tau\left(  \Delta\right)  \right\}  , \label{mu3}%
\end{align}
with $T^{+}/T\rightarrow\infty,$ $\tau\left(  \Delta\right)  \rightarrow0,$
and $\tau\left(  \Delta\right)  /\Delta^{1/2}\rightarrow\infty,$ and define
the statistic:\footnote{The first term on the RHS of $\widehat{\mu
}_{3,T,\Delta}$ can also be expressed as $\frac{1}{T}\sum_{k=n^{+}-n+1}%
^{n^{+}}\left(  \ln X_{k\Delta}-\ln X_{(k-1)\Delta}-\frac{\ln X_{n\Delta}-\ln
X_{\Delta}}{n}\right)  ^{3}.$ However, it is necessary to use a longer sample
size for the second term.}%
\begin{equation}
S_{T,\Delta}=\frac{T^{1/2}}{\Delta}\widehat{\mu}_{3,T,\Delta}. \label{St}%
\end{equation}


\noindent The logic underlying the suggested statistic is the following. As
outlined in the proof of Theorem 1,%
\begin{align*}
&  \frac{1}{T}\sum_{k=1}^{n}\left(  \ln X_{k\Delta}-\ln X_{(k-1)\Delta}%
-\frac{\ln X_{n\Delta}-\ln X_{\Delta}}{n}\right)  ^{3}\\
&  =\mathrm{E}\left(  Z_{k}1_{\Delta_{N_{(k+1)\Delta}}}\right)  ^{3}+\frac
{3}{2}\Delta\rho^{3}\mathrm{E}\left(  V_{k\Delta}^{1/2}g\left(  V_{k\Delta
},\theta\right)  \right)  +o_{p}\left(  \frac{\Delta}{T^{1/2}}\right)  ,
\end{align*}
so that $\left\vert \frac{1}{\sqrt{T}\Delta}\sum_{k=1}^{n}\left(  \ln
X_{k\Delta}-\ln X_{(k-1)\Delta}-\frac{\ln X_{n\Delta}-\ln X_{\Delta}}%
{n}\right)  ^{3}\right\vert $ may diverge to infinity in probability either
because of the presence of jumps or because of the presence of leverage, or
both. Hence, we need to correct for the skewness due to the presence of
leverage. This is the role played by the second term on the RHS of
(\ref{mu3}). In fact, regardless of the presence of jumps, provided that
$\tau\left(  \Delta\right)  \rightarrow0$ at an appropriate rate,%
\begin{align*}
&  \frac{1}{T^{+}}\sum_{k=1}^{n^{+}}\left(  \ln X_{k\Delta}-\ln X_{(k-1)\Delta
}-\frac{\ln X_{n^{+}\Delta}-\ln X_{\Delta}}{n^{+}}\right)  ^{3}1\left\{
\left\vert \ln X_{k\Delta}-\ln X_{(k-1)\Delta}\right\vert \leq\tau\left(
\Delta\right)  \right\} \\
&  =\frac{3}{2}\Delta\rho^{3}\mathrm{E}\left(  V_{k\Delta}^{1/2}g\left(
V_{k\Delta},\theta\right)  \right)  +o_{p}\left(  \frac{\Delta}{T^{+1/2}%
}\right)  .
\end{align*}
The requirement of $T^{+}/T\rightarrow\infty$ ensures that the contribution of
leverage estimation error is negligible.

Furthermore, as shown in a number of lemmata in the Appendix, the thresholding
effect is asymptotically negligible under the null, and thus for $T=T^{+}$ the
statistic is degenerate under $H_{0}^{\lambda}.$ On the other hand, if
$T^{+}/T\rightarrow0,$ the limiting behaviour is driven by the threshold term,
and the statistic lacks power against jumps.

With regard to the $\tau\left(  \Delta\right)  $ term in (\ref{mu3}), note
that thresholding is a well established technique for disentangling the jump
component from the continuous component in various estimation and testing
frameworks (see Mancini (2009) and Mancini and Ren\`{o} (2011)). In these
papers the threshold sequence, $\tau\left(  \Delta\right)  ,$ is selected so
that $\tau\left(  \Delta\right)  \rightarrow0$ and $\frac{\tau\left(
\Delta\right)  }{\sqrt{\Delta\log\left(  1/\Delta\right)  }}\rightarrow0,$
which is dictated by the law of the iterated logarithm of the Brownian
component of the model. In the theorems below we would require mildly stronger
conditions on $\tau\left(  \Delta\right)  $ because the time span is growing.

Before establishing the asymptotic properties of $S_{T,\Delta},$ it should be
pointed out that a central limit theorem for realized third moments has been
proven in Li, Mykland, Renault, Zhang and Zheng (2014). Their Theorem 2
establishes asymptotic mixed normality for tricity in the case of unequal
random times and finite time spans. Namely, they show that $\frac{1}{\Delta
}\sum_{t_{k}\leq1}\left(  \ln X_{t_{k+1}}-\ln X_{t_{k}}\right)  ^{3}$ has a
mixed normal limiting distribution under the null of exogenous sampling time,
and diverges otherwise.

\noindent In the sequel, we need the following assumption.

\noindent\textbf{Assumption A: (i)} $\ln X_{t}$ and $V_{t}$ are generated as
in (\ref{X}) and (\ref{vc}), with $\mu(v,\theta)$ and $g\left(  v,\theta
\right)  $ twice continuously differentiable, satisfying local Lipschitz and
growth conditions for all $\theta\in\Theta,$\thinspace\textbf{(ii)} $V_{t}$ is
geometrically ergodic, \textbf{(iii)} \textrm{E}$\left(  V_{t}^{\frac{m}{2}%
}\right)  <\infty$ and \textrm{E}$\left(  g(V_{t})^{2}\right)  <\infty$ for
even integer $m>6,$ (\textbf{iv}) $N_{t}$ satisfies (\ref{J1})-(\ref{J3}), and
$\lambda_{t}$ is either constant or it satisfies (\ref{lambda}), with
$\lambda_{\infty}\geq0,$ $\beta\geq0,$ $a>0,$ and $a>\beta$, and (\textbf{v})
the jump size, $Z_{k},$ is independently and identically distributed, with
density $f(z;\gamma),$ and $\mathrm{E}\left(  \left\vert Z_{k}\right\vert
^{\kappa}\right)  <\infty,$ for $\kappa\geq6.$

\noindent\textbf{Theorem 1:}\textit{ Let Assumption}\textbf{ A}\textit{ hold.
Also, assume that as }$T\rightarrow\infty,$ $\Delta\rightarrow0,$
$T\Delta\rightarrow\infty,$ $\sqrt{T}\Delta\rightarrow0,$ and $\frac
{\Delta^{\frac{1}{2}-\frac{3}{m}}}{\tau\left(  \Delta\right)  }\rightarrow0,$
\textit{for even }$m>6,$\textit{ and }$T^{+}/T\rightarrow\infty.$\textit{
Then,}

\noindent\textit{(i) Under }$H_{0}^{\lambda}:$%
\[
S_{T,\Delta}\overset{d}{\rightarrow}N\left(  0,\omega_{0}\right)  ,
\]
\textit{where}%
\[
\omega_{0}=\left(  15\left(  1-\rho^{2}\right)  ^{3}+15\rho^{6}+45\left(
1-\rho^{2}\right)  ^{2}\rho^{2}+45\left(  1-\rho^{2}\right)  \rho^{4}\right)
\mathrm{E}\left(  V_{k\Delta}^{3}\right)  .
\]
\textit{ }\noindent\textit{(ii) Under }$H_{A}^{\lambda(1)},$\textit{ there
exists an }$\varepsilon>0,$\textit{ such that:}%
\[
\lim_{T\rightarrow\infty,\Delta\rightarrow0}\Pr\left(  \frac{\Delta}{\sqrt{T}%
}\left\vert S_{T,\Delta}\right\vert >\varepsilon\right)  =1.
\]
\noindent\textit{(iii) Under }$H_{A}^{\lambda(2)}$\textit{, there exists an
}$\varepsilon>0,$\textit{ such that:}%
\[
\lim_{T\rightarrow\infty,\Delta\rightarrow0}\Pr\left(  \Delta\left\vert
S_{T,\Delta}\right\vert >\varepsilon\right)  =1.
\]


\noindent\textbf{Remark 1: }It follows immediately that $S_{T,\Delta}$
converges to a normal random variable under the null hypothesis, diverges at
rate $\frac{\sqrt{T}}{\Delta}$ under the alternative of jumps with non-zero
third moment, and diverges at the slower rate of $\frac{1}{\Delta}$ under the
alternative of jumps symmetric around zero. As shown in the appendix (see
equation (\ref{Z3})), as $\Delta\rightarrow0$ and $T\rightarrow\infty,$ we
have that $\widehat{\mu}_{3,T,\Delta}=\lambda\mathrm{E}\left(  Z^{3}\right)
+2\lambda^{2}\Delta\mathrm{E}\left(  Z\right)  \mathrm{E}\left(  Z^{2}\right)
+o_{p}\left(  \Delta\right)  .$ Now, if $\mathrm{E}\left(  Z_{k}^{3}\right)
\neq0,$ (i.e., under $H_{A}^{\lambda(1)}),$ the test has a well defined Pitman
drift against $\sqrt{T}-$alternatives. On the other hand, if jumps are
symmetric around zero (i.e., $\mathrm{E}\left(  Z_{k}^{3}\right)
=\mathrm{E}\left(  Z_{k}\right)  =0),$ then $\lambda\mathrm{E}\left(
Z_{k}^{3}\right)  $ is not identified, and under $H_{A}^{\lambda(2)}$ the
Pitman drift is zero. Indeed, $\widehat{\mu}_{3,T,\Delta}%
\overset{p}{\rightarrow}0$ regardless of whether $\lambda=0$ or $\lambda>0$ in
this case$.$ Although it is not possible to distinguish between $H_{0}%
^{\lambda}$ and $H_{A}^{\lambda(2)}$ on the basis of the different locations
of the limiting distribution (i.e., the Pitman drift), it is possible to
distinguish between them on the basis of different scales of the limiting
distribution of $\frac{T^{1/2}}{\Delta}\widehat{\mu}_{3,T,\Delta}.$ This is
because the order of magnitude of the variance of $\frac{T^{1/2}}{\Delta
}\widehat{\mu}_{3,T,\Delta}$ is larger when $\lambda>0$ and $\mathrm{E}\left(
Z_{k}^{3}\right)  =\mathrm{E}\left(  Z_{k}\right)  =0$ than when $\lambda=0.$
Broadly speaking, under $H_{0}^{\lambda},$ $S_{T,\Delta}%
\overset{d}{\rightarrow}N\left(  0,\omega_{0}\right)  ,$ while under
$H_{A}^{\lambda(2)},$ $\Delta S_{T,\Delta}\overset{d}{\rightarrow}N\left(
0,\omega_{1}\right)  ,$ with $\omega_{1}\neq\omega_{0}$. This is what allows
one to distinguish between $H_{0}^{\lambda}$ and $H_{A}^{\lambda(2)}.$

\noindent\textbf{Remark 2: }The test has power not only against constant
intensity and self-exciting intensity, as in (\ref{lambda}), but also against
affine jump diffusions where intensity is an affine function of volatility,
for example.

\noindent\textbf{Remark 3: }As the variance of the statistic is of larger
order under the alternative of positive jump intensity, we cannot construct a
variance estimator which is consistent under all hypotheses. Thus, our aim is
to construct an estimator for the variance of $S_{T,\Delta}$ which is
consistent under the null and bounded in probability under the (union of)
alternatives. This is done by using a threshold variance estimator, which
filters out the contribution of the jump component. In particular, define:%
\begin{align}
&  \widehat{\sigma}_{\lambda,T,\Delta}^{2}\nonumber\\
&  =\frac{1}{T\Delta^{2}}\sum_{k=1}^{n}\left(  \ln X_{k\Delta}-\ln
X_{(k-1)\Delta}-\frac{\ln X_{n\Delta}-\ln X_{\Delta}}{n}\right)  ^{6}1\left\{
\left\vert \ln X_{k\Delta}-\ln X_{(k-1)\Delta}\right\vert \leq\tau\left(
\Delta\right)  \right\}  . \label{jsig2}%
\end{align}
It follows that the t-statistic version of the jump test is,
\[
t_{\lambda,T,\Delta}=\frac{S_{T,\Delta}}{\widehat{\sigma}_{\lambda,T,\Delta}%
}.
\]
The following corollary summarizes the limiting behavior of $t_{\lambda
,T,\Delta}.$

\noindent\textbf{Corollary 2: }\textit{Let Assumption\textbf{ A} hold. Also,
assume that as }$T\rightarrow\infty,$ $\Delta\rightarrow0,$ $T\Delta
\rightarrow\infty,$ $\sqrt{T}\Delta\rightarrow0,$ $T^{+}/T\rightarrow\infty,$
$\frac{\Delta^{\frac{1}{2}-\frac{3}{m}}}{\tau\left(  \Delta\right)
}\rightarrow0$ \textit{for even} $m>6,$ \textit{and} $\tau^{7}\left(
\Delta\right)  \Delta^{-2}\rightarrow0.$ \textit{Then,}

\noindent\textit{(i) Under }$H_{0}^{\lambda}:$%
\[
t_{\lambda,T,\Delta}\overset{d}{\rightarrow}N\left(  0,1\right)  .
\]
\noindent\textit{(ii) Under }$H_{A}^{\lambda(1)},$\textit{ there exists an
}$\varepsilon>0,$\textit{ such that:}%
\[
\lim_{T\rightarrow\infty,\Delta\rightarrow0}\Pr\left(  \frac{\Delta}{\sqrt{T}%
}\left\vert t_{\lambda,T,\Delta}\right\vert >\varepsilon\right)  =1.
\]
\noindent\textit{(iii) Under }$H_{A}^{\lambda(2)}$\textit{ there exists an
}$\varepsilon>0,$\textit{ such that:}%
\[
\lim_{T\rightarrow\infty,\Delta\rightarrow0}\Pr\left(  \Delta\left\vert
t_{\lambda,T,\Delta}\right\vert >\varepsilon\right)  =1.
\]


\noindent\textbf{Remark 4: }Note that the variance estimator can be
constructed using the entire time span of $T^{+}$ observations, and the
statement in Corollary 2 still holds, provided that we replace $\sqrt{T}%
\Delta\rightarrow0$ and $\sqrt{T}\tau^{2}\left(  \Delta\right)  \rightarrow0$
with $\sqrt{T^{+}}\Delta\rightarrow0$ and $\sqrt{T^{+}}\tau^{2}\left(
\Delta\right)  \rightarrow0.$ In general, the price of having a statistic
which allows for possible leverage effects is that we need to use a longer
time span for estimating the leverage contribution. A possible alternative
approach would be to pretest for $H_{0}^{\rho}:\rho=0$ vs. $H_{A}^{\rho}%
:\rho\neq0.$ Here, under $H_{0}^{\rho},$
\[
\frac{\Delta}{\sqrt{T^{+}}}\sum_{k=1}^{n^{+}}\left(  \ln X_{k\Delta}-\ln
X_{(k-1)\Delta}-\frac{\ln X_{n^{+}\Delta}-\ln X_{\Delta}}{n^{+}}\right)
^{3}1\left\{  \left\vert \ln X_{k\Delta}-\ln X_{(k-1)\Delta}\right\vert
\leq\tau\left(  \Delta\right)  \right\}
\]
is asymptotically normal, while it diverges to (likely minus) infinity under
$H_{A}^{\rho}.$ This means that if we do not reject the null of no leverage,
we do not need to recenter the statistic in (\ref{mu3}) and we can use all
$T^{+}$ observations for tricity estimation.

\noindent\textbf{Remark 5: }Consider selection of the threshold sequence. From
Corollary 2, it follows that $\tau\left(  \Delta\right)  $ should approach
zero faster than $T^{-1/4}$ and faster than $\Delta^{2/7},$ but slower than
$\Delta^{\frac{1}{2}-\frac{3}{m}},$ where $m$ denotes the number of finite
moments of $V_{t}^{1/2}$ , with $m>6$ by Assumption A(iii). For example, if
$V_{t}$ follows a square root process, so that all finite moments exist, we
can set $\tau\left(  \Delta\right)  =c\Delta^{\eta},$ with $\frac{2}{7}%
<\eta<\frac{1}{2}$ and $\Delta=T^{-\delta},$ where $\frac{7}{8}<\delta<1.$
These conditions ensure that $T\Delta\rightarrow\infty,$ $\sqrt{T}%
\Delta\rightarrow0,$ and $\sqrt{T}\tau^{2}\left(  \Delta\right)
\rightarrow0.$ Additionally, in order to implement the statistic, we choose
the constant $c$ in a data driven manner. A natural solution is to use
$\widehat{\sigma}_{\mu_{Z}}$ as defined in (\ref{sigmaz}) below.

\noindent\textbf{Remark 6: }Since the suggested statistic is not robust to the
presence of microstructure noise, the optimal discrete interval, $\Delta,$ is
the highest frequency at which microstructure noise doesn't bind. Visual
inspection of the signature plots of Andersen, Bollerslev and Diebold (2000)
provides a useful tool for the choice of interval. It should also be noted
that the statistic is constructed over an increasing time span; and hence it
is not straightforward to ascertain whether simple pre-averaging will yield a
statistic that is robust to microstructure noise (as in the case of the
realized pre-average power variation discussed in Podolskji and Vetter
(2009b)). In our Monte Carlo experiment, we implement values of $\Delta$ for
which noise is either not binding or moderately binding. Careful exploration
of this issue is left to future research.

\noindent\textbf{Remark 7: }In this paper, we only derive tests for the null
of zero jump intensity in asset returns. However, the same approach can be
used for testing equivalent hypotheses for volatility. Such tests would
require estimators of the spot volatility, say $V_{k\Delta}^{2},$ which can be
constructed using a finer grid of observations than that used in the above
tests, such as if there were $M$ observations over each interval of order
$\Delta.$ The order of magnitude of the error due to the estimation of the
spot volatility is derived in Bandi and Ren\`{o} (2012), under various settings.

\section{Testing for Self-Exciting Jumps}

If the null hypothesis of zero intensity is rejected, one can proceed to test
the null of no self-excitation or path dependence. If the drift in (\ref{X})
and the intensity are constant, and if the jump size is independently
distributed, then $\left(  \ln X_{(k+1)\Delta}-\ln X_{k\Delta}-\frac{\ln
X_{n\Delta}-\ln X_{\Delta}}{n}\right)  $ is a martingale difference process
and so it is uncorrelated. In fact, if $\lambda_{t}$ is generated as in
(\ref{lambda}), then it follows from A\"{\i}t-Sahalia, Cacho-Diaz and Laeven
(2015), that
\begin{align}
&  \mathrm{E}\left(  \left(  \ln X_{(k+\tau)\Delta}-\ln X_{(k+\tau-1)\Delta
}-\frac{\ln X_{n\Delta}-\ln X_{\Delta}}{n}\right)  \left(  X_{k\Delta}-\ln
X_{(k-1)\Delta}-\frac{\ln X_{n\Delta}-\ln X_{\Delta}}{n}\right)  \right)
\nonumber\\
&  =\frac{\beta\lambda_{\infty}\left(  2a-\beta\right)  }{2\left(
a-\beta\right)  }\exp\left(  -\left(  a-\beta\right)  \tau\right)  \left(
\mathrm{E}\left(  Z\right)  \right)  ^{2}\Delta^{2}+o\left(  \Delta
^{2}\right)  . \label{cross}%
\end{align}
In this case, it is natural to test null hypothesis of $\beta=0,$ against the
alternative that $\beta>0$, in order to test for path dependence$.$ In turn,
it is immediate to see that in order to identify $\beta,$ we require not only
$\lambda_{\infty}>0,$ but also $\mathrm{E}\left(  Z_{k}\right)  \neq0.$ In
fact, failure to reject the null may be simply due to the fact that
$\mathrm{E}\left(  Z_{k}\right)  =0.$ Hence, before testing for jump
self-excitation, it remains to pretest for the null of $\mathrm{E}\left(
Z_{k}\right)  =0$ versus $\mathrm{E}\left(  Z_{k}\right)  \neq0.$

\subsection{Test of $\left(  Z_{k}\right)  =0$ (Zero Mean Jump Test)}

We test the null of zero mean jumps, against its negation. The hypotheses of
interest are:%
\[
H_{0}^{\mu_{Z}}:\mathrm{E}\left(  Z_{k}\right)  =0\text{ }%
\]
and%
\[
H_{A}^{\mu_{Z}}:\mathrm{E}\left(  Z_{k}\right)  \neq0.
\]
Let%
\begin{align}
\widehat{\mu}_{T,\Delta}^{Z}  &  =\frac{1}{T}\sum_{k=0}^{n}\left(  \ln
X_{k\Delta}-\ln X_{(k-1)\Delta}\right) \nonumber\\
&  -\frac{1}{T^{+}}\sum_{k=1}^{n^{+}}\left(  \ln X_{k\Delta}-\ln
X_{(k-1)\Delta}\right)  1\left\{  \left\vert \ln X_{k\Delta}-\ln
X_{(k-1)\Delta}\right\vert \leq\tau\left(  \Delta\right)  \right\}
\label{muz}%
\end{align}
and
\begin{equation}
\widehat{\sigma}_{\mu_{Z}}^{2}=\frac{1}{T}\sum_{k=0}^{n}\left(  \ln
X_{k\Delta}-\ln X_{(k-1)\Delta}-\frac{\ln X_{n\Delta}-\ln X_{\Delta}}%
{n}\right)  ^{2} \label{sigmaz}%
\end{equation}
and define
\begin{equation}
t_{\mu_{Z},T,\Delta}=\sqrt{T}\frac{\widehat{\mu}_{T,\Delta}^{Z}}%
{\widehat{\sigma}_{\mu_{Z}}}. \label{tzero}%
\end{equation}
Since \textrm{E}$\left(  \frac{1}{T}\sum_{k=0}^{n}\left(  \ln X_{k\Delta}-\ln
X_{(k-1)\Delta}\right)  \right)  =\mu+\lambda\mathrm{E}\left(  Z_{k}\right)
+O\left(  \Delta\right)  ,$ from the first term on the RHS of (\ref{muz}), we
cannot disentangle the contribution to the mean of the continuous and jump
components. However, thanks to the thresholding, the second term on the RHS of
(\ref{muz}) provides a $\sqrt{T^{+}}-$consistent estimator of the drift,
$\mu;$ and as $T^{+}/T\rightarrow\infty,$ estimation error is negligible. As
in the case of Theorem 1, if $T=T^{+},$ the statistic is degenerate under the
null. Note also that the first term on the RHS of (\ref{muz}) is not
recentered using $\frac{\ln X_{n\Delta}-\ln X_{\Delta}}{n}.$ This is because
otherwise we would recenter both the continuous as well as the jump
components, and the statistic would not have any power against non-zero jump mean.

\noindent\textbf{Theorem 3: }\textit{Let Assumption\textbf{ A} hold. Also,
assume that as }$T\rightarrow\infty,$ $\Delta\rightarrow0,$ $T\Delta
\rightarrow\infty,$ $\sqrt{T}\Delta\rightarrow0,$ $T^{+}/T\rightarrow\infty,$
$\frac{\Delta^{\frac{1}{2}-\frac{3}{m}}}{\tau\left(  \Delta\right)
}\rightarrow0$ \textit{for even} $m>6,$ \textit{and} $\sqrt{T}\tau^{2}\left(
\Delta\right)  \rightarrow0.$\textit{\footnote{Note that $\sqrt{T}%
\Delta\rightarrow0$ is implied by $\frac{\Delta^{\frac{1}{2}-\frac{3}{m}}%
}{\tau\left(  \Delta\right)  }\rightarrow0$ for even $m>6,$ and $\sqrt{T}%
\tau^{2}\left(  \Delta\right)  \rightarrow0,$ so that the conditions of the
theorem can be restated without the requirement that $\sqrt{T}\Delta
\rightarrow0$.}} \textit{Then,}

\noindent\textit{(i) Under }$H_{0}^{\mu_{Z}}:$%
\[
t_{\mu_{Z},T,\Delta}\overset{d}{\rightarrow}N\left(  0,1\right)  .
\]
\noindent\textit{(ii) Under }$H_{A}^{\mu_{Z}},$ \textit{there exists an
}$\varepsilon>0$\textit{ such that: }%
\[
\lim_{T\rightarrow\infty,\Delta\rightarrow0}\Pr\left(  \left\vert \frac
{t_{\mu_{Z},T,\Delta}}{\sqrt{T}}\right\vert >\varepsilon\right)  =1,
\]
where $\hat{\mu}_{T,\Delta}^{Z},\hat{\sigma}_{\mu_{Z}}$, and $t_{\mu
_{Z},T,\Delta}$ are defined as in (\ref{muz}), (\ref{sigmaz}), and
(\ref{tzero}), respectively.

\noindent\textbf{Remark 1: }Note that $\widehat{\sigma}_{\mu_{Z}}^{2}$ in
(\ref{sigmaz}) can be constructed using the entire time span, $T^{+}$,
provided that we replace $\sqrt{T}\Delta\rightarrow0$ and $\sqrt{T}\tau
^{2}\left(  \Delta\right)  \rightarrow0$ with $\sqrt{T^{+}}\Delta\rightarrow0$
and $\sqrt{T^{+}}\tau^{2}\left(  \Delta\right)  \rightarrow0.$

\subsection{Test of $\beta=0$ (Self Excitement Test)}

Note that for this test, we can use the entire time span, $T^{+},$ as leverage
plays no role in autocorrelation calculations. Our objective is to test the
following hypotheses:%
\[
H_{0}^{\beta}:\beta=0\text{ }%
\]
and%
\[
H_{A}^{\beta}:\beta>0,
\]
under the maintained assumption that $\mathrm{E}\left(  Z_{k}\right)  \neq0.$
Define the statistic:%
\[
S_{T^{+},\Delta}^{\beta}=\max\left\{  0,t_{\beta,T^{+},\Delta}\right\}  ,
\]
where%
\begin{equation}
t_{\beta,T^{+},\Delta}=\frac{\sqrt{\frac{T^{+}}{\Delta}}\widehat{\beta}%
_{T^{+},\Delta}}{\widehat{\sigma}_{\beta,T^{+},\Delta}}, \label{t-beta}%
\end{equation}
with%
\begin{equation}
\widehat{\beta}_{T^{+},\Delta}=\frac{1}{T^{+}}\sum_{k=2}^{n^{+}-1}\left(  \ln
X_{(k+1)\Delta}-\ln X_{k\Delta}-\frac{\ln X_{n^{+}\Delta}-\ln X_{\Delta}%
}{n^{+}}\right)  \left(  \ln X_{k\Delta}-\ln X_{(k-1)\Delta}-\frac{\ln
X_{n^{+}\Delta}-\ln X_{\Delta}}{n^{+}}\right)  \label{beta}%
\end{equation}
and%
\begin{align}
&  \widehat{\sigma}_{\beta,T^{+},\Delta}^{2}\nonumber\\
&  =\frac{1}{T^{+}\Delta}\sum_{k=2}^{n^{+}-1}\left(  \ln X_{(k+1)\Delta}-\ln
X_{k\Delta}-\frac{\ln X_{n^{+}\Delta}-\ln X_{\Delta}}{n^{+}}\right)
^{2}\left(  \ln X_{(k+1)\Delta}-\ln X_{k\Delta}-\frac{\ln X_{n^{+}\Delta}-\ln
X_{\Delta}}{n^{+}}\right)  ^{2}. \label{sig2b}%
\end{align}
From (\ref{cross}), and recalling that $a>0,$ $\beta\geq0,$ and $a>\beta,$ it
follows immediately that the autocorrelation can never be negative. This is
why the test is one-sided. Additionally, recall that $T\Delta\rightarrow
\infty$ implies $T^{+}\Delta\rightarrow\infty.$\medskip\ The following result
thus holds.

\noindent\textbf{Theorem 4: }\textit{Let Assumption\textbf{ A} hold. Also,
assume that }$E\left(  Z\right)  \neq0,$\textit{ }$\lambda_{\infty}>0,$
\textit{and as }$n\rightarrow\infty,$\textit{ }$T\rightarrow\infty,$\textit{
}$\Delta\rightarrow0$ \textit{and} $T\Delta\rightarrow\infty.$ \textit{Then,}

\noindent\textit{(i) Under }$H_{0}:$%
\[
S_{T^{+},\Delta}^{\beta}\overset{d}{\rightarrow}\max\left\{  0,\mathcal{Z}%
\right\}  ,
\]
\textit{where }$\mathcal{Z}$\textit{ is a standard normal random variable.}

\noindent\textit{(ii) Under }$H_{A},$\textit{ there exists an }$\varepsilon
>0$\textit{ such that: }%
\[
\lim_{T^{+}\rightarrow\infty,\Delta\rightarrow0}\Pr\left(  \frac{1}%
{\sqrt{T^{+}\Delta}}S_{T^{+},\Delta}^{\beta}>\varepsilon\right)  =1.
\]


\noindent It follows that $S_{T^{+},\Delta}^{\beta}$ converges to an
half-normal random variable under the null, and diverges at rate $\sqrt
{T^{+}\Delta}$ under the alternative.

\noindent\textbf{Remark 1: }The test statistic is only a function of the first
autocovariance term. It follows immediately that one can construct a test
based on an increasing number of autocovariance terms, with the number of
terms chosen adaptively (see, e.g. Escanciano and Lobato (2009)).

\noindent\textbf{Remark 2: }If the nulls of zero intensity, zero jump mean and
no self-excitation are all rejected, then one can proceed to estimate the full
Hawkes diffusion using GMM, as in A\"{\i}t-Sahalia, Cacho-Diaz and Laeven (2015).

\noindent\textbf{Remark 3: }In this section, we consider self-exciting
intensity. However, from an empirical point of view, an interesting case is
that of financial contagion, where the contagion is due to \textquotedblleft
common\textquotedblright\ jumps. In this case, the jump intensity is an
increasing function not only of its own past jumps but also of past jumps in
other assets. In order to test for (no) cross-excitation, it suffices to
construct a statistic based on cross correlations instead of autocorrelations
(see Theorem 4 in A\"{\i}t-Sahalia, Cacho-Diaz and Laeven (2015)). For
example, let:
\begin{align*}
&  \widehat{\beta}_{T^{+},\Delta}^{(I,II)}\\
&  =\frac{1}{T^{+}}\sum_{k=2}^{n^{+}-1}\left(  \ln X_{(k+1)\Delta}^{(I)}-\ln
X_{k\Delta}^{(I)}-\frac{\ln X_{n^{+}\Delta}^{(I)}-\ln X_{\Delta}^{(I)}}{n^{+}%
}\right)  \left(  \ln X_{k\Delta}^{(II)}-\ln X_{(k-1)\Delta}^{(II)}-\frac{\ln
X_{n^{+}\Delta}^{(II)}-\ln X_{\Delta}^{(II)}}{n^{+}}\right)  ,
\end{align*}
and note that if the jump intensity in asset $II$ does not depend on past
jumps in asset $I$, then $\widehat{\beta}_{T,\Delta}^{(I,II)}%
\overset{p}{\rightarrow}0.$ On the other hand, if the intensity in asset $II$
increases when there is a jump in asset $I$, then$\left\vert \widehat{\beta
}_{T^{+},\Delta}^{(I,II)}\right\vert $ has a strictly positive probability limit.

\section{Monte Carlo Experiment}

In this section we present the findings of a Monte Carlo experiment designed
to evaluate the finite sample properties of: (i) the \textquotedblleft jump
test\textquotedblright\ for the null of zero jump intensity, based on
$t_{\lambda,T,\Delta}=\frac{S_{T,\Delta}}{\widehat{\sigma}_{\lambda,T,\Delta}%
},$ where $S_{T,\Delta}=\frac{T^{1/2}}{\Delta}\widehat{\mu}_{3,T,\Delta},$
$\widehat{\mu}_{3,T,\Delta}$ is defined in (\ref{mu3}), and $\widehat{\sigma
}_{\lambda,T,\Delta}^{2}$ is defined in (\ref{jsig2}); (ii) the
\textquotedblleft zero mean jump test\textquotedblright\ for the null of zero
mean jumps based on $t_{\mu_{Z},T,\Delta}=\sqrt{T}\frac{\widehat{\mu
}_{T,\Delta}^{Z}}{\widehat{\sigma}_{\mu_{Z}}}$, where $\widehat{\mu}%
_{T,\Delta}^{Z}$ is defined in (\ref{muz}) and $\widehat{\sigma}_{\mu_{Z}}%
^{2}$ is defined in (\ref{sigmaz}); and the \textquotedblleft self excitement
test\textquotedblright\ for the null of\ no jump-path dependence, based on
$S_{T^{+},\Delta}^{\beta}=\max\left\{  0,t_{\beta,T^{+},\Delta}\right\}  ,$
where $t_{\beta,T^{+},\Delta}=$ $\frac{\sqrt{\frac{T^{+}}{\Delta}%
}\widehat{\beta}_{T^{+},\Delta}}{\widehat{\sigma}_{\beta,T^{+},\Delta}},$
$\widehat{\beta}_{T^{+},\Delta}$ is defined in (\ref{beta}), and
$\widehat{\sigma}_{\beta,T^{+},\Delta}^{2}$ is defined in (\ref{sig2b}).

Data used in the experiment are generated according to the following data
generating processes (DGPs):%
\[
\mathrm{d\ln}X_{t}=\mu\mathrm{d}t+\sqrt{V_{t}}\mathrm{d}W_{1,t}+Z_{t}%
\mathrm{d}N_{t},
\]
where volatility is modeled as a square-root process:%
\[
\mathrm{d}V_{t}=\kappa_{v}(\theta_{v}-V_{t})\mathrm{d}t+\zeta\sqrt{V_{t}%
}\mathrm{d}W_{2,t},
\]
with \textrm{E}$\left(  W_{1,t}W_{2,t}\right)  =\rho.$ We set $\mu=0.1$,
$\rho=\{0,-0.25\}$, $\kappa_{v}=5,$ $\theta_{v}=0.16,$ and $\zeta=0.5.$
Additionally, $N_{t}$ satisfies the conditions in (\ref{J1})-(\ref{J3}). The
jump size, $Z_{k},$ is identically and independently distributed with density
$f(z;\gamma).$ We consider three jump densities: $f(z;\gamma)=$ $N(0.0,\sigma
)$, $f(z;\gamma)=$ $N(0.5,\sigma),$ and $f(z;\gamma)=\varsigma e^{-\varsigma
z}.$ For the cases where $Z_{k}$ is a normal random variable, $\sigma
=\{0.1,0.2,0.3,0.4,0.7\}$; and for the case where $Z_{t}$ is characterized by
the exponential density, $\varsigma=5.$ The jump intensity evolves according
to:%
\begin{equation}
\lambda_{t}=\lambda_{\infty}+\beta\int_{0}^{t}\exp\left(  -a\left(
t-s\right)  \right)  \mathrm{d}N_{s}, \label{lambdaB}%
\end{equation}
where $\lambda_{\infty}=\{0.3,0.5,0.7,0.9\}$ and $(a,\beta)=\{(0,0),(3,2),$
$(5,4),$ $(7,5)\}.$ Note that the case where $(a,\beta)=(0,0)$ is consistent
with both the case of no jumps (i.e., $\lambda_{\infty}=0)$ and with the case
of constant jump intensity (i.e., $\lambda_{t}=\lambda_{\infty}>0$, for all
$t).$ In the constant jump intensity case, we consider Poisson jumps, with
parameter $\lambda_{\infty}.$

We simulate observations using a Milstein discretization scheme, with discrete
interval $h=1/312$, and consider two intra-daily sampling frequencies:
$\Delta=1/78$ and $\Delta=1/156$. In an empirical context, these values are
consistent with 5-minute and 2.5-minute sampling frequencies, assuming a 6.5
hour trading day (see e.g., A\"{\i}t-Sahalia and Jacod (2009)). Loosely
speaking, we view our values of $\Delta$ as associated with noise which is
either not binding or moderately binding. Recall also that for the
$t_{\lambda,T,\Delta}$ and $t_{\mu_{Z},T,\Delta}$ tests, a key assumption is
that $T\Delta\rightarrow\infty$ and $\sqrt{T}\Delta\rightarrow0$, leading to a
restriction that $1/\Delta<T<1/\Delta^{2}.$ In particular, when $\Delta=1/78$
we set $T=\{60,70,80,90,100,110,120,130\}$ and when $\Delta=1/156$ we set
$T=\{160,180,200,220,240,260,280,300\}.$ Notice that all values of $T$ satisfy
the condition, with the exception of $T=60$ and $T=70.$ These sample sizes are
included in order to provide some evidence on the performance of the tests
when the condition is broken. In all experiments, we perform $1000$ Monte
Carlo replications.

When implementing the test for no jumps $t_{\lambda,T,\Delta},$ in order to
disentangle the contribution of jumps from that of leverage, we also need to
select the thresholding sequence $\tau(\Delta)$ and the \textquotedblleft
longer\textquotedblright\ time span $T^{+}.$ We set $T^{+}=10T,$ to satisfy
the requirement that $T^{+}/T\rightarrow\infty,$ which ensures that the
contribution of leverage estimation error is negligible. We then set
$\tau\left(  \Delta\right)  =c\Delta^{\eta},$ with $\frac{2}{7}<\eta<\frac
{1}{2},$ and given that for most choices of $T,$ $\Delta=T^{-\delta},$
$\frac{7}{8}<\delta<1,$ we have $\tau\left(  \Delta\right)  =cT^{-\delta\eta
},$ and for $\delta\eta\in(\frac{1}{4},\frac{1}{2}),$ the rate conditions
$\frac{\Delta^{\frac{1}{2}}}{\tau\left(  \Delta\right)  }\rightarrow0$,
$\sqrt{\Delta}\tau\left(  \Delta\right)  ^{-1}$ and $\sqrt{T}\tau^{2}\left(
\Delta\right)  \rightarrow0$ are satisfied.\footnote{Note tht the condition
$\sqrt{T}\tau(\Delta)\rightarrow0$ s required only for Theorem 3.} Finally, we
set $c=\widehat{\sigma}_{\mu_{Z}}$, where $\widehat{\sigma}_{\mu_{Z}}^{2}$ is
defined in (\ref{sigthresh}). The choice of $\delta\eta$ is also important in
finite sample applications. Consider the case where there are no jumps, so
that $\widehat{\sigma}_{\lambda,T,\Delta}^{2}$, which is used in the
construction of $t_{\lambda,T,\Delta},$ is a consistent variance estimator.
Recalling (\ref{mu3})-(\ref{jsig2}), it is immediate to see that $\tau\left(
\Delta\right)  $ plays a role both in the numerator and the denominator of
$t_{\lambda,T,\Delta}$. In our experimental setup, small thresholds (e.g.
$\delta\eta=0.4)$ result in a too small variance estimator, leading to an
oversized test. Not surprisingly, the empirical power is not affected by the
choice of the threshold parameter. Below, we only report results for the case
where $\delta\eta=0.251.$

The findings from our Monte Carlo experiment are reported in Tables 1-5. In
these tables, rejection frequencies based on tests implemented at a 10\%
nominal level are reported for various values of $\lambda_{\infty}$,
$(a,\beta),\sigma$, and $\Delta$.

Tables 1 and 2 contain results for the jump test. Empirical rejection
frequencies under $H_{0}^{\lambda}:\lambda_{\infty}=0$ are given in the first
4 rows of entries in Table 1. When $\Delta=1/78$, rejection frequencies are
very near nominal levels only for $T$ $=80$ and $T=90$. Indeed, when $T$ is
increased to $130$, the empirical size deteriorates substantively and is over
$20\%$. However, rejection frequencies are close to the nominal 10\% level for
$T=240,260,280,$ and $300$ when $\Delta^{-1}=1/156.$ This finding is quite
interesting, and it suggests that the range of permissible $T$ and $\Delta$
permutations for which the size properties of our jump test are adequate is
wide. Of course, for extremely large values of $T$, performance should be
again expected to deteriorate, since we require that $T<1/\Delta^{2}.$ Also,
the empirical size is not affected by the presence of leverage.

Empirical rejection frequencies under $H_{A}^{\lambda(1)}:\lambda_{\infty}>0$
and \textrm{E}$\left(  Z_{k}^{3}\right)  \neq0$ are given in the last 16 rows
of entries in Table 1. Recall, that in this case all jump densities have
non-zero third moment, and the test has well defined Pitman drift against
$\sqrt{T}-$ alternatives. Not surprisingly, rejection frequencies are thus
near unit, regardless of jump density specification.

The more challenging alternative is $H_{A}^{\lambda(2)}:\lambda_{\infty}>0$
and \textrm{E}$\left(  Z_{k}^{3}\right)  =0.$ In this case, the test has zero
Pitman drift, and that the ability of the test to distinguish between
$H_{0}^{\lambda}$ and $H_{A}^{\lambda(2)}$ derives solely from the different
order of magnitude of the variance under the two hypotheses. Results for DGPs
generated under $H_{A}^{\lambda(2)}$ are gathered in Table 2, for
$\sigma=0.1,0.2,0.4,$ and $\lambda_{\infty}=0.3,0.5,0.7$. As might be
expected, the power increases as $\sigma$ and $\lambda_{\infty}$ increase.
However, the value of $\sigma$ plays a much bigger role than that of
$\lambda_{\infty}.$ This is not surprising, given that what drives the power
is the order of magnitude of the variance.

Table 3 summarizes experimental findings for the zero mean jump test, based on
$t_{\mu_{Z},T,\Delta}=\sqrt{T}\frac{\widehat{\mu}_{T,\Delta}^{Z}%
}{\widehat{\sigma}_{\mu_{Z}}}.$ This test can be thought of as a pre-test,
prior to testing for self excitement, and after testing for jumps, say. The
reason for this is that in order to identify $\beta$ when testing
$H_{0}^{\beta}:\beta=0$ vs. $H_{A}^{\beta}:\beta>0$, we require not only that
$\lambda_{\infty}>0$ but also that $E(Z_{k})\neq0.$ Inspection of the
rejection frequencies reported in the table indicates that the test is well
sized, for all values of $T$, when $\lambda_{\infty}=0.3$ and $\sigma=0.2.$
Similar results obtain for other values of $\lambda_{\infty}$ and $\sigma,$
and are hence not reported. Overall, the power is quite good, except for the
cases in which $T$ is small, $\lambda_{\infty}$ is small, $\Delta^{-1}=78,$
and there is no self-excitation.

Finally, Tables 4 (empirical size) and 5 (empirical power) summarize
experimental findings for our self excitement test, based on $S_{T^{+},\Delta
}^{\beta}=\max\left\{  0,t_{\beta,T^{+},\Delta}\right\}  ,$ where
$t_{\beta,T^{+},\Delta}=$ $\frac{\sqrt{\frac{T^{+}}{\Delta}}\widehat{\beta
}_{T^{+},\Delta}}{\widehat{\sigma}_{\beta,T^{+},\Delta}}$. Although $192$
($\lambda_{\infty}$,$\sigma,\Delta^{-1},T)$ permutations are reported in Table
4, even cursory examination of the table indicates that the test is very well
sized, with rejection frequencies very close to the nominal 10\% level, in all cases.

It remains only to examine the performance of the self excitement test, under
$H_{A}^{\beta}:\beta>0$. The findings are reported in Table 5, where empirical
power is summarized for various values of $\lambda_{\infty}$,$(a,\beta
),\Delta^{-1}$, and $T$.\footnote{As the test is based on the first
autocovariance term, leverage plays no role in the test statistic. Not
surprisingly, then, including leverage (or not)\ has no qualitatively
noteworthy impact on our findings, and only results for the non-zero leverage
case are reported.} For the case where $Z_{k}$ is a normal random variable, we
report rejection frequencies for only one value of $\sigma$ (i.e.,
$\sigma=0.2)$. However, it should be noted that empirical power generally
declines as $\sigma$ increases from $0.1$ to $0.7$, for fixed values of the
other parameters. One possible explanation for this finding is that
\textquotedblleft noisiness\textquotedblright\ is induced when estimating the
first autocovariance term, when jumps are extremely large.

Empirical power also declines when the value of $a$ is increased, with
$(a-\beta)$ fixed (compare rejection frequencies for $(a,\beta)=(3,2)$ with
those for $(a,\beta)=(5,4)$). The reason for this follows from (\ref{cross}%
)-(\ref{lambdaB}), where it is immediate to see that the smaller is $a$ and
the smaller is $(a-\beta),$ the higher is the degree of self-excitation. This
means that our lowest degree of self-excitation is associated with the case
where $(a,\beta)=(7,5)$. Empirical power is correspondingly the lowest in this
case, bottoming out at around $0.30$. When the degree of self-excitation is
strongest (i.e., $(a,\beta)=(3,2))$ and there are \textquotedblleft
enough\textquotedblright\ jumps (i.e., $\lambda_{\infty}=0.7)$, rejection
frequencies are (roughly)\ in the range $0.70$ to $0.80$.

In summary, all of our tests perform as expected, given the asymptotic theory
describing their large sample behavior. Moreover, the finite sample
performance of the tests is found to be good, in all cases, with the important
obvious caveat that jump process characteristics and sampling \ frequencies
affect the ability of the tests to perform adequately.

\section{Concluding Remarks}

If the intensity parameter in a jump diffusion model is identically zero, then
parameters characterizing the jump size density cannot be identified. In
general, this lack of identification precludes consistent estimation of
identified parameters. In the extant literature, there are a large variety of
tests for the null of no jumps versus the alternative of jumps, including
tests based on the comparison of two realized volatility measures, one which
is robust, and the other which is not robust to the presence of jumps (see,
e.g. Barndorff-Nielsen, Shephard and Winkel (2006) and Podolskji and Vetter
(2009a)), tests based on a thresholding approach (see, e.g. Corsi, Pirino, and
Ren\`{o} (2010), Lee and Mykland (2008), and Lee, Loretan and Ploberger
(2013)), and tests based on power variation (see e.g. A\"{\i}t-Sahalia and
Jacod (2009)). One feature of these tests is that they are based on
observations drawn on a given finite time span, and thus they can only detect
realized jumps. This paper introduces a test which is instead able to detect
jumps in the data generating process. Our test is based on realized tricity
and make use of high frequency observations measured over a long time-span.
Importantly, the test is robust to the presence of leverage. It has a normal
limiting distribution, and so inference is straightforward. A so-called
\textquotedblleft self-excitement\textquotedblright\ test is also introduced,
which is designed to have power against path dependent intensity, thus
providing a direct test for the Hawkes diffusion model of A\"{\i}t-Sahalia,
Cacho-Diaz and Laeven (2015). The finite sample behavior of the suggested
statistics is studied via Monte Carlo experimentation, and is found to be
adequate under a variety of realistic data generating processes.\pagebreak

\section{Appendix}

\noindent\textbf{Lemma 1: }Let Assumptions \textbf{A(i)-(iii)} hold. Also, as
$T\rightarrow\infty,$ $T\Delta^{2}\rightarrow0,$ $\tau\left(  \Delta\right)
\rightarrow0$ and $\frac{\Delta^{\frac{1}{2}-\frac{3}{m}}}{\tau\left(
\Delta\right)  }\rightarrow0,$ with $m>6$ and even. Then if $\lambda=0,$%
\[
P\left(  \max_{k\leq n}\left\vert \ln X_{k\Delta}-\ln X_{(k-1)\Delta
}\right\vert >\varepsilon\tau\left(  \Delta\right)  \right)  \rightarrow0.
\]


\noindent\textbf{Proof of Lemma 1: }Recalling that $T\Delta^{2}\rightarrow0,$%
\begin{align*}
&  P\left(  \max_{k\leq n}\left\vert \ln X_{k\Delta}-\ln X_{(k-1)\Delta
}\right\vert >\varepsilon\tau\left(  \Delta\right)  \right) \\
&  \leq\sum_{k=1}^{n}P\left(  \left\vert \ln X_{k\Delta}-\ln X_{(k-1)\Delta
}\right\vert >\varepsilon\tau\left(  \Delta\right)  \right) \\
&  =\sum_{k=1}^{n}P\left(  \left\vert \mu\Delta+V_{k\Delta}^{1/2}\sqrt
{1-\rho^{2}}\left(  W_{1,\left(  k+1\right)  \Delta}-W_{1,k\Delta}\right)
+V_{k\Delta}^{1/2}\rho\left(  W_{2,\left(  k+1\right)  \Delta}-W_{2,k\Delta
}\right)  \right\vert >\varepsilon\tau\left(  \Delta\right)  \right) \\
&  \leq\frac{T}{\Delta}\frac{1}{\varepsilon^{m}\tau\left(  \Delta\right)
^{m}}\mathrm{E}\left(  \left\vert \mu\Delta+V_{k\Delta}^{1/2}\sqrt{1-\rho^{2}%
}\left(  W_{1,\left(  k+1\right)  \Delta}-W_{1,k\Delta}\right)  +V_{k\Delta
}^{1/2}\rho\left(  W_{2,\left(  k+1\right)  \Delta}-W_{2,k\Delta}\right)
\right\vert ^{m}\right) \\
&  \leq C\mathrm{E}\left(  V_{k\Delta}^{m/2}\right)  \Delta^{\frac{m}{2}%
-3}\tau\left(  \Delta\right)  ^{-m}\rightarrow0\text{,}%
\end{align*}
for $\frac{\Delta^{\frac{1}{2}-\frac{3}{m}}}{\tau\left(  \Delta\right)
}\rightarrow0$ and $m>6,$ as $\mathrm{E}\left(  V_{k\Delta}^{m/2}\right)  $ is
finite, by A(iii).\medskip

\noindent\textbf{Lemma 2: }Let Assumptions \textbf{A(iv)-(v)} hold. Then, for
all $l\geq2$ even,%
\[
\mathrm{E}\left(  \frac{1}{T}\sum_{k=1}^{n}\left(  Z_{k}1_{\Delta
_{N_{(k+1)\Delta}}}-\Delta\mathrm{E}\left(  Z_{k}\right)  \lambda\right)
^{l}1\left\{  \left\vert Z_{k}1_{\Delta_{N_{(k+1)\Delta}}}\right\vert \leq
\tau\left(  \Delta\right)  \right\}  \right)  \leq C\tau\left(  \Delta\right)
^{l+1}.
\]


\noindent\textbf{Proof of Lemma 2: }For $l$ even, and for $c_{l-k,k}>0,$
whenever $l$ and $k$ are even,%
\begin{align*}
&  \mathrm{E}\left(  \frac{1}{T}\sum_{k=1}^{n}\left(  Z_{k}1_{\Delta
_{N_{(k+1)\Delta}}}-\Delta\mathrm{E}\left(  Z_{k}\right)  \lambda\right)
^{l}1\left\{  \left\vert Z_{k}1_{\Delta_{N_{(k+1)\Delta}}}\right\vert \leq
\tau\left(  \Delta\right)  \right\}  \right) \\
&  =\frac{1}{\Delta}\int_{-\tau\left(  \Delta\right)  }^{\tau\left(
\Delta\right)  }\left(  \left(  z-\Delta\mathrm{E}\left(  Z\right)
\lambda\right)  \right)  ^{l}\mathrm{E}\left(  1_{\Delta_{N_{(k+1)\Delta}}%
}\right)  f_{Z}(z)\mathrm{d}z\\
&  =\lambda\int_{-\tau\left(  \Delta\right)  }^{\tau\left(  \Delta\right)
}\left(  z-\Delta\mathrm{E}\left(  Z_{k}\right)  \lambda\right)  ^{l}%
f_{Z}(z)\mathrm{d}z\\
&  =\lambda\sum_{k=0}^{l}c_{l-k,k}\int_{-\tau\left(  \Delta\right)  }%
^{\tau\left(  \Delta\right)  }Z^{l-k}\left(  \Delta\mathrm{E}\left(
Z_{k}\right)  \lambda\right)  ^{k}f_{Z}(z)\mathrm{d}z\left(  1+o(\Delta
)\right) \\
&  =\lambda c_{l,0}\int_{-\tau\left(  \Delta\right)  }^{\tau\left(
\Delta\right)  }Z^{l}f_{Z}(z)\mathrm{d}z\left(  1+o(\Delta)\right) \\
&  \leq C\tau\left(  \Delta\right)  ^{l+1}.
\end{align*}
\medskip\noindent\textbf{Lemma 3: }Let Assumptions \textbf{A(i)-(v)} hold. If
as $T,\Delta^{-1}\rightarrow\infty,$ $T^{+}/T\rightarrow\infty,$ $\frac
{\Delta^{\frac{1}{2}-\frac{3}{m}}}{\tau\left(  \Delta\right)  }\rightarrow0,$
with $m>6$ and $\sqrt{T}\tau^{2}\left(  \Delta\right)  \rightarrow0,$ then,%
\[
\sqrt{T}\left(  \frac{1}{T^{+}}\sum_{k=1}^{n^{+}}\left(  \ln X_{k\Delta}-\ln
X_{(k-1)\Delta}\right)  1\left\{  \left\vert \ln X_{k\Delta}-\ln
X_{(k-1)\Delta}\right\vert \leq\tau\left(  \Delta\right)  \right\}
-\mu\right)  =o_{p}(1).
\]
\smallskip\noindent\textbf{Proof of Lemma 3: } Note that $\sqrt{T}\tau\left(
\Delta\right)  \rightarrow0$ implies that $T\Delta^{2}\rightarrow0,$ and so by
Lemma 1,%
\begin{align*}
&  \frac{\sqrt{T}}{T^{+}}\sum_{k=1}^{n^{+}}\left(  \ln X_{k\Delta}-\ln
X_{(k-1)\Delta}\right)  1\left\{  \left\vert \ln X_{k\Delta}-\ln
X_{(k-1)\Delta}\right\vert \leq\tau\left(  \Delta\right)  \right\} \\
&  =\frac{\sqrt{T}}{T^{+}}\sum_{k=1}^{n^{+}}\left(  \mu\Delta+V_{k\Delta
}^{1/2}\sqrt{1-\rho^{2}}\left(  W_{1,\left(  k+1\right)  \Delta}-W_{1,k\Delta
}\right)  +V_{k\Delta}^{1/2}\rho\left(  W_{2,\left(  k+1\right)  \Delta
}-W_{2,k\Delta}\right)  \right)  \left(  1-1_{\Delta_{N_{(k+1)\Delta}}}\right)
\\
&  +\frac{\sqrt{T}}{T^{+}}\sum_{k=1}^{n^{+}}\left(  \mu\Delta+V_{k\Delta
}^{1/2}\sqrt{1-\rho^{2}}\left(  W_{1,\left(  k+1\right)  \Delta}-W_{1,k\Delta
}\right)  +V_{k\Delta}^{1/2}\rho\left(  W_{2,\left(  k+1\right)  \Delta
}-W_{2,k\Delta}\right)  +Z_{k}\right) \\
&  \times1\left\{  \left\vert Z_{k}1_{\Delta_{N_{(k+1)\Delta}}}\right\vert
\leq\tau\left(  \Delta\right)  \right\}  +o_{p}(1)\\
&  =I_{T,\Delta}+II_{T,\Delta}.
\end{align*}
We need to show that: (i) $I_{T,\Delta}=\sqrt{T}\mu+o_{p}(1);$ and (ii)
$II_{T,\Delta}=o_{p}(1).$

For $T^{+}/T\rightarrow\infty,$%
\[
\frac{\sqrt{T}}{T^{+}}\sum_{k=1}^{n^{+}}\left(  V_{k\Delta}^{1/2}\sqrt
{1-\rho^{2}}\left(  W_{1,\left(  k+1\right)  \Delta}-W_{1,k\Delta}\right)
+V_{k\Delta}^{1/2}\rho\left(  W_{2,\left(  k+1\right)  \Delta}-W_{2,k\Delta
}\right)  \right)  \left(  1-1_{\Delta_{N_{(k+1)\Delta}}}\right)  =o_{p}(1)
\]
and
\[
\frac{\sqrt{T}}{T^{+}}\sum_{k=1}^{n^{+}}1_{\Delta_{N_{(k+1)\Delta}}}=o_{p}(1)
\]
as $\mathrm{var}\left(  1_{\Delta_{N_{(k+1)\Delta}}}\right)  =O\left(
\Delta\right)  $ and $T/T^{+}\rightarrow0.$ Thus, (i) is established. With
regard to (ii), as
\begin{align*}
\mathrm{E}\left(  1\left\{  \left\vert Z_{k}1_{\Delta_{N_{(k+1)\Delta}}%
}\right\vert \leq\tau\left(  \Delta\right)  \right\}  \right)   &
=\mathrm{E}\left(  1\left\{  \left\vert Z_{k}\right\vert \leq\tau\left(
\Delta\right)  \right\}  \right)  \mathrm{E}\left(  1_{\Delta_{N_{(k+1)\Delta
}}}\right) \\
&  \leq O\left(  \tau\left(  \Delta\right)  \Delta\right)
\end{align*}
it follows that%
\begin{align*}
&  \frac{\sqrt{T}}{T^{+}}\sum_{k=1}^{n^{+}}\left(  \mu\Delta+V_{k\Delta}%
^{1/2}\sqrt{1-\rho^{2}}\left(  W_{1,\left(  k+1\right)  \Delta}-W_{1,k\Delta
}\right)  +V_{k\Delta}^{1/2}\rho\left(  W_{2,\left(  k+1\right)  \Delta
}-W_{2,k\Delta}\right)  \right) \\
&  \times1\left\{  \left\vert Z_{k}1_{\Delta_{N_{(k+1)\Delta}}}\right\vert
\leq\tau\left(  \Delta\right)  \right\}  =o_{p}(1).
\end{align*}
Finally,%
\begin{align*}
&  \frac{\sqrt{T}}{T^{+}}\sum_{k=1}^{n^{+}}\mathrm{E}\left(  \left\vert
Z_{k}\right\vert \right)  1\left\{  \left\vert Z_{k}1_{\Delta_{N_{(k+1)\Delta
}}}\right\vert \leq\tau\left(  \Delta\right)  \right\} \\
&  =\frac{\sqrt{T}}{\Delta}\int_{-\tau\left(  \Delta\right)  }^{\tau\left(
\Delta\right)  }|z|f_{Z}(z)\mathrm{d}z\mathrm{E}\left(  1_{\Delta
_{N_{(k+1)\Delta}}}\right) \\
&  \leq C\sqrt{T}\tau^{2}\left(  \Delta\right)  =o(1).
\end{align*}
Then (ii) follows by a straightforward application of the Markov
inequality.\medskip

\noindent\textbf{Proof of Theorem 1:}

\noindent Part (i): From the multivariate Milstein formula (see Kloeden and
Platen (1999), Section 10.3), we have that:%
\begin{align}
&  \ln X_{\left(  k+1\right)  \Delta}-\ln X_{k\Delta}\nonumber\\
&  =\left(  \mu\Delta+V_{k\Delta}^{1/2}\sqrt{1-\rho^{2}}\left(  W_{1,\left(
k+1\right)  \Delta}-W_{1,k\Delta}\right)  +V_{k\Delta}^{1/2}\rho\left(
W_{2,\left(  k+1\right)  \Delta}-W_{2,k\Delta}\right)  \right. \nonumber\\
&  +\frac{1}{4}\rho V_{(k-1)\Delta}^{-1/2}g\left(  V_{(k-1)\Delta}%
,\theta\right)  \left(  \left(  W_{2,\left(  k+1\right)  \Delta}-W_{2,k\Delta
}\right)  ^{2}-\Delta\right) \nonumber\\
&  \left.  +\frac{1}{4}\sqrt{1-\rho^{2}}V_{(k-1)\Delta}^{-1/2}g\left(
V_{k\Delta},\theta\right)  \int_{k\Delta}^{(k+1)\Delta}\int_{k\Delta}^{s_{2}%
}\mathrm{d}W_{2,s_{1}}\mathrm{d}W_{1,s_{2}}\right)  \left(  1+o_{p}(1)\right)
\label{DC}%
\end{align}
Also,%
\begin{equation}
\frac{\ln X_{T}}{n}=\left(  \mu\Delta+\sqrt{1-\rho^{2}}\frac{\Delta}{T}%
\int_{0}^{T}V_{s}^{1/2}\mathrm{d}W_{1,s}+\rho\frac{\Delta}{T}\int_{0}^{T}%
V_{s}^{1/2}\mathrm{d}W_{2,s}\right)  \left(  1+o_{p}(1)\right)  \label{XT}%
\end{equation}
and%
\[
\frac{\ln X_{\Delta}}{n}=\left(  \mu\frac{\Delta^{2}}{T}+\sqrt{1-\rho^{2}%
}\frac{\Delta}{T}\int_{0}^{\Delta}V_{s}^{1/2}\mathrm{d}W_{1,s}+\rho
\frac{\Delta}{T}\int_{0}^{\Delta}V_{s}^{1/2}\mathrm{d}W_{2,s}\right)  \left(
1+o_{p}(1)\right)  .
\]
Thus, $\frac{\ln X_{\Delta}}{n}=O_{p}\left(  \frac{\Delta^{3/2}}{T}\right)  $
and $\frac{\ln X_{\Delta}}{n^{+}}=O_{p}\left(  \frac{\Delta^{3/2}}{T^{+}%
}\right)  .$ These terms can thus be ignored given that they are $o_{p}\left(
\Delta\right)  .$ Now,%

\begin{align*}
&  \left(  \ln X_{k\Delta}-\ln X_{(k-1)\Delta}-\frac{\ln X_{T}}{n}\right)
^{3}\\
&  =\left(  \sqrt{1-\rho^{2}}V_{k\Delta}^{1/2}\left(  W_{1,\left(  k+1\right)
\Delta}-W_{1,k\Delta}\right)  +\rho V_{k\Delta}^{1/2}\left(  W_{2,\left(
k+1\right)  \Delta}-W_{2,k\Delta}\right)  \right. \\
&  +\frac{1}{4}\rho V_{k\Delta}^{-1/2}g\left(  V_{k\Delta},\theta\right)
\left(  \left(  W_{2,\left(  k+1\right)  \Delta}-W_{2,k\Delta}\right)
^{2}-\Delta\right)  +\\
&  \left.  +\frac{1}{4}\sqrt{1-\rho^{2}}V_{(k-1)\Delta}^{-1/2}g\left(
V_{k\Delta},\theta\right)  \int_{k\Delta}^{(k+1)\Delta}\int_{k\Delta}^{s_{2}%
}\mathrm{d}W_{2,s_{1}}\mathrm{d}W_{1,s_{2}}\right)  ^{3}\left(  1+o_{p}%
(1)\right)  .
\end{align*}
Straightforward but tedious algebra shows that%
\begin{align*}
&  \mathrm{E}\left(  \ln X_{k\Delta}-\ln X_{(k-1)\Delta}-\frac{\ln X_{T}}%
{n}\right)  ^{3}\\
&  =\frac{3}{2}\rho^{3}\mathrm{E}\left(  V_{k\Delta}^{1/2}g\left(  V_{k\Delta
},\theta\right)  \right)  \Delta^{2}.
\end{align*}
Because of Lemma 1,%
\begin{align*}
\widehat{\mu}_{3,T,\Delta}  &  =\frac{1}{T}\sum_{k=1}^{n}\left(  \ln
X_{k\Delta}-\ln X_{(k-1)\Delta}-\frac{\ln X_{T}}{n}\right)  ^{3}\\
&  -\frac{1}{T^{+}}\sum_{k=1}^{n^{+}}\left(  \ln X_{k\Delta}-\ln
X_{(k-1)\Delta}-\frac{\ln X_{T}}{n^{+}}\right)  ^{3}\left(  1+o_{p}(1)\right)
.
\end{align*}
Now, write%
\begin{align*}
&  \frac{\sqrt{T}}{\Delta}\widehat{\mu}_{3,T,\Delta}\\
&  =\frac{\Delta}{\sqrt{T}}\sum_{k=1}^{n}\left(  \frac{1}{\Delta^{2}}\left(
\ln X_{k\Delta}-\ln X_{(k-1)\Delta}-\frac{\ln X_{n\Delta}-\ln X_{\Delta}}%
{n}\right)  ^{3}-\frac{3}{2}\rho^{3}\mathrm{E}\left(  V_{k\Delta}%
^{1/2}g\left(  V_{k\Delta},\theta\right)  \right)  \right) \\
&  -\frac{\sqrt{T}}{T^{+}}\Delta\sum_{k=1}^{n^{+}}\left(  \frac{1}{\Delta^{2}%
}\left(  \ln X_{k\Delta}-\ln X_{(k-1)\Delta}-\frac{\ln X_{n^{+}\Delta}-\ln
X_{\Delta}}{n^{+}}\right)  ^{3}-\frac{3}{2}\rho^{3}\mathrm{E}\left(
V_{k\Delta}^{1/2}g\left(  V_{k\Delta},\theta\right)  \right)  \right)  \left(
1+o_{p}(1)\right) \\
&  =\left(  I_{T,\Delta}+II_{T,T^{+},\Delta}\right)  \left(  1+o_{p}%
(1)\right)  .
\end{align*}
It is immediate to see that \textrm{E}$\left(  I_{T,\Delta}\right)  =0.$ Also,
recalling that for $m$ even, the $m-$th central moment of a standard normal is
equal to $\frac{m!}{2^{m/2}(m/2)!},$%
\begin{align*}
&  \omega_{0}=\mathrm{var}\left(  I_{T,\Delta}\right) \\
&  =\mathrm{var}\left(  \frac{1}{\sqrt{T}\Delta}\sum_{k=1}^{n}\left(  \left(
1-\rho^{2}\right)  ^{1/2}V_{k\Delta}^{1/2}\left(  W_{1,\left(  k+1\right)
\Delta}-W_{1,k\Delta}\right)  +\rho V_{k\Delta}^{1/2}\left(  W_{2,\left(
k+1\right)  \Delta}-W_{2,k\Delta}\right)  \right)  ^{3}\right)  +o(1)\\
&  =\mathrm{E}\left(  \left(  \left(  1-\rho^{2}\right)  ^{1/2}V_{k\Delta
}^{1/2}\left(  W_{1,\left(  k+1\right)  \Delta}-W_{1,k\Delta}\right)  +\rho
V_{k\Delta}^{1/2}\left(  W_{2,\left(  k+1\right)  \Delta}-W_{2,k\Delta
}\right)  \right)  ^{6}\right) \\
&  =15\left(  1-\rho^{2}\right)  ^{3}\mathrm{E}\left(  V_{k\Delta}^{3}\right)
+15\rho^{6}\mathrm{E}\left(  V_{k\Delta}^{3}\right)  +45\left(  1-\rho
^{2}\right)  ^{2}\rho^{2}\mathrm{E}\left(  V_{k\Delta}^{3}\right)  +45\left(
1-\rho^{2}\right)  \rho^{4}\mathrm{E}\left(  V_{k\Delta}^{3}\right)  .
\end{align*}
Hence, by the central limit theorem for martingale differences,%
\[
I_{T,\Delta}\overset{d}{\rightarrow}N\left(  0,\omega_{0}\right)  .
\]


\noindent Since $T^{+}/T\rightarrow\infty,$ $II_{T,T^{+},\Delta}$ is of
smaller probability order than $I_{T,\Delta},$ and thus is $o_{p}(1).$ The
statement in Part (i) then follows.

\medskip

\noindent Part (ii): Let%
\[
\ln X_{\left(  k+1\right)  \Delta}-\ln X_{k\Delta}=\left(  \ln X_{\left(
k+1\right)  \Delta}^{c}-\ln X_{k\Delta}^{c}\right)  +\left(  \ln X_{\left(
k+1\right)  \Delta}^{d}-\ln X_{k\Delta}^{d}\right)  ,
\]
where $\left(  \ln X_{\left(  k+1\right)  \Delta}^{c}-\ln X_{k\Delta}%
^{c}\right)  $ is defined as in the RHS of (\ref{DC}), and%
\[
\left(  \ln X_{\left(  k+1\right)  \Delta}^{d}-\ln X_{k\Delta}^{d}\right)
=Z_{k}1_{\Delta N_{(k+1)\Delta}},
\]
where $Z_{k}$ denotes a draw from the jump size density, say $f_{Z},$ and
$1_{\Delta N_{(k+1)\Delta}}=1,$ if $\Delta N_{(k+1)\Delta}=1,$ and equals zero
otherwise. Also%
\[
\frac{\ln X_{T}}{n}=\frac{\ln X_{T}^{c}}{n}+\frac{1}{n}\sum_{i=0}^{N_{T}}%
Z_{i}=\frac{\ln X_{T}^{c}}{n}+\Delta\lambda\mathrm{E}\left(  Z_{k}\right)
+o_{p}(1),
\]
with $\frac{\ln X_{T}^{c}}{n}$ defined as in the RHS of (\ref{XT}). Write,%
\begin{align*}
\frac{\sqrt{T}}{\Delta}\widehat{\mu}_{3,T,\Delta}  &  =\left(  \frac{1}%
{\Delta\sqrt{T}}\sum_{k=1}^{n}\left(  \ln X_{k\Delta}-\ln X_{(k-1)\Delta
}-\frac{\ln X_{n\Delta}-\ln X_{\Delta}}{n}\right)  ^{3}1\left\{  \left\vert
\ln X_{k\Delta}-\ln X_{(k-1)\Delta}\right\vert \leq\tau\left(  \Delta\right)
\right\}  \right. \\
&  \left.  -\frac{\sqrt{T}}{\Delta T^{+}}\sum_{k=1}^{n^{+}}\left(  \ln
X_{k\Delta}-\ln X_{(k-1)\Delta}-\frac{\ln X_{n^{+}\Delta}-\ln X_{\Delta}%
}{n^{+}}\right)  ^{3}1\left\{  \left\vert \ln X_{k\Delta}-\ln X_{(k-1)\Delta
}\right\vert \leq\tau\left(  \Delta\right)  \right\}  \right) \\
&  +\frac{1}{\Delta\sqrt{T}}\sum_{k=1}^{n}\left(  \ln X_{k\Delta}-\ln
X_{(k-1)\Delta}-\frac{\ln X_{n\Delta}-\ln X_{\Delta}}{n}\right)  ^{3}1\left\{
\left\vert \ln X_{k\Delta}-\ln X_{(k-1)\Delta}\right\vert >\tau\left(
\Delta\right)  \right\} \\
&  =A_{T,T^{+},\Delta}+B_{T,\Delta}.
\end{align*}
By Lemma 1,%
\begin{align*}
A_{T,T^{+},\Delta}  &  =\frac{1}{\Delta\sqrt{T}}\sum_{k=1}^{n}\left(  \left(
\ln X_{k\Delta}^{c}-\ln X_{(k-1)\Delta}^{c}-\frac{\ln X_{n\Delta}^{c}}%
{n}\right)  ^{3}-\frac{3}{2}\rho^{3}\mathrm{E}\left(  V_{k\Delta}%
^{1/2}g\left(  V_{k\Delta},\theta\right)  \right)  \right) \\
&  +\left(  \frac{1}{\Delta\sqrt{T}}\sum_{k=1}^{n}\left(  Z_{k}1_{\Delta
N_{(k+1)\Delta}}-\Delta\lambda\mathrm{E}\left(  Z_{k}\right)  \right)
^{3}1\left\{  \left\vert \ln X_{k\Delta}-\ln X_{(k-1)\Delta}\right\vert
\leq\tau\left(  \Delta\right)  \right\}  \right. \\
&  \left.  -\frac{\sqrt{T}}{\Delta T^{+}}\sum_{k=1}^{n^{+}}\left(
Z_{k}1_{\Delta N_{(k+1)\Delta}}-\Delta\lambda\mathrm{E}\left(  Z_{k}\right)
\right)  ^{3}1\left\{  \left\vert \ln X_{k\Delta}-\ln X_{(k-1)\Delta
}\right\vert \leq\tau\left(  \Delta\right)  \right\}  \right) \\
&  +\mathrm{\ cross}\text{ \textrm{terms}}\\
&  =A_{1,T,T^{+},\Delta}+A_{2,T,T^{+},\Delta}+\mathrm{\ cross}\text{
\textrm{terms.}}%
\end{align*}
By the same argument as that used in Part (i), $A_{1,T,T^{+},\Delta
}\overset{d}{\rightarrow}N(0,\omega_{0}),$ while $A_{2,T,T^{+},\Delta
}+\mathrm{\ cross}$ \textrm{terms} is of a smaller probability order than
$B_{T,\Delta}.$

\noindent Because of Lemma 1,
\[
B_{T,\Delta}=\frac{1}{\sqrt{T}}\sum_{k=1}^{n}\frac{1}{\Delta}\left(
Z_{k}1_{\Delta N_{(k+1)\Delta}}-\Delta\lambda\mathrm{E}\left(  Z_{k}\right)
_{i}\right)  ^{3}+o_{p}(1).
\]


\noindent Now,%
\begin{align}
&  \frac{1}{\sqrt{T}}\sum_{k=1}^{n}\frac{1}{\Delta}\left(  Z_{k}1_{\Delta
N_{(k+1)\Delta}}-\Delta\lambda\mathrm{E}\left(  Z_{k}\right)  \right)
^{3}\nonumber\\
&  =\frac{1}{\sqrt{T}}\sum_{k=1}^{n}\frac{1}{\Delta}Z_{k}^{3}1_{\Delta
N_{(k+1)\Delta}}-2\lambda\mathrm{E}\left(  Z_{k}\right)  \frac{\Delta}%
{\sqrt{T}}\sum_{k=1}^{n}\frac{1}{\Delta}Z_{i}^{2}1_{\Delta N_{(k+1)\Delta}%
}\nonumber\\
&  +2\lambda^{2}\mathrm{E}\left(  Z_{k}\right)  ^{2}\frac{\Delta^{2}}{\sqrt
{T}}\sum_{k=1}^{n}\frac{1}{\Delta}Z_{k}1_{\Delta N_{(k+1)\Delta}}-\sqrt
{T}\Delta\lambda^{3}\mathrm{E}\left(  Z_{k}\right)  ^{3}\nonumber\\
&  =\frac{1}{\sqrt{T}}\sum_{k=1}^{n}\left(  \frac{1}{\Delta}Z_{k}^{3}1_{\Delta
N_{(k+1)\Delta}}-\lambda\mathrm{E}\left(  Z_{k}^{3}\right)  \right)
+\frac{\sqrt{T}}{\Delta}\lambda\mathrm{E}\left(  Z_{k}^{3}\right) \nonumber\\
&  -2\lambda\mathrm{E}\left(  Z_{k}\right)  \frac{\Delta}{\sqrt{T}}\sum
_{k=1}^{n}\left(  \frac{1}{\Delta}Z_{k}^{2}1_{\Delta N_{(k+1)\Delta}}%
-\lambda\mathrm{E}\left(  Z_{k}^{2}\right)  \right)  -\sqrt{T}2\lambda
^{2}\mathrm{E}\left(  Z_{k}\right)  \mathrm{E}\left(  Z_{k}^{2}\right)
+o_{p}(1), \label{Z3}%
\end{align}
since $\sqrt{T}\Delta\lambda^{3}\mathrm{E}\left(  Z_{k}\right)  ^{3}=o(1)$ and
$\frac{\Delta^{2}}{\sqrt{T}}\sum_{k=1}^{n}\frac{1}{\Delta}Z_{k}1_{\Delta
N_{(k+1)\Delta}}=o_{p}(1),$ for $\sqrt{T}\Delta\rightarrow0.$

\noindent From (\ref{Z3}), we see that the statistic has $\frac{\sqrt{T}%
}{\Delta}$ Pitman drift, whenever $\mathrm{E}\left(  Z_{k}^{3}\right)  \neq0.$
The statement in Part (ii) then follows.

\medskip

\noindent Part (iii): When $\mathrm{E}\left(  Z_{k}^{3}\right)  =\mathrm{E}%
\left(  Z_{k}\right)  =0,$%
\begin{align*}
&  \frac{1}{\sqrt{T}}\sum_{k=1}^{n}\frac{1}{\Delta}\left(  Z_{k}1_{\Delta
N_{(k+1)\Delta}}-\Delta\lambda\mathrm{E}\left(  Z_{k}\right)  \right)  ^{3}\\
&  =\frac{1}{\sqrt{T}}\sum_{k=1}^{n}\frac{1}{\Delta}Z_{k}^{3}1_{\Delta
N_{(k+1)\Delta}}+o_{p}(1).
\end{align*}
We now show that $\mathrm{var}\left(  S_{T,\Delta}\right)  =O\left(  \frac
{1}{\Delta^{2}}\right)  ,$ regardless of whether the jump intensity is
constant or path dependent.

\noindent If $\beta=0$ (no path dependent intensity), then:%
\begin{align*}
&  \mathrm{var}\left(  S_{T,\Delta}\right) \\
&  =\mathrm{var}\left(  \frac{1}{\sqrt{T}\Delta}\sum_{k=1}^{n}Z_{k}%
^{3}1_{\Delta N_{(k+1)\Delta}}\right)  \left(  1+o(1)\right) \\
&  =\frac{1}{T\Delta^{2}}\sum_{k=1}^{n}\mathrm{var}\left(  Z_{k}^{3}1_{\Delta
N_{(k+1)\Delta}}\right)  \left(  1+o(1)\right) \\
&  =\frac{1}{\Delta^{3}}\mathrm{var}\left(  Z_{k}^{3}1_{\Delta N_{(k+1)\Delta
}}\right)  \left(  1+o(1)\right)  =O\left(  \frac{1}{\Delta^{2}}\right)  .
\end{align*}
Alternatively, if $\beta>0,$ one must take autocovariance terms into account
when carrying out similar calculations. However, given A(iv), the order of
magnitude of the variance is still $O\left(  \frac{1}{\Delta^{2}}\right)  $.
Given that $\sqrt{T}\Delta\rightarrow0,$ $S_{T,\Delta}$ is of probability
order $\Delta^{-1}$, and the statement in Part (iii) follows.

\bigskip

\noindent\textbf{Proof of Corollary 2:}

\noindent Part (i). We need to show that $\widehat{\sigma}_{\lambda,T,\Delta
}^{2}-\omega_{0}=o_{p}(1),$ with $\omega_{0}$ defined as in the statement of
Theorem 1. By Lemma 1,%
\[
\widehat{\sigma}_{\lambda,T,\Delta}^{2}=\frac{1}{T\Delta^{2}}\sum_{k=1}%
^{n}\left(  \ln X_{k\Delta}-\ln X_{(k-1)\Delta}-\frac{\ln X_{n\Delta}-\ln
X_{\Delta}}{n}\right)  ^{6}+o_{p}(1).
\]
The statement of Part (i) follows directly by the law of large numbers (for
iid processes if $\beta=0$ and for ergodic mixing processes if $\beta>0).$

\noindent Parts (ii)-(iii): We need to show that $\widehat{\sigma}%
_{\lambda,T,\Delta}^{2}=O_{p}(1).$Now, note that:%
\begin{align}
\widehat{\sigma}_{\lambda,T,\Delta}^{2}  &  =\frac{1}{T\Delta^{2}}\sum
_{k=1}^{n}\left(  \ln X_{k\Delta}^{c}-\ln X_{(k-1)\Delta}^{c}-\frac{\ln
X_{n\Delta}^{c}-\ln X_{\Delta}^{c}}{n}\right)  ^{6}\nonumber\\
&  +\frac{1}{T\Delta^{2}}\sum_{k=1}^{n}\left(  Z_{k}1_{\Delta N_{(k+1)\Delta}%
}-\Delta\lambda\mathrm{E}\left(  Z\right)  \right)  ^{6}1\left\{  \left\vert
\ln X_{k\Delta}-\ln X_{(k-1)\Delta}\right\vert \leq\tau\left(  \Delta\right)
\right\} \nonumber\\
&  +\mathrm{\ cross}\text{ }\mathrm{terms}\text{ }+o_{p}(1). \label{sigma-l}%
\end{align}
The first term on the RHS of (\ref{sigma-l}) is a consistent estimator of
$\omega_{0}$. It suffices to show that the second term on the RHS of
(\ref{sigma-l}) is $O_{p}(1).$ This follows because the cross term cannot be
of a larger order than the second term. Given Lemma 1,%
\begin{align*}
&  \frac{1}{T\Delta^{2}}\sum_{k=1}^{n}\left(  Z_{k}1_{\Delta N_{(k+1)\Delta}%
}-\Delta\lambda\mathrm{E}\left(  Z_{k}\right)  \right)  ^{6}1\left\{
\left\vert \ln X_{k\Delta}-\ln X_{(k-1)\Delta}\right\vert \leq\tau\left(
\Delta\right)  \right\} \\
&  =O_{p}(1)\left(  \frac{1}{T\Delta^{2}}\sum_{k=1}^{n}\left(  Z_{k}1_{\Delta
N_{(k+1)\Delta}}-\Delta\lambda\mathrm{E}\left(  Z_{k}\right)  \right)
^{6}1\left\{  \left\vert Z_{k}1_{\Delta_{N_{(k+1)\Delta}}}\right\vert \leq
\tau\left(  \Delta\right)  \right\}  \right)  ,
\end{align*}
and by Lemma 2,%

\begin{align*}
&  P\left(  \frac{1}{T\Delta^{2}}\sum_{k=1}^{n}\left(  Z_{k}1_{\Delta
_{N_{(k+1)\Delta}}}-\mathrm{E}\left(  Z_{k}1_{\Delta_{N_{(k+1)\Delta}}%
}\right)  \right)  ^{6}1\left\{  \left\vert Z_{k}1_{\Delta_{N_{(k+1)\Delta}}%
}\right\vert \leq\tau\left(  \Delta\right)  \right\}  >\varepsilon\right) \\
&  \leq\frac{1}{\Delta^{2}\varepsilon}\mathrm{E}\left(  \frac{1}{T}\sum
_{k=1}^{n}\left(  Z_{k}1_{\Delta_{N_{(k+1)\Delta}}}-\mathrm{E}\left(
Z_{k}1_{\Delta_{N_{(k+1)\Delta}}}\right)  \right)  ^{6}1\left\{  \left\vert
Z_{k}1_{\Delta_{N_{(k+1)\Delta}}}\right\vert \leq\tau\left(  \Delta\right)
\right\}  \right) \\
&  \rightarrow0,
\end{align*}
provided that $\tau\left(  \Delta\right)  ^{7}\Delta^{-2}\rightarrow0.$

\medskip

\noindent\textbf{Proof of Theorem 3:}

\noindent Part (i): By Lemma 3,%
\begin{align*}
&  \sqrt{T}\widehat{\mu}_{T,\Delta}^{Z}\\
&  =\frac{1}{\sqrt{T}}\sum_{k=0}^{n}\left(  \sqrt{1-\rho^{2}}V_{k\Delta}%
^{1/2}\left(  W_{1,\left(  k+1\right)  \Delta}-W_{1,k\Delta}\right)  +\rho
V_{k\Delta}^{1/2}\left(  W_{2,\left(  k+1\right)  \Delta}-W_{2,k\Delta
}\right)  \right. \\
&  \left.  +\mu\Delta+Z_{k}1_{\Delta N_{(k+1)\Delta}}\right)  -\sqrt{T}%
\mu+o_{p}(1).
\end{align*}
Thus,%
\begin{align*}
&  \sqrt{T}\widehat{\mu}_{T,\Delta}^{Z}\\
&  =\frac{1}{\sqrt{T}}\sum_{k=0}^{n}\left(  \sqrt{1-\rho^{2}}V_{k\Delta}%
^{1/2}\left(  W_{1,\left(  k+1\right)  \Delta}-W_{1,k\Delta}\right)  +\rho
V_{k\Delta}^{1/2}\left(  W_{2,\left(  k+1\right)  \Delta}-W_{2,k\Delta
}\right)  +Z_{k}1_{\Delta N_{(k+1)\Delta}}\right) \\
&  \overset{d}{\rightarrow}N\left(  0,\sigma_{\mu_{Z}}^{2}\right)  ,
\end{align*}
with $\sigma_{\mu_{Z}}^{2}=\mathrm{E}\left(  V_{k\Delta}\right)
+\lambda\mathrm{E}\left(  Z_{k}^{2}\right)  .$ As $\widehat{\sigma}_{\mu_{Z}%
}^{2}=\sigma_{\mu_{Z}}^{2}+o_{p}(1),$ the statement in Part (i) follows
directly.\medskip

\noindent(ii) The proof is immediate, as%
\begin{align*}
&  \sqrt{T}\widehat{\mu}_{T,\Delta}^{Z}\\
&  =\frac{1}{\sqrt{T}}\sum_{k=0}^{n}\left(  \sqrt{1-\rho^{2}}V_{k\Delta}%
^{1/2}\left(  W_{1,\left(  k+1\right)  \Delta}-W_{1,k\Delta}\right)  +\rho
V_{k\Delta}^{1/2}\left(  W_{2,\left(  k+1\right)  \Delta}-W_{2,k\Delta
}\right)  +Z_{k}1_{\Delta N_{(k+1)\Delta}}\right)  +\lambda\sqrt{T}%
\mathrm{E}\left(  Z_{k}\right)  .
\end{align*}


\medskip

\noindent\textbf{Proof of Theorem 4:}

\noindent Part (i): Note that,%
\begin{align}
&  \sqrt{\frac{T^{+}}{\Delta}}\widehat{\beta}_{T,\Delta}\nonumber\\
&  =\frac{1}{\sqrt{T^{+}\Delta}}\sum_{k=1}^{n^{+}-1}\left(  \left(
\sqrt{1-\rho^{2}}V_{k\Delta}^{1/2}\left(  W_{1,\left(  k+1\right)  \Delta
}-W_{1,k\Delta}\right)  +\rho V_{k\Delta}^{1/2}\left(  W_{2,\left(
k+1\right)  \Delta}-W_{2,k\Delta}\right)  \right.  \right. \nonumber\\
&  \left.  +Z_{k}1_{\Delta N_{(k+1)\Delta}}-\Delta\lambda\mathrm{E}\left(
Z\right)  \right)  \left(  \sqrt{1-\rho^{2}}V_{k\Delta}^{1/2}\left(
W_{1,k\Delta}-W_{1,(k-1)\Delta}\right)  \right. \nonumber\\
&  \left.  \left.  \rho V_{k\Delta}^{1/2}\left(  W_{2,k\Delta}%
-W_{2,(k-1)\Delta}\right)  +Z_{k-1}1_{\Delta N_{k\Delta}}-\Delta
\lambda\mathrm{E}\left(  Z\right)  \right)  \right)  +o_{p}\left(  1\right)
\nonumber\\
&  =\frac{1}{\sqrt{T^{+}\Delta}}\sum_{k=1}^{n^{+}-1}\left(  1-\rho^{2}\right)
V_{k\Delta}\left(  W_{1,\left(  k+1\right)  \Delta}-W_{1,k\Delta}\right)
\left(  W_{1,k\Delta}-W_{1,(k-1)\Delta}\right) \nonumber\\
&  +\frac{2}{\sqrt{T^{+}\Delta}}\sum_{k=1}^{n^{+}-1}\sqrt{1-\rho^{2}}\rho
V_{k\Delta}\left(  W_{1,\left(  k+1\right)  \Delta}-W_{1,k\Delta}\right)
\left(  W_{2,k\Delta}-W_{2,(k-1)\Delta}\right) \nonumber\\
&  +\frac{1}{\sqrt{T^{+}\Delta}}\sum_{k=1}^{n^{+}-1}\rho^{2}V_{k\Delta}\left(
W_{2,\left(  k+1\right)  \Delta}-W_{2,k\Delta}\right)  \left(  W_{2,k\Delta
}-W_{2,(k-1)\Delta}\right) \nonumber\\
&  +\frac{1}{\sqrt{T^{+}\Delta}}\sum_{k=1}^{n^{+}-1}\left(  Z_{k}%
Z_{k-1}1_{\Delta N_{(k+1)\Delta}}1_{\Delta N_{k\Delta}}-\Delta^{2}\lambda
^{2}\mathrm{E}\left(  Z\right)  ^{2}\right)  +o_{p}\left(  1\right)  .
\label{tau}%
\end{align}
Under the null of $\beta=0,$%
\begin{align*}
&  \mathrm{E}\left(  Z_{k}Z_{k-1}1_{\Delta N_{(k+1)\Delta}}1_{\Delta
N_{(k+1)\Delta}}\right) \\
&  =\mathrm{E}\left(  Z_{k}\right)  ^{2}\mathrm{E}\left(  1_{\Delta
N_{(k+1)\Delta}}\right)  \mathrm{E}\left(  1_{\Delta N_{k\Delta}}\right)
=\Delta^{2}\lambda^{2}\mathrm{E}\left(  Z_{k}\right)  ^{2}.
\end{align*}
Thus, under $H_{0},$ all of the terms on the RHS of (\ref{tau}) have zero
mean. Also,%
\begin{align*}
\sigma_{\beta}^{2}  &  =\mathrm{var}\left(  \sqrt{\frac{T^{+}}{\Delta}%
}\widehat{\tau}_{T,\Delta}\right) \\
&  =\left(  \left(  1-\rho^{2}\right)  ^{2}+4\rho\left(  1-\rho^{2}\right)
+\rho^{4}\right)  \mathrm{E}\left(  V_{k\Delta}^{2}\right)  +\lambda
^{2}\left(  \mathrm{E}\left(  Z_{k}^{2}\right)  \right)  ^{2},
\end{align*}
and since $\widehat{\sigma}_{\beta,T,\Delta}^{2}=\sigma_{\beta,T,\Delta}%
^{2}+o_{p}(1),$ by the central limit theorem for martingale differences,%
\[
\sqrt{\frac{T^{+}}{\Delta}}t_{\beta,T^{+},\Delta}\rightarrow N\left(
0,\sigma_{\beta}^{2}\right)  .
\]
The statement in Part (i) follows from the continuous mapping theorem.

\medskip

\noindent Part (ii): The first three terms on the RHS of (\ref{tau}) are
asymptotically normal, under both hypotheses. With regard to the fourth term,
note that, under the alternative, from Hawkes (1971),%
\begin{align*}
&  \mathrm{E}\left(  Z_{k}Z_{k-1}1_{\Delta N_{(k+1)\Delta}}1_{\Delta
N_{k\Delta}}-\Delta^{2}\lambda^{2}\mathrm{E}\left(  Z_{k}\right)  ^{2}\right)
\\
&  =\Delta^{2}\frac{\beta\lambda\left(  2a-\beta\right)  }{2\left(
a-\beta\right)  }\exp\left(  -\left(  a-\beta\right)  \right)  \mathrm{E}%
\left(  Z_{k}\right)  ^{2},
\end{align*}
and so the fourth term on the RHS of (\ref{tau}) can be written as:%
\begin{align*}
&  \frac{1}{\sqrt{T^{+}\Delta}}\sum_{k=1}^{n^{+}-1}\left(  \left(
Z_{k}Z_{k-1}1_{\Delta N_{(k+1)\Delta}}1_{\Delta N_{k\Delta}}-\Delta^{2}%
\lambda^{2}\mathrm{E}\left(  Z_{k}\right)  ^{2}\right)  \right. \\
&  \left.  -\Delta^{2}\frac{\beta\lambda\left(  2a-\beta\right)  }{2\left(
a-\beta\right)  }\exp\left(  -\left(  a-\beta\right)  \right)  \mathrm{E}%
\left(  Z_{k}\right)  ^{2}\right) \\
&  +\sqrt{T^{+}\Delta}\frac{\beta\lambda\left(  2a-\beta\right)  }{2\left(
a-\beta\right)  }\exp\left(  -\left(  a-\beta\right)  \right)  \mathrm{E}%
\left(  Z_{k}\right)  ^{2}\\
&  =O_{p}(1)+\sqrt{T^{+}\Delta}\frac{\beta\lambda\left(  2a-\beta\right)
}{2\left(  a-\beta\right)  }\exp\left(  -\left(  a-\beta\right)  \right)
\mathrm{E}\left(  Z_{k}\right)  ^{2}.
\end{align*}


\pagebreak

\section{Reference}

\noindent A\"{\i}t-Sahalia, Y., J. Cacho-Diaz and R. Laeven (2015). Modeling
Financial Contagion Using Mutually Exciting Jump Processes. \textit{Journal of
Financial Economics, }117, 585-606.\smallskip

\noindent A\"{\i}t-Sahalia, Y., J. Jacod (2009). Testing for Jumps in a
Discretely Observed Process. \textit{Annals of Statistics, }37,
2202-2244.\smallskip

\noindent A\"{\i}t-Sahalia, Y., J. Jacod and J. Li (2012). Testing for Jumps
in Noisy High Frequency Data. \textit{Journal of Econometrics, }168,
207-222.\smallskip

\noindent Andersen, T.G., T. Bollerslev and F.X. Diebold (2000). Great
Realizations. \textit{Risk, }105-108.\smallskip

\noindent Amaya, D., P. Christoffersen, K. Jacobs and A. Vasquez (2015). Does
Realized Skewness Predict the Cross-Section of Equity Returns? \textit{Journal
of Financial Economics, }118, 135-167.\smallskip

\noindent Andersen, Benzoni and Lund (2002). An Empirical Investigation of
Continuous Time Equity Return Models. \textit{Journal of Finance, }62,
1239-1283\smallskip.

\noindent Andrews, D.W.K. (1999). Estimation When a Parameter is on the
Boundary. \textit{Econometrica, }67, 1341-1383.\smallskip

\noindent Andrews, D.W.K. (2001). Testing When a Parameter is on the Boundary
of the Maintained Hypothesis. \textit{Econometrica, }69, 683-734.\smallskip

\noindent Andrews, D.W.K. and X. Cheng (2012). Estimation and Inference with
Weak, Semi-strong and Strong Identification. \textit{Econometrica, }80,
2153-2211.\smallskip

\noindent Bandi, F.M. and R. Ren\`{o} (2012). Time Varying Leverage Effects.
\textit{Journal of Econometrics, }169, 94-113.\smallskip

\noindent Barndorff-Nielsen, O.E. and N. Shephard (2004). Power and Bipower
Variation with Stochastic Volatility and Jumps. \textit{Journal of Financial
Econometrics, }2, 1-48.\smallskip

\noindent Barndorff-Nielsen, O.E., N. Shephard and M. Winkel (2006). Limit
Theorem for Multipower Variation in the Presence of Jumps. \textit{Stochastic
Processes and Their Applications, }116, 796-806.\smallskip

\noindent Beg, A.B.M.R.A., M.J. Silvapulle and P. Silvapulle (2001). Testing
Against Inequality Constraints When Some Nuisance Parameters are Present Only
Under the Alternative: Test of ARCH in ARCH-M Models. \textit{Journal of
Business and Economic Statistics, }19, 245-253.\smallskip

\noindent Benjamini, Y., and Y. Hochberg (1995). Controlling the False
Discovery Rate: A Practical and Powerful Approach to Multiple Testing.
\textit{Journal of the Royal Statistical Society, B}, 57, 289-300.\smallskip

\noindent Bowsher, C.G. (2007). Modeling Security Market Event in Continuous
Time: Intensity-Based Multivariate, Point Process Models. \textit{Journal of
Econometrics, }141, 876-912.\smallskip

\noindent Chacko, G., and L.M. Viceira (2003). Spectral GMM\ estimation of
continuous-time Processes. \textit{Journal of Econometrics,} 116,
259-292.\smallskip

\noindent Corradi, V. and N.R. Swanson (2011). Predictive Density Construction
and Testing with Multiple Possibly Misspecified Diffusion Models.
\textit{Journal of Econometrics, }161, 304-324.\smallskip

\noindent Corsi, F., D. Pirino and R. Ren\`{o} (2010). Threshold Bipower
Variation and the Impact of Jumps on Volatility Forecasting. \textit{Journal
of Econometrics,} 159, 276-288.\smallskip

\noindent Duffie, D. and K. Singleton (1993). Simulated Moment Estimation of
Markov Models of Asset Prices.\textit{\ Econometrica,} 61, 929-952.\smallskip

\noindent Duffie, D., J. Pan and K. Singleton (2000). Transform Analysis and
Asset Pricing for Affine Jump Diffusion. \textit{Econometrica, }68,
1343-1376.\smallskip

\noindent Fermanian, J.-D. and B. Salani\'{e} (2004). A Nonparametric
Simulated Maximum Likelihood Estimation Method. \textit{Econometric Theory,}
20, 701-734.\smallskip

\noindent Eraker, B., M. Johannes, and N. Polson (2003). The Impact of Jumps
in Volatility and Returns. \textit{Journal of Finance} 58,
1269-1300.\smallskip

\noindent Escanciano, J.C. and I.N. Lobato (2009). An Automatic Data-Driven
Portmanteau Test for Testing Serial Autocorrelation. \textit{Journal of
Econometrics, }151, 140-149.\smallskip

\noindent Gallant, A.R. and G. Tauchen (1996). Which Moments to Match.
\textit{Econometric Theory,} 12, 657-681.\smallskip

\noindent Gourieroux, C., A. Monfort, and E. Renault (1993). Indirect
Inference. \textit{Journal of Applied Econometrics,} 8, 203-227.\smallskip

\noindent Hawkes, A.G. (1971). Spectra of Some Self-Exciting and Mutually
Exciting Point Processes. \textit{Biometrika, }58, 83-90.\smallskip

\noindent Holm, S. (1979). A Simple Sequentially Rejective Multiple Test
Procedure. \textit{Scandinavian Journal of Statistics}, 6, 65--70.\smallskip

\noindent Huang, X. and G.E. Tauchen (2005). The Relative Contribution of
Jumps to Total Price Variance. \textit{Journal of Financial Econometrics, }3,
456-499.\smallskip

\noindent Jacod, J. (2012). Statistics and High Frequency Data, in
\textit{Statistical Methods for Stochastic Differential Equations, }M.
Kessler, A. Limdner and M. Sorensen Eds, Chapman \& Hall, Monographs on
Statistics and Probability, v. 124.\smallskip

\noindent Jiang, G.J. and J.L. Knight (2002). Estimation of Continuous-time
Processes via the Empirical Characteristic Function. \textit{Journal of
Business and Economic Statistics,} 20, 198-212.\smallskip

\noindent Kloeden, P.E. and E. Platen (1999). \textit{Numerical Solutions of
Stochastic Differential Equations}. Springer Verlag.\smallskip

\noindent Li, Y., P.A. Mykland, E. Renault, L. Zhang, Z. Zheng (2014).
Realized Volatility when Sampling Times are Possibly Endogenous.
\textit{Econometric Theory, }30, 580-05.\smallskip

\noindent Lee, S. and P.A. Mykland (2008). Jumps in Financial Markets: A New
Nonparametric Test and Jump Dynamics. \textit{Review of Financial Studies,
}21, 2535-2563.\smallskip

\noindent Lee, T., M. Loretan and W. Ploberger (2013). Rate-Optimal Tests for
Jumps in Diffusion Processes. \textit{Statistical Papers. }54,
1009-1041.\smallskip

\noindent Mancini, C. (2009). Nonparametric Threshold Estimation for Models
with Stochastic Diffusion Coefficient and Jumps. \textit{Scandinavian Journal
of Statistics, }36, 270-296.\smallskip

\noindent Mancini, C. and R. Ren\`{o} (2011). Threshold Estimation of Markov
Models with Jumps and Interest Rate Modelling. \textit{Journal of
Econometrics, }160, 77-92.

\noindent Podolskij, M. and M. Vetter (2009a). Bipower Type Estimation in a
Noisy Diffusion Setting. \textit{Stochastic Processes and Their Applications.
}119, 2803-2832.\smallskip

\noindent Podolskij, M. and M. Vetter (2009b). Estimation of Volatility
Functionals in the Simultaneous Presence of Microstructure Noise and Jumps.
\textit{Bernoulli, }15, 634-658.\smallskip

\noindent Romano, J.P. and M. Wolf (2005). Stepwise Multiple Testing as
Formalized Data Snooping. \textit{Econometrica}, 73, 1237-1282.\smallskip

\noindent Silvapulle, M.J. and P.K. Sen (2005). \textit{Constrained
Statistical Inference: Inequality, Order and Shape Restrictions.
}Wiley.\smallskip

\noindent Singleton, K.J. (2001). Estimation of Affine Asset Pricing Models
Using Empirical Characteristic Function. \textit{Journal of Econometrics,}
102, 111-141.\smallskip

\noindent Storey, J.D. (2003). The Positive False Discovery Rate: a Bayesian
Interpretation and the $q$-value. \textit{Annals of Statistics}, 31,
2013-2035.\smallskip

\noindent White, H. (2000). A Reality Check For Data Snooping.
\textit{Econometrica}, 68, 1097-1127.

\clearpage
\allowdisplaybreaks


%% tables


\textwidth=6.5in \textheight=9.5in \topmargin=-1.2in \baselineskip=1pt \renewcommand{\baselinestretch}{1.2}

\newpage

%TABLE 1 Jump Intensity Test %%%%%%%%%%%%%%%%%%%%%%%%%%%%%%%%%%%%%%%%%%%%%%%%%%%%%%%%%%%%%%%%%%%%%%%%%%%%%%%%%%%%%%%%%%%%%%
\begin{table}[ptb]
\caption{\textbf{Monte Carlo Experiments - Jump Test (Empirical Size and
Empirical Power, $E(Z_{k}) \ne0$)} *}%
\label{tb1}%
\begin{tabular}
[c]{llllllllllll}
&  &  &  &  &  &  &  &  &  &  & \\
&  &  &  &  &  &  &  &  &  &  & \\\hline\hline
$\lambda_{\infty}$ & $(a, \beta)$ & $\sigma$ & $\Delta$ & $T1$ & $T2$ & $T3$ &
$T4$ & $T5$ & $T6$ & $T7$ & $T8$\\\hline
\multicolumn{12}{c}{\textbf{\emph{EMPIRICAL SIZE}}}\\\hline
\multicolumn{12}{c}{\textbf{\emph{No Leverage}}}\\\hline
-- & -- & -- & 1/78 & 0.072 & 0.086 & 0.106 & 0.148 & 0.142 & 0.166 & 0.188 &
0.210\\
-- & -- & -- & 1/156 & 0.032 & 0.040 & 0.052 & 0.056 & 0.080 & 0.090 & 0.098 &
0.108\\\hline
\multicolumn{12}{c}{\textbf{\emph{Leverage}}}\\\hline
-- & -- & -- & 1/78 & 0.060 & 0.102 & 0.114 & 0.132 & 0.154 & 0.182 & 0.204 &
0.222\\
-- & -- & -- & 1/156 & 0.036 & 0.040 & 0.044 & 0.064 & 0.070 & 0.088 & 0.096 &
0.104\\\hline
\multicolumn{12}{c}{\textbf{\emph{EMPIRICAL POWER}}}\\\hline
\multicolumn{12}{c}{\textbf{\emph{Constant Intensity, $Z_{k}$ is $Exp$(5)}}%
}\\\hline
0.3 & (0,0) & -- & 1/78 & 0.992 & 0.996 & 0.998 & 0.998 & 1.000 & 1.000 &
1.000 & 1.000\\
0.3 & (0,0) & -- & 1/156 & 1.000 & 1.000 & 1.000 & 1.000 & 1.000 & 1.000 &
1.000 & 1.000\\\hline
\multicolumn{12}{c}{\textbf{\emph{Constant Intensity, $Z_{k}$ is $N(0.5,
\sigma^{2} )$}}}\\\hline
0.3 & (0,0) & 0.2 & 1/78 & 0.976 & 0.986 & 0.998 & 0.998 & 1.000 & 1.000 &
1.000 & 1.000\\
0.3 & (0,0) & 0.2 & 1/156 & 1.000 & 1.000 & 1.000 & 1.000 & 1.000 & 1.000 &
1.000 & 1.000\\\hline
\multicolumn{12}{c}{\textbf{\emph{Self Excitement, $Z_{k}$ is $Exp$(5)}}%
}\\\hline
0.3 & (3,2) & -- & 1/78 & 0.992 & 0.996 & 0.996 & 1.000 & 1.000 & 1.000 &
1.000 & 1.000\\
0.3 & (3,2) & -- & 1/156 & 1.000 & 1.000 & 1.000 & 1.000 & 1.000 & 1.000 &
1.000 & 1.000\\
0.3 & (5,4) & -- & 1/78 & 0.992 & 0.994 & 0.998 & 0.998 & 0.998 & 1.000 &
1.000 & 1.000\\
0.3 & (5,4) & -- & 1/156 & 1.000 & 1.000 & 1.000 & 1.000 & 1.000 & 1.000 &
1.000 & 1.000\\
0.3 & (7,5) & -- & 1/78 & 0.988 & 0.992 & 0.996 & 1.000 & 1.000 & 1.000 &
1.000 & 1.000\\
0.3 & (7,5) & -- & 1/156 & 1.000 & 1.000 & 1.000 & 1.000 & 1.000 & 1.000 &
1.000 & 1.000\\\hline
\multicolumn{12}{c}{\textbf{\emph{Self Excitement, $Z_{k}$ is $N(0.5,
\sigma^{2} )$}}}\\\hline
0.3 & (3,2) & 0.2 & 1/78 & 0.992 & 0.994 & 0.996 & 1.000 & 1.000 & 1.000 &
1.000 & 1.000\\
0.3 & (3,2) & 0.2 & 1/156 & 1.000 & 1.000 & 1.000 & 1.000 & 1.000 & 1.000 &
1.000 & 1.000\\
0.3 & (5,4) & 0.2 & 1/78 & 0.982 & 0.988 & 0.996 & 0.996 & 0.996 & 1.000 &
1.000 & 1.000\\
0.3 & (5,4) & 0.2 & 1/156 & 1.000 & 1.000 & 1.000 & 1.000 & 1.000 & 1.000 &
1.000 & 1.000\\
0.3 & (7,5) & 0.2 & 1/78 & 0.978 & 0.988 & 0.992 & 0.996 & 0.996 & 1.000 &
1.000 & 1.000\\
0.3 & (7,5) & 0.2 & 1/156 & 1.000 & 1.000 & 1.000 & 1.000 & 1.000 & 1.000 &
1.000 & 1.000\\\hline\hline
\vspace{0.02in} &  &  &  &  &  &  &  &  &  &  &
\end{tabular}
{\footnotesize \begin{minipage}{1\columnwidth} \footnotesize * Entries in the table are rejection frequencies for the \textquotedblleft jump
test\textquotedblright\, based on
$t_{\lambda,T,\Delta}$. Results are tabulated
for the following sample size ($T$) and discretization ($\Delta$)
permutations: $\Delta$ = 1/78 -- $T1$%
:$T$=60, $T2$:$T$=70, $T3$:$T$=80, $T4$:$T$=90, $T5$:$T$=100, $T6$:$T$=110,
$T7$:$T$=120, $T8$:$T$=130. For $\Delta$ =
1/156 -- $T1$:$T$=160, $T2$:$T$=180, $T3$:$T$=200, $T4$:$T$=220, $T5$:$T$=240,
$T6$:$T$=260, $T7$:$T$=280, $T8$:$T$=300. In all experiments, we perform $1000$ Monte Carlo replications. For
complete details, refer to Section 5 of the paper.
\end{minipage}}\end{table}

\newpage

%TABLE 2 Jump Intensity Test %%%%%%%%%%%%%%%%%%%%%%%%%%%%%%%%%%%%%%%%%%%%%%%%%%%%%%%%%%%%%%%%%%%%%%%%%%%%%%%%%%%%%%%%%%%%%%
\begin{table}[ptb]
\caption{\textbf{Monte Carlo Experiments - Jump Test (Empirical Power,
$E(Z_{k}) =0$) *}}%
\label{tb2}%
\begin{tabular}
[c]{llllllllllll}
&  &  &  &  &  &  &  &  &  &  & \\
&  &  &  &  &  &  &  &  &  &  & \\\hline\hline
$\lambda_{\infty}$ & $(a, \beta)$ & $\sigma$ & $\Delta$ & $T1$ & $T2$ & $T3$ &
$T4$ & $T5$ & $T6$ & $T7$ & $T8$\\\hline
\multicolumn{12}{c}{\textbf{\emph{Constant Intensity, $Z_{k}$ is $N(0.0,
\sigma^{2} )$}}}\\\hline
0.3 & (0,0) & 0.1 & 1/78 & 0.078 & 0.132 & 0.136 & 0.142 & 0.190 & 0.202 &
0.224 & 0.240\\
0.3 & (0,0) & 0.1 & 1/156 & 0.166 & 0.176 & 0.200 & 0.216 & 0.222 & 0.262 &
0.274 & 0.268\\
0.3 & (0,0) & 0.2 & 1/78 & 0.382 & 0.464 & 0.482 & 0.498 & 0.524 & 0.540 &
0.574 & 0.578\\
0.3 & (0,0) & 0.2 & 1/156 & 0.768 & 0.794 & 0.794 & 0.798 & 0.802 & 0.818 &
0.834 & 0.848\\
0.3 & (0,0) & 0.4 & 1/78 & 0.796 & 0.844 & 0.874 & 0.890 & 0.908 & 0.904 &
0.916 & 0.928\\
0.3 & (0,0) & 0.4 & 1/156 & 0.972 & 0.970 & 0.966 & 0.970 & 0.968 & 0.988 &
0.982 & 0.984\\
0.5 & (0,0) & 0.1 & 1/78 & 0.086 & 0.130 & 0.132 & 0.178 & 0.194 & 0.218 &
0.244 & 0.258\\
0.5 & (0,0) & 0.1 & 1/156 & 0.224 & 0.274 & 0.288 & 0.296 & 0.306 & 0.304 &
0.344 & 0.334\\
0.5 & (0,0) & 0.2 & 1/78 & 0.462 & 0.520 & 0.576 & 0.588 & 0.596 & 0.636 &
0.636 & 0.648\\
0.5 & (0,0) & 0.2 & 1/156 & 0.832 & 0.838 & 0.856 & 0.850 & 0.860 & 0.882 &
0.892 & 0.886\\
0.5 & (0,0) & 0.4 & 1/78 & 0.888 & 0.894 & 0.918 & 0.936 & 0.936 & 0.950 &
0.956 & 0.946\\
0.5 & (0,0) & 0.4 & 1/156 & 0.978 & 0.982 & 0.962 & 0.982 & 0.978 & 0.974 &
0.978 & 0.976\\
0.7 & (0,0) & 0.1 & 1/78 & 0.122 & 0.170 & 0.192 & 0.188 & 0.214 & 0.224 &
0.262 & 0.286\\
0.7 & (0,0) & 0.1 & 1/156 & 0.294 & 0.300 & 0.326 & 0.348 & 0.362 & 0.388 &
0.364 & 0.370\\
0.7 & (0,0) & 0.2 & 1/78 & 0.574 & 0.604 & 0.644 & 0.654 & 0.660 & 0.676 &
0.682 & 0.706\\
0.7 & (0,0) & 0.2 & 1/156 & 0.856 & 0.864 & 0.896 & 0.886 & 0.866 & 0.874 &
0.886 & 0.872\\
0.7 & (0,0) & 0.4 & 1/78 & 0.918 & 0.914 & 0.934 & 0.938 & 0.940 & 0.956 &
0.958 & 0.944\\
0.7 & (0,0) & 0.4 & 1/156 & 0.972 & 0.972 & 0.984 & 0.976 & 0.986 & 0.980 &
0.980 & 0.984\\\hline\hline
\vspace{0.02in} &  &  &  &  &  &  &  &  &  &  &
\end{tabular}
{\footnotesize \begin{minipage}{1\columnwidth} \footnotesize * See notes to Table 1.
\end{minipage}}\end{table}

\newpage

%TABLE 3 Test of $E(Z)=0$ %%%%%%%%%%%%%%%%%%%%%%%%%%%%%%%%%%%%%%%%%%%%%%%%%%%%%%%%%%%%%%%%%%%%%%%%%%%%%%%%%%%%%%%%%%%%%%
\begin{table}[ptb]
\caption{\textbf{Monte Carlo Experiments - Test of $E(Z_{k})=0$ *}}%
\label{tb3}%
\begin{tabular}
[c]{llllllllllll}
&  &  &  &  &  &  &  &  &  &  & \\
&  &  &  &  &  &  &  &  &  &  & \\\hline\hline
$\lambda_{\infty}$ & $(a, \beta)$ & $\sigma$ & $\Delta$ & $T1$ & $T2$ & $T3$ &
$T4$ & $T5$ & $T6$ & $T7$ & $T8$\\\hline
\multicolumn{12}{c}{\textbf{\emph{EMPIRICAL SIZE}}}\\\hline
\multicolumn{12}{c}{\textbf{\emph{Constant Intensity, $Z_{k}$ is $N(0.0,
\sigma^{2} )$}}}\\\hline
0.3 & (0,0) & 0.2 & 1/78 & 0.070 & 0.062 & 0.062 & 0.078 & 0.082 & 0.082 &
0.086 & 0.100\\
0.3 & (0,0) & 0.2 & 1/156 & 0.082 & 0.086 & 0.076 & 0.066 & 0.070 & 0.068 &
0.092 & 0.090\\
\multicolumn{12}{c}{\textbf{\emph{Self Excitement, $Z_{k}$ is $N(0.0,
\sigma^{2} )$}}}\\
0.3 & (5,4) & 0.2 & 1/78 & 0.060 & 0.056 & 0.068 & 0.080 & 0.060 & 0.078 &
0.080 & 0.082\\
0.3 & (5,4) & 0.2 & 1/156 & 0.088 & 0.088 & 0.090 & 0.094 & 0.094 & 0.078 &
0.094 & 0.082\\\hline
\multicolumn{12}{c}{\textbf{\emph{EMPIRICAL POWER}}}\\\hline
\multicolumn{12}{c}{\textbf{\emph{Constant Intensity, $Z_{k}$ is $Exp$(5)}}%
}\\\hline
0.3 & (0,0) & -- & 1/78 & 0.182 & 0.236 & 0.264 & 0.272 & 0.320 & 0.324 &
0.334 & 0.366\\
0.3 & (0,0) & -- & 1/156 & 0.858 & 0.880 & 0.928 & 0.934 & 0.964 & 0.972 &
0.970 & 0.982\\
0.5 & (0,0) & -- & 1/78 & 0.414 & 0.458 & 0.480 & 0.508 & 0.562 & 0.584 &
0.616 & 0.650\\
0.5 & (0,0) & -- & 1/156 & 0.998 & 0.998 & 0.998 & 1.000 & 1.000 & 1.000 &
1.000 & 1.000\\
0.7 & (0,0) & -- & 1/78 & 0.580 & 0.652 & 0.716 & 0.742 & 0.784 & 0.814 &
0.836 & 0.870\\
0.7 & (0,0) & -- & 1/156 & 1.000 & 1.000 & 1.000 & 1.000 & 1.000 & 1.000 &
1.000 & 1.000\\\hline
\multicolumn{12}{c}{\textbf{\emph{Constant Intensity, $Z_{k}$ is $N(0.5,
\sigma^{2} )$}}}\\\hline
0.3 & (0,0) & 0.2 & 1/78 & 0.138 & 0.174 & 0.170 & 0.190 & 0.208 & 0.220 &
0.208 & 0.226\\
0.3 & (0,0) & 0.2 & 1/156 & 0.676 & 0.718 & 0.728 & 0.778 & 0.810 & 0.868 &
0.872 & 0.886\\
0.5 & (0,0) & 0.2 & 1/78 & 0.274 & 0.310 & 0.324 & 0.352 & 0.394 & 0.428 &
0.428 & 0.454\\
0.5 & (0,0) & 0.2 & 1/156 & 0.954 & 0.966 & 0.978 & 0.986 & 0.996 & 0.998 &
0.998 & 1.000\\
0.7 & (0,0) & 0.2 & 1/78 & 0.396 & 0.450 & 0.502 & 0.552 & 0.596 & 0.624 &
0.642 & 0.694\\
0.7 & (0,0) & 0.2 & 1/156 & 1.000 & 1.000 & 1.000 & 1.000 & 1.000 & 1.000 &
1.000 & 1.000\\\hline
\multicolumn{12}{c}{\textbf{\emph{Self Excitement, $Z_{k}$ is $Exp$(5)}}%
}\\\hline
0.3 & (5,4) & -- & 1/78 & 0.650 & 0.664 & 0.674 & 0.690 & 0.694 & 0.704 &
0.708 & 0.716\\
0.3 & (5,4) & -- & 1/156 & 0.928 & 0.946 & 0.968 & 0.972 & 0.982 & 0.990 &
0.994 & 0.994\\
0.5 & (5,4) & -- & 1/78 & 0.836 & 0.846 & 0.854 & 0.872 & 0.872 & 0.874 &
0.894 & 0.904\\
0.5 & (5,4) & -- & 1/156 & 0.998 & 1.000 & 1.000 & 1.000 & 1.000 & 1.000 &
1.000 & 1.000\\
0.7 & (5,4) & -- & 1/78 & 0.950 & 0.954 & 0.964 & 0.970 & 0.972 & 0.978 &
0.976 & 0.980\\
0.7 & (5,4) & -- & 1/156 & 0.998 & 1.000 & 1.000 & 1.000 & 1.000 & 1.000 &
1.000 & 1.000\\\hline
\multicolumn{12}{c}{\textbf{\emph{Self Excitement, $Z_{k}$ is $N(0.5,
\sigma^{2} )$}}}\\\hline
0.3 & (5,4) & 0.2 & 1/78 & 0.602 & 0.624 & 0.628 & 0.624 & 0.644 & 0.638 &
0.642 & 0.658\\
0.3 & (5,4) & 0.2 & 1/156 & 0.852 & 0.870 & 0.878 & 0.886 & 0.906 & 0.940 &
0.944 & 0.958\\
0.5 & (5,4) & 0.2 & 1/78 & 0.784 & 0.804 & 0.812 & 0.820 & 0.810 & 0.828 &
0.828 & 0.844\\
0.5 & (5,4) & 0.2 & 1/156 & 0.980 & 0.988 & 0.992 & 0.998 & 1.000 & 1.000 &
1.000 & 1.000\\
0.7 & (5,4) & 0.2 & 1/78 & 0.916 & 0.928 & 0.922 & 0.936 & 0.942 & 0.946 &
0.960 & 0.956\\
0.7 & (5,4) & 0.2 & 1/156 & 0.996 & 0.994 & 0.998 & 1.000 & 1.000 & 1.000 &
1.000 & 1.000\\\hline\hline
\vspace{0.02in} &  &  &  &  &  &  &  &  &  &  &
\end{tabular}
{\footnotesize \begin{minipage}{1\columnwidth} \footnotesize * See notes to Table 1. Entries in the table are rejection frequencies for the
zero mean jump test, based on $t_{\mu_{Z},T,\Delta}=\sqrt{T}\frac
{\widehat{\mu}_{T,\Delta}^{Z}}{\widehat{\sigma}_{\mu_{Z}}}$.
\end{minipage}}\end{table}

\newpage

%TABLE 4 Self Excitement Test %%%%%%%%%%%%%%%%%%%%%%%%%%%%%%%%%%%%%%%%%%%%%%%%%%%%%%%%%%%%%%%%%%%%%%%%%%%%%%%%%%%%%%%%%%%%%%
\begin{table}[ptb]
\caption{\textbf{Monte Carlo Experiments - Self Excitement Test (Empirical
Size) *} }%
\label{tb4}%
\begin{tabular}
[c]{llllllllllll}
&  &  &  &  &  &  &  &  &  &  & \\
&  &  &  &  &  &  &  &  &  &  & \\\hline\hline
$\lambda_{\infty}$ & $(a, \beta)$ & $\sigma$ & $\Delta$ & $T1$ & $T2$ & $T3$ &
$T4$ & $T5$ & $T6$ & $T7$ & $T8$\\\hline
\multicolumn{12}{c}{\textbf{\emph{$Z_{k}$ is $Exp$(5)}}}\\\hline
0.3 & (0,0) & -- & 1/78 & 0.114 & 0.108 & 0.120 & 0.138 & 0.124 & 0.146 &
0.126 & 0.130\\
0.3 & (0,0) & -- & 1/156 & 0.092 & 0.094 & 0.080 & 0.090 & 0.082 & 0.092 &
0.080 & 0.076\\
0.5 & (0,0) & -- & 1/78 & 0.088 & 0.088 & 0.096 & 0.098 & 0.108 & 0.118 &
0.122 & 0.114\\
0.5 & (0,0) & -- & 1/156 & 0.070 & 0.076 & 0.068 & 0.070 & 0.088 & 0.078 &
0.076 & 0.068\\
0.7 & (0,0) & -- & 1/78 & 0.088 & 0.098 & 0.122 & 0.128 & 0.108 & 0.124 &
0.118 & 0.122\\
0.7 & (0,0) & -- & 1/156 & 0.076 & 0.076 & 0.084 & 0.074 & 0.090 & 0.078 &
0.082 & 0.086\\\hline
\multicolumn{12}{c}{\textbf{\emph{$Z_{k}$ is $N(0.5, \sigma^{2} )$}}}\\\hline
0.3 & (0,0) & 0.1 & 1/78 & 0.116 & 0.124 & 0.126 & 0.124 & 0.136 & 0.144 &
0.138 & 0.146\\
0.3 & (0,0) & 0.1 & 1/156 & 0.084 & 0.106 & 0.094 & 0.092 & 0.094 & 0.086 &
0.078 & 0.070\\
0.3 & (0,0) & 0.2 & 1/78 & 0.118 & 0.122 & 0.136 & 0.142 & 0.136 & 0.150 &
0.146 & 0.134\\
0.3 & (0,0) & 0.2 & 1/156 & 0.102 & 0.098 & 0.086 & 0.088 & 0.094 & 0.092 &
0.084 & 0.080\\
0.3 & (0,0) & 0.4 & 1/78 & 0.120 & 0.116 & 0.132 & 0.140 & 0.130 & 0.146 &
0.150 & 0.132\\
0.3 & (0,0) & 0.4 & 1/156 & 0.082 & 0.098 & 0.086 & 0.094 & 0.088 & 0.088 &
0.076 & 0.084\\
0.5 & (0,0) & 0.1 & 1/78 & 0.094 & 0.098 & 0.116 & 0.132 & 0.134 & 0.134 &
0.142 & 0.136\\
0.5 & (0,0) & 0.1 & 1/156 & 0.082 & 0.084 & 0.078 & 0.090 & 0.100 & 0.086 &
0.088 & 0.074\\
0.5 & (0,0) & 0.2 & 1/78 & 0.094 & 0.094 & 0.102 & 0.122 & 0.116 & 0.126 &
0.122 & 0.124\\
0.5 & (0,0) & 0.2 & 1/156 & 0.086 & 0.086 & 0.088 & 0.088 & 0.092 & 0.080 &
0.076 & 0.066\\
0.5 & (0,0) & 0.4 & 1/78 & 0.110 & 0.096 & 0.120 & 0.116 & 0.118 & 0.134 &
0.124 & 0.122\\
0.5 & (0,0) & 0.4 & 1/156 & 0.096 & 0.090 & 0.078 & 0.092 & 0.092 & 0.088 &
0.090 & 0.084\\
0.7 & (0,0) & 0.1 & 1/78 & 0.104 & 0.112 & 0.128 & 0.132 & 0.130 & 0.132 &
0.104 & 0.112\\
0.7 & (0,0) & 0.1 & 1/156 & 0.086 & 0.086 & 0.078 & 0.082 & 0.094 & 0.084 &
0.094 & 0.098\\
0.7 & (0,0) & 0.2 & 1/78 & 0.092 & 0.102 & 0.116 & 0.132 & 0.118 & 0.126 &
0.120 & 0.126\\
0.7 & (0,0) & 0.2 & 1/156 & 0.072 & 0.078 & 0.074 & 0.084 & 0.084 & 0.082 &
0.092 & 0.086\\
0.7 & (0,0) & 0.4 & 1/78 & 0.076 & 0.094 & 0.092 & 0.104 & 0.110 & 0.122 &
0.104 & 0.098\\
0.7 & (0,0) & 0.4 & 1/156 & 0.078 & 0.068 & 0.086 & 0.072 & 0.092 & 0.092 &
0.098 & 0.084\\\hline\hline
\vspace{0.02in} &  &  &  &  &  &  &  &  &  &  &
\end{tabular}
{\footnotesize \begin{minipage}{1\columnwidth} \footnotesize * See notes to Table 1. Entries in the table are rejection frequencies for the
\textquotedblleft self excitement test\textquotedblright\ of the null of\ no
jump path dependence, based on $S_{T^{+},\Delta}^{\beta}=\max\left\{
0,t_{\beta,T^{+},\Delta}\right\}.$
\end{minipage}}\end{table}

\newpage

%TABLE 5 Self Excitement Test %%%%%%%%%%%%%%%%%%%%%%%%%%%%%%%%%%%%%%%%%%%%%%%%%%%%%%%%%%%%%%%%%%%%%%%%%%%%%%%%%%%%%%%%%%%%%%
\begin{table}[ptb]
\caption{\textbf{Monte Carlo Experiments - Self Excitement Test (Empirical
Power) *}}%
\label{tb5}%
\begin{tabular}
[c]{llllllllllll}
&  &  &  &  &  &  &  &  &  &  & \\
&  &  &  &  &  &  &  &  &  &  & \\\hline\hline
$\lambda_{\infty}$ & $(a, \beta)$ & $\sigma$ & $\Delta$ & $T1$ & $T2$ & $T3$ &
$T4$ & $T5$ & $T6$ & $T7$ & $T8$\\\hline
\multicolumn{12}{c}{\textbf{\emph{$Z_{k}$ is $Exp$(5)}}}\\\hline
0.3 & (3,2) & -- & 1/78 & 0.570 & 0.568 & 0.564 & 0.560 & 0.558 & 0.572 &
0.560 & 0.554\\
0.3 & (3,2) & -- & 1/156 & 0.566 & 0.558 & 0.556 & 0.552 & 0.554 & 0.558 &
0.548 & 0.554\\
0.3 & (5,4) & -- & 1/78 & 0.418 & 0.414 & 0.426 & 0.420 & 0.424 & 0.430 &
0.420 & 0.422\\
0.3 & (5,4) & -- & 1/156 & 0.430 & 0.432 & 0.422 & 0.428 & 0.436 & 0.430 &
0.428 & 0.424\\
0.3 & (7,5) & -- & 1/78 & 0.332 & 0.324 & 0.338 & 0.342 & 0.322 & 0.336 &
0.316 & 0.324\\
0.3 & (7,5) & - & 1/156 & 0.318 & 0.320 & 0.324 & 0.314 & 0.324 & 0.318 &
0.300 & 0.310\\
0.5 & (3,2) & -- & 1/78 & 0.704 & 0.708 & 0.704 & 0.706 & 0.706 & 0.696 &
0.700 & 0.700\\
0.5 & (3,2) & -- & 1/156 & 0.704 & 0.704 & 0.706 & 0.708 & 0.700 & 0.710 &
0.710 & 0.704\\
0.5 & (5,4) & -- & 1/78 & 0.550 & 0.548 & 0.544 & 0.534 & 0.540 & 0.536 &
0.520 & 0.524\\
0.5 & (5,4) & -- & 1/156 & 0.520 & 0.520 & 0.516 & 0.516 & 0.530 & 0.530 &
0.526 & 0.524\\
0.5 & (7,5) & -- & 1/78 & 0.446 & 0.424 & 0.438 & 0.444 & 0.434 & 0.434 &
0.430 & 0.436\\
0.5 & (7,5) & -- & 1/156 & 0.430 & 0.432 & 0.452 & 0.436 & 0.432 & 0.432 &
0.412 & 0.410\\
0.7 & (3,2) & -- & 1/78 & 0.788 & 0.782 & 0.780 & 0.770 & 0.776 & 0.772 &
0.768 & 0.772\\
0.7 & (3,2) & -- & 1/156 & 0.780 & 0.780 & 0.772 & 0.770 & 0.762 & 0.752 &
0.760 & 0.750\\
0.7 & (5,4) & -- & 1/78 & 0.644 & 0.636 & 0.648 & 0.636 & 0.620 & 0.616 &
0.622 & 0.604\\
0.7 & (5,4) & -- & 1/156 & 0.618 & 0.612 & 0.614 & 0.626 & 0.616 & 0.606 &
0.608 & 0.620\\
0.7 & (7,5) & -- & 1/78 & 0.510 & 0.522 & 0.500 & 0.498 & 0.492 & 0.500 &
0.486 & 0.492\\
0.7 & (7,5) & -- & 1/156 & 0.486 & 0.474 & 0.468 & 0.456 & 0.446 & 0.438 &
0.446 & 0.444\\\hline
\multicolumn{12}{c}{\textbf{\emph{$Z_{k}$ is $N(0.5, \sigma^{2} )$}}}\\\hline
0.3 & (3,2) & 0.2 & 1/78 & 0.560 & 0.568 & 0.566 & 0.560 & 0.552 & 0.562 &
0.548 & 0.548\\
0.3 & (3,2) & 0.2 & 1/156 & 0.534 & 0.540 & 0.534 & 0.542 & 0.540 & 0.544 &
0.540 & 0.542\\
0.3 & (5,4) & 0.2 & 1/78 & 0.388 & 0.394 & 0.398 & 0.382 & 0.388 & 0.392 &
0.378 & 0.372\\
0.3 & (5,4) & 0.2 & 1/156 & 0.396 & 0.378 & 0.382 & 0.384 & 0.402 & 0.384 &
0.380 & 0.380\\
0.3 & (7,5) & 0.2 & 1/78 & 0.292 & 0.308 & 0.302 & 0.304 & 0.286 & 0.292 &
0.284 & 0.284\\
0.3 & (7,5) & 0.2 & 1/156 & 0.284 & 0.278 & 0.272 & 0.278 & 0.268 & 0.274 &
0.268 & 0.262\\
0.5 & (3,2) & 0.2 & 1/78 & 0.676 & 0.686 & 0.666 & 0.674 & 0.668 & 0.658 &
0.642 & 0.640\\
0.5 & (3,2) & 0.2 & 1/156 & 0.654 & 0.652 & 0.638 & 0.632 & 0.630 & 0.624 &
0.634 & 0.626\\
0.5 & (5,4) & 0.2 & 1/78 & 0.506 & 0.510 & 0.502 & 0.480 & 0.464 & 0.462 &
0.460 & 0.442\\
0.5 & (5,4) & 0.2 & 1/156 & 0.476 & 0.464 & 0.458 & 0.450 & 0.450 & 0.452 &
0.454 & 0.442\\
0.5 & (7,5) & 0.2 & 1/78 & 0.386 & 0.388 & 0.380 & 0.372 & 0.362 & 0.356 &
0.346 & 0.344\\
0.5 & (7,5) & 0.2 & 1/156 & 0.380 & 0.390 & 0.384 & 0.386 & 0.390 & 0.364 &
0.358 & 0.356\\
0.7 & (3,2) & 0.2 & 1/78 & 0.706 & 0.692 & 0.692 & 0.686 & 0.688 & 0.694 &
0.682 & 0.668\\
0.7 & (3,2) & 0.2 & 1/156 & 0.708 & 0.696 & 0.694 & 0.668 & 0.674 & 0.662 &
0.654 & 0.654\\
0.7 & (5,4) & 0.2 & 1/78 & 0.586 & 0.576 & 0.566 & 0.572 & 0.560 & 0.556 &
0.538 & 0.530\\
0.7 & (5,4) & 0.2 & 1/156 & 0.560 & 0.562 & 0.562 & 0.556 & 0.544 & 0.532 &
0.530 & 0.522\\
0.7 & (7,5) & 0.2 & 1/78 & 0.428 & 0.432 & 0.428 & 0.416 & 0.406 & 0.402 &
0.402 & 0.396\\
0.7 & (7,5) & 0.2 & 1/156 & 0.420 & 0.418 & 0.416 & 0.420 & 0.410 & 0.404 &
0.390 & 0.384\\\hline\hline
\vspace{0.02in} &  &  &  &  &  &  &  &  &  &  &
\end{tabular}
{\footnotesize \begin{minipage}{1\columnwidth} \footnotesize * See notes to Table 4.\end{minipage}}%
\end{table}

\newpage


\end{document}