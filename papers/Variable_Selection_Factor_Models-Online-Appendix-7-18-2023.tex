%2multibyte Version: 5.50.0.2960 CodePage: 936

\documentclass[10pt]{article}
%%%%%%%%%%%%%%%%%%%%%%%%%%%%%%%%%%%%%%%%%%%%%%%%%%%%%%%%%%%%%%%%%%%%%%%%%%%%%%%%%%%%%%%%%%%%%%%%%%%%%%%%%%%%%%%%%%%%%%%%%%%%%%%%%%%%%%%%%%%%%%%%%%%%%%%%%%%%%%%%%%%%%%%%%%%%%%%%%%%%%%%%%%%%%%%%%%%%%%%%%%%%%%%%%%%%%%%%%%%%%%%%%%%%%%%%%%%%%%%%%%%%%%%%%%%%
\usepackage{amssymb}
\usepackage{amsfonts}
\usepackage{amsmath}
\usepackage{geometry}
\usepackage[onehalfspacing]{setspace}

\setcounter{MaxMatrixCols}{10}
%TCIDATA{OutputFilter=LATEX.DLL}
%TCIDATA{Version=5.50.0.2960}
%TCIDATA{Codepage=936}
%TCIDATA{<META NAME="SaveForMode" CONTENT="1">}
%TCIDATA{BibliographyScheme=Manual}
%TCIDATA{Created=Sunday, July 18, 2004 16:10:34}
%TCIDATA{LastRevised=Tuesday, July 18, 2023 19:00:44}
%TCIDATA{<META NAME="GraphicsSave" CONTENT="32">}
%TCIDATA{<META NAME="DocumentShell" CONTENT="General\Blank Document">}
%TCIDATA{CSTFile=article.cst}
%TCIDATA{PageSetup=72,72,72,72,0}
%TCIDATA{AllPages=
%F=36,\PARA{038<p type="texpara" tag="Body Text" >\hfill \thepage}
%}


\newtheorem{theorem}{Theorem}
\newtheorem{acknowledgement}[theorem]{Acknowledgement}
\newtheorem{algorithm}[theorem]{Algorithm}
\newtheorem{axiom}[theorem]{Axiom}
\newtheorem{case}[theorem]{Case}
\newtheorem{claim}[theorem]{Claim}
\newtheorem{conclusion}[theorem]{Conclusion}
\newtheorem{condition}[theorem]{Condition}
\newtheorem{conjecture}[theorem]{Conjecture}
\newtheorem{corollary}[theorem]{Corollary}
\newtheorem{criterion}[theorem]{Criterion}
\newtheorem{definition}[theorem]{Definition}
\newtheorem{example}[theorem]{Example}
\newtheorem{exercise}[theorem]{Exercise}
\newtheorem{lemma}[theorem]{Lemma}
\newtheorem{notation}[theorem]{Notation}
\newtheorem{problem}[theorem]{Problem}
\newtheorem{proposition}[theorem]{Proposition}
\newtheorem{remark}[theorem]{Remark}
\newtheorem{solution}[theorem]{Solution}
\newtheorem{summary}[theorem]{Summary}
\newenvironment{proof}[1][Proof]{\noindent\textbf{#1.} }{\ \rule{0.5em}{0.5em}}
\renewcommand{\baselinestretch}{1.3} 
\textwidth=6.8in
\textheight=8.7in
\oddsidemargin=0in
\evensidemargin=0in
\topmargin=0in
\baselineskip=10pt
\linespread{1.0}
\input{tcilatex}
\geometry{left=1in,right=1in,top=1.25in,bottom=1.25in}

\begin{document}


\begin{center}
\vspace{0.22in}{\Large {\ }}

{\Large {Online Appendix: Selecting the Relevant Variables for Factor
Estimation in FAVAR Models, With An Application to Forecasting the Yield
Curve$^{\ast }$}}

\bigskip \vspace{0.5in}

John C. Chao$^{1},$ Kaiwen Qiu$^{2}$, and Norman R. Swanson$^{2}$

\medskip

$^{1}$University of Maryland and $^{2}$Rutgers University

\medskip \vspace{0.22in}

July 18, 2023

\bigskip \bigskip

Abstract
\end{center}

\begin{spacing}{1.01}
\noindent This Online Appendix contains additional supporting lemmas whose results are
used in the proofs of Theorems 1 and 2 of the main paper as well as Lemmas
A1-A2 of that paper.
\end{spacing}

\bigskip \bigskip \bigskip

\noindent \textit{Keywords: }Factor analysis, forecasting, variable
selection.

\noindent

\bigskip \bigskip

{\footnotesize 
\begin{spacing}{1.01}
\noindent $^{\ast }$\textit{Corresponding Author:} Norman R. Swanson, Department of Economics, 9500 Hamilton Street, Rutgers University,
nswanson@econ.rutgers.edu.

\medskip

\noindent John C. Chao, Department of Economics, 7343 Preinkert
Drive, University of Maryland, jcchao@umd.edu. Kaiwen Qiu, Department of Economics, 9500 Hamilton Street, Rutgers University,
kq60@econ.rutgers.edu. The authors are grateful to Simon Freyaldenhoven,
Yuan Liao, Minchul Shin, Jim Stock, Timothy Vogelsang, Endong Wang, Xiye Yang, Bo Zhou and seminar participants at the University of Glasgow, the University of California Riverside, the Federal
Reserve Bank of Pihadelphia, the 2022 Summer Econometrics Society Meetings, the 2022 International Association of Applied Econometrics Association meetings, the DC-MD-VA Econometrics Workshop, and the 2022 NBER-NSF Time Series Conference for useful comments received on earlier versions
of this paper. Chao thanks the University of Maryland for research support.
\end{spacing}}

\newpage

\noindent \noindent \setcounter{page}{1}

\section*{Additional Supporting Lemmas and Their Proofs}

In this Online Appendix, we state and prove a number of additional
supporting lemmas. The results given by these lemmas are used to prove
Theorems 1 and 2 as well as Lemmas A1-A2 of the main paper and, thus, help
to deliver the main results of the paper.

\bigskip

\noindent \textbf{Lemma OA-1: }Let $a$ and $\theta $ be real numbers such
that $a>0$ and $\theta \geq 1$. Also, let $G$ be a finite non-negative
integer. Then, \textbf{\ }%
\begin{equation*}
\dsum\limits_{m=1}^{\infty }m^{G}\exp \left\{ -am^{{\large \theta }}\right\}
<\infty
\end{equation*}

\noindent \textbf{Proof of Lemma OA-1: }By the integral test,%
\begin{equation*}
\dsum\limits_{m=1}^{\infty }m^{G}\exp \left\{ -am^{{\large \theta }}\right\}
<\infty \text{ for finite non-negative integer }G
\end{equation*}%
if%
\begin{equation*}
\dint\nolimits_{1}^{\infty }x^{G}\exp \left\{ -ax^{{\large \theta }}\right\}
dx<\infty \text{ for finite non-negative integer }G
\end{equation*}%
In addition, note that since, by assumption, $a>0$ and $\theta \geq 1$, we
have%
\begin{equation*}
\dint\nolimits_{1}^{\infty }x^{G}\exp \left\{ -ax^{{\large \theta }}\right\}
dx\leq \dint\nolimits_{1}^{\infty }x^{G}\exp \left\{ -ax\right\} dx
\end{equation*}

We will first consider the case where $G=0$. In this case, note that%
\begin{equation*}
\dint\nolimits_{1}^{\infty }x^{0}\exp \left\{ -ax\right\}
dx=\dint\nolimits_{1}^{\infty }\exp \left\{ -ax\right\} dx
\end{equation*}%
Let $u=-ax$, so that $-\frac{du}{a}=dx$; and we have%
\begin{eqnarray}
\dint\nolimits_{1}^{\infty }\exp \left\{ -ax\right\} dx &=&-\frac{1}{a}%
\dint\nolimits_{-a}^{-\infty }\exp \left\{ u\right\} du  \notag \\
&=&\frac{1}{a}\dint\nolimits_{-\infty }^{-a}\exp \left\{ u\right\} du  \notag
\\
&=&\frac{\exp \left\{ -a\right\} }{a}  \notag \\
&<&\infty \text{ for any }a>0\text{.}  \label{G=0 Case}
\end{eqnarray}

Next, consider the case where $G$ is an integer such that $G\geq 1$. Here,
we will show that%
\begin{equation*}
\dint\nolimits_{1}^{\infty }x^{G}\exp \left\{ -ax\right\} dx=\left[ \frac{1}{%
a}+\dsum\limits_{k=1}^{G}\frac{1}{a}\left( \dprod\limits_{j=0}^{k-1}\frac{G-j%
}{a}\right) \right] \exp \left\{ -a\right\} <\infty \text{ }
\end{equation*}%
using mathematical induction. To proceed, first consider the case where $G=1$%
. Let 
\begin{eqnarray*}
u &=&x,\text{ }du=dx \\
dv &=&\exp \left\{ -ax\right\} dx,\text{ }v=-\frac{1}{a}\exp \left\{
-ax\right\} \text{;}
\end{eqnarray*}%
and making use of integration-by-parts, we have%
\begin{eqnarray*}
\dint\nolimits_{1}^{\infty }x\exp \left\{ -ax\right\} dx &=&\left. -\frac{x}{%
a}\exp \left\{ -ax\right\} \right\vert _{1}^{\infty
}+\dint\nolimits_{1}^{\infty }\frac{1}{a}\exp \left\{ -ax\right\} dx \\
&=&\frac{1}{a}\exp \left\{ -a\right\} -\left. \frac{1}{a^{2}}\exp \left\{
-ax\right\} \right\vert _{1}^{\infty } \\
&=&\frac{1}{a}\exp \left\{ -a\right\} +\frac{1}{a^{2}}\exp \left\{ -a\right\}
\\
&=&\left( \frac{1}{a}+\frac{1}{a^{2}}\right) \exp \left\{ -a\right\} \\
&=&\left\{ \frac{1}{a}+\dsum\limits_{k=1}^{1}\frac{1}{a}\left(
\dprod\limits_{j=0}^{k-1}\frac{1-j}{a}\right) \right\} \exp \left\{
-a\right\} <\infty
\end{eqnarray*}%
Next, for $G=2$, let 
\begin{eqnarray*}
u &=&x^{2},\text{ }du=2xdx \\
dv &=&\exp \left\{ -ax\right\} dx,\text{ }v=-\frac{1}{a}\exp \left\{
-ax\right\} \text{;}
\end{eqnarray*}%
and we again make use of integration-by-parts to obtain%
\begin{eqnarray*}
\dint\nolimits_{1}^{\infty }x^{2}\exp \left\{ -ax\right\} dx &=&\left. -%
\frac{x^{2}}{a}\exp \left\{ -ax\right\} \right\vert _{1}^{\infty }+\frac{2}{a%
}\dint\nolimits_{1}^{\infty }x\exp \left\{ -ax\right\} dx \\
&=&\frac{1}{a}\exp \left\{ -a\right\} +\frac{2}{a}\left( \frac{1}{a}+\frac{1%
}{a^{2}}\right) \exp \left\{ -a\right\} \\
&=&\frac{1}{a}\exp \left\{ -a\right\} +2\left( \frac{1}{a^{2}}+\frac{1}{a^{3}%
}\right) \exp \left\{ -a\right\} \\
&=&\left( \frac{1}{a}+\frac{2}{a^{2}}+\frac{2}{a^{3}}\right) \exp \left\{
-a\right\} \\
&=&\left[ \frac{1}{a}+\dsum\limits_{k=1}^{2}\frac{1}{a}\left(
\dprod\limits_{j=0}^{k-1}\frac{2-j}{a}\right) \right] \exp \left\{ -a\right\}
\\
&<&\infty
\end{eqnarray*}%
Now, suppose that, for some $G\geq 2$,%
\begin{equation*}
\dint\nolimits_{1}^{\infty }x^{G-1}\exp \left\{ -ax\right\} dx=\left[ \frac{1%
}{a}+\dsum\limits_{k=1}^{G-1}\frac{1}{a}\left( \dprod\limits_{j=0}^{k-1}%
\frac{G-1-j}{a}\right) \right] \exp \left\{ -a\right\} \text{;}
\end{equation*}%
then, let 
\begin{eqnarray*}
u &=&x^{G},\text{ }du=Gx^{G-1}dx \\
dv &=&\exp \left\{ -ax\right\} dx,\text{ }v=-\frac{1}{a}\exp \left\{
-ax\right\} \text{;}
\end{eqnarray*}%
and, using integration-by-parts, we have%
\begin{eqnarray}
\dint\nolimits_{1}^{\infty }x^{G}\exp \left\{ -ax\right\} dx &=&\left. -%
\frac{x^{G}}{a}\exp \left\{ -ax\right\} \right\vert _{1}^{\infty }+\frac{G}{a%
}\dint\nolimits_{1}^{\infty }x^{G-1}\exp \left\{ -ax\right\} dx  \notag \\
&=&\frac{1}{a}\exp \left\{ -a\right\} +\frac{G}{a}\left[ \frac{1}{a}%
+\dsum\limits_{k=1}^{G-1}\frac{1}{a}\left( \dprod\limits_{j=0}^{k-1}\frac{%
G-1-j}{a}\right) \right] \exp \left\{ -a\right\}  \notag \\
&=&\frac{1}{a}\exp \left\{ -a\right\} +\left[ \frac{G}{a^{2}}%
+\dsum\limits_{k=1}^{G-1}\frac{1}{a}\frac{G}{a}\left(
\dprod\limits_{j=0}^{k-1}\frac{G-\left( j+1\right) }{a}\right) \right] \exp
\left\{ -a\right\}  \notag \\
&=&\left\{ \frac{1}{a}+\frac{G}{a^{2}}+\frac{1}{a}\frac{G}{a}\left( \frac{G-1%
}{a}\right) +\frac{1}{a}\frac{G}{a}\left( \frac{G-1}{a}\right) \left( \frac{%
G-2}{a}\right) \right.  \notag \\
&&\left. +\cdot \cdot \cdot +\frac{1}{a}\frac{G}{a}\left( \frac{G-1}{a}%
\right) \left( \frac{G-2}{a}\right) \times \cdot \cdot \cdot \times \left( 
\frac{1}{a}\right) \right\} \exp \left\{ -a\right\}  \notag \\
&=&\left\{ \frac{1}{a}+\dsum\limits_{k=1}^{G}\frac{1}{a}\left(
\dprod\limits_{j=0}^{k-1}\frac{G-j}{a}\right) \right\} \exp \left\{
-a\right\}  \notag \\
&<&\text{ }\infty \text{.}  \label{G>=1 Case}
\end{eqnarray}

In view of expressions (\ref{G=0 Case}) and (\ref{G>=1 Case}), it then
follows by the integral test for series convergence that 
\begin{equation*}
\dsum\limits_{m=1}^{\infty }m^{G}\exp \left\{ -am^{{\large \theta }}\right\}
<\infty \text{ }
\end{equation*}%
for any finite non-negative integer $G$ and for any constants $a$ and $%
\theta $ such that $a>0$ and $\theta \geq 1$. $\square $

\medskip

\noindent \textbf{Lemma OA-2: }Let $\left\{ V_{t}\right\} $ be a sequence of
random variables (or random vectors) defined on some probability space $%
\left( \Omega ,\mathcal{F},P\right) $, and let%
\begin{equation*}
X_{t}=g\left( V_{t},V_{t-1},...,V_{t-\varkappa }\right)
\end{equation*}%
be a measurable function for some finite positive integer $\varkappa $. In
addition, defne $\mathcal{G}_{-\infty }^{t}=\sigma \left(
....,X_{t-1},X_{t}\right) $, $\mathcal{G}_{t+m}^{\infty }=\sigma \left(
X_{t+m},X_{t+m+1},....\right) $, $\mathcal{F}_{-\infty }^{t}=\sigma \left(
....,V_{t-1},V_{t}\right) $, and

\noindent $\mathcal{F}_{t+m-\varkappa }^{\infty }=\sigma \left(
V_{t+m-\varkappa },V_{t+m+1-\varkappa },....\right) $. Under this setting,
the following results hold.

\begin{enumerate}
\item[(a)] Let 
\begin{eqnarray*}
\beta _{V,m-\varkappa } &=&\sup_{t}\beta \left( \mathcal{F}_{-\infty }^{t},%
\mathcal{F}_{t+m-\varkappa }^{\infty }\right) =\sup_{t}E\left[ \sup \left\{
\left\vert P\left( B|\mathcal{F}_{-\infty }^{t}\right) -P\left( B\right)
\right\vert :B\in \mathcal{F}_{t+m-\varkappa }^{\infty }\right\} \right] , \\
\beta _{X,m} &=&\sup_{t}\beta \left( \mathcal{G}_{-\infty }^{t},\mathcal{G}%
_{t+m}^{\infty }\right) =\sup_{t}E\left[ \sup \left\{ \left\vert P\left( H|%
\mathcal{G}_{-\infty }^{t}\right) -P\left( H\right) \right\vert :H\in 
\mathcal{G}_{t+m}^{\infty }\right\} \right] \text{.}
\end{eqnarray*}%
If $\left\{ V_{t}\right\} $ is $\beta $-mixing with 
\begin{equation*}
\beta _{V,m-\varkappa }\leq \overline{C}_{1}\exp \left\{ -C_{2}\left(
m-\varkappa \right) \right\} \text{ }
\end{equation*}%
for all $m\geq \varkappa $ and for some positive constants $\overline{C}_{1}$
and $C_{2}$; then $X_{t}$ is also $\beta $-mixing with $\beta $-mixing
coefficient satisfying%
\begin{equation*}
\beta _{X,m}\leq C_{1}\exp \left\{ -C_{2}m\right\} \text{ for all }m\geq
\varkappa \text{,}
\end{equation*}%
where $C_{1}$ is a positive constant such that $C_{1}\geq \overline{C}%
_{1}\exp \left\{ C_{2}\varkappa \right\} $.

\item[(b)] Let 
\begin{eqnarray*}
\alpha _{V,m-\varkappa } &=&\sup_{t}\alpha \left( \mathcal{F}_{-\infty }^{t},%
\mathcal{F}_{t+m-\varkappa }^{\infty }\right) =\sup_{t}\sup_{G\in \mathcal{F}%
_{-\infty }^{t},H\in \mathcal{F}_{t+m-\varkappa }^{\infty }}\left\vert
P\left( G\cap H\right) -P\left( G\right) P\left( H\right) \right\vert , \\
\alpha _{X,m} &=&\sup_{t}\alpha \left( \mathcal{G}_{-\infty }^{t},\mathcal{G}%
_{t+m}^{\infty }\right) =\sup_{t}\sup_{G\in \mathcal{G}_{-\infty }^{t},H\in 
\mathcal{G}_{t+m}^{\infty }}\left\vert P\left( G\cap H\right) -P\left(
G\right) P\left( H\right) \right\vert
\end{eqnarray*}%
If $\left\{ V_{t}\right\} $ is $\alpha $-mixing with 
\begin{equation*}
\alpha _{V,m-\varkappa }\leq \overline{C}_{1}\exp \left\{ -C_{2}\left(
m-\varkappa \right) \right\} \text{ }
\end{equation*}%
for all $m\geq \varkappa $ and for some positive constants $\overline{C}_{1}$
and $C_{2}$; then $X_{t}$ is also $\alpha $-mixing with $\alpha $-mixing
coefficient satisfying%
\begin{equation*}
\alpha _{X,m}\leq C_{1}\exp \left\{ -C_{2}m\right\} \text{ for all }m\geq
\varkappa \text{,}
\end{equation*}%
where $C_{1}$ is a positive constant such that $C_{1}\geq \overline{C}%
_{1}\exp \left\{ C_{2}\varkappa \right\} $.
\end{enumerate}

\noindent

\noindent \textbf{Proof of Lemma OA-2: }

To show part (a), note first that it is well known that%
\begin{eqnarray*}
\beta _{X,m} &=&\sup_{t}E\left[ \sup \left\{ \left\vert P\left( H|\mathcal{G}%
_{-\infty }^{t}\right) -P\left( H\right) \right\vert :H\in \mathcal{G}%
_{t+m}^{\infty }\right\} \right] \\
&=&\sup_{t}\left\{ \frac{1}{2}\sup
\dsum\limits_{i=1}^{I}\dsum\limits_{j=1}^{J}\left\vert P\left( G_{i}\cap
H_{j}\right) -P\left( G_{i}\right) P\left( H_{j}\right) \right\vert \right\}
\end{eqnarray*}%
where the second supremum on the last line above is taken over all pairs of
finite partitions $\left\{ G_{1},...,G_{I}\right\} $ and $\left\{
H_{1},...,H_{J}\right\} $ of $\Omega $ such that $G_{i}\in \mathcal{G}%
_{-\infty }^{t}$ for $i=1,...,I$ and $H_{j}\in \mathcal{G}_{t+m}^{\infty }$
for $j=1,....,J$. See, for example, Borovkova, Burton, and Dehling (2001).
Similarly, \ 
\begin{eqnarray*}
\beta _{V,m-\varkappa } &=&\sup_{t}E\left[ \sup \left\{ \left\vert P\left( B|%
\mathcal{F}_{-\infty }^{t}\right) -P\left( B\right) \right\vert :B\in 
\mathcal{F}_{t+m-\varkappa }^{\infty }\right\} \right] \\
&=&\sup_{t}\left\{ \frac{1}{2}\sup
\dsum\limits_{i=1}^{L}\dsum\limits_{j=1}^{M}\left\vert P\left( A_{i}\cap
B_{j}\right) -P\left( A_{i}\right) P\left( B_{j}\right) \right\vert \right\}
\end{eqnarray*}%
where, similar to the definition of $\beta _{X,m}$, the second supremum on
the last line above is taken over all pairs of finite partitions $\left\{
A_{1},...,A_{L}\right\} $ and $\left\{ B_{1},...,B_{M}\right\} $ of $\Omega $
such that $A_{i}\in \mathcal{F}_{-\infty }^{t}$ for $i=1,...,I$ and $%
B_{j}\in \mathcal{F}_{t+m-\varkappa }^{\infty }$ for $j=1,....,M$. \
Moreover, since $X_{t}$ is measurable on any $\sigma $-field on which $%
V_{t},V_{t-1},...,V_{t-\varkappa }$ are measurable, we also have%
\begin{equation*}
\mathcal{G}_{-\infty }^{t}=\sigma \left( ....,X_{t-1},X_{t}\right) \subseteq
\sigma \left( ....,V_{t-1},V_{t}\right) =\mathcal{F}_{-\infty }^{t}
\end{equation*}%
and%
\begin{equation*}
\mathcal{G}_{t+m}^{\infty }=\sigma \left( X_{t+m},X_{t+m+1},....\right)
\subseteq \sigma \left( V_{t+m-\varkappa },V_{t+m+1-\varkappa },....\right) =%
\mathcal{F}_{t+m-\varkappa }^{\infty }\text{.}
\end{equation*}%
It, thus, follows that, for all $m\geq \varkappa $,%
\begin{eqnarray*}
\beta _{X,m} &=&\sup_{t}\left\{ \frac{1}{2}\sup
\dsum\limits_{i=1}^{I}\dsum\limits_{j=1}^{J}\left\vert P\left( G_{i}\cap
H_{j}\right) -P\left( G_{i}\right) P\left( H_{j}\right) \right\vert \right\}
\\
&\leq &\sup_{t}\left\{ \frac{1}{2}\sup
\dsum\limits_{i=1}^{L}\dsum\limits_{j=1}^{M}\left\vert P\left( A_{i}\cap
B_{j}\right) -P\left( A_{i}\right) P\left( B_{j}\right) \right\vert \right\}
\\
&=&\beta _{V,m-\varkappa } \\
&\leq &\overline{C}_{1}\exp \left\{ -C_{2}\left( m-\varkappa \right) \right\}
\\
&=&\overline{C}_{1}\exp \left\{ C_{2}\varkappa \right\} \exp \left\{
-C_{2}m\right\} \\
&\leq &C_{1}\exp \left\{ -C_{2}m\right\} \text{ }
\end{eqnarray*}%
for some positive constant $C_{1}\geq \overline{C}_{1}\exp \left\{
C_{2}\varkappa \right\} $ which exists given that $\varkappa $ is fixed.
Moreover, we have%
\begin{equation*}
\beta _{X,m}\leq C_{1}\exp \left\{ -C_{2}m\right\} \rightarrow 0\text{ as }%
m\rightarrow \infty \text{,}
\end{equation*}%
which establishes the required result for part (a).

Part (b) can be shown in a manner similar to part (a), so to avoid
redundancy, we do not include an explicit proof here. $\square $

\medskip

\noindent \textbf{Remark: }Note that part (b) of Lemma OA-2 is similar to a
result given in Theorem 14.1 of Davidson (1994) but adapted to suit our
situation and our notatons here. Indeed, parts (a) and (b) of this lemma are
both well-known results in the probability literature. We have chosen to
state these results explicitly here only so that we can more easily refer to
them in the proofs of some of our other results.

\bigskip

\noindent \textbf{Lemma OA-3: }Let $\left\{ X_{t}\right\} $ be a sequence of
random variables that is $\alpha $-mixing. Let $p>1$ and $r\geq p/\left(
p-1\right) $, and let $q=\max \left\{ p,r\right\} $. Suppose that, for all $%
t $, 
\begin{equation*}
\left\Vert X_{t}\right\Vert _{q}=\left( E\left\vert X_{t}\right\vert
^{q}\right) ^{\frac{{\large 1}}{{\large q}}}<\infty \text{ }
\end{equation*}%
Then, 
\begin{equation*}
\left\vert Cov\left( X_{t},X_{t+m}\right) \right\vert \leq 2\left(
2^{1-1/p}+1\right) \alpha _{m}^{1-1/p-1/r}\left\Vert X_{t}\right\Vert
_{p}\left\Vert X_{t+m}\right\Vert _{r}
\end{equation*}%
where%
\begin{equation*}
\alpha _{m}=\sup_{t}\alpha \left( \mathcal{F}_{-\infty }^{t},\mathcal{F}%
_{t+m}^{\infty }\right) =\sup_{G\in \mathcal{F}_{-\infty }^{t},H\in \mathcal{%
F}_{t+m}^{\infty }}\left\vert P\left( G\cap H\right) -P\left( G\right)
P\left( H\right) \right\vert \text{.}
\end{equation*}

\noindent \textbf{Remark: }This is Corollary 14.3 of Davidson (1994). For a
proof, see pages 212-213 of Davidson (1994).

\medskip

\noindent \textbf{Lemma OA-4: }Suppose that Assumption 2-3 hold. Let $\tau
_{1}=\left\lfloor T_{0}^{\alpha _{{\large 1}}}\right\rfloor $, where $%
1>\alpha _{1}>0$ and $T_{0}=T-p+1$. Then,

\begin{enumerate}
\item[(a)] 
\begin{equation*}
\frac{1}{\tau _{1}^{2}}\dsum\limits_{\substack{ g,h=\left( r-1\right) \tau
+p  \\ g\leq h}}^{\left( r-1\right) \tau +\tau _{1}+p-1}\left\vert E\left[
u_{ig}u_{ih}\right] \right\vert =O\left( \frac{1}{\tau _{1}}\right)
\end{equation*}

\item[(b)] 
\begin{equation*}
\frac{1}{\tau _{1}^{3}}\dsum\limits_{\substack{ h,v,w=\left( r-1\right) \tau
+p  \\ h\leq v\leq w}}^{\left( r-1\right) \tau +\tau _{1}+p-1}\left\vert
E\left( u_{ih}u_{iv}u_{iw}\right) \right\vert =O\left( \frac{1}{\tau _{1}^{2}%
}\right)
\end{equation*}

\item[(c)] \textbf{\ }%
\begin{equation*}
\frac{1}{\tau _{1}^{4}}\dsum\limits_{\substack{ g,h,v,w=\left( r-1\right)
\tau +p  \\ g\leq h\leq v\leq w}}^{\left( r-1\right) \tau +\tau
_{1}+p-1}\left\vert E\left[ u_{ig}u_{ih}u_{iv}u_{iw}\right] \right\vert
=O\left( \frac{1}{\tau _{1}^{2}}\right)
\end{equation*}
\end{enumerate}

\noindent \textbf{Proof of Lemma OA-4:}

To show part (a), first write%
\begin{equation}
\frac{1}{\tau _{1}^{2}}\dsum\limits_{\substack{ g,h=\left( r-1\right) \tau
+p  \\ g\leq h}}^{\left( r-1\right) \tau +\tau _{1}+p-1}\left\vert E\left[
u_{ig}u_{ih}\right] \right\vert =\frac{1}{\tau _{1}^{2}}\dsum\limits_{g=%
\left( r-1\right) \tau +p}^{\left( r-1\right) \tau +\tau _{1}+p-1}E\left[
u_{ig}^{2}\right] +\frac{1}{\tau _{1}^{2}}\dsum\limits_{\substack{ %
g,h=\left( r-1\right) \tau +p  \\ g<h}}^{\left( r-1\right) \tau +\tau
_{1}+p-1}\left\vert E\left[ u_{ig}u_{ih}\right] \right\vert  \label{Euguh}
\end{equation}%
Consider now the first term on the right-hand side of expression (\ref{Euguh}%
). Note that, trivially, by Assumption 2-3(b), there exists a positive
constant $C$ such that%
\begin{equation}
\frac{1}{\tau _{1}^{2}}\dsum\limits_{g=\left( r-1\right) \tau +p}^{\left(
r-1\right) \tau +\tau _{1}+p-1}E\left[ u_{ig}^{2}\right] \leq \frac{C}{\tau
_{1}}=O\left( \frac{1}{\tau _{1}}\right)  \label{term2 1}
\end{equation}%
For the second term on the right-hand side of expression (\ref{Euguh}), note
that by Assumption 2-3(c), $\left\{ u_{it}\right\} _{t=-\infty }^{\infty }$
is $\beta $-mixing with $\beta $ mixing coefficient satisfying 
\begin{equation*}
\beta _{i}\left( m\right) \leq a_{1}\exp \left\{ -a_{2}m\right\} \text{.}
\end{equation*}%
for every $i$. Since $\alpha _{i,m}\leq \beta _{i}\left( m\right) $, it
follows that $\left\{ u_{it}\right\} _{t=-\infty }^{\infty }$ is $\alpha $%
-mixing as well, with $\alpha $ mixing coefficient satisfying%
\begin{equation*}
\alpha _{i,m}\leq a_{1}\exp \left\{ -a_{2}m\right\} \text{ for every }i\text{%
. }
\end{equation*}

\noindent Hence, in this case, we can apply Lemma OA-3 with $p=6$ and $r=5/4$
to obtain%
\begin{eqnarray*}
&&\frac{1}{\tau _{1}^{2}}\dsum\limits_{\substack{ g,h=\left( r-1\right) \tau
+p  \\ g<h}}^{\left( r-1\right) \tau +\tau _{1}+p-1}\left\vert E\left[
u_{ig}u_{ih}\right] \right\vert \\
&\leq &\frac{1}{\tau _{1}^{2}}\dsum\limits_{\substack{ g,h=\left( r-1\right)
\tau +p  \\ g<h}}^{\left( r-1\right) \tau +\tau _{1}+p-1}2\left( 2^{1-\frac{%
{\large 1}}{{\large 6}}}+1\right) \left[ a_{1}\exp \left\{ -a_{2}\left(
h-g\right) \right\} \right] ^{1-\frac{{\large 1}}{{\large 6}}-\frac{{\large 4%
}}{{\large 5}}}\left( E\left\vert u_{ig}\right\vert ^{6}\right) ^{\frac{%
{\large 1}}{{\large 6}}}\left( E\left\vert u_{ih}\right\vert ^{\frac{{\large %
5}}{{\large 4}}}\right) ^{\frac{{\large 4}}{{\large 5}}}
\end{eqnarray*}%
Next, by application of Liapunov's inequality, we have that there exists
some positive constant $\overline{C}$ such that%
\begin{eqnarray*}
\left( E\left\vert u_{ig}\right\vert ^{6}\right) ^{\frac{{\large 1}}{{\large %
6}}}\left( E\left\vert u_{ih}\right\vert ^{\frac{{\large 5}}{{\large 4}}%
}\right) ^{\frac{{\large 4}}{{\large 5}}} &\leq &\left( E\left\vert
u_{ig}\right\vert ^{6}\right) ^{\frac{{\large 1}}{{\large 6}}}\left(
E\left\vert u_{ih}\right\vert ^{6}\right) ^{\frac{{\large 1}}{{\large 6}}} \\
&\leq &\left( \sup_{t}E\left\vert u_{it}\right\vert ^{6}\right) ^{\frac{%
{\large 1}}{{\large 3}}}\text{ \ } \\
&=&\overline{C}^{\frac{{\large 1}}{{\large 3}}}<\infty \text{ \ }\left( 
\text{by Assumption 2-3(b)}\right)
\end{eqnarray*}%
Moreover, let $\varrho =h-g$, so that $h=g+\varrho $. Using these notations
and the boundedness of $\left( E\left\vert u_{ig}\right\vert ^{6}\right) ^{%
\frac{{\large 1}}{{\large 6}}}\left( E\left\vert u_{ih}\right\vert ^{\frac{%
{\large 5}}{{\large 4}}}\right) ^{\frac{{\large 4}}{{\large 5}}}$ as shown
above, we can further write%
\begin{eqnarray}
&&\frac{1}{\tau _{1}^{2}}\dsum\limits_{\substack{ g,h=\left( r-1\right) \tau
+p  \\ g<h}}^{\left( r-1\right) \tau +\tau _{1}+p-1}\left\vert E\left[
u_{ig}u_{ih}\right] \right\vert  \notag \\
&&\frac{1}{\tau _{1}^{2}}\dsum\limits_{\substack{ g,h=\left( r-1\right) \tau
+p  \\ g<h}}^{\left( r-1\right) \tau +\tau _{1}+p-1}2\left( 2^{1-\frac{%
{\large 1}}{{\large 6}}}+1\right) \left[ a_{1}\exp \left\{ -a_{2}\left(
h-g\right) \right\} \right] ^{1-\frac{{\large 1}}{{\large 6}}-\frac{{\large 4%
}}{{\large 5}}}\left( E\left\vert u_{ig}\right\vert ^{6}\right) ^{\frac{%
{\large 1}}{{\large 6}}}\left( E\left\vert u_{ih}\right\vert ^{\frac{{\large %
5}}{{\large 4}}}\right) ^{\frac{{\large 4}}{{\large 5}}}  \notag \\
&\leq &\frac{\overline{C}^{\frac{{\large 1}}{{\large 3}}}}{\tau _{1}^{2}}%
\dsum\limits_{\substack{ g,h=\left( r-1\right) \tau +p  \\ g<h}}^{\left(
r-1\right) \tau +\tau _{1}+p-1}2\left( 2^{\frac{{\large 5}}{{\large 6}}%
}+1\right) \left[ a_{1}\exp \left\{ -a_{2}\left( h-g\right) \right\} \right]
^{\frac{{\large 1}}{{\large 30}}}  \notag \\
&\leq &\frac{C^{\ast }}{\tau _{1}^{2}}\dsum\limits_{\substack{ g,h=\left(
r-1\right) \tau +p  \\ g<h}}^{\left( r-1\right) \tau +\tau _{1}+p-1}\exp
\left\{ -\frac{a_{2}}{30}\varrho \right\} \text{ }  \notag \\
&&\left( \text{for some constant }C^{\ast }\text{ such that }2\left( 2^{%
\frac{{\large 5}}{{\large 6}}}+1\right) \overline{C}^{\frac{{\large 1}}{%
{\large 3}}}a_{1}^{\frac{{\large 1}}{{\large 30}}}\leq C^{\ast }<\infty
\right)  \notag \\
&\leq &\frac{C^{\ast }}{\tau _{1}^{2}}\dsum\limits_{\substack{ g=\left(
r-1\right) \tau +p}}^{\left( r-1\right) \tau +\tau
_{1}+p-1}\dsum\limits_{\varrho =1}^{\infty }\exp \left\{ -\frac{a_{2}}{30}%
\varrho \right\}  \notag \\
&=&\frac{C^{\ast }}{\tau _{1}}\dsum\limits_{\substack{ \varrho =1}}^{\infty
}\exp \left\{ -\frac{a_{2}}{30}\varrho \right\} \text{ }  \notag \\
&=&O\left( \frac{1}{\tau _{1}}\right) \text{ \ }\left( \text{given Lemma OA-1%
}\right)  \label{term2 2}
\end{eqnarray}%
It follows from expressions (\ref{Euguh}), (\ref{term2 1}), and (\ref{term2
2}) that%
\begin{eqnarray*}
\frac{1}{\tau _{1}^{2}}\dsum\limits_{\substack{ g,h=\left( r-1\right) \tau
+p  \\ g\leq h}}^{\left( r-1\right) \tau +\tau _{1}+p-1}\left\vert E\left[
u_{ig}u_{ih}\right] \right\vert &=&\frac{1}{\tau _{1}^{2}}%
\dsum\limits_{g=\left( r-1\right) \tau +p}^{\left( r-1\right) \tau +\tau
_{1}+p-1}E\left[ u_{ig}^{2}\right] +\frac{1}{\tau _{1}^{2}}\dsum\limits 
_{\substack{ g,h=\left( r-1\right) \tau +p  \\ g<h}}^{\left( r-1\right) \tau
+\tau _{1}+p-1}\left\vert E\left[ u_{ig}u_{ih}\right] \right\vert \\
&=&O\left( \frac{1}{\tau _{1}}\right) +O\left( \frac{1}{\tau _{1}}\right) \\
&=&O\left( \frac{1}{\tau _{1}}\right) \text{.}
\end{eqnarray*}

To show part (b), first write%
\begin{eqnarray}
\frac{1}{\tau _{1}^{3}}\dsum\limits_{\substack{ h,v,w=\left( r-1\right) \tau
+p  \\ h\leq v\leq w}}^{\left( r-1\right) \tau +\tau _{1}+p-1}\left\vert
E\left( u_{ih}u_{iv}u_{iw}\right) \right\vert &=&\frac{1}{\tau _{1}^{3}}%
\dsum\limits_{\substack{ h=\left( r-1\right) \tau +p}}^{\left( r-1\right)
\tau +\tau _{1}+p-1}E\left\vert u_{ih}\right\vert ^{3}+\frac{1}{\tau _{1}^{3}%
}\dsum\limits_{\substack{ h,v,w=\left( r-1\right) \tau +p  \\ h\leq v\leq w 
\\ v-h>w-v,\text{ }v-h>0}}^{\left( r-1\right) \tau +\tau _{1}+p-1}\left\vert
E\left( u_{ih}u_{iv}u_{iw}\right) \right\vert  \notag \\
&&+\frac{1}{\tau _{1}^{3}}\dsum\limits_{\substack{ h,v,w=\left( r-1\right)
\tau +p  \\ h\leq v\leq w  \\ w-v\geq v-h,\text{ }w-v>0}}^{\left( r-1\right)
\tau +\tau _{1}+p-1}\left\vert E\left( u_{ih}u_{iv}u_{iw}\right) \right\vert
\label{Euhuvuw}
\end{eqnarray}%
For the first term on the right-hand side of expression (\ref{Euhuvuw})
above, note that, trivially, we can apply Assumption 2-3(b) to obtain%
\begin{equation}
\frac{1}{\tau _{1}^{3}}\dsum\limits_{\substack{ h=\left( r-1\right) \tau +p}}%
^{\left( r-1\right) \tau +\tau _{1}+p-1}E\left\vert u_{ih}\right\vert
^{3}\leq \frac{C}{\tau _{1}^{2}}=O\left( \frac{1}{\tau _{1}^{2}}\right) 
\text{.}  \label{term3 1}
\end{equation}

Next, for the second term on the right-hand side of expression (\ref{Euhuvuw}%
) above, we can apply Lemma OA-3 with $p=6$ and $r=5/4$ to obtain%
\begin{eqnarray*}
&&\frac{1}{\tau _{1}^{3}}\dsum\limits_{\substack{ h,v,w=\left( r-1\right)
\tau +p  \\ h\leq v\leq w  \\ v-h>w-v,\text{ }v-h>0}}^{\left( r-1\right)
\tau +\tau _{1}+p-1}\left\vert E\left( u_{ih}u_{iv}u_{iw}\right) \right\vert
\\
&\leq &\frac{1}{\tau _{1}^{3}}\dsum\limits_{\substack{ h,v,w=\left(
r-1\right) \tau +p  \\ h\leq v\leq w  \\ v-h>w-v,\text{ }v-h>0}}^{\left(
r-1\right) \tau +\tau _{1}+p-1}2\left( 2^{1-\frac{{\large 1}}{{\large 6}}%
}+1\right) \left[ a_{1}\exp \left\{ -a_{2}\left( v-h\right) \right\} \right]
^{1-\frac{{\large 1}}{{\large 6}}-\frac{{\large 4}}{{\large 5}}}\left(
E\left\vert u_{ih}\right\vert ^{6}\right) ^{\frac{{\large 1}}{{\large 6}}%
}\left( E\left\vert u_{iv}u_{iw}\right\vert ^{\frac{{\large 5}}{{\large 4}}%
}\right) ^{\frac{{\large 4}}{{\large 5}}}
\end{eqnarray*}%
Next, by application of H\"{o}lder's inequality, we have%
\begin{eqnarray*}
\left( E\left\vert u_{ih}\right\vert ^{6}\right) ^{\frac{{\large 1}}{{\large %
6}}}\left( E\left\vert u_{iv}u_{iw}\right\vert ^{\frac{{\large 5}}{{\large 4}%
}}\right) ^{\frac{{\large 4}}{{\large 5}}} &\leq &\left( E\left\vert
u_{ih}\right\vert ^{6}\right) ^{\frac{{\large 1}}{{\large 6}}}\left( \left(
E\left\vert u_{iv}\right\vert ^{\frac{{\large 5}}{{\large 2}}}\right) ^{%
\frac{{\large 1}}{{\large 2}}}\left( E\left\vert u_{iw}\right\vert ^{\frac{%
{\large 5}}{{\large 2}}}\right) ^{\frac{{\large 1}}{{\large 2}}}\right) ^{%
\frac{{\large 4}}{{\large 5}}} \\
&=&\left( E\left\vert u_{ih}\right\vert ^{6}\right) ^{\frac{{\large 1}}{%
{\large 6}}}\left( E\left\vert u_{iv}\right\vert ^{\frac{{\large 5}}{{\large %
2}}}\right) ^{\frac{{\large 2}}{{\large 5}}}\left( E\left\vert
u_{iw}\right\vert ^{\frac{{\large 5}}{{\large 2}}}\right) ^{\frac{{\large 2}%
}{{\large 5}}} \\
&\leq &\left( E\left\vert u_{ih}\right\vert ^{6}\right) ^{\frac{{\large 1}}{%
{\large 6}}}\left( E\left\vert u_{iv}\right\vert ^{6}\right) ^{\frac{{\large %
1}}{{\large 6}}}\left( E\left\vert u_{iw}\right\vert ^{6}\right) ^{\frac{%
{\large 1}}{{\large 6}}} \\
&&\left( \text{by Liapunov's inequality}\right) \\
&=&\overline{C}^{\frac{{\large 1}}{{\large 2}}}<\infty \text{ }\left( \text{%
by Assumption 2-3(b)}\right)
\end{eqnarray*}%
Moreover, let $\varrho _{1}=v-h$ and $\varrho _{2}=w-v$, so that $%
v=h+\varrho _{1}$ and $w=v+$ $\varrho _{2}=h+\varrho _{1}+$ $\varrho _{2}$.
Using these notations and the boundedness of $\left( E\left\vert
u_{ih}\right\vert ^{6}\right) ^{\frac{{\large 1}}{{\large 6}}}\left(
E\left\vert u_{iv}u_{iw}\right\vert ^{\frac{{\large 5}}{{\large 4}}}\right)
^{\frac{{\large 4}}{{\large 5}}}$ as shown above, we can further write%
\begin{eqnarray}
&&\frac{1}{\tau _{1}^{3}}\dsum\limits_{\substack{ h,v,w=\left( r-1\right)
\tau +p  \\ h\leq v\leq w  \\ v-h>w-v,\text{ }v-h>0}}^{\left( r-1\right)
\tau +\tau _{1}+p-1}\left\vert E\left( u_{ih}u_{iv}u_{iw}\right) \right\vert
\notag \\
&\leq &\frac{1}{\tau _{1}^{3}}\dsum\limits_{\substack{ h,v,w=\left(
r-1\right) \tau +p  \\ h\leq v\leq w  \\ v-h>w-v,\text{ }v-h>0}}^{\left(
r-1\right) \tau +\tau _{1}+p-1}2\left( 2^{1-\frac{{\large 1}}{{\large 6}}%
}+1\right) \left[ a_{1}\exp \left\{ -a_{2}\left( v-h\right) \right\} \right]
^{1-\frac{{\large 1}}{{\large 6}}-\frac{{\large 4}}{{\large 5}}}\left(
E\left\vert u_{ih}\right\vert ^{6}\right) ^{\frac{{\large 1}}{{\large 6}}%
}\left( E\left\vert u_{iv}u_{iw}\right\vert ^{\frac{{\large 5}}{{\large 4}}%
}\right) ^{\frac{{\large 4}}{{\large 5}}}  \notag \\
&\leq &\frac{\overline{C}^{\frac{{\large 1}}{{\large 2}}}}{\tau _{1}^{3}}%
\dsum\limits_{\substack{ h,v,w=\left( r-1\right) \tau +p  \\ h\leq v\leq w 
\\ v-h>w-v,\text{ }v-h>0}}^{\left( r-1\right) \tau +\tau _{1}+p-1}2\left( 2^{%
\frac{{\large 5}}{{\large 6}}}+1\right) \left[ a_{1}\exp \left\{
-a_{2}\left( v-h\right) \right\} \right] ^{\frac{{\large 1}}{{\large 30}}} 
\notag \\
&\leq &\frac{C^{\ast }}{\tau _{1}^{3}}\dsum\limits_{\substack{ h,v,w=\left(
r-1\right) \tau +p  \\ h\leq v\leq w  \\ v-h>w-v,\text{ }v-h>0}}^{\left(
r-1\right) \tau +\tau _{1}+p-1}\exp \left\{ -\frac{a_{2}}{30}\varrho
_{1}\right\} \text{ }  \notag \\
&&\left( \text{for some constant }C^{\ast }\text{ such that }2\left( 2^{%
\frac{{\large 5}}{{\large 6}}}+1\right) \overline{C}^{\frac{{\large 1}}{%
{\large 2}}}a_{1}^{\frac{{\large 1}}{{\large 30}}}\leq C^{\ast }<\infty
\right)  \notag \\
&\leq &\frac{C^{\ast }}{\tau _{1}^{3}}\dsum\limits_{\substack{ h=\left(
r-1\right) \tau +p}}^{\left( r-1\right) \tau +\tau
_{1}+p-1}\dsum\limits_{\varrho _{{\large 1}}=1}^{\infty
}\dsum\limits_{\varrho _{{\large 2}}=0}^{\varrho _{{\large 1}}-1}\exp
\left\{ -\frac{a_{2}}{30}\varrho _{1}\right\}  \notag \\
&\leq &\frac{C^{\ast }}{\tau _{1}^{3}}\dsum\limits_{\substack{ h=\left(
r-1\right) \tau +p}}^{\left( r-1\right) \tau +\tau _{1}+p-1}\dsum\limits 
_{\substack{ \varrho _{{\large 1}}=1}}^{\infty }\varrho _{1}\exp \left\{ -%
\frac{a_{2}}{30}\varrho _{1}\right\}  \notag \\
&=&\frac{C^{\ast }}{\tau _{1}^{2}}\dsum\limits_{\substack{ \varrho _{{\large %
1}}=1}}^{\infty }\varrho _{1}\exp \left\{ -\frac{a_{2}}{30}\varrho
_{1}\right\} \text{ }  \notag \\
&=&O\left( \frac{1}{\tau _{1}^{2}}\right) \text{ \ }\left( \text{given Lemma
OA-1}\right)  \label{term3 2}
\end{eqnarray}

Similarly, for the third term on the right-hand side of expression (\ref%
{Euhuvuw}), we can apply Lemma OA-3 with $p=6$ and $r=5/4$ to obtain%
\begin{eqnarray*}
&&\frac{1}{\tau _{1}^{3}}\dsum\limits_{\substack{ h,v,w=\left( r-1\right)
\tau +p  \\ h\leq v\leq w  \\ w-v\geq v-h,\text{ }w-v>0}}^{\left( r-1\right)
\tau +\tau _{1}+p-1}\left\vert E\left( u_{ih}u_{iv}u_{iw}\right) \right\vert
\\
&\leq &\frac{1}{\tau _{1}^{3}}\dsum\limits_{\substack{ h,v,w=\left(
r-1\right) \tau +p  \\ h\leq v\leq w  \\ w-v\geq v-h,\text{ }w-v>0}}^{\left(
r-1\right) \tau +\tau _{1}+p-1}2\left( 2^{1-\frac{{\large 1}}{{\large 6}}%
}+1\right) \left[ a_{1}\exp \left\{ -a_{2}\left( w-v\right) \right\} \right]
^{1-\frac{{\large 4}}{{\large 5}}-\frac{{\large 1}}{{\large 6}}}\left(
E\left\vert u_{ih}u_{iv}\right\vert ^{\frac{{\large 5}}{{\large 4}}}\right)
^{\frac{{\large 4}}{{\large 5}}}\left( E\left\vert u_{iw}\right\vert
^{6}\right) ^{\frac{{\large 1}}{{\large 6}}}
\end{eqnarray*}%
Next, by applying H\"{o}lder's inequality, we have%
\begin{eqnarray*}
\left( E\left\vert u_{ih}u_{iv}\right\vert ^{\frac{{\large 5}}{{\large 4}}%
}\right) ^{\frac{{\large 4}}{{\large 5}}}\left( E\left\vert
u_{iw}\right\vert ^{6}\right) ^{\frac{{\large 1}}{{\large 6}}} &\leq &\left(
\left( E\left\vert u_{ih}\right\vert ^{\frac{{\large 5}}{{\large 2}}}\right)
^{\frac{{\large 1}}{{\large 2}}}\left( E\left\vert u_{iv}\right\vert ^{\frac{%
{\large 5}}{{\large 2}}}\right) ^{\frac{{\large 1}}{{\large 2}}}\right) ^{%
\frac{{\large 4}}{{\large 5}}}\left( E\left\vert u_{iw}\right\vert
^{6}\right) ^{\frac{{\large 1}}{{\large 6}}} \\
&=&\left( E\left\vert u_{ih}\right\vert ^{\frac{{\large 5}}{{\large 2}}%
}\right) ^{\frac{{\large 2}}{{\large 5}}}\left( E\left\vert
u_{iv}\right\vert ^{\frac{{\large 5}}{{\large 2}}}\right) ^{\frac{{\large 2}%
}{{\large 5}}}\left( E\left\vert u_{iw}\right\vert ^{6}\right) ^{\frac{%
{\large 1}}{{\large 6}}} \\
&\leq &\left( E\left\vert u_{ih}\right\vert ^{6}\right) ^{\frac{{\large 1}}{%
{\large 6}}}\left( E\left\vert u_{iv}\right\vert ^{6}\right) ^{\frac{{\large %
1}}{{\large 6}}}\left( E\left\vert u_{iw}\right\vert ^{6}\right) ^{\frac{%
{\large 1}}{{\large 6}}} \\
&&\left( \text{by Liapunov's inequality}\right) \text{\ } \\
&=&\overline{C}^{\frac{{\large 1}}{{\large 2}}}<\infty \text{ }\left( \text{%
by Assumption 2-3(b)}\right)
\end{eqnarray*}%
Moreover, let $\varrho _{1}=v-h$ and $\varrho _{2}=w-v$, so that $%
v=h+\varrho _{1}$ and $w=v+$ $\varrho _{2}=h+\varrho _{1}+$ $\varrho _{2}$.
Using these notations and the boundedness of $\left( E\left\vert
u_{ih}u_{iv}\right\vert ^{\frac{{\large 5}}{{\large 4}}}\right) ^{\frac{%
{\large 4}}{{\large 5}}}\left( E\left\vert u_{iw}\right\vert ^{6}\right) ^{%
\frac{{\large 1}}{{\large 6}}}$ as shown above, we can further write%
\begin{eqnarray}
&&\frac{1}{\tau _{1}^{3}}\dsum\limits_{\substack{ h,v,w=\left( r-1\right)
\tau +p  \\ h\leq v\leq w  \\ w-v\geq v-h,\text{ }w-v>0}}^{\left( r-1\right)
\tau +\tau _{1}+p-1}\left\vert E\left( u_{ih}u_{iv}u_{iw}\right) \right\vert
\notag \\
&\leq &\frac{1}{\tau _{1}^{3}}\dsum\limits_{\substack{ h,v,w=\left(
r-1\right) \tau +p  \\ h\leq v\leq w  \\ w-v\geq v-h,\text{ }w-v>0}}^{\left(
r-1\right) \tau +\tau _{1}+p-1}2\left( 2^{1-\frac{{\large 1}}{{\large 6}}%
}+1\right) \left[ a_{1}\exp \left\{ -a_{2}\left( w-v\right) \right\} \right]
^{1-\frac{{\large 4}}{{\large 5}}-\frac{{\large 1}}{{\large 6}}}\left(
E\left\vert u_{ih}u_{iv}\right\vert ^{\frac{{\large 5}}{{\large 4}}}\right)
^{\frac{{\large 4}}{{\large 5}}}\left( E\left\vert u_{iw}\right\vert
^{6}\right) ^{\frac{{\large 1}}{{\large 6}}}  \notag \\
&\leq &\frac{\overline{C}^{\frac{{\large 1}}{{\large 2}}}}{\tau _{1}^{3}}%
\dsum\limits_{\substack{ h,v,w=\left( r-1\right) \tau +p  \\ h\leq v\leq w 
\\ w-v\geq v-h,\text{ }w-v>0}}^{\left( r-1\right) \tau +\tau
_{1}+p-1}2\left( 2^{\frac{{\large 5}}{{\large 6}}}+1\right) \left[ a_{1}\exp
\left\{ -a_{2}\left( w-v\right) \right\} \right] ^{\frac{{\large 1}}{{\large %
30}}}  \notag \\
&\leq &\frac{C^{\ast }}{\tau _{1}^{3}}\dsum\limits_{\substack{ h,v,w=\left(
r-1\right) \tau +p  \\ h\leq v\leq w  \\ w-v\geq v-h,\text{ }w-v>0}}^{\left(
r-1\right) \tau +\tau _{1}+p-1}\exp \left\{ -\frac{a_{2}}{30}\varrho
_{2}\right\} \text{ }  \notag \\
&&\left( \text{for some constant }C^{\ast }\text{ such that }2\left( 2^{%
\frac{{\large 5}}{{\large 6}}}+1\right) \overline{C}^{\frac{{\large 1}}{%
{\large 2}}}a_{1}^{\frac{{\large 1}}{{\large 30}}}\leq C^{\ast }<\infty
\right)  \notag \\
&\leq &\frac{C^{\ast }}{\tau _{1}^{3}}\dsum\limits_{\substack{ h=\left(
r-1\right) \tau +p}}^{\left( r-1\right) \tau +\tau _{1}+p-1}\dsum\limits 
_{\substack{ \varrho _{{\large 2}}=1}}^{\infty }\dsum\limits_{\varrho _{%
{\large 1}}=0}^{\varrho _{{\large 2}}}\exp \left\{ -\frac{a_{2}}{30}\varrho
_{2}\right\}  \notag \\
&=&\frac{C^{\ast }}{\tau _{1}^{3}}\dsum\limits_{\substack{ h=\left(
r-1\right) \tau +p}}^{\left( r-1\right) \tau +\tau _{1}+p-1}\dsum\limits 
_{\substack{ \varrho _{{\large 2}}=1}}^{\infty }\left( \varrho _{2}+1\right)
\exp \left\{ -\frac{a_{2}}{30}\varrho _{2}\right\}  \notag \\
&=&\frac{C^{\ast }}{\tau _{1}^{2}}\left[ \dsum\limits_{\substack{ \varrho _{%
{\large 2}}=1}}^{\infty }\varrho _{2}\exp \left\{ -\frac{a_{2}}{30}\varrho
_{2}\right\} +\dsum\limits_{\substack{ \varrho _{{\large 2}}=1}}^{\infty
}\exp \left\{ -\frac{a_{2}}{30}\varrho _{2}\right\} \right]  \notag \\
&=&O\left( \frac{1}{\tau _{1}^{2}}\right) \text{ \ }\left( \text{given Lemma
OA-1}\right)  \label{term3 3}
\end{eqnarray}

It follows from expressions (\ref{Euhuvuw}), (\ref{term3 1}), (\ref{term3 2}%
), and (\ref{term3 3}) that%
\begin{eqnarray*}
\frac{1}{\tau _{1}^{3}}\dsum\limits_{\substack{ h,v,w=\left( r-1\right) \tau
+p  \\ h\leq v\leq w}}^{\left( r-1\right) \tau +\tau _{1}+p-1}\left\vert
E\left( u_{ih}u_{iv}u_{iw}\right) \right\vert &=&\frac{1}{\tau _{1}^{3}}%
\dsum\limits_{\substack{ h=\left( r-1\right) \tau +p}}^{\left( r-1\right)
\tau +\tau _{1}+p-1}E\left\vert u_{ih}\right\vert ^{3}+\frac{1}{\tau _{1}^{3}%
}\dsum\limits_{\substack{ h,v,w=\left( r-1\right) \tau +p  \\ h\leq v\leq w 
\\ v-h>w-v,\text{ }v-h>0}}^{\left( r-1\right) \tau +\tau _{1}+p-1}\left\vert
E\left( u_{ih}u_{iv}u_{iw}\right) \right\vert \\
&&+\frac{1}{\tau _{1}^{3}}\dsum\limits_{\substack{ h,v,w=\left( r-1\right)
\tau +p  \\ h\leq v\leq w  \\ w-v\geq v-h,\text{ }w-v>0}}^{\left( r-1\right)
\tau +\tau _{1}+p-1}\left\vert E\left( u_{ih}u_{iv}u_{iw}\right) \right\vert
\\
&=&O\left( \frac{1}{\tau _{1}^{2}}\right) +O\left( \frac{1}{\tau _{1}^{2}}%
\right) +O\left( \frac{1}{\tau _{1}^{2}}\right) \\
&=&O\left( \frac{1}{\tau _{1}^{2}}\right) \text{. }
\end{eqnarray*}

Finally, to show part (c), we first write%
\begin{eqnarray}
&&\frac{1}{\tau _{1}^{4}}\dsum\limits_{\substack{ g,h,v,w=\left( r-1\right)
\tau +p  \\ g\leq h\leq v\leq w}}^{\left( r-1\right) \tau +\tau
_{1}+p-1}\left\vert E\left[ u_{ig}u_{ih}u_{iv}u_{iw}\right] \right\vert 
\notag \\
&=&\frac{1}{\tau _{1}^{4}}\dsum\limits_{\substack{ g,h=\left( r-1\right)
\tau +p  \\ g\leq h}}^{\left( r-1\right) \tau +\tau _{1}+p-1}\left\vert E%
\left[ u_{ig}u_{ih}^{3}\right] \right\vert +\frac{1}{\tau _{1}^{4}}%
\dsum\limits_{\substack{ g,h,v,w=\left( r-1\right) \tau +p  \\ g\leq h\leq
v\leq w  \\ w-v>v-h,\text{ }w-v>0}}^{\left( r-1\right) \tau +\tau
_{1}+p-1}\left\vert E\left[ u_{ig}u_{ih}u_{iv}u_{iw}\right] \right\vert 
\notag \\
&&+\frac{1}{\tau _{1}^{4}}\dsum\limits_{\substack{ g,h,v,w=\left( r-1\right)
\tau +p  \\ g\leq h\leq v\leq w  \\ w-v\leq v-h,\text{ }v-h>0}}^{\left(
r-1\right) \tau +\tau _{1}+p-1}\left\vert E\left[ u_{ig}u_{ih}u_{iv}u_{iw}%
\right] \right\vert  \notag \\
&=&\frac{1}{\tau _{1}^{4}}\dsum\limits_{\substack{ g,h=\left( r-1\right)
\tau +p  \\ g\leq h}}^{\left( r-1\right) \tau +\tau _{1}+p-1}\left\vert E%
\left[ u_{ig}u_{ih}^{3}\right] \right\vert +\frac{1}{\tau _{1}^{4}}%
\dsum\limits_{\substack{ g,h,v,w=\left( r-1\right) \tau +p  \\ g\leq h\leq
v\leq w  \\ w-v>v-h,\text{ }w-v>0}}^{\left( r-1\right) \tau +\tau
_{1}+p-1}\left\vert E\left[ \left\{ u_{ig}u_{ih}-E\left( u_{ig}u_{ih}\right)
+E\left( u_{ig}u_{ih}\right) \right\} u_{iv}u_{iw}\right] \right\vert  \notag
\\
&&+\frac{1}{\tau _{1}^{4}}\dsum\limits_{\substack{ g,h,v,w=\left( r-1\right)
\tau +p  \\ g\leq h\leq v\leq w  \\ w-v\leq v-h,\text{ }v-h>0}}^{\left(
r-1\right) \tau +\tau _{1}+p-1}\left\vert E\left[ \left\{
u_{ig}u_{ih}-E\left( u_{ig}u_{ih}\right) +E\left( u_{ig}u_{ih}\right)
\right\} u_{iv}u_{iw}\right] \right\vert  \notag \\
&\leq &\frac{1}{\tau _{1}^{4}}\dsum\limits_{\substack{ g,h=\left( r-1\right)
\tau +p  \\ g\leq h}}^{\left( r-1\right) \tau +\tau _{1}+p-1}\left\vert E%
\left[ u_{ig}u_{ih}^{3}\right] \right\vert +\frac{1}{\tau _{1}^{4}}%
\dsum\limits_{\substack{ g,h,v,w=\left( r-1\right) \tau +p  \\ g\leq h\leq
v\leq w  \\ w-v>v-h,\text{ }w-v>0}}^{\left( r-1\right) \tau +\tau
_{1}+p-1}\left\vert E\left[ \left\{ u_{ig}u_{ih}-E\left( u_{ig}u_{ih}\right)
\right\} u_{iv}u_{iw}\right] \right\vert  \notag \\
&&+\frac{1}{\tau _{1}^{4}}\dsum\limits_{\substack{ g,h,v,w=\left( r-1\right)
\tau +p  \\ g\leq h\leq v\leq w  \\ w-v\leq v-h,\text{ }v-h>0}}^{\left(
r-1\right) \tau +\tau _{1}+p-1}\left\vert E\left[ \left\{
u_{ig}u_{ih}-E\left( u_{ig}u_{ih}\right) \right\} u_{iv}u_{iw}\right]
\right\vert  \notag \\
&&+\frac{1}{\tau _{1}^{4}}\dsum\limits_{\substack{ g,h,v,w=\left( r-1\right)
\tau +p  \\ g\leq h\leq v\leq w  \\ w-h>0}}^{\left( r-1\right) \tau +\tau
_{1}+p-1}\left\vert E\left( u_{ig}u_{ih}\right) \right\vert \left\vert
E\left( u_{iv}u_{iw}\right) \right\vert  \label{Euguhuvuw}
\end{eqnarray}%
For the first term on the right-hand side of expression (\ref{Euguhuvuw})
above, note that, trivially, by Jensen's inequality and H\"{o}lder's
inequality, we have%
\begin{eqnarray}
\frac{1}{\tau _{1}^{4}}\dsum\limits_{\substack{ g,h=\left( r-1\right) \tau
+p  \\ g\leq h}}^{\left( r-1\right) \tau +\tau _{1}+p-1}\left\vert E\left[
u_{ig}u_{ih}^{3}\right] \right\vert &\leq &\frac{1}{\tau _{1}^{4}}%
\dsum\limits_{\substack{ g,h=\left( r-1\right) \tau +p  \\ g\leq h}}^{\left(
r-1\right) \tau +\tau _{1}+p-1}E\left[ \left\vert
u_{ig}u_{ih}^{3}\right\vert \right]  \notag \\
&\leq &\frac{1}{\tau _{1}^{4}}\dsum\limits_{\substack{ g,h=\left( r-1\right)
\tau +p  \\ g\leq h}}^{\left( r-1\right) \tau +\tau _{1}+p-1}\sqrt{%
E\left\vert u_{ig}\right\vert ^{2}}\sqrt{E\left\vert u_{ih}\right\vert ^{6}}
\notag \\
&\leq &\frac{1}{\tau _{1}^{4}}\dsum\limits_{\substack{ g,h=\left( r-1\right)
\tau +p  \\ g\leq h}}^{\left( r-1\right) \tau +\tau _{1}+p-1}\left(
E\left\vert u_{ih}\right\vert ^{6}\right) ^{\frac{{\large 1}}{{\large 6}}}%
\sqrt{E\left\vert u_{ih}\right\vert ^{6}}  \notag \\
&&\left( \text{by Liapunov's inequality}\right)  \notag \\
&\leq &\frac{\overline{C}^{\frac{{\large 2}}{{\large 3}}}\tau _{1}^{2}}{\tau
_{1}^{4}}\text{ \ }\left( \text{by Assumption 2-3(b)}\right)  \notag \\
&=&O\left( \frac{1}{\tau _{1}^{2}}\right)  \label{term4 1}
\end{eqnarray}

Next, for the second term on the right-hand side of expression (\ref%
{Euguhuvuw}), we can apply Lemma OA-3 with $p=4/3$ and $r=6$ to obtain%
\begin{eqnarray*}
&&\frac{1}{\tau _{1}^{4}}\dsum\limits_{\substack{ g,h,v,w=\left( r-1\right)
\tau +p  \\ g\leq h\leq v\leq w  \\ w-v>v-h,\text{ }w-v>0}}^{\left(
r-1\right) \tau +\tau _{1}+p-1}\left\vert E\left[ \left\{
u_{ig}u_{ih}-E\left( u_{ig}u_{ih}\right) \right\} u_{iv}u_{iw}\right]
\right\vert \\
&\leq &\frac{1}{\tau _{1}^{4}}\dsum\limits_{\substack{ g,h,v,w=\left(
r-1\right) \tau +p  \\ g\leq h\leq v\leq w  \\ w-v>v-h,\text{ }w-v>0}}%
^{\left( r-1\right) \tau +\tau _{1}+p-1}\left\{ 2\left( 2^{1-\frac{{\large 3}%
}{{\large 4}}}+1\right) \left[ a_{1}\exp \left\{ -a_{2}\left( w-v\right)
\right\} \right] ^{1-\frac{{\large 3}}{{\large 4}}-\frac{{\large 1}}{{\large %
6}}}\right. \\
&&\text{ \ \ \ \ \ \ \ \ \ \ \ \ \ \ \ \ \ \ \ \ \ \ \ \ \ }\left. \times
\left( E\left\vert \left\{ u_{ig}u_{ih}-E\left( u_{ig}u_{ih}\right) \right\}
u_{iv}\right\vert ^{\frac{{\large 4}}{{\large 3}}}\right) ^{\frac{{\large 3}%
}{{\large 4}}}\left( E\left\vert u_{iw}\right\vert ^{6}\right) ^{\frac{%
{\large 1}}{{\large 6}}}\right\}
\end{eqnarray*}%
Next, by repeated application of H\"{o}lder's inequality, we have%
\begin{eqnarray*}
E\left\vert \left\{ u_{ig}u_{ih}-E\left( u_{ig}u_{ih}\right) \right\}
u_{iv}\right\vert ^{\frac{{\large 4}}{{\large 3}}} &\leq &\left[ E\left\vert
u_{ig}u_{ih}-E\left( u_{ig}u_{ih}\right) \right\vert ^{\frac{{\large 12}}{%
{\large 7}}}\right] ^{\frac{{\large 7}}{{\large 9}}}\left[ E\left\vert
u_{iv}\right\vert ^{6}\right] ^{\frac{{\large 2}}{{\large 9}}} \\
&\leq &\left[ 2^{\frac{{\large 5}}{{\large 7}}}\left( E\left\vert
u_{ig}u_{ih}\right\vert ^{\frac{{\large 12}}{{\large 7}}}+\left\vert E\left[
u_{ig}u_{ih}\right] \right\vert ^{\frac{{\large 12}}{{\large 7}}}\right) %
\right] ^{\frac{{\large 7}}{{\large 9}}}\left[ E\left\vert u_{iv}\right\vert
^{6}\right] ^{\frac{{\large 2}}{{\large 9}}} \\
&&\left( \text{by Lo\`{e}ve's }c_{r}\text{ inequality}\right) \\
&\leq &\left[ 2^{\frac{{\large 5}}{{\large 7}}}\left( E\left\vert
u_{ig}u_{ih}\right\vert ^{\frac{{\large 12}}{{\large 7}}}+E\left\vert
u_{ig}u_{ih}\right\vert ^{\frac{{\large 12}}{{\large 7}}}\right) \right] ^{%
\frac{{\large 7}}{{\large 9}}}\left[ E\left\vert u_{iv}\right\vert ^{6}%
\right] ^{\frac{{\large 2}}{{\large 9}}} \\
&&\left( \text{by Jensen's inequality}\right) \\
&=&\left[ 2^{\frac{{\large 12}}{{\large 7}}}E\left\vert
u_{ig}u_{ih}\right\vert ^{\frac{{\large 12}}{{\large 7}}}\right] ^{\frac{%
{\large 7}}{{\large 9}}}\left[ E\left\vert u_{iv}\right\vert ^{6}\right] ^{%
\frac{{\large 2}}{{\large 9}}} \\
&\leq &2^{\frac{{\large 4}}{{\large 3}}}\left[ \left( E\left\vert
u_{ig}\right\vert ^{\frac{{\large 24}}{{\large 7}}}\right) ^{\frac{{\large 1}%
}{{\large 2}}}\left( E\left\vert u_{ih}\right\vert ^{\frac{{\large 24}}{%
{\large 7}}}\right) ^{\frac{{\large 1}}{{\large 2}}}\right] ^{\frac{{\large 7%
}}{{\large 9}}}\left[ E\left\vert u_{iv}\right\vert ^{6}\right] ^{\frac{%
{\large 2}}{{\large 9}}} \\
&=&2^{\frac{{\large 4}}{{\large 3}}}\left[ \left( E\left\vert
u_{ig}\right\vert ^{\frac{{\large 24}}{{\large 7}}}\right) ^{\frac{{\large 7}%
}{{\large 24}}}\left( E\left\vert u_{ih}\right\vert ^{\frac{{\large 24}}{%
{\large 7}}}\right) ^{\frac{{\large 7}}{{\large 24}}}\right] ^{\frac{{\large %
4}}{{\large 3}}}\left[ E\left\vert u_{iv}\right\vert ^{6}\right] ^{\frac{%
{\large 2}}{{\large 9}}} \\
&\leq &2^{\frac{{\large 4}}{{\large 3}}}\left[ \left( E\left\vert
u_{ig}\right\vert ^{6}\right) ^{\frac{{\large 1}}{{\large 6}}}\left(
E\left\vert u_{ih}\right\vert ^{6}\right) ^{\frac{{\large 1}}{{\large 6}}}%
\right] ^{\frac{{\large 4}}{{\large 3}}}\left[ E\left\vert u_{iv}\right\vert
^{6}\right] ^{\frac{{\large 2}}{{\large 9}}} \\
&\leq &2^{\frac{{\large 4}}{{\large 3}}}\left( \overline{C}\right) ^{\frac{%
{\large 2}}{{\large 9}}}\left( \overline{C}\right) ^{\frac{{\large 2}}{%
{\large 9}}}\left( \overline{C}\right) ^{\frac{{\large 2}}{{\large 9}}}\text{
\ }\left( \text{by Assumption 2-3(b) }\right) \\
&=&2^{\frac{{\large 4}}{{\large 3}}}\overline{C}^{\frac{{\large 2}}{{\large 3%
}}}
\end{eqnarray*}%
Moreover, let $\varrho _{1}=v-h$ and $\varrho _{2}=w-v$ so that $v=h+$ $%
\varrho _{1}$ and $w=v+\varrho _{2}=h+\varrho _{1}+\varrho _{2}$. Using
these notations and the boundedness of $E\left\vert \left\{
u_{ig}u_{ih}-E\left( u_{ig}u_{ih}\right) \right\} u_{iv}\right\vert ^{\frac{%
{\large 4}}{{\large 3}}}$ as shown above, we can further write%
\begin{eqnarray}
&&\frac{1}{\tau _{1}^{4}}\dsum\limits_{\substack{ g,h,v,w=\left( r-1\right)
\tau +p  \\ g\leq h\leq v\leq w  \\ w-v>v-h,\text{ }w-v>0}}^{\left(
r-1\right) \tau +\tau _{1}+p-1}\left\vert E\left[ \left\{
u_{ig}u_{ih}-E\left( u_{ig}u_{ih}\right) \right\} u_{iv}u_{iw}\right]
\right\vert  \notag \\
&\leq &\frac{1}{\tau _{1}^{4}}\dsum\limits_{\substack{ g,h,v,w=\left(
r-1\right) \tau +p  \\ g\leq h\leq v\leq w  \\ w-v>v-h,\text{ }w-v>0}}%
^{\left( r-1\right) \tau +\tau _{1}+p-1}\left\{ 2\left( 2^{1-\frac{{\large 3}%
}{{\large 4}}}+1\right) \left[ a_{1}\exp \left\{ -a_{2}\left( w-v\right)
\right\} \right] ^{1-\frac{{\large 3}}{{\large 4}}-\frac{{\large 1}}{{\large %
6}}}\right.  \notag \\
&&\text{ \ \ \ \ \ \ \ \ \ \ \ \ \ \ \ \ \ \ \ \ \ \ \ \ }\left. \times
\left( E\left\vert \left\{ u_{ig}u_{ih}-E\left( u_{ig}u_{ih}\right) \right\}
u_{iv}\right\vert ^{\frac{{\large 4}}{{\large 3}}}\right) ^{\frac{{\large 3}%
}{{\large 4}}}\left( E\left\vert u_{iw}\right\vert ^{6}\right) ^{\frac{%
{\large 1}}{{\large 6}}}\right\}  \notag \\
&\leq &\frac{1}{\tau _{1}^{4}}\dsum\limits_{\substack{ g,h,v,w=\left(
r-1\right) \tau +p  \\ g\leq h\leq v\leq w  \\ w-v>v-h,\text{ }w-v>0}}%
^{\left( r-1\right) \tau +\tau _{1}+p-1}2\left( 2^{\frac{{\large 1}}{{\large %
4}}}+1\right) \left[ a_{1}\exp \left\{ -a_{2}\left( w-v\right) \right\} %
\right] ^{\frac{{\large 1}}{{\large 12}}}\left( 2^{\frac{{\large 4}}{{\large %
3}}}\overline{C}^{\frac{{\large 2}}{{\large 3}}}\right) ^{\frac{{\large 3}}{%
{\large 4}}}\left( \overline{C}\right) ^{\frac{{\large 1}}{{\large 6}}} 
\notag \\
&\leq &\frac{C^{\ast }}{\tau _{1}^{4}}\dsum\limits_{\substack{ %
g,h,v,w=\left( r-1\right) \tau +p  \\ g\leq h\leq v\leq w  \\ w-v>v-h,\text{ 
}w-v>0}}^{\left( r-1\right) \tau +\tau _{1}+p-1}\exp \left\{ -\frac{a_{2}}{12%
}\varrho _{2}\right\} \text{ }  \notag \\
&&\left( \text{for some constant }C^{\ast }\text{ such that }4\left( 2^{%
\frac{{\large 1}}{{\large 4}}}+1\right) \overline{C}^{\frac{{\large 2}}{%
{\large 3}}}a_{1}^{\frac{{\large 1}}{{\large 12}}}\leq C^{\ast }<\infty
\right)  \notag \\
&\leq &\frac{C^{\ast }}{\tau _{1}^{4}}\dsum\limits_{\substack{ g=\left(
r-1\right) \tau +p}}^{\left( r-1\right) \tau +\tau _{1}+p-1}\dsum\limits 
_{\substack{ h=\left( r-1\right) \tau +p}}^{\left( r-1\right) \tau +\tau
_{1}+p-1}\dsum\limits_{\varrho _{{\large 2}}=1}^{\infty
}\dsum\limits_{\varrho _{{\large 1}}=0}^{\varrho _{{\large 2}}-1}\exp
\left\{ -\frac{a_{2}}{12}\varrho _{2}\right\}  \notag \\
&\leq &\frac{C^{\ast }}{\tau _{1}^{2}}\dsum\limits_{\substack{ \rho _{%
{\large 2}}=1}}^{\infty }\varrho _{{\large 2}}\exp \left\{ -\frac{a_{2}}{12}%
\varrho _{2}\right\}  \notag \\
&=&O\left( \frac{1}{\tau _{1}^{2}}\right) \text{ \ }\left( \text{given Lemma
OA-1}\right)  \label{term4 2}
\end{eqnarray}

Similarly, for the third term on the right-hand side of expression (\ref%
{Euguhuvuw}) above, we can apply Lemma OA-3 with $p=2$ and $r=3$ to obtain%
\begin{eqnarray*}
&&\frac{1}{\tau _{1}^{4}}\dsum\limits_{\substack{ g,h,v,w=\left( r-1\right)
\tau +p  \\ g\leq h\leq v\leq w  \\ w-v\leq v-h,\text{ }v-h>0}}^{\left(
r-1\right) \tau +\tau _{1}+p-1}\left\vert E\left[ \left\{
u_{ig}u_{ih}-E\left( u_{ig}u_{ih}\right) \right\} u_{iv}u_{iw}\right]
\right\vert \\
&\leq &\frac{1}{\tau _{1}^{4}}\dsum\limits_{\substack{ g,h,v,w=\left(
r-1\right) \tau +p  \\ g\leq h\leq v\leq w  \\ w-v\leq v-h,\text{ }v-h>0}}%
^{\left( r-1\right) \tau +\tau _{1}+p-1}\left\{ 2\left( 2^{1-\frac{{\large 1}%
}{{\large 2}}}+1\right) \left[ a_{1}\exp \left\{ -a_{2}\left( v-h\right)
\right\} \right] ^{1-\frac{{\large 1}}{{\large 2}}-\frac{{\large 1}}{{\large %
3}}}\right. \\
&&\text{ \ \ \ \ \ \ \ \ \ \ \ \ \ \ \ \ \ \ \ \ \ \ \ }\left. \text{\ }%
\times \left( E\left\vert \left\{ u_{ig}u_{ih}-E\left( u_{ig}u_{ih}\right)
\right\} \right\vert ^{2}\right) ^{\frac{{\large 1}}{{\large 2}}}\left(
E\left\vert u_{iv}u_{iw}\right\vert ^{3}\right) ^{\frac{{\large 1}}{{\large 3%
}}}\right\}
\end{eqnarray*}%
Next, applications of H\"{o}lder's inequality yield%
\begin{eqnarray*}
E\left\vert u_{iv}u_{iw}\right\vert ^{3} &\leq &\left( E\left\vert
u_{iv}\right\vert ^{6}\right) ^{\frac{{\large 1}}{{\large 2}}}\left(
E\left\vert u_{iw}\right\vert ^{6}\right) ^{\frac{{\large 1}}{{\large 2}}} \\
&\leq &\left( \overline{C}\right) ^{\frac{{\large 1}}{{\large 2}}}\left( 
\overline{C}\right) ^{\frac{{\large 1}}{{\large 2}}}\text{ \ }\left( \text{%
by Assumption 2-3(b)}\right) \\
&=&\overline{C}<\infty
\end{eqnarray*}%
and%
\begin{eqnarray*}
E\left\vert \left\{ u_{ig}u_{ih}-E\left( u_{ig}u_{ih}\right) \right\}
\right\vert ^{2} &\leq &2\left( E\left\vert u_{ig}u_{ih}\right\vert
^{2}+E\left\vert u_{ig}u_{ih}\right\vert ^{2}\right) \\
&&\left( \text{by Lo\`{e}ve's }c_{r}\text{ inequality and Jensen's inequality%
}\right) \\
&=&4E\left\vert u_{ig}u_{ih}\right\vert ^{2} \\
&\leq &4\left[ \left( E\left\vert u_{ig}\right\vert ^{4}\right) ^{\frac{%
{\large 1}}{{\large 4}}}\left( E\left\vert u_{ih}\right\vert ^{4}\right) ^{%
\frac{{\large 1}}{{\large 4}}}\right] ^{2} \\
&\leq &4\left[ \left( E\left\vert u_{ig}\right\vert ^{6}\right) ^{\frac{%
{\large 1}}{{\large 6}}}\left( E\left\vert u_{ih}\right\vert ^{6}\right) ^{%
\frac{{\large 1}}{{\large 6}}}\right] ^{2}\text{ \ }\left( \text{by
Liapunov's inequality}\right) \\
&\leq &4\left( \sup_{i,t}E\left\vert u_{it}\right\vert ^{6}\right) ^{\frac{%
{\large 2}}{{\large 3}}} \\
&\leq &4\left( \overline{C}\right) ^{\frac{{\large 2}}{{\large 3}}}<\infty 
\text{ \ }\left( \text{by Assumption 2-3(b) }\right) \text{ }
\end{eqnarray*}%
Moreover, let $\varrho _{1}=v-h$ and $\varrho _{2}=w-v$ so that $v=h+$ $%
\varrho _{1}$ and $w=v+\varrho _{2}=h+\varrho _{1}+\varrho _{2}$. Using
these notations and the boundedness of $E\left\vert u_{iv}u_{iw}\right\vert
^{3}$ and $E\left\vert \left\{ u_{ig}u_{ih}-E\left( u_{ig}u_{ih}\right)
\right\} \right\vert ^{2}$ as shown above, we can further write%
\begin{eqnarray}
&&\frac{1}{\tau _{1}^{4}}\dsum\limits_{\substack{ g,h,v,w=\left( r-1\right)
\tau +p  \\ g\leq h\leq v\leq w  \\ w-v\leq v-h,\text{ }v-h>0}}^{\left(
r-1\right) \tau +\tau _{1}+p-1}\left\vert E\left[ \left\{
u_{ig}u_{ih}-E\left( u_{ig}u_{ih}\right) \right\} u_{iv}u_{iw}\right]
\right\vert  \notag \\
&\leq &\frac{1}{\tau _{1}^{4}}\dsum\limits_{\substack{ g,h,v,w=\left(
r-1\right) \tau +p  \\ g\leq h\leq v\leq w  \\ w-v\leq v-h,\text{ }v-h>0}}%
^{\left( r-1\right) \tau +\tau _{1}+p-1}\left\{ 2\left( 2^{1-\frac{{\large 1}%
}{{\large 2}}}+1\right) \left[ a_{1}\exp \left\{ -a_{2}\left( v-h\right)
\right\} \right] ^{1-\frac{{\large 1}}{{\large 2}}-\frac{{\large 1}}{{\large %
3}}}\right.  \notag \\
&&\text{ \ \ \ \ \ \ \ \ \ \ \ \ \ \ \ \ \ \ \ \ \ \ \ \ }\left. \times
\left( E\left\vert \left\{ u_{ig}u_{ih}-E\left( u_{ig}u_{ih}\right) \right\}
\right\vert ^{2}\right) ^{\frac{{\large 1}}{{\large 2}}}\left( E\left\vert
u_{iv}u_{iw}\right\vert ^{3}\right) ^{\frac{{\large 1}}{{\large 3}}}\right\}
\notag \\
&\leq &\frac{1}{\tau _{1}^{4}}\dsum\limits_{\substack{ g,h,v,w=\left(
r-1\right) \tau +p  \\ g\leq h\leq v\leq w  \\ w-v\leq v-h,\text{ }v-h>0}}%
^{\left( r-1\right) \tau +\tau _{1}+p-1}2\left( 2^{\frac{{\large 1}}{{\large %
2}}}+1\right) \left[ a_{1}\exp \left\{ -a_{2}\left( v-h\right) \right\} %
\right] ^{\frac{{\large 1}}{{\large 6}}}\left( 4\overline{C}^{\frac{{\large 2%
}}{{\large 3}}}\right) ^{\frac{{\large 1}}{{\large 2}}}\left( \overline{C}%
\right) ^{\frac{{\large 1}}{{\large 3}}}  \notag \\
&\leq &\frac{C^{\ast }}{\tau _{1}^{4}}\dsum\limits_{\substack{ %
g,h,v,w=\left( r-1\right) \tau +p  \\ g\leq h\leq v\leq w  \\ w-v\leq v-h,%
\text{ }v-h>0}}^{\left( r-1\right) \tau +\tau _{1}+p-1}\exp \left\{ -\frac{%
a_{2}}{6}\varrho _{1}\right\} \text{ }  \notag \\
&&\left( \text{for some constant }C^{\ast }\text{ such that }4\left( 2^{%
\frac{{\large 1}}{{\large 2}}}+1\right) \overline{C}^{\frac{{\large 2}}{%
{\large 3}}}a_{1}^{\frac{{\large 1}}{{\large 6}}}\leq C^{\ast }<\infty
\right)  \notag \\
&\leq &\frac{C^{\ast }}{\tau _{1}^{4}}\dsum\limits_{\substack{ g=\left(
r-1\right) \tau +p}}^{\left( r-1\right) \tau +\tau _{1}+p-1}\dsum\limits 
_{\substack{ h=\left( r-1\right) \tau +p}}^{\left( r-1\right) \tau +\tau
_{1}+p-1}\dsum\limits_{\varrho _{{\large 1}}=1}^{\infty
}\dsum\limits_{\varrho _{{\large 2}}=0}^{\varrho _{{\large 1}}}\exp \left\{ -%
\frac{a_{2}}{6}\varrho _{1}\right\}  \notag \\
&=&\frac{C^{\ast }}{\tau _{1}^{2}}\dsum\limits_{\substack{ \rho _{{\large 1}%
}=1}}^{\infty }\left( \varrho _{{\large 1}}+1\right) \exp \left\{ -\frac{%
a_{2}}{6}\varrho _{1}\right\}  \notag \\
&=&O\left( \frac{1}{\tau _{1}^{2}}\right) \text{ \ }\left( \text{given Lemma
OA-1}\right)  \label{term4 3}
\end{eqnarray}

Finally, consider the fourth term on the right-hand side of expression (\ref%
{Euguhuvuw}) above. For this term, we apply the result given in part (a) to
obtain 
\begin{eqnarray}
&&\frac{1}{\tau _{1}^{4}}\dsum\limits_{\substack{ g,h,v,w=\left( r-1\right)
\tau +p  \\ g\leq h\leq v\leq w  \\ w-h>0}}^{\left( r-1\right) \tau +\tau
_{1}+p-1}\left\vert E\left( u_{ig}u_{ih}\right) \right\vert \left\vert
E\left( u_{iv}u_{iw}\right) \right\vert  \notag \\
&\leq &\left( \frac{1}{\tau _{1}^{2}}\dsum\limits_{\substack{ g,h=\left(
r-1\right) \tau +p  \\ g\leq h}}^{\left( r-1\right) \tau +\tau
_{1}+p-1}\left\vert E\left( u_{ig}u_{ih}\right) \right\vert \right) \left( 
\frac{1}{\tau _{1}^{2}}\dsum\limits_{\substack{ v,w=\left( r-1\right) \tau
+p  \\ v\leq w}}^{\left( r-1\right) \tau +\tau _{1}+p-1}\left\vert E\left(
u_{iv}u_{iw}\right) \right\vert \right)  \notag \\
&=&O\left( \frac{1}{\tau _{1}^{2}}\right) \text{.}  \label{term4 4}
\end{eqnarray}

\noindent It follows from expressions (\ref{Euguhuvuw})-(\ref{term4 4}) that%
\begin{eqnarray*}
&&\frac{1}{\tau _{1}^{4}}\dsum\limits_{\substack{ g,h,v,w=\left( r-1\right)
\tau +p  \\ g\leq h\leq v\leq w}}^{\left( r-1\right) \tau +\tau
_{1}+p-1}\left\vert E\left[ u_{ig}u_{ih}u_{iv}u_{iw}\right] \right\vert \\
&\leq &\frac{1}{\tau _{1}^{4}}\dsum\limits_{\substack{ g,h=\left( r-1\right)
\tau +p  \\ g\leq h}}^{\left( r-1\right) \tau +\tau _{1}+p-1}\left\vert E%
\left[ u_{ig}u_{ih}^{3}\right] \right\vert +\frac{1}{\tau _{1}^{4}}%
\dsum\limits_{\substack{ g,h,v,w=\left( r-1\right) \tau +p  \\ g\leq h\leq
v\leq w  \\ w-v>v-h,\text{ }w-v>0}}^{\left( r-1\right) \tau +\tau
_{1}+p-1}\left\vert E\left[ \left\{ u_{ig}u_{ih}-E\left( u_{ig}u_{ih}\right)
\right\} u_{iv}u_{iw}\right] \right\vert \\
&&+\frac{1}{\tau _{1}^{4}}\dsum\limits_{\substack{ g,h,v,w=\left( r-1\right)
\tau +p  \\ g\leq h\leq v\leq w  \\ w-v\leq v-h,\text{ }v-h>0}}^{\left(
r-1\right) \tau +\tau _{1}+p-1}\left\vert E\left[ \left\{
u_{ig}u_{ih}-E\left( u_{ig}u_{ih}\right) \right\} u_{iv}u_{iw}\right]
\right\vert \\
&&+\frac{1}{\tau _{1}^{4}}\dsum\limits_{\substack{ g,h,v,w=\left( r-1\right)
\tau +p  \\ g\leq h\leq v\leq w  \\ w-h>0}}^{\left( r-1\right) \tau +\tau
_{1}+p-1}\left\vert E\left( u_{ig}u_{ih}\right) \right\vert \left\vert
E\left( u_{iv}u_{iw}\right) \right\vert \\
&=&O\left( \frac{1}{\tau _{1}^{2}}\right) \text{. }\square
\end{eqnarray*}

\medskip

\noindent \textbf{Lemma OA-5:} Suppose that Assumptions 2-1, 2-2(a)-(b),
2-5, and 2-6 hold. Then, there exists a positve constant $\overline{C}$ such
that 
\begin{equation*}
E\left\Vert \underline{W}_{t}\right\Vert _{2}^{6}\leq \overline{C}<\infty 
\text{ for all }t
\end{equation*}%
and, thus,%
\begin{equation*}
E\left\Vert \underline{Y}_{t}\right\Vert _{2}^{6}\leq \overline{C}<\infty 
\text{ and }E\left\Vert \underline{F}_{t}\right\Vert _{2}^{6}\leq \overline{C%
}<\infty \text{ for all }t\text{,}
\end{equation*}%
where%
\begin{equation*}
\underset{dp\times 1}{\underline{Y}_{t}}=\left( 
\begin{array}{c}
Y_{t} \\ 
Y_{t-1} \\ 
\vdots \\ 
Y_{t-p{\LARGE +}1}%
\end{array}%
\right) \text{, and}\underset{Kp\times 1}{\underline{F}_{t}}=\left( 
\begin{array}{c}
F_{t} \\ 
F_{t-1} \\ 
\vdots \\ 
F_{t-p{\LARGE +}1}%
\end{array}%
\right) \text{. }
\end{equation*}

\medskip

\noindent \textbf{Proof of Lemma OA-5:}

To proceed, note that, given Assumption 2-1, we can write the vector
moving-average (VMA) representation of the companion form of the FAVAR model
as 
\begin{eqnarray}
\underline{W}_{t} &=&\left( I_{\left( d{\LARGE +}K\right) p}-A\right)
^{-1}\alpha +\dsum\limits_{j=0}^{\infty }A^{j}E_{t-j}  \notag \\
&=&\left( I_{\left( d{\LARGE +}K\right) p}-A\right) ^{-1}J_{d{\LARGE +}%
K}^{\prime }J_{d{\LARGE +}K}\alpha +\dsum\limits_{j=0}^{\infty }A^{j}J_{d%
{\LARGE +}K}^{\prime }J_{d{\LARGE +}K}E_{t-j}  \notag \\
&=&\left( I_{\left( d{\LARGE +}K\right) p}-A\right) ^{-1}J_{d{\LARGE +}%
K}^{\prime }\mu +\dsum\limits_{j=0}^{\infty }A^{j}J_{d{\LARGE +}K}^{\prime
}\varepsilon _{t-j}\text{,}  \label{VMA representation of W}
\end{eqnarray}%
where%
\begin{eqnarray*}
\underline{W}_{t} &=&\left( 
\begin{array}{c}
W_{t} \\ 
W_{t-1} \\ 
\vdots \\ 
W_{t-p{\LARGE +}2} \\ 
W_{t-p{\large +}1}%
\end{array}%
\right) ,\text{ }E_{t}=\left( 
\begin{array}{c}
\varepsilon _{t} \\ 
0 \\ 
\vdots \\ 
0 \\ 
0%
\end{array}%
\right) \text{, }\alpha =\left( 
\begin{array}{c}
\mu \\ 
0 \\ 
\vdots \\ 
0 \\ 
0%
\end{array}%
\right) \text{, } \\
\underset{\left( d{\LARGE +}K\right) \times \left( d{\LARGE +}K\right) p}{%
J_{d{\LARGE +}K}} &=&\left[ 
\begin{array}{ccccc}
I_{d{\LARGE +}K} & 0 & \cdots & 0 & 0%
\end{array}%
\right] \text{, and }A=\left( 
\begin{array}{ccccc}
A_{1} & A_{2} & \cdots & A_{p-1} & A_{p} \\ 
I_{d{\LARGE +}K} & 0 & \cdots & \cdots & 0 \\ 
0 & \ddots & \ddots &  & \vdots \\ 
\vdots & \ddots & \ddots & \ddots & \vdots \\ 
0 & \cdots & 0 & I_{d{\LARGE +}K} & 0%
\end{array}%
\right) \text{.}
\end{eqnarray*}%
By the triangle inequality, 
\begin{equation*}
\left\Vert \underline{W}_{t}\right\Vert _{2}\leq \left\Vert \left( I_{\left(
d{\LARGE +}K\right) p}-A\right) ^{-1}J_{d{\LARGE +}K}^{\prime }\mu
\right\Vert _{2}+\left\Vert \dsum\limits_{j=0}^{\infty }A^{j}J_{d{\LARGE +}%
K}^{\prime }\varepsilon _{t-j}\right\Vert _{2}
\end{equation*}%
Moreover, using the inequality $\left\vert
\dsum\nolimits_{i=1}^{m}a_{i}\right\vert ^{r}\leq
m^{r-1}\dsum\nolimits_{i=1}^{m}\left\vert a_{i}\right\vert ^{r}$ for $r\geq
1 $, we get 
\begin{equation*}
\left\Vert \underline{W}_{t}\right\Vert _{2}^{6}\leq 2^{5}\left( \left\Vert
\left( I_{\left( d{\LARGE +}K\right) p}-A\right) ^{-1}J_{d{\LARGE +}%
K}^{\prime }\mu \right\Vert _{2}^{6}+\left\Vert \dsum\limits_{j=0}^{\infty
}A^{j}J_{d{\LARGE +}K}^{\prime }\varepsilon _{t-j}\right\Vert _{2}^{6}\right)
\end{equation*}%
so that%
\begin{equation}
E\left\Vert \underline{W}_{t}\right\Vert _{2}^{6}\leq 32\left\Vert \left(
I_{\left( d{\LARGE +}K\right) p}-A\right) ^{-1}J_{d{\LARGE +}K}^{\prime }\mu
\right\Vert _{2}^{6}+32E\left\Vert \dsum\limits_{j=0}^{\infty }A^{j}J_{d%
{\LARGE +}K}^{\prime }\varepsilon _{t-j}\right\Vert _{2}^{6}
\label{EW^6 ineq}
\end{equation}

Focusing first on the first term on the right-hand side of the inequality (%
\ref{EW^6 ineq}), we note that 
\begin{eqnarray*}
\left\Vert \left( I_{\left( d{\LARGE +}K\right) p}-A\right) ^{-1}J_{d{\LARGE %
+}K}^{\prime }\mu \right\Vert _{2}^{6} &=&\left( \mu ^{\prime }J_{d{\LARGE +}%
K}\left( I_{\left( d{\LARGE +}K\right) p}-A\right) ^{-1\prime }\left(
I_{\left( d{\LARGE +}K\right) p}-A\right) ^{-1}J_{d{\LARGE +}K}^{\prime }\mu
\right) ^{3} \\
&=&\left( \mu ^{\prime }J_{d{\LARGE +}K}\left[ \left( I_{\left( d{\LARGE +}%
K\right) p}-A\right) \left( I_{\left( d{\LARGE +}K\right) p}-A\right)
^{\prime }\right] ^{-1}J_{d{\LARGE +}K}^{\prime }\mu \right) ^{3} \\
&\leq &\left( \frac{1}{\lambda _{\min }\left\{ \left( I_{\left( d{\LARGE +}%
K\right) p}-A\right) \left( I_{\left( d{\LARGE +}K\right) p}-A\right)
^{\prime }\right\} }\right) ^{3}\left( \mu ^{\prime }J_{d{\LARGE +}K}J_{d%
{\LARGE +}K}^{\prime }\mu \right) ^{3} \\
&=&\left( \frac{1}{\lambda _{\min }\left\{ \left( I_{\left( d{\LARGE +}%
K\right) p}-A\right) \left( I_{\left( d{\LARGE +}K\right) p}-A\right)
^{\prime }\right\} }\right) ^{3}\left( \mu ^{\prime }\mu \right) ^{3}
\end{eqnarray*}%
Now, by Assumption 2-6, there exists a constant $\underline{C}>0$ such that 
\begin{eqnarray*}
\lambda _{\min }\left\{ \left( I_{\left( d{\LARGE +}K\right) p}-A\right)
\left( I_{\left( d{\LARGE +}K\right) p}-A\right) ^{\prime }\right\}
&=&\lambda _{\min }\left\{ \left( I_{\left( d{\LARGE +}K\right) p}-A\right)
^{\prime }\left( I_{\left( d{\LARGE +}K\right) p}-A\right) \right\} \\
&=&\sigma _{\min }^{2}\left( I_{\left( d{\LARGE +}K\right) p}-A\right) \\
&\geq &\underline{C}\lambda _{\min }^{2}\left( I_{\left( d{\LARGE +}K\right)
p}-A\right) \\
&\geq &\underline{C}\left[ 1-\phi _{\max }\right] ^{2} \\
&>&0
\end{eqnarray*}%
where $\phi _{\max }=\max \left\{ \left\vert \lambda _{\max }\left( A\right)
\right\vert ,\left\vert \lambda _{\min }\left( A\right) \right\vert \right\} 
$ and where $0<\phi _{\max }<1$ since, by Assumption 2-1, all eigenvalues of 
$A$ have modulus less than $1$. It follows by Assumption 2-5 that, there
exists a positive constant $\overline{C}_{1}$ such that%
\begin{eqnarray*}
\left\Vert \left( I_{\left( d{\LARGE +}K\right) p}-A\right) ^{-1}J_{d{\LARGE %
+}K}^{\prime }\mu \right\Vert _{2}^{6} &\leq &\left( \frac{1}{\lambda _{\min
}\left\{ \left( I_{\left( d{\LARGE +}K\right) p}-A\right) \left( I_{\left( d%
{\LARGE +}K\right) p}-A\right) ^{\prime }\right\} }\right) ^{3}\left( \mu
^{\prime }\mu \right) ^{3} \\
&\leq &\frac{\left\Vert \mu \right\Vert _{2}^{6}}{\underline{C}^{3}\left[
1-\phi _{\max }\right] ^{6}}\leq \overline{C}_{1}<\infty \text{. }
\end{eqnarray*}

To show the boundedness of the second term on the right-hand side of the
inequality (\ref{EW^6 ineq}), let $e_{g,\left( d+K\right) p}$ be a $\left(
d+K\right) p\times 1$ elementary vector whose $g^{th}$ component is $1$ and
all other components are $0$ for $g\in \left\{ 1,2,...,\left( d+K\right)
p\right\} $, and note that 
\begin{eqnarray*}
\left\Vert \dsum\limits_{j=0}^{\infty }A^{j}J_{d{\LARGE +}K}^{\prime
}\varepsilon _{t-j}\right\Vert _{2}^{2} &=&\dsum\limits_{g=1}^{\left( d%
{\LARGE +}K\right) p}\left( \dsum\limits_{j=0}^{\infty }e_{g,\left(
d+K\right) p}^{\prime }A^{j}J_{d{\LARGE +}K}^{\prime }\varepsilon
_{t-j}\right) ^{2} \\
&=&\dsum\limits_{g=1}^{\left( d{\LARGE +}K\right)
p}\dsum\limits_{j=0}^{\infty }\dsum\limits_{k=0}^{\infty }e_{g,\left(
d+K\right) p}^{\prime }A^{j}J_{d{\LARGE +}K}^{\prime }\varepsilon
_{t-j}\varepsilon _{t-k}^{\prime }J_{d{\LARGE +}K}\left( A^{\prime }\right)
^{k}e_{g,\left( d+K\right) p}
\end{eqnarray*}%
from which we obtain, by applying the inequality $\left\vert
\dsum\nolimits_{i=1}^{m}a_{i}\right\vert ^{r}\leq
m^{r-1}\dsum\nolimits_{i=1}^{m}\left\vert a_{i}\right\vert ^{r}$ for $r\geq
1 $%
\begin{eqnarray*}
&&\left\Vert \dsum\limits_{j=0}^{\infty }A^{j}J_{d{\LARGE +}K}^{\prime
}\varepsilon _{t-j}\right\Vert _{2}^{6} \\
&=&\left[ \dsum\limits_{g=1}^{\left( d{\LARGE +}K\right) p}\left(
\dsum\limits_{j=0}^{\infty }e_{g,\left( d+K\right) p}^{\prime }A^{j}J_{d%
{\LARGE +}K}^{\prime }\varepsilon _{t-j}\right) ^{2}\right] ^{3} \\
&\leq &\left[ \left( d+K\right) p\right] ^{2}\dsum\limits_{g=1}^{\left( d%
{\LARGE +}K\right) p}\left( \dsum\limits_{j=0}^{\infty }e_{g,\left(
d+K\right) p}^{\prime }A^{j}J_{d{\LARGE +}K}^{\prime }\varepsilon
_{t-j}\right) ^{6} \\
&=&\left[ \left( d+K\right) p\right] ^{2}\dsum\limits_{g=1}^{\left( d{\LARGE %
+}K\right) p}\left\{ \dsum\limits_{j=0}^{\infty }\dsum\limits_{k=0}^{\infty
}\dsum\limits_{i=0}^{\infty }\dsum\limits_{\ell =0}^{\infty
}\dsum\limits_{r=0}^{\infty }\dsum\limits_{s=0}^{\infty }e_{g,\left(
d+K\right) p}^{\prime }A^{j}J_{d{\LARGE +}K}^{\prime }\varepsilon
_{t-j}\varepsilon _{t-k}^{\prime }J_{d{\LARGE +}K}\left( A^{\prime }\right)
^{k}e_{g,\left( d+K\right) p}\right. \\
&&\text{ \ \ \ \ \ \ \ \ }\left. \times e_{g,\left( d+K\right) p}^{\prime
}A^{i}J_{d}^{\prime }\varepsilon _{t-i}\varepsilon _{t-\ell }^{\prime }J_{d%
{\LARGE +}K}\left( A^{\prime }\right) ^{\ell }e_{g,\left( d+K\right)
p}e_{g,\left( d+K\right) p}^{\prime }A^{r}J_{d{\LARGE +}K}^{\prime
}\varepsilon _{t-r}\varepsilon _{t-s}^{\prime }J_{d}\left( A^{\prime
}\right) ^{s}e_{g,\left( d+K\right) p}\right\}
\end{eqnarray*}%
Hence,%
\begin{eqnarray*}
&&E\left\Vert \dsum\limits_{j=0}^{\infty }A^{j}J_{d{\LARGE +}K}^{\prime
}\varepsilon _{t-j}\right\Vert _{2}^{6} \\
&\leq &\left[ \left( d+K\right) p\right] ^{2}\dsum\limits_{g=1}^{\left( d%
{\LARGE +}K\right) p}\dsum\limits_{j=0}^{\infty }E\left\vert e_{g,\left(
d+K\right) p}^{\prime }A^{j}J_{d{\LARGE +}K}^{\prime }\varepsilon
_{t-j}\right\vert ^{6} \\
&&+\left[ \left( d+K\right) p\right] ^{2}\dsum\limits_{g=1}^{\left( d{\LARGE %
+}K\right) p}\binom{6}{3}\left( \dsum\limits_{j=0}^{\infty }E\left\vert
e_{g,\left( d+K\right) p}^{\prime }A^{j}J_{d{\LARGE +}K}^{\prime
}\varepsilon _{t-j}\right\vert ^{3}\right) ^{2} \\
&&+\left[ \left( d+K\right) p\right] ^{2}\dsum\limits_{g=1}^{\left( d{\LARGE %
+}K\right) p}\binom{6}{2}\binom{4}{2}\left( \dsum\limits_{j=0}^{\infty
}E\left\vert e_{g,\left( d+K\right) p}^{\prime }A^{j}J_{d{\LARGE +}%
K}^{\prime }\varepsilon _{t-j}\right\vert ^{2}\right) ^{3} \\
&&+\left[ \left( d+K\right) p\right] ^{2}\dsum\limits_{g=1}^{\left( d{\LARGE %
+}K\right) p}\binom{6}{4}\dsum\limits_{j=0}^{\infty }E\left\vert e_{g,\left(
d+K\right) p}^{\prime }A^{j}J_{d{\LARGE +}K}^{\prime }\varepsilon
_{t-j}\right\vert ^{4}\dsum\limits_{k=0}^{\infty }E\left\vert e_{g,\left(
d+K\right) p}^{\prime }A^{k}J_{d{\LARGE +}K}^{\prime }\varepsilon
_{t-k}\right\vert ^{2} \\
&=&\left[ \left( d+K\right) p\right] ^{2}\dsum\limits_{g=1}^{\left( d{\LARGE %
+}K\right) p}\dsum\limits_{j=0}^{\infty }E\left\vert e_{g,\left( d+K\right)
p}^{\prime }A^{j}J_{d{\LARGE +}K}^{\prime }\varepsilon _{t-j}\right\vert ^{6}
\\
&&+20\left[ \left( d+K\right) p\right] ^{2}\dsum\limits_{g=1}^{\left( d%
{\LARGE +}K\right) p}\left( \dsum\limits_{j=0}^{\infty }E\left\vert
e_{g,\left( d+K\right) p}^{\prime }A^{j}J_{d{\LARGE +}K}^{\prime
}\varepsilon _{t-j}\right\vert ^{3}\right) ^{2} \\
&&+90\left[ \left( d+K\right) p\right] ^{2}\dsum\limits_{g=1}^{\left( d%
{\LARGE +}K\right) p}\left( \dsum\limits_{j=0}^{\infty }E\left\vert
e_{g,\left( d+K\right) p}^{\prime }A^{j}J_{d{\LARGE +}K}^{\prime
}\varepsilon _{t-j}\right\vert ^{2}\right) ^{3} \\
&&+15\left[ \left( d+K\right) p\right] ^{2}\dsum\limits_{g=1}^{\left( d%
{\LARGE +}K\right) p}\dsum\limits_{j=0}^{\infty }E\left\vert e_{g,\left(
d+K\right) p}^{\prime }A^{j}J_{d{\LARGE +}K}^{\prime }\varepsilon
_{t-j}\right\vert ^{4}\dsum\limits_{k=0}^{\infty }E\left\vert e_{g,\left(
d+K\right) p}^{\prime }A^{k}J_{d{\LARGE +}K}^{\prime }\varepsilon
_{t-k}\right\vert ^{2}
\end{eqnarray*}%
Next, applying the Cauchy-Schwarz inequality, we further obtain%
\begin{eqnarray*}
&&E\left\Vert \dsum\limits_{j=0}^{\infty }A^{j}J_{d{\LARGE +}K}^{\prime
}\varepsilon _{t-j}\right\Vert _{2}^{6} \\
&\leq &\left[ \left( d+K\right) p\right] ^{2}\dsum\limits_{g=1}^{\left( d%
{\LARGE +}K\right) p}\dsum\limits_{j=0}^{\infty }\left[ e_{g,\left(
d+K\right) p}^{\prime }A^{j}J_{d{\LARGE +}K}^{\prime }J_{d{\LARGE +}K}\left(
A^{j}\right) ^{\prime }e_{g,\left( d+K\right) p}\right] ^{3}E\left\Vert
\varepsilon _{t-j}\right\Vert _{2}^{6} \\
&&+20\left[ \left( d+K\right) p\right] ^{2}\dsum\limits_{g=1}^{\left( d%
{\LARGE +}K\right) p}\left( \dsum\limits_{j=0}^{\infty }\left[ e_{g,\left(
d+K\right) p}^{\prime }A^{j}J_{d{\LARGE +}K}^{\prime }J_{d{\LARGE +}K}\left(
A^{j}\right) ^{\prime }e_{g,\left( d+K\right) p}\right] ^{\frac{{\Large 3}}{%
{\Large 2}}}E\left\Vert \varepsilon _{t-j}\right\Vert _{2}^{3}\right) ^{2} \\
&&+90\left[ \left( d+K\right) p\right] ^{2}\dsum\limits_{g=1}^{\left( d%
{\LARGE +}K\right) p}\left( \dsum\limits_{j=0}^{\infty }\left[ e_{g,\left(
d+K\right) p}^{\prime }A^{j}J_{d{\LARGE +}K}^{\prime }J_{d{\LARGE +}K}\left(
A^{j}\right) ^{\prime }e_{g,\left( d+K\right) p}\right] E\left\Vert
\varepsilon _{t-j}\right\Vert _{2}^{2}\right) ^{3} \\
&&+15\left[ \left( d+K\right) p\right] ^{2}\dsum\limits_{g=1}^{\left( d%
{\LARGE +}K\right) p}\left\{ \dsum\limits_{j=0}^{\infty }\left[ e_{g,\left(
d+K\right) p}^{\prime }A^{j}J_{d{\LARGE +}K}^{\prime }J_{d{\LARGE +}K}\left(
A^{j}\right) ^{\prime }e_{g,\left( d+K\right) p}\right] ^{2}E\left\Vert
\varepsilon _{t-j}\right\Vert _{2}^{4}\right. \\
&&\text{ \ \ \ \ \ \ \ \ \ \ \ \ \ \ \ \ \ \ \ \ \ \ \ \ \ \ \ \ \ \ \ \ \ \
\ }\left. \times \dsum\limits_{k=0}^{\infty }\left[ e_{g,\left( d+K\right)
p}^{\prime }A^{k}J_{d{\LARGE +}K}^{\prime }J_{d{\LARGE +}K}\left(
A^{k}\right) ^{\prime }e_{g,\left( d+K\right) p}\right] E\left\Vert
\varepsilon _{t-k}\right\Vert _{2}^{2}\right\} \\
&\leq &\left[ \left( d+K\right) p\right] ^{2}\dsum\limits_{g=1}^{\left( d%
{\LARGE +}K\right) p}\dsum\limits_{j=0}^{\infty }\left[ e_{g,\left(
d+K\right) p}^{\prime }A^{j}\left( A^{j}\right) ^{\prime }e_{g,\left(
d+K\right) p}\right] ^{3}E\left\Vert \varepsilon _{t-j}\right\Vert _{2}^{6}
\\
&&+20\left[ \left( d+K\right) p\right] ^{2}\dsum\limits_{g=1}^{\left( d%
{\LARGE +}K\right) p}\left( \dsum\limits_{j=0}^{\infty }\left[ e_{g,\left(
d+K\right) p}^{\prime }A^{j}\left( A^{j}\right) ^{\prime }e_{g,\left(
d+K\right) p}\right] ^{\frac{{\Large 3}}{{\Large 2}}}E\left\Vert \varepsilon
_{t-j}\right\Vert _{2}^{3}\right) ^{2} \\
&&+90\left[ \left( d+K\right) p\right] ^{2}\dsum\limits_{g=1}^{\left( d%
{\LARGE +}K\right) p}\left( \dsum\limits_{j=0}^{\infty }\left[ e_{g,\left(
d+K\right) p}^{\prime }A^{j}\left( A^{j}\right) ^{\prime }e_{g,\left(
d+K\right) p}\right] E\left\Vert \varepsilon _{t-j}\right\Vert
_{2}^{2}\right) ^{3} \\
&&+15\left[ \left( d+K\right) p\right] ^{2}\dsum\limits_{g=1}^{\left( d%
{\LARGE +}K\right) p}\left\{ \dsum\limits_{j=0}^{\infty }\left[ e_{g,\left(
d+K\right) p}^{\prime }A^{j}\left( A^{j}\right) ^{\prime }e_{g,\left(
d+K\right) p}\right] ^{2}E\left\Vert \varepsilon _{t-j}\right\Vert
_{2}^{4}\right. \\
&&\text{ \ \ \ \ \ \ \ \ \ \ \ \ \ \ \ \ \ \ \ \ \ \ \ \ \ \ \ \ \ \ \ \ \ \ 
}\left. \times \dsum\limits_{k=0}^{\infty }\left[ e_{g,\left( d+K\right)
p}^{\prime }A^{k}\left( A^{k}\right) ^{\prime }e_{g,\left( d+K\right) p}%
\right] E\left\Vert \varepsilon _{t-k}\right\Vert _{2}^{2}\right\}
\end{eqnarray*}%
In addition, observe that, for every $g\in \left\{ 1,2,...,\left( d+K\right)
p\right\} $%
\begin{eqnarray*}
&&e_{g,\left( d+K\right) p}^{\prime }A^{j}\left( A^{j}\right) ^{\prime
}e_{g,\left( d+K\right) p} \\
&\leq &\lambda _{\max }\left\{ A^{j}\left( A^{j}\right) ^{\prime }\right\} \\
&=&\lambda _{\max }\left\{ \left( A^{j}\right) ^{\prime }A^{j}\right\} \\
&=&\sigma _{\max }^{2}\left( A^{j}\right) \\
&\leq &C\max \left\{ \left\vert \lambda _{\max }\left( A^{j}\right)
\right\vert ^{2},\left\vert \lambda _{\min }\left( A^{j}\right) \right\vert
^{2}\right\} \text{ }\left( \text{by Assumption 2-6}\right) \\
&=&C\max \left\{ \left\vert \lambda _{\max }\left( A\right) \right\vert
^{2j},\left\vert \lambda _{\min }\left( A\right) \right\vert ^{2j}\right\} \\
&=&C\phi _{\max }^{2j}\text{ \ }
\end{eqnarray*}%
where $\phi _{\max }=\max \left\{ \left\vert \lambda _{\max }\left( A\right)
\right\vert ,\left\vert \lambda _{\min }\left( A\right) \right\vert \right\} 
$ and where $0<\phi _{\max }<1$ given that Assumption 2-1 implies that all
eigenvalues of $A$ have modulus less than $1$. Now, in light of Assumption
2-2(b), we can set $C\geq 1$ to be a constant such that $E\left\Vert
\varepsilon _{t-j}\right\Vert _{2}^{6}\leq C<\infty $, so that, by
Liapunov's inequality,%
\begin{eqnarray*}
E\left\Vert \varepsilon _{t-j}\right\Vert _{2}^{2} &\leq &\left( E\left\Vert
\varepsilon _{t-j}\right\Vert _{2}^{6}\right) ^{\frac{{\Large 1}}{{\Large 3}}%
}\leq C^{\frac{{\Large 1}}{{\Large 3}}}\text{, }E\left\Vert \varepsilon
_{t-j}\right\Vert _{2}^{3}\leq \left( E\left\Vert \varepsilon
_{t-j}\right\Vert _{2}^{6}\right) ^{\frac{{\Large 1}}{{\Large 2}}}\leq C^{%
\frac{{\Large 1}}{{\Large 2}}}\text{,} \\
E\left\Vert \varepsilon _{t-j}\right\Vert _{2}^{4} &\leq &\left( E\left\Vert
\varepsilon _{t-j}\right\Vert _{2}^{6}\right) ^{\frac{{\Large 2}}{{\Large 3}}%
}\leq C^{\frac{{\Large 2}}{{\Large 3}}}\text{, }
\end{eqnarray*}%
and, thus, 
\begin{eqnarray*}
&&E\left\Vert \dsum\limits_{j=0}^{\infty }A^{j}J_{d{\LARGE +}K}^{\prime
}\varepsilon _{t-j}\right\Vert _{2}^{6} \\
&\leq &\left[ \left( d+K\right) p\right] ^{2}\dsum\limits_{g=1}^{\left( d%
{\LARGE +}K\right) p}\dsum\limits_{j=0}^{\infty }\left[ e_{g,\left(
d+K\right) p}^{\prime }A^{j}\left( A^{j}\right) ^{\prime }e_{g,\left(
d+K\right) p}\right] ^{3}E\left\Vert \varepsilon _{t-j}\right\Vert _{2}^{6}
\\
&&+20\left[ \left( d+K\right) p\right] ^{2}\dsum\limits_{g=1}^{\left( d%
{\LARGE +}K\right) p}\left( \dsum\limits_{j=0}^{\infty }\left[ e_{g,\left(
d+K\right) p}^{\prime }A^{j}\left( A^{j}\right) ^{\prime }e_{g,\left(
d+K\right) p}\right] ^{\frac{{\Large 3}}{{\Large 2}}}E\left\Vert \varepsilon
_{t-j}\right\Vert _{2}^{3}\right) ^{2} \\
&&+90\left[ \left( d+K\right) p\right] ^{2}\dsum\limits_{g=1}^{\left( d%
{\LARGE +}K\right) p}\left( \dsum\limits_{j=0}^{\infty }\left[ e_{g,\left(
d+K\right) p}^{\prime }A^{j}\left( A^{j}\right) ^{\prime }e_{g,\left(
d+K\right) p}\right] E\left\Vert \varepsilon _{t-j}\right\Vert
_{2}^{2}\right) ^{3} \\
&&+15\left[ \left( d+K\right) p\right] ^{2}\dsum\limits_{g=1}^{\left( d%
{\LARGE +}K\right) p}\left\{ \dsum\limits_{j=0}^{\infty }\left[ e_{g,\left(
d+K\right) p}^{\prime }A^{j}\left( A^{j}\right) ^{\prime }e_{g,\left(
d+K\right) p}\right] ^{2}E\left\Vert \varepsilon _{t-j}\right\Vert
_{2}^{4}\right. \\
&&\text{ \ \ \ \ \ \ \ \ \ \ \ \ \ \ \ \ \ \ \ \ \ \ \ \ \ \ \ \ \ \ \ \ \ \ 
}\left. \times \dsum\limits_{k=0}^{\infty }\left[ e_{g,\left( d+K\right)
p}^{\prime }A^{k}\left( A^{k}\right) ^{\prime }e_{g,\left( d+K\right) p}%
\right] E\left\Vert \varepsilon _{t-k}\right\Vert _{2}^{2}\right\} \\
&\leq &C\left[ \left( d+K\right) p\right] ^{2}\left\{
\dsum\limits_{g=1}^{\left( d{\LARGE +}K\right) p}\dsum\limits_{j=0}^{\infty
}\phi _{\max }^{6j}+20\dsum\limits_{g=1}^{\left( d{\LARGE +}K\right)
p}\left( \dsum\limits_{j=0}^{\infty }\phi _{\max }^{3j}\right)
^{2}+90\dsum\limits_{g=1}^{\left( d{\LARGE +}K\right) p}\left(
\dsum\limits_{j=0}^{\infty }\phi _{\max }^{2j}\right) ^{3}\right. \\
&&\text{ \ \ \ \ \ \ \ \ \ \ \ \ \ \ \ \ \ \ \ \ \ \ \ }\left.
+15\dsum\limits_{g=1}^{\left( d{\LARGE +}K\right) p}\left(
\dsum\limits_{j=0}^{\infty }\phi _{\max }^{4j}\right) \left(
\dsum\limits_{k=0}^{\infty }\phi _{\max }^{2k}\right) \right\} \\
&\leq &C\left[ \left( d+K\right) p\right] ^{3} \\
&&\times \left\{ \frac{1}{1-\phi _{\max }^{6}}+20\left( \frac{1}{1-\phi
_{\max }^{3}}\right) ^{2}+90\left( \frac{1}{1-\phi _{\max }^{2}}\right)
^{3}+15\left( \frac{1}{1-\phi _{\max }^{4}}\right) \left( \frac{1}{1-\phi
_{\max }^{2}}\right) \right\} \\
&\leq &\overline{C}_{2}<\infty
\end{eqnarray*}%
for some constant such that%
\begin{eqnarray*}
&&\overline{C}_{2} \\
&\geq &C\left[ \left( d+K\right) p\right] ^{3} \\
&&\times \left\{ \frac{1}{1-\phi _{\max }^{6}}+20\left( \frac{1}{1-\phi
_{\max }^{3}}\right) ^{2}+90\left( \frac{1}{1-\phi _{\max }^{2}}\right)
^{3}+15\left( \frac{1}{1-\phi _{\max }^{4}}\right) \left( \frac{1}{1-\phi
_{\max }^{2}}\right) \right\} \text{.}
\end{eqnarray*}

Putting everything together, we see that 
\begin{eqnarray*}
E\left\Vert \underline{W}_{t}\right\Vert _{2}^{6} &\leq &32\left\Vert \left(
I_{\left( d{\LARGE +}K\right) p}-A\right) ^{-1}J_{d{\LARGE +}K}^{\prime }\mu
\right\Vert _{2}^{6}+32E\left\Vert \dsum\limits_{j=0}^{\infty }A^{j}J_{d%
{\LARGE +}K}^{\prime }\varepsilon _{t-j}\right\Vert _{2}^{6} \\
&\leq &32\left( \overline{C}_{1}+\overline{C}_{2}\right) \\
&\leq &\overline{C}<\infty
\end{eqnarray*}%
for a constant $\overline{C}$ such that $0<32\left( \overline{C}_{1}+%
\overline{C}_{2}\right) \leq \overline{C}<\infty $.

In addition, define$\mathcal{P}_{\left( d{\LARGE +}K\right) p}$ to be the $%
\left( d{\LARGE +}K\right) p\times \left( d{\LARGE +}K\right) p$ permutation
matrix such that%
\begin{equation}
\mathcal{P}_{\left( d{\LARGE +}K\right) p}\underline{W}_{t}=\left( 
\begin{array}{c}
\underset{dp\times 1}{\underline{Y}_{t}} \\ 
\underset{Kp\times 1}{\underline{F}_{t}}%
\end{array}%
\right) \text{;}  \label{definition permutation matrix}
\end{equation}%
and let $S_{d}^{\prime }=\left( 
\begin{array}{cc}
I_{dp} & \underset{dp\times Kp}{0}%
\end{array}%
\right) $ and $S_{K}^{\prime }=\left( 
\begin{array}{cc}
\underset{Kp\times dp}{0} & I_{Kp}%
\end{array}%
\right) $. Note that%
\begin{eqnarray*}
S_{d}^{\prime }\mathcal{P}_{\left( d{\LARGE +}K\right) p}\underline{W}_{t}
&=&\left( 
\begin{array}{cc}
I_{dp} & \underset{dp\times Kp}{0}%
\end{array}%
\right) \left( 
\begin{array}{c}
\underset{dp\times 1}{\underline{Y}_{t}} \\ 
\underset{Kp\times 1}{\underline{F}_{t}}%
\end{array}%
\right) =\underline{Y}_{t}, \\
S_{K}^{\prime }\mathcal{P}_{\left( d{\LARGE +}K\right) p}\underline{W}_{t}
&=&\left( 
\begin{array}{cc}
\underset{Kp\times dp}{0} & I_{Kp}%
\end{array}%
\right) \left( 
\begin{array}{c}
\underset{dp\times 1}{\underline{Y}_{t}} \\ 
\underset{Kp\times 1}{\underline{F}_{t}}%
\end{array}%
\right) =\underline{F}_{t}\text{.}
\end{eqnarray*}%
so that%
\begin{eqnarray*}
\left\Vert \underline{Y}_{t}\right\Vert _{2} &\leq &\left\Vert S_{d}^{\prime
}\right\Vert _{2}\left\Vert \mathcal{P}_{\left( d{\LARGE +}K\right)
p}\right\Vert _{2}\left\Vert \underline{W}_{t}\right\Vert _{2} \\
&=&\sqrt{\lambda _{\max }\left( S_{d}S_{d}^{\prime }\right) }\sqrt{\lambda
_{\max }\left( \mathcal{P}_{\left( d{\LARGE +}K\right) p}^{\prime }\mathcal{P%
}_{\left( d{\LARGE +}K\right) p}\right) }\left\Vert \underline{W}%
_{t}\right\Vert _{2} \\
&=&\sqrt{\lambda _{\max }\left( S_{d}^{\prime }S_{d}\right) }\sqrt{\lambda
_{\max }\left( I_{\left( d{\LARGE +}K\right) p}\right) }\left\Vert 
\underline{W}_{t}\right\Vert _{2} \\
&=&\sqrt{\lambda _{\max }\left( I_{dp}\right) }\sqrt{\lambda _{\max }\left(
I_{\left( d{\LARGE +}K\right) p}\right) }\left\Vert \underline{W}%
_{t}\right\Vert _{2} \\
&=&\left\Vert \underline{W}_{t}\right\Vert _{2}
\end{eqnarray*}%
and%
\begin{eqnarray*}
\left\Vert \underline{F}_{t}\right\Vert _{2} &\leq &\left\Vert S_{K}^{\prime
}\right\Vert _{2}\left\Vert \mathcal{P}_{\left( d{\LARGE +}K\right)
p}\right\Vert _{2}\left\Vert \underline{W}_{t}\right\Vert _{2} \\
&=&\sqrt{\lambda _{\max }\left( S_{K}S_{K}^{\prime }\right) }\sqrt{\lambda
_{\max }\left( \mathcal{P}_{\left( d{\LARGE +}K\right) p}^{\prime }\mathcal{P%
}_{\left( d{\LARGE +}K\right) p}\right) }\left\Vert \underline{W}%
_{t}\right\Vert _{2} \\
&=&\sqrt{\lambda _{\max }\left( S_{K}^{\prime }S_{K}\right) }\sqrt{\lambda
_{\max }\left( I_{\left( d{\LARGE +}K\right) p}\right) }\left\Vert 
\underline{W}_{t}\right\Vert _{2} \\
&=&\sqrt{\lambda _{\max }\left( I_{Kp}\right) }\sqrt{\lambda _{\max }\left(
I_{\left( d{\LARGE +}K\right) p}\right) }\left\Vert \underline{W}%
_{t}\right\Vert _{2} \\
&=&\left\Vert \underline{W}_{t}\right\Vert _{2}
\end{eqnarray*}%
It further follows that%
\begin{equation*}
E\left\Vert \underline{Y}_{t}\right\Vert _{2}^{6}\leq E\left\Vert \underline{%
W}_{t}\right\Vert _{2}^{6}\leq \overline{C}<\infty \text{ and }E\left\Vert 
\underline{F}_{t}\right\Vert _{2}^{6}\leq E\left\Vert \underline{W}%
_{t}\right\Vert _{2}^{6}\leq \overline{C}<\infty \text{. }\square
\end{equation*}

\bigskip

\noindent \textbf{Lemma OA-6: }Suppose that Assumptions 2-1, 2-2(a)-(b),
2-3, 2-5, 2-6, and 2-9(b) hold. Then, the following statements are true as $%
N_{1},T\rightarrow \infty $

\begin{enumerate}
\item[(a)] 
\begin{equation*}
\max_{1\leq \ell \leq d}\max_{i\in H^{{\large c}}}\left\vert \frac{1}{q}%
\dsum\limits_{r=1}^{q}\frac{1}{\tau _{1}}\dsum\limits_{t=\left( r-1\right)
\tau {\LARGE +}p}^{\left( r-1\right) \tau +\tau _{1}{\LARGE +}p-1}\gamma
_{i}^{\prime }\underline{F}_{t}\varepsilon _{\ell ,t{\LARGE +}1}\right\vert 
\overset{p}{\rightarrow }0\text{. }
\end{equation*}

\item[(b)] 
\begin{equation*}
\max_{1\leq \ell \leq d}\max_{i\in H^{{\large c}}}\frac{1}{q}%
\dsum\limits_{r=1}^{q}\left( \frac{1}{\tau _{1}}\dsum\limits_{t=\left(
r-1\right) \tau +p}^{\left( r-1\right) \tau +\tau _{1}+p-1}\gamma
_{i}^{\prime }\underline{F}_{t}\varepsilon _{\ell ,t{\LARGE +}1}\right) ^{2}%
\overset{p}{\rightarrow }0
\end{equation*}

\item[(c)] 
\begin{equation*}
\max_{1\leq \ell \leq d}\max_{i\in H^{{\large c}}}\left\vert \frac{1}{q}%
\dsum\limits_{r=1}^{q}\frac{1}{\tau _{1}}\dsum\limits_{t=\left( r-1\right)
\tau {\LARGE +}p}^{\left( r-1\right) \tau +\tau _{1}{\LARGE +}p-1}y_{\ell ,t%
{\LARGE +}1}u_{it}\right\vert \overset{p}{\rightarrow }0\text{.}
\end{equation*}

\item[(d)] 
\begin{equation*}
\max_{1\leq \ell \leq d}\max_{i\in H^{{\large c}}}\frac{1}{q}%
\dsum\limits_{r=1}^{q}\left( \frac{1}{\tau _{1}}\dsum\limits_{t=\left(
r-1\right) \tau +p}^{\left( r-1\right) \tau +\tau _{1}+p-1}y_{\ell ,t{\LARGE %
+}1}u_{it}\right) ^{2}\overset{p}{\rightarrow }0
\end{equation*}

\item[(e)] 
\begin{equation*}
\max_{1\leq \ell \leq d}\max_{i\in H^{{\large c}}}\left\vert \frac{1}{q}%
\dsum\limits_{r=1}^{q}\left( \frac{1}{\tau _{1}}\dsum\limits_{t=\left(
r-1\right) \tau +p}^{\left( r-1\right) \tau +\tau _{1}+p-1}\gamma
_{i}^{\prime }\underline{F}_{t}\varepsilon _{\ell ,t{\LARGE +}1}\right)
\left( \frac{1}{\tau _{1}}\dsum\limits_{t=\left( r-1\right) \tau +p}^{\left(
r-1\right) \tau +\tau _{1}+p-1}y_{\ell ,t{\LARGE +}1}u_{it}\right)
\right\vert \overset{p}{\rightarrow }0
\end{equation*}
\end{enumerate}

\medskip

\noindent \textbf{Proof of Lemma OA-6. }

To show part (a), first write%
\begin{eqnarray*}
&&P\left\{ \max_{1\leq \ell \leq d}\max_{i\in H^{{\large c}}}\left\vert 
\frac{1}{q}\dsum\limits_{r=1}^{q}\frac{1}{\tau _{1}}\dsum\limits_{t=\left(
r-1\right) \tau {\LARGE +}p}^{\left( r-1\right) \tau +\tau _{1}{\LARGE +}%
p-1}\gamma _{i}^{\prime }\underline{F}_{t}\varepsilon _{\ell ,t{\LARGE +}%
1}\right\vert \geq \epsilon \right\} \\
&=&P\left\{ \max_{1\leq \ell \leq d}\max_{i\in H^{{\large c}}}\left( \frac{1%
}{q}\dsum\limits_{r=1}^{q}\frac{1}{\tau _{1}}\dsum\limits_{t=\left(
r-1\right) \tau {\LARGE +}p}^{\left( r-1\right) \tau +\tau _{1}{\LARGE +}%
p-1}\gamma _{i}^{\prime }\underline{F}_{t}\varepsilon _{\ell ,t{\LARGE +}%
1}\right) ^{6}\geq \epsilon ^{6}\right\} \\
&\leq &P\left\{ \max_{1\leq \ell \leq d}\max_{i\in H^{{\large c}}}\frac{1}{q}%
\dsum\limits_{r=1}^{q}\left( \frac{1}{\tau _{1}}\dsum\limits_{t=\left(
r-1\right) \tau {\LARGE +}p}^{\left( r-1\right) \tau +\tau _{1}{\LARGE +}%
p-1}\gamma _{i}^{\prime }\underline{F}_{t}\varepsilon _{\ell ,t{\LARGE +}%
1}\right) ^{6}\geq \epsilon ^{6}\right\} \\
&&\left( \text{by Jensen's inequality}\right) \\
&\leq &P\left\{ \dsum\limits_{\ell =1}^{d}\dsum\limits_{i\in H^{{\large c}}}%
\frac{1}{q}\dsum\limits_{r=1}^{q}\left( \frac{1}{\tau _{1}}%
\dsum\limits_{t=\left( r-1\right) \tau {\LARGE +}p}^{\left( r-1\right) \tau
+\tau _{1}{\LARGE +}p-1}\gamma _{i}^{\prime }\underline{F}_{t}\varepsilon
_{\ell ,t{\LARGE +}1}\right) ^{6}\geq \epsilon ^{6}\right\} \\
&\leq &\frac{1}{\epsilon ^{6}}\frac{1}{q}\dsum\limits_{r=1}^{q}\dsum%
\limits_{\ell =1}^{d}\dsum\limits_{i\in H^{{\large c}}}E\left( \frac{1}{\tau
_{1}}\dsum\limits_{t=\left( r-1\right) \tau {\LARGE +}p}^{\left( r-1\right)
\tau +\tau _{1}{\LARGE +}p-1}\gamma _{i}^{\prime }\underline{F}%
_{t}\varepsilon _{\ell ,t{\LARGE +}1}\right) ^{6}
\end{eqnarray*}%
Next, note that%
\begin{eqnarray*}
&&\frac{1}{q}\dsum\limits_{r=1}^{q}\dsum\limits_{\ell
=1}^{d}\dsum\limits_{i\in H^{{\large c}}}E\left( \frac{1}{\tau _{1}}%
\dsum\limits_{t=\left( r-1\right) \tau {\LARGE +}p}^{\left( r-1\right) \tau
+\tau _{1}{\LARGE +}p-1}\gamma _{i}^{\prime }\underline{F}_{t}\varepsilon
_{\ell ,t{\LARGE +}1}\right) ^{6} \\
&\leq &\frac{1}{q\tau _{1}^{6}}\dsum\limits_{r=1}^{q}\dsum\limits_{\ell
=1}^{d}\dsum\limits_{i\in H^{{\large c}}}\dsum\limits_{t=\left( r-1\right)
\tau {\LARGE +}p}^{\left( r-1\right) \tau +\tau _{1}{\LARGE +}p-1}E\left[
\gamma _{i}^{\prime }\underline{F}_{t}\varepsilon _{\ell ,t{\LARGE +}1}%
\right] ^{6} \\
&&+\frac{20}{q\tau _{1}^{6}}\dsum\limits_{r=1}^{q}\dsum\limits_{\ell
=1}^{d}\dsum\limits_{i\in H^{{\large c}}}\dsum\limits_{t=\left( r-1\right)
\tau {\LARGE +}p}^{\left( r-1\right) \tau +\tau _{1}{\LARGE +}%
p-1}\dsum\limits_{\substack{ s=\left( r-1\right) \tau {\LARGE +}p  \\ s\neq
t }}^{\left( r-1\right) \tau +\tau _{1}{\LARGE +}p-1}E\left[ \left\vert
\gamma _{i}^{\prime }\underline{F}_{t}\varepsilon _{\ell ,t{\LARGE +}%
1}\right\vert \right] ^{3}E\left[ \left\vert \gamma _{i}^{\prime }\underline{%
F}_{s}\varepsilon _{\ell ,s{\LARGE +}1}\right\vert \right] ^{3} \\
&&+\frac{15}{q\tau _{1}^{6}}\dsum\limits_{r=1}^{q}\dsum\limits_{\ell
=1}^{d}\dsum\limits_{i\in H^{{\large c}}}\dsum\limits_{t=\left( r-1\right)
\tau {\LARGE +}p}^{\left( r-1\right) \tau +\tau _{1}{\LARGE +}%
p-1}\dsum\limits_{\substack{ s=\left( r-1\right) \tau {\LARGE +}p  \\ s\neq
t }}^{\left( r-1\right) \tau +\tau _{1}{\LARGE +}p-1}E\left[ \gamma
_{i}^{\prime }\underline{F}_{t}\varepsilon _{\ell ,t{\LARGE +}1}\right] ^{4}E%
\left[ \gamma _{i}^{\prime }\underline{F}_{s}\varepsilon _{\ell ,s{\LARGE +}%
1}\right] ^{2} \\
&&+\frac{90}{q\tau _{1}^{6}}\dsum\limits_{r=1}^{q}\dsum\limits_{\ell
=1}^{d}\dsum\limits_{i\in H^{{\large c}}}\dsum\limits_{t=\left( r-1\right)
\tau {\LARGE +}p}^{\left( r-1\right) \tau +\tau _{1}{\LARGE +}%
p-1}\dsum\limits_{\substack{ s=\left( r-1\right) \tau {\LARGE +}p  \\ s\neq
t }}^{\left( r-1\right) \tau +\tau _{1}{\LARGE +}p-1}\dsum\limits_{\substack{
r=\left( r-1\right) \tau {\LARGE +}p  \\ r\neq t,\text{ }r\neq s}}^{\left(
r-1\right) \tau +\tau _{1}{\LARGE +}p-1}\left\{ E\left[ \gamma _{i}^{\prime }%
\underline{F}_{t}\varepsilon _{\ell ,t{\LARGE +}1}\right] ^{2}E\left[ \gamma
_{i}^{\prime }\underline{F}_{s}\varepsilon _{\ell ,s{\LARGE +}1}\right]
^{2}\right. \\
&&\text{ \ \ \ \ \ \ \ \ \ \ \ \ \ \ \ \ \ \ \ \ \ \ \ \ \ \ \ \ \ \ \ \ \ \
\ \ \ \ \ \ \ \ \ \ \ \ \ \ \ \ \ \ \ \ \ \ \ \ \ \ \ \ \ \ \ \ \ \ \ \ \ \
\ \ \ \ \ \ \ \ \ \ \ \ \ }\left. \times E\left[ \gamma _{i}^{\prime }%
\underline{F}_{s}\varepsilon _{\ell ,r{\LARGE +}1}\right] ^{2}\right\} \\
&\leq &\frac{1}{q\tau _{1}^{6}}\dsum\limits_{r=1}^{q}\dsum\limits_{\ell
=1}^{d}\dsum\limits_{i\in H^{{\large c}}}\dsum\limits_{t=\left( r-1\right)
\tau {\LARGE +}p}^{\left( r-1\right) \tau +\tau _{1}{\LARGE +}p-1}E\left[
\left( \gamma _{i}^{\prime }\underline{F}_{t}\right) ^{6}\right] E\left[
\varepsilon _{\ell ,t{\LARGE +}1}^{6}\right] \\
&&+\frac{20}{q\tau _{1}^{6}}\dsum\limits_{r=1}^{q}\dsum\limits_{\ell
=1}^{d}\dsum\limits_{i\in H^{{\large c}}}\dsum\limits_{t=\left( r-1\right)
\tau {\LARGE +}p}^{\left( r-1\right) \tau +\tau _{1}{\LARGE +}%
p-1}\dsum\limits_{\substack{ s=\left( r-1\right) \tau {\LARGE +}p  \\ s\neq
t }}^{\left( r-1\right) \tau +\tau _{1}{\LARGE +}p-1}\frac{1}{64}E\left[
\gamma _{i}^{\prime }\underline{F}_{t}\underline{F}_{t}^{\prime }\gamma
_{i}+\varepsilon _{\ell ,t{\LARGE +}1}^{2}\right] ^{3}E\left[ \gamma
_{i}^{\prime }\underline{F}_{s}\underline{F}_{s}^{\prime }\gamma
_{i}+\varepsilon _{\ell ,s{\LARGE +}1}^{2}\right] ^{3} \\
&&+\frac{15}{q\tau _{1}^{6}}\dsum\limits_{r=1}^{q}\dsum\limits_{\ell
=1}^{d}\dsum\limits_{i\in H^{{\large c}}}\dsum\limits_{t=\left( r-1\right)
\tau {\LARGE +}p}^{\left( r-1\right) \tau +\tau _{1}{\LARGE +}%
p-1}\dsum\limits_{\substack{ s=\left( r-1\right) \tau {\LARGE +}p  \\ s\neq
t }}^{\left( r-1\right) \tau +\tau _{1}{\LARGE +}p-1}E\left[ \gamma
_{i}^{\prime }\underline{F}_{t}\underline{F}_{t}^{\prime }\gamma _{i}\right]
^{2}E\left[ \varepsilon _{\ell ,t{\LARGE +}1}^{4}\right] E\left[ \gamma
_{i}^{\prime }\underline{F}_{s}\underline{F}_{s}^{\prime }\gamma _{i}\right]
E\left[ \varepsilon _{\ell ,s{\LARGE +}1}^{2}\right] \\
&&+\frac{90}{q\tau _{1}^{6}}\dsum\limits_{r=1}^{q}\dsum\limits_{\ell
=1}^{d}\dsum\limits_{i\in H^{{\large c}}}\left\{ \dsum\limits_{t=\left(
r-1\right) \tau {\LARGE +}p}^{\left( r-1\right) \tau +\tau _{1}{\LARGE +}%
p-1}\dsum\limits_{\substack{ s=\left( r-1\right) \tau {\LARGE +}p  \\ s\neq
t }}^{\left( r-1\right) \tau +\tau _{1}{\LARGE +}p-1}E\left[ \gamma
_{i}^{\prime }\underline{F}_{t}\underline{F}_{t}^{\prime }\gamma _{i}\right]
E\left[ \varepsilon _{\ell ,t{\LARGE +}1}^{2}\right] E\left[ \gamma
_{i}^{\prime }\underline{F}_{s}\underline{F}_{s}^{\prime }\gamma _{i}\right]
E\left[ \varepsilon _{\ell ,s{\LARGE +}1}^{2}\right] \right. \\
&&\text{ \ \ \ \ \ \ \ \ \ \ \ \ \ \ \ \ \ \ \ \ \ \ }\left. \times
\dsum\limits_{\substack{ r=\left( r-1\right) \tau {\LARGE +}p  \\ r\neq t,%
\text{ }r\neq s}}^{\left( r-1\right) \tau +\tau _{1}{\LARGE +}p-1}E\left[
\gamma _{i}^{\prime }\underline{F}_{r}\underline{F}_{r}^{\prime }\gamma _{i}%
\right] E\left[ \varepsilon _{\ell ,r{\LARGE +}1}^{2}\right] \right\}
\end{eqnarray*}%
\begin{eqnarray*}
&\leq &\frac{1}{q\tau _{1}^{6}}\dsum\limits_{r=1}^{q}\dsum\limits_{\ell
=1}^{d}\dsum\limits_{i\in H^{{\large c}}}\dsum\limits_{t=\left( r-1\right)
\tau {\LARGE +}p}^{\left( r-1\right) \tau +\tau _{1}{\LARGE +}p-1}\left\Vert
\gamma _{i}\right\Vert _{2}^{6}E\left[ \left\Vert \underline{F}%
_{t}\right\Vert _{2}^{6}\right] E\left[ \varepsilon _{\ell ,t{\LARGE +}1}^{6}%
\right] \\
&&+\frac{\left( 20\cdot 16\right) }{64q\tau _{1}^{6}}\dsum\limits_{r=1}^{q}%
\dsum\limits_{\ell =1}^{d}\dsum\limits_{i\in H^{{\large c}%
}}\dsum\limits_{t=\left( r-1\right) \tau {\LARGE +}p}^{\left( r-1\right)
\tau +\tau _{1}{\LARGE +}p-1}\dsum\limits_{\substack{ s=\left( r-1\right)
\tau {\LARGE +}p  \\ s\neq t}}^{\left( r-1\right) \tau +\tau _{1}{\LARGE +}%
p-1}\left\{ \left( E\left[ \left( \gamma _{i}^{\prime }\underline{F}%
_{t}\right) ^{6}\right] +E\left[ \varepsilon _{\ell ,t{\LARGE +}1}^{6}\right]
\right) \right. \\
&&\text{ \ \ \ \ \ \ \ \ \ \ \ \ \ \ \ \ \ \ \ \ \ \ \ \ \ \ \ \ \ \ \ \ \ \
\ \ \ \ \ \ \ \ \ \ \ \ \ \ \ \ \ \ \ \ \ \ \ \ \ \ \ \ \ \ \ \ \ \ \ \ \ }%
\left. \times \left( E\left[ \left( \gamma _{i}^{\prime }\underline{F}%
_{s}\right) ^{6}\right] +E\left[ \varepsilon _{\ell ,s{\LARGE +}1}^{6}\right]
\right) \right\} \\
&&+\frac{15}{q\tau _{1}^{6}}\dsum\limits_{r=1}^{q}\dsum\limits_{\ell
=1}^{d}\dsum\limits_{i\in H^{{\large c}}}\dsum\limits_{t=\left( r-1\right)
\tau {\LARGE +}p}^{\left( r-1\right) \tau +\tau _{1}{\LARGE +}%
p-1}\dsum\limits_{\substack{ s=\left( r-1\right) \tau {\LARGE +}p  \\ s\neq
t }}^{\left( r-1\right) \tau +\tau _{1}{\LARGE +}p-1}\left\{ \left\Vert
\gamma _{i}\right\Vert _{2}^{4}E\left[ \left\Vert \underline{F}%
_{t}\right\Vert _{2}^{4}\right] E\left[ \varepsilon _{\ell ,t{\LARGE +}1}^{4}%
\right] \right. \\
&&\text{ \ \ \ \ \ \ \ \ \ \ \ \ \ \ \ \ \ \ \ \ \ \ \ \ \ \ \ \ \ \ \ \ \ \
\ \ \ \ \ \ \ \ \ \ \ \ \ \ \ \ \ \ \ \ \ \ \ \ \ \ \ \ \ \ \ }\left. \times
\left\Vert \gamma _{i}\right\Vert _{2}^{2}E\left[ \left\Vert \underline{F}%
_{s}\right\Vert _{2}^{2}\right] E\left[ \varepsilon _{\ell ,s{\LARGE +}1}^{2}%
\right] \right\} \\
&&+\frac{90}{q\tau _{1}^{6}}\dsum\limits_{r=1}^{q}\dsum\limits_{\ell
=1}^{d}\dsum\limits_{i\in H^{{\large c}}}\left\{ \dsum\limits_{t=\left(
r-1\right) \tau {\LARGE +}p}^{\left( r-1\right) \tau +\tau _{1}{\LARGE +}%
p-1}\left\Vert \gamma _{i}\right\Vert _{2}^{2}E\left[ \left\Vert \underline{F%
}_{t}\right\Vert _{2}^{2}\right] E\left[ \varepsilon _{\ell ,t{\LARGE +}%
1}^{2}\right] \right. \\
&&\text{ \ \ }\left. \times \dsum\limits_{\substack{ s=\left( r-1\right)
\tau {\LARGE +}p  \\ s\neq t}}^{\left( r-1\right) \tau +\tau _{1}{\LARGE +}%
p-1}\left\Vert \gamma _{i}\right\Vert _{2}^{2}E\left[ \left\Vert \underline{F%
}_{s}\right\Vert _{2}^{2}\right] E\left[ \varepsilon _{\ell ,s{\LARGE +}%
1}^{2}\right] \dsum\limits_{\substack{ r=\left( r-1\right) \tau {\LARGE +}p 
\\ r\neq t,\text{ }r\neq s}}^{\left( r-1\right) \tau +\tau _{1}{\LARGE +}%
p-1}\left\Vert \gamma _{i}\right\Vert _{2}^{2}E\left[ \left\Vert \underline{F%
}_{r}\right\Vert _{2}^{2}\right] E\left[ \varepsilon _{\ell ,r{\LARGE +}%
1}^{2}\right] \right\} \\
&\leq &C\left( \frac{N_{1}}{\tau _{1}^{5}}+5\frac{N_{1}}{\tau _{1}^{4}}+15%
\frac{N_{1}}{\tau _{1}^{4}}+90\frac{N_{1}}{\tau _{1}^{3}}\right) \text{ } \\
&&\left( \text{applying Assumptions 2-2(b), Assumption 2-5, and Lemma OA-5}%
\right) \\
&=&O\left( \frac{N_{1}}{\tau _{1}^{3}}\right) \text{.}
\end{eqnarray*}%
It follows that%
\begin{equation*}
P\left\{ \max_{1\leq \ell \leq d}\max_{i\in H^{{\large c}}}\left\vert \frac{1%
}{q}\dsum\limits_{r=1}^{q}\frac{1}{\tau _{1}}\dsum\limits_{t=\left(
r-1\right) \tau {\LARGE +}p}^{\left( r-1\right) \tau +\tau _{1}{\LARGE +}%
p-1}\gamma _{i}^{\prime }\underline{F}_{t}\varepsilon _{\ell ,t{\LARGE +}%
1}\right\vert \geq \epsilon \right\} =O\left( \frac{N_{1}}{\tau _{1}^{3}}%
\right) =o\left( 1\right) \text{.}
\end{equation*}

To show part (b), note that, for any $\epsilon >0$%
\begin{eqnarray*}
&&P\left\{ \max_{1\leq \ell \leq d}\max_{i\in H^{{\large c}}}\frac{1}{q}%
\dsum\limits_{r=1}^{q}\left( \frac{1}{\tau _{1}}\dsum\limits_{t=\left(
r-1\right) \tau +p}^{\left( r-1\right) \tau +\tau _{1}+p-1}\gamma
_{i}^{\prime }\underline{F}_{t}\varepsilon _{\ell ,t{\LARGE +}1}\right)
^{2}\geq \epsilon \right\} \\
&=&P\left\{ \max_{1\leq \ell \leq d}\max_{i\in H^{{\large c}}}\left\vert 
\frac{1}{q}\dsum\limits_{r=1}^{q}\left( \frac{1}{\tau _{1}}%
\dsum\limits_{t=\left( r-1\right) \tau +p}^{\left( r-1\right) \tau +\tau
_{1}+p-1}\gamma _{i}^{\prime }\underline{F}_{t}\varepsilon _{\ell ,t{\LARGE +%
}1}\right) ^{2}\right\vert ^{3}\geq \epsilon ^{3}\right\} \\
&\leq &P\left\{ \max_{1\leq \ell \leq d}\max_{i\in H^{{\large c}}}\frac{1}{q}%
\dsum\limits_{r=1}^{q}\left( \frac{1}{\tau _{1}}\dsum\limits_{t=\left(
r-1\right) \tau +p}^{\left( r-1\right) \tau +\tau _{1}+p-1}\gamma
_{i}^{\prime }\underline{F}_{t}\varepsilon _{\ell ,t{\LARGE +}1}\right)
^{6}\geq \epsilon ^{3}\right\} \\
&&\left( \text{by Jensen't inequality}\right) \\
&\leq &P\left\{ \dsum\limits_{\ell =1}^{d}\dsum\limits_{i\in H^{{\large c}}}%
\frac{1}{q}\dsum\limits_{r=1}^{q}\left( \frac{1}{\tau _{1}}%
\dsum\limits_{t=\left( r-1\right) \tau {\LARGE +}p}^{\left( r-1\right) \tau
+\tau _{1}{\LARGE +}p-1}\gamma _{i}^{\prime }\underline{F}_{t}\varepsilon
_{\ell ,t{\LARGE +}1}\right) ^{6}\geq \epsilon ^{3}\right\} \\
&\leq &\frac{1}{\epsilon ^{3}}\frac{1}{q}\dsum\limits_{r=1}^{q}\dsum%
\limits_{\ell =1}^{d}\dsum\limits_{i\in H^{{\large c}}}E\left( \frac{1}{\tau
_{1}}\dsum\limits_{t=\left( r-1\right) \tau {\LARGE +}p}^{\left( r-1\right)
\tau +\tau _{1}{\LARGE +}p-1}\gamma _{i}^{\prime }\underline{F}%
_{t}\varepsilon _{\ell ,t{\LARGE +}1}\right) ^{6}
\end{eqnarray*}

\noindent The rest of the proof for part (b) then follows in a manner
similar to the argument given for part (a) above.

To show part (c), first note that, for any $\epsilon >0$,%
\begin{eqnarray}
&&P\left\{ \max_{1\leq \ell \leq d}\max_{i\in H^{{\large c}}}\left\vert 
\frac{1}{q}\dsum\limits_{r=1}^{q}\frac{1}{\tau _{1}}\dsum\limits_{t=\left(
r-1\right) \tau {\LARGE +}p}^{\left( r-1\right) \tau +\tau _{1}{\LARGE +}%
p-1}y_{\ell ,t{\LARGE +}1}u_{it}\right\vert \geq \epsilon \right\}  \notag \\
&=&P\left\{ \max_{1\leq \ell \leq d}\max_{i\in H^{{\large c}}}\left( \frac{1%
}{q}\dsum\limits_{r=1}^{q}\frac{1}{\tau _{1}}\dsum\limits_{t=\left(
r-1\right) \tau {\LARGE +}p}^{\left( r-1\right) \tau +\tau _{1}{\LARGE +}%
p-1}y_{\ell ,t{\LARGE +}1}u_{it}\right) ^{6}\geq \epsilon ^{6}\right\} 
\notag \\
&\leq &P\left\{ \max_{1\leq \ell \leq d}\max_{i\in H^{{\large c}}}\frac{1}{q}%
\dsum\limits_{r=1}^{q}\left( \frac{1}{\tau _{1}}\dsum\limits_{t=\left(
r-1\right) \tau {\LARGE +}p}^{\left( r-1\right) \tau +\tau _{1}{\LARGE +}%
p-1}y_{\ell ,t{\LARGE +}1}u_{it}\right) ^{6}\geq \epsilon ^{6}\right\} 
\notag \\
&&\left( \text{by convexity or Jensen's inequality}\right)  \notag \\
&\leq &P\left\{ \dsum\limits_{\ell =1}^{d}\dsum\limits_{i\in H^{{\large c}}}%
\frac{1}{q}\dsum\limits_{r=1}^{q}\left( \frac{1}{\tau _{1}}%
\dsum\limits_{t=\left( r-1\right) \tau {\LARGE +}p}^{\left( r-1\right) \tau
+\tau _{1}{\LARGE +}p-1}y_{\ell ,t{\LARGE +}1}u_{it}\right) ^{6}\geq
\epsilon ^{6}\right\}  \notag \\
&\leq &\frac{1}{\epsilon ^{6}}\frac{1}{q}\dsum\limits_{r=1}^{q}\dsum%
\limits_{\ell =1}^{d}\dsum\limits_{i\in H^{{\large c}}}E\left( \frac{1}{\tau
_{1}}\dsum\limits_{t=\left( r-1\right) \tau {\LARGE +}p}^{\left( r-1\right)
\tau +\tau _{1}{\LARGE +}p-1}y_{\ell ,t{\LARGE +}1}u_{it}\right) ^{6}
\label{Pmaxyu}
\end{eqnarray}%
Now, there exists a constant $C_{1}>1$ such that 
\begin{eqnarray*}
&&\frac{1}{q}\dsum\limits_{r=1}^{q}\dsum\limits_{\ell
=1}^{d}\dsum\limits_{i\in H^{{\large c}}}E\left( \frac{1}{\tau _{1}}%
\dsum\limits_{t=\left( r-1\right) \tau {\LARGE +}p}^{\left( r-1\right) \tau
+\tau _{1}{\LARGE +}p-1}y_{\ell ,t{\LARGE +}1}u_{it}\right) ^{6} \\
&\leq &\frac{C_{1}}{q\tau _{1}^{6}}\dsum\limits_{r=1}^{q}\dsum\limits_{i\in
H^{{\large c}}}\left\{ \dsum\limits_{\substack{ t,s,g,h,v,w=\left(
r-1\right) \tau +p  \\ t\leq s\leq g\leq h\leq v\leq w}}^{\left( r-1\right)
\tau +\tau _{1}+p-1}\left\vert E\left[ u_{it}u_{is}u_{ig}u_{ih}u_{iv}u_{iw}%
\right] \right\vert \right. \\
&&\text{ \ \ \ \ \ \ \ \ \ \ \ \ \ \ \ \ \ \ \ \ \ \ \ \ }\left. \times
\dsum\limits_{\ell =1}^{d}\left\vert E\left[ y_{\ell ,t{\LARGE +}1}y_{\ell ,s%
{\LARGE +}1}y_{\ell ,g{\LARGE +}1}y_{\ell ,h{\LARGE +}1}y_{\ell ,v{\LARGE +}%
1}y_{\ell ,w{\LARGE +}1}\right] \right\vert \right\}
\end{eqnarray*}%
Next, note that, by repeated application of H\"{o}lder's inequality, we have
by Lemma OA-5 that there exists a positive constant $\overline{C}$ such that%
\begin{eqnarray*}
&&\dsum\limits_{\ell =1}^{d}\left\vert E\left[ y_{\ell ,t{\LARGE +}1}y_{\ell
,s{\LARGE +}1}y_{\ell ,g{\LARGE +}1}y_{\ell ,h{\LARGE +}1}y_{\ell ,v{\LARGE +%
}1}y_{\ell ,w{\LARGE +}1}\right] \right\vert \\
&\leq &\dsum\limits_{\ell =1}^{d}\left( E\left[ y_{\ell ,t{\LARGE +}%
1}^{2}y_{\ell ,s{\LARGE +}1}^{2}y_{\ell ,g{\LARGE +}1}^{2}\right] \right) ^{%
\frac{{\large 1}}{{\large 2}}}\left( E\left[ y_{\ell ,h{\LARGE +}%
1}^{2}y_{\ell ,v{\LARGE +}1}^{2}y_{\ell ,w{\LARGE +}1}^{2}\right] \right) ^{%
\frac{{\large 1}}{{\large 2}}} \\
&\leq &\dsum\limits_{\ell =1}^{d}\left( \left\{ E\left[ y_{\ell ,t{\LARGE +}%
1}^{6}\right] \right\} ^{\frac{{\large 1}}{{\large 3}}}\left( E\left[
\left\vert y_{\ell ,s{\LARGE +}1}y_{\ell ,g{\LARGE +}1}\right\vert ^{3}%
\right] \right) ^{\frac{{\large 2}}{{\large 3}}}\right) ^{\frac{{\large 1}}{%
{\large 2}}}\left( \left\{ E\left[ y_{\ell ,h{\LARGE +}1}^{6}\right]
\right\} ^{\frac{{\large 1}}{{\large 3}}}\left( E\left[ \left\vert y_{\ell ,v%
{\LARGE +}1}y_{\ell ,w{\LARGE +}1}\right\vert ^{3}\right] \right) ^{\frac{%
{\large 2}}{{\large 3}}}\right) ^{\frac{{\large 1}}{{\large 2}}} \\
&\leq &\dsum\limits_{\ell =1}^{d}\left[ \left( \left\{ E\left[ y_{\ell ,t%
{\LARGE +}1}^{6}\right] \right\} ^{\frac{{\large 1}}{{\large 3}}}\left\{ E%
\left[ y_{\ell ,s{\LARGE +}1}^{6}\right] \right\} ^{\frac{{\large 1}}{%
{\large 3}}}\left\{ E\left[ y_{\ell ,g{\LARGE +}1}^{6}\right] \right\} ^{%
\frac{{\large 1}}{{\large 3}}}\right) ^{\frac{{\large 1}}{{\large 2}}}\right.
\\
&&\text{ \ \ }\left. \times \left( \left\{ E\left[ y_{\ell ,h{\LARGE +}1}^{6}%
\right] \right\} ^{\frac{{\large 1}}{{\large 3}}}\left\{ E\left[ y_{\ell ,v%
{\LARGE +}1}^{6}\right] \right\} ^{\frac{{\large 1}}{{\large 3}}}\left\{ E%
\left[ y_{\ell ,w{\LARGE +}1}^{6}\right] \right\} ^{\frac{{\large 1}}{%
{\large 3}}}\right) ^{\frac{{\large 1}}{{\large 2}}}\right] \\
&\leq &\dsum\limits_{\ell =1}^{d}\left\{ E\left[ y_{\ell ,t{\LARGE +}1}^{6}%
\right] \right\} ^{\frac{{\large 1}}{{\large 6}}}\left\{ E\left[ y_{\ell ,s%
{\LARGE +}1}^{6}\right] \right\} ^{\frac{{\large 1}}{{\large 6}}}\left\{ E%
\left[ y_{\ell ,g{\LARGE +}1}^{6}\right] \right\} ^{\frac{{\large 1}}{%
{\large 6}}}\left\{ E\left[ y_{\ell ,h{\LARGE +}1}^{6}\right] \right\} ^{%
\frac{{\large 1}}{{\large 6}}}\left\{ E\left[ y_{\ell ,v{\LARGE +}1}^{6}%
\right] \right\} ^{\frac{{\large 1}}{{\large 6}}}\left\{ E\left[ y_{\ell ,w%
{\LARGE +}1}^{6}\right] \right\} ^{\frac{{\large 1}}{{\large 6}}} \\
&\leq &d\max_{1\leq \ell \leq d}\sup_{t}E\left[ y_{\ell ,t}^{6}\right] \\
&\leq &\overline{C}<\infty \text{ } \\
&&\left( \text{since, given that }y_{\ell ,t}=\mathbf{e}_{\ell ,dp}^{\prime }%
\underline{Y}_{t}\text{; }E\left[ y_{\ell ,t}^{6}\right] \leq E\left\Vert 
\underline{Y}_{t}\right\Vert _{2}^{6}\leq \overline{C}\text{ by Lemma OA-5}%
\right. \text{ } \\
&&\left. \text{where }\overline{C}\text{ is a constant not depending on }%
\ell \text{ or }t\text{ }\right)
\end{eqnarray*}%
Hence, we can write%
\begin{eqnarray}
&&\frac{1}{q}\dsum\limits_{r=1}^{q}\dsum\limits_{\ell
=1}^{d}\dsum\limits_{i\in H^{{\large c}}}E\left( \frac{1}{\tau _{1}}%
\dsum\limits_{t=\left( r-1\right) \tau {\LARGE +}p}^{\left( r-1\right) \tau
+\tau _{1}{\LARGE +}p-1}y_{\ell ,t{\LARGE +}1}u_{it}\right) ^{6}  \notag \\
&\leq &\frac{C_{1}\overline{C}}{q\tau _{1}^{6}}\dsum\limits_{r=1}^{q}\dsum%
\limits_{i\in H^{{\large c}}}\dsum\limits_{\substack{ t,s,g,h,v,w=\left(
r-1\right) \tau +p  \\ t\leq s\leq g\leq h\leq v\leq w}}^{\left( r-1\right)
\tau +\tau _{1}+p-1}\left\vert E\left[ u_{it}u_{is}u_{ig}u_{ih}u_{iv}u_{iw}%
\right] \right\vert  \notag \\
&\leq &\frac{C_{1}\overline{C}}{q\tau _{1}^{6}}\dsum\limits_{r=1}^{q}\dsum%
\limits_{i\in H^{{\large c}}}\dsum\limits_{\substack{ t,s,g=\left(
r-1\right) \tau +p  \\ t\leq s\leq g}}^{\left( r-1\right) \tau +\tau
_{1}+p-1}\left\vert E\left[ u_{it}u_{is}u_{ig}^{4}\right] \right\vert  \notag
\\
&&+\frac{C_{1}\overline{C}}{q\tau _{1}^{6}}\dsum\limits_{r=1}^{q}\dsum%
\limits_{i\in H}\dsum\limits_{\substack{ t,s,g,h,v,w=\left( r-1\right) \tau
+p  \\ t\leq s\leq g\leq h\leq v\leq w  \\ w-v\geq \max \left\{
v-h,h-g\right\} ,w-v>0}}^{\left( r-1\right) \tau +\tau _{1}+p-1}\left\vert E%
\left[ u_{it}u_{is}u_{ig}u_{ih}u_{iv}u_{iw}\right] \right\vert  \notag \\
&&+\frac{C_{1}\overline{C}}{q\tau _{1}^{6}}\dsum\limits_{r=1}^{q}\dsum%
\limits_{i\in H^{{\large c}}}\dsum\limits_{\substack{ t,s,g,h,v,w=\left(
r-1\right) \tau +p  \\ t\leq s\leq g\leq h\leq v\leq w  \\ v-h\geq \max
\left\{ w-v,h-g\right\} ,v-h>0}}^{\left( r-1\right) \tau +\tau
_{1}+p-1}\left\vert E\left[ u_{it}u_{is}u_{ig}u_{ih}u_{iv}u_{iw}\right]
\right\vert  \notag \\
&&+\frac{C_{1}\overline{C}}{q\tau _{1}^{6}}\dsum\limits_{r=1}^{q}\dsum%
\limits_{i\in H}\dsum\limits_{\substack{ t,s,g,h,v,w=\left( r-1\right) \tau
+p  \\ t\leq s\leq g\leq h\leq v\leq w  \\ h-g\geq \max \left\{
w-v,v-h\right\} ,h-g>0}}^{\left( r-1\right) \tau +\tau _{1}+p-1}\left\vert E%
\left[ u_{it}u_{is}u_{ig}u_{ih}u_{iv}u_{iw}\right] \right\vert  \notag
\end{eqnarray}%
\begin{eqnarray}
&\leq &\frac{C_{1}\overline{C}}{q\tau _{1}^{6}}\dsum\limits_{r=1}^{q}\dsum%
\limits_{i\in H^{{\large c}}}\dsum\limits_{\substack{ t,s,g=\left(
r-1\right) \tau +p  \\ t\leq s\leq g}}^{\left( r-1\right) \tau +\tau
_{1}+p-1}\left\vert E\left[ u_{it}u_{is}u_{ig}^{4}\right] \right\vert  \notag
\\
&&+\frac{C_{1}\overline{C}}{q\tau _{1}^{6}}\dsum\limits_{r=1}^{q}\dsum%
\limits_{i\in H}\dsum\limits_{\substack{ t,s,g,h,v,w=\left( r-1\right) \tau
+p  \\ t\leq s\leq g\leq h\leq v\leq w  \\ w-v\geq \max \left\{
v-h,h-g\right\} ,w-v>0}}^{\left( r-1\right) \tau +\tau _{1}+p-1}\left\vert E%
\left[ u_{it}u_{is}u_{ig}u_{ih}u_{iv}u_{iw}\right] \right\vert  \notag \\
&&+\frac{C_{1}\overline{C}}{q\tau _{1}^{6}}\dsum\limits_{r=1}^{q}\dsum%
\limits_{i\in H^{{\large c}}}\dsum\limits_{\substack{ t,s,g,h,v,w=\left(
r-1\right) \tau +p  \\ t\leq s\leq g\leq h\leq v\leq w  \\ v-h\geq \max
\left\{ w-v,h-g\right\} ,v-h>0}}^{\left( r-1\right) \tau +\tau
_{1}+p-1}\left\vert E\left[ \left\{ u_{it}u_{is}u_{ig}u_{ih}-E\left(
u_{it}u_{is}u_{ig}u_{ih}\right) \right\} u_{iv}u_{iw}\right] \right\vert 
\notag \\
&&+\frac{C_{1}\overline{C}}{q\tau _{1}^{6}}\dsum\limits_{r=1}^{q}\dsum%
\limits_{i\in H^{{\large c}}}\dsum\limits_{\substack{ t,s,g,h,v,w=\left(
r-1\right) \tau +p  \\ t\leq s\leq g\leq h\leq v\leq w  \\ v-h\geq \max
\left\{ w-v,h-g\right\} ,v-h>0}}^{\left( r-1\right) \tau +\tau
_{1}+p-1}\left\vert E\left( u_{it}u_{is}u_{ig}u_{ih}\right) \right\vert
\left\vert E\left( u_{iv}u_{iw}\right) \right\vert  \notag \\
&&+\frac{C_{1}\overline{C}}{q\tau _{1}^{6}}\dsum\limits_{r=1}^{q}\dsum%
\limits_{i\in H^{{\large c}}}\dsum\limits_{\substack{ t,s,g,h,v,w=\left(
r-1\right) \tau +p  \\ t\leq s\leq g\leq h\leq v\leq w  \\ h-g\geq \max
\left\{ w-v,v-h\right\} ,h-g>0}}^{\left( r-1\right) \tau +\tau
_{1}+p-1}\left\vert E\left[ \left\{ u_{it}u_{is}u_{ig}-E\left(
u_{it}u_{is}u_{ig}\right) \right\} u_{ih}u_{iv}u_{iw}\right] \right\vert 
\notag \\
&&+\frac{C_{1}\overline{C}}{q\tau _{1}^{6}}\dsum\limits_{r=1}^{q}\dsum%
\limits_{i\in H^{{\large c}}}\dsum\limits_{\substack{ t,s,g,h,v,w=\left(
r-1\right) \tau +p  \\ t\leq s\leq g\leq h\leq v\leq w  \\ h-g\geq \max
\left\{ w-v,v-h\right\} ,h-g>0}}^{\left( r-1\right) \tau +\tau
_{1}+p-1}\left\vert E\left( u_{it}u_{is}u_{ig}\right) \right\vert \left\vert
E\left( u_{ih}u_{iv}u_{iw}\right) \right\vert  \notag \\
&=&\mathcal{T}_{1}+\mathcal{T}_{2}+\mathcal{T}_{3}+\mathcal{T}_{4}+\mathcal{T%
}_{5}+\mathcal{T}_{6},\text{ }\left( say\right) \text{.}  \label{Eyu^6}
\end{eqnarray}

Consider first $\mathcal{T}_{1}$. Note that%
\begin{eqnarray}
\mathcal{T}_{1} &=&\frac{C_{1}\overline{C}}{q\tau _{1}^{6}}%
\dsum\limits_{r=1}^{q}\dsum\limits_{i\in H^{{\large c}}}\dsum\limits 
_{\substack{ t,s,g=\left( r-1\right) \tau +p  \\ t\leq s\leq g}}^{\left(
r-1\right) \tau +\tau _{1}+p-1}\left\vert E\left[ u_{it}u_{is}u_{ig}^{4}%
\right] \right\vert  \notag \\
&\leq &\frac{C_{1}\overline{C}}{q\tau _{1}^{6}}\dsum\limits_{r=1}^{q}\dsum%
\limits_{i\in H^{{\large c}}}\dsum\limits_{\substack{ t,s,g=\left(
r-1\right) \tau +p  \\ t\leq s\leq g}}^{\left( r-1\right) \tau +\tau
_{1}+p-1}E\left[ \left\vert u_{it}u_{is}u_{ig}^{4}\right\vert \right]  \notag
\\
&\leq &\frac{C_{1}\overline{C}}{q\tau _{1}^{6}}\dsum\limits_{r=1}^{q}\dsum%
\limits_{i\in H^{{\large c}}}\dsum\limits_{\substack{ t,s,g=\left(
r-1\right) \tau +p  \\ t\leq s\leq g}}^{\left( r-1\right) \tau +\tau
_{1}+p-1}\left( E\left[ \left\vert u_{it}u_{is}\right\vert ^{3}\right]
\right) ^{\frac{{\large 1}}{{\large 3}}}\left( E\left[ \left\vert
u_{ig}\right\vert ^{6}\right] \right) ^{\frac{{\large 2}}{{\large 3}}}\text{ 
}\left( \text{by H\"{o}lder's inequality}\right)  \notag \\
&\leq &\frac{C_{1}\overline{C}}{q\tau _{1}^{6}}\dsum\limits_{r=1}^{q}\dsum%
\limits_{i\in H^{{\large c}}}\dsum\limits_{\substack{ t,s,g=\left(
r-1\right) \tau +p  \\ t\leq s\leq g}}^{\left( r-1\right) \tau +\tau
_{1}+p-1}\left( \left[ E\left\{ \left\vert u_{it}\right\vert ^{6}\right\} %
\right] ^{\frac{{\large 1}}{{\large 2}}}\left[ E\left\{ \left\vert
u_{is}\right\vert ^{6}\right\} \right] ^{\frac{{\large 1}}{{\large 2}}%
}\right) ^{\frac{{\large 1}}{{\large 3}}}\left( E\left[ \left\vert
u_{ig}\right\vert ^{6}\right] \right) ^{\frac{{\large 2}}{{\large 3}}} 
\notag \\
&&\left( \text{by further application of H\"{o}lder's inequality}\right) 
\notag \\
&=&\frac{C_{1}\overline{C}}{q\tau _{1}^{6}}\dsum\limits_{r=1}^{q}\dsum%
\limits_{i\in H^{{\large c}}}\dsum\limits_{\substack{ t,s,g=\left(
r-1\right) \tau +p  \\ t\leq s\leq g}}^{\left( r-1\right) \tau +\tau
_{1}+p-1}\left( E\left\{ \left\vert u_{it}\right\vert ^{6}\right\} \right) ^{%
\frac{{\large 1}}{{\large 6}}}\left( E\left\{ \left\vert u_{it}\right\vert
^{6}\right\} \right) ^{\frac{{\large 1}}{{\large 6}}}\left( E\left[
\left\vert u_{ig}\right\vert ^{6}\right] \right) ^{\frac{{\large 2}}{{\large %
3}}}  \notag \\
&\leq &\frac{C_{1}\overline{C}}{q\tau _{1}^{6}}\dsum\limits_{r=1}^{q}\dsum%
\limits_{i\in H^{{\large c}}}\dsum\limits_{\substack{ t,s,g=\left(
r-1\right) \tau +p  \\ t\leq s\leq g}}^{\left( r-1\right) \tau +\tau
_{1}+p-1}\overline{C}\text{ \ }\left( \text{by Assumption 2-3(b)}\right) 
\notag \\
&\leq &C_{1}\overline{C}^{2}\frac{N_{1}}{\tau _{1}^{5}}  \notag \\
&=&O\left( \frac{N_{1}}{\tau _{1}^{5}}\right) \text{.}  \label{T1}
\end{eqnarray}

Next, consider $\mathcal{T}_{2}$. For this term, note first that by
Assumption 2-3(c), $\left\{ u_{it}\right\} _{t=-\infty }^{\infty }$ is $%
\beta $-mixing with $\beta $ mixing coefficient satisfying 
\begin{equation*}
\beta _{i}\left( m\right) \leq a_{1}\exp \left\{ -a_{2}m\right\}
\end{equation*}%
for every $i$. Since $\alpha _{i,m}\leq \beta _{i}\left( m\right) $, it
follows that $\left\{ u_{it}\right\} _{t=-\infty }^{\infty }$ is $\alpha $%
-mixing as well, with $\alpha $ mixing coefficient satisfying%
\begin{equation*}
\alpha _{i,m}\leq a_{1}\exp \left\{ -a_{2}m\right\} \text{ for every }i\text{%
.}
\end{equation*}

\noindent Hence, we apply Lemma OA-3 with $p=5/4$ and $r=6$ to obtain%
\begin{eqnarray*}
&&\mathcal{T}_{2} \\
&=&\frac{C_{1}\overline{C}}{q\tau _{1}^{6}}\dsum\limits_{r=1}^{q}\dsum%
\limits_{i\in H^{{\large c}}}\dsum\limits_{\substack{ t,s,g,h,v,w=\left(
r-1\right) \tau +p  \\ t\leq s\leq g\leq h\leq v\leq w  \\ w-v\geq \max
\left\{ v-h,h-g\right\} ,w-v>0}}^{\left( r-1\right) \tau +\tau
_{1}+p-1}\left\vert E\left[ u_{it}u_{is}u_{ig}u_{ih}u_{iv}u_{iw}\right]
\right\vert \\
&\leq &\frac{C_{1}\overline{C}}{q\tau _{1}^{6}}\dsum\limits_{r=1}^{q}\dsum%
\limits_{i\in H^{{\large c}}}\dsum\limits_{\substack{ t,s,g,h,v,w=\left(
r-1\right) \tau +p  \\ t\leq s\leq g\leq h\leq v\leq w  \\ w-v\geq \max
\left\{ v-h,h-g\right\} ,w-v>0}}^{\left( r-1\right) \tau +\tau
_{1}+p-1}\left\{ 2\left( 2^{1-\frac{{\large 4}}{{\large 5}}}+1\right) \left[
a_{1}\exp \left\{ -a_{2}\left( w-v\right) \right\} \right] ^{1-\frac{{\large %
4}}{{\large 5}}-\frac{{\large 1}}{{\large 6}}}\right. \\
&&\text{ \ \ \ \ \ \ \ \ \ \ \ \ \ \ \ \ \ \ \ \ \ \ \ \ \ \ \ \ \ \ \ \ \ \
\ \ \ \ \ \ \ \ \ \ \ \ \ \ \ \ \ \ \ \ \ }\left. \times \left( E\left\vert
u_{it}u_{is}u_{ig}u_{ih}u_{iv}\right\vert ^{\frac{{\large 5}}{{\large 4}}%
}\right) ^{\frac{{\large 4}}{{\large 5}}}\left( E\left\vert
u_{iw}\right\vert ^{6}\right) ^{\frac{{\large 1}}{{\large 6}}}\right\}
\end{eqnarray*}%
Next, by Liapunov's inequality and Assumption 2-3(b), we obtain%
\begin{equation*}
\left( E\left\vert u_{iw}\right\vert ^{6}\right) ^{\frac{{\large 1}}{{\large %
6}}}\leq \left( E\left\vert u_{iw}\right\vert ^{7}\right) ^{\frac{{\large 1}%
}{{\large 7}}}\leq \overline{C}^{\frac{{\large 1}}{{\large 7}}}
\end{equation*}%
Making use of this bound and by repeated application of H\"{o}lder's
inequality, we have 
\begin{eqnarray*}
&&E\left\vert u_{it}u_{is}u_{ig}u_{ih}u_{iv}\right\vert ^{\frac{{\large 5}}{%
{\large 4}}} \\
&\leq &\left[ E\left\vert u_{it}u_{is}u_{ig}\right\vert ^{\frac{{\large 25}}{%
{\large 12}}}\right] ^{\frac{{\large 3}}{{\large 5}}}\left[ E\left\vert
u_{ih}u_{iv}\right\vert ^{\frac{{\large 25}}{{\large 8}}}\right] ^{\frac{%
{\large 2}}{{\large 5}}} \\
&\leq &\left[ \left( E\left\vert u_{it}u_{is}\right\vert ^{\frac{{\large 150}%
}{{\large 47}}}\right) ^{\frac{{\large 47}}{{\large 72}}}\left( E\left\vert
u_{ig}\right\vert ^{6}\right) ^{\frac{{\large 25}}{{\large 72}}}\right] ^{%
\frac{{\large 3}}{{\large 5}}}\left[ \left( E\left\vert u_{ih}\right\vert ^{%
\frac{{\large 25}}{{\large 4}}}\right) ^{\frac{{\large 1}}{{\large 2}}%
}\left( E\left\vert u_{iv}\right\vert ^{\frac{{\large 25}}{{\large 4}}%
}\right) ^{\frac{{\large 1}}{{\large 2}}}\right] ^{\frac{{\large 2}}{{\large %
5}}} \\
&\leq &\left[ \left( \sqrt{E\left\vert u_{it}\right\vert ^{\frac{{\large 300}%
}{{\large 47}}}}\sqrt{E\left\vert u_{is}\right\vert ^{\frac{{\large 300}}{%
{\large 47}}}}\right) ^{\frac{{\large 47}}{{\large 72}}}\left( E\left\vert
u_{ig}\right\vert ^{6}\right) ^{\frac{{\large 25}}{{\large 72}}}\right] ^{%
\frac{{\large 3}}{{\large 5}}}\left[ \left( E\left\vert u_{ih}\right\vert ^{%
\frac{{\large 25}}{{\large 4}}}\right) ^{\frac{{\large 1}}{{\large 2}}%
}\left( E\left\vert u_{iv}\right\vert ^{\frac{{\large 25}}{{\large 4}}%
}\right) ^{\frac{{\large 1}}{{\large 2}}}\right] ^{\frac{{\large 2}}{{\large %
5}}} \\
&=&\left( E\left\vert u_{it}\right\vert ^{\frac{{\large 300}}{{\large 47}}%
}\right) ^{\frac{{\large 141}}{{\large 720}}}\left( E\left\vert
u_{is}\right\vert ^{\frac{{\large 300}}{{\large 47}}}\right) ^{\frac{{\large %
141}}{{\large 720}}}\left( E\left\vert u_{ih}\right\vert ^{6}\right) ^{\frac{%
{\large 15}}{{\large 72}}}\left( E\left\vert u_{iv}\right\vert ^{\frac{%
{\large 25}}{{\large 4}}}\right) ^{\frac{{\large 1}}{{\large 5}}}\left(
E\left\vert u_{iw}\right\vert ^{\frac{{\large 25}}{{\large 4}}}\right) ^{%
\frac{{\large 1}}{{\large 5}}} \\
&=&\left[ \left( E\left\vert u_{it}\right\vert ^{\frac{{\large 300}}{{\large %
47}}}\right) ^{\frac{{\large 47}}{{\large 300}}}\left( E\left\vert
u_{is}\right\vert ^{\frac{{\large 300}}{{\large 47}}}\right) ^{\frac{{\large %
47}}{{\large 300}}}\right] ^{\frac{{\large 5}}{{\large 4}}}\left[ \left(
E\left\vert u_{ih}\right\vert ^{6}\right) ^{\frac{{\large 1}}{{\large 6}}}%
\right] ^{\frac{{\large 5}}{{\large 4}}}\left[ \left( E\left\vert
u_{iv}\right\vert ^{\frac{{\large 25}}{{\large 4}}}\right) ^{\frac{{\large 4}%
}{{\large 25}}}\right] ^{\frac{{\large 5}}{{\large 4}}}\left[ \left(
E\left\vert u_{iw}\right\vert ^{\frac{{\large 25}}{{\large 4}}}\right) ^{%
\frac{{\large 4}}{{\large 25}}}\right] ^{\frac{{\large 5}}{{\large 4}}} \\
&\leq &\left[ \left( E\left\vert u_{it}\right\vert ^{7}\right) ^{\frac{%
{\large 1}}{{\large 7}}}\left( E\left\vert u_{is}\right\vert ^{7}\right) ^{%
\frac{{\large 1}}{{\large 7}}}\right] ^{\frac{{\large 5}}{{\large 4}}}\left[
\left( E\left\vert u_{ih}\right\vert ^{7}\right) ^{\frac{{\large 1}}{{\large %
7}}}\right] ^{\frac{{\large 5}}{{\large 4}}}\left[ \left( E\left\vert
u_{iv}\right\vert ^{7}\right) ^{\frac{{\large 1}}{{\large 7}}}\right] ^{%
\frac{{\large 5}}{{\large 4}}}\left[ \left( E\left\vert u_{iw}\right\vert
^{7}\right) ^{\frac{{\large 1}}{{\large 7}}}\right] ^{\frac{{\large 5}}{%
{\large 4}}} \\
&\leq &\left( \overline{C}\right) ^{\frac{{\large 5}}{{\large 28}}}\left( 
\overline{C}\right) ^{\frac{{\large 5}}{{\large 28}}}\left( \overline{C}%
\right) ^{\frac{{\large 5}}{{\large 28}}}\left( \overline{C}\right) ^{\frac{%
{\large 5}}{{\large 28}}}\left( \overline{C}\right) ^{\frac{{\large 5}}{%
{\large 28}}}\text{ \ }\left( \text{by Assumption 2-3(b)}\right) \\
&=&\overline{C}^{\frac{{\large 25}}{{\large 28}}}
\end{eqnarray*}%
Moreover, let $\rho _{1}=h-g$, $\rho _{2}=v-h$, and $\rho _{3}=w-v$, so that 
$h=g+\rho _{1}$, $v=h+$ $\rho _{2}=g+\rho _{1}+$ $\rho _{2}$, $w=v+\rho
_{3}=g+\rho _{1}+$ $\rho _{2}+\rho _{3}$. Using these notations and the
boundedness of $E\left\vert u_{it}u_{is}u_{ig}u_{ih}u_{iv}\right\vert ^{%
\frac{{\large 5}}{{\large 4}}}$ as shown above, we can further write%
\begin{eqnarray}
&&\mathcal{T}_{2}  \notag \\
&\leq &\frac{C_{1}\overline{C}}{q\tau _{1}^{6}}\dsum\limits_{r=1}^{q}\dsum%
\limits_{i\in H^{{\large c}}}\dsum\limits_{\substack{ t,s,g,h,v,w=\left(
r-1\right) \tau +p  \\ t\leq s\leq g\leq h\leq v\leq w  \\ w-v\geq \max
\left\{ v-h,h-g\right\} ,w-v>0}}^{\left( r-1\right) \tau +\tau
_{1}+p-1}\left\{ 2\left( 2^{1-\frac{{\large 4}}{{\large 5}}}+1\right) \left[
a_{1}\exp \left\{ -a_{2}\left( w-v\right) \right\} \right] ^{1-\frac{{\large %
4}}{{\large 5}}-\frac{{\large 1}}{{\large 6}}}\right.  \notag \\
&&\text{ \ \ \ \ \ \ \ \ \ \ \ \ \ \ \ \ \ \ \ \ \ \ \ \ \ \ \ \ \ \ \ \ \ \
\ \ \ \ \ \ \ \ \ \ \ \ \ \ \ \ \ \ }\left. \times \left( E\left\vert
u_{it}u_{is}u_{ig}u_{ih}u_{iv}\right\vert ^{\frac{{\large 5}}{{\large 4}}%
}\right) ^{\frac{{\large 4}}{{\large 5}}}\left( E\left\vert
u_{iw}\right\vert ^{6}\right) ^{\frac{{\large 1}}{{\large 6}}}\right\} 
\notag \\
&\leq &\frac{C_{1}\overline{C}}{q\tau _{1}^{6}}\dsum\limits_{r=1}^{q}\dsum%
\limits_{i\in H^{{\large c}}}\dsum\limits_{\substack{ t,s,g,h,v,w=\left(
r-1\right) \tau +p  \\ t\leq s\leq g\leq h\leq v\leq w  \\ w-v\geq \max
\left\{ v-h,h-g\right\} ,w-v>0}}^{\left( r-1\right) \tau +\tau
_{1}+p-1}2\left( 2^{\frac{{\large 1}}{{\large 5}}}+1\right) \left[ a_{1}\exp
\left\{ -a_{2}\left( w-v\right) \right\} \right] ^{\frac{{\large 1}}{{\large %
30}}}\overline{C}^{\frac{{\large 25}}{{\large 28}}}\overline{C}^{\frac{%
{\large 1}}{{\large 7}}}  \notag \\
&\leq &\frac{C_{1}\overline{C}^{\frac{{\large 57}}{{\large 28}}}}{q\tau
_{1}^{6}}\dsum\limits_{r=1}^{q}\dsum\limits_{i\in H^{{\large c}%
}}\dsum\limits _{\substack{ t,s,g,h,v,w=\left( r-1\right) \tau +p  \\ t\leq
s\leq g\leq h\leq v\leq w  \\ w-v\geq \max \left\{ v-h,h-g\right\} ,w-v>0}}%
^{\left( r-1\right) \tau +\tau _{1}+p-1}2\left( 2^{\frac{{\large 1}}{{\large %
5}}}+1\right) \left[ a_{1}\exp \left\{ -a_{2}\left( w-v\right) \right\} %
\right] ^{\frac{{\large 1}}{{\large 30}}}  \notag \\
&\leq &\frac{C^{\ast }}{q\tau _{1}^{6}}\dsum\limits_{r=1}^{q}\dsum\limits_{i%
\in H^{{\large c}}}\dsum\limits_{\substack{ t,s,g,h,v,w=\left( r-1\right)
\tau +p  \\ t\leq s\leq g\leq h\leq v\leq w  \\ w-v\geq \max \left\{
v-h,h-g\right\} ,w-v>0}}^{\left( r-1\right) \tau +\tau _{1}+p-1}\exp \left\{
-\frac{a_{2}}{30}\rho _{3}\right\} \text{ }  \notag \\
&&\left( \text{for some constant }C^{\ast }\text{ such that }2\left( 2^{%
\frac{{\large 1}}{{\large 5}}}+1\right) C_{1}\overline{C}^{\frac{{\large 57}%
}{{\large 28}}}a_{1}^{\frac{{\large 1}}{{\large 30}}}\leq C^{\ast }<\infty
\right)  \notag \\
&\leq &\frac{C^{\ast }}{q\tau _{1}^{6}}\dsum\limits_{r=1}^{q}\dsum\limits_{i%
\in H^{{\large c}}}\dsum\limits_{\substack{ t=\left( r-1\right) \tau +p}}%
^{\left( r-1\right) \tau +\tau _{1}+p-1}\dsum\limits_{\substack{ s=\left(
r-1\right) \tau +p}}^{\left( r-1\right) \tau +\tau _{1}+p-1}\dsum\limits 
_{\substack{ g=\left( r-1\right) \tau +p}}^{\left( r-1\right) \tau +\tau
_{1}+p-1}\dsum\limits_{\substack{ \rho _{{\large 3}}=1}}^{\infty
}\dsum\limits_{\rho _{{\large 1}}=0}^{\rho _{{\large 3}}}\dsum\limits_{\rho
_{{\large 2}}=0}^{\rho _{{\large 3}}}\exp \left\{ -\frac{a_{2}}{30}\rho
_{3}\right\}  \notag \\
&\leq &\frac{C^{\ast }}{q\tau _{1}^{6}}\dsum\limits_{r=1}^{q}\dsum\limits_{i%
\in H^{{\large c}}}\dsum\limits_{\substack{ t=\left( r-1\right) \tau +p}}%
^{\left( r-1\right) \tau +\tau _{1}+p-1}\dsum\limits_{\substack{ s=\left(
r-1\right) \tau +p}}^{\left( r-1\right) \tau +\tau _{1}+p-1}\dsum\limits 
_{\substack{ g=\left( r-1\right) \tau +p}}^{\left( r-1\right) \tau +\tau
_{1}+p-1}\dsum\limits_{\substack{ \rho _{{\large 3}}=1}}^{\infty }\left(
\rho _{3}+1\right) ^{2}\exp \left\{ -\frac{a_{2}}{30}\rho _{3}\right\} 
\notag \\
&=&C^{\ast }\frac{N_{1}}{\tau _{1}^{3}}\left[ \dsum\limits_{\substack{ \rho
_{{\large 3}}=1}}^{\infty }\rho _{3}^{2}\exp \left\{ -\frac{a_{2}}{30}\rho
_{3}\right\} +2\dsum\limits_{\substack{ \rho _{{\large 3}}=1}}^{\infty }\rho
_{3}\exp \left\{ -\frac{a_{2}}{30}\rho _{3}\right\} +\dsum\limits_{\substack{
\rho _{{\large 3}}=1}}^{\infty }\exp \left\{ -\frac{a_{2}}{30}\rho
_{3}\right\} \right]  \notag \\
&=&O\left( \frac{N_{1}}{\tau _{1}^{3}}\right) \text{ \ }\left( \text{by
Lemma OA-1}\right) \text{. }  \label{T2}
\end{eqnarray}

Now, consider $\mathcal{T}_{3}$. Here, we can apply Lemma OA-3 with $p=3/2$
and $r=7/2$ to obtain%
\begin{eqnarray*}
\mathcal{T}_{3} &=&\frac{C_{1}\overline{C}}{q\tau _{1}^{6}}%
\dsum\limits_{r=1}^{q}\dsum\limits_{i\in H^{{\large c}}}\dsum\limits 
_{\substack{ t,s,g,h,v,w=\left( r-1\right) \tau +p  \\ t\leq s\leq g\leq
h\leq v\leq w  \\ v-h\geq \max \left\{ w-v,h-g\right\} ,v-h>0}}^{\left(
r-1\right) \tau +\tau _{1}+p-1}\left\vert E\left[ \left\{
u_{it}u_{is}u_{ig}u_{ih}-E\left( u_{it}u_{is}u_{ig}u_{ih}\right) \right\}
u_{iv}u_{iw}\right] \right\vert \\
&\leq &\frac{C_{1}\overline{C}}{q\tau _{1}^{6}}\dsum\limits_{r=1}^{q}\dsum%
\limits_{i\in H^{{\large c}}}\dsum\limits_{\substack{ t,s,g,h,v,w=\left(
r-1\right) \tau +p  \\ t\leq s\leq g\leq h\leq v\leq w  \\ v-h\geq \max
\left\{ w-v,h-g\right\} ,v-h>0}}^{\left( r-1\right) \tau +\tau
_{1}+p-1}\left\{ 2\left( 2^{1-\frac{{\large 2}}{{\large 3}}}+1\right) \left[
a_{1}\exp \left\{ -a_{2}\left( v-h\right) \right\} \right] ^{1-\frac{{\large %
2}}{{\large 3}}-\frac{{\large 2}}{{\large 7}}}\right. \\
&&\text{ \ \ \ \ \ \ \ \ \ \ \ \ \ \ \ \ \ \ \ \ \ \ \ \ \ \ \ \ \ }\left.
\times \left( E\left\vert \left\{ u_{it}u_{is}u_{ig}u_{ih}-E\left(
u_{it}u_{is}u_{ig}u_{ih}\right) \right\} \right\vert ^{\frac{{\large 3}}{%
{\large 2}}}\right) ^{\frac{{\large 2}}{{\large 3}}}\left( E\left\vert
u_{iv}u_{iw}\right\vert ^{\frac{{\large 7}}{{\large 2}}}\right) ^{\frac{%
{\large 2}}{{\large 7}}}\right\}
\end{eqnarray*}%
Next, note that applications of H\"{o}lder's inequality yield%
\begin{eqnarray*}
E\left\vert u_{iv}u_{iw}\right\vert ^{\frac{{\large 7}}{{\large 2}}} &\leq
&\left( E\left\vert u_{iv}\right\vert ^{7}\right) ^{\frac{{\large 1}}{%
{\large 2}}}\left( E\left\vert u_{iw}\right\vert ^{7}\right) ^{\frac{{\large %
1}}{{\large 2}}} \\
&\leq &\left( \overline{C}\right) ^{\frac{{\large 1}}{{\large 2}}}\left( 
\overline{C}\right) ^{\frac{{\large 1}}{{\large 2}}}\text{ \ }\left( \text{%
by Assumption 2-3(b)}\right) \\
&=&\overline{C}<\infty
\end{eqnarray*}%
and%
\begin{eqnarray*}
E\left\vert \left\{ u_{it}u_{is}u_{ig}u_{ih}-E\left(
u_{it}u_{is}u_{ig}u_{ih}\right) \right\} \right\vert ^{\frac{{\large 3}}{%
{\large 2}}} &\leq &2^{\frac{{\large 1}}{{\large 2}}}\left( E\left\vert
u_{it}u_{is}u_{ig}u_{ih}\right\vert ^{\frac{{\large 3}}{{\large 2}}%
}+E\left\vert u_{it}u_{is}u_{ig}u_{ih}\right\vert ^{\frac{{\large 3}}{%
{\large 2}}}\right) \\
&&\left( \text{by Lo\`{e}ve's }c_{r}\text{ inequality}\right) \\
&\leq &2^{\frac{{\large 3}}{{\large 2}}}E\left\vert
u_{it}u_{is}u_{ig}u_{ih}\right\vert ^{\frac{{\large 3}}{{\large 2}}} \\
&\leq &2^{\frac{{\large 3}}{{\large 2}}}\left( E\left\vert
u_{it}u_{is}\right\vert ^{3}\right) ^{\frac{{\large 1}}{{\large 2}}}\left(
E\left\vert u_{ig}u_{ih}\right\vert ^{3}\right) ^{\frac{{\large 1}}{{\large 2%
}}} \\
&\leq &2^{\frac{{\large 3}}{{\large 2}}}\left( \left( E\left\vert
u_{it}\right\vert ^{6}\right) ^{\frac{{\large 1}}{{\large 2}}}\left(
E\left\vert u_{is}\right\vert ^{6}\right) ^{\frac{{\large 1}}{{\large 2}}%
}\right) ^{\frac{{\large 1}}{{\large 2}}}\left( \left( E\left\vert
u_{ig}\right\vert ^{6}\right) ^{\frac{{\large 1}}{{\large 2}}}\left(
E\left\vert u_{ih}\right\vert ^{6}\right) ^{\frac{{\large 1}}{{\large 2}}%
}\right) ^{\frac{{\large 1}}{{\large 2}}} \\
&\leq &2^{\frac{{\large 3}}{{\large 2}}}\left[ \left( E\left\vert
u_{it}\right\vert ^{6}\right) ^{\frac{{\large 1}}{{\large 6}}}\left(
E\left\vert u_{is}\right\vert ^{6}\right) ^{\frac{{\large 1}}{{\large 6}}%
}\left( E\left\vert u_{ig}\right\vert ^{6}\right) ^{\frac{{\large 1}}{%
{\large 6}}}\left( E\left\vert u_{ih}\right\vert ^{6}\right) ^{\frac{{\large %
1}}{{\large 6}}}\right] ^{\frac{{\large 3}}{{\large 2}}} \\
&\leq &2^{\frac{{\large 3}}{{\large 2}}}\left[ \left( E\left\vert
u_{it}\right\vert ^{7}\right) ^{\frac{{\large 1}}{{\large 7}}}\left(
E\left\vert u_{is}\right\vert ^{7}\right) ^{\frac{{\large 1}}{{\large 7}}%
}\left( E\left\vert u_{ig}\right\vert ^{7}\right) ^{\frac{{\large 1}}{%
{\large 7}}}\left( E\left\vert u_{ih}\right\vert ^{7}\right) ^{\frac{{\large %
1}}{{\large 7}}}\right] ^{\frac{{\large 3}}{{\large 2}}} \\
&&\left( \text{by Liapunov's inequality}\right) \\
&\leq &2^{\frac{{\large 3}}{{\large 2}}}\left[ \left( \sup_{i,t}E\left\vert
u_{it}\right\vert ^{7}\right) ^{\frac{{\large 4}}{{\large 7}}}\right] ^{%
\frac{{\large 3}}{{\large 2}}}\text{ \ } \\
&=&2^{\frac{{\large 3}}{{\large 2}}}\overline{C}^{\frac{{\large 6}}{{\large 7%
}}}\text{ \ }\left( \text{by Assumption 2-3(b)}\right)
\end{eqnarray*}%
Again, let $\rho _{1}=h-g$, $\rho _{2}=v-h$, and $\rho _{3}=w-v$, so that $%
h=g+\rho _{1}$, $v=h+$ $\rho _{2}=g+\rho _{1}+$ $\rho _{2}$, $w=v+\rho
_{3}=g+\rho _{1}+$ $\rho _{2}+\rho _{3}$. Using these notations and the
boundedness of $E\left\vert u_{iv}u_{iw}\right\vert ^{\frac{{\large 7}}{%
{\large 2}}}$ and $E\left\vert \left\{ u_{it}u_{is}u_{ig}u_{ih}-E\left(
u_{it}u_{is}u_{ig}u_{ih}\right) \right\} \right\vert ^{\frac{{\large 3}}{%
{\large 2}}}$ as shown above, we can further write%
\begin{eqnarray}
\mathcal{T}_{3} &=&\frac{C_{1}\overline{C}}{q\tau _{1}^{6}}%
\dsum\limits_{r=1}^{q}\dsum\limits_{i\in H^{{\large c}}}\dsum\limits 
_{\substack{ t,s,g,h,v,w=\left( r-1\right) \tau +p  \\ t\leq s\leq g\leq
h\leq v\leq w  \\ v-h\geq \max \left\{ w-v,h-g\right\} ,v-h>0}}^{\left(
r-1\right) \tau +\tau _{1}+p-1}\left\vert E\left[ \left\{
u_{it}u_{is}u_{ig}u_{ih}-E\left( u_{it}u_{is}u_{ig}u_{ih}\right) \right\}
u_{iv}u_{iw}\right] \right\vert  \notag \\
&\leq &\frac{C_{1}\overline{C}}{q\tau _{1}^{6}}\dsum\limits_{r=1}^{q}\dsum%
\limits_{i\in H^{{\large c}}}\dsum\limits_{\substack{ t,s,g,h,v,w=\left(
r-1\right) \tau +p  \\ t\leq s\leq g\leq h\leq v\leq w  \\ v-h\geq \max
\left\{ w-v,h-g\right\} ,v-h>0}}^{\left( r-1\right) \tau +\tau
_{1}+p-1}\left\{ 2\left( 2^{1-\frac{{\large 2}}{{\large 3}}}+1\right) \left[
a_{1}\exp \left\{ -a_{2}\left( v-h\right) \right\} \right] ^{1-\frac{{\large %
2}}{{\large 3}}-\frac{{\large 2}}{{\large 7}}}\right.  \notag \\
&&\text{ \ \ \ \ \ \ \ \ \ \ \ \ \ \ \ \ \ \ \ \ \ \ \ \ \ \ \ \ \ \ \ }%
\left. \times \left( E\left\vert \left\{ u_{it}u_{is}u_{ig}u_{ih}-E\left(
u_{it}u_{is}u_{ig}u_{ih}\right) \right\} \right\vert ^{\frac{{\large 3}}{%
{\large 2}}}\right) ^{\frac{{\large 2}}{{\large 3}}}\left( E\left\vert
u_{iv}u_{iw}\right\vert ^{\frac{{\large 7}}{{\large 2}}}\right) ^{\frac{%
{\large 2}}{{\large 7}}}\right\}  \notag \\
&\leq &\frac{C_{1}\overline{C}}{q\tau _{1}^{6}}\dsum\limits_{r=1}^{q}\dsum%
\limits_{i\in H^{{\large c}}}\dsum\limits_{\substack{ t,s,g,h,v,w=\left(
r-1\right) \tau +p  \\ t\leq s\leq g\leq h\leq v\leq w  \\ v-h\geq \max
\left\{ w-v,h-g\right\} ,v-h>0}}^{\left( r-1\right) \tau +\tau
_{1}+p-1}2\left( 2^{\frac{{\large 1}}{{\large 3}}}+1\right) \left[ a_{1}\exp
\left\{ -a_{2}\left( v-h\right) \right\} \right] ^{\frac{{\large 1}}{{\large %
21}}}\left( 2^{\frac{{\large 3}}{{\large 2}}}\overline{C}^{\frac{{\large 6}}{%
{\large 7}}}\right) ^{\frac{{\large 2}}{{\large 3}}}\left( \overline{C}%
\right) ^{\frac{{\large 2}}{{\large 7}}}  \notag \\
&\leq &\frac{C^{\ast }}{q\tau _{1}^{6}}\dsum\limits_{r=1}^{q}\dsum\limits_{i%
\in H^{{\large c}}}\dsum\limits_{\substack{ t,s,g,h,v,w=\left( r-1\right)
\tau +p  \\ t\leq s\leq g\leq h\leq v\leq w  \\ v-h\geq \max \left\{
w-v,h-g\right\} ,v-h>0}}^{\left( r-1\right) \tau +\tau _{1}+p-1}\exp \left\{
-\frac{a_{2}}{21}\varrho _{2}\right\}  \notag \\
&&\left( \text{for some constant }C^{\ast }\text{ such that }4\left( 2^{%
\frac{{\large 1}}{{\large 3}}}+1\right) C_{1}\overline{C}^{\frac{{\large 13}%
}{{\large 7}}}a_{1}^{\frac{{\large 1}}{{\large 21}}}\leq C^{\ast }<\infty
\right)  \notag \\
&\leq &\frac{C^{\ast }}{q\tau _{1}^{6}}\dsum\limits_{r=1}^{q}\dsum\limits_{i%
\in H^{{\large c}}}\dsum\limits_{\substack{ t=\left( r-1\right) \tau +p}}%
^{\left( r-1\right) \tau +\tau _{1}+p-1}\dsum\limits_{\substack{ s=\left(
r-1\right) \tau +p}}^{\left( r-1\right) \tau +\tau _{1}+p-1}\dsum\limits 
_{\substack{ g=\left( r-1\right) \tau +p}}^{\left( r-1\right) \tau +\tau
_{1}+p-1}\dsum\limits_{\varrho _{{\large 2}}=1}^{\infty
}\dsum\limits_{\varrho _{{\large 1}}=0}^{\varrho _{{\large 2}%
}}\dsum\limits_{\varrho _{{\large 3}}=0}^{\varrho _{{\large 2}}}\exp \left\{
-\frac{a_{2}}{21}\varrho _{2}\right\}  \notag \\
&\leq &C^{\ast }\frac{N_{1}}{\tau _{1}^{3}}\dsum\limits_{\varrho _{{\large 2}%
}=1}^{\infty }\left( \varrho _{{\large 2}}+1\right) ^{2}\exp \left\{ -\frac{%
a_{2}}{21}\varrho _{2}\right\}  \notag \\
&=&C^{\ast }\frac{N_{1}}{\tau _{1}^{3}}\left[ \dsum\limits_{\varrho _{%
{\large 2}}=1}^{\infty }\varrho _{{\large 2}}^{2}\exp \left\{ -\frac{a_{2}}{%
21}\varrho _{2}\right\} +2\dsum\limits_{\varrho _{{\large 2}}=1}^{\infty
}\varrho _{{\large 2}}\exp \left\{ -\frac{a_{2}}{21}\varrho _{2}\right\}
+\dsum\limits_{\varrho _{{\large 2}}=1}^{\infty }\exp \left\{ -\frac{a_{2}}{%
21}\varrho _{2}\right\} \right]  \notag \\
&=&O\left( \frac{N_{1}}{\tau _{1}^{3}}\right) \text{ \ }\left( \text{by
Lemma OA-1}\right) \text{.}  \label{T3}
\end{eqnarray}

Turning our attention to the term $\mathcal{T}_{4}$, note that, from the
upper bounds given in the proofs of parts (a) and (c) of Lemma OA-4, it is
clear that there exists a positive constant $C$ such that%
\begin{equation*}
\frac{1}{\tau _{1}^{4}}\dsum\limits_{\substack{ t,s,g,h=\left( r-1\right)
\tau +p  \\ t\leq s\leq g\leq h}}^{\left( r-1\right) \tau +\tau
_{1}+p-1}\left\vert E\left( u_{it}u_{is}u_{ig}u_{ih}\right) \right\vert \leq 
\frac{C}{\tau _{1}^{2}}
\end{equation*}%
and%
\begin{equation*}
\frac{1}{\tau _{1}^{2}}\dsum\limits_{\substack{ v,w=\left( r-1\right) \tau
+p  \\ v\leq w}}^{\left( r-1\right) \tau +\tau _{1}+p-1}\left\vert E\left(
u_{iv}u_{iw}\right) \right\vert \leq \frac{C}{\tau _{1}}
\end{equation*}%
from which it follows that%
\begin{eqnarray}
\mathcal{T}_{4} &=&\frac{C_{1}\overline{C}}{q\tau _{1}^{6}}%
\dsum\limits_{r=1}^{q}\dsum\limits_{i\in H^{{\large c}}}\dsum\limits 
_{\substack{ t,s,g,h,v,w=\left( r-1\right) \tau +p  \\ t\leq s\leq g\leq
h\leq v\leq w  \\ v-h\geq \max \left\{ w-v,h-g\right\} ,v-h>0}}^{\left(
r-1\right) \tau +\tau _{1}+p-1}\left\vert E\left(
u_{it}u_{is}u_{ig}u_{ih}\right) \right\vert \left\vert E\left(
u_{iv}u_{iw}\right) \right\vert  \notag \\
&\leq &\frac{C_{1}\overline{C}}{q}\dsum\limits_{r=1}^{q}\dsum\limits_{i\in
H^{{\large c}}}\left( \frac{1}{\tau _{1}^{4}}\dsum\limits_{\substack{ %
t,s,g,h=\left( r-1\right) \tau +p  \\ t\leq s\leq g\leq h}}^{\left(
r-1\right) \tau +\tau _{1}+p-1}\left\vert E\left(
u_{it}u_{is}u_{ig}u_{ih}\right) \right\vert \right) \left( \frac{1}{\tau
_{1}^{2}}\dsum\limits_{\substack{ v,w=\left( r-1\right) \tau +p  \\ v\leq w}}%
^{\left( r-1\right) \tau +\tau _{1}+p-1}\left\vert E\left(
u_{iv}u_{iw}\right) \right\vert \right)  \notag \\
&\leq &\frac{C_{1}\overline{C}}{q}\dsum\limits_{r=1}^{q}\dsum\limits_{i\in
H^{{\large c}}}\left( \frac{C}{\tau _{1}^{2}}\right) \left( \frac{C}{\tau
_{1}}\right)  \notag \\
&=&C_{1}\overline{C}C^{2}\frac{N_{1}}{\tau _{1}^{3}}  \notag \\
&=&O\left( \frac{N_{1}}{\tau _{1}^{3}}\right) \text{. }  \label{T4}
\end{eqnarray}

Consider now $\mathcal{T}_{5}$. In this case, we apply Lemma OA-3 with $p=2$
and $r=9/4$ to obtain%
\begin{eqnarray*}
\mathcal{T}_{5} &=&\frac{C_{1}\overline{C}}{q\tau _{1}^{6}}%
\dsum\limits_{r=1}^{q}\dsum\limits_{i\in H^{{\large c}}}\dsum\limits 
_{\substack{ t,s,g,h,v,w=\left( r-1\right) \tau +p  \\ t\leq s\leq g\leq
h\leq v\leq w  \\ h-g\geq \max \left\{ w-v,v-h\right\} ,h-g>0}}^{\left(
r-1\right) \tau +\tau _{1}+p-1}\left\vert E\left[ \left\{
u_{it}u_{is}u_{ig}-E\left( u_{it}u_{is}u_{ig}\right) \right\}
u_{ih}u_{iv}u_{iw}\right] \right\vert \\
&\leq &\frac{C_{1}\overline{C}}{q\tau _{1}^{6}}\dsum\limits_{r=1}^{q}\dsum%
\limits_{i\in H^{{\large c}}}\dsum\limits_{\substack{ t,s,g,h,v,w=\left(
r-1\right) \tau +p  \\ t\leq s\leq g\leq h\leq v\leq w  \\ h-g\geq \max
\left\{ w-v,v-h\right\} ,h-g>0}}^{\left( r-1\right) \tau +\tau
_{1}+p-1}\left\{ 2\left( 2^{1-\frac{{\large 1}}{{\large 2}}}+1\right) \left[
a_{1}\exp \left\{ -a_{2}\left( h-g\right) \right\} \right] ^{1-\frac{{\large %
1}}{{\large 2}}-\frac{{\large 4}}{{\large 9}}}\right. \\
&&\text{ \ \ \ \ \ \ \ \ \ \ \ \ \ \ \ \ \ \ \ \ \ \ \ \ \ \ \ \ \ \ \ \ \ \
\ \ \ \ }\left. \times \left( E\left\vert \left\{ u_{it}u_{is}u_{ig}-E\left(
u_{it}u_{is}u_{ig}\right) \right\} \right\vert ^{2}\right) ^{\frac{{\large 1}%
}{{\large 2}}}\left( E\left\vert u_{ih}u_{iv}u_{iw}\right\vert ^{\frac{%
{\large 9}}{{\large 4}}}\right) ^{\frac{{\large 4}}{{\large 9}}}\right\}
\end{eqnarray*}%
Next, by repeated application of H\"{o}lder's inequality, we obtain%
\begin{eqnarray*}
&&E\left\vert u_{ih}u_{iv}u_{iw}\right\vert ^{\frac{{\large 9}}{{\large 4}}}
\\
&\leq &\left[ E\left\vert u_{ih}\right\vert ^{7}\right] ^{\frac{{\large 9}}{%
{\large 28}}}\left[ E\left\vert u_{iv}u_{iw}\right\vert ^{\frac{{\large 63}}{%
{\large 19}}}\right] ^{\frac{{\large 19}}{{\large 28}}} \\
&\leq &\left[ E\left\vert u_{ih}\right\vert ^{7}\right] ^{\frac{{\large 9}}{%
{\large 28}}}\left[ \left( E\left\vert u_{iv}\right\vert ^{\frac{{\large 126}%
}{{\large 19}}}\right) ^{\frac{{\large 1}}{{\large 2}}}\left( E\left\vert
u_{iw}\right\vert ^{\frac{{\large 126}}{{\large 19}}}\right) ^{\frac{{\large %
1}}{{\large 2}}}\right] ^{\frac{{\large 19}}{{\large 28}}} \\
&=&\left[ E\left\vert u_{ih}\right\vert ^{7}\right] ^{\frac{{\large 9}}{%
{\large 28}}}\left( E\left\vert u_{iv}\right\vert ^{\frac{{\large 126}}{%
{\large 19}}}\right) ^{\frac{{\large 19}}{{\large 56}}}\left( E\left\vert
u_{iw}\right\vert ^{\frac{{\large 126}}{{\large 19}}}\right) ^{\frac{{\large %
19}}{{\large 56}}} \\
&=&\left[ E\left\vert u_{ih}\right\vert ^{7}\right] ^{\frac{{\large 9}}{%
{\large 28}}}\left[ \left( E\left\vert u_{iv}\right\vert ^{\frac{{\large 126}%
}{{\large 19}}}\right) ^{\frac{{\large 19}}{{\large 126}}}\left( E\left\vert
u_{iw}\right\vert ^{\frac{{\large 126}}{{\large 19}}}\right) ^{\frac{{\large %
19}}{{\large 126}}}\right] ^{\frac{{\large 9}}{{\large 4}}} \\
&\leq &\left[ E\left\vert u_{ih}\right\vert ^{7}\right] ^{\frac{{\large 9}}{%
{\large 28}}}\left[ \left( E\left\vert u_{iv}\right\vert ^{7}\right) ^{\frac{%
{\large 1}}{{\large 7}}}\left( E\left\vert u_{iw}\right\vert ^{7}\right) ^{%
\frac{{\large 1}}{{\large 7}}}\right] ^{\frac{{\large 9}}{{\large 4}}}\text{ 
}\left( \text{by Liapunov's inequality}\right) \\
&\leq &\left( \sup_{i,t}E\left\vert u_{it}\right\vert ^{7}\right) ^{\frac{%
{\large 27}}{{\large 28}}}\text{ } \\
&\leq &\overline{C}^{\frac{{\large 27}}{{\large 28}}}\text{ \ \ }\left( 
\text{by Assumption 2-3(b)}\right)
\end{eqnarray*}%
and%
\begin{eqnarray*}
E\left\vert \left\{ u_{it}u_{is}u_{ig}-E\left( u_{it}u_{is}u_{ig}\right)
\right\} \right\vert ^{2} &\leq &2\left( E\left\vert
u_{it}u_{is}u_{ig}\right\vert ^{2}+E\left\vert u_{it}u_{is}u_{ig}\right\vert
^{2}\right) \\
&&\left( \text{by Lo\`{e}ve's }c_{r}\text{ inequality}\right) \\
&\leq &4E\left\vert u_{it}u_{is}u_{ig}\right\vert ^{2} \\
&\leq &4\left( E\left\vert u_{it}\right\vert ^{6}\right) ^{\frac{{\large 1}}{%
{\large 3}}}\left( E\left\vert u_{is}u_{ig}\right\vert ^{3}\right) ^{\frac{%
{\large 2}}{{\large 3}}} \\
&\leq &4\left( E\left\vert u_{it}\right\vert ^{6}\right) ^{\frac{{\large 1}}{%
{\large 3}}}\left( \sqrt{E\left\vert u_{is}\right\vert ^{6}}\sqrt{%
E\left\vert u_{ig}\right\vert ^{6}}\right) ^{\frac{{\large 2}}{{\large 3}}}
\\
&=&4\left[ \left( E\left\vert u_{it}\right\vert ^{6}\right) ^{\frac{{\large 1%
}}{{\large 6}}}\right] ^{2}\left[ \left( E\left\vert u_{is}\right\vert
^{6}\right) ^{\frac{{\large 1}}{{\large 6}}}\left( E\left\vert
u_{ig}\right\vert ^{6}\right) ^{\frac{{\large 1}}{{\large 6}}}\right] ^{2} \\
&\leq &4\left[ \left( E\left\vert u_{it}\right\vert ^{7}\right) ^{\frac{%
{\large 1}}{{\large 7}}}\right] ^{2}\left[ \left( E\left\vert
u_{is}\right\vert ^{7}\right) ^{\frac{{\large 1}}{{\large 7}}}\left(
E\left\vert u_{ig}\right\vert ^{7}\right) ^{\frac{{\large 1}}{{\large 7}}}%
\right] ^{2} \\
&&\left( \text{by Liapunov's inequality}\right) \\
&\leq &4\left[ \left( \sup_{i,t}E\left\vert u_{it}\right\vert ^{7}\right) ^{%
\frac{{\large 1}}{{\large 7}}}\right] ^{6} \\
&\leq &4\overline{C}^{\frac{{\large 6}}{{\large 7}}}\text{ }\left( \text{by
Assumption 2-3(b)}\right)
\end{eqnarray*}%
Define again $\rho _{1}=h-g$, $\rho _{2}=v-h$, and $\rho _{3}=w-v$, so that $%
h=g+\rho _{1}$, $v=h+$ $\rho _{2}=g+\rho _{1}+$ $\rho _{2}$, $w=v+\rho
_{3}=g+\rho _{1}+$ $\rho _{2}+\rho _{3}$. Using these notations and the
boundedness of $E\left\vert u_{ih}u_{iv}u_{iw}\right\vert ^{\frac{{\large 9}%
}{{\large 4}}}$ and $E\left\vert \left\{ u_{it}u_{is}u_{ig}-E\left(
u_{it}u_{is}u_{ig}\right) \right\} \right\vert ^{2}$ as shown above, we can
further write%
\begin{eqnarray}
&&\mathcal{T}_{5}  \notag \\
&\leq &\frac{C_{1}\overline{C}}{q\tau _{1}^{6}}\dsum\limits_{r=1}^{q}\dsum%
\limits_{i\in H^{{\large c}}}\dsum\limits_{\substack{ t,s,g,h,v,w=\left(
r-1\right) \tau +p  \\ t\leq s\leq g\leq h\leq v\leq w  \\ h-g\geq \max
\left\{ w-v,v-h\right\} ,h-g>0}}^{\left( r-1\right) \tau +\tau
_{1}+p-1}\left\{ 2\left( 2^{1-\frac{{\large 1}}{{\large 2}}}+1\right) \left[
a_{1}\exp \left\{ -a_{2}\left( h-g\right) ^{{\large \theta }}\right\} \right]
^{1-\frac{{\large 1}}{{\large 2}}-\frac{{\large 4}}{{\large 9}}}\right. 
\notag \\
&&\text{ \ \ \ \ \ \ \ \ \ \ \ \ \ \ \ \ \ \ \ \ \ \ \ \ \ \ \ \ \ \ \ \ \ \
\ \ \ \ }\left. \times \left( E\left\vert \left\{ u_{it}u_{is}u_{ig}-E\left(
u_{it}u_{is}u_{ig}\right) \right\} \right\vert ^{2}\right) ^{\frac{{\large 1}%
}{{\large 2}}}\left( E\left\vert u_{ih}u_{iv}u_{iw}\right\vert ^{\frac{%
{\large 9}}{{\large 4}}}\right) ^{\frac{{\large 4}}{{\large 9}}}\right\} 
\notag \\
&\leq &\frac{C_{1}\overline{C}}{q\tau _{1}^{6}}\dsum\limits_{r=1}^{q}\dsum%
\limits_{i\in H^{{\large c}}}\dsum\limits_{\substack{ t,s,g,h,v,w=\left(
r-1\right) \tau +p  \\ t\leq s\leq g\leq h\leq v\leq w  \\ h-g\geq \max
\left\{ w-v,v-h\right\} ,h-g>0}}^{\left( r-1\right) \tau +\tau
_{1}+p-1}2\left( 2^{\frac{{\large 1}}{{\large 2}}}+1\right) \left[ a_{1}\exp
\left\{ -a_{2}\left( h-g\right) \right\} \right] ^{\frac{{\large 1}}{{\large %
18}}}\left( 4\overline{C}^{\frac{{\large 6}}{{\large 7}}}\right) ^{\frac{%
{\large 1}}{{\large 2}}}\left( \overline{C}^{\frac{{\large 27}}{{\large 28}}%
}\right) ^{\frac{{\large 4}}{{\large 9}}}  \notag \\
&\leq &\frac{C^{\ast }}{q\tau _{1}^{6}}\dsum\limits_{r=1}^{q}\dsum\limits_{i%
\in H^{{\large c}}}\dsum\limits_{\substack{ t,s,g,h,v,w=\left( r-1\right)
\tau +p  \\ t\leq s\leq g\leq h\leq v\leq w  \\ h-g\geq \max \left\{
w-v,v-h\right\} ,h-g>0}}^{\left( r-1\right) \tau +\tau _{1}+p-1}\exp \left\{
-\frac{a_{2}}{18}\varrho _{1}\right\}  \notag \\
&&\left( \text{for some constant }C^{\ast }\text{ such that }4\left( 2^{%
\frac{{\large 1}}{{\large 2}}}+1\right) C_{1}\overline{C}^{\frac{{\large 13}%
}{{\large 7}}}a_{1}^{\frac{{\large 1}}{{\large 18}}}\leq C^{\ast }<\infty
\right)  \notag \\
&\leq &\frac{C^{\ast }}{q\tau _{1}^{6}}\dsum\limits_{r=1}^{q}\dsum\limits_{i%
\in H^{{\large c}}}\dsum\limits_{\substack{ t=\left( r-1\right) \tau +p}}%
^{\left( r-1\right) \tau +\tau _{1}+p-1}\dsum\limits_{\substack{ s=\left(
r-1\right) \tau +p}}^{\left( r-1\right) \tau +\tau _{1}+p-1}\dsum\limits 
_{\substack{ g=\left( r-1\right) \tau +p}}^{\left( r-1\right) \tau +\tau
_{1}+p-1}\dsum\limits_{\varrho _{{\large 1}}=1}^{\infty
}\dsum\limits_{\varrho _{{\large 2}}=0}^{\varrho _{{\large 1}%
}}\dsum\limits_{\varrho _{{\large 3}}=0}^{\varrho _{{\large 1}}}\exp \left\{
-\frac{a_{2}}{18}\varrho _{1}\right\}  \notag \\
&\leq &C^{\ast }\frac{N_{1}}{\tau _{1}^{3}}\dsum\limits_{\varrho _{{\large 1}%
}=1}^{\infty }\left( \varrho _{{\large 1}}+1\right) ^{2}\exp \left\{ -\frac{%
a_{2}}{18}\varrho _{1}\right\}  \notag \\
&=&C^{\ast }\frac{N_{1}}{\tau _{1}^{3}}\left[ \dsum\limits_{\varrho _{%
{\large 1}}=1}^{\infty }\varrho _{{\large 1}}^{2}\exp \left\{ -\frac{a_{2}}{%
18}\varrho _{1}\right\} +2\dsum\limits_{\varrho _{{\large 1}}=1}^{\infty
}\varrho _{{\large 1}}\exp \left\{ -\frac{a_{2}}{18}\varrho _{1}\right\}
+\dsum\limits_{\varrho _{{\large 1}}=1}^{\infty }\exp \left\{ -\frac{a_{2}}{%
18}\varrho _{1}\right\} \right]  \notag \\
&=&O\left( \frac{N_{1}}{\tau _{1}^{3}}\right) \text{ \ }\left( \text{by
Lemma OA-1}\right)  \label{T5}
\end{eqnarray}

Finally, consider $\mathcal{T}_{6}$. Note that, from the upper bounds given
in the proofs of part (b) of Lemma OA-4, it is clear that there exists a
positive constant $C$ such that%
\begin{equation*}
\frac{1}{\tau _{1}^{3}}\dsum\limits_{\substack{ t,s,g=\left( r-1\right) \tau
+p  \\ t\leq s\leq g}}^{\left( r-1\right) \tau +\tau _{1}+p-1}\left\vert
E\left( u_{it}u_{is}u_{ig}\right) \right\vert \leq \frac{C}{\tau _{1}^{2}}
\end{equation*}%
and%
\begin{equation*}
\frac{1}{\tau _{1}^{3}}\dsum\limits_{\substack{ h,v,w=\left( r-1\right) \tau
+p  \\ h\leq v\leq w}}^{\left( r-1\right) \tau +\tau _{1}+p-1}\left\vert
E\left( u_{ih}u_{iv}u_{iw}\right) \right\vert \leq \frac{C}{\tau _{1}^{2}}
\end{equation*}%
from which it follows that%
\begin{eqnarray}
\mathcal{T}_{6} &=&\frac{C_{1}\overline{C}}{q\tau _{1}^{6}}%
\dsum\limits_{r=1}^{q}\dsum\limits_{i\in H^{{\large c}}}\dsum\limits 
_{\substack{ t,s,g,h,v,w=\left( r-1\right) \tau +p  \\ t\leq s\leq g\leq
h\leq v\leq w  \\ h-g\geq \max \left\{ w-v,v-h\right\} ,h-g>0}}^{\left(
r-1\right) \tau +\tau _{1}+p-1}\left\vert E\left( u_{it}u_{is}u_{ig}\right)
\right\vert \left\vert E\left( u_{ih}u_{iv}u_{iw}\right) \right\vert  \notag
\\
&\leq &\frac{C_{1}\overline{C}}{q}\dsum\limits_{r=1}^{q}\dsum\limits_{i\in
H^{{\large c}}}\left( \frac{1}{\tau _{1}^{3}}\dsum\limits_{\substack{ %
t,s,g=\left( r-1\right) \tau +p  \\ t\leq s\leq g}}^{\left( r-1\right) \tau
+\tau _{1}+p-1}\left\vert E\left( u_{it}u_{is}u_{ig}\right) \right\vert
\right) \left( \frac{1}{\tau _{1}^{3}}\dsum\limits_{\substack{ h,v,w=\left(
r-1\right) \tau +p  \\ h\leq v\leq w}}^{\left( r-1\right) \tau +\tau
_{1}+p-1}\left\vert E\left( u_{ih}u_{iv}u_{iw}\right) \right\vert \right) 
\notag \\
&\leq &\frac{C_{1}\overline{C}}{q}\dsum\limits_{r=1}^{q}\dsum\limits_{i\in
H^{{\large c}}}\left( \frac{C}{\tau _{1}^{2}}\right) \left( \frac{C}{\tau
_{1}^{2}}\right)  \notag \\
&=&C_{1}C\overline{C}^{2}\frac{N_{1}}{\tau _{1}^{4}}  \notag \\
&=&O\left( \frac{N_{1}}{\tau _{1}^{4}}\right) \text{. }  \label{T6}
\end{eqnarray}

It follows from expressions (\ref{Pmaxyu})-(\ref{T6}) that, for any $%
\epsilon >0$,

\begin{eqnarray*}
&&P\left\{ \max_{1\leq \ell \leq d}\max_{i\in H^{{\large c}}}\left\vert 
\frac{1}{q}\dsum\limits_{r=1}^{q}\frac{1}{\tau _{1}}\dsum\limits_{t=\left(
r-1\right) \tau {\LARGE +}p}^{\left( r-1\right) \tau +\tau _{1}{\LARGE +}%
p-1}y_{\ell ,t{\LARGE +}1}u_{it}\right\vert \geq \epsilon \right\} \\
&\leq &\frac{1}{\epsilon ^{6}}\frac{1}{q}\dsum\limits_{r=1}^{q}\dsum%
\limits_{\ell =1}^{d}\dsum\limits_{i\in H^{{\large c}}}E\left( \frac{1}{\tau
_{1}}\dsum\limits_{t=\left( r-1\right) \tau {\LARGE +}p}^{\left( r-1\right)
\tau +\tau _{1}{\LARGE +}p-1}y_{\ell ,t{\LARGE +}1}u_{it}\right) ^{6} \\
&\leq &\frac{1}{\epsilon ^{4}}\left( \mathcal{T}_{1}+\mathcal{T}_{2}+%
\mathcal{T}_{3}+\mathcal{T}_{4}+\mathcal{T}_{5}+\mathcal{T}_{6}\right) \\
&=&O\left( \frac{N_{1}}{\tau _{1}^{5}}\right) +O\left( \frac{N_{1}}{\tau
_{1}^{3}}\right) +O\left( \frac{N_{1}}{\tau _{1}^{3}}\right) +O\left( \frac{%
N_{1}}{\tau _{1}^{3}}\right) +O\left( \frac{N_{1}}{\tau _{1}^{3}}\right)
+O\left( \frac{N_{1}}{\tau _{1}^{4}}\right) \\
&=&O\left( \frac{N_{1}}{\tau _{1}^{3}}\right) \\
&=&o\left( 1\right) \text{ \ }\left( \text{by Assumption 2-9(b) which
stipulates that }\frac{N_{1}}{\tau _{1}^{3}}\sim \frac{N_{1}}{T^{3\alpha _{%
{\large 1}}}}\rightarrow 0\right)
\end{eqnarray*}%
which proves the required result.

Turning our attention to part (d), note that, for any $\epsilon >0$,%
\begin{eqnarray*}
&&P\left\{ \max_{1\leq \ell \leq d}\max_{i\in H^{{\large c}}}\frac{1}{q}%
\dsum\limits_{r=1}^{q}\left( \frac{1}{\tau _{1}}\dsum\limits_{t=\left(
r-1\right) \tau +p}^{\left( r-1\right) \tau +\tau _{1}+p-1}y_{\ell ,t{\LARGE %
+}1}u_{it}\right) ^{2}\geq \epsilon \right\} \\
&=&P\left\{ \max_{1\leq \ell \leq d}\max_{i\in H^{{\large c}}}\left\vert 
\frac{1}{q}\dsum\limits_{r=1}^{q}\left( \frac{1}{\tau _{1}}%
\dsum\limits_{t=\left( r-1\right) \tau +p}^{\left( r-1\right) \tau +\tau
_{1}+p-1}y_{\ell ,t{\LARGE +}1}u_{it}\right) ^{2}\right\vert ^{3}\geq
\epsilon ^{3}\right\} \\
&\leq &P\left\{ \max_{1\leq \ell \leq d}\max_{i\in H^{{\large c}}}\frac{1}{q}%
\dsum\limits_{r=1}^{q}\left( \frac{1}{\tau _{1}}\dsum\limits_{t=\left(
r-1\right) \tau +p}^{\left( r-1\right) \tau +\tau _{1}+p-1}y_{\ell ,t{\LARGE %
+}1}u_{it}\right) ^{6}\geq \epsilon ^{3}\right\} \\
&&\left( \text{by Jensen's inequality}\right) \\
&\leq &P\left\{ \dsum\limits_{\ell =1}^{d}\dsum\limits_{i\in H^{{\large c}}}%
\frac{1}{q}\dsum\limits_{r=1}^{q}\left( \frac{1}{\tau _{1}}%
\dsum\limits_{t=\left( r-1\right) \tau +p}^{\left( r-1\right) \tau +\tau
_{1}+p-1}y_{\ell ,t{\LARGE +}1}u_{it}\right) ^{6}\geq \epsilon ^{3}\right\}
\\
&\leq &\frac{1}{\epsilon ^{3}}\frac{1}{q}\dsum\limits_{r=1}^{q}\dsum%
\limits_{\ell =1}^{d}\dsum\limits_{i\in H^{{\large c}}}E\left( \frac{1}{\tau
_{1}}\dsum\limits_{t=\left( r-1\right) \tau {\LARGE +}p}^{\left( r-1\right)
\tau +\tau _{1}{\LARGE +}p-1}y_{\ell ,t{\LARGE +}1}u_{it}\right) ^{6}\text{. 
}
\end{eqnarray*}%
The rest of the proof for part (d) then follows in a manner similar to the
argument given for part (c) above.

For part (e), note that, by the Cauchy-Schwarz inequality,%
\begin{eqnarray*}
&&\max_{1\leq \ell \leq d}\max_{i\in H^{{\large c}}}\left\vert \frac{1}{q}%
\dsum\limits_{r=1}^{q}\left( \frac{1}{\tau _{1}}\dsum\limits_{t=\left(
r-1\right) \tau +p}^{\left( r-1\right) \tau +\tau _{1}+p-1}\gamma
_{i}^{\prime }\underline{F}_{t}\varepsilon _{\ell ,t{\LARGE +}1}\right)
\left( \frac{1}{\tau _{1}}\dsum\limits_{t=\left( r-1\right) \tau +p}^{\left(
r-1\right) \tau +\tau _{1}+p-1}y_{\ell ,t{\LARGE +}1}u_{it}\right)
\right\vert \\
&\leq &\max_{1\leq \ell \leq d}\max_{i\in H^{{\large c}}}\sqrt{\frac{1}{q}%
\dsum\limits_{r=1}^{q}\left( \frac{1}{\tau _{1}}\dsum\limits_{t=\left(
r-1\right) \tau +p}^{\left( r-1\right) \tau +\tau _{1}+p-1}\gamma
_{i}^{\prime }\underline{F}_{t}\varepsilon _{\ell ,t{\LARGE +}1}\right) ^{2}}%
\sqrt{\frac{1}{q}\dsum\limits_{r=1}^{q}\left( \frac{1}{\tau _{1}}%
\dsum\limits_{t=\left( r-1\right) \tau +p}^{\left( r-1\right) \tau +\tau
_{1}+p-1}y_{\ell ,t{\LARGE +}1}u_{it}\right) ^{2}} \\
&\leq &\left\{ \sqrt{\max_{1\leq \ell \leq d}\max_{i\in H^{{\large c}}}\frac{%
1}{q}\dsum\limits_{r=1}^{q}\left( \frac{1}{\tau _{1}}\dsum\limits_{t=\left(
r-1\right) \tau +p}^{\left( r-1\right) \tau +\tau _{1}+p-1}\gamma
_{i}^{\prime }\underline{F}_{t}\varepsilon _{\ell ,t{\LARGE +}1}\right) ^{2}}%
\right. \\
&&\left. \times \sqrt{\max_{1\leq \ell \leq d}\max_{i\in H^{{\large c}}}%
\frac{1}{q}\dsum\limits_{r=1}^{q}\left( \frac{1}{\tau _{1}}%
\dsum\limits_{t=\left( r-1\right) \tau +p}^{\left( r-1\right) \tau +\tau
_{1}+p-1}y_{\ell ,t{\LARGE +}1}u_{it}\right) ^{2}}\right\} \\
&=&o_{p}\left( 1\right) \text{,}
\end{eqnarray*}%
where the convergence in probability to zero in the last line above follows
from applying the results in parts (b) and (d) of this lemma. $\square $

\medskip

\noindent \textbf{Lemma OA-7:} Suppose that Assumptions 2-1 and 2-6 hold.
Then, the following statements are true.

\begin{enumerate}
\item[(a)] There exists a positive constant $C^{\dagger }$ such that 
\begin{equation*}
\left\Vert A_{YY}\right\Vert _{2}\leq C^{\dagger }\phi _{\max }
\end{equation*}%
where $\phi _{\max }=\max \left\{ \left\vert \lambda _{\max }\left( A\right)
\right\vert ,\left\vert \lambda _{\min }\left( A\right) \right\vert \right\} 
$ with $0<\phi _{\max }<1$.

\item[(b)] There exists a positive constant $C^{\dagger }$ such that 
\begin{equation*}
\left\Vert A_{YF}\right\Vert _{2}\leq C^{\dagger }\phi _{\max }
\end{equation*}%
where $\phi _{\max }$ is as defined in part (a).
\end{enumerate}

\medskip

\noindent \textbf{Proof of Lemma OA-7: }

To proceed, recall first that the FAVAR model, i.e.,%
\begin{eqnarray*}
Y_{t} &=&\mu _{Y}+A_{YY}\underline{Y}_{t-1}+A_{YF}\underline{F}%
_{t-1}+\varepsilon _{t}^{Y} \\
F_{t} &=&\mu _{F}+A_{FY}\underline{Y}_{t-1}+A_{FF}\underline{F}%
_{t-1}+\varepsilon _{t}^{F}\text{,}
\end{eqnarray*}%
can be written in the companion form%
\begin{equation*}
\underline{W}_{t}=\alpha +A\underline{W}_{t-1}+E_{t}
\end{equation*}%
where $\underline{W}_{t}=\left( 
\begin{array}{ccccc}
W_{t}^{\prime } & W_{t-1}^{\prime } & \cdots & W_{t-p{\LARGE +}2}^{\prime }
& W_{t-p{\LARGE +}1}^{\prime }%
\end{array}%
\right) ^{\prime }$ with $W_{t}=\left( 
\begin{array}{cc}
Y_{t}^{\prime } & F_{t}^{\prime }%
\end{array}%
\right) ^{\prime }$ and where 
\begin{equation*}
\alpha =\left( 
\begin{array}{c}
\mu \\ 
0 \\ 
\vdots \\ 
0 \\ 
0%
\end{array}%
\right) \text{, }A=\left( 
\begin{array}{ccccc}
A_{1} & A_{2} & \cdots & A_{p-1} & A_{p} \\ 
I_{d+K} & 0 & \cdots & 0 & 0 \\ 
0 & I_{d+K} & \ddots & \vdots & 0 \\ 
\vdots & \ddots & \ddots & 0 & \vdots \\ 
0 & \cdots & 0 & I_{d+K} & 0%
\end{array}%
\right) \text{, and }E_{t}=\left( 
\begin{array}{c}
\varepsilon _{t} \\ 
0 \\ 
\vdots \\ 
0 \\ 
0%
\end{array}%
\right)
\end{equation*}%
with $\mu =\left( 
\begin{array}{cc}
\mu _{Y}^{\prime } & \mu _{F}^{\prime }%
\end{array}%
\right) ^{\prime }$, $\varepsilon _{t}=\left( 
\begin{array}{cc}
\varepsilon _{t}^{Y\prime } & \varepsilon _{t}^{F\prime }%
\end{array}%
\right) ^{\prime }$, and%
\begin{equation*}
A_{\ell }=\left( 
\begin{array}{cc}
A_{YY,\ell } & A_{YF,\ell } \\ 
A_{FY,\ell } & A_{FF,\ell }%
\end{array}%
\right) \text{ for }\ell =1,...,p\text{.}
\end{equation*}%
Let $\mathcal{P}_{\left( d{\LARGE +}K\right) p}$ be the $\left( d{\LARGE +}%
K\right) p\times \left( d{\LARGE +}K\right) p$ permutation matrix defined by
expression (\ref{definition permutation matrix}) in the proof of Lemma OA-5;
and it is easy to see that $\overline{A}=\mathcal{P}_{\left( d{\LARGE +}%
K\right) p}A\mathcal{P}_{\left( d{\LARGE +}K\right) p}^{\prime }$ has the
partitioned form 
\begin{equation*}
\overline{A}=\mathcal{P}_{\left( d{\LARGE +}K\right) p}A\mathcal{P}_{\left( d%
{\LARGE +}K\right) p}^{\prime }=\left( 
\begin{array}{cc}
\underset{d\times dp}{\overline{A}_{11}} & \underset{d\times Kp}{\overline{A}%
_{12}} \\ 
\underset{d\left( p-1\right) \times dp}{\overline{A}_{21}} & \underset{%
d\left( p-1\right) \times Kp}{\overline{A}_{22}} \\ 
\underset{K\times dp}{\overline{A}_{31}} & \underset{K\times Kp}{\overline{A}%
_{32}} \\ 
\underset{K\left( p-1\right) \times dp}{\overline{A}_{41}} & \underset{%
K\left( p-1\right) \times Kp}{\overline{A}_{42}}%
\end{array}%
\right)
\end{equation*}%
where $\overline{A}_{11}=A_{YY}$ and $\overline{A}_{12}=A_{YF}$, i.e., the
first $d$ rows of the matrix $\overline{A}$ as given by the submatrix $\left[
\begin{array}{cc}
A_{YY} & A_{YF}%
\end{array}%
\right] $.

Now, to show part (a), let $\overline{\upsilon }\in \mathbb{R}^{dp}$ such
that $\left\Vert \overline{\upsilon }\right\Vert _{2}=1$ and such that%
\begin{equation*}
\left\Vert A_{YY}\right\Vert _{2}=\overline{\upsilon }^{\prime
}A_{YY}^{\prime }A_{YY}\overline{\upsilon }=\max_{\left\Vert \upsilon
\right\Vert _{2}=1}\upsilon ^{\prime }A_{YY}^{\prime }A_{YY}\upsilon =%
\overline{\upsilon }^{\prime }\overline{A}_{11}^{\prime }\overline{A}_{11}%
\overline{\upsilon }
\end{equation*}%
and let $S_{d}=\left( 
\begin{array}{cc}
I_{dp} & \underset{dp\times Kp}{0}%
\end{array}%
\right) ^{\prime }$. It follows that%
\begin{eqnarray*}
\left\Vert A_{YY}\right\Vert _{2} &=&\sqrt{\overline{\upsilon }^{\prime
}A_{YY}^{\prime }A_{YY}\overline{\upsilon }} \\
&=&\sqrt{\overline{\upsilon }^{\prime }\overline{A}_{11}^{\prime }\overline{A%
}_{11}\overline{\upsilon }} \\
&\leq &\sqrt{\overline{\upsilon }^{\prime }\overline{A}_{11}^{\prime }%
\overline{A}_{11}\overline{\upsilon }+\overline{\upsilon }^{\prime }%
\overline{A}_{21}^{\prime }\overline{A}_{21}\overline{\upsilon }+\overline{%
\upsilon }^{\prime }\overline{A}_{31}^{\prime }\overline{A}_{31}\overline{%
\upsilon }+\overline{\upsilon }^{\prime }\overline{A}_{41}^{\prime }%
\overline{A}_{41}\overline{\upsilon }} \\
&=&\sqrt{\overline{\upsilon }^{\prime }S_{d}^{\prime }\overline{A}^{\prime }%
\overline{A}S_{d}\overline{\upsilon }} \\
&=&\sqrt{\overline{\upsilon }^{\prime }S_{d}^{\prime }\mathcal{P}_{\left( d%
{\LARGE +}K\right) p}A^{\prime }\mathcal{P}_{\left( d{\LARGE +}K\right)
p}^{\prime }\mathcal{P}_{\left( d{\LARGE +}K\right) p}A\mathcal{P}_{\left( d%
{\LARGE +}K\right) p}^{\prime }S_{d}\overline{\upsilon }} \\
&=&\sqrt{\overline{\upsilon }^{\prime }S_{d}^{\prime }\mathcal{P}_{\left( d%
{\LARGE +}K\right) p}A^{\prime }A\mathcal{P}_{\left( d{\LARGE +}K\right)
p}^{\prime }S_{d}\overline{\upsilon }}\text{ }\left( \text{since }\mathcal{P}%
_{\left( d{\LARGE +}K\right) p}\text{ is an orthogonal matrix}\right) \\
&\leq &\sqrt{\max_{\left\Vert \upsilon \right\Vert _{2}=1}\upsilon ^{\prime
}A^{\prime }A\upsilon }\text{ }\left( \text{noting that }\left\Vert \mathcal{%
P}_{\left( d{\LARGE +}K\right) p}^{\prime }S_{d}\overline{\upsilon }%
\right\Vert _{2}=\sqrt{\overline{\upsilon }^{\prime }S_{d}^{\prime }\mathcal{%
P}_{\left( d{\LARGE +}K\right) p}\mathcal{P}_{\left( d{\LARGE +}K\right)
p}^{\prime }S_{d}\overline{\upsilon }}=1\right) \\
&=&\left\Vert A\right\Vert _{2} \\
&=&\sigma _{\max }\left( A\right) \\
&\leq &C^{\dagger }\phi _{\max }\text{ }\left( \text{by Assumption 2-6}%
\right)
\end{eqnarray*}%
where $\phi _{\max }=\max \left\{ \left\vert \lambda _{\max }\left( A\right)
\right\vert ,\left\vert \lambda _{\min }\left( A\right) \right\vert \right\} 
$. Note further that $0<\phi _{\max }<1$ since, by Assumption 2-1, all
eigenvalues of $A$ have modulus less than $1$.

To show part (b), let $\widetilde{\upsilon }\in \mathbb{R}^{Kp}$ such that $%
\left\Vert \widetilde{\upsilon }\right\Vert _{2}=1$ and such that%
\begin{equation*}
\left\Vert A_{YF}\right\Vert _{2}=\widetilde{\upsilon }^{\prime
}A_{YF}^{\prime }A_{YF}\widetilde{\upsilon }=\max_{\left\Vert \upsilon
\right\Vert _{2}=1}\upsilon ^{\prime }A_{YF}^{\prime }A_{YF}\upsilon =%
\widetilde{\upsilon }^{\prime }\overline{A}_{12}^{\prime }\overline{A}_{12}%
\widetilde{\upsilon }
\end{equation*}%
and let%
\begin{equation*}
\underset{\left( d+K\right) p\times Kp}{S_{K}}=\left( 
\begin{array}{c}
0 \\ 
I_{Kp}%
\end{array}%
\right) \text{. }
\end{equation*}%
It follows that%
\begin{eqnarray*}
\left\Vert A_{YF}\right\Vert _{2} &=&\sqrt{\widetilde{\upsilon }^{\prime
}A_{YF}^{\prime }A_{YF}\widetilde{\upsilon }} \\
&=&\sqrt{\widetilde{\upsilon }^{\prime }\overline{A}_{12}^{\prime }\overline{%
A}_{12}\widetilde{\upsilon }} \\
&\leq &\sqrt{\widetilde{\upsilon }^{\prime }\overline{A}_{12}^{\prime }%
\overline{A}_{12}\widetilde{\upsilon }+\widetilde{\upsilon }^{\prime }%
\overline{A}_{22}^{\prime }\overline{A}_{22}\widetilde{\upsilon }+\widetilde{%
\upsilon }^{\prime }\overline{A}_{32}^{\prime }\overline{A}_{32}\widetilde{%
\upsilon }+\widetilde{\upsilon }^{\prime }\overline{A}_{42}^{\prime }%
\overline{A}_{42}\widetilde{\upsilon }} \\
&=&\sqrt{\widetilde{\upsilon }^{\prime }S_{K}^{\prime }\overline{A}^{\prime }%
\overline{A}S_{K}\widetilde{\upsilon }} \\
&=&\sqrt{\widetilde{\upsilon }^{\prime }S_{K}^{\prime }\mathcal{P}_{\left( d%
{\LARGE +}K\right) p}A^{\prime }\mathcal{P}_{\left( d{\LARGE +}K\right)
p}^{\prime }\mathcal{P}_{\left( d{\LARGE +}K\right) p}A\mathcal{P}_{\left( d%
{\LARGE +}K\right) p}^{\prime }S_{K}\widetilde{\upsilon }} \\
&=&\sqrt{\widetilde{\upsilon }^{\prime }S_{K}^{\prime }\mathcal{P}_{\left( d%
{\LARGE +}K\right) p}A^{\prime }A\mathcal{P}_{\left( d{\LARGE +}K\right)
p}^{\prime }S_{K}\widetilde{\upsilon }}\text{ }\left( \text{since }\mathcal{P%
}_{\left( d{\LARGE +}K\right) p}\text{ is an orthogonal matrix}\right) \\
&\leq &\sqrt{\max_{\left\Vert \upsilon \right\Vert _{2}=1}\upsilon ^{\prime
}A^{\prime }A\upsilon }\text{ }\left( \text{noting that }\left\Vert \mathcal{%
P}_{\left( d{\LARGE +}K\right) p}^{\prime }S_{K}\widetilde{\upsilon }%
\right\Vert _{2}=\sqrt{\widetilde{\upsilon }^{\prime }S_{K}^{\prime }%
\mathcal{P}_{\left( d{\LARGE +}K\right) p}\mathcal{P}_{\left( d{\LARGE +}%
K\right) p}^{\prime }S_{K}\widetilde{\upsilon }}=1\right) \\
&=&\left\Vert A\right\Vert _{2} \\
&=&\sigma _{\max }\left( A\right) \\
&\leq &C^{\dagger }\phi _{\max }\text{ }\left( \text{by Assumption 2-6}%
\right)
\end{eqnarray*}%
where $\phi _{\max }=\max \left\{ \left\vert \lambda _{\max }\left( A\right)
\right\vert ,\left\vert \lambda _{\min }\left( A\right) \right\vert \right\} 
$. As noted in the proof for part (a), $0<\phi _{\max }<1$ since, by
Assumption 2-1, all eigenvalues of $A$ have modulus less than $1$. $\square $

\medskip

\noindent \textbf{Lemma OA-8: }Consider the linear process%
\begin{equation*}
\xi _{t}=\dsum\limits_{j=0}^{\infty }\Psi _{j}\varepsilon _{t-j}
\end{equation*}%
Suppose the process satisfies the following assumptions

\begin{enumerate}
\item[(i)] Let $\left\{ \varepsilon _{t}\right\} $ is an independent
sequence of random vectors with $E\left[ \varepsilon _{t}\right] =0$ for all 
$t$. For some $\delta >0$, suppose that there exists a positive constant $K$
such that 
\begin{equation*}
E\left\Vert \varepsilon _{t}\right\Vert _{2}^{1+\delta }\leq K<\infty \text{
for all }t.
\end{equation*}

\item[(ii)] Suppose that $\varepsilon _{t}$ has p.d.f. $g_{\varepsilon _{t}}$
such that, for some positive constant $M<\infty $,%
\begin{equation*}
\sup_{t}\dint \left\vert g_{\varepsilon _{t}}\left( \upsilon -u\right)
-g_{\varepsilon _{t}}\left( \upsilon \right) \right\vert d\varepsilon \leq
M\left\vert u\right\vert
\end{equation*}%
whenever $\left\vert u\right\vert \leq \overline{\kappa }$ for some constant 
$\overline{\kappa }>0$.

\item[(iii)] Suppose that%
\begin{equation*}
\dsum\limits_{j=0}^{\infty }\left\Vert \Psi _{j}\right\Vert _{2}<\infty
\end{equation*}%
and%
\begin{equation*}
\det \left\{ \dsum\limits_{j=0}^{\infty }\Psi _{j}z^{j}\right\} \neq 0\text{
for all }z\text{ with }\left\vert z\right\vert \leq 1
\end{equation*}
\end{enumerate}

Under these conditions, suppose further that%
\begin{equation*}
\dsum\limits_{j=0}^{\infty }\left( \dsum\limits_{k=j}^{\infty }\left\Vert
\Psi _{j}\right\Vert _{2}\right) ^{\frac{{\LARGE \delta }}{{\LARGE 1+\delta }%
}}<\infty \text{;}
\end{equation*}%
then, for some positive constant $\overline{K}$, 
\begin{equation*}
\beta _{\xi }\left( m\right) \leq \overline{K}\dsum\limits_{j=m}^{\infty
}\left( \dsum\limits_{k=j}^{\infty }\left\Vert \Psi _{k}\right\Vert
_{2}\right) ^{\frac{{\LARGE \delta }}{{\LARGE 1+\delta }}}
\end{equation*}%
where%
\begin{equation*}
\beta _{\xi }\left( m\right) =\sup_{t}E\left[ \sup \left\{ \left\vert
P\left( B|\mathcal{F}_{\xi ,-\infty }^{t}\right) -P\left( B\right)
\right\vert :B\in \mathcal{F}_{\xi ,t+m}^{\infty }\right\} \right] \text{.}
\end{equation*}%
with $\mathcal{F}_{\xi ,-\infty }^{t}=\sigma \left( ....,\xi _{t-2},\xi
_{t-1},\xi _{t}\right) $ and $\mathcal{F}_{\xi ,t+m}^{\infty }=\sigma \left(
\xi _{t+m},\xi _{t+m+1},\xi _{t+m+2},....\right) $.

\noindent \textbf{Remark: }This is Theorem 2.1 of Pham and Tran (1985)
restated here in our notation. For a proof, see Pham and Tran (1985).

\medskip

\noindent \textbf{Lemma OA-9: }Let $A$ be an $n\times n$ square matrix with
(ordered) singular values given by%
\begin{equation*}
\sigma _{\left( 1\right) }\left( A\right) \geq \sigma _{\left( 2\right)
}\left( A\right) \geq \cdot \cdot \cdot \geq \sigma _{\left( n\right)
}\left( A\right) \geq 0\text{.}
\end{equation*}%
Suppose that $A$ is diagonalizable, i.e.,%
\begin{equation*}
A=S\Lambda S^{-1}
\end{equation*}%
where $\Lambda $ is diagonal matrix whose diagonal elements are the
eigenvalues of $A$. Let the modulus of these eigenvalues be ordered as
follows: 
\begin{equation*}
\left\vert \lambda _{\left( 1\right) }\left( A\right) \right\vert \geq
\left\vert \lambda _{\left( 2\right) }\left( A\right) \right\vert \geq \cdot
\cdot \cdot \geq \left\vert \lambda _{\left( n\right) }\left( A\right)
\right\vert \text{.}
\end{equation*}%
Then, for $k\in \left\{ 1,...,n\right\} $ and for any positive integer $j$,
we have%
\begin{equation*}
\chi \left( S\right) ^{-1}\left\vert \lambda _{\left( k\right) }\left(
A^{j}\right) \right\vert \leq \sigma _{\left( k\right) }\left( A^{j}\right)
\leq \chi \left( S\right) \left\vert \lambda _{\left( k\right) }\left(
A^{j}\right) \right\vert \text{ }
\end{equation*}%
where%
\begin{equation*}
\chi \left( S\right) =\sigma _{\left( 1\right) }\left( S\right) \sigma
_{\left( 1\right) }\left( S^{-1}\right) \text{.}
\end{equation*}%
\textbf{Proof of Lemma OA-9: }Observe first that we can assume, without loss
of generality, that the decomposition%
\begin{equation*}
A=S\Lambda S^{-1}=S\cdot diag\left( \lambda _{1},\lambda _{2},...,\lambda
_{n}\right) \cdot S^{-1}
\end{equation*}%
is such that 
\begin{equation*}
\lambda _{i}=\lambda _{\left( i\right) }\left( A\right) \text{ for }i=1,...,n
\end{equation*}%
with%
\begin{equation*}
\left\vert \lambda _{\left( 1\right) }\left( A\right) \right\vert \geq
\left\vert \lambda _{\left( 2\right) }\left( A\right) \right\vert \geq \cdot
\cdot \cdot \geq \left\vert \lambda _{\left( n\right) }\left( A\right)
\right\vert \text{. }
\end{equation*}%
This is because suppose we have the alternative representation where 
\begin{equation*}
A=\widetilde{S}\widetilde{\Lambda }\widetilde{S}^{-1}=\widetilde{S}\cdot
diag\left( \widetilde{\lambda }_{1},\widetilde{\lambda }_{2},...,\widetilde{%
\lambda }_{n}\right) \cdot \widetilde{S}^{-1}
\end{equation*}%
and where $\widetilde{\lambda }_{i}\neq \lambda _{\left( i\right) }\left(
A\right) $ for at least some of the $i^{\prime }s$. Then, we can always
define a permutation matrix $\mathcal{P}$ such that%
\begin{equation*}
\mathcal{P}^{\prime }\widetilde{\Lambda }\mathcal{P}=\Lambda
\end{equation*}%
so that, given that $\mathcal{P}$ is an orthogonal matrix, we have 
\begin{equation*}
A=\widetilde{S}\widetilde{\Lambda }\widetilde{S}^{-1}=\widetilde{S}\mathcal{%
PP}^{\prime }\widetilde{\Lambda }\mathcal{PP}^{\prime }\widetilde{S}%
^{-1}=S\Lambda S^{-1}
\end{equation*}%
where $S=\widetilde{S}\mathcal{P}$ and, thus, $S^{-1}=\left( \widetilde{S}%
\mathcal{P}\right) ^{-1}=\mathcal{P}^{\prime }\widetilde{S}^{-1}$.

Next, note that, for any positive integer $j$,%
\begin{equation*}
A^{j}=S\Lambda S^{-1}\times S\Lambda S^{-1}\times \cdot \cdot \cdot \times
S\Lambda S^{-1}=S\Lambda ^{j}S^{-1}
\end{equation*}%
where 
\begin{equation*}
\Lambda ^{j}=diag\left( \lambda _{1}^{j},\lambda _{2}^{j},...,\lambda
_{n}^{j}\right) =diag\left( \lambda _{\left( 1\right) }^{j}\left( A\right)
,\lambda _{\left( 2\right) }^{j}\left( A\right) ,...,\lambda _{\left(
n\right) }^{j}\left( A\right) \right) \text{.}
\end{equation*}%
Moreover, since $\lambda _{\left( k\right) }\left( A^{j}\right) =\lambda
_{\left( k\right) }^{j}\left( A\right) $ for any $k\in \left\{
1,...,m\right\} $, we also have%
\begin{equation*}
\Lambda ^{j}=diag\left( \lambda _{1}^{j},\lambda _{2}^{j},...,\lambda
_{n}^{j}\right) =diag\left( \lambda _{\left( 1\right) }\left( A^{j}\right)
,\lambda _{\left( 2\right) }\left( A^{j}\right) ,...,\lambda _{\left(
n\right) }\left( A^{j}\right) \right) \text{.}
\end{equation*}%
In addition, let $\overline{\lambda _{\left( k\right) }\left( A^{j}\right) }$
denote the complex conjugate of $\lambda _{\left( k\right) }\left(
A^{j}\right) $ for $k\in \left\{ 1,...,m\right\} $, and note that, by
definition, 
\begin{equation*}
\sigma _{\left( k\right) }\left( \Lambda ^{j}\right) =\sqrt{\overline{%
\lambda _{\left( k\right) }\left( A^{j}\right) }\lambda _{\left( k\right)
}\left( A^{j}\right) }=\left\vert \lambda _{\left( k\right) }\left(
A^{j}\right) \right\vert
\end{equation*}%
Since $\left\vert \lambda _{\left( k\right) }\left( A^{j}\right) \right\vert
=\left\vert \lambda _{\left( k\right) }^{j}\left( A\right) \right\vert
=\left\vert \lambda _{\left( k\right) }\left( A\right) \right\vert ^{j}$,
the ordering%
\begin{equation*}
\left\vert \lambda _{\left( 1\right) }\left( A\right) \right\vert \geq
\left\vert \lambda _{\left( 2\right) }\left( A\right) \right\vert \geq \cdot
\cdot \cdot \geq \left\vert \lambda _{\left( n\right) }\left( A\right)
\right\vert
\end{equation*}%
implies that%
\begin{equation*}
\left\vert \lambda _{\left( 1\right) }\left( A^{j}\right) \right\vert \geq
\left\vert \lambda _{\left( 2\right) }\left( A^{j}\right) \right\vert \geq
\cdot \cdot \cdot \geq \left\vert \lambda _{\left( n\right) }\left(
A^{j}\right) \right\vert
\end{equation*}%
and, thus, 
\begin{equation*}
\sigma _{\left( 1\right) }\left( \Lambda ^{j}\right) \geq \sigma _{\left(
2\right) }\left( \Lambda ^{j}\right) \geq \cdot \cdot \cdot \geq \sigma
_{\left( n\right) }\left( \Lambda ^{j}\right)
\end{equation*}%
for any positive integer $j$.

Now, apply the inequality%
\begin{equation*}
\sigma _{\left( i+\ell -1\right) }\left( BC\right) \leq \sigma _{\left(
i\right) }\left( B\right) \sigma _{\left( \ell \right) }\left( C\right)
\end{equation*}%
for $i,\ell \in \left\{ 1,...,n\right\} $ and $i+\ell \leq n+1$; we have%
\begin{eqnarray*}
\sigma _{\left( k\right) }\left( A^{j}\right) &=&\sigma _{\left( k\right)
}\left( S\Lambda ^{j}S^{-1}\right) \\
&\leq &\sigma _{\left( k\right) }\left( S\Lambda ^{j}\right) \sigma _{\left(
1\right) }\left( S^{-1}\right) \\
&\leq &\sigma _{\left( k\right) }\left( \Lambda ^{j}\right) \sigma _{\left(
1\right) }\left( S\right) \sigma _{\left( 1\right) }\left( S^{-1}\right) \\
&=&\sigma _{\left( 1\right) }\left( S\right) \sigma _{\left( 1\right)
}\left( S^{-1}\right) \left\vert \lambda _{\left( k\right) }\left(
A^{j}\right) \right\vert \\
&=&\chi \left( S\right) \left\vert \lambda _{\left( k\right) }\left(
A^{j}\right) \right\vert \text{ for any }k\in \left\{ 1,...,n\right\}
\end{eqnarray*}%
Moreover, for any $k\in \left\{ 1,...,n\right\} $, 
\begin{eqnarray*}
\left\vert \lambda _{\left( k\right) }\left( A^{j}\right) \right\vert
&=&\sigma _{\left( k\right) }\left( \Lambda ^{j}\right) \\
&=&\sigma _{\left( k\right) }\left( S^{-1}S\Lambda ^{j}S^{-1}S\right) \\
&=&\sigma _{\left( k\right) }\left( S^{-1}A^{j}S\right) \\
&\leq &\sigma _{\left( 1\right) }\left( S^{-1}\right) \sigma _{\left(
k\right) }\left( A^{j}\right) \sigma _{\left( 1\right) }\left( S\right)
\end{eqnarray*}%
or%
\begin{equation*}
\frac{\left\vert \lambda _{\left( k\right) }\left( A^{j}\right) \right\vert 
}{\chi \left( S\right) }=\frac{\left\vert \lambda _{\left( k\right) }\left(
A^{j}\right) \right\vert }{\sigma _{\left( 1\right) }\left( S\right) \sigma
_{\left( 1\right) }\left( S^{-1}\right) }\leq \sigma _{\left( k\right)
}\left( A^{j}\right)
\end{equation*}%
Putting these two inequalities together, we have, for any $k\in \left\{
1,...,n\right\} $ and for all positive integer $j$, 
\begin{equation*}
\chi \left( S\right) ^{-1}\left\vert \lambda _{\left( k\right) }\left(
A^{j}\right) \right\vert \leq \sigma _{\left( k\right) }\left( A^{j}\right)
\leq \chi \left( S\right) \left\vert \lambda _{\left( k\right) }\left(
A^{j}\right) \right\vert \text{. }\square
\end{equation*}

\noindent \textbf{Remark: }Note that the case where $j=1$ in Lemma OA-9 has
previously been obtained in Theorem 1 of Ruhe (1975). Hence, Lemma C-9 can
be viewed as providing an extension to the first part of that theorem.

\medskip

\noindent \textbf{Lemma OA-10: }Let $\rho $ be such that $\left\vert \rho
\right\vert <1$. Then,%
\begin{equation*}
\dsum\limits_{j=0}^{\infty }\left( j+1\right) \rho ^{j}=\frac{1}{\left(
1-\rho \right) ^{2}}<\infty
\end{equation*}

\noindent \textbf{Proof of Lemma OA-10:} Define%
\begin{equation*}
S_{n}\left( \rho \right) =1+\rho +\rho ^{2}+\cdot \cdot \cdot +\rho ^{n}=%
\frac{1-\rho ^{n{\LARGE +}1}}{1-\rho }
\end{equation*}%
Note that%
\begin{eqnarray*}
S_{n}^{\prime }\left( \rho \right) &=&1+2\rho +3\rho ^{2}+\cdot \cdot \cdot
+n\rho ^{n-1} \\
&=&-\frac{\left( n+1\right) \rho ^{n}}{1-\rho }+\frac{1-\rho ^{n{\LARGE +}1}%
}{\left( 1-\rho \right) ^{2}} \\
&=&\frac{1-\rho ^{n{\LARGE +}1}-\left( n+1\right) \rho ^{n}\left( 1-\rho
\right) }{\left( 1-\rho \right) ^{2}} \\
&=&\frac{1-\rho ^{n{\LARGE +}1}-\left( n+1\right) \rho ^{n}+\left(
n+1\right) \rho ^{n{\LARGE +}1}}{\left( 1-\rho \right) ^{2}} \\
&=&\frac{1-\left( n+1\right) \rho ^{n}+n\rho ^{n{\LARGE +}1}}{\left( 1-\rho
\right) ^{2}} \\
&=&\frac{1-\rho ^{n}-n\rho ^{n}\left( 1-\rho \right) }{\left( 1-\rho \right)
^{2}}
\end{eqnarray*}%
It follows that%
\begin{equation*}
S_{n}^{\prime }\left( \rho \right) =\dsum\limits_{j=0}^{n-1}\left(
j+1\right) \rho ^{j}=\frac{1-\rho ^{n}-n\rho ^{n}\left( 1-\rho \right) }{%
\left( 1-\rho \right) ^{2}}\rightarrow \frac{1}{\left( 1-\rho \right) ^{2}}%
\text{ as }n\rightarrow \infty \text{. }\square
\end{equation*}

\medskip

\noindent \textbf{Lemma OA-11: }Let $W_{t}=\left( Y_{t}^{\prime
},F_{t}^{\prime }\right) ^{\prime }$ be generated by the factor-augmented
VAR process%
\begin{equation*}
W_{t+1}=\mu +A_{1}W_{t}+\cdot \cdot \cdot +A_{p}W_{t-p+1}+\varepsilon _{t+1}
\end{equation*}%
described in section 3 of the main paper. Under Assumptions 2-1, 2-2, and
2-6; $\left\{ W_{t}\right\} $ is a $\beta $-mixing process with $\beta $%
-mixing coefficient $\beta _{W}\left( m\right) $ such that%
\begin{equation*}
\beta _{W}\left( m\right) \leq C_{1}\exp \left\{ -C_{2}m\right\}
\end{equation*}%
for some positive constants $C_{1}$ and $C_{2}$. Here, 
\begin{equation*}
\beta _{W}\left( m\right) =\sup_{t}E\left[ \sup \left\{ \left\vert P\left( B|%
\mathcal{A}_{-\infty }^{t}\right) -P\left( B\right) \right\vert :B\in 
\mathcal{A}_{t+m}^{\infty }\right\} \right]
\end{equation*}%
with $\mathcal{A}_{-\infty }^{t}=\sigma \left(
...,W_{t-2},W_{t-1},W_{t}\right) $ and $\mathcal{A}_{t+m}^{\infty }=\sigma
\left( W_{t+m},W_{t+m+1},W_{t+m+2},....\right) $.

\medskip

\noindent \textbf{Proof of Lemma OA-11:}

To prove this lemma, we shall verify the conditions of Lemma OA-8 given
above for the vector moving-average representation of $W_{t}$, i.e., 
\begin{equation*}
W_{t}=J_{d{\LARGE +}K}\left( I_{\left( d{\LARGE +}K\right) p}-A\right)
^{-1}J_{d{\LARGE +}K}^{\prime }\mu +\dsum\limits_{j=0}^{\infty }J_{d{\LARGE +%
}K}A^{j}J_{d{\LARGE +}K}^{\prime }\varepsilon _{t-j}=\mu _{\ast
}+\dsum\limits_{j=0}^{\infty }\Psi _{j}\varepsilon _{t-j}\text{,}
\end{equation*}%
where%
\begin{eqnarray*}
\text{ }\mu _{\ast } &=&J_{d{\LARGE +}K}\left( I_{\left( d{\LARGE +}K\right)
p}-A\right) ^{-1}J_{d{\LARGE +}K}^{\prime }\mu \text{, }\Psi _{j}=J_{d%
{\LARGE +}K}A^{j}J_{d{\LARGE +}K}^{\prime }\text{, } \\
\underset{\left( d{\LARGE +}K\right) \times \left( d{\LARGE +}K\right) p}{%
J_{d{\LARGE +}K}} &=&\left[ 
\begin{array}{ccccc}
I_{d{\LARGE +}K} & 0 & \cdots & 0 & 0%
\end{array}%
\right] \text{, and }A=\left( 
\begin{array}{ccccc}
A_{1} & A_{2} & \cdots & A_{p-1} & A_{p} \\ 
I_{d{\LARGE +}K} & 0 & \cdots & \cdots & 0 \\ 
0 & \ddots & \ddots &  & \vdots \\ 
\vdots & \ddots & \ddots & \ddots & \vdots \\ 
0 & \cdots & 0 & I_{d{\LARGE +}K} & 0%
\end{array}%
\right)
\end{eqnarray*}%
To proceed, set%
\begin{equation}
\xi _{t}=\dsum\limits_{j=0}^{\infty }\Psi _{j}\varepsilon _{t-j}  \label{xi}
\end{equation}%
and note first that, setting $\delta =5$ in Lemma OA-8, and we see that
Assumptions (i) and (ii) of Lemma OA-8 are the same as the conditions
specified in Assumption 2-2 (a)-(c). Next, note that, since in this case $%
\Psi _{j}=J_{d{\LARGE +}K}A^{j}J_{d{\LARGE +}K}^{\prime }$, we have 
\begin{eqnarray*}
\left\Vert \Psi _{j}\right\Vert _{2} &\leq &\left\Vert J_{d{\LARGE +}%
K}\right\Vert _{2}\left\Vert A^{j}\right\Vert _{2}\left\Vert J_{d{\LARGE +}%
K}^{\prime }\right\Vert _{2} \\
&\leq &\sqrt{\lambda _{\max }\left( J_{d{\LARGE +}K}^{\prime }J_{d{\LARGE +}%
K}\right) }\left( \sqrt{\lambda _{\max }\left\{ \left( A^{j}\right) ^{\prime
}A^{j}\right\} }\right) \sqrt{\lambda _{\max }\left( J_{d{\LARGE +}K}J_{d%
{\LARGE +}K}^{\prime }\right) } \\
&=&\lambda _{\max }\left( J_{d{\LARGE +}K}J_{d{\LARGE +}K}^{\prime }\right)
\left( \sqrt{\lambda _{\max }\left\{ \left( A^{j}\right) ^{\prime
}A^{j}\right\} }\right) \\
&=&\sqrt{\lambda _{\max }\left\{ \left( A^{j}\right) ^{\prime }A^{j}\right\} 
} \\
&=&\sigma _{\max }\left( A^{j}\right) \\
&\leq &C\left[ \max \left\{ \left\vert \lambda _{\max }\left( A^{j}\right)
\right\vert ,\left\vert \lambda _{\min }\left( A^{j}\right) \right\vert
\right\} \right] \text{ \ }\left( \text{by Assumption 2-6}\right) \\
&=&C\left[ \max \left\{ \left\vert \lambda _{\max }\left( A\right)
\right\vert ,\left\vert \lambda _{\min }\left( A\right) \right\vert \right\} %
\right] ^{j} \\
&=&C\phi _{\max }^{j}
\end{eqnarray*}%
where $\phi _{\max }=\max \left\{ \left\vert \lambda _{\max }\left( A\right)
\right\vert ,\left\vert \lambda _{\min }\left( A\right) \right\vert \right\} 
$ and where $0<\phi _{\max }<1$ since, by Assumption 2-1, all eigenvalues of 
$A$ have modulus less than $1$. It follows that%
\begin{equation*}
\dsum\limits_{j=0}^{\infty }\left\Vert \Psi _{j}\right\Vert _{2}\leq
C\dsum\limits_{j=0}^{\infty }\phi _{\max }^{j}=\frac{C}{1-\phi _{\max }}%
<\infty \text{.}
\end{equation*}%
Moreover, by Assumption 2-1, 
\begin{equation*}
\det \left\{ I_{\left( d+K\right) p}-A_{1}z-\cdot \cdot \cdot
-A_{p}z^{p}\right\} \neq 0\text{ for all }z\text{ such that }\left\vert
z\right\vert \leq 1
\end{equation*}%
and, by definition,%
\begin{equation*}
\dsum\limits_{j=0}^{\infty }\Psi _{j}z^{j}=\Psi \left( z\right) =\left(
I_{\left( d+K\right) p}-A_{1}z-\cdot \cdot \cdot -A_{p}z^{p}\right) ^{-1}%
\text{ for all }z\text{ such that }\left\vert z\right\vert \leq 1
\end{equation*}%
so that%
\begin{equation*}
\Psi \left( z\right) \left( I_{\left( d+K\right) p}-A_{1}z-\cdot \cdot \cdot
-A_{p}z^{p}\right) =I_{\left( d+K\right) p}\text{ for all }z\text{ such that 
}\left\vert z\right\vert \leq 1
\end{equation*}%
In addition, since%
\begin{eqnarray*}
&&\det \left\{ \Psi \left( z\right) \right\} \det \left\{ I_{\left(
d+K\right) p}-A_{1}z-\cdot \cdot \cdot -A_{p}z^{p}\right\} \\
&=&\det \left\{ \Psi \left( z\right) \left( I_{\left( d+K\right)
p}-A_{1}z-\cdot \cdot \cdot -A_{p}z^{p}\right) \right\} \\
&=&\det \left\{ I_{\left( d+K\right) p}\right\} \\
&=&1\text{,}
\end{eqnarray*}%
and since 
\begin{equation*}
\left\vert \det \left\{ I_{\left( d+K\right) p}-A_{1}z-\cdot \cdot \cdot
-A_{p}z^{p}\right\} \right\vert <\infty \text{ for all }z\text{ such that }%
\left\vert z\right\vert \leq 1\text{,}
\end{equation*}%
it follows that%
\begin{eqnarray*}
\det \left\{ \dsum\limits_{j=0}^{\infty }\Psi _{j}z^{j}\right\} &=&\det
\left\{ \Psi \left( z\right) \right\} \\
&=&\frac{1}{\det \left\{ I_{\left( d+K\right) p}-A_{1}z-\cdot \cdot \cdot
-A_{p}z^{p}\right\} } \\
&\neq &0\text{ for all }z\text{ such that }\left\vert z\right\vert \leq 1%
\text{.}
\end{eqnarray*}

Finally, note that, setting $\delta =5$,%
\begin{eqnarray*}
\dsum\limits_{j=0}^{\infty }\left( \dsum\limits_{k=j}^{\infty }\left\Vert
\Psi _{k}\right\Vert _{2}\right) ^{\frac{{\LARGE \delta }}{{\LARGE 1+\delta }%
}} &=&\dsum\limits_{j=0}^{\infty }\left( \dsum\limits_{k=j}^{\infty
}\left\Vert \Psi _{k}\right\Vert _{2}\right) ^{\frac{{\large 5}}{{\large 6}}}
\\
&\leq &\dsum\limits_{j=0}^{\infty }\left( \dsum\limits_{k=j}^{\infty }C\phi
_{\max }^{k}\right) ^{\frac{{\large 5}}{{\large 6}}} \\
&=&C^{\frac{{\large 5}}{{\large 6}}}\dsum\limits_{j=0}^{\infty }\left(
\dsum\limits_{k=j}^{\infty }\phi _{\max }^{k}\right) ^{\frac{{\large 5}}{%
{\large 6}}} \\
&\leq &C^{\frac{{\large 5}}{{\large 6}}}\dsum\limits_{j=0}^{\infty
}\dsum\limits_{k=j}^{\infty }\left( \phi _{\max }^{\frac{{\large 5}}{{\large %
6}}}\right) ^{k}\text{ } \\
&&\left( \text{by the inequality }\left\vert \dsum\limits_{i=1}^{\infty
}a_{i}\right\vert ^{r}\leq \dsum\limits_{i=1}^{\infty }\left\vert
a_{i}\right\vert ^{r}\text{ for }r\leq 1\right) \\
&=&C^{\frac{{\large 5}}{{\large 6}}}\dsum\limits_{j=0}^{\infty }\left(
j+1\right) \left( \phi _{\max }^{\frac{{\large 5}}{{\large 6}}}\right) ^{j}
\\
&=&C^{\frac{{\large 5}}{{\large 6}}}\left[ 1-\phi _{\max }^{\frac{{\large 5}%
}{{\large 6}}}\right] ^{-2}\text{ }\left( \text{by Lemma OA-10}\right) \\
&<&\infty \text{ \ }\left( \text{since }0<\phi _{\max }^{\frac{{\large 5}}{%
{\large 6}}}<1\text{ given that }0<\phi _{\max }<1\right) \text{.}
\end{eqnarray*}

Hence, all conditions of Lemma OA-8 are fulfilled. Applying Lemma OA-8, we
then obtain that there exists a constant $\overline{C}$ such that%
\begin{eqnarray*}
\beta _{\xi }\left( m\right) &\leq &\overline{C}\dsum\limits_{j=m}^{\infty
}\left( \dsum\limits_{k=j}^{\infty }\left\Vert \Psi _{k}\right\Vert
_{2}\right) ^{\frac{{\large 5}}{{\large 6}}} \\
&\leq &\overline{C}\dsum\limits_{j=m}^{\infty }\left(
\dsum\limits_{k=j}^{\infty }C\phi _{\max }^{k}\right) ^{\frac{{\large 5}}{%
{\large 6}}} \\
&=&\overline{C}C^{\frac{{\large 5}}{{\large 6}}}\dsum\limits_{j=m}^{\infty
}\left( \dsum\limits_{k=j}^{\infty }\phi _{\max }^{k}\right) ^{\frac{{\large %
5}}{{\large 6}}} \\
&\leq &\overline{C}C^{\frac{{\large 5}}{{\large 6}}}\dsum\limits_{j=m}^{%
\infty }\dsum\limits_{k=j}^{\infty }\left( \phi _{\max }^{\frac{{\large 5}}{%
{\large 6}}}\right) ^{k} \\
&=&\overline{C}C^{\frac{{\large 5}}{{\large 6}}}\left( \phi _{\max }^{\frac{%
{\large 5}}{{\large 6}}}\right) ^{m}\dsum\limits_{j=0}^{\infty }\left(
j+1\right) \left( \phi _{\max }^{\frac{{\large 5}}{{\large 6}}}\right) ^{j}
\\
&=&\overline{C}C^{\frac{{\large 5}}{{\large 6}}}\left( \phi _{\max }^{\frac{%
{\large 5}}{{\large 6}}}\right) ^{m}\left[ 1-\phi _{\max }^{\frac{{\large 5}%
}{{\large 6}}}\right] ^{-2} \\
&=&\overline{C}C^{\frac{{\large 5}}{{\large 6}}}\left[ 1-\phi _{\max }^{%
\frac{{\large 5}}{{\large 6}}}\right] ^{-2}\exp \left\{ -\left[ \frac{5}{6}%
\left\vert \ln \phi _{\max }\right\vert \right] m\right\} \text{ }\left( 
\text{since }0<\phi _{\max }<1\right) \\
&\leq &C_{1}\exp \left\{ -C_{2}m\right\} \rightarrow 0\text{ as }%
m\rightarrow \infty \text{. }
\end{eqnarray*}%
for some positive constants $C_{1}$ and $C_{2}$ such that 
\begin{equation*}
C_{1}\geq \overline{C}C^{\frac{{\large 5}}{{\large 6}}}\left[ 1-\phi _{\max
}^{\frac{{\large 5}}{{\large 6}}}\right] ^{-2}\text{ and }C_{2}\leq \frac{5}{%
6}\left\vert \ln \phi _{\max }\right\vert
\end{equation*}%
It follows that the process $\left\{ \xi _{t}\right\} $ (as defined in
expression (\ref{xi})) is $\beta $ mixing with beta coefficient $\beta _{\xi
}\left( m\right) $ satisfying 
\begin{equation*}
\beta _{\xi }\left( m\right) \leq C_{1}\exp \left\{ -C_{2}m\right\} \text{. }
\end{equation*}%
Since%
\begin{equation*}
W_{t}=\mu _{\ast }+\dsum\limits_{j=0}^{\infty }\Psi _{j}\varepsilon
_{t-j}=\mu _{\ast }+\xi _{t}
\end{equation*}%
and since $\mu _{\ast }$ is a nonrandom parameter, we can then apply part
(a) of Lemma OA-2 to deduce that $\left\{ W_{t}\right\} $ is a $\beta $
mixing process with $\beta $ coefficient $\beta _{W}\left( m\right) $
satisfying the inequality%
\begin{equation*}
\beta _{W}\left( m\right) \leq C_{1}\exp \left\{ -C_{2}m\right\} \text{. }%
\square
\end{equation*}

\medskip

\noindent \noindent \textbf{Lemma OA-12: }Let $\underline{Y}_{t}=\left( 
\begin{array}{ccccc}
Y_{t}^{\prime } & Y_{t-1}^{\prime } & \cdots & Y_{t-p{\LARGE +}2}^{\prime }
& Y_{t-p{\LARGE +}1}^{\prime }%
\end{array}%
\right) ^{\prime }$ and

\noindent $\underline{F}_{t}=\left( 
\begin{array}{ccccc}
F_{t}^{\prime } & F_{t-1}^{\prime } & \cdots & F_{t-p{\LARGE +}2}^{\prime }
& F_{t-p{\LARGE +}1}^{\prime }%
\end{array}%
\right) ^{\prime }$. Under Assumptions 2-1, 2-2, 2-5, 2-6, and 2-9(b); the
following statements are true as $N,T\rightarrow \infty $

\begin{enumerate}
\item[(a)] 
\begin{equation*}
\max_{1\leq \ell \leq d}\max_{i\in H^{{\large c}}}\left\vert \frac{1}{q}%
\dsum\limits_{r=1}^{q}\frac{1}{\tau _{1}}\dsum\limits_{t=\left( r-1\right)
\tau +p}^{\left( r-1\right) \tau +\tau _{1}+p-1}\gamma _{i}^{\prime }\left( 
\underline{F}_{t}\underline{Y}_{t}^{\prime }-E\left[ \underline{F}_{t}%
\underline{Y}_{t}^{\prime }\right] \right) \alpha _{YY,\ell }\right\vert 
\overset{p}{\rightarrow }0
\end{equation*}

\item[(b)] 
\begin{equation*}
\max_{1\leq \ell \leq d}\max_{i\in H^{{\large c}}}\left\vert \frac{1}{q}%
\dsum\limits_{r=1}^{q}\frac{1}{\tau _{1}}\dsum\limits_{t=\left( r-1\right)
\tau +p}^{\left( r-1\right) \tau +\tau _{1}+p-1}\gamma _{i}^{\prime }\left( 
\underline{F}_{t}\underline{F}_{t}^{\prime }-E\left[ \underline{F}_{t}%
\underline{F}_{t}^{\prime }\right] \right) \alpha _{YF,\ell }\right\vert 
\overset{p}{\rightarrow }0
\end{equation*}

\item[(c)] 
\begin{equation*}
\max_{1\leq \ell \leq d}\max_{i\in H^{{\large c}}}\left\vert \frac{1}{q}%
\dsum\limits_{r=1}^{q}\frac{1}{\tau _{1}}\dsum\limits_{t=\left( r-1\right)
\tau +p}^{\left( r-1\right) \tau +\tau _{1}+p-1}\gamma _{i}^{\prime }\left( 
\underline{F}_{t}-E\left[ \underline{F}_{t}\right] \right) \mu _{Y,\ell
}\right\vert \overset{p}{\rightarrow }0
\end{equation*}

\item[(d)] 
\begin{eqnarray*}
&&\max_{1\leq \ell \leq d}\max_{i\in H^{{\large c}}}\frac{1}{q}%
\dsum\limits_{r=1}^{q}\left( \frac{1}{\tau _{1}}\dsum\limits_{t=\left(
r-1\right) \tau +p}^{\left( r-1\right) \tau +\tau _{1}+p-1}\gamma
_{i}^{\prime }\left\{ \left( \underline{F}_{t}-E\left[ \underline{F}_{t}%
\right] \right) \mu _{Y,\ell }+\left( \underline{F}_{t}\underline{Y}%
_{t}^{\prime }-E\left[ \underline{F}_{t}\underline{Y}_{t}^{\prime }\right]
\right) \alpha _{YY,\ell }\right. \right. \\
&&\text{ \ \ \ \ \ \ \ \ \ \ \ \ \ \ \ \ \ \ \ \ \ \ \ \ \ \ \ \ \ \ \ \ \ \
\ \ \ \ \ \ \ \ \ \ \ \ \ \ \ \ }\left. \text{\ }\left. +\left( \underline{F}%
_{t}\underline{F}_{t}^{\prime }-E\left[ \underline{F}_{t}\underline{F}%
_{t}^{\prime }\right] \right) \alpha _{YF,\ell }\right\} \right) ^{2} \\
&&\overset{p}{\rightarrow }0
\end{eqnarray*}

\item[(e)] 
\begin{equation*}
\max_{1\leq \ell \leq d}\max_{i\in H^{{\large c}}}\frac{1}{q}%
\dsum\limits_{r=1}^{q}\left( \frac{1}{\tau _{1}}\dsum\limits_{t=\left(
r-1\right) \tau +p}^{\left( r-1\right) \tau +\tau _{1}+p-1}\gamma
_{i}^{\prime }\underline{F}_{t}\left[ \mu _{Y,\ell }+\underline{Y}%
_{t}^{\prime }\alpha _{YY,\ell }+\underline{F}_{t}^{\prime }\alpha _{YF,\ell
}\right] \right) ^{2}=O_{p}\left( 1\right) \text{.}
\end{equation*}

\item[(f)] 
\begin{eqnarray*}
&&\max_{1\leq \ell \leq d}\max_{i\in H^{{\large c}}}\left\vert \frac{1}{q}%
\dsum\limits_{r=1}^{q}\left\{ \left( \frac{1}{\tau _{1}}\dsum\limits_{t=%
\left( r-1\right) \tau +p}^{\left( r-1\right) \tau +\tau _{1}+p-1}\gamma
_{i}^{\prime }\left( \underline{F}_{t}-E\left[ \underline{F}_{t}\right]
\right) \mu _{Y,\ell }\right. \right. \right. \\
&&\text{ \ \ \ \ }\left. \left. +\frac{1}{\tau _{1}}\dsum\limits_{t=\left(
r-1\right) \tau +p}^{\left( r-1\right) \tau +\tau _{1}+p-1}\left\{ \gamma
_{i}^{\prime }\left( \underline{F}_{t}\underline{Y}_{t}^{\prime }-E\left[ 
\underline{F}_{t}\underline{Y}_{t}^{\prime }\right] \right) \alpha _{YY,\ell
}+\gamma _{i}^{\prime }\left( \underline{F}_{t}\underline{F}_{t}^{\prime }-E%
\left[ \underline{F}_{t}\underline{F}_{t}^{\prime }\right] \right) \alpha
_{YF,\ell }\right\} \right\} \right) \\
&&\text{ \ \ \ }\left. \text{\ }\left. \times \left( \frac{1}{\tau _{1}}%
\dsum\limits_{t=\left( r-1\right) \tau +p}^{\left( r-1\right) \tau +\tau
_{1}+p-1}\left\{ \gamma _{i}^{\prime }E\left[ \underline{F}_{t}\right] \mu
_{Y,\ell }+\gamma _{i}^{\prime }E\left[ \underline{F}_{t}\underline{Y}%
_{t}^{\prime }\right] \alpha _{YY,\ell }+\gamma _{i}^{\prime }E\left[ 
\underline{F}_{t}\underline{F}_{t}^{\prime }\right] \alpha _{YF,\ell
}\right\} \right) \right\} \right\vert \\
&&\overset{p}{\rightarrow }0
\end{eqnarray*}

\item[(g)] 
\begin{eqnarray*}
&&\max_{1\leq \ell \leq d}\max_{i\in H^{{\large c}}}\left\vert \frac{1}{q}%
\dsum\limits_{r=1}^{q}\left( \frac{1}{\tau _{1}}\dsum\limits_{t=\left(
r-1\right) \tau +p}^{\left( r-1\right) \tau +\tau _{1}+p-1}\gamma
_{i}^{\prime }\underline{F}_{t}\left[ \mu _{Y,\ell }+\underline{Y}%
_{t}^{\prime }\alpha _{YY,\ell }+\underline{F}_{t}^{\prime }\alpha _{YF,\ell
}\right] \right) \right. \\
&&\text{ \ \ \ \ \ \ \ \ \ \ \ \ \ \ \ \ \ \ \ \ \ \ \ }\left. \times \left( 
\frac{1}{\tau _{1}}\dsum\limits_{t=\left( r-1\right) \tau +p}^{\left(
r-1\right) \tau +\tau _{1}+p-1}y_{\ell ,t{\LARGE +}1}u_{it}\right)
\right\vert \\
&&\overset{p}{\rightarrow }0
\end{eqnarray*}

\item[(h)] 
\begin{eqnarray*}
&&\max_{1\leq \ell \leq d}\max_{i\in H^{{\large c}}}\left\vert \frac{1}{q}%
\dsum\limits_{r=1}^{q}\left( \frac{1}{\tau _{1}}\dsum\limits_{t=\left(
r-1\right) \tau +p}^{\left( r-1\right) \tau +\tau _{1}+p-1}\gamma
_{i}^{\prime }\underline{F}_{t}\left[ \mu _{Y,\ell }+\underline{Y}%
_{t}^{\prime }\alpha _{YY,\ell }+\underline{F}_{t}^{\prime }\alpha _{YF,\ell
}\right] \right) \right. \\
&&\text{ \ \ \ \ \ \ \ \ \ \ \ \ \ \ \ \ \ \ \ \ \ }\left. \text{\ \ }\times
\left( \frac{1}{\tau _{1}}\dsum\limits_{t=\left( r-1\right) \tau +p}^{\left(
r-1\right) \tau +\tau _{1}+p-1}\gamma _{i}^{\prime }\underline{F}%
_{t}\varepsilon _{\ell ,t{\LARGE +}1}\right) \right\vert \\
&&\overset{p}{\rightarrow }0
\end{eqnarray*}
\end{enumerate}

\medskip

\noindent \textbf{Proof of Lemma OA-12:}

To show part (a), note that, for any $\epsilon >0$,

\begin{eqnarray}
&&P\left\{ \max_{1\leq \ell \leq d}\max_{i{\large \in }H^{{\large c}%
}}\left\vert \frac{1}{q}\dsum\limits_{r=1}^{q}\frac{1}{\tau _{1}}%
\dsum\limits_{t=\left( r-1\right) \tau +p}^{\left( r-1\right) \tau +\tau
_{1}+p-1}\gamma _{i}^{\prime }\left( \underline{F}_{t}\underline{Y}%
_{t}^{\prime }-E\left[ \underline{F}_{t}\underline{Y}_{t}^{\prime }\right]
\right) \alpha _{YY,\ell }\right\vert \geq \epsilon \right\}  \notag \\
&=&P\left\{ \max_{1\leq \ell \leq d}\max_{i{\large \in }H^{{\large c}%
}}\left( \frac{1}{q}\dsum\limits_{r=1}^{q}\frac{1}{\tau _{1}}%
\dsum\limits_{t=\left( r-1\right) \tau +p}^{\left( r-1\right) \tau +\tau
_{1}+p-1}\gamma _{i}^{\prime }\left( \underline{F}_{t}\underline{Y}%
_{t}^{\prime }-E\left[ \underline{F}_{t}\underline{Y}_{t}^{\prime }\right]
\right) \alpha _{YY,\ell }\right) ^{2}\geq \epsilon ^{2}\right\}  \notag \\
&\leq &P\left\{ \max_{1\leq \ell \leq d}\max_{i{\large \in }H^{{\large c}}}%
\frac{1}{q}\dsum\limits_{r=1}^{q}\left( \frac{1}{\tau _{1}}%
\dsum\limits_{t=\left( r-1\right) \tau +p}^{\left( r-1\right) \tau +\tau
_{1}+p-1}\gamma _{i}^{\prime }\left( \underline{F}_{t}\underline{Y}%
_{t}^{\prime }-E\left[ \underline{F}_{t}\underline{Y}_{t}^{\prime }\right]
\right) \alpha _{YY,\ell }\right) ^{2}\geq \epsilon ^{2}\right\} \text{ } 
\notag \\
&&\left( \text{by Jensen's inequality}\right)  \notag \\
&=&P\left\{ \max_{i{\large \in }H^{{\large c}}}\max_{1\leq \ell \leq d}\frac{%
1}{q}\dsum\limits_{r=1}^{q}\left( \gamma _{i}^{\prime }\left[ \frac{1}{\tau
_{1}}\dsum\limits_{t=\left( r-1\right) \tau +p}^{\left( r-1\right) \tau
+\tau _{1}+p-1}\left( \underline{F}_{t}\underline{Y}_{t}^{\prime }-E\left[ 
\underline{F}_{t}\underline{Y}_{t}^{\prime }\right] \right) \alpha _{YY,\ell
}\right] \right) ^{2}\geq \epsilon ^{2}\right\}  \notag \\
&\leq &P\left\{ \max_{i{\large \in }H^{{\large c}}}\dsum\limits_{\ell =1}^{d}%
\frac{1}{q}\dsum\limits_{r=1}^{q}\left( \gamma _{i}^{\prime }\left[ \frac{1}{%
\tau _{1}}\dsum\limits_{t=\left( r-1\right) \tau +p}^{\left( r-1\right) \tau
+\tau _{1}+p-1}\left( \underline{F}_{t}\underline{Y}_{t}^{\prime }-E\left[ 
\underline{F}_{t}\underline{Y}_{t}^{\prime }\right] \right) \alpha _{YY,\ell
}\right] \right) ^{2}\geq \epsilon ^{2}\right\}  \notag \\
&\leq &P\left\{ \max_{i{\large \in }H^{{\large c}}}\left\Vert \gamma
_{i}\right\Vert _{2}^{2}\dsum\limits_{\ell =1}^{d}\left( \frac{1}{q}%
\dsum\limits_{r=1}^{q}\left[ \frac{1}{\tau _{1}}\dsum\limits_{t=\left(
r-1\right) \tau +p}^{\left( r-1\right) \tau +\tau _{1}+p-1}\left( \underline{%
F}_{t}\underline{Y}_{t}^{\prime }-E\left[ \underline{F}_{t}\underline{Y}%
_{t}^{\prime }\right] \right) \alpha _{YY,\ell }\right] ^{\prime }\right.
\right.  \notag \\
&&\text{ \ \ \ \ \ \ \ \ \ \ \ \ \ \ \ \ \ \ \ \ \ \ \ \ \ \ \ \ }\left.
\left. \times \left[ \frac{1}{\tau _{1}}\dsum\limits_{t=\left( r-1\right)
\tau +p}^{\left( r-1\right) \tau +\tau _{1}+p-1}\left( \underline{F}_{t}%
\underline{Y}_{t}^{\prime }-E\left[ \underline{F}_{t}\underline{Y}%
_{t}^{\prime }\right] \right) \alpha _{YY,\ell }\right] \right) \geq
\epsilon ^{2}\right\}  \notag \\
&=&P\left\{ \max_{i{\large \in }H^{{\large c}}}\left\Vert \gamma
_{i}\right\Vert _{2}^{2}\dsum\limits_{\ell =1}^{d}\frac{1}{q}%
\dsum\limits_{r=1}^{q}\frac{1}{\tau _{1}^{2}}\dsum\limits_{t=\left(
r-1\right) \tau +p}^{\left( r-1\right) \tau +\tau
_{1}+p-1}\dsum\limits_{s=\left( r-1\right) \tau +p}^{\left( r-1\right) \tau
+\tau _{1}+p-1}\alpha _{YY,\ell }^{\prime }\left( \underline{F}_{t}%
\underline{Y}_{t}^{\prime }-E\left[ \underline{F}_{t}\underline{Y}%
_{t}^{\prime }\right] \right) ^{\prime }\right.  \notag \\
&&\text{ \ \ \ \ \ \ \ \ \ \ \ \ \ \ \ \ \ \ \ \ \ \ \ \ \ \ \ \ \ \ \ \ \ \
\ \ \ \ \ \ \ \ \ \ \ \ \ \ \ \ \ \ \ \ \ \ \ \ \ \ \ \ \ \ \ \ \ \ \ \ }%
\left. \times \left( \underline{F}_{s}\underline{Y}_{s}^{\prime }-E\left[ 
\underline{F}_{s}\underline{Y}_{s}^{\prime }\right] \right) \alpha _{YY,\ell
}\geq \epsilon ^{2}\right\}  \notag \\
&\leq &\frac{\max_{i{\large \in }H^{{\large c}}}\left\Vert \gamma
_{i}\right\Vert _{2}^{2}}{\epsilon ^{2}}\dsum\limits_{\ell =1}^{d}\frac{1}{q}%
\dsum\limits_{r=1}^{q}\frac{1}{\tau _{1}^{2}}\dsum\limits_{t=\left(
r-1\right) \tau +p}^{\left( r-1\right) \tau +\tau
_{1}+p-1}\dsum\limits_{s=\left( r-1\right) \tau +p}^{\left( r-1\right) \tau
+\tau _{1}+p-1}\left\{ \alpha _{YY,\ell }^{\prime }\right.  \notag \\
&&\text{ \ \ \ \ \ \ \ \ \ \ \ \ \ \ \ \ \ \ \ \ \ \ \ \ \ \ \ \ \ \ \ \ \ \
\ \ \ \ }\left. \times E\left[ \left( \underline{F}_{t}\underline{Y}%
_{t}^{\prime }-E\left[ \underline{F}_{t}\underline{Y}_{t}^{\prime }\right]
\right) ^{\prime }\left( \underline{F}_{s}\underline{Y}_{s}^{\prime }-E\left[
\underline{F}_{s}\underline{Y}_{s}^{\prime }\right] \right) \right] \alpha
_{YY,\ell }\right\}  \notag \\
&&\left( \text{by Markov's inequality}\right)  \notag \\
&\leq &\frac{C}{\epsilon ^{2}}\dsum\limits_{\ell =1}^{d}\frac{1}{q}%
\dsum\limits_{r=1}^{q}\frac{1}{\tau _{1}^{2}}\dsum\limits_{t=\left(
r-1\right) \tau +p}^{\left( r-1\right) \tau +\tau
_{1}+p-1}\dsum\limits_{s=\left( r-1\right) \tau +p}^{\left( r-1\right) \tau
+\tau _{1}+p-1}\left\{ \alpha _{YY,\ell }^{\prime }\right.  \notag \\
&&\text{ \ \ \ \ \ \ \ \ \ \ \ \ \ \ \ \ \ \ \ \ \ \ \ \ \ \ }\left. \times E%
\left[ \left( \underline{F}_{t}\underline{Y}_{t}^{\prime }-E\left[ 
\underline{F}_{t}\underline{Y}_{t}^{\prime }\right] \right) ^{\prime }\left( 
\underline{F}_{s}\underline{Y}_{s}^{\prime }-E\left[ \underline{F}_{s}%
\underline{Y}_{s}^{\prime }\right] \right) \right] \alpha _{YY,\ell
}\right\} \text{ }  \label{PmaxFY bd} \\
&&\left( \text{by Assumption 2-5}\right)  \notag
\end{eqnarray}%
Next, write%
\begin{eqnarray}
&&\dsum\limits_{\ell =1}^{d}\left( \frac{1}{q}\dsum\limits_{r=1}^{q}\frac{1}{%
\tau _{1}^{2}}\dsum\limits_{t=\left( r-1\right) \tau +p}^{\left( r-1\right)
\tau +\tau _{1}+p-1}\dsum\limits_{s=\left( r-1\right) \tau +p}^{\left(
r-1\right) \tau +\tau _{1}+p-1}\left\{ \alpha _{YY,\ell }^{\prime }\right.
\right.  \notag \\
&&\text{ \ \ \ \ \ \ \ \ \ \ \ \ \ \ \ \ \ \ \ \ \ \ \ \ \ }\left. \text{\ }%
\left. \times E\left[ \left( \underline{F}_{t}\underline{Y}_{t}^{\prime }-E%
\left[ \underline{F}_{t}\underline{Y}_{t}^{\prime }\right] \right) ^{\prime
}\left( \underline{F}_{s}\underline{Y}_{s}^{\prime }-E\left[ \underline{F}%
_{s}\underline{Y}_{s}^{\prime }\right] \right) \right] \alpha _{YY,\ell
}\right\} \right) \text{ }  \notag \\
&=&\dsum\limits_{\ell =1}^{d}\left( \frac{1}{q}\dsum\limits_{r=1}^{q}\frac{1%
}{\tau _{1}^{2}}\dsum\limits_{t=\left( r-1\right) \tau +p}^{\left(
r-1\right) \tau +\tau _{1}+p-1}\alpha _{YY,\ell }^{\prime }E\left[ \left( 
\underline{F}_{t}\underline{Y}_{t}^{\prime }-E\left[ \underline{F}_{t}%
\underline{Y}_{t}^{\prime }\right] \right) ^{\prime }\left( \underline{F}_{t}%
\underline{Y}_{t}^{\prime }-E\left[ \underline{F}_{t}\underline{Y}%
_{t}^{\prime }\right] \right) \right] \alpha _{YY,\ell }\right)  \notag \\
&&+\dsum\limits_{\ell =1}^{d}\left( \frac{2}{q}\dsum\limits_{r=1}^{q}\frac{1%
}{\tau _{1}^{2}}\dsum\limits_{t=\left( r-1\right) \tau +p}^{\left(
r-1\right) \tau +\tau _{1}+p-2}\dsum\limits_{m=1}^{\left( r-1\right) \tau
+\tau _{1}+p-t-1}\left\{ \alpha _{YY,\ell }^{\prime }\right. \right.  \notag
\\
&&\text{ \ \ \ \ \ \ \ \ \ \ \ \ \ \ \ \ \ \ \ \ \ }\left. \text{\ }\left.
\times E\left[ \left( \underline{F}_{t}\underline{Y}_{t}^{\prime }-E\left[ 
\underline{F}_{t}\underline{Y}_{t}^{\prime }\right] \right) ^{\prime }\left( 
\underline{F}_{t+m}\underline{Y}_{t+m}^{\prime }-E\left[ \underline{F}_{t+m}%
\underline{Y}_{t+m}^{\prime }\right] \right) \right] \alpha _{YY,\ell
}\right\} \right)  \notag \\
&\leq &\dsum\limits_{\ell =1}^{d}\left( \frac{1}{q}\dsum\limits_{r=1}^{q}%
\frac{1}{\tau _{1}^{2}}\dsum\limits_{t=\left( r-1\right) \tau +p}^{\left(
r-1\right) \tau +\tau _{1}+p-1}\alpha _{YY,\ell }^{\prime }E\left[ \left( 
\underline{F}_{t}\underline{Y}_{t}^{\prime }-E\left[ \underline{F}_{t}%
\underline{Y}_{t}^{\prime }\right] \right) ^{\prime }\left( \underline{F}_{t}%
\underline{Y}_{t}^{\prime }-E\left[ \underline{F}_{t}\underline{Y}%
_{t}^{\prime }\right] \right) \right] \alpha _{YY,\ell }\right)  \notag \\
&&+\dsum\limits_{\ell =1}^{d}\left( \frac{2}{q}\dsum\limits_{r=1}^{q}\frac{1%
}{\tau _{1}^{2}}\dsum\limits_{t=\left( r-1\right) \tau +p}^{\left(
r-1\right) \tau +\tau _{1}+p-2}\dsum\limits_{m=1}^{\left( r-1\right) \tau
+\tau _{1}+p-t-1}\left\vert \alpha _{YY,\ell }^{\prime }\right. \right. 
\notag \\
&&\text{ \ \ \ \ \ \ \ \ \ \ }\left. \left. \times E\left[ \left( \underline{%
F}_{t}\underline{Y}_{t}^{\prime }-E\left[ \underline{F}_{t}\underline{Y}%
_{t}^{\prime }\right] \right) ^{\prime }\left( \underline{F}_{t+m}\underline{%
Y}_{t+m}^{\prime }-E\left[ \underline{F}_{t+m}\underline{Y}_{t+m}^{\prime }%
\right] \right) \right] \alpha _{YY,\ell }\right\vert \right)
\label{FY 2nd moment}
\end{eqnarray}%
Let $e_{\ell ,d}$ be a $d\times 1$ elementary vector whose $\ell ^{th}$
component is $1$ and all other components are $0$, and note that%
\begin{eqnarray}
&&\dsum\limits_{\ell =1}^{d}\left( \frac{1}{q}\dsum\limits_{r=1}^{q}\frac{1}{%
\tau _{1}^{2}}\dsum\limits_{t=\left( r-1\right) \tau +p}^{\left( r-1\right)
\tau +\tau _{1}+p-1}\alpha _{YY,\ell }^{\prime }E\left[ \left( \underline{F}%
_{t}\underline{Y}_{t}^{\prime }-E\left[ \underline{F}_{t}\underline{Y}%
_{t}^{\prime }\right] \right) ^{\prime }\left( \underline{F}_{t}\underline{Y}%
_{t}^{\prime }-E\left[ \underline{F}_{t}\underline{Y}_{t}^{\prime }\right]
\right) \right] \alpha _{YY,\ell }\right)  \notag \\
&=&\dsum\limits_{\ell =1}^{d}\left( \frac{1}{q\tau _{1}^{2}}%
\dsum\limits_{r=1}^{q}\dsum\limits_{t=\left( r-1\right) \tau +p}^{\left(
r-1\right) \tau +\tau _{1}+p-1}e_{\ell ,d}^{\prime }A_{YY}E\left[ \left( 
\underline{F}_{t}\underline{Y}_{t}^{\prime }-E\left[ \underline{F}_{t}%
\underline{Y}_{t}^{\prime }\right] \right) ^{\prime }\left( \underline{F}_{t}%
\underline{Y}_{t}^{\prime }-E\left[ \underline{F}_{t}\underline{Y}%
_{t}^{\prime }\right] \right) \right] A_{YY}^{\prime }e_{\ell ,d}\right) 
\notag \\
&=&\dsum\limits_{\ell =1}^{d}\frac{1}{q\tau _{1}^{2}}\dsum\limits_{r=1}^{q}%
\left( \dsum\limits_{t=\left( r-1\right) \tau +p}^{\left( r-1\right) \tau
+\tau _{1}+p-1}e_{\ell ,d}^{\prime }A_{YY}E\left[ \underline{Y}_{t}%
\underline{F}_{t}^{\prime }\underline{F}_{t}\underline{Y}_{t}^{\prime }%
\right] A_{YY}^{\prime }e_{\ell ,d}\right.  \notag \\
&&\text{ \ \ \ \ \ \ \ \ \ \ \ \ \ \ \ }\left. -\dsum\limits_{t=\left(
r-1\right) \tau +p}^{\left( r-1\right) \tau +\tau _{1}+p-1}e_{\ell
,d}^{\prime }A_{YY}E\left[ \underline{Y}_{t}\underline{F}_{t}^{\prime }%
\right] E\left[ \underline{F}_{t}\underline{Y}_{t}^{\prime }\right]
A_{YY}^{\prime }e_{\ell ,d}\right)  \notag \\
&\leq &\dsum\limits_{\ell =1}^{d}\frac{1}{q\tau _{1}^{2}}\dsum%
\limits_{r=1}^{q}\dsum\limits_{t=\left( r-1\right) \tau +p}^{\left(
r-1\right) \tau +\tau _{1}+p-1}E\left[ \left\Vert \underline{F}%
_{t}\right\Vert _{2}^{2}\left( e_{\ell ,d}^{\prime }A_{YY}\underline{Y}%
_{t}\right) ^{2}\right]  \notag \\
&\leq &\dsum\limits_{\ell =1}^{d}\frac{1}{q\tau _{1}^{2}}\dsum%
\limits_{r=1}^{q}\dsum\limits_{t=\left( r-1\right) \tau +p}^{\left(
r-1\right) \tau +\tau _{1}+p-1}\sqrt{E\left[ \left\Vert \underline{F}%
_{t}\right\Vert _{2}^{4}\right] }\sqrt{E\left( e_{\ell ,d}^{\prime }A_{YY}%
\underline{Y}_{t}\underline{Y}_{t}^{\prime }A_{YY}^{\prime }e_{\ell
,d}\right) ^{2}}\text{ }\left( \text{by CS inequality}\right)  \notag \\
&\leq &\dsum\limits_{\ell =1}^{d}\frac{1}{q\tau _{1}^{2}}\dsum%
\limits_{r=1}^{q}\dsum\limits_{t=\left( r-1\right) \tau +p}^{\left(
r-1\right) \tau +\tau _{1}+p-1}\sqrt{E\left[ \left\Vert \underline{F}%
_{t}\right\Vert _{2}^{4}\right] }\sqrt{E\left[ \left\Vert \underline{Y}%
_{t}\right\Vert _{2}^{4}\right] }\sqrt{\left( e_{\ell ,d}^{\prime
}A_{YY}A_{YY}^{\prime }e_{\ell ,d}\right) ^{2}}  \notag \\
&\leq &\dsum\limits_{\ell =1}^{d}\frac{1}{q\tau _{1}^{2}}\dsum%
\limits_{r=1}^{q}\dsum\limits_{t=\left( r-1\right) \tau +p}^{\left(
r-1\right) \tau +\tau _{1}+p-1}\sqrt{E\left[ \left\Vert \underline{F}%
_{t}\right\Vert _{2}^{4}\right] }\sqrt{E\left[ \left\Vert \underline{Y}%
_{t}\right\Vert _{2}^{4}\right] }\left\Vert A_{YY}\right\Vert _{2}^{2}\sqrt{%
\left( e_{\ell ,d}^{\prime }e_{\ell ,d}\right) ^{2}}  \notag \\
&\leq &\frac{d\left( C^{\dagger }\right) ^{2}}{q\tau _{1}^{2}}%
\dsum\limits_{r=1}^{q}\dsum\limits_{t=\left( r-1\right) \tau +p}^{\left(
r-1\right) \tau +\tau _{1}+p-1}\sqrt{E\left[ \left\Vert \underline{F}%
_{t}\right\Vert _{2}^{4}\right] }\sqrt{E\left[ \left\Vert \underline{Y}%
_{t}\right\Vert _{2}^{4}\right] }\phi _{\max }^{2}\text{ \ }  \notag \\
&&\left( \text{by part (a)\ of Lemma OA-7 and by the fact that }e_{\ell ,d}%
\text{ is an elementary vector}\right)  \notag \\
&\leq &\frac{\overline{C}}{\tau _{1}}=O\left( \frac{1}{\tau _{1}}\right) 
\text{. }  \label{FY 2nd moment 1}
\end{eqnarray}%
for some positive constant $\overline{C}\geq d\left( C^{\dagger }\right) ^{2}%
\sqrt{E\left[ \left\Vert \underline{F}_{t}\right\Vert _{2}^{4}\right] }\sqrt{%
E\left[ \left\Vert \underline{Y}_{t}\right\Vert _{2}^{4}\right] }\phi _{\max
}^{2}$, which exists in light of Lemma OA-5 and the fact that $0<\phi _{\max
}<1$ given Assumption 2-1.

To analyze the second term on the right-hand side of expression (\ref{FY 2nd
moment}), note first that by Lemma OA-11, $\left\{ \left( Y_{t}^{\prime
},F_{t}^{\prime }\right) ^{\prime }\right\} $ is $\beta $-mixing with $\beta 
$ mixing coefficient satisfying 
\begin{equation*}
\beta _{W}\left( m\right) \leq C_{1}\exp \left\{ -C_{2}m\right\} \text{ for
some positive constants }C_{1}\text{ and }C_{2}\text{.}
\end{equation*}%
Since $\alpha _{W,m}\leq \beta _{W}\left( m\right) $, it follows that $%
W_{t}=\left( Y_{t}^{\prime },F_{t}^{\prime }\right) ^{\prime }$ is $\alpha $%
-mixing as well, with $\alpha $ mixing coefficient satisfying%
\begin{equation*}
\alpha _{W,m}\leq C_{1}\exp \left\{ -C_{2}m\right\}
\end{equation*}

\noindent Moreover, by applying part (b) of Lemma OA-2, we further deduce
that $X_{1t}=\underline{F}_{t}\underline{Y}_{t}^{\prime }A_{YY}^{\prime
}e_{\ell ,d}$ is also $\alpha $-mixing with $\alpha $ mixing coefficient
satisfying%
\begin{eqnarray*}
\alpha _{X_{1},m} &\leq &C_{1}\exp \left\{ -C_{2}\left( m-p+1\right) \right\}
\\
&\leq &C_{1}^{\ast }\exp \left\{ -C_{2}m\right\}
\end{eqnarray*}%
for some positive constant $C_{1}^{\ast }\geq C_{1}\exp \left\{ C_{2}\left(
p-1\right) \right\} $. Hence, we can apply Lemma OA-3 with $p=3$ and $r=3$
to obtain%
\begin{eqnarray*}
&&\left\vert \alpha _{YY,\ell }^{\prime }E\left[ \left( \underline{F}_{t}%
\underline{Y}_{t}^{\prime }-E\left[ \underline{F}_{t}\underline{Y}%
_{t}^{\prime }\right] \right) ^{\prime }\left( \underline{F}_{t+m}\underline{%
Y}_{t+m}^{\prime }-E\left[ \underline{F}_{t+m}\underline{Y}_{t+m}^{\prime }%
\right] \right) \right] \alpha _{YY,\ell }\right\vert \\
&=&\left\vert e_{\ell ,d}^{\prime }A_{YY}E\left[ \left( \underline{F}_{t}%
\underline{Y}_{t}^{\prime }-E\left[ \underline{F}_{t}\underline{Y}%
_{t}^{\prime }\right] \right) ^{\prime }\left( \underline{F}_{t+m}\underline{%
Y}_{t+m}^{\prime }-E\left[ \underline{F}_{t+m}\underline{Y}_{t+m}^{\prime }%
\right] \right) \right] A_{YY}^{\prime }e_{\ell ,d}\right\vert \\
&=&\left\vert \dsum\limits_{h=1}^{Kp}e_{\ell ,d}^{\prime }A_{YY}E\left[
\left( \underline{F}_{t}\underline{Y}_{t}^{\prime }-E\left[ \underline{F}_{t}%
\underline{Y}_{t}^{\prime }\right] \right) ^{\prime
}e_{h,Kp}e_{h,Kp}^{\prime }\left( \underline{F}_{t+m}\underline{Y}%
_{t+m}^{\prime }-E\left[ \underline{F}_{t+m}\underline{Y}_{t+m}^{\prime }%
\right] \right) \right] A_{YY}^{\prime }e_{\ell ,d}\right\vert \\
&\leq &\dsum\limits_{h=1}^{Kp}\left\{ 2\left( 2^{\frac{{\large 2}}{{\large 3}%
}}+1\right) \alpha _{X_{1},m}^{\frac{{\Large 1}}{{\Large 3}}}\left(
E\left\vert e_{\ell ,d}^{\prime }A_{YY}\left( \underline{F}_{t}\underline{Y}%
_{t}^{\prime }-E\left[ \underline{F}_{t}\underline{Y}_{t}^{\prime }\right]
\right) ^{\prime }e_{h,Kp}\right\vert ^{3}\right) ^{\frac{{\large 1}}{%
{\large 3}}}\right. \\
&&\text{ \ \ \ \ \ }\left. \times \left( E\left\vert e_{h,Kp}^{\prime
}\left( \underline{F}_{t+m}\underline{Y}_{t+m}^{\prime }-E\left[ \underline{F%
}_{t+m}\underline{Y}_{t+m}^{\prime }\right] \right) A_{YY}^{\prime }e_{\ell
,d}\right\vert ^{3}\right) ^{1/3}\right\}
\end{eqnarray*}%
where $\alpha _{X,m}$ denotes the $\alpha $ mixing coefficient for the
process $\left\{ X_{1t}\right\} $ and where, by our previous calculations, 
\begin{equation*}
\alpha _{X_{1},m}^{\frac{{\Large 1}}{{\Large 3}}}\leq \left( C_{1}^{\ast
}\right) ^{\frac{{\Large 1}}{{\Large 3}}}\exp \left\{ -\frac{C_{2}m}{3}%
\right\} \text{ for all }m\text{ sufficiently large.}
\end{equation*}%
It further follows that there exists a positive constant $C_{3}$ such that%
\begin{eqnarray*}
\dsum\limits_{m=1}^{\infty }\alpha _{X_{1},m}^{\frac{{\Large 1}}{{\Large 3}}%
} &\leq &\left( C_{1}^{\ast }\right) ^{\frac{{\Large 1}}{{\Large 3}}%
}\dsum\limits_{m=1}^{\infty }\exp \left\{ -\frac{C_{2}m}{3}\right\} \\
&\leq &\left( C_{1}^{\ast }\right) ^{\frac{{\Large 1}}{{\Large 3}}%
}\dsum\limits_{m=0}^{\infty }\exp \left\{ -\frac{C_{2}m}{3}\right\} \\
&=&\left( C_{1}^{\ast }\right) ^{\frac{{\Large 1}}{{\Large 3}}}\left[ 1-\exp
\left\{ -\frac{C_{2}}{3}\right\} \right] ^{-1} \\
&\leq &C_{3}
\end{eqnarray*}%
where the last inequality stems from the fact that $\dsum\nolimits_{m=0}^{%
\infty }\exp \left\{ -\left( C_{2}m/3\right) \right\} $ is a convergent
geometric series given that $0<\exp \left\{ -\left( C_{2}/3\right) \right\}
<1$ for $C_{2}>0$. Next, note that%
\begin{eqnarray*}
&&E\left\vert e_{\ell ,d}^{\prime }A_{YY}\left( \underline{F}_{t}\underline{Y%
}_{t}^{\prime }-E\left[ \underline{F}_{t}\underline{Y}_{t}^{\prime }\right]
\right) ^{\prime }e_{h,Kp}\right\vert ^{3} \\
&\leq &2^{2}\left\{ E\left\vert e_{\ell ,d}^{\prime }A_{YY}\underline{Y}_{t}%
\underline{F}_{t}^{\prime }e_{h,Kp}\right\vert ^{3}+\left\vert E\left[
e_{\ell ,d}^{\prime }A_{YY}\underline{Y}_{t}\underline{F}_{t}^{\prime
}e_{h,Kp}\right] \right\vert ^{3}\right\} \left( \text{by Lo\`{e}ve's }c_{r}%
\text{ inequality}\right) \\
&\leq &2^{2}\left\{ E\left\vert e_{\ell ,d}^{\prime }A_{YY}\underline{Y}_{t}%
\underline{F}_{t}^{\prime }e_{h,Kp}\right\vert ^{3}+\left( E\left[
\left\vert e_{\ell ,d}^{\prime }A_{YY}\underline{Y}_{t}\underline{F}%
_{t}^{\prime }e_{h,Kp}\right\vert \right] \right) ^{3}\right\} \text{ }%
\left( \text{by Jensen's inequality}\right) \\
&\leq &2^{2}\left\{ E\left\vert \frac{e_{\ell ,d}^{\prime }A_{YY}\underline{Y%
}_{t}\underline{Y}_{t}^{\prime }A_{YY}^{\prime }e_{\ell ,d}}{2}+\frac{%
e_{h,Kp}^{\prime }\underline{F}_{t}\underline{F}_{t}^{\prime }e_{h,Kp}}{2}%
\right\vert ^{3}+\left( E\left[ \left\vert e_{\ell ,d}^{\prime }A_{YY}%
\underline{Y}_{t}\underline{F}_{t}^{\prime }e_{h,Kp}\right\vert \right]
\right) ^{3}\right\} \\
&\leq &\frac{4}{8}\left[ E\left\vert e_{\ell ,d}^{\prime }A_{YY}\underline{Y}%
_{t}\underline{Y}_{t}^{\prime }A_{YY}^{\prime }e_{\ell ,d}\right\vert
^{3}+E\left\vert e_{h,Kp}^{\prime }\underline{F}_{t}\underline{F}%
_{t}^{\prime }e_{h,Kp}\right\vert ^{3}\right] \\
&&+4\left( \sqrt{E\left[ e_{\ell ,d}^{\prime }A_{YY}\underline{Y}_{t}%
\underline{Y}_{t}^{\prime }A_{YY}^{\prime }e_{\ell ,d}\right] }\sqrt{E\left[
e_{h,Kp}\underline{F}_{t}\underline{F}_{t}^{\prime }e_{h,Kp}\right] }\right)
^{3} \\
&&\left( \text{by Lo\`{e}ve's }c_{r}\text{ inequality and by the CS
inequality}\right) \\
&\leq &\frac{1}{2}\left\vert e_{\ell ,d}^{\prime }A_{YY}A_{YY}^{\prime
}e_{\ell ,d}\right\vert ^{3}E\left\Vert \underline{Y}_{t}\right\Vert
_{2}^{6}+\frac{1}{2}E\left\Vert \underline{F}_{t}\right\Vert
_{2}^{6}+4\left( E\left\Vert \underline{Y}_{t}\right\Vert _{2}^{2}\right) ^{%
\frac{{\large 3}}{{\large 2}}}\left( e_{\ell ,d}^{\prime
}A_{YY}A_{YY}^{\prime }e_{\ell ,d}\right) ^{\frac{{\large 3}}{{\large 2}}%
}\left( E\left\Vert \underline{F}_{t}\right\Vert _{2}^{2}\right) ^{\frac{%
{\large 3}}{{\large 2}}} \\
&\leq &\frac{1}{2}\left\Vert e_{\ell ,d}\right\Vert _{2}^{6}\left(
C^{\dagger }\right) ^{6}\phi _{\max }^{6}E\left\Vert \underline{Y}%
_{t}\right\Vert _{2}^{6}+\frac{1}{2}E\left\Vert \underline{F}_{t}\right\Vert
_{2}^{6}+4\left( E\left\Vert \underline{Y}_{t}\right\Vert _{2}^{2}\right) ^{%
\frac{{\large 3}}{{\large 2}}}\left\Vert e_{\ell ,d}\right\Vert
_{2}^{3}\left( C^{\dagger }\right) ^{3}\phi _{\max }^{3}\left( E\left\Vert 
\underline{F}_{t}\right\Vert _{2}^{2}\right) ^{\frac{{\large 3}}{{\large 2}}}
\\
&=&\frac{1}{2}\left( C^{\dagger }\right) ^{6}\phi _{\max }^{6}E\left\Vert 
\underline{Y}_{t}\right\Vert _{2}^{6}+\frac{1}{2}E\left\Vert \underline{F}%
_{t}\right\Vert _{2}^{6}+4\left( E\left\Vert \underline{Y}_{t}\right\Vert
_{2}^{2}\right) ^{\frac{{\large 3}}{{\large 2}}}\left( C^{\dagger }\right)
^{3}\phi _{\max }^{3}\left( E\left\Vert \underline{F}_{t}\right\Vert
_{2}^{2}\right) ^{\frac{{\large 3}}{{\large 2}}} \\
&&\left( \text{since }\left\Vert e_{\ell ,d}\right\Vert _{2}=1\text{ for
every }\ell \in \left\{ 1,...,d\right\} \text{ given that }e_{\ell ,d}\text{%
's are elementary vectors}\right) \\
&\leq &C_{4}
\end{eqnarray*}%
for some positive constant $C_{4}\geq \left( 1/2\right) \left( C^{\dagger
}\right) ^{6}\phi _{\max }^{6}E\left\Vert \underline{Y}_{t}\right\Vert
_{2}^{6}+\left( 1/2\right) E\left\Vert \underline{F}_{t}\right\Vert _{2}^{6}$

\noindent $+4\left( E\left\Vert \underline{Y}_{t}\right\Vert _{2}^{2}\right)
^{\frac{{\large 3}}{{\large 2}}}\left( C^{\dagger }\right) ^{3}\phi _{\max
}^{3}\left( E\left\Vert \underline{F}_{t}\right\Vert _{2}^{2}\right) ^{\frac{%
{\large 3}}{{\large 2}}}$ which exists in light of Lemma OA-5 and the fact
that $0<\phi _{\max }<1$ given Assumption 2-1. In a similar way, we can also
show that there exists a positive constant $C_{5}$ such that 
\begin{eqnarray*}
&&E\left\vert e_{h,Kp}^{\prime }\left( \underline{F}_{t+m}\underline{Y}%
_{t+m}^{\prime }-E\left[ \underline{F}_{t+m}\underline{Y}_{t+m}^{\prime }%
\right] \right) A_{YY}^{\prime }e_{\ell ,d}\right\vert ^{3} \\
&\leq &\left( 1/2\right) \left\Vert e_{\ell ,d}\right\Vert _{2}^{6}\left(
C^{\dagger }\right) ^{6}\phi _{\max }^{6}E\left\Vert \underline{Y}%
_{t+m}\right\Vert _{2}^{6}+\left( 1/2\right) E\left\Vert \underline{F}%
_{t+m}\right\Vert _{2}^{6} \\
&&+4\left( E\left\Vert \underline{Y}_{t+m}\right\Vert _{2}^{2}\right) ^{%
\frac{{\large 3}}{{\large 2}}}\left\Vert e_{\ell ,d}\right\Vert
_{2}^{3}\left( C^{\dagger }\right) ^{3}\phi _{\max }^{3}\left( E\left\Vert 
\underline{F}_{t+m}\right\Vert _{2}^{2}\right) ^{\frac{{\large 3}}{{\large 2}%
}} \\
&\leq &C_{5}<\infty
\end{eqnarray*}

\noindent Hence,%
\begin{eqnarray}
&&\frac{2}{\tau _{1}^{2}}\dsum\limits_{\ell =1}^{d}\dsum\limits_{t=\left(
r-1\right) \tau +p}^{\left( r-1\right) \tau +\tau
_{1}+p-2}\dsum\limits_{m=1}^{\left( r-1\right) \tau +\tau
_{1}+p-t-1}\left\vert e_{\ell ,d}^{\prime }A_{YY}\right.  \notag \\
&&\text{ \ \ \ \ \ \ \ \ \ \ \ \ \ \ \ \ \ \ \ \ \ \ \ \ \ \ \ \ \ \ }\left.
\times E\left[ \left( \underline{F}_{t}\underline{Y}_{t}^{\prime }-E\left[ 
\underline{F}_{t}\underline{Y}_{t}^{\prime }\right] \right) ^{\prime }\left( 
\underline{F}_{t+m}\underline{Y}_{t+m}^{\prime }-E\left[ \underline{F}_{t+m}%
\underline{Y}_{t+m}^{\prime }\right] \right) \right] A_{YY}^{\prime }e_{\ell
,d}\right\vert  \notag \\
&\leq &\frac{4\left( 2^{\frac{{\large 2}}{{\large 3}}}+1\right) }{\tau
_{1}^{2}}\dsum\limits_{\ell =1}^{d}\dsum\limits_{t=\left( r-1\right) \tau
+p}^{\left( r-1\right) \tau +\tau _{1}+p-2}\dsum\limits_{m=1}^{\left(
r-1\right) \tau +\tau _{1}+p-t-1}\dsum\limits_{h=1}^{Kp}\alpha _{X_{1},m}^{%
\frac{{\Large 1}}{{\Large 3}}}\left( E\left\vert e_{\ell ,d}^{\prime
}A_{YY}\left( \underline{F}_{t}\underline{Y}_{t}^{\prime }-E\left[ 
\underline{F}_{t}\underline{Y}_{t}^{\prime }\right] \right) ^{\prime
}e_{h,Kp}\right\vert ^{3}\right) ^{\frac{{\large 1}}{{\large 3}}}  \notag \\
&&\text{ \ \ \ \ \ \ \ \ \ \ \ \ \ \ \ \ \ \ \ \ \ \ \ \ \ \ \ \ \ \ \ \ \ \
\ \ \ \ \ \ \ \ \ \ \ \ \ \ }\times \left( E\left\vert e_{h,Kp}^{\prime
}\left( \underline{F}_{t+m}\underline{Y}_{t+m}^{\prime }-E\left[ \underline{F%
}_{t+m}\underline{Y}_{t+m}^{\prime }\right] \right) A_{YY}^{\prime }e_{\ell
,d}\right\vert ^{3}\right) ^{1/3}  \notag \\
&\leq &\frac{4dKp\left( 2^{\frac{{\large 2}}{{\large 3}}}+1\right) C_{4}^{%
\frac{{\large 1}}{{\large 3}}}C_{5}^{\frac{{\large 1}}{{\large 3}}}}{\tau
_{1}^{2}}\dsum\limits_{t=\left( r-1\right) \tau +p}^{\left( r-1\right) \tau
+\tau _{1}+p-2}\dsum\limits_{m=1}^{\infty }\left( C_{1}^{\ast }\right) ^{%
\frac{{\Large 1}}{{\Large 3}}}\exp \left\{ -\frac{C_{2}m}{3}\right\}  \notag
\\
&\leq &\frac{C^{\ast }}{\tau _{1}}\left( \frac{\tau _{1}-1}{\tau _{1}}%
\right) \dsum\limits_{m=1}^{\infty }\exp \left\{ -\frac{C_{2}m}{3}\right\} 
\text{ }\left( \text{where }C^{\ast }\geq 4dKp\left( 2^{\frac{{\large 2}}{%
{\large 3}}}+1\right) \left( C_{1}^{\ast }\right) ^{\frac{{\Large 1}}{%
{\Large 3}}}C_{4}^{\frac{{\large 1}}{{\large 3}}}C_{5}^{\frac{{\large 1}}{%
{\large 3}}}\right)  \notag \\
&\leq &\frac{C^{\ast }}{\tau _{1}}\dsum\limits_{m=1}^{\infty }\exp \left\{ -%
\frac{C_{2}m}{3}\right\}  \notag \\
&=&O\left( \frac{1}{\tau _{1}}\right)  \label{FY 2nd moment 2}
\end{eqnarray}%
It then follows from expressions (\ref{PmaxFY bd}), (\ref{FY 2nd moment}), (%
\ref{FY 2nd moment 1}), and (\ref{FY 2nd moment 2}) that%
\begin{eqnarray*}
&&P\left\{ \max_{1\leq \ell \leq d}\max_{i{\large \in H}^{c}}\left\vert 
\frac{1}{q}\dsum\limits_{r=1}^{q}\frac{1}{\tau _{1}}\dsum\limits_{t=\left(
r-1\right) \tau +p}^{\left( r-1\right) \tau +\tau _{1}+p-1}\gamma
_{i}^{\prime }\left( \underline{F}_{t}\underline{Y}_{t}^{\prime }-E\left[ 
\underline{F}_{t}\underline{Y}_{t}^{\prime }\right] \right) \alpha _{YY,\ell
}\right\vert \geq \epsilon \right\} \\
&\leq &\frac{C}{\epsilon ^{2}}\dsum\limits_{\ell =1}^{d}\left( \frac{1}{q}%
\dsum\limits_{r=1}^{q}\frac{1}{\tau _{1}^{2}}\dsum\limits_{t=\left(
r-1\right) \tau +p}^{\left( r-1\right) \tau +\tau
_{1}+p-1}\dsum\limits_{s=\left( r-1\right) \tau +p}^{\left( r-1\right) \tau
+\tau _{1}+p-1}e_{\ell ,d}^{\prime }A_{YY}\right. \\
&&\text{ \ \ \ \ \ \ \ \ \ \ \ \ \ \ \ \ \ \ \ \ \ \ \ \ \ \ \ }\left.
\times E\left[ \left( \underline{F}_{t}\underline{Y}_{t}^{\prime }-E\left[ 
\underline{F}_{t}\underline{Y}_{t}^{\prime }\right] \right) ^{\prime }\left( 
\underline{F}_{s}\underline{Y}_{s}^{\prime }-E\left[ \underline{F}_{s}%
\underline{Y}_{s}^{\prime }\right] \right) \right] A_{YY}^{\prime }e_{\ell
,d}\right) \\
&\leq &\text{ }\frac{C}{\epsilon ^{2}}\dsum\limits_{\ell =1}^{d}\left( \frac{%
1}{q}\dsum\limits_{r=1}^{q}\frac{1}{\tau _{1}^{2}}\dsum\limits_{t=\left(
r-1\right) \tau +p}^{\left( r-1\right) \tau +\tau _{1}+p-1}e_{\ell
,d}^{\prime }A_{YY}E\left[ \left( \underline{F}_{t}\underline{Y}_{t}^{\prime
}-E\left[ \underline{F}_{t}\underline{Y}_{t}^{\prime }\right] \right)
^{\prime }\left( \underline{F}_{t}\underline{Y}_{t}^{\prime }-E\left[ 
\underline{F}_{t}\underline{Y}_{t}^{\prime }\right] \right) \right]
A_{YY}^{\prime }e_{\ell ,d}\right) \\
&&+\frac{C}{\epsilon ^{2}}\dsum\limits_{\ell =1}^{d}\frac{1}{q}%
\dsum\limits_{r=1}^{q}\frac{2}{\tau _{1}^{2}}\dsum\limits_{t=\left(
r-1\right) \tau +p}^{\left( r-1\right) \tau +\tau
_{1}+p-2}\dsum\limits_{m=1}^{\left( r-1\right) \tau +\tau
_{1}+p-t-1}\left\vert e_{\ell ,d}^{\prime }A_{YY}\right. \\
&&\text{ \ \ \ \ \ \ \ \ \ \ \ \ \ \ \ \ \ \ \ \ \ \ \ }\left. \times E\left[
\left( \underline{F}_{t}\underline{Y}_{t}^{\prime }-E\left[ \underline{F}_{t}%
\underline{Y}_{t}^{\prime }\right] \right) ^{\prime }\left( \underline{F}%
_{t+m}\underline{Y}_{t+m}^{\prime }-E\left[ \underline{F}_{t+m}\underline{Y}%
_{t+m}^{\prime }\right] \right) \right] A_{YY}^{\prime }e_{\ell
,d}\right\vert \\
&\leq &\frac{C}{\epsilon ^{2}}\frac{1}{q}\dsum\limits_{r=1}^{q}\frac{%
\overline{C}}{\tau _{1}}+\frac{C}{\epsilon ^{2}}\frac{1}{q}%
\dsum\limits_{r=1}^{q}\frac{C^{\ast }}{\tau _{1}}\dsum\limits_{m=1}^{\infty
}\exp \left\{ -\frac{C_{2}m}{3}\right\} \\
&=&\frac{C\overline{C}}{\epsilon ^{2}}\frac{1}{\tau _{1}}+\frac{CC^{\ast }}{%
\epsilon ^{2}}\frac{1}{\tau _{1}}\dsum\limits_{m=1}^{\infty }\exp \left\{ -%
\frac{C_{2}m}{3}\right\} \\
&=&O\left( \frac{1}{\tau _{1}}\right) +O\left( \frac{1}{\tau _{1}}\right) \\
&=&O\left( \frac{1}{\tau _{1}}\right) =o\left( 1\right) \text{.}
\end{eqnarray*}

Next, to show part (b), note that, for any $\epsilon >0$,%
\begin{eqnarray}
&&P\left\{ \max_{1\leq \ell \leq d}\max_{i{\large \in H}^{{\large c}%
}}\left\vert \frac{1}{q}\dsum\limits_{r=1}^{q}\frac{1}{\tau _{1}}%
\dsum\limits_{t=\left( r-1\right) \tau +p}^{\left( r-1\right) \tau +\tau
_{1}+p-1}\gamma _{i}^{\prime }\left( \underline{F}_{t}\underline{F}%
_{t}^{\prime }-E\left[ \underline{F}_{t}\underline{F}_{t}^{\prime }\right]
\right) \alpha _{YF,\ell }\right\vert \geq \epsilon \right\}  \notag \\
&=&P\left\{ \max_{1\leq \ell \leq d}\max_{i{\large \in H}^{{\large c}%
}}\left( \frac{1}{q}\dsum\limits_{r=1}^{q}\frac{1}{\tau _{1}}%
\dsum\limits_{t=\left( r-1\right) \tau +p}^{\left( r-1\right) \tau +\tau
_{1}+p-1}\gamma _{i}^{\prime }\left( \underline{F}_{t}\underline{F}%
_{t}^{\prime }-E\left[ \underline{F}_{t}\underline{F}_{t}^{\prime }\right]
\right) \alpha _{YF,\ell }\right) ^{2}\geq \epsilon ^{2}\right\}  \notag \\
&\leq &P\left\{ \max_{1\leq \ell \leq d}\max_{i{\large \in H}^{{\large c}}}%
\frac{1}{q}\dsum\limits_{r=1}^{q}\left( \frac{1}{\tau _{1}}%
\dsum\limits_{t=\left( r-1\right) \tau +p}^{\left( r-1\right) \tau +\tau
_{1}+p-1}\gamma _{i}^{\prime }\left( \underline{F}_{t}\underline{F}%
_{t}^{\prime }-E\left[ \underline{F}_{t}\underline{F}_{t}^{\prime }\right]
\right) \alpha _{YF,\ell }\right) ^{2}\geq \epsilon ^{2}\right\} \text{ } 
\notag \\
&&\left( \text{by Jensen's inequality}\right)  \notag \\
&=&P\left\{ \max_{1\leq \ell \leq d}\max_{i{\large \in H}^{{\large c}}}\frac{%
1}{q}\dsum\limits_{r=1}^{q}\left( \gamma _{i}^{\prime }\left[ \frac{1}{\tau
_{1}}\dsum\limits_{t=\left( r-1\right) \tau +p}^{\left( r-1\right) \tau
+\tau _{1}+p-1}\left( \underline{F}_{t}\underline{F}_{t}^{\prime }-E\left[ 
\underline{F}_{t}\underline{F}_{t}^{\prime }\right] \right) \alpha _{YF,\ell
}\right] \right) ^{2}\geq \epsilon ^{2}\right\}  \notag \\
&\leq &P\left\{ \max_{i{\large \in H}^{{\large c}}}\dsum\limits_{\ell =1}^{d}%
\frac{1}{q}\dsum\limits_{r=1}^{q}\left( \gamma _{i}^{\prime }\left[ \frac{1}{%
\tau _{1}}\dsum\limits_{t=\left( r-1\right) \tau +p}^{\left( r-1\right) \tau
+\tau _{1}+p-1}\left( \underline{F}_{t}\underline{F}_{t}^{\prime }-E\left[ 
\underline{F}_{t}\underline{F}_{t}^{\prime }\right] \right) \alpha _{YF,\ell
}\right] \right) ^{2}\geq \epsilon ^{2}\right\}  \notag \\
&\leq &P\left\{ \max_{i{\large \in H}^{{\large c}}}\left\Vert \gamma
_{i}\right\Vert _{2}^{2}\dsum\limits_{\ell =1}^{d}\left( \frac{1}{q}%
\dsum\limits_{r=1}^{q}\left[ \frac{1}{\tau _{1}}\dsum\limits_{t=\left(
r-1\right) \tau +p}^{\left( r-1\right) \tau +\tau _{1}+p-1}\left( \underline{%
F}_{t}\underline{F}_{t}^{\prime }-E\left[ \underline{F}_{t}\underline{F}%
_{t}^{\prime }\right] \right) \alpha _{YF,\ell }\right] \right. \right.
^{\prime }  \notag \\
&&\text{ \ \ \ \ \ \ \ \ \ \ \ \ \ \ \ \ \ \ \ \ \ \ \ \ \ \ \ \ }\left.
\left. \times \left[ \frac{1}{\tau _{1}}\dsum\limits_{t=\left( r-1\right)
\tau +p}^{\left( r-1\right) \tau +\tau _{1}+p-1}\left( \underline{F}_{t}%
\underline{F}_{t}^{\prime }-E\left[ \underline{F}_{t}\underline{F}%
_{t}^{\prime }\right] \right) \alpha _{YF,\ell }\right] \right) \geq
\epsilon ^{2}\right\}  \notag \\
&=&P\left\{ \max_{i{\large \in H}^{{\large c}}}\left\Vert \gamma
_{i}\right\Vert _{2}^{2}\dsum\limits_{\ell =1}^{d}\frac{1}{q}%
\dsum\limits_{r=1}^{q}\frac{1}{\tau _{1}^{2}}\dsum\limits_{t=\left(
r-1\right) \tau +p}^{\left( r-1\right) \tau +\tau
_{1}+p-1}\dsum\limits_{s=\left( r-1\right) \tau +p}^{\left( r-1\right) \tau
+\tau _{1}+p-1}\alpha _{YF,\ell }^{\prime }\right.  \notag \\
&&\text{ \ \ \ \ \ \ \ \ \ \ \ \ \ \ \ \ \ \ \ \ \ \ \ \ \ \ \ \ \ \ \ \ \ \
\ }\left. \times \left( \underline{F}_{t}\underline{F}_{t}^{\prime }-E\left[ 
\underline{F}_{t}\underline{F}_{t}^{\prime }\right] \right) ^{\prime }\left( 
\underline{F}_{s}\underline{F}_{s}^{\prime }-E\left[ \underline{F}_{s}%
\underline{F}_{s}^{\prime }\right] \right) \alpha _{YF,\ell }\geq \epsilon
^{2}\right\}  \notag \\
&\leq &\frac{\max_{i{\large \in H}^{{\large c}}}\left\Vert \gamma
_{i}\right\Vert _{2}^{2}}{\epsilon ^{2}}\dsum\limits_{\ell =1}^{d}\left( 
\frac{1}{q}\dsum\limits_{r=1}^{q}\frac{1}{\tau _{1}^{2}}\dsum\limits_{t=%
\left( r-1\right) \tau +p}^{\left( r-1\right) \tau +\tau
_{1}+p-1}\dsum\limits_{s=\left( r-1\right) \tau +p}^{\left( r-1\right) \tau
+\tau _{1}+p-1}\alpha _{YF,\ell }^{\prime }\right.  \notag \\
&&\text{ \ \ \ \ \ \ \ \ \ \ \ \ \ \ \ \ \ \ \ \ \ \ \ \ \ \ \ \ \ }\left.
\times E\left[ \left( \underline{F}_{t}\underline{F}_{t}^{\prime }-E\left[ 
\underline{F}_{t}\underline{F}_{t}^{\prime }\right] \right) ^{\prime }\left( 
\underline{F}_{s}\underline{F}_{s}^{\prime }-E\left[ \underline{F}_{s}%
\underline{F}_{s}^{\prime }\right] \right) \right] \alpha _{YF,\ell }\right)
\notag \\
&&\left( \text{by Markov's inequality}\right)  \notag \\
&\leq &\frac{C}{\epsilon ^{2}}\dsum\limits_{\ell =1}^{d}\left( \frac{1}{q}%
\dsum\limits_{r=1}^{q}\frac{1}{\tau _{1}^{2}}\dsum\limits_{t=\left(
r-1\right) \tau +p}^{\left( r-1\right) \tau +\tau
_{1}+p-1}\dsum\limits_{s=\left( r-1\right) \tau +p}^{\left( r-1\right) \tau
+\tau _{1}+p-1}\alpha _{YF,\ell }^{\prime }\right.  \notag \\
&&\text{ \ \ \ \ \ \ \ \ \ \ \ \ \ \ \ \ \ \ \ \ \ \ \ \ \ \ \ \ \ \ \ }%
\left. \times E\left[ \left( \underline{F}_{t}\underline{F}_{t}^{\prime }-E%
\left[ \underline{F}_{t}\underline{F}_{t}^{\prime }\right] \right) ^{\prime
}\left( \underline{F}_{s}\underline{F}_{s}^{\prime }-E\left[ \underline{F}%
_{s}\underline{F}_{s}^{\prime }\right] \right) \right] \alpha _{YF,\ell
}\right) \text{ \ }  \label{PmaxFF bd} \\
&&\left( \text{by Assumption 2-5}\right)  \notag
\end{eqnarray}%
Note first that%
\begin{eqnarray}
&&\dsum\limits_{\ell =1}^{d}\left( \frac{1}{q}\dsum\limits_{r=1}^{q}\frac{1}{%
\tau _{1}^{2}}\dsum\limits_{t=\left( r-1\right) \tau +p}^{\left( r-1\right)
\tau +\tau _{1}+p-1}\dsum\limits_{s=\left( r-1\right) \tau +p}^{\left(
r-1\right) \tau +\tau _{1}+p-1}\alpha _{YF,\ell }^{\prime }\right.  \notag \\
&&\text{ \ \ \ \ \ \ \ \ \ \ \ \ \ \ \ \ \ \ \ \ \ \ \ \ \ \ \ }\left.
\times E\left[ \left( \underline{F}_{t}\underline{F}_{t}^{\prime }-E\left[ 
\underline{F}_{t}\underline{F}_{t}^{\prime }\right] \right) ^{\prime }\left( 
\underline{F}_{s}\underline{F}_{s}^{\prime }-E\left[ \underline{F}_{s}%
\underline{F}_{s}^{\prime }\right] \right) \right] \alpha _{YF,\ell }\right)
\notag \\
&=&\dsum\limits_{\ell =1}^{d}\left( \frac{1}{q}\dsum\limits_{r=1}^{q}\frac{1%
}{\tau _{1}^{2}}\dsum\limits_{t=\left( r-1\right) \tau +p}^{\left(
r-1\right) \tau +\tau _{1}+p-1}\alpha _{YF,\ell }^{\prime }E\left[ \left( 
\underline{F}_{t}\underline{F}_{t}^{\prime }-E\left[ \underline{F}_{t}%
\underline{F}_{t}^{\prime }\right] \right) ^{\prime }\left( \underline{F}_{t}%
\underline{F}_{t}^{\prime }-E\left[ \underline{F}_{t}\underline{F}%
_{t}^{\prime }\right] \right) \right] \alpha _{YF,\ell }\right)  \notag \\
&&+\dsum\limits_{\ell =1}^{d}\left( \frac{2}{q}\dsum\limits_{r=1}^{q}\frac{1%
}{\tau _{1}^{2}}\dsum\limits_{t=\left( r-1\right) \tau +p}^{\left(
r-1\right) \tau +\tau _{1}+p-2}\dsum\limits_{m=1}^{\left( r-1\right) \tau
+\tau _{1}+p-t-1}\alpha _{YF,\ell }^{\prime }\right.  \notag \\
&&\text{ \ \ \ \ \ \ \ \ \ \ \ \ \ \ \ \ \ \ \ \ \ \ \ \ }\left. \times E%
\left[ \left( \underline{F}_{t}\underline{F}_{t}^{\prime }-E\left[ 
\underline{F}_{t}\underline{F}_{t}^{\prime }\right] \right) ^{\prime }\left( 
\underline{F}_{t+m}\underline{F}_{t+m}^{\prime }-E\left[ \underline{F}_{t+m}%
\underline{F}_{t+m}^{\prime }\right] \right) \right] \alpha _{YF,\ell
}\right)  \notag \\
&\leq &\dsum\limits_{\ell =1}^{d}\left( \frac{1}{q}\dsum\limits_{r=1}^{q}%
\frac{1}{\tau _{1}^{2}}\dsum\limits_{t=\left( r-1\right) \tau +p}^{\left(
r-1\right) \tau +\tau _{1}+p-1}\alpha _{YF,\ell }^{\prime }E\left[ \left( 
\underline{F}_{t}\underline{F}_{t}^{\prime }-E\left[ \underline{F}_{t}%
\underline{F}_{t}^{\prime }\right] \right) ^{\prime }\left( \underline{F}_{t}%
\underline{F}_{t}^{\prime }-E\left[ \underline{F}_{t}\underline{F}%
_{t}^{\prime }\right] \right) \right] \alpha _{YF,\ell }\right)  \notag \\
&&+\dsum\limits_{\ell =1}^{d}\frac{2}{q}\dsum\limits_{r=1}^{q}\frac{1}{\tau
_{1}^{2}}\dsum\limits_{t=\left( r-1\right) \tau +p}^{\left( r-1\right) \tau
+\tau _{1}+p-2}\dsum\limits_{m=1}^{\left( r-1\right) \tau +\tau
_{1}+p-t-1}\left\vert \alpha _{YF,\ell }^{\prime }\right.  \notag \\
&&\text{ \ \ \ \ \ \ \ \ \ \ \ \ \ \ \ \ \ \ \ \ \ \ }\left. \times E\left[
\left( \underline{F}_{t}\underline{F}_{t}^{\prime }-E\left[ \underline{F}_{t}%
\underline{F}_{t}^{\prime }\right] \right) ^{\prime }\left( \underline{F}%
_{t+m}\underline{F}_{t+m}^{\prime }-E\left[ \underline{F}_{t+m}\underline{F}%
_{t+m}^{\prime }\right] \right) \right] \alpha _{YF,\ell }\right\vert
\label{FF 2nd moment}
\end{eqnarray}%
Consider the first term on the majorant side of expression (\ref{FF 2nd
moment}), whose order of magnitude we can analyze as follows%
\begin{eqnarray}
&&\dsum\limits_{\ell =1}^{d}\left( \frac{1}{q}\dsum\limits_{r=1}^{q}\frac{1}{%
\tau _{1}^{2}}\dsum\limits_{t=\left( r-1\right) \tau +p}^{\left( r-1\right)
\tau +\tau _{1}+p-1}\alpha _{YF,\ell }^{\prime }E\left[ \left( \underline{F}%
_{t}\underline{F}_{t}^{\prime }-E\left[ \underline{F}_{t}\underline{F}%
_{t}^{\prime }\right] \right) ^{\prime }\left( \underline{F}_{t}\underline{F}%
_{t}^{\prime }-E\left[ \underline{F}_{t}\underline{F}_{t}^{\prime }\right]
\right) \right] \alpha _{YF,\ell }\right)  \notag \\
&=&\dsum\limits_{\ell =1}^{d}\left( \frac{1}{q\tau _{1}^{2}}%
\dsum\limits_{r=1}^{q}\dsum\limits_{t=\left( r-1\right) \tau +p}^{\left(
r-1\right) \tau +\tau _{1}+p-1}e_{\ell ,d}^{\prime }A_{YF}E\left[ \left( 
\underline{F}_{t}\underline{F}_{t}^{\prime }-E\left[ \underline{F}_{t}%
\underline{F}_{t}^{\prime }\right] \right) ^{\prime }\left( \underline{F}_{t}%
\underline{F}_{t}^{\prime }-E\left[ \underline{F}_{t}\underline{F}%
_{t}^{\prime }\right] \right) \right] A_{YF}^{\prime }e_{\ell ,d}\right) 
\notag \\
&=&\dsum\limits_{\ell =1}^{d}\frac{1}{q\tau _{1}^{2}}\dsum\limits_{r=1}^{q}%
\left( \dsum\limits_{t=\left( r-1\right) \tau +p}^{\left( r-1\right) \tau
+\tau _{1}+p-1}\left\{ e_{\ell ,d}^{\prime }A_{YF}E\left[ \underline{F}_{t}%
\underline{F}_{t}^{\prime }\underline{F}_{t}\underline{F}_{t}^{\prime }%
\right] A_{YF}^{\prime }e_{\ell ,d}-e_{\ell ,d}^{\prime }A_{YF}E\left[ 
\underline{F}_{t}\underline{F}_{t}^{\prime }\right] E\left[ \underline{F}_{t}%
\underline{F}_{t}^{\prime }\right] A_{YF}^{\prime }e_{\ell ,d}\right\}
\right)  \notag \\
&\leq &\dsum\limits_{\ell =1}^{d}\frac{1}{q\tau _{1}^{2}}\dsum%
\limits_{r=1}^{q}\dsum\limits_{t=\left( r-1\right) \tau +p}^{\left(
r-1\right) \tau +\tau _{1}+p-1}E\left[ \left\Vert \underline{F}%
_{t}\right\Vert _{2}^{2}\left( e_{\ell ,d}^{\prime }A_{YF}\underline{F}%
_{t}\right) ^{2}\right]  \notag \\
&\leq &\dsum\limits_{\ell =1}^{d}\frac{1}{q\tau _{1}^{2}}\dsum%
\limits_{r=1}^{q}\dsum\limits_{t=\left( r-1\right) \tau +p}^{\left(
r-1\right) \tau +\tau _{1}+p-1}\sqrt{E\left[ \left\Vert \underline{F}%
_{t}\right\Vert _{2}^{4}\right] }\sqrt{E\left( e_{\ell ,d}^{\prime }A_{YF}%
\underline{F}_{t}\underline{F}_{t}^{\prime }A_{YF}^{\prime }e_{\ell
,d}\right) ^{2}}\text{ }\left( \text{by CS inequality}\right)  \notag \\
&\leq &\dsum\limits_{\ell =1}^{d}\frac{1}{q\tau _{1}^{2}}\dsum%
\limits_{r=1}^{q}\dsum\limits_{t=\left( r-1\right) \tau +p}^{\left(
r-1\right) \tau +\tau _{1}+p-1}\sqrt{E\left[ \left\Vert \underline{F}%
_{t}\right\Vert _{2}^{4}\right] }\sqrt{E\left[ \left\Vert \underline{F}%
_{t}\right\Vert _{2}^{4}\right] }\sqrt{\left( e_{\ell ,d}^{\prime
}A_{YF}A_{YF}^{\prime }e_{\ell ,d}\right) ^{2}}  \notag \\
&\leq &\frac{1}{q\tau _{1}^{2}}\dsum\limits_{\ell
=1}^{d}\dsum\limits_{r=1}^{q}\dsum\limits_{t=\left( r-1\right) \tau
+p}^{\left( r-1\right) \tau +\tau _{1}+p-1}\sqrt{E\left[ \left\Vert 
\underline{F}_{t}\right\Vert _{2}^{4}\right] }\sqrt{E\left[ \left\Vert 
\underline{F}_{t}\right\Vert _{2}^{4}\right] }\left\Vert A_{YF}\right\Vert
_{2}^{2}\sqrt{\left( e_{\ell ,d}^{\prime }e_{\ell ,d}\right) ^{2}}  \notag \\
&\leq &\frac{\left( C^{\dagger }\right) ^{2}}{q\tau _{1}^{2}}%
\dsum\limits_{\ell =1}^{d}\dsum\limits_{r=1}^{q}\dsum\limits_{t=\left(
r-1\right) \tau +p}^{\left( r-1\right) \tau +\tau _{1}+p-1}E\left[
\left\Vert \underline{F}_{t}\right\Vert _{2}^{4}\right] \phi _{\max }^{2} 
\notag \\
&&\left( \text{by part (b) of Lemma OA-7 and by the fact that }e_{\ell ,d}%
\text{ is an elementary vector}\right)  \notag \\
&\leq &\frac{\overline{C}}{\tau _{1}}=O\left( \frac{1}{\tau _{1}}\right) 
\text{. }  \label{FF 2nd moment 1}
\end{eqnarray}%
for some positive constant $\overline{C}\geq d\left( C^{\dagger }\right)
^{2}E\left[ \left\Vert \underline{F}_{t}\right\Vert _{2}^{4}\right] \phi
_{\max }^{2}$, which exists in light of Lemma OA-5 and the fact that $0<\phi
_{\max }<1$ given Assumption 2-1.

To analyze the second term on the right-hand side of expression (\ref{FF 2nd
moment}), note first that by Lemma OA-11, $\left\{ F_{t}\right\} $ is $\beta 
$-mixing with $\beta $ mixing coefficient satisfying 
\begin{equation*}
\beta _{F}\left( m\right) \leq C_{1}\exp \left\{ -C_{2}m\right\} \text{ for
some positive constants }C_{1}\text{ and }C_{2}\text{.}
\end{equation*}%
Since $\alpha _{F,m}\leq \beta _{F}\left( m\right) $, it follows that $F_{t}$
is $\alpha $-mixing as well, with $\alpha $ mixing coefficient satisfying%
\begin{equation*}
\alpha _{F,m}\leq C_{1}\exp \left\{ -C_{2}m\right\}
\end{equation*}

\noindent Moreover, by applying part (b) of Lemma OA-2, we further deduce
that $X_{2t}=\underline{F}_{t}\underline{F}_{t}^{\prime }A_{YF}^{\prime
}e_{\ell ,d}$ is also $\alpha $-mixing with $\alpha $ mixing coefficient
satisfying%
\begin{eqnarray*}
\alpha _{X_{2},m} &\leq &C_{1}\exp \left\{ -C_{2}\left( m-p+1\right) \right\}
\\
&\leq &C_{1}^{\ast }\exp \left\{ -C_{2}m\right\}
\end{eqnarray*}%
for some positive constant $C_{1}^{\ast }\geq C_{1}\exp \left\{ C_{2}\left(
p-1\right) \right\} $. Hence, we can apply Lemma OA-3 with $p=3$ and $r=3$
to obtain

\begin{eqnarray*}
&&\left\vert \alpha _{YF,\ell }^{\prime }E\left[ \left( \underline{F}_{t}%
\underline{F}_{t}^{\prime }-E\left[ \underline{F}_{t}\underline{F}%
_{t}^{\prime }\right] \right) ^{\prime }\left( \underline{F}_{t+m}\underline{%
F}_{t+m}^{\prime }-E\left[ \underline{F}_{t+m}\underline{F}_{t+m}^{\prime }%
\right] \right) \right] \alpha _{YF,\ell }\right\vert \\
&=&\left\vert e_{\ell ,d}^{\prime }A_{YF}E\left[ \left( \underline{F}_{t}%
\underline{F}_{t}^{\prime }-E\left[ \underline{F}_{t}\underline{F}%
_{t}^{\prime }\right] \right) ^{\prime }\left( \underline{F}_{t+m}\underline{%
F}_{t+m}^{\prime }-E\left[ \underline{F}_{t+m}\underline{F}_{t+m}^{\prime }%
\right] \right) \right] A_{YF}^{\prime }e_{\ell ,d}\right\vert \\
&=&\left\vert \dsum\limits_{h=1}^{Kp}e_{\ell ,d}^{\prime }A_{YF}E\left[
\left( \underline{F}_{t}\underline{F}_{t}^{\prime }-E\left[ \underline{F}_{t}%
\underline{F}_{t}^{\prime }\right] \right) ^{\prime
}e_{h,Kp}e_{h,Kp}^{\prime }\left( \underline{F}_{t+m}\underline{F}%
_{t+m}^{\prime }-E\left[ \underline{F}_{t+m}\underline{F}_{t+m}^{\prime }%
\right] \right) \right] A_{YF}^{\prime }e_{\ell ,d}\right\vert \\
&\leq &\dsum\limits_{h=1}^{Kp}\left\{ 2\left( 2^{\frac{{\large 2}}{{\large 3}%
}}+1\right) \alpha _{X_{2},m}^{\frac{{\Large 1}}{{\Large 3}}}\left(
E\left\vert e_{\ell ,d}^{\prime }A_{YF}\left( \underline{F}_{t}\underline{F}%
_{t}^{\prime }-E\left[ \underline{F}_{t}\underline{F}_{t}^{\prime }\right]
\right) ^{\prime }e_{h,Kp}\right\vert ^{3}\right) ^{\frac{{\large 1}}{%
{\large 3}}}\right. \\
&&\text{ \ \ \ \ \ }\left. \times \left( E\left\vert e_{h,Kp}^{\prime
}\left( \underline{F}_{t+m}\underline{F}_{t+m}^{\prime }-E\left[ \underline{F%
}_{t+m}\underline{F}_{t+m}^{\prime }\right] \right) A_{YF}^{\prime }e_{\ell
,d}\right\vert ^{3}\right) ^{1/3}\right\}
\end{eqnarray*}%
where $\alpha _{X_{2},m}$ denotes the alpha mixing coefficient for the
process $\left\{ X_{2t}\right\} $ and where, by our previous calculations, 
\begin{equation*}
\alpha _{X_{2},m}^{\frac{{\Large 1}}{{\Large 3}}}\leq \left( C_{1}^{\ast
}\right) ^{\frac{{\Large 1}}{{\Large 3}}}\exp \left\{ -\frac{C_{2}m}{3}%
\right\} \text{ for all }m\text{ sufficiently large,}
\end{equation*}%
It further follows that there exists a positive constant $C_{3}$ such that%
\begin{eqnarray*}
\dsum\limits_{m=1}^{\infty }\alpha _{X_{2},m}^{\frac{{\Large 1}}{{\Large 3}}%
} &\leq &\left( C_{1}^{\ast }\right) ^{\frac{{\Large 1}}{{\Large 3}}%
}\dsum\limits_{m=1}^{\infty }\exp \left\{ -\frac{C_{2}m}{3}\right\} \\
&\leq &\left( C_{1}^{\ast }\right) ^{\frac{{\Large 1}}{{\Large 3}}%
}\dsum\limits_{m=0}^{\infty }\exp \left\{ -\frac{C_{2}m}{3}\right\} \\
&=&\left( C_{1}^{\ast }\right) ^{\frac{{\Large 1}}{{\Large 3}}}\left[ 1-\exp
\left\{ -\frac{C_{2}}{3}\right\} \right] ^{-1} \\
&\leq &C_{3}
\end{eqnarray*}

Next, note that%
\begin{eqnarray*}
&&E\left\vert e_{\ell ,d}^{\prime }A_{YF}\left( \underline{F}_{t}\underline{F%
}_{t}^{\prime }-E\left[ \underline{F}_{t}\underline{F}_{t}^{\prime }\right]
\right) ^{\prime }e_{h,Kp}\right\vert ^{3} \\
&\leq &2^{2}\left\{ E\left\vert e_{\ell ,d}^{\prime }A_{YF}\underline{F}_{t}%
\underline{F}_{t}^{\prime }e_{h,Kp}\right\vert ^{3}+\left\vert E\left[
e_{\ell ,d}^{\prime }A_{YF}\underline{F}_{t}\underline{F}_{t}^{\prime
}e_{h,Kp}\right] \right\vert ^{3}\right\} \left( \text{by Lo\`{e}ve's }c_{r}%
\text{ inequality}\right) \\
&\leq &2^{2}\left\{ E\left\vert e_{\ell ,d}^{\prime }A_{YF}\underline{F}_{t}%
\underline{F}_{t}^{\prime }e_{h,Kp}\right\vert ^{3}+\left( E\left[
\left\vert e_{\ell ,d}^{\prime }A_{YF}\underline{F}_{t}\underline{F}%
_{t}^{\prime }e_{h,Kp}\right\vert \right] \right) ^{3}\right\} \text{ }%
\left( \text{by Jensen's inequality}\right) \\
&\leq &2^{2}\left\{ E\left\vert \frac{e_{\ell ,d}^{\prime }A_{YF}\underline{F%
}_{t}\underline{F}_{t}^{\prime }A_{YF}^{\prime }e_{\ell ,d}}{2}+\frac{%
e_{h,Kp}^{\prime }\underline{F}_{t}\underline{F}_{t}^{\prime }e_{h,Kp}}{2}%
\right\vert ^{3}+\left( E\left[ \left\vert e_{\ell ,d}^{\prime }A_{YF}%
\underline{F}_{t}\underline{F}_{t}^{\prime }e_{h,Kp}\right\vert \right]
\right) ^{3}\right\} \\
&\leq &\frac{4}{8}\left[ E\left\vert e_{\ell ,d}^{\prime }A_{YF}\underline{F}%
_{t}\underline{F}_{t}^{\prime }A_{YF}^{\prime }e_{\ell ,d}\right\vert
^{3}+E\left\vert e_{h,Kp}^{\prime }\underline{F}_{t}\underline{F}%
_{t}^{\prime }e_{h,Kp}^{\prime }\right\vert ^{3}\right] \\
&&+4\left( \sqrt{E\left[ e_{\ell ,d}^{\prime }A_{YF}\underline{F}_{t}%
\underline{F}_{t}^{\prime }A_{YF}^{\prime }e_{\ell ,d}\right] }\sqrt{E\left[
e_{h,Kp}\underline{F}_{t}\underline{F}_{t}^{\prime }e_{h,Kp}\right] }\right)
^{3} \\
&&\left( \text{by Lo\`{e}ve's }c_{r}\text{ inequality and by the CS
inequality}\right) \\
&\leq &\frac{1}{2}\left\vert e_{\ell ,d}^{\prime }A_{YF}A_{YF}^{\prime
}e_{\ell ,d}\right\vert ^{3}E\left\Vert \underline{F}_{t}\right\Vert
_{2}^{6}+\frac{1}{2}E\left\Vert \underline{F}_{t}\right\Vert
_{2}^{6}+4\left( E\left\Vert \underline{F}_{t}\right\Vert _{2}^{2}\right)
^{3}\left( e_{\ell ,d}^{\prime }A_{YF}A_{YF}^{\prime }e_{\ell ,d}\right) ^{%
\frac{{\large 3}}{{\large 2}}} \\
&\leq &\frac{1}{2}\left\Vert e_{\ell ,d}\right\Vert _{2}^{6}\left(
C^{\dagger }\right) ^{6}\phi _{\max }^{6}E\left\Vert \underline{F}%
_{t}\right\Vert _{2}^{6}+\frac{1}{2}E\left\Vert \underline{F}_{t}\right\Vert
_{2}^{6}+4\left( E\left\Vert \underline{F}_{t}\right\Vert _{2}^{2}\right)
^{3}\left\Vert e_{\ell ,d}\right\Vert _{2}^{3}\left( C^{\dagger }\right)
^{3}\phi _{\max }^{3} \\
&=&\frac{1}{2}\left( C^{\dagger }\right) ^{6}\phi _{\max }^{6}E\left\Vert 
\underline{F}_{t}\right\Vert _{2}^{6}+\frac{1}{2}E\left\Vert \underline{F}%
_{t}\right\Vert _{2}^{6}+4\left( E\left\Vert \underline{F}_{t}\right\Vert
_{2}^{2}\right) ^{3}\left( C^{\dagger }\right) ^{3}\phi _{\max }^{3} \\
&&\left( \text{since }\left\Vert e_{\ell ,d}\right\Vert _{2}=1\text{ for
every }\ell \in \left\{ 1,...,d\right\} \text{ given that }e_{\ell ,d}\text{%
's are elementary vectors}\right) \\
&\leq &C_{6}
\end{eqnarray*}%
for some positive constant $C_{6}\geq \left( 1/2\right) \left( C^{\dagger
}\right) ^{6}\phi _{\max }^{6}E\left\Vert \underline{F}_{t}\right\Vert
_{2}^{6}+\left( 1/2\right) E\left\Vert \underline{F}_{t}\right\Vert
_{2}^{6}+4\left( E\left\Vert \underline{F}_{t}\right\Vert _{2}^{2}\right)
^{3}\left( C^{\dagger }\right) ^{3}\phi _{\max }^{3}$ which exists in light
of Lemma OA-5 and the fact that $0<\phi _{\max }<1$ given Assumption 2-1. In
a similar way, we can also show that there exists a positive constant $C_{7}$
such that 
\begin{eqnarray*}
&&E\left\vert e_{h,Kp}^{\prime }\left( \underline{F}_{t+m}\underline{F}%
_{t+m}^{\prime }-E\left[ \underline{F}_{t+m}\underline{F}_{t+m}^{\prime }%
\right] \right) A_{YY}^{\prime }e_{\ell ,d}\right\vert ^{3} \\
&\leq &\frac{1}{2}\left\Vert e_{\ell ,d}\right\Vert _{2}^{6}\left(
C^{\dagger }\right) ^{6}\phi _{\max }^{6}E\left\Vert \underline{F}%
_{t+m}\right\Vert _{2}^{6}+\frac{1}{2}E\left\Vert \underline{F}%
_{t+m}\right\Vert _{2}^{6} \\
&&+4\left( E\left\Vert \underline{F}_{t+m}\right\Vert _{2}^{2}\right)
^{3}\left\Vert e_{\ell ,d}\right\Vert _{2}^{3}\left( C^{\dagger }\right)
^{3}\phi _{\max }^{3} \\
&\leq &C_{7}<\infty
\end{eqnarray*}

\noindent Hence,%
\begin{eqnarray}
&&\frac{2}{\tau _{1}^{2}}\dsum\limits_{\ell =1}^{d}\dsum\limits_{t=\left(
r-1\right) \tau +p}^{\left( r-1\right) \tau +\tau
_{1}+p-2}\dsum\limits_{m=1}^{\left( r-1\right) \tau +\tau
_{1}+p-t-1}\left\vert e_{\ell ,d}^{\prime }A_{YF}\right.  \notag \\
&&\text{ \ \ \ \ \ \ \ \ \ \ \ \ \ \ \ \ \ \ \ \ \ \ \ }\left. \times E\left[
\left( \underline{F}_{t}\underline{F}_{t}^{\prime }-E\left[ \underline{F}_{t}%
\underline{F}_{t}^{\prime }\right] \right) ^{\prime }\left( \underline{F}%
_{t+m}\underline{F}_{t+m}^{\prime }-E\left[ \underline{F}_{t+m}\underline{F}%
_{t+m}^{\prime }\right] \right) \right] A_{YF}^{\prime }e_{\ell
,d}\right\vert  \notag \\
&\leq &\frac{4\left( 2^{\frac{{\large 2}}{{\large 3}}}+1\right) }{\tau
_{1}^{2}}\dsum\limits_{\ell =1}^{d}\dsum\limits_{t=\left( r-1\right) \tau
+p}^{\left( r-1\right) \tau +\tau _{1}+p-2}\dsum\limits_{m=1}^{\left(
r-1\right) \tau +\tau _{1}+p-t-1}\dsum\limits_{h=1}^{Kp}\left\{ \alpha
_{X_{2},m}^{\frac{{\Large 1}}{{\Large 3}}}\left( E\left\vert e_{\ell
,d}^{\prime }A_{YF}\left( \underline{F}_{t}\underline{F}_{t}^{\prime }-E%
\left[ \underline{F}_{t}\underline{F}_{t}^{\prime }\right] \right) ^{\prime
}e_{h,Kp}\right\vert ^{3}\right) ^{\frac{{\large 1}}{{\large 3}}}\right. 
\notag \\
&&\text{ \ \ \ \ \ \ \ \ \ \ \ \ \ \ \ \ \ \ \ \ \ \ \ \ \ \ \ \ \ \ \ \ \ \
\ \ \ \ \ \ \ \ \ \ \ \ }\left. \text{\ }\times \left( E\left\vert
e_{h,Kp}^{\prime }\left( \underline{F}_{t+m}\underline{F}_{t+m}^{\prime }-E%
\left[ \underline{F}_{t+m}\underline{F}_{t+m}^{\prime }\right] \right)
A_{YF}^{\prime }e_{\ell ,d}\right\vert ^{3}\right) ^{1/3}\right\}  \notag \\
&\leq &\frac{4dKp\left( 2^{\frac{{\large 2}}{{\large 3}}}+1\right) C_{6}^{%
\frac{{\large 1}}{{\large 3}}}C_{7}^{\frac{{\large 1}}{{\large 3}}}}{\tau
_{1}^{2}}\dsum\limits_{t=\left( r-1\right) \tau +p}^{\left( r-1\right) \tau
+\tau _{1}+p-2}\dsum\limits_{m=1}^{\infty }\left( C_{1}^{\ast }\right) ^{%
\frac{{\Large 1}}{{\Large 3}}}\exp \left\{ -\frac{C_{2}m}{3}\right\}  \notag
\\
&\leq &\frac{C^{\ast }}{\tau _{1}}\left( \frac{\tau _{1}-1}{\tau _{1}}%
\right) \dsum\limits_{m=1}^{\infty }\exp \left\{ -\frac{C_{2}m}{3}\right\} 
\text{ }\left( \text{where }C^{\ast }\geq 4dKp\left( 2^{\frac{{\large 2}}{%
{\large 3}}}+1\right) \left( C_{1}^{\ast }\right) ^{\frac{{\Large 1}}{%
{\Large 3}}}C_{6}^{\frac{{\large 1}}{{\large 3}}}C_{7}^{\frac{{\large 1}}{%
{\large 3}}}\right)  \notag \\
&\leq &\frac{C^{\ast }}{\tau _{1}}\dsum\limits_{m=1}^{\infty }\exp \left\{ -%
\frac{C_{2}m}{3}\right\}  \notag \\
&=&O\left( \frac{1}{\tau _{1}}\right)  \label{FF 2nd moment 2}
\end{eqnarray}%
It then follows from expressions (\ref{PmaxFF bd}), (\ref{FF 2nd moment}), (%
\ref{FF 2nd moment 1}), and (\ref{FF 2nd moment 2}) that%
\begin{eqnarray*}
&&P\left\{ \max_{1\leq \ell \leq d}\max_{i{\large \in }H^{{\large c}%
}}\left\vert \frac{1}{q}\dsum\limits_{r=1}^{q}\frac{1}{\tau _{1}}%
\dsum\limits_{t=\left( r-1\right) \tau +p}^{\left( r-1\right) \tau +\tau
_{1}+p-1}\gamma _{i}^{\prime }\left( \underline{F}_{t}\underline{F}%
_{t}^{\prime }-E\left[ \underline{F}_{t}\underline{F}_{t}^{\prime }\right]
\right) \alpha _{YF,\ell }\right\vert \geq \epsilon \right\} \\
&\leq &\frac{C}{\epsilon ^{2}}\dsum\limits_{\ell =1}^{d}\left( \frac{1}{q}%
\dsum\limits_{r=1}^{q}\frac{1}{\tau _{1}^{2}}\dsum\limits_{t=\left(
r-1\right) \tau +p}^{\left( r-1\right) \tau +\tau
_{1}+p-1}\dsum\limits_{s=\left( r-1\right) \tau +p}^{\left( r-1\right) \tau
+\tau _{1}+p-1}\alpha _{YF,\ell }^{\prime }E\left[ \left( \underline{F}_{t}%
\underline{F}_{t}^{\prime }-E\left[ \underline{F}_{t}\underline{F}%
_{t}^{\prime }\right] \right) ^{\prime }\left( \underline{F}_{s}\underline{F}%
_{s}^{\prime }-E\left[ \underline{F}_{s}\underline{F}_{s}^{\prime }\right]
\right) \right] \alpha _{YF,\ell }\right) \\
&\leq &\text{ }\frac{C}{\epsilon ^{2}}\dsum\limits_{\ell =1}^{d}\left( \frac{%
1}{q}\dsum\limits_{r=1}^{q}\frac{1}{\tau _{1}^{2}}\dsum\limits_{t=\left(
r-1\right) \tau +p}^{\left( r-1\right) \tau +\tau _{1}+p-1}\alpha _{YF,\ell
}^{\prime }E\left[ \left( \underline{F}_{t}\underline{F}_{t}^{\prime }-E%
\left[ \underline{F}_{t}\underline{F}_{t}^{\prime }\right] \right) ^{\prime
}\left( \underline{F}_{t}\underline{F}_{t}^{\prime }-E\left[ \underline{F}%
_{t}\underline{F}_{t}^{\prime }\right] \right) \right] \alpha _{YF,\ell
}\right) \\
&&+\frac{C}{\epsilon ^{2}}\dsum\limits_{\ell =1}^{d}\frac{1}{q}%
\dsum\limits_{r=1}^{q}\frac{2}{\tau _{1}^{2}}\dsum\limits_{t=\left(
r-1\right) \tau +p}^{\left( r-1\right) \tau +\tau
_{1}+p-2}\dsum\limits_{m=1}^{\left( r-1\right) \tau +\tau
_{1}+p-t-1}\left\vert \alpha _{YF,\ell }^{\prime }\right. \\
&&\text{ \ \ \ \ \ \ \ \ \ \ \ \ \ \ \ \ \ \ \ \ \ \ \ \ \ \ \ \ \ \ \ \ \ \
\ \ }\left. \times E\left[ \left( \underline{F}_{t}\underline{F}_{t}^{\prime
}-E\left[ \underline{F}_{t}\underline{F}_{t}^{\prime }\right] \right)
^{\prime }\left( \underline{F}_{t+m}\underline{F}_{t+m}^{\prime }-E\left[ 
\underline{F}_{t+m}\underline{F}_{t+m}^{\prime }\right] \right) \right]
\alpha _{YF,\ell }\right\vert \\
&\leq &\frac{C}{\epsilon ^{2}}\frac{1}{q}\dsum\limits_{r=1}^{q}\frac{%
\overline{C}}{\tau _{1}}+\frac{C}{\epsilon ^{2}}\frac{1}{q}%
\dsum\limits_{r=1}^{q}\frac{C^{\ast }}{\tau _{1}}\dsum\limits_{m=1}^{\infty
}\exp \left\{ -\frac{C_{2}m}{3}\right\} \\
&=&\frac{C\overline{C}}{\epsilon ^{2}}\frac{1}{\tau _{1}}+\frac{CC^{\ast }}{%
\epsilon ^{2}}\frac{1}{\tau _{1}}\dsum\limits_{m=1}^{\infty }\exp \left\{ -%
\frac{C_{2}m}{3}\right\} \\
&=&O\left( \frac{1}{\tau _{1}}\right) +O\left( \frac{1}{\tau _{1}}\right) \\
&=&O\left( \frac{1}{\tau _{1}}\right) =o\left( 1\right) \text{.}
\end{eqnarray*}

Now, to show part (c), note that, for any $\epsilon >0$,%
\begin{eqnarray}
&&P\left\{ \max_{1\leq \ell \leq d}\max_{i{\large \in }H^{{\large c}%
}}\left\vert \frac{1}{q}\dsum\limits_{r=1}^{q}\frac{1}{\tau _{1}}%
\dsum\limits_{t=\left( r-1\right) \tau +p}^{\left( r-1\right) \tau +\tau
_{1}+p-1}\gamma _{i}^{\prime }\left( \underline{F}_{t}-E\left[ \underline{F}%
_{t}\right] \right) \mu _{Y,\ell }\right\vert \geq \epsilon \right\}  \notag
\\
&=&P\left\{ \max_{1\leq \ell \leq d}\max_{i{\large \in }H^{{\large c}%
}}\left( \frac{1}{q}\dsum\limits_{r=1}^{q}\frac{1}{\tau _{1}}%
\dsum\limits_{t=\left( r-1\right) \tau +p}^{\left( r-1\right) \tau +\tau
_{1}+p-1}\gamma _{i}^{\prime }\left( \underline{F}_{t}-E\left[ \underline{F}%
_{t}\right] \right) \mu _{Y,\ell }\right) ^{2}\geq \epsilon ^{2}\right\} 
\notag \\
&\leq &P\left\{ \max_{1\leq \ell \leq d}\max_{i{\large \in }H^{{\large c}}}%
\frac{1}{q}\dsum\limits_{r=1}^{q}\left( \frac{1}{\tau _{1}}%
\dsum\limits_{t=\left( r-1\right) \tau +p}^{\left( r-1\right) \tau +\tau
_{1}+p-1}\gamma _{i}^{\prime }\left( \underline{F}_{t}-E\left[ \underline{F}%
_{t}\right] \right) \mu _{Y,\ell }\right) ^{2}\geq \epsilon ^{2}\right\} 
\text{ }\left( \text{by Jensen's inequality}\right)  \notag \\
&=&P\left\{ \max_{1\leq \ell \leq d}\max_{i{\large \in }H^{{\large c}}}\frac{%
1}{q}\dsum\limits_{r=1}^{q}\left( \gamma _{i}^{\prime }\left[ \frac{1}{\tau
_{1}}\dsum\limits_{t=\left( r-1\right) \tau +p}^{\left( r-1\right) \tau
+\tau _{1}+p-1}\left( \underline{F}_{t}-E\left[ \underline{F}_{t}\right]
\right) \mu _{Y,\ell }\right] \right) ^{2}\geq \epsilon ^{2}\right\}  \notag
\\
&\leq &P\left\{ \max_{i{\large \in }H^{{\large c}}}\dsum\limits_{\ell =1}^{d}%
\frac{1}{q}\dsum\limits_{r=1}^{q}\left( \gamma _{i}^{\prime }\left[ \frac{1}{%
\tau _{1}}\dsum\limits_{t=\left( r-1\right) \tau +p}^{\left( r-1\right) \tau
+\tau _{1}+p-1}\left( \underline{F}_{t}-E\left[ \underline{F}_{t}\right]
\right) \mu _{Y,\ell }\right] \right) ^{2}\geq \epsilon ^{2}\right\}  \notag
\\
&\leq &P\left\{ \max_{i{\large \in }H^{{\large c}}}\left\Vert \gamma
_{i}\right\Vert _{2}^{2}\dsum\limits_{\ell =1}^{d}\left( \frac{1}{q}%
\dsum\limits_{r=1}^{q}\left[ \frac{1}{\tau _{1}}\dsum\limits_{t=\left(
r-1\right) \tau +p}^{\left( r-1\right) \tau +\tau _{1}+p-1}\left( \underline{%
F}_{t}-E\left[ \underline{F}_{t}\right] \right) \mu _{Y,\ell }\right]
^{\prime }\right. \right.  \notag \\
&&\text{ \ \ \ \ \ \ \ \ \ \ \ \ \ \ \ \ \ \ \ \ \ \ \ \ \ \ \ \ \ \ \ \ }%
\left. \left. \times \left[ \frac{1}{\tau _{1}}\dsum\limits_{t=\left(
r-1\right) \tau +p}^{\left( r-1\right) \tau +\tau _{1}+p-1}\left( \underline{%
F}_{t}-E\left[ \underline{F}_{t}\right] \right) \mu _{Y,\ell }\right]
\right) \geq \epsilon ^{2}\right\}  \notag \\
&=&P\left\{ \max_{i{\large \in }H^{{\large c}}}\left\Vert \gamma
_{i}\right\Vert _{2}^{2}\dsum\limits_{\ell =1}^{d}\frac{1}{q}%
\dsum\limits_{r=1}^{q}\frac{1}{\tau _{1}^{2}}\dsum\limits_{t=\left(
r-1\right) \tau +p}^{\left( r-1\right) \tau +\tau
_{1}+p-1}\dsum\limits_{s=\left( r-1\right) \tau +p}^{\left( r-1\right) \tau
+\tau _{1}+p-1}\mu _{Y,\ell }\left( \underline{F}_{t}-E\left[ \underline{F}%
_{t}\right] \right) ^{\prime }\left( \underline{F}_{s}-E\left[ \underline{F}%
_{s}\right] \right) \mu _{Y,\ell }\geq \epsilon ^{2}\right\}  \notag \\
&\leq &\frac{\max_{i{\large \in }H^{{\large c}}}\left\Vert \gamma
_{i}\right\Vert _{2}^{2}}{\epsilon ^{2}}\dsum\limits_{\ell =1}^{d}\left( 
\frac{1}{q}\dsum\limits_{r=1}^{q}\frac{1}{\tau _{1}^{2}}\dsum\limits_{t=%
\left( r-1\right) \tau +p}^{\left( r-1\right) \tau +\tau
_{1}+p-1}\dsum\limits_{s=\left( r-1\right) \tau +p}^{\left( r-1\right) \tau
+\tau _{1}+p-1}\mu _{Y,\ell }^{2}E\left[ \left( \underline{F}_{t}-E\left[ 
\underline{F}_{t}\right] \right) ^{\prime }\left( \underline{F}_{s}-E\left[ 
\underline{F}_{s}\right] \right) \right] \right)  \notag \\
&&\left( \text{by Markov's inequality}\right)  \notag \\
&\leq &\frac{C}{\epsilon ^{2}}\dsum\limits_{\ell =1}^{d}\left( \frac{1}{q}%
\dsum\limits_{r=1}^{q}\frac{1}{\tau _{1}^{2}}\dsum\limits_{t=\left(
r-1\right) \tau +p}^{\left( r-1\right) \tau +\tau
_{1}+p-1}\dsum\limits_{s=\left( r-1\right) \tau +p}^{\left( r-1\right) \tau
+\tau _{1}+p-1}\mu _{Y,\ell }^{2}E\left[ \left( \underline{F}_{t}-E\left[ 
\underline{F}_{t}\right] \right) ^{\prime }\left( \underline{F}_{s}-E\left[ 
\underline{F}_{s}\right] \right) \right] \right) \text{ }  \label{PmaxF bd}
\\
&&\left( \text{by Assumption 2-5}\right)  \notag
\end{eqnarray}%
Note that%
\begin{eqnarray}
&&\text{ }\dsum\limits_{\ell =1}^{d}\left( \frac{1}{q}\dsum\limits_{r=1}^{q}%
\frac{1}{\tau _{1}^{2}}\dsum\limits_{t=\left( r-1\right) \tau +p}^{\left(
r-1\right) \tau +\tau _{1}+p-1}\dsum\limits_{s=\left( r-1\right) \tau
+p}^{\left( r-1\right) \tau +\tau _{1}+p-1}\mu _{Y,\ell }^{2}E\left[ \left( 
\underline{F}_{t}-E\left[ \underline{F}_{t}\right] \right) ^{\prime }\left( 
\underline{F}_{s}-E\left[ \underline{F}_{s}\right] \right) \right] \right) 
\notag \\
&=&\dsum\limits_{\ell =1}^{d}\left( \frac{1}{q}\dsum\limits_{r=1}^{q}\frac{1%
}{\tau _{1}^{2}}\dsum\limits_{t=\left( r-1\right) \tau +p}^{\left(
r-1\right) \tau +\tau _{1}+p-1}\mu _{Y,\ell }^{2}E\left[ \left( \underline{F}%
_{t}-E\left[ \underline{F}_{t}\right] \right) ^{\prime }\left( \underline{F}%
_{t}-E\left[ \underline{F}_{t}\right] \right) \right] \right)  \notag \\
&&+\dsum\limits_{\ell =1}^{d}\left( \frac{2}{q}\dsum\limits_{r=1}^{q}\frac{1%
}{\tau _{1}^{2}}\dsum\limits_{t=\left( r-1\right) \tau +p}^{\left(
r-1\right) \tau +\tau _{1}+p-2}\dsum\limits_{m=1}^{\left( r-1\right) \tau
+\tau _{1}+p-t-1}\mu _{Y,\ell }^{2}E\left[ \left( \underline{F}_{t}-E\left[ 
\underline{F}_{t}\right] \right) ^{\prime }\left( \underline{F}_{t+m}-E\left[
\underline{F}_{t+m}\right] \right) \right] \right)  \notag \\
&\leq &\dsum\limits_{\ell =1}^{d}\left( \frac{1}{q}\dsum\limits_{r=1}^{q}%
\frac{1}{\tau _{1}^{2}}\dsum\limits_{t=\left( r-1\right) \tau +p}^{\left(
r-1\right) \tau +\tau _{1}+p-1}\mu _{Y,\ell }^{2}E\left[ \left( \underline{F}%
_{t}-E\left[ \underline{F}_{t}\right] \right) ^{\prime }\left( \underline{F}%
_{t}-E\left[ \underline{F}_{t}\right] \right) \right] \right)  \notag \\
&&+\frac{2}{q\tau _{1}^{2}}\dsum\limits_{r=1}^{q}\dsum\limits_{t=\left(
r-1\right) \tau +p}^{\left( r-1\right) \tau +\tau
_{1}+p-2}\dsum\limits_{m=1}^{\left( r-1\right) \tau +\tau
_{1}+p-t-1}\left\vert E\left[ \left( \underline{F}_{t}-E\left[ \underline{F}%
_{t}\right] \right) ^{\prime }\left( \underline{F}_{t+m}-E\left[ \underline{F%
}_{t+m}\right] \right) \right] \right\vert \dsum\limits_{\ell =1}^{d}\mu
_{Y,\ell }^{2}  \label{F 2nd moment}
\end{eqnarray}%
Consider the first term on the majorant side of expression (\ref{F 2nd
moment}), whose order of magnitude we can analyze as follows%
\begin{eqnarray}
&&\dsum\limits_{\ell =1}^{d}\left( \frac{1}{q}\dsum\limits_{r=1}^{q}\frac{1}{%
\tau _{1}^{2}}\dsum\limits_{t=\left( r-1\right) \tau +p}^{\left( r-1\right)
\tau +\tau _{1}+p-1}\mu _{Y,\ell }^{2}E\left[ \left( \underline{F}_{t}-E%
\left[ \underline{F}_{t}\right] \right) ^{\prime }\left( \underline{F}_{t}-E%
\left[ \underline{F}_{t}\right] \right) \right] \right)  \notag \\
&=&\dsum\limits_{\ell =1}^{d}\frac{1}{q\tau _{1}^{2}}\dsum\limits_{r=1}^{q}%
\left( \dsum\limits_{t=\left( r-1\right) \tau +p}^{\left( r-1\right) \tau
+\tau _{1}+p-1}\mu _{Y,\ell }^{2}\left\{ E\left[ \underline{F}_{t}^{\prime }%
\underline{F}_{t}\right] -E\left[ \underline{F}_{t}\right] ^{\prime }E\left[ 
\underline{F}_{t}\right] \right\} \right)  \notag \\
&\leq &\dsum\limits_{\ell =1}^{d}\frac{1}{q\tau _{1}^{2}}\dsum%
\limits_{r=1}^{q}\dsum\limits_{t=\left( r-1\right) \tau +p}^{\left(
r-1\right) \tau +\tau _{1}+p-1}\mu _{Y,\ell }^{2}E\left[ \left\Vert 
\underline{F}_{t}\right\Vert _{2}^{2}\right]  \notag \\
&=&\frac{1}{q\tau _{1}^{2}}\dsum\limits_{r=1}^{q}\dsum\limits_{t=\left(
r-1\right) \tau +p}^{\left( r-1\right) \tau +\tau _{1}+p-1}E\left[
\left\Vert \underline{F}_{t}\right\Vert _{2}^{2}\right] \dsum\limits_{\ell
=1}^{d}\left( \mu _{Y,\ell }^{2}\right)  \notag \\
&\leq &\frac{1}{q\tau _{1}^{2}}\dsum\limits_{r=1}^{q}\dsum\limits_{t=\left(
r-1\right) \tau +p}^{\left( r-1\right) \tau +\tau _{1}+p-1}E\left[
\left\Vert \underline{F}_{t}\right\Vert _{2}^{2}\right] \left\Vert \mu
_{Y}\right\Vert _{2}^{2}  \notag \\
&\leq &\frac{\overline{C}}{\tau _{1}}=O\left( \frac{1}{\tau _{1}}\right) 
\text{. }  \label{F 2nd moment 1}
\end{eqnarray}%
for some positive constant $\overline{C}\geq \left\Vert \mu _{Y}\right\Vert
_{2}^{2}E\left[ \left\Vert \underline{F}_{t}\right\Vert _{2}^{2}\right] $,
which exists in light of Assumption 2-5 and Lemma OA-5.

To analyze the second term on the right-hand side of expression (\ref{F 2nd
moment}), note first that by the same argument as given for part (b) above,
we can apply Lemma OA-11 to deduce that $\left\{ F_{t}\right\} $ is $\beta $%
-mixing and, thus, also $\alpha $-mixing with $\alpha $ mixing coefficient
satisfying%
\begin{equation*}
\alpha _{F,m}\leq C_{1}\exp \left\{ -C_{2}m\right\}
\end{equation*}

\noindent Hence, we can apply Lemma OA-3 with $p=3$ and $r=3$ to obtain%
\begin{eqnarray*}
&&\left\vert E\left[ \left( \underline{F}_{t}-E\left[ \underline{F}_{t}%
\right] \right) ^{\prime }\left( \underline{F}_{t+m}-E\left[ \underline{F}%
_{t+m}\right] \right) \right] \right\vert \dsum\limits_{\ell =1}^{d}\mu
_{Y,\ell }^{2} \\
&=&\left\vert \dsum\limits_{h=1}^{Kp}E\left[ \left( \underline{F}_{t}-E\left[
\underline{F}_{t}\right] \right) ^{\prime }e_{h,Kp}e_{h,Kp}^{\prime }\left( 
\underline{F}_{t+m}-E\left[ \underline{F}_{t+m}\right] \right) \right]
\right\vert \dsum\limits_{\ell =1}^{d}\mu _{Y,\ell }^{2} \\
&\leq &\dsum\limits_{h=1}^{Kp}2\left( 2^{\frac{{\large 2}}{{\large 3}}%
}+1\right) \alpha _{F,m}^{\frac{{\Large 1}}{{\Large 3}}}\left( E\left\vert
\left( \underline{F}_{t}-E\left[ \underline{F}_{t}\right] \right) ^{\prime
}e_{h,Kp}\right\vert ^{3}\right) ^{\frac{{\large 1}}{{\large 3}}}\left(
E\left\vert e_{h,Kp}^{\prime }\left( \underline{F}_{t+m}-E\left[ \underline{F%
}_{t+m}\right] \right) \right\vert ^{3}\right) ^{1/3}\dsum\limits_{\ell
=1}^{d}\mu _{Y,\ell }^{2}
\end{eqnarray*}%
Moreover, there exists a positive constant $C_{3}$ such that 
\begin{equation*}
\dsum\limits_{m=1}^{\infty }\alpha _{F,m}^{\frac{{\Large 1}}{{\Large 3}}%
}\leq C_{1}^{\frac{{\Large 1}}{{\Large 3}}}\dsum\limits_{m=1}^{\infty }\exp
\left\{ -\frac{C_{2}m}{3}\right\} =C_{1}^{\frac{{\Large 1}}{{\Large 3}}}%
\left[ 1-\exp \left\{ -\frac{C_{2}}{3}\right\} \right] ^{-1}\leq C_{3}
\end{equation*}%
where again the last inequality stems from the fact that $%
\dsum\nolimits_{m=0}^{\infty }\exp \left\{ -\left( C_{2}m/3\right) \right\} $
is a convergent geometric series given that $0<\exp \left\{ -\left(
C_{2}/3\right) \right\} <1$ for $C_{2}>0$. Next, note that%
\begin{eqnarray*}
&&E\left\vert \left( \underline{F}_{t}-E\left[ \underline{F}_{t}\right]
\right) ^{\prime }e_{h,Kp}\right\vert ^{3} \\
&\leq &2^{2}\left\{ E\left\vert \underline{F}_{t}^{\prime
}e_{h,Kp}\right\vert ^{3}+\left\vert E\left[ \underline{F}_{t}^{\prime
}e_{h,Kp}\right] \right\vert ^{3}\right\} \left( \text{by Lo\`{e}ve's }c_{r}%
\text{ inequality}\right) \\
&\leq &2^{2}\left\{ E\left\vert \underline{F}_{t}^{\prime
}e_{h,Kp}\right\vert ^{3}+\left( E\left[ \left\vert \underline{F}%
_{t}^{\prime }e_{h,Kp}\right\vert \right] \right) ^{3}\right\} \text{ }%
\left( \text{by Jensen's inequality}\right) \\
&\leq &2^{2}\left\{ E\left[ \left( \underline{F}_{t}^{\prime }\underline{F}%
_{t}\right) ^{\frac{{\large 3}}{{\large 2}}}\left( e_{h,Kp}^{\prime
}e_{h,Kp}\right) ^{\frac{{\large 3}}{{\large 2}}}\right] +\left( \sqrt{E%
\left[ \underline{F}_{t}^{\prime }\underline{F}_{t}\right] }\sqrt{%
e_{h,Kp}^{\prime }e_{h,Kp}}\right) ^{3}\right\} \text{ }\left( \text{by CS
inequality}\right) \\
&\leq &4\left\{ E\left[ \left\Vert \underline{F}_{t}\right\Vert _{2}^{3}%
\right] +\left( E\left[ \left\Vert \underline{F}_{t}\right\Vert _{2}^{2}%
\right] \right) ^{\frac{{\large 3}}{{\large 2}}}\right\} \\
&\leq &C_{8}
\end{eqnarray*}%
for some positive constant $C_{8}\geq 4\left\{ E\left[ \left\Vert \underline{%
F}_{t}\right\Vert _{2}^{3}\right] +\left( E\left[ \left\Vert \underline{F}%
_{t}\right\Vert _{2}^{2}\right] \right) ^{\frac{{\large 3}}{{\large 2}}%
}\right\} $ which exists in light of the result given in Lemma OA-5. In a
similar way, we can also show that there exists a positive constant $C_{9}$
such that 
\begin{eqnarray*}
E\left\vert e_{\ell }^{\prime }\left( \underline{F}_{t+m}-E\left[ \underline{%
F}_{t+m}\right] \right) \right\vert ^{3} &\leq &4\left\{ E\left[ \left\Vert 
\underline{F}_{t+m}\right\Vert _{2}^{3}\right] +\left( E\left[ \left\Vert 
\underline{F}_{t+m}\right\Vert _{2}^{2}\right] \right) ^{\frac{{\large 3}}{%
{\large 2}}}\right\} \\
&\leq &C_{9}<\infty
\end{eqnarray*}

\noindent Finally, by Assumption 2-5, there exists a positive constant $%
C_{10}$ such that $\max_{1\leq \ell \leq d}\mu _{Y,\ell }^{2}\leq \left\Vert
\mu _{Y}\right\Vert _{2}^{2}\leq C_{10}<\infty $. Hence,%
\begin{eqnarray}
&&\frac{2}{q}\dsum\limits_{r=1}^{q}\frac{1}{\tau _{1}^{2}}%
\dsum\limits_{t=\left( r-1\right) \tau +p}^{\left( r-1\right) \tau +\tau
_{1}+p-2}\dsum\limits_{m=1}^{\left( r-1\right) \tau +\tau
_{1}+p-t-1}\left\vert E\left[ \left( \underline{F}_{t}-E\left[ \underline{F}%
_{t}\right] \right) ^{\prime }\left( \underline{F}_{t+m}-E\left[ \underline{F%
}_{t+m}\right] \right) \right] \right\vert \dsum\limits_{\ell =1}^{d}\mu
_{Y,\ell }^{2}  \notag \\
&\leq &\dsum\limits_{h=1}^{Kp}\frac{4\left( 2^{\frac{{\large 2}}{{\large 3}}%
}+1\right) }{\tau _{1}^{2}}\left\Vert \mu _{Y}\right\Vert _{2}^{2}  \notag \\
&&\times \frac{1}{q}\dsum\limits_{r=1}^{q}\dsum\limits_{t=\left( r-1\right)
\tau +p}^{\left( r-1\right) \tau +\tau _{1}+p-2}\dsum\limits_{m=1}^{\left(
r-1\right) \tau +\tau _{1}+p-t-1}\left\{ \alpha _{F,m}^{\frac{{\Large 1}}{%
{\Large 3}}}\left( E\left\vert \left( \underline{F}_{t}-E\left[ \underline{F}%
_{t}\right] \right) ^{\prime }e_{h,Kp}\right\vert ^{3}\right) ^{\frac{%
{\large 1}}{{\large 3}}}\right.  \notag \\
&&\text{ \ \ \ \ \ \ \ \ \ \ \ \ \ \ \ \ \ \ \ \ \ \ \ \ \ \ \ \ \ \ \ \ \ \
\ \ \ \ \ \ \ \ \ \ \ \ \ \ \ \ \ \ \ \ }\left. \times \left( E\left\vert
e_{h,Kp}^{\prime }\left( \underline{F}_{t+m}-E\left[ \underline{F}_{t+m}%
\right] \right) \right\vert ^{3}\right) ^{1/3}\right\}  \notag \\
&\leq &\frac{4Kp\left( 2^{\frac{{\large 2}}{{\large 3}}}+1\right) C_{8}^{%
\frac{{\large 1}}{{\large 3}}}C_{9}^{\frac{{\large 1}}{{\large 3}}}C_{10}}{%
\tau _{1}^{2}}\dsum\limits_{t=\left( r-1\right) \tau +p}^{\left( r-1\right)
\tau +\tau _{1}+p-2}\dsum\limits_{m=1}^{\infty }C_{1}^{\frac{{\Large 1}}{%
{\Large 3}}}\exp \left\{ -\frac{C_{2}m}{3}\right\}  \notag \\
&\leq &\frac{C^{\ast }}{\tau _{1}}\left( \frac{\tau _{1}-1}{\tau _{1}}%
\right) \dsum\limits_{m=1}^{\infty }\exp \left\{ -\frac{C_{2}m}{3}\right\} 
\text{ }\left( \text{where }C^{\ast }\geq 4Kp\left( 2^{\frac{{\large 2}}{%
{\large 3}}}+1\right) C_{1}^{\frac{{\Large 1}}{{\Large 3}}}C_{8}^{\frac{%
{\large 1}}{{\large 3}}}C_{9}^{\frac{{\large 1}}{{\large 3}}}C_{10}\right) 
\notag \\
&\leq &\frac{C^{\ast }}{\tau _{1}}\dsum\limits_{m=1}^{\infty }\exp \left\{ -%
\frac{C_{2}m}{3}\right\}  \notag \\
&=&O\left( \frac{1}{\tau _{1}}\right)  \label{F 2nd moment 2}
\end{eqnarray}%
It then follows from expressions (\ref{PmaxF bd}), (\ref{F 2nd moment}), (%
\ref{F 2nd moment 1}), and (\ref{F 2nd moment 2}) that%
\begin{eqnarray*}
&&P\left\{ \max_{1\leq \ell \leq d}\max_{i{\large \in }H^{{\large c}%
}}\left\vert \frac{1}{q}\dsum\limits_{r=1}^{q}\frac{1}{\tau _{1}}%
\dsum\limits_{t=\left( r-1\right) \tau +p}^{\left( r-1\right) \tau +\tau
_{1}+p-1}\gamma _{i}^{\prime }\left( \underline{F}_{t}-E\left[ \underline{F}%
_{t}\right] \right) \mu _{Y,\ell }\right\vert \geq \epsilon \right\} \\
&\leq &\text{ }\frac{C}{\epsilon ^{2}}\dsum\limits_{\ell =1}^{d}\left( \frac{%
1}{q}\dsum\limits_{r=1}^{q}\frac{1}{\tau _{1}^{2}}\dsum\limits_{t=\left(
r-1\right) \tau +p}^{\left( r-1\right) \tau +\tau
_{1}+p-1}\dsum\limits_{s=\left( r-1\right) \tau +p}^{\left( r-1\right) \tau
+\tau _{1}+p-1}\mu _{Y,\ell }^{2}E\left[ \left( \underline{F}_{t}-E\left[ 
\underline{F}_{t}\right] \right) ^{\prime }\left( \underline{F}_{s}-E\left[ 
\underline{F}_{s}\right] \right) \right] \right) \\
&\leq &\frac{C}{\epsilon ^{2}}\dsum\limits_{\ell =1}^{d}\left( \frac{1}{q}%
\dsum\limits_{r=1}^{q}\frac{1}{\tau _{1}^{2}}\dsum\limits_{t=\left(
r-1\right) \tau +p}^{\left( r-1\right) \tau +\tau _{1}+p-1}\mu _{Y,\ell
}^{2}E\left[ \left( \underline{F}_{t}-E\left[ \underline{F}_{t}\right]
\right) ^{\prime }\left( \underline{F}_{t}-E\left[ \underline{F}_{t}\right]
\right) \right] \right) \\
&&+\frac{C}{\epsilon ^{2}}\frac{1}{q}\dsum\limits_{r=1}^{q}\frac{2}{\tau
_{1}^{2}}\dsum\limits_{t=\left( r-1\right) \tau +p}^{\left( r-1\right) \tau
+\tau _{1}+p-2}\dsum\limits_{m=1}^{\left( r-1\right) \tau +\tau
_{1}+p-t-1}\left\vert E\left[ \left( \underline{F}_{t}-E\left[ \underline{F}%
_{t}\right] \right) ^{\prime }\left( \underline{F}_{t+m}-E\left[ \underline{F%
}_{t+m}\right] \right) \right] \right\vert \dsum\limits_{\ell =1}^{d}\mu
_{Y,\ell }^{2} \\
&\leq &\frac{C}{\epsilon ^{2}}\frac{1}{q}\dsum\limits_{r=1}^{q}\frac{%
\overline{C}}{\tau _{1}}+\frac{C}{\epsilon ^{2}}\frac{1}{q}%
\dsum\limits_{r=1}^{q}\frac{C^{\ast }}{\tau _{1}}\dsum\limits_{m=1}^{\infty
}\exp \left\{ -\frac{C_{2}m}{3}\right\} \\
&=&\frac{C\overline{C}}{\epsilon ^{2}}\frac{1}{\tau _{1}}+\frac{CC^{\ast }}{%
\epsilon ^{2}}\frac{1}{\tau _{1}}\dsum\limits_{m=1}^{\infty }\exp \left\{ -%
\frac{C_{2}m}{3}\right\} \\
&=&O\left( \frac{1}{\tau _{1}}\right) +O\left( \frac{1}{\tau _{1}}\right) \\
&=&O\left( \frac{1}{\tau _{1}}\right) =o\left( 1\right) \text{.}
\end{eqnarray*}

Turning our attention to part (d), note that, by apply Lo\`{e}ve's $c_{r}$
inequality, we obtain%
\begin{eqnarray*}
&&\max_{1\leq \ell \leq d}\max_{i\in H^{{\large c}}}\frac{1}{q}%
\dsum\limits_{r=1}^{q}\left( \frac{1}{\tau _{1}}\dsum\limits_{t=\left(
r-1\right) \tau +p}^{\left( r-1\right) \tau +\tau _{1}+p-1}\gamma
_{i}^{\prime }\left\{ \left( \underline{F}_{t}-E\left[ \underline{F}_{t}%
\right] \right) \mu _{Y,\ell }+\left( \underline{F}_{t}\underline{Y}%
_{t}^{\prime }-E\left[ \underline{F}_{t}\underline{Y}_{t}^{\prime }\right]
\right) \alpha _{YY,\ell }\right. \right. \\
&&\text{ \ \ \ \ \ \ \ \ \ \ \ \ \ \ \ \ \ \ \ \ \ \ \ \ \ \ \ \ \ \ \ \ \ \
\ \ \ \ \ \ \ }\left. \text{\ }\left. +\left( \underline{F}_{t}\underline{F}%
_{t}^{\prime }-E\left[ \underline{F}_{t}\underline{F}_{t}^{\prime }\right]
\right) \alpha _{YF,\ell }\right\} \right) ^{2} \\
&\leq &3\max_{1\leq \ell \leq d}\max_{i{\large \in }H^{{\large c}}}\frac{1}{q%
}\dsum\limits_{r=1}^{q}\left( \frac{1}{\tau _{1}}\dsum\limits_{t=\left(
r-1\right) \tau +p}^{\left( r-1\right) \tau +\tau _{1}+p-1}\gamma
_{i}^{\prime }\left( \underline{F}_{t}-E\left[ \underline{F}_{t}\right]
\right) \mu _{Y,\ell }\right) ^{2} \\
&&+3\max_{1\leq \ell \leq d}\max_{i{\large \in }H^{{\large c}}}\frac{1}{q}%
\dsum\limits_{r=1}^{q}\left( \frac{1}{\tau _{1}}\dsum\limits_{t=\left(
r-1\right) \tau +p}^{\left( r-1\right) \tau +\tau _{1}+p-1}\gamma
_{i}^{\prime }\left( \underline{F}_{t}\underline{Y}_{t}^{\prime }-E\left[ 
\underline{F}_{t}\underline{Y}_{t}^{\prime }\right] \right) \alpha _{YY,\ell
}\right) ^{2} \\
&&+3\max_{1\leq \ell \leq d}\max_{i{\large \in }H^{{\large c}}}\frac{1}{q}%
\dsum\limits_{r=1}^{q}\left( \frac{1}{\tau _{1}}\dsum\limits_{t=\left(
r-1\right) \tau +p}^{\left( r-1\right) \tau +\tau _{1}+p-1}\gamma
_{i}^{\prime }\left( \underline{F}_{t}\underline{F}_{t}^{\prime }-E\left[ 
\underline{F}_{t}\underline{F}_{t}^{\prime }\right] \right) \alpha _{YF,\ell
}\right) ^{2}
\end{eqnarray*}%
It follows from the arguments given in the proofs of parts (a)-(c) above
that, for any $\epsilon >0$,%
\begin{eqnarray*}
&&P\left\{ \max_{1\leq \ell \leq d}\max_{i{\large \in }H^{{\large c}}}\frac{1%
}{q}\dsum\limits_{r=1}^{q}\left( \frac{1}{\tau _{1}}\dsum\limits_{t=\left(
r-1\right) \tau +p}^{\left( r-1\right) \tau +\tau _{1}+p-1}\gamma
_{i}^{\prime }\left( \underline{F}_{t}\underline{Y}_{t}^{\prime }-E\left[ 
\underline{F}_{t}\underline{Y}_{t}^{\prime }\right] \right) \alpha _{YY,\ell
}\right) ^{2}\geq \epsilon \right\} \text{ } \\
&\leq &\frac{C}{\epsilon ^{2}}\frac{1}{q\tau _{1}^{2}} \\
&&\times \dsum\limits_{\ell =1}^{d}\left(
\dsum\limits_{r=1}^{q}\dsum\limits_{t=\left( r-1\right) \tau +p}^{\left(
r-1\right) \tau +\tau _{1}+p-1}\dsum\limits_{s=\left( r-1\right) \tau
+p}^{\left( r-1\right) \tau +\tau _{1}+p-1}\alpha _{YY,\ell }^{\prime }E 
\left[ \left( \underline{F}_{t}\underline{Y}_{t}^{\prime }-E\left[ 
\underline{F}_{t}\underline{Y}_{t}^{\prime }\right] \right) ^{\prime }\left( 
\underline{F}_{s}\underline{Y}_{s}^{\prime }-E\left[ \underline{F}_{s}%
\underline{Y}_{s}^{\prime }\right] \right) \right] \alpha _{YY,\ell }\right) 
\text{ } \\
&=&o\left( 1\right) \text{,}
\end{eqnarray*}%
\begin{eqnarray*}
&&P\left\{ \max_{1\leq \ell \leq d}\max_{i{\large \in }H^{{\large c}}}\frac{1%
}{q}\dsum\limits_{r=1}^{q}\left( \frac{1}{\tau _{1}}\dsum\limits_{t=\left(
r-1\right) \tau +p}^{\left( r-1\right) \tau +\tau _{1}+p-1}\gamma
_{i}^{\prime }\left( \underline{F}_{t}\underline{F}_{t}^{\prime }-E\left[ 
\underline{F}_{t}\underline{F}_{t}^{\prime }\right] \right) \alpha _{YF,\ell
}\right) ^{2}\geq \epsilon \right\} \text{ } \\
&\leq &\frac{C}{\epsilon ^{2}}\frac{1}{q\tau _{1}^{2}} \\
&&\times \dsum\limits_{\ell =1}^{d}\left(
\dsum\limits_{r=1}^{q}\dsum\limits_{t=\left( r-1\right) \tau +p}^{\left(
r-1\right) \tau +\tau _{1}+p-1}\dsum\limits_{s=\left( r-1\right) \tau
+p}^{\left( r-1\right) \tau +\tau _{1}+p-1}\alpha _{YF,\ell }^{\prime }E 
\left[ \left( \underline{F}_{t}\underline{F}_{t}^{\prime }-E\left[ 
\underline{F}_{t}\underline{F}_{t}^{\prime }\right] \right) ^{\prime }\left( 
\underline{F}_{s}\underline{F}_{s}^{\prime }-E\left[ \underline{F}_{s}%
\underline{F}_{s}^{\prime }\right] \right) \right] \alpha _{YF,\ell }\right)
\\
&=&o\left( 1\right) \text{ }
\end{eqnarray*}%
and%
\begin{eqnarray*}
&&P\left\{ \max_{1\leq \ell \leq d}\max_{i{\large \in }H^{{\large c}}}\frac{1%
}{q}\dsum\limits_{r=1}^{q}\left( \frac{1}{\tau _{1}}\dsum\limits_{t=\left(
r-1\right) \tau +p}^{\left( r-1\right) \tau +\tau _{1}+p-1}\gamma
_{i}^{\prime }\left( \underline{F}_{t}-E\left[ \underline{F}_{t}\right]
\right) \mu _{Y,\ell }\right) ^{2}\geq \epsilon \right\} \\
&\leq &\frac{C}{\epsilon ^{2}}\frac{1}{q\tau _{1}^{2}} \\
&&\times \dsum\limits_{\ell =1}^{d}\left(
\dsum\limits_{r=1}^{q}\dsum\limits_{t=\left( r-1\right) \tau +p}^{\left(
r-1\right) \tau +\tau _{1}+p-1}\dsum\limits_{s=\left( r-1\right) \tau
+p}^{\left( r-1\right) \tau +\tau _{1}+p-1}\mu _{Y,\ell }^{2}E\left[ \left( 
\underline{F}_{t}-E\left[ \underline{F}_{t}\right] \right) ^{\prime }\left( 
\underline{F}_{s}-E\left[ \underline{F}_{s}\right] \right) \right] \right) \\
&=&o\left( 1\right) ,
\end{eqnarray*}%
from which we deduce via the Slutsky's theorem that%
\begin{eqnarray*}
&&\max_{1\leq \ell \leq d}\max_{i\in H^{{\large c}}}\frac{1}{q}%
\dsum\limits_{r=1}^{q}\left( \frac{1}{\tau _{1}}\dsum\limits_{t=\left(
r-1\right) \tau +p}^{\left( r-1\right) \tau +\tau _{1}+p-1}\gamma
_{i}^{\prime }\left\{ \left( \underline{F}_{t}-E\left[ \underline{F}_{t}%
\right] \right) \mu _{Y,\ell }+\left( \underline{F}_{t}\underline{Y}%
_{t}^{\prime }-E\left[ \underline{F}_{t}\underline{Y}_{t}^{\prime }\right]
\right) \alpha _{YY,\ell }\right. \right. \\
&&\text{ \ \ \ \ \ \ \ \ \ \ \ \ \ \ \ \ \ \ \ \ \ \ \ \ \ \ }\left. \left.
+\left( \underline{F}_{t}\underline{F}_{t}^{\prime }-E\left[ \underline{F}%
_{t}\underline{F}_{t}^{\prime }\right] \right) \alpha _{YF,\ell }\right\}
\right) ^{2} \\
&=&o_{p}\left( 1\right)
\end{eqnarray*}%
as required.

To show part (e), note that%
\begin{eqnarray*}
&&\max_{1\leq \ell \leq d}\max_{i\in H^{{\large c}}}\frac{1}{q}%
\dsum\limits_{r=1}^{q}\left( \frac{1}{\tau _{1}}\dsum\limits_{t=\left(
r-1\right) \tau +p}^{\left( r-1\right) \tau +\tau _{1}+p-1}\gamma
_{i}^{\prime }\underline{F}_{t}\left[ \mu _{Y,\ell }+\underline{Y}%
_{t}^{\prime }\alpha _{YY,\ell }+\underline{F}_{t}^{\prime }\alpha _{YF,\ell
}\right] \right) ^{2} \\
&\leq &\max_{1\leq \ell \leq d}\max_{i\in H^{{\large c}}}\frac{1}{q}%
\dsum\limits_{r=1}^{q}\left( \frac{1}{\tau _{1}}\dsum\limits_{t=\left(
r-1\right) \tau +p}^{\left( r-1\right) \tau +\tau _{1}+p-1}\gamma
_{i}^{\prime }\left\{ \left( \underline{F}_{t}-E\left[ \underline{F}_{t}%
\right] \right) \mu _{Y,\ell }+\left( \underline{F}_{t}\underline{Y}%
_{t}^{\prime }-E\left[ \underline{F}_{t}\underline{Y}_{t}^{\prime }\right]
\right) \alpha _{YY,\ell }\right. \right. \\
&&\text{ \ \ \ \ \ \ \ \ \ \ \ \ \ \ \ \ \ \ \ \ \ \ \ \ \ \ \ \ \ \ \ \ \ \
\ \ \ \ \ \ \ \ \ \ \ \ \ }\left. +\left( \underline{F}_{t}\underline{F}%
_{t}^{\prime }-E\left[ \underline{F}_{t}\underline{F}_{t}^{\prime }\right]
\right) \alpha _{YF,\ell }\right\} \\
&&\text{\ \ \ \ \ \ \ \ \ \ \ \ \ \ \ \ \ \ \ \ \ }\left. +\frac{1}{\tau _{1}%
}\dsum\limits_{t=\left( r-1\right) \tau +p}^{\left( r-1\right) \tau +\tau
_{1}+p-1}\gamma _{i}^{\prime }\left\{ E\left[ \underline{F}_{t}\right] \mu
_{Y,\ell }+E\left[ \underline{F}_{t}\underline{Y}_{t}^{\prime }\right]
\alpha _{YY,\ell }+E\left[ \underline{F}_{t}\underline{F}_{t}^{\prime }%
\right] \alpha _{YF,\ell }\right\} \right) ^{2} \\
&\leq &\max_{1\leq \ell \leq d}\max_{i\in H^{{\large c}}}\frac{2}{q}%
\dsum\limits_{r=1}^{q}\left( \frac{1}{\tau _{1}}\dsum\limits_{t=\left(
r-1\right) \tau +p}^{\left( r-1\right) \tau +\tau _{1}+p-1}\gamma
_{i}^{\prime }\left\{ \left( \underline{F}_{t}-E\left[ \underline{F}_{t}%
\right] \right) \mu _{Y,\ell }+\left( \underline{F}_{t}\underline{Y}%
_{t}^{\prime }-E\left[ \underline{F}_{t}\underline{Y}_{t}^{\prime }\right]
\right) \alpha _{YY,\ell }\right. \right. \\
&&\text{ \ \ \ \ \ \ \ \ \ \ \ \ \ \ \ \ \ \ \ \ \ \ \ \ \ }\left. \text{\ }%
\left. +\left( \underline{F}_{t}\underline{F}_{t}^{\prime }-E\left[ 
\underline{F}_{t}\underline{F}_{t}^{\prime }\right] \right) \alpha _{YF,\ell
}\right\} \right) ^{2} \\
&&+\max_{1\leq \ell \leq d}\max_{i\in H^{{\large c}}}\frac{2}{q}%
\dsum\limits_{r=1}^{q}\left( \frac{1}{\tau _{1}}\dsum\limits_{t=\left(
r-1\right) \tau +p}^{\left( r-1\right) \tau +\tau _{1}+p-1}\gamma
_{i}^{\prime }\left\{ E\left[ \underline{F}_{t}\right] \mu _{Y,\ell }+E\left[
\underline{F}_{t}\underline{Y}_{t}^{\prime }\right] \alpha _{YY,\ell }+E%
\left[ \underline{F}_{t}\underline{F}_{t}^{\prime }\right] \alpha _{YF,\ell
}\right\} \right) ^{2}\text{ \ } \\
&&\left( \text{by Lo\`{e}ve's }c_{r}\text{ inequality}\right) \\
&=&o_{p}\left( 1\right) +O\left( 1\right) \text{ } \\
&&\left( \text{applying the results given in part (d) of this lemma and in
Lemma A1 of the main paper}\right) \\
&=&O_{p}\left( 1\right) \text{.}
\end{eqnarray*}

To show part (f), we apply the Cauchy-Schwarz inequality as well as part (d)
of this lemma and Lemma A1 of the main paper to obtain%
\begin{eqnarray*}
&&\max_{1\leq \ell \leq d}\max_{i\in H^{{\large c}}}\left\vert \frac{1}{q}%
\dsum\limits_{r=1}^{q}\left\{ \left( \frac{1}{\tau _{1}}\dsum\limits_{t=%
\left( r-1\right) \tau +p}^{\left( r-1\right) \tau +\tau _{1}+p-1}\left\{
\gamma _{i}^{\prime }\left( \underline{F}_{t}-E\left[ \underline{F}_{t}%
\right] \right) \mu _{Y,\ell }+\gamma _{i}^{\prime }\left( \underline{F}_{t}%
\underline{Y}_{t}^{\prime }-E\left[ \underline{F}_{t}\underline{Y}%
_{t}^{\prime }\right] \right) \alpha _{YY,\ell }\right. \right. \right.
\right. \\
&&\text{ \ \ \ \ \ \ \ \ \ \ \ \ \ \ \ \ \ \ \ \ \ \ \ \ \ \ }\left. \text{\ 
}\left. +\gamma _{i}^{\prime }\left( \underline{F}_{t}\underline{F}%
_{t}^{\prime }-E\left[ \underline{F}_{t}\underline{F}_{t}^{\prime }\right]
\right) \alpha _{YF,\ell }\right\} \right) \\
&&\text{ \ \ \ \ \ \ \ \ \ \ \ \ \ }\left. \text{\ }\left. \times \left( 
\frac{1}{\tau _{1}}\dsum\limits_{t=\left( r-1\right) \tau +p}^{\left(
r-1\right) \tau +\tau _{1}+p-1}\left\{ \gamma _{i}^{\prime }E\left[ 
\underline{F}_{t}\right] \mu _{Y,\ell }+\gamma _{i}^{\prime }E\left[ 
\underline{F}_{t}\underline{Y}_{t}^{\prime }\right] \alpha _{YY,\ell
}+\gamma _{i}^{\prime }E\left[ \underline{F}_{t}\underline{F}_{t}^{\prime }%
\right] \alpha _{YF,\ell }\right\} \right) \right\} \right\vert \\
&\leq &\max_{1\leq \ell \leq d}\max_{i\in H^{{\large c}}}\frac{1}{q}%
\dsum\limits_{r=1}^{q}\left\vert \left( \frac{1}{\tau _{1}}%
\dsum\limits_{t=\left( r-1\right) \tau +p}^{\left( r-1\right) \tau +\tau
_{1}+p-1}\left\{ \gamma _{i}^{\prime }\left( \underline{F}_{t}-E\left[ 
\underline{F}_{t}\right] \right) \mu _{Y,\ell }+\gamma _{i}^{\prime }\left( 
\underline{F}_{t}\underline{Y}_{t}^{\prime }-E\left[ \underline{F}_{t}%
\underline{Y}_{t}^{\prime }\right] \right) \alpha _{YY,\ell }\right. \right.
\right. \\
&&\text{ \ \ \ \ \ \ \ \ \ \ \ \ \ \ \ \ \ \ \ \ \ \ \ \ \ \ \ \ \ \ \ \ \ \
\ \ \ \ \ \ \ \ \ \ \ \ \ }\left. \left. +\gamma _{i}^{\prime }\left( 
\underline{F}_{t}\underline{F}_{t}^{\prime }-E\left[ \underline{F}_{t}%
\underline{F}_{t}^{\prime }\right] \right) \alpha _{YF,\ell }\right\} \right)
\\
&&\text{\ \ \ \ \ \ \ \ \ \ \ \ \ \ \ \ \ \ }\left. \text{\ }\times \left( 
\frac{1}{\tau _{1}}\dsum\limits_{t=\left( r-1\right) \tau +p}^{\left(
r-1\right) \tau +\tau _{1}+p-1}\left\{ \gamma _{i}^{\prime }E\left[ 
\underline{F}_{t}\right] \mu _{Y,\ell }+\gamma _{i}^{\prime }E\left[ 
\underline{F}_{t}\underline{Y}_{t}^{\prime }\right] \alpha _{YY,\ell
}+\gamma _{i}^{\prime }E\left[ \underline{F}_{t}\underline{F}_{t}^{\prime }%
\right] \alpha _{YF,\ell }\right\} \right) \right\vert \\
&\leq &\left[ \max_{1\leq \ell \leq d}\max_{i\in H^{{\large c}}}\frac{1}{q}%
\dsum\limits_{r=1}^{q}\left( \frac{1}{\tau _{1}}\dsum\limits_{t=\left(
r-1\right) \tau +p}^{\left( r-1\right) \tau +\tau _{1}+p-1}\left\{ \gamma
_{i}^{\prime }\left( \underline{F}_{t}-E\left[ \underline{F}_{t}\right]
\right) \mu _{Y,\ell }+\gamma _{i}^{\prime }\left( \underline{F}_{t}%
\underline{Y}_{t}^{\prime }-E\left[ \underline{F}_{t}\underline{Y}%
_{t}^{\prime }\right] \right) \alpha _{YY,\ell }\right. \right. \right. \\
&&\text{ \ \ \ \ \ \ \ \ \ \ \ \ \ \ \ \ \ }\left. \text{\ \ }\left. \text{\ 
}\left. +\gamma _{i}^{\prime }\left( \underline{F}_{t}\underline{F}%
_{t}^{\prime }-E\left[ \underline{F}_{t}\underline{F}_{t}^{\prime }\right]
\right) \alpha _{YF,\ell }\right\} \right) ^{2}\right] ^{1/2} \\
&&\times \left[ \max_{1\leq \ell \leq d}\max_{i\in H^{{\large c}}}\frac{1}{q}%
\dsum\limits_{r=1}^{q}\left( \frac{1}{\tau _{1}}\dsum\limits_{t=\left(
r-1\right) \tau +p}^{\left( r-1\right) \tau +\tau _{1}+p-1}\left\{ \gamma
_{i}^{\prime }E\left[ \underline{F}_{t}\right] \mu _{Y,\ell }+\gamma
_{i}^{\prime }E\left[ \underline{F}_{t}\underline{Y}_{t}^{\prime }\right]
\alpha _{YY,\ell }+\gamma _{i}^{\prime }E\left[ \underline{F}_{t}\underline{F%
}_{t}^{\prime }\right] \alpha _{YF,\ell }\right\} \right) ^{2}\right] ^{1/2}
\\
&=&o_{p}\left( 1\right) O\left( 1\right) \\
&=&o_{p}\left( 1\right) \text{. }
\end{eqnarray*}

\noindent \qquad For part (g), we apply the Cauchy-Schwarz inequality as
well as part (d) of Lemma OA-6 and part (e) of this lemma to obtain%
\begin{eqnarray*}
&&\max_{1\leq \ell \leq d}\max_{i\in H^{{\large c}}}\left\vert \frac{1}{q}%
\dsum\limits_{r=1}^{q}\left( \frac{1}{\tau _{1}}\dsum\limits_{t=\left(
r-1\right) \tau +p}^{\left( r-1\right) \tau +\tau _{1}+p-1}\gamma
_{i}^{\prime }\underline{F}_{t}\left[ \mu _{Y,\ell }+\underline{Y}%
_{t}^{\prime }\alpha _{YY,\ell }+\underline{F}_{t}^{\prime }\alpha _{YF,\ell
}\right] \right) \right. \\
&&\text{ \ \ \ \ \ \ \ \ \ \ \ \ \ \ \ \ \ \ \ \ }\left. \times \left( \frac{%
1}{\tau _{1}}\dsum\limits_{t=\left( r-1\right) \tau +p}^{\left( r-1\right)
\tau +\tau _{1}+p-1}y_{\ell ,t{\LARGE +}1}u_{it}\right) \right\vert \\
&\leq &\max_{1\leq \ell \leq d}\max_{i\in H^{{\large c}}}\frac{1}{q}%
\dsum\limits_{r=1}^{q}\left\vert \left( \frac{1}{\tau _{1}}%
\dsum\limits_{t=\left( r-1\right) \tau +p}^{\left( r-1\right) \tau +\tau
_{1}+p-1}\gamma _{i}^{\prime }\underline{F}_{t}\left[ \mu _{Y,\ell }+%
\underline{Y}_{t}^{\prime }\alpha _{YY,\ell }+\underline{F}_{t}^{\prime
}\alpha _{YF,\ell }\right] \right) \right. \\
&&\text{ \ \ \ \ \ \ \ \ \ \ \ \ \ \ \ \ \ \ \ \ \ \ \ }\left. \times \left( 
\frac{1}{\tau _{1}}\dsum\limits_{t=\left( r-1\right) \tau +p}^{\left(
r-1\right) \tau +\tau _{1}+p-1}y_{\ell ,t{\LARGE +}1}u_{it}\right)
\right\vert \\
&\leq &\sqrt{\max_{1\leq \ell \leq d}\max_{i\in H^{{\large c}}}\frac{1}{q}%
\dsum\limits_{r=1}^{q}\left( \frac{1}{\tau _{1}}\dsum\limits_{t=\left(
r-1\right) \tau +p}^{\left( r-1\right) \tau +\tau _{1}+p-1}\gamma
_{i}^{\prime }\underline{F}_{t}\left[ \mu _{Y,\ell }+\underline{Y}%
_{t}^{\prime }\alpha _{YY,\ell }+\underline{F}_{t}^{\prime }\alpha _{YF,\ell
}\right] \right) ^{2}} \\
&&\times \sqrt{\max_{1\leq \ell \leq d}\max_{i\in H^{{\large c}}}\frac{1}{q}%
\dsum\limits_{r=1}^{q}\left( \frac{1}{\tau _{1}}\dsum\limits_{t=\left(
r-1\right) \tau +p}^{\left( r-1\right) \tau +\tau _{1}+p-1}y_{\ell ,t{\LARGE %
+}1}u_{it}\right) ^{2}} \\
&=&O_{p}\left( 1\right) o_{p}\left( 1\right) \\
&=&o_{p}\left( 1\right)
\end{eqnarray*}

Finally, for part (h), we apply the Cauchy-Schwarz inequality as well as
part (b) of Lemma OA-6 and part (e) of this lemma to obtain%
\begin{eqnarray*}
&&\max_{1\leq \ell \leq d}\max_{i\in H^{{\large c}}}\left\vert \frac{1}{q}%
\dsum\limits_{r=1}^{q}\left( \frac{1}{\tau _{1}}\dsum\limits_{t=\left(
r-1\right) \tau +p}^{\left( r-1\right) \tau +\tau _{1}+p-1}\gamma
_{i}^{\prime }\underline{F}_{t}\left[ \mu _{Y,\ell }+\underline{Y}%
_{t}^{\prime }\alpha _{YY,\ell }+\underline{F}_{t}^{\prime }\alpha _{YF,\ell
}\right] \right) \right. \\
&&\text{ \ \ \ \ \ \ \ \ \ \ \ \ \ \ \ \ \ \ \ \ \ }\left. \times \left( 
\frac{1}{\tau _{1}}\dsum\limits_{t=\left( r-1\right) \tau +p}^{\left(
r-1\right) \tau +\tau _{1}+p-1}\gamma _{i}^{\prime }\underline{F}%
_{t}\varepsilon _{\ell ,t{\LARGE +}1}\right) \right\vert \\
&\leq &\max_{1\leq \ell \leq d}\max_{i\in H^{{\large c}}}\frac{1}{q}%
\dsum\limits_{r=1}^{q}\left\vert \left( \frac{1}{\tau _{1}}%
\dsum\limits_{t=\left( r-1\right) \tau +p}^{\left( r-1\right) \tau +\tau
_{1}+p-1}\gamma _{i}^{\prime }\underline{F}_{t}\left[ \mu _{Y,\ell }+%
\underline{Y}_{t}^{\prime }\alpha _{YY,\ell }+\underline{F}_{t}^{\prime
}\alpha _{YF,\ell }\right] \right) \right. \\
&&\text{ \ \ \ \ \ \ \ \ \ \ \ \ \ \ \ \ \ \ \ \ \ \ }\left. \times \left( 
\frac{1}{\tau _{1}}\dsum\limits_{t=\left( r-1\right) \tau +p}^{\left(
r-1\right) \tau +\tau _{1}+p-1}\gamma _{i}^{\prime }\underline{F}%
_{t}\varepsilon _{\ell ,t{\LARGE +}1}\right) \right\vert \\
&\leq &\sqrt{\max_{1\leq \ell \leq d}\max_{i\in H^{{\large c}}}\frac{1}{q}%
\dsum\limits_{r=1}^{q}\left( \frac{1}{\tau _{1}}\dsum\limits_{t=\left(
r-1\right) \tau +p}^{\left( r-1\right) \tau +\tau _{1}+p-1}\gamma
_{i}^{\prime }\underline{F}_{t}\left[ \mu _{Y,\ell }+\underline{Y}%
_{t}^{\prime }\alpha _{YY,\ell }+\underline{F}_{t}^{\prime }\alpha _{YF,\ell
}\right] \right) ^{2}} \\
&&\times \sqrt{\max_{1\leq \ell \leq d}\max_{i\in H^{{\large c}}}\frac{1}{q}%
\dsum\limits_{r=1}^{q}\left( \frac{1}{\tau _{1}}\dsum\limits_{t=\left(
r-1\right) \tau +p}^{\left( r-1\right) \tau +\tau _{1}+p-1}\gamma
_{i}^{\prime }\underline{F}_{t}\varepsilon _{\ell ,t{\LARGE +}1}\right) ^{2}}
\\
&=&O_{p}\left( 1\right) o_{p}\left( 1\right) \\
&=&o_{p}\left( 1\right) \text{. }\square
\end{eqnarray*}

\medskip

\noindent

\noindent \textbf{Lemma OA-13: }Let $a,b\in \mathbb{R}$ such that $a\geq 0$
and $b\geq 0$. Then,%
\begin{equation*}
\left\vert \sqrt{a}-\sqrt{b}\right\vert \leq \sqrt{\left\vert a-b\right\vert 
}
\end{equation*}%
\textbf{Proof of Lemma OA-13}: Note that%
\begin{eqnarray*}
\left( \sqrt{a}-\sqrt{b}\right) ^{2} &=&a-2\sqrt{a}\sqrt{b}+b \\
&=&\sqrt{a}\left( \sqrt{a}-\sqrt{b}\right) +\sqrt{b}\left( \sqrt{b}-\sqrt{a}%
\right) \\
&\leq &\sqrt{a}\left\vert \sqrt{a}-\sqrt{b}\right\vert +\sqrt{b}\left\vert 
\sqrt{b}-\sqrt{a}\right\vert \\
&=&\left( \sqrt{a}+\sqrt{b}\right) \left\vert \sqrt{a}-\sqrt{b}\right\vert \\
&=&\left\vert \left( \sqrt{a}+\sqrt{b}\right) \left( \sqrt{a}-\sqrt{b}%
\right) \right\vert \\
&=&\left\vert a-b\right\vert
\end{eqnarray*}%
Taking principal square root on both sides, we obtain%
\begin{equation*}
\left\vert \sqrt{a}-\sqrt{b}\right\vert \leq \sqrt{\left\vert a-b\right\vert 
}\text{. }\square
\end{equation*}

\medskip

\noindent \textbf{Lemma OA-14:}%
\begin{equation*}
P\left\{ \dbigcap\limits_{i=1}^{m}A_{i}\right\} \geq
\dsum\limits_{i=1}^{m}P\left( A_{i}\right) -\left( m-1\right)
\end{equation*}%
\textbf{Proof of Lemma OA-14:} \ 
\begin{eqnarray*}
P\left\{ \dbigcap\limits_{i=1}^{m}A_{i}\right\} &=&1-P\left\{ \left(
\dbigcap\limits_{i=1}^{m}A_{i}\right) ^{c}\right\} \\
&=&1-P\left\{ \dbigcup\limits_{i=1}^{m}A_{i}^{c}\right\} \text{ \ (by
DeMorgan's Law)} \\
&\geq &1-\dsum\limits_{i=1}^{m}P\left( A_{i}^{c}\right) \\
&=&1-\dsum\limits_{i=1}^{m}\left[ 1-P\left( A_{i}\right) \right] \\
&=&\dsum\limits_{i=1}^{m}P\left( A_{i}\right) -m+1 \\
&=&\dsum\limits_{i=1}^{m}P\left( A_{i}\right) -\left( m-1\right) \text{. }%
\square
\end{eqnarray*}

\medskip

\noindent

\noindent \textbf{Lemma OA-15: }

\begin{enumerate}
\item[(a)] For $t>0$,%
\begin{equation*}
\overline{\Phi }\left( t\right) =1-\Phi \left( t\right) \leq \frac{\phi
\left( t\right) }{t},
\end{equation*}%
where $\phi \left( t\right) $ and $\Phi \left( t\right) $ denote,
respectively, the pdf and the cdf of a standard normal random variable.

\item[(b)] Let $N=N_{1}+N_{2}$. Specify $\varphi $ such that $\varphi
\rightarrow 0$ as $N_{1},N_{2}\rightarrow \infty $ and such that, for some
constant $a>0$,%
\begin{equation*}
\varphi \geq \frac{1}{N^{a}}
\end{equation*}%
for all $N_{1},N_{2}$ sufficiently large. Then, for all $N_{1},N_{2}$
sufficiently large such that 
\begin{equation*}
1-\frac{\varphi }{2N}\geq \Phi \left( 2\right)
\end{equation*}%
we have%
\begin{equation*}
\Phi ^{-1}\left( 1-\frac{\varphi }{2N}\right) \leq \sqrt{2\left( 1+a\right) }%
\sqrt{\ln N}\text{.}
\end{equation*}
\end{enumerate}

\noindent \textbf{Proof of Lemma OA-15: }

\begin{enumerate}
\item[(a)] 
\begin{eqnarray*}
1-\Phi \left( t\right) &=&\dint\nolimits_{t}^{\infty }\frac{1}{\sqrt{2\pi }}%
\exp \left\{ -\frac{z^{2}}{2}\right\} dz \\
&=&\dint\nolimits_{t}^{\infty }\frac{1}{z}\frac{z}{\sqrt{2\pi }}\exp \left\{
-\frac{z^{2}}{2}\right\} dz \\
&\leq &\frac{1}{t}\dint\nolimits_{t}^{\infty }\frac{z}{\sqrt{2\pi }}\exp
\left\{ -\frac{z^{2}}{2}\right\} dz
\end{eqnarray*}%
Let%
\begin{equation*}
u=-\frac{z^{2}}{2}\text{ and }du=-zdz
\end{equation*}%
so that%
\begin{eqnarray*}
\dint\nolimits_{t}^{\infty }\frac{z}{\sqrt{2\pi }}\exp \left\{ -\frac{z^{2}}{%
2}\right\} dz &=&-\dint\nolimits_{{\LARGE -}\frac{{\LARGE t}^{{\large 2}}}{%
{\LARGE 2}}}^{-\infty }\frac{1}{\sqrt{2\pi }}\exp \left\{ u\right\} du \\
&=&\dint\nolimits_{-\infty }^{{\LARGE -}\frac{{\LARGE t}^{{\large 2}}}{%
{\LARGE 2}}}\frac{1}{\sqrt{2\pi }}\exp \left\{ u\right\} du \\
&=&\frac{1}{\sqrt{2\pi }}\exp \left\{ -\frac{t^{2}}{2}\right\} \\
&=&\phi \left( t\right)
\end{eqnarray*}%
It follows that%
\begin{equation*}
\overline{\Phi }\left( t\right) =1-\Phi \left( t\right) \leq \frac{\phi
\left( t\right) }{t}\text{.}
\end{equation*}

\item[(b)] Let $t>0$ and set%
\begin{equation*}
\Phi \left( t\right) =\Pr \left( Z\leq t\right) =1-\frac{\varphi }{2N}\text{.%
}
\end{equation*}%
It follows that%
\begin{equation*}
\Phi ^{-1}\left( \Phi \left( t\right) \right) =\Phi ^{-1}\left( 1-\frac{%
\varphi }{2N}\right) =t
\end{equation*}%
and, by the result given in part (a) above,%
\begin{equation*}
1-\Phi \left( t\right) =1-\left( 1-\frac{\varphi }{2N}\right) =\frac{\varphi 
}{2N}\leq \frac{\phi \left( t\right) }{t}\text{.}
\end{equation*}%
The latter inequality implies that%
\begin{equation*}
t\leq \phi \left( t\right) \frac{2N}{\varphi }
\end{equation*}%
so that%
\begin{eqnarray*}
\ln t &\leq &\ln \phi \left( t\right) +\ln 2+\ln \left( \frac{N}{\varphi }%
\right) \\
&=&-\frac{1}{2}t^{2}-\frac{1}{2}\ln 2-\frac{1}{2}\ln \pi +\ln 2+\ln \left( 
\frac{N}{\varphi }\right) \\
&=&-\frac{1}{2}t^{2}+\frac{1}{2}\ln 2-\frac{1}{2}\ln \pi +\ln \left( \frac{N%
}{\varphi }\right) \\
&<&-\frac{1}{2}t^{2}+\frac{1}{2}\ln 2+\ln \left( \frac{N}{\varphi }\right) \\
&<&-\frac{1}{2}t^{2}+\ln 2+\ln \left( \frac{N}{\varphi }\right)
\end{eqnarray*}%
or%
\begin{eqnarray*}
t^{2} &\leq &2\left( \ln 2-\ln t\right) +2\ln \left( \frac{N}{\varphi }%
\right) \\
&=&2\ln \left( \frac{2}{t}\right) +2\ln \left( \frac{N}{\varphi }\right) \\
&\leq &2\ln \left( \frac{N}{\varphi }\right) \text{ for any }t=\Phi
^{-1}\left( 1-\frac{\varphi }{2N}\right) \geq 2
\end{eqnarray*}%
so that%
\begin{equation*}
t\leq \sqrt{2}\sqrt{\ln \left( \frac{N}{\varphi }\right) }\text{ for any }%
t=\Phi ^{-1}\left( 1-\frac{\varphi }{2N}\right) \geq 2
\end{equation*}%
Hence, for $N_{1},N_{2}$ sufficiently large so that%
\begin{equation*}
1-\frac{\varphi }{2N}\geq \Phi \left( 2\right) \text{ or, equivalently, }%
t=\Phi ^{-1}\left( 1-\frac{\varphi }{2N}\right) \geq 2\text{,}
\end{equation*}%
we have%
\begin{eqnarray*}
\Phi ^{-1}\left( 1-\frac{\varphi }{2N}\right) &=&t \\
&\leq &\sqrt{2}\sqrt{\ln \left( \frac{N}{\varphi }\right) } \\
&=&\sqrt{2}\sqrt{\ln N-\ln \varphi } \\
&=&\sqrt{2}\sqrt{\ln N}\sqrt{1-\frac{\ln \varphi }{\ln N}} \\
&\leq &\sqrt{2}\sqrt{\ln N}\sqrt{1-\frac{\ln N^{-a}}{\ln N}} \\
&=&\sqrt{2\left( 1+a\right) }\sqrt{\ln N}\text{. }\square
\end{eqnarray*}
\end{enumerate}

\medskip

\noindent \textbf{Lemma QA-16: }Suppose that Assumptions 2-1, 2-2, 2-3, 2-5,
2-6, and 2-8 hold and suppose that $N_{1},N_{2},T\rightarrow \infty $ such
that $N_{1}/\tau _{1}^{3}=N_{1}/\left\lfloor T_{0}^{\alpha _{{\large 1}%
}}\right\rfloor ^{3}\rightarrow 0$. Then, the following statements are true.

\begin{enumerate}
\item[(a)] 
\begin{equation*}
\max_{1\leq \ell \leq d}\max_{i{\large \in }H^{{\large c}}}\left\vert \frac{%
\overline{S}_{i,\ell ,T}-\mu _{i,\ell ,T}}{\mu _{i,\ell ,T}}\right\vert 
\overset{p}{\rightarrow }0
\end{equation*}

\item[(b)] 
\begin{equation*}
\max_{1\leq \ell \leq d}\max_{i{\large \in }H^{{\large c}}}\left\vert \frac{%
\overline{V}_{i,\ell ,T}-\pi _{i,\ell ,T}}{\pi _{i,\ell ,T}}\right\vert 
\overset{p}{\rightarrow }0
\end{equation*}
\end{enumerate}

\noindent where%
\begin{equation*}
\overline{S}_{i,\ell ,T}=\dsum\limits_{r=1}^{q}\dsum\limits_{t=\left(
r-1\right) \tau +p}^{\left( r-1\right) \tau +\tau _{1}+p-1}Z_{it}y_{\ell ,t%
{\LARGE +}1}\text{ and }\overline{V}_{i,\ell ,T}=\dsum\limits_{r=1}^{q}\left[
\dsum\limits_{t=\left( r-1\right) \tau +p}^{\left( r-1\right) \tau +\tau
_{1}+p-1}Z_{it}y_{\ell ,t{\LARGE +}1}\right] ^{2}
\end{equation*}

\noindent \textbf{Proof of Lemma QA-16:}

To show part (a), note first that by applying parts (a) and (c) of Lemma
OA-6, parts (a)-(c) of Lemma OA-12, and the Slutsky theorem; we obtain 
\begin{eqnarray*}
&&\max_{1\leq \ell \leq d}\max_{i{\large \in }H^{{\large c}}}\left\vert 
\frac{\overline{S}_{i,\ell ,T}-\mu _{i,\ell ,T}}{q\tau _{1}}\right\vert \\
&=&\max_{1\leq \ell \leq d}\max_{i{\large \in }H^{{\large c}}}\left\vert 
\frac{1}{q}\dsum\limits_{r=1}^{q}\frac{1}{\tau _{1}}\dsum\limits_{t=\left(
r-1\right) \tau +p}^{\left( r-1\right) \tau +\tau _{1}+p-1}\gamma
_{i}^{\prime }\underline{F}_{t}\left[ \mu _{Y,\ell }+\underline{Y}%
_{t}^{\prime }\alpha _{YY,\ell }+\underline{F}_{t}^{\prime }\alpha _{YF,\ell
}\right] \right. \\
&&\text{ \ \ \ \ \ \ \ \ \ \ }+\frac{1}{q}\dsum\limits_{r=1}^{q}\frac{1}{%
\tau _{1}}\dsum\limits_{t=\left( r-1\right) \tau +p}^{\left( r-1\right) \tau
+\tau _{1}+p-1}\gamma _{i}^{\prime }\underline{F}_{t}\varepsilon _{\ell ,t%
{\LARGE +}1}+\frac{1}{q}\dsum\limits_{r=1}^{q}\frac{1}{\tau _{1}}%
\dsum\limits_{t=\left( r-1\right) \tau +p}^{\left( r-1\right) \tau +\tau
_{1}+p-1}y_{\ell ,t{\LARGE +}1}u_{it} \\
&&\text{ \ \ \ \ \ \ \ \ }\left. -\frac{1}{q}\dsum\limits_{r=1}^{q}\frac{1}{%
\tau _{1}}\dsum\limits_{t=\left( r-1\right) \tau +p}^{\left( r-1\right) \tau
+\tau _{1}+p-1}\left\{ \gamma _{i}^{\prime }E\left[ \underline{F}_{t}\right]
\mu _{Y,\ell }+\gamma _{i}^{\prime }E\left[ \underline{F}_{t}\underline{Y}%
_{t}^{\prime }\right] \alpha _{YY,\ell }+\gamma _{i}^{\prime }E\left[ 
\underline{F}_{t}\underline{F}_{t}^{\prime }\right] \alpha _{YF,\ell
}\right\} \right\vert \\
&\leq &\max_{1\leq \ell \leq d}\max_{i{\large \in }H^{{\large c}}}\left\vert 
\frac{1}{q}\dsum\limits_{r=1}^{q}\frac{1}{\tau _{1}}\dsum\limits_{t=\left(
r-1\right) \tau +p}^{\left( r-1\right) \tau +\tau _{1}+p-1}\gamma
_{i}^{\prime }\left( \underline{F}_{t}-E\left[ \underline{F}_{t}\right]
\right) \mu _{Y,\ell }\right\vert \\
&&+\max_{1\leq \ell \leq d}\max_{i{\large \in }H^{{\large c}}}\left\vert 
\frac{1}{q}\dsum\limits_{r=1}^{q}\frac{1}{\tau _{1}}\dsum\limits_{t=\left(
r-1\right) \tau +p}^{\left( r-1\right) \tau +\tau _{1}+p-1}\gamma
_{i}^{\prime }\left( \underline{F}_{t}\underline{Y}_{t}^{\prime }-E\left[ 
\underline{F}_{t}\underline{Y}_{t}^{\prime }\right] \right) \alpha _{YY,\ell
}\right\vert \\
&&+\max_{1\leq \ell \leq d}\max_{i{\large \in }H^{{\large c}}}\left\vert 
\frac{1}{q}\dsum\limits_{r=1}^{q}\frac{1}{\tau _{1}}\dsum\limits_{t=\left(
r-1\right) \tau +p}^{\left( r-1\right) \tau +\tau _{1}+p-1}\gamma
_{i}^{\prime }\left( \underline{F}_{t}\underline{F}_{t}^{\prime }-E\left[ 
\underline{F}_{t}\underline{F}_{t}^{\prime }\right] \right) \alpha _{YF,\ell
}\right\vert \\
&&+\max_{1\leq \ell \leq d}\max_{i{\large \in }H^{{\large c}}}\left\vert 
\frac{1}{q}\dsum\limits_{r=1}^{q}\frac{1}{\tau _{1}}\dsum\limits_{t=\left(
r-1\right) \tau +p}^{\left( r-1\right) \tau +\tau _{1}+p-1}\gamma
_{i}^{\prime }\underline{F}_{t}\varepsilon _{\ell ,t{\LARGE +}1}\right\vert
\\
&&+\max_{1\leq \ell \leq d}\max_{i{\large \in }H^{{\large c}}}\left\vert 
\frac{1}{q}\dsum\limits_{r=1}^{q}\frac{1}{\tau _{1}}\dsum\limits_{t=\left(
r-1\right) \tau +p}^{\left( r-1\right) \tau +\tau _{1}+p-1}y_{\ell ,t{\LARGE %
+}1}u_{it}\right\vert \\
&=&o_{p}\left( 1\right)
\end{eqnarray*}%
Moreover, by Assumption 2-8, there exist a positive constant $\underline{c}$
such that for all $N$ and $T$ sufficiently large 
\begin{eqnarray*}
&&\min_{1\leq \ell \leq d}\min_{{\large i\in }H^{{\large c}}}\left\vert 
\frac{\mu _{i,\ell ,T}}{q\tau _{1}}\right\vert \\
&=&\min_{1\leq \ell \leq d}\min_{{\large i\in }H^{{\large c}}}\left\vert 
\frac{1}{q}\dsum\limits_{r=1}^{q}\frac{1}{\tau _{1}}\dsum\limits_{t=\left(
r-1\right) \tau +p}^{\left( r-1\right) \tau +\tau _{1}+p-1}\gamma
_{i}^{\prime }\left\{ E\left[ \underline{F}_{t}\right] \mu _{Y,\ell }+E\left[
\underline{F}_{t}\underline{Y}_{t}^{\prime }\right] \alpha _{YY,\ell }+E%
\left[ \underline{F}_{t}\underline{F}_{t}^{\prime }\right] \alpha _{YF,\ell
}\right\} \right\vert \\
&\geq &\underline{c}>0
\end{eqnarray*}%
It follows that%
\begin{equation*}
\max_{1\leq \ell \leq d}\max_{i{\large \in }H^{{\large c}}}\left\vert \frac{%
\overline{S}_{i,\ell ,T}-\mu _{i,\ell ,T}}{\mu _{i,\ell ,T}}\right\vert \leq
\max_{1\leq \ell \leq d}\max_{i{\large \in }H^{{\large c}}}\left\vert \frac{%
\overline{S}_{i,\ell ,T}-\mu _{i,\ell ,T}}{q\tau _{1}}\right\vert
/\min_{1\leq \ell \leq d}\min_{{\large i\in }H^{{\large c}}}\left\vert \frac{%
\mu _{i,\ell ,T}}{q\tau _{1}}\right\vert =o_{p}\left( 1\right) \text{.}
\end{equation*}

Now, for part (b), note that, applying parts (d), (f), (g), and (h) of Lemma
OA-12, parts (b), (d), and (e) of Lemma OA-6, and the Slutsky theorem; we
have

\begin{eqnarray*}
&&\max_{1\leq \ell \leq d}\max_{i{\large \in }H^{{\large c}}}\left\vert 
\frac{\overline{V}_{i,\ell ,T}-\pi _{i,\ell ,T}}{q\tau _{1}^{2}}\right\vert
\\
&=&\max_{1\leq \ell \leq d}\max_{i{\large \in }H^{{\large c}}}\frac{1}{q}%
\dsum\limits_{r=1}^{q}\left( \frac{1}{\tau _{1}}\dsum\limits_{t=\left(
r-1\right) \tau +p}^{\left( r-1\right) \tau +\tau _{1}+p-1}\gamma
_{i}^{\prime }\left\{ \left( \underline{F}_{t}-E\left[ \underline{F}_{t}%
\right] \right) \mu _{Y,\ell }+\left( \underline{F}_{t}\underline{Y}%
_{t}^{\prime }-E\left[ \underline{F}_{t}\underline{Y}_{t}^{\prime }\right]
\right) \alpha _{YY,\ell }\right. \right. \\
&&\text{ \ \ \ \ \ \ \ \ \ \ \ \ \ \ \ \ \ \ \ \ \ \ \ \ \ \ \ }\left.
\left. +\left( \underline{F}_{t}\underline{F}_{t}^{\prime }-E\left[ 
\underline{F}_{t}\underline{F}_{t}^{\prime }\right] \right) \alpha _{YF,\ell
}\right\} \right) ^{2} \\
&&+\max_{1\leq \ell \leq d}\max_{i{\large \in }H^{{\large c}}}\left\vert 
\frac{2}{q}\dsum\limits_{r=1}^{q}\left\{ \left( \frac{1}{\tau _{1}}%
\dsum\limits_{t=\left( r-1\right) \tau +p}^{\left( r-1\right) \tau +\tau
_{1}+p-1}\gamma _{i}^{\prime }\left\{ \left( \underline{F}_{t}-E\left[ 
\underline{F}_{t}\right] \right) \mu _{Y,\ell }+\left( \underline{F}_{t}%
\underline{Y}_{t}^{\prime }-E\left[ \underline{F}_{t}\underline{Y}%
_{t}^{\prime }\right] \right) \alpha _{YY,\ell }\right. \right. \right.
\right. \\
&&\text{ \ \ \ \ \ \ \ \ \ \ \ \ \ \ \ \ \ \ \ \ \ \ \ \ \ \ \ }\left. \text{%
\ }\left. +\left( \underline{F}_{t}\underline{F}_{t}^{\prime }-E\left[ 
\underline{F}_{t}\underline{F}_{t}^{\prime }\right] \right) \alpha _{YF,\ell
}\right\} \right) \\
&&\text{ \ \ \ \ \ \ \ \ \ \ }\left. \text{\ \ \ }\left. \times \left( \frac{%
1}{\tau _{1}}\dsum\limits_{t=\left( r-1\right) \tau +p}^{\left( r-1\right)
\tau +\tau _{1}+p-1}\gamma _{i}^{\prime }\left\{ E\left[ \underline{F}_{t}%
\right] \mu _{Y,\ell }+E\left[ \underline{F}_{t}\underline{Y}_{t}^{\prime }%
\right] \alpha _{YY,\ell }+E\left[ \underline{F}_{t}\underline{F}%
_{t}^{\prime }\right] \alpha _{YF,\ell }\right\} \right) \right\} \right\vert
\\
&&+\max_{1\leq \ell \leq d}\max_{i{\large \in }H^{{\large c}}}\frac{1}{q}%
\dsum\limits_{r=1}^{q}\left( \frac{1}{\tau _{1}}\dsum\limits_{t=\left(
r-1\right) \tau +p}^{\left( r-1\right) \tau +\tau _{1}+p-1}\gamma
_{i}^{\prime }\underline{F}_{t}\varepsilon _{\ell ,t{\LARGE +}1}\right) ^{2}
\\
&&+\max_{1\leq \ell \leq d}\max_{i{\large \in }H^{{\large c}}}\frac{1}{q}%
\dsum\limits_{r=1}^{q}\left( \frac{1}{\tau _{1}}\dsum\limits_{t=\left(
r-1\right) \tau +p}^{\left( r-1\right) \tau +\tau _{1}+p-1}y_{\ell ,t{\LARGE %
+}1}u_{it}\right) ^{2} \\
&&+2\max_{1\leq \ell \leq d}\max_{i{\large \in }H^{{\large c}}}\left\vert 
\frac{1}{q}\dsum\limits_{r=1}^{q}\left( \frac{1}{\tau _{1}}%
\dsum\limits_{t=\left( r-1\right) \tau +p}^{\left( r-1\right) \tau +\tau
_{1}+p-1}\gamma _{i}^{\prime }\underline{F}_{t}\varepsilon _{\ell ,t{\LARGE +%
}1}\right) \left( \frac{1}{\tau _{1}}\dsum\limits_{t=\left( r-1\right) \tau
+p}^{\left( r-1\right) \tau +\tau _{1}+p-1}y_{\ell ,t{\LARGE +}%
1}u_{it}\right) \right\vert \\
&&+2\max_{1\leq \ell \leq d}\max_{i{\large \in }H^{{\large c}}}\left\vert 
\frac{1}{q}\dsum\limits_{r=1}^{q}\left( \frac{1}{\tau _{1}}%
\dsum\limits_{t=\left( r-1\right) \tau +p}^{\left( r-1\right) \tau +\tau
_{1}+p-1}\gamma _{i}^{\prime }\underline{F}_{t}\left[ \mu _{Y,\ell }+%
\underline{Y}_{t}^{\prime }\alpha _{YY,\ell }+\underline{F}_{t}^{\prime
}\alpha _{YF,\ell }\right] \right) \right. \\
&&\text{ \ \ \ \ \ \ \ \ \ \ \ \ \ \ \ \ \ \ \ \ }\left. \text{\ }\times
\left( \frac{1}{\tau _{1}}\dsum\limits_{t=\left( r-1\right) \tau +p}^{\left(
r-1\right) \tau +\tau _{1}+p-1}y_{\ell ,t{\LARGE +}1}u_{it}\right)
\right\vert \\
&&+2\max_{1\leq \ell \leq d}\max_{i{\large \in }H^{{\large c}}}\left\vert 
\frac{1}{q}\dsum\limits_{r=1}^{q}\left( \frac{1}{\tau _{1}}%
\dsum\limits_{t=\left( r-1\right) \tau +p}^{\left( r-1\right) \tau +\tau
_{1}+p-1}\gamma _{i}^{\prime }\underline{F}_{t}\left[ \mu _{Y,\ell }+%
\underline{Y}_{t}^{\prime }\alpha _{YY,\ell }+\underline{F}_{t}^{\prime
}\alpha _{YF,\ell }\right] \right) \right. \\
&&\text{ \ \ \ \ \ \ \ \ \ \ \ \ \ \ \ \ \ \ \ \ }\left. \times \left( \frac{%
1}{\tau _{1}}\dsum\limits_{t=\left( r-1\right) \tau +p}^{\left( r-1\right)
\tau +\tau _{1}+p-1}\gamma _{i}^{\prime }\underline{F}_{t}\varepsilon _{\ell
,t{\LARGE +}1}\right) \right\vert \\
&=&o_{p}\left( 1\right)
\end{eqnarray*}%
Moreover, note that, for all $N$ and $T$ sufficiently large,%
\begin{eqnarray*}
&&\min_{1\leq \ell \leq d}\min_{{\large i\in }H^{{\large c}}}\frac{\pi
_{i,\ell ,T}}{q\tau _{1}^{2}} \\
&=&\min_{1\leq \ell \leq d}\min_{{\large i\in }H^{{\large c}}}\frac{1}{q}%
\dsum\limits_{r=1}^{q}\left( \frac{1}{\tau _{1}}\dsum\limits_{t=\left(
r-1\right) \tau +p}^{\left( r-1\right) \tau +\tau _{1}+p-1}\left\{ \gamma
_{i}^{\prime }E\left[ \underline{F}_{t}\right] \mu _{Y,\ell }+\gamma
_{i}^{\prime }E\left[ \underline{F}_{t}\underline{Y}_{t}^{\prime }\right]
\alpha _{YY,\ell }+\gamma _{i}^{\prime }E\left[ \underline{F}_{t}\underline{F%
}_{t}^{\prime }\right] \alpha _{YF,\ell }\right\} \right) ^{2} \\
&=&\min_{1\leq \ell \leq d}\min_{{\large i\in }H^{{\large c}}}\frac{1}{q}%
\dsum\limits_{r=1}^{q}\left( \frac{1}{\tau _{1}}\dsum\limits_{t=\left(
r-1\right) \tau +p}^{\left( r-1\right) \tau +\tau _{1}+p-1}\gamma
_{i}^{\prime }\left\{ E\left[ \underline{F}_{t}\right] \mu _{Y,\ell }+E\left[
\underline{F}_{t}\underline{Y}_{t}^{\prime }\right] \alpha _{YY,\ell }+E%
\left[ \underline{F}_{t}\underline{F}_{t}^{\prime }\right] \alpha _{YF,\ell
}\right\} \right) ^{2} \\
&\geq &\min_{1\leq \ell \leq d}\min_{{\large i\in }H^{{\large c}}}\left( 
\frac{1}{q}\dsum\limits_{r=1}^{q}\frac{1}{\tau _{1}}\dsum\limits_{t=\left(
r-1\right) \tau +p}^{\left( r-1\right) \tau +\tau _{1}+p-1}\gamma
_{i}^{\prime }\left\{ E\left[ \underline{F}_{t}\right] \mu _{Y,\ell }+E\left[
\underline{F}_{t}\underline{Y}_{t}^{\prime }\right] \alpha _{YY,\ell }+E%
\left[ \underline{F}_{t}\underline{F}_{t}^{\prime }\right] \alpha _{YF,\ell
}\right\} \right) ^{2}\text{ \ } \\
&&\left( \text{by Jensen's inequality}\right) \\
&=&\min_{1\leq \ell \leq d}\min_{{\large i\in }H^{{\large c}}}\left\vert 
\frac{1}{q}\dsum\limits_{r=1}^{q}\frac{1}{\tau _{1}}\dsum\limits_{t=\left(
r-1\right) \tau +p}^{\left( r-1\right) \tau +\tau _{1}+p-1}\gamma
_{i}^{\prime }\left\{ E\left[ \underline{F}_{t}\right] \mu _{Y,\ell }+E\left[
\underline{F}_{t}\underline{Y}_{t}^{\prime }\right] \alpha _{YY,\ell }+E%
\left[ \underline{F}_{t}\underline{F}_{t}^{\prime }\right] \alpha _{YF,\ell
}\right\} \right\vert ^{2} \\
&=&\left( \min_{1\leq \ell \leq d}\min_{{\large i\in }H^{{\large c}%
}}\left\vert \frac{1}{q}\dsum\limits_{r=1}^{q}\frac{1}{\tau _{1}}%
\dsum\limits_{t=\left( r-1\right) \tau +p}^{\left( r-1\right) \tau +\tau
_{1}+p-1}\gamma _{i}^{\prime }\left\{ E\left[ \underline{F}_{t}\right] \mu
_{Y,\ell }+E\left[ \underline{F}_{t}\underline{Y}_{t}^{\prime }\right]
\alpha _{YY,\ell }+E\left[ \underline{F}_{t}\underline{F}_{t}^{\prime }%
\right] \alpha _{YF,\ell }\right\} \right\vert \right) ^{2} \\
&\geq &\underline{c}^{2}>0\text{ \ }\left( \text{by Assumption 2-8}\right) 
\text{. }
\end{eqnarray*}%
It follows that%
\begin{equation*}
\max_{1\leq \ell \leq d}\max_{i{\large \in }H^{{\large c}}}\left\vert \frac{%
\overline{V}_{i,\ell ,T}-\pi _{i,\ell ,T}}{\pi _{i,\ell ,T}}\right\vert \leq
\max_{1\leq \ell \leq d}\max_{i{\large \in }H^{{\large c}}}\left\vert \frac{%
\overline{V}_{i,\ell ,T}-\pi _{i,\ell ,T}}{q\tau _{1}^{2}}\right\vert
/\min_{1\leq \ell \leq d}\min_{{\large i\in }H^{{\large c}}}\left( \frac{\pi
_{i,\ell ,T}}{q\tau _{1}^{2}}\right) =o_{p}\left( 1\right) \text{. }\square
\end{equation*}

\bigskip

\begin{thebibliography}{9}
\bibitem{} Borovkova, S., R. Burton, and H. Dehling (2001):
\textquotedblleft Limit Theorems for Functionals of Mixing Processes to
U-Statistics and Dimension Estimation,\textquotedblright\ \textit{%
Transactions of the American Mathematical Society}, 353, 4261-4318.

\bibitem{} Davidson. J. (1994): \textit{Stochastic Limit Theory: An
Introduction for Econometricians}. New York: Oxford University Press.

\bibitem{} Pham, T. D. and L. T. Tran (1985): \textquotedblleft Some Mixing
Properties of Time Series Models,\textquotedblright\ \textit{Stochastic
Processes and Their Applications}, 19, 297-303.

\bibitem{} Ruhe, A. (1975): \textquotedblleft On the Closeness of
Eigenvalues and Singular Values for Almost Normal
Matrices,\textquotedblright\ \textit{Linear Algebra and Its Applications},
11, 87-94.
\end{thebibliography}

\medskip

\end{document}
